%% Generated by Sphinx.
\def\sphinxdocclass{report}
\documentclass[letterpaper,10pt,english]{sphinxmanual}
\ifdefined\pdfpxdimen
   \let\sphinxpxdimen\pdfpxdimen\else\newdimen\sphinxpxdimen
\fi \sphinxpxdimen=.75bp\relax

\PassOptionsToPackage{warn}{textcomp}
\usepackage[utf8]{inputenc}
\ifdefined\DeclareUnicodeCharacter
% support both utf8 and utf8x syntaxes
  \ifdefined\DeclareUnicodeCharacterAsOptional
    \def\sphinxDUC#1{\DeclareUnicodeCharacter{"#1}}
  \else
    \let\sphinxDUC\DeclareUnicodeCharacter
  \fi
  \sphinxDUC{00A0}{\nobreakspace}
  \sphinxDUC{2500}{\sphinxunichar{2500}}
  \sphinxDUC{2502}{\sphinxunichar{2502}}
  \sphinxDUC{2514}{\sphinxunichar{2514}}
  \sphinxDUC{251C}{\sphinxunichar{251C}}
  \sphinxDUC{2572}{\textbackslash}
\fi
\usepackage{cmap}
\usepackage[T1]{fontenc}
\usepackage{amsmath,amssymb,amstext}
\usepackage{babel}



\usepackage{times}
\expandafter\ifx\csname T@LGR\endcsname\relax
\else
% LGR was declared as font encoding
  \substitutefont{LGR}{\rmdefault}{cmr}
  \substitutefont{LGR}{\sfdefault}{cmss}
  \substitutefont{LGR}{\ttdefault}{cmtt}
\fi
\expandafter\ifx\csname T@X2\endcsname\relax
  \expandafter\ifx\csname T@T2A\endcsname\relax
  \else
  % T2A was declared as font encoding
    \substitutefont{T2A}{\rmdefault}{cmr}
    \substitutefont{T2A}{\sfdefault}{cmss}
    \substitutefont{T2A}{\ttdefault}{cmtt}
  \fi
\else
% X2 was declared as font encoding
  \substitutefont{X2}{\rmdefault}{cmr}
  \substitutefont{X2}{\sfdefault}{cmss}
  \substitutefont{X2}{\ttdefault}{cmtt}
\fi


\usepackage[Bjarne]{fncychap}
\usepackage{sphinx}

\fvset{fontsize=\small}
\usepackage{geometry}


% Include hyperref last.
\usepackage{hyperref}
% Fix anchor placement for figures with captions.
\usepackage{hypcap}% it must be loaded after hyperref.
% Set up styles of URL: it should be placed after hyperref.
\urlstyle{same}

\addto\captionsenglish{\renewcommand{\contentsname}{Contents:}}

\usepackage{sphinxmessages}
\setcounter{tocdepth}{1}


% Jupyter Notebook code cell colors
\definecolor{nbsphinxin}{HTML}{307FC1}
\definecolor{nbsphinxout}{HTML}{BF5B3D}
\definecolor{nbsphinx-code-bg}{HTML}{F5F5F5}
\definecolor{nbsphinx-code-border}{HTML}{E0E0E0}
\definecolor{nbsphinx-stderr}{HTML}{FFDDDD}
% ANSI colors for output streams and traceback highlighting
\definecolor{ansi-black}{HTML}{3E424D}
\definecolor{ansi-black-intense}{HTML}{282C36}
\definecolor{ansi-red}{HTML}{E75C58}
\definecolor{ansi-red-intense}{HTML}{B22B31}
\definecolor{ansi-green}{HTML}{00A250}
\definecolor{ansi-green-intense}{HTML}{007427}
\definecolor{ansi-yellow}{HTML}{DDB62B}
\definecolor{ansi-yellow-intense}{HTML}{B27D12}
\definecolor{ansi-blue}{HTML}{208FFB}
\definecolor{ansi-blue-intense}{HTML}{0065CA}
\definecolor{ansi-magenta}{HTML}{D160C4}
\definecolor{ansi-magenta-intense}{HTML}{A03196}
\definecolor{ansi-cyan}{HTML}{60C6C8}
\definecolor{ansi-cyan-intense}{HTML}{258F8F}
\definecolor{ansi-white}{HTML}{C5C1B4}
\definecolor{ansi-white-intense}{HTML}{A1A6B2}
\definecolor{ansi-default-inverse-fg}{HTML}{FFFFFF}
\definecolor{ansi-default-inverse-bg}{HTML}{000000}

% Define an environment for non-plain-text code cell outputs (e.g. images)
\makeatletter
\newenvironment{nbsphinxfancyoutput}{%
    % Avoid fatal error with framed.sty if graphics too long to fit on one page
    \let\sphinxincludegraphics\nbsphinxincludegraphics
    \nbsphinx@image@maxheight\textheight
    \advance\nbsphinx@image@maxheight -2\fboxsep   % default \fboxsep 3pt
    \advance\nbsphinx@image@maxheight -2\fboxrule  % default \fboxrule 0.4pt
    \advance\nbsphinx@image@maxheight -\baselineskip
\def\nbsphinxfcolorbox{\spx@fcolorbox{nbsphinx-code-border}{white}}%
\def\FrameCommand{\nbsphinxfcolorbox\nbsphinxfancyaddprompt\@empty}%
\def\FirstFrameCommand{\nbsphinxfcolorbox\nbsphinxfancyaddprompt\sphinxVerbatim@Continues}%
\def\MidFrameCommand{\nbsphinxfcolorbox\sphinxVerbatim@Continued\sphinxVerbatim@Continues}%
\def\LastFrameCommand{\nbsphinxfcolorbox\sphinxVerbatim@Continued\@empty}%
\MakeFramed{\advance\hsize-\width\@totalleftmargin\z@\linewidth\hsize\@setminipage}%
\lineskip=1ex\lineskiplimit=1ex\raggedright%
}{\par\unskip\@minipagefalse\endMakeFramed}
\makeatother
\newbox\nbsphinxpromptbox
\def\nbsphinxfancyaddprompt{\ifvoid\nbsphinxpromptbox\else
    \kern\fboxrule\kern\fboxsep
    \copy\nbsphinxpromptbox
    \kern-\ht\nbsphinxpromptbox\kern-\dp\nbsphinxpromptbox
    \kern-\fboxsep\kern-\fboxrule\nointerlineskip
    \fi}
\newlength\nbsphinxcodecellspacing
\setlength{\nbsphinxcodecellspacing}{0pt}

% Define support macros for attaching opening and closing lines to notebooks
\newsavebox\nbsphinxbox
\makeatletter
\newcommand{\nbsphinxstartnotebook}[1]{%
    \par
    % measure needed space
    \setbox\nbsphinxbox\vtop{{#1\par}}
    % reserve some space at bottom of page, else start new page
    \needspace{\dimexpr2.5\baselineskip+\ht\nbsphinxbox+\dp\nbsphinxbox}
    % mimick vertical spacing from \section command
      \addpenalty\@secpenalty
      \@tempskipa 3.5ex \@plus 1ex \@minus .2ex\relax
      \addvspace\@tempskipa
      {\Large\@tempskipa\baselineskip
             \advance\@tempskipa-\prevdepth
             \advance\@tempskipa-\ht\nbsphinxbox
             \ifdim\@tempskipa>\z@
               \vskip \@tempskipa
             \fi}
    \unvbox\nbsphinxbox
    % if notebook starts with a \section, prevent it from adding extra space
    \@nobreaktrue\everypar{\@nobreakfalse\everypar{}}%
    % compensate the parskip which will get inserted by next paragraph
    \nobreak\vskip-\parskip
    % do not break here
    \nobreak
}% end of \nbsphinxstartnotebook

\newcommand{\nbsphinxstopnotebook}[1]{%
    \par
    % measure needed space
    \setbox\nbsphinxbox\vbox{{#1\par}}
    \nobreak % it updates page totals
    \dimen@\pagegoal
    \advance\dimen@-\pagetotal \advance\dimen@-\pagedepth
    \advance\dimen@-\ht\nbsphinxbox \advance\dimen@-\dp\nbsphinxbox
    \ifdim\dimen@<\z@
      % little space left
      \unvbox\nbsphinxbox
      \kern-.8\baselineskip
      \nobreak\vskip\z@\@plus1fil
      \penalty100
      \vskip\z@\@plus-1fil
      \kern.8\baselineskip
    \else
      \unvbox\nbsphinxbox
    \fi
}% end of \nbsphinxstopnotebook

% Ensure height of an included graphics fits in nbsphinxfancyoutput frame
\newdimen\nbsphinx@image@maxheight % set in nbsphinxfancyoutput environment
\newcommand*{\nbsphinxincludegraphics}[2][]{%
    \gdef\spx@includegraphics@options{#1}%
    \setbox\spx@image@box\hbox{\includegraphics[#1,draft]{#2}}%
    \in@false
    \ifdim \wd\spx@image@box>\linewidth
      \g@addto@macro\spx@includegraphics@options{,width=\linewidth}%
      \in@true
    \fi
    % no rotation, no need to worry about depth
    \ifdim \ht\spx@image@box>\nbsphinx@image@maxheight
      \g@addto@macro\spx@includegraphics@options{,height=\nbsphinx@image@maxheight}%
      \in@true
    \fi
    \ifin@
      \g@addto@macro\spx@includegraphics@options{,keepaspectratio}%
    \fi
    \setbox\spx@image@box\box\voidb@x % clear memory
    \expandafter\includegraphics\expandafter[\spx@includegraphics@options]{#2}%
}% end of "\MakeFrame"-safe variant of \sphinxincludegraphics
\makeatother

\makeatletter
\renewcommand*\sphinx@verbatim@nolig@list{\do\'\do\`}
\begingroup
\catcode`'=\active
\let\nbsphinx@noligs\@noligs
\g@addto@macro\nbsphinx@noligs{\let'\PYGZsq}
\endgroup
\makeatother
\renewcommand*\sphinxbreaksbeforeactivelist{\do\<\do\"\do\'}
\renewcommand*\sphinxbreaksafteractivelist{\do\.\do\,\do\:\do\;\do\?\do\!\do\/\do\>\do\-}
\makeatletter
\fvset{codes*=\sphinxbreaksattexescapedchars\do\^\^\let\@noligs\nbsphinx@noligs}
\makeatother



\title{muse\sphinxhyphen{}documentation}
\date{Nov 05, 2020}
\release{0.1}
\author{Sustainable Gas Institute}
\newcommand{\sphinxlogo}{\vbox{}}
\renewcommand{\releasename}{Release}
\makeindex
\begin{document}

\pagestyle{empty}
\sphinxmaketitle
\pagestyle{plain}
\sphinxtableofcontents
\pagestyle{normal}
\phantomsection\label{\detokenize{index::doc}}



\chapter{Installation}
\label{\detokenize{installing-muse:installation}}\label{\detokenize{installing-muse::doc}}
There are two ways to install MUSE: one for users who do not wish to modify the source code of MUSE, and another for developers who do.

\begin{sphinxadmonition}{note}{Note:}
Windows users and developers may need to install \sphinxhref{https://visualstudio.microsoft.com/downloads/\#build-tools-for-visual-studio-2019}{Windows Build Tools}. These tools include C/C++ compilers which are needed to build some python dependencies.

MacOS includes compilers by default, hence no action is needed for Mac users.

Linux users may need to install a C compiler, whether GNU gcc or Clang, as well python development packages, depending on their distribution.
\end{sphinxadmonition}


\section{For users}
\label{\detokenize{installing-muse:for-users}}
MUSE is developed using python, an open\sphinxhyphen{}source programming language, which means that there are two steps to the installation process. First, python should be installed. Then so should MUSE.

The simplest method to install python is by downloading the \sphinxhref{https://www.anaconda.com/distribution/\#download-section}{Anaconda distribution}. Make sure to choose the appropriate operating system (e.g. windows), python version 3.7, and the 64 bit installer. Once this has been done follow the steps for the anaconda installer, as prompted.

After python is installed we can install MUSE. MUSE can be installed via the \sphinxhref{https://docs.anaconda.com/anaconda/user-guide/getting-started/\#write-a-python-program-using-anaconda-prompt-or-terminal}{Anaconda Prompt} (or any terminal on Mac and Linux). This is a command\sphinxhyphen{}line interface to python and the python eco\sphinxhyphen{}system. In the anaconda prompt, run:

\begin{sphinxVerbatim}[commandchars=\\\{\}]
python \PYGZhy{}m pip install \PYGZhy{}\PYGZhy{}user git+https://github.com/SGIModel/StarMuse
\end{sphinxVerbatim}

It should now be possible to run muse. Again, this can be done in the anaconda prompt as follows:

\begin{sphinxVerbatim}[commandchars=\\\{\}]
python \PYGZhy{}m muse \PYGZhy{}\PYGZhy{}help
\end{sphinxVerbatim}

\begin{sphinxadmonition}{note}{Note:}
Although not strictly necessary, users are encouraged to create an \sphinxhref{https://www.anaconda.com/what-is-anaconda/}{Anaconda virtual environment} and install MUSE there, as shown in {\hyperref[\detokenize{installing-muse:installation-devs}]{\sphinxcrossref{\DUrole{std,std-ref}{For developers}}}}.
\end{sphinxadmonition}


\section{For developers}
\label{\detokenize{installing-muse:for-developers}}\label{\detokenize{installing-muse:installation-devs}}
Although not strictly necessary, creating an \sphinxhref{https://www.anaconda.com/what-is-anaconda/}{Anaconda virtual environment} is highly
recommended. Anaconda will isolate users and developers from changes occuring on their
operating system, and from conflicts between python packages. It also ensures reproducibility
from day to day.

Create a virtual env including python with:

\begin{sphinxVerbatim}[commandchars=\\\{\}]
conda create \PYGZhy{}n muse \PYG{n+nv}{python}\PYG{o}{=}\PYG{l+m}{3}.7
\end{sphinxVerbatim}

Activate the environment with:

\begin{sphinxVerbatim}[commandchars=\\\{\}]
conda activate muse
\end{sphinxVerbatim}

Later, to recover the system\sphinxhyphen{}wide “normal” python, deactivate the environment with:

\begin{sphinxVerbatim}[commandchars=\\\{\}]
conda deactivate
\end{sphinxVerbatim}

The simplest approach is to first download the muse code with \sphinxhref{https://git-scm.com/}{git}:

\begin{sphinxVerbatim}[commandchars=\\\{\}]
git clone https://github.com/SGIModel/StarMuse.git muse
\end{sphinxVerbatim}

For interested users, there are plenty of \sphinxhref{http://try.github.io/}{good} tutorials for \sphinxhref{https://git-scm.com/}{git}.
Next, it is possible to install the working directory into the conda environment:

\begin{sphinxVerbatim}[commandchars=\\\{\}]
\PYG{c+c1}{\PYGZsh{} On Linux and Mac}
\PYG{n+nb}{cd} muse
conda activate muse
python \PYGZhy{}m pip install \PYGZhy{}e \PYG{l+s+s2}{\PYGZdq{}.[dev,docs]\PYGZdq{}}

\PYG{c+c1}{\PYGZsh{} On Windows}
dir muse
conda activate muse
python \PYGZhy{}m pip install \PYGZhy{}e \PYG{l+s+s2}{\PYGZdq{}.[dev,docs]\PYGZdq{}}
\end{sphinxVerbatim}

The quotation marks are needed on some systems or shells, and do not hurt on any. The
downloaded code can then be modified. The changes will be automatically reflected in the
conda environment.

Tests can be run with the command \sphinxhref{https://docs.pytest.org/en/latest/}{pytest}, from the testing framework of the same name.

The documentation can be built with:

\begin{sphinxVerbatim}[commandchars=\\\{\}]
python setup.py docs
\end{sphinxVerbatim}

The main page for the documentation can then be found at
\sphinxtitleref{build\textbackslash{}sphinx\textbackslash{}html\textbackslash{}index.html} (or \sphinxtitleref{build/sphinx/html/index.html} on Mac and Linux).
The file can viewed from any web browser.


\chapter{Running your first example}
\label{\detokenize{running-muse-example:Running-your-first-example}}\label{\detokenize{running-muse-example::doc}}
In this section we run an example simulation of MUSE and visualise the results. There are a number of different examples in the source code, which can be found \sphinxhref{dead-link}{INSERT LINK HERE}.

Once python and MUSE have been installed, we can run an example. To do this open anaconda prompt. Then change directory to where you have downloaded the MUSE source code.

Navigate to the following link for MacOS or Linux based operating systems:

\sphinxcode{\sphinxupquote{\{MUSE\_download\_location\}/StarMuse/run/example/default/}}

Change \sphinxcode{\sphinxupquote{\{MUSE\_download\_location\}}} to the location you downloaded MUSE to, for example \sphinxcode{\sphinxupquote{Users/\{my\_name\}/Documents/}} using the \sphinxcode{\sphinxupquote{cd}} command, or “change directory” command. Once we have navigated to the directory containing the example settings \sphinxcode{\sphinxupquote{settings.toml}} we can run the simulation using the following command in the anaconda prompt or terminal:

\sphinxcode{\sphinxupquote{python \sphinxhyphen{}m muse settings.toml}}

If running correctly, your prompt should output text similar to that which can be found here.

It is also possible to run MUSE directly in python using the following code:

{
\sphinxsetup{VerbatimColor={named}{nbsphinx-code-bg}}
\sphinxsetup{VerbatimBorderColor={named}{nbsphinx-code-border}}
\begin{sphinxVerbatim}[commandchars=\\\{\}]
\llap{\color{nbsphinxin}[ ]:\,\hspace{\fboxrule}\hspace{\fboxsep}}\PYG{k+kn}{from} \PYG{n+nn}{muse} \PYG{k+kn}{import} \PYG{n}{examples}
\PYG{n}{model} \PYG{o}{=} \PYG{n}{examples}\PYG{o}{.}\PYG{n}{model}\PYG{p}{(}\PYG{l+s+s2}{\PYGZdq{}}\PYG{l+s+s2}{default}\PYG{l+s+s2}{\PYGZdq{}}\PYG{p}{)}
\PYG{n}{model}\PYG{o}{.}\PYG{n}{run}\PYG{p}{(}\PYG{p}{)}
\end{sphinxVerbatim}
}


\section{Results}
\label{\detokenize{running-muse-example:Results}}
If the default MUSE example has run successfully, you should now have a folder called \sphinxcode{\sphinxupquote{Results}} in the same directory as \sphinxcode{\sphinxupquote{settings.toml}}.

This directory should contain results for each sector (\sphinxcode{\sphinxupquote{Gas}},\sphinxcode{\sphinxupquote{Power}} and \sphinxcode{\sphinxupquote{Residential}}) as well as results for the entire simulation in the form of \sphinxcode{\sphinxupquote{MCACapacity.csv}} and \sphinxcode{\sphinxupquote{MCAPrices.csv}}.
\begin{itemize}
\item {} 
\sphinxcode{\sphinxupquote{MCACapacity.csv}} contains information about the capacity each agent has for each technology per year.

\item {} 
\sphinxcode{\sphinxupquote{MCAPrices.csv}} has the price of each commodity per year and timeslice. eg. the cost of electricity at night for electricity in 2020.

\end{itemize}

Within each of the sector result folders, there is an output for \sphinxcode{\sphinxupquote{Capacity}} for each commodity in each year. The years into the future, which the simulation has not run to, refers to the capacity as it retires. Within the \sphinxcode{\sphinxupquote{Residential}} folder there is also a folder for \sphinxcode{\sphinxupquote{Supply}} within each year. This refers to how much end\sphinxhyphen{}use commodity was output.

It is possible to extend the outputs using hooks. To see how to do this refer to the developer guide here.


\section{Visualisation}
\label{\detokenize{running-muse-example:Visualisation}}
Now, we can visualise the results of our first simulation! For this we will need to install a data management package called \sphinxcode{\sphinxupquote{pandas}}, as well as data visualisation software, \sphinxcode{\sphinxupquote{matplotlib}} and \sphinxcode{\sphinxupquote{seaborn}}.

This can be done with the following command:

\sphinxcode{\sphinxupquote{python \sphinxhyphen{}m pip install \sphinxhyphen{}\sphinxhyphen{}user pandas}}

To install \sphinxcode{\sphinxupquote{matplotlib}} and \sphinxcode{\sphinxupquote{seaborn}} replace \sphinxcode{\sphinxupquote{pandas}} in the previous command with the respective package name.

Firstly, we import the packages required to visualise the results.

{
\sphinxsetup{VerbatimColor={named}{nbsphinx-code-bg}}
\sphinxsetup{VerbatimBorderColor={named}{nbsphinx-code-border}}
\begin{sphinxVerbatim}[commandchars=\\\{\}]
\llap{\color{nbsphinxin}[1]:\,\hspace{\fboxrule}\hspace{\fboxsep}}\PYG{k+kn}{import} \PYG{n+nn}{pandas} \PYG{k}{as} \PYG{n+nn}{pd}
\PYG{k+kn}{import} \PYG{n+nn}{matplotlib}\PYG{n+nn}{.}\PYG{n+nn}{pyplot} \PYG{k}{as} \PYG{n+nn}{plt}
\PYG{k+kn}{import} \PYG{n+nn}{seaborn} \PYG{k}{as} \PYG{n+nn}{sns}
\end{sphinxVerbatim}
}

Next, we load the dataset of interest to us for this example: the \sphinxcode{\sphinxupquote{MCACapacity.csv}} file. We do this using pandas.

{
\sphinxsetup{VerbatimColor={named}{nbsphinx-code-bg}}
\sphinxsetup{VerbatimBorderColor={named}{nbsphinx-code-border}}
\begin{sphinxVerbatim}[commandchars=\\\{\}]
\llap{\color{nbsphinxin}[2]:\,\hspace{\fboxrule}\hspace{\fboxsep}}\PYG{n}{capacity\PYGZus{}results} \PYG{o}{=} \PYG{n}{pd}\PYG{o}{.}\PYG{n}{read\PYGZus{}csv}\PYG{p}{(}\PYG{l+s+s2}{\PYGZdq{}}\PYG{l+s+s2}{Results/MCAcapacity.csv}\PYG{l+s+s2}{\PYGZdq{}}\PYG{p}{)}
\PYG{n}{capacity\PYGZus{}results}\PYG{o}{.}\PYG{n}{head}\PYG{p}{(}\PYG{p}{)}
\end{sphinxVerbatim}
}

{

\kern-\sphinxverbatimsmallskipamount\kern-\baselineskip
\kern+\FrameHeightAdjust\kern-\fboxrule
\vspace{\nbsphinxcodecellspacing}

\sphinxsetup{VerbatimColor={named}{white}}
\sphinxsetup{VerbatimBorderColor={named}{nbsphinx-code-border}}
\begin{sphinxVerbatim}[commandchars=\\\{\}]
\llap{\color{nbsphinxout}[2]:\,\hspace{\fboxrule}\hspace{\fboxsep}}   technology region agent      type       sector  capacity  year
0   gasboiler     R1    A1  retrofit  residential      10.0  2020
1     gasCCGT     R1    A1  retrofit        power       1.0  2020
2  gassupply1     R1    A1  retrofit          gas      15.0  2020
3   gasboiler     R1    A1  retrofit  residential       5.0  2025
4    heatpump     R1    A1  retrofit  residential      19.0  2025
\end{sphinxVerbatim}
}

Using the \sphinxcode{\sphinxupquote{head}} command we print the first five rows of our dataset. Next, we will visualise each of the sectors, with capacity on the y\sphinxhyphen{}axis and year on the x\sphinxhyphen{}axis.

Don’t worry too much about the code if some of it is unfamiliar. We effectively split the data into each sector and then plot a line plot for each.

{
\sphinxsetup{VerbatimColor={named}{nbsphinx-code-bg}}
\sphinxsetup{VerbatimBorderColor={named}{nbsphinx-code-border}}
\begin{sphinxVerbatim}[commandchars=\\\{\}]
\llap{\color{nbsphinxin}[3]:\,\hspace{\fboxrule}\hspace{\fboxsep}}\PYG{k}{for} \PYG{n}{sector\PYGZus{}name}\PYG{p}{,} \PYG{n}{results} \PYG{o+ow}{in} \PYG{n}{capacity\PYGZus{}results}\PYG{o}{.}\PYG{n}{groupby}\PYG{p}{(}\PYG{l+s+s2}{\PYGZdq{}}\PYG{l+s+s2}{sector}\PYG{l+s+s2}{\PYGZdq{}}\PYG{p}{)}\PYG{p}{:}
    \PYG{n+nb}{print}\PYG{p}{(}\PYG{l+s+s2}{\PYGZdq{}}\PYG{l+s+si}{\PYGZob{}\PYGZcb{}}\PYG{l+s+s2}{ sector}\PYG{l+s+s2}{\PYGZdq{}}\PYG{o}{.}\PYG{n}{format}\PYG{p}{(}\PYG{n}{sector\PYGZus{}name}\PYG{p}{)}\PYG{p}{)}
    \PYG{n}{sns}\PYG{o}{.}\PYG{n}{lineplot}\PYG{p}{(}\PYG{n}{data}\PYG{o}{=}\PYG{n}{results}\PYG{p}{,} \PYG{n}{x}\PYG{o}{=}\PYG{l+s+s2}{\PYGZdq{}}\PYG{l+s+s2}{year}\PYG{l+s+s2}{\PYGZdq{}}\PYG{p}{,} \PYG{n}{y}\PYG{o}{=}\PYG{l+s+s2}{\PYGZdq{}}\PYG{l+s+s2}{capacity}\PYG{l+s+s2}{\PYGZdq{}}\PYG{p}{,} \PYG{n}{hue}\PYG{o}{=}\PYG{l+s+s2}{\PYGZdq{}}\PYG{l+s+s2}{technology}\PYG{l+s+s2}{\PYGZdq{}}\PYG{p}{)}
    \PYG{n}{plt}\PYG{o}{.}\PYG{n}{show}\PYG{p}{(}\PYG{p}{)}
    \PYG{n}{plt}\PYG{o}{.}\PYG{n}{close}\PYG{p}{(}\PYG{p}{)}
\end{sphinxVerbatim}
}

{

\kern-\sphinxverbatimsmallskipamount\kern-\baselineskip
\kern+\FrameHeightAdjust\kern-\fboxrule
\vspace{\nbsphinxcodecellspacing}

\sphinxsetup{VerbatimColor={named}{white}}
\sphinxsetup{VerbatimBorderColor={named}{nbsphinx-code-border}}
\begin{sphinxVerbatim}[commandchars=\\\{\}]
gas sector
\end{sphinxVerbatim}
}

\hrule height -\fboxrule\relax
\vspace{\nbsphinxcodecellspacing}

\makeatletter\setbox\nbsphinxpromptbox\box\voidb@x\makeatother

\begin{nbsphinxfancyoutput}

\noindent\sphinxincludegraphics[width=382\sphinxpxdimen,height=262\sphinxpxdimen]{{running-muse-example_12_1}.png}

\end{nbsphinxfancyoutput}

{

\kern-\sphinxverbatimsmallskipamount\kern-\baselineskip
\kern+\FrameHeightAdjust\kern-\fboxrule
\vspace{\nbsphinxcodecellspacing}

\sphinxsetup{VerbatimColor={named}{white}}
\sphinxsetup{VerbatimBorderColor={named}{nbsphinx-code-border}}
\begin{sphinxVerbatim}[commandchars=\\\{\}]
power sector
\end{sphinxVerbatim}
}

\hrule height -\fboxrule\relax
\vspace{\nbsphinxcodecellspacing}

\makeatletter\setbox\nbsphinxpromptbox\box\voidb@x\makeatother

\begin{nbsphinxfancyoutput}

\noindent\sphinxincludegraphics[width=382\sphinxpxdimen,height=262\sphinxpxdimen]{{running-muse-example_12_3}.png}

\end{nbsphinxfancyoutput}

{

\kern-\sphinxverbatimsmallskipamount\kern-\baselineskip
\kern+\FrameHeightAdjust\kern-\fboxrule
\vspace{\nbsphinxcodecellspacing}

\sphinxsetup{VerbatimColor={named}{white}}
\sphinxsetup{VerbatimBorderColor={named}{nbsphinx-code-border}}
\begin{sphinxVerbatim}[commandchars=\\\{\}]
residential sector
\end{sphinxVerbatim}
}

\hrule height -\fboxrule\relax
\vspace{\nbsphinxcodecellspacing}

\makeatletter\setbox\nbsphinxpromptbox\box\voidb@x\makeatother

\begin{nbsphinxfancyoutput}

\noindent\sphinxincludegraphics[width=382\sphinxpxdimen,height=262\sphinxpxdimen]{{running-muse-example_12_5}.png}

\end{nbsphinxfancyoutput}

In this toy example, we can see that the end\sphinxhyphen{}use technology of choice in the residential sector becomes a heatpump. The heatpump displaces the gas boiler. Therefore, the supply of gas crashes due to a reduced demand. To account for the increase in demand for electricity, the agent invests heavily in wind turbines.


\section{Next steps}
\label{\detokenize{running-muse-example:Next-steps}}
If you want to jump straight into customising your own example scenarios, head to the link {\hyperref[\detokenize{user-guide/index::doc}]{\sphinxcrossref{\DUrole{doc}{here}}}}. If you would like a little bit of background based on how MUSE works first, head to the next section!


\chapter{MUSE Overview}
\label{\detokenize{overview:muse-overview}}\label{\detokenize{overview::doc}}
\begin{sphinxadmonition}{note}{Note:}
TODO: Potentially find introductory image to place here.
\end{sphinxadmonition}

MUSE is an open source agent\sphinxhyphen{}based modelling environment that can be used to simulate change in an energy system over time. An example of the type of question MUSE can help in answering is:
\begin{itemize}
\item {} 
How may a carbon budget affect investments made in the power sector over the next 30 years?

\end{itemize}

MUSE can incorporate residential, power, industrial and conversion sectors, meaning many questions can be explored using MUSE, as per the wishes of the user.

MUSE is an agent\sphinxhyphen{}based modelling environment, where the agents are investors and consumers. In MUSE, this means that investment decisions are made from the point of view of the investor and consumer. These agents can be heterogenous, enabling for differering investment strategies between agents, as in the real world.

MUSE is technology rich and can model energy production, conversion and end\sphinxhyphen{}use technologies. So, for example, MUSE can enable the user to develop a power sector with solar photovoltaics, wind turbines and gas power plants which produce energy for appliances like electric stoves, heaters and lighting in the residential sector. Agents invest within these sectors, investing in technologies such as electric stoves in the residential sector or gas power plants in the power sectors. The investments made depend on the agent’s investment strategies.

Every sector is a user configurable module. This means that a user can configure any number of sectors, cointaining custom, user\sphinxhyphen{}defined technologies. In practice, this configuration is carried out using a selection of {\hyperref[\detokenize{inputs/index:input-files}]{\sphinxcrossref{\DUrole{std,std-ref}{Input Files}}}}. In addition, MUSE can model any geographical region around the world and over any time scale, from a single year through to 100 years or more. Within a year, MUSE allows for a user\sphinxhyphen{}defined temporal granularity. This allows for the year to be split into different seasons and times, where energy demand may differ.

MUSE differs from the vast majority of energy systems models, which are intertemporal optimisation, by allowing agents to have “limited foresight”. This enables these agents to invest under uncertainty of the future, as in the real world. In addition, MUSE is a “partial equilibrium” model, in the sense that it balances supply and demand of each energy commodity in the system.


\section{What questions can MUSE answer?}
\label{\detokenize{overview:what-questions-can-muse-answer}}
MUSE allows for users to investigate how an energy system may evolve over a time period, based upon investors using different decision metrics or objectives such as the \sphinxhref{https://en.wikipedia.org/wiki/Net\_present\_value}{net present value}, \sphinxhref{https://en.wikipedia.org/wiki/Levelized\_cost\_of\_energy}{levelized cost of electricity} or a custom\sphinxhyphen{}defined function. In addition to this, it can simulate how investors search for technology options, and how different objectives are combined to reach an investment decision.

The search for new technologies can depend on several factors such as agents’ budgets, technology maturity or preferences on the fuel\sphinxhyphen{}type. For instance, an investor in the power sector may decide that they want to focus on renewable energy, whereas another may prefer the perceived most profitable option.

Examples of the questions MUSE can answer include:
\begin{itemize}
\item {} 
\sphinxhref{https://www.sciencedirect.com/science/article/pii/S0306261920308072}{How may India’s steel industry decarbonise?}

\item {} 
\sphinxhref{https://www.sciencedirect.com/science/article/pii/S036054421930177X}{How might residential consumers change their investment decisions over time?}

\item {} 
How might a carbon tax impact investments made in the power sector?

\end{itemize}


\section{How to use MUSE}
\label{\detokenize{overview:how-to-use-muse}}
There are a huge number of ways that MUSE could be used. The energy field is varied and diverse, and many different scenarios can be explored. Users can model the impact of changes in technology prices, demand, policy instruments, sector interactions and much, much more. People are always thinking of new ways that MUSE can be used. So, get creative!

A simulation model of a geographical region or world can be developed and is made up of the following features:
\begin{enumerate}
\sphinxsetlistlabels{\arabic}{enumi}{enumii}{}{.}%
\item {} 
\sphinxstylestrong{Sectors} such as the power sector, gas production sector and the residential sector.

\item {} 
\sphinxstylestrong{Agents} such as a high\sphinxhyphen{}income subsection of the population in the UK or a risk\sphinxhyphen{}averse generation company. These agents are responsible for making investments in energy technologies.

\item {} 
\sphinxstylestrong{Technologies} which the agents choose to adopt. Technologies either produce an energy commodity (e.g. electricity), or a service demand (e.g. building space heating).

\item {} 
\sphinxstylestrong{Service demands} are demands that must be serviced such as lighting, heating or steel production.

\item {} 
\sphinxstylestrong{Market clearing algorithm} is the algorithm which determines global commodity prices based upon the balancing of supply and demand from each of the sectors.

\item {} 
\sphinxstylestrong{Equilibrium prices} are the prices determined by the market clearing algorithm and can determine the investments made by agents in various sectors. This allows for the model to project how the system may develop over a time period.

\end{enumerate}

These features are described in more detail in the rest of this documentation.


\section{What are MUSE’s unique features?}
\label{\detokenize{overview:what-are-muse-s-unique-features}}
MUSE is a generalisable agent\sphinxhyphen{}based modelling environment and simulates energy transitions from the point of view of the investor and consumer agents. This means that users can define their own agents based upon their needs. The fact that MUSE is an agent\sphinxhyphen{}based model means that each of these agents can have different investment behaviours.

Additionally, agent\sphinxhyphen{}based models allow for agents to model imperfect information and limited foresight. An example of this is the ability to model the uncertainty residential users face when predicting the price of gas over the next 25 years. This is a unique feature to agent\sphinxhyphen{}based models when compared to intertemporal optimisation models and more closely models the real world. Many energy systems models are intertemporal optimisation models, which consider the viewpoint of a single benevolent decision maker, with perfect foresight and knowledge. These models optimise energy system investment and operation.

Whilst such intertemporal optimisation models are certainly useful, MUSE is different in that it models the incentives and challenges faced by investors. It can, therefore, be used to investigate different research questions, from the point of view of the investor and consumer. These questions are up to you, so impress us!

MUSE is completely open source, and ready for development.


\section{Visualisation of MUSE}
\label{\detokenize{overview:visualisation-of-muse}}
\noindent{\hspace*{\fill}\sphinxincludegraphics[width=550\sphinxpxdimen]{{muse_overview}.jpg}\hspace*{\fill}}

The figure above displays the key sectors of MUSE:
\begin{itemize}
\item {} 
Primary supply sectors

\item {} 
Conversion sectors

\item {} 
Demand sectors

\item {} 
Climate model

\item {} 
Market clearing algorithm (MCA)

\end{itemize}


\section{How MUSE works}
\label{\detokenize{overview:how-muse-works}}
MUSE works by iterating between sectors shown above to ensure that energy service demands are met by the technologies chosen by the agents. Next, we detail the calculations made by MUSE throughout the simulation.
\begin{enumerate}
\sphinxsetlistlabels{\arabic}{enumi}{enumii}{}{.}%
\item {} 
The energy service demand is calculated. For example, how much electricity, gas and oil demand is there for cooking, building space heating and lighting in the residential sector?

\item {} \begin{description}
\item[{A demand sector is solved. That is, agents choose which end\sphinxhyphen{}use technologies to serve the demands in the sector. For example, electric stoves are compared to gas stoves to meet demand for cooking. These technologies are chosen based upon their:}] \leavevmode\begin{enumerate}
\sphinxsetlistlabels{\roman}{enumii}{enumiii}{}{.}%
\item {} 
Search space (which technologies are they willing to consider?)

\item {} 
Their objectives (which metrics do they consider important?)

\item {} 
Their decision rules (how do they choose to combine their metrics if they have multiple?)

\end{enumerate}

\end{description}

\item {} 
The decisions made by the agents in the demand sectors then leads to a certain level of demand for energy commodities, such as electricity, gas and oil, as a whole. This demand is then passed to the MCA.

\item {} 
The MCA then sends these demands to the sectors that supply these energy commodities (supply or conversion sectors).

\item {} 
The supply and conversion sectors are solved: agents in these sectors use the same approach (i.e. search space, objectives, decision rules) to decide which technologies to investment in to serve the energy commodity demand. For example, agents in the power sector may decide to invest in solar photovoltaics, wind turbines and gas power plants to service the electricity demand.

\item {} 
As a result of these decisions a price for each energy commodity is formed based upon supply and demand. This is passed to the MCA.

\item {} 
The MCA then sends these prices back to the demand sectors, which are solved again as above.

\item {} 
This process repeats itself until commodity supply and demand converges for each energy commodity. Once these converge, the model has found a “partial equilibrium” and it moves forward to the next time period.

\end{enumerate}


\chapter{Key MUSE Components}
\label{\detokenize{muse-components:key-muse-components}}\label{\detokenize{muse-components::doc}}
MUSE is made up of five key components:
\begin{itemize}
\item {} 
Service Demand

\item {} 
Technologies

\item {} 
Sectors

\item {} 
Agents

\item {} 
Market Clearing Algorithm

\end{itemize}

In this section we will briefly explore what these components do and and how they interact.


\section{Service Demand}
\label{\detokenize{muse-components:service-demand}}
The service demand is a user input which defines the demand that an end\sphinxhyphen{}use sector has. An example of this is the service demand commodity of heat or cooling that the residential sector requires. End\sphinxhyphen{}use in this case, refers to the energy which is utilised at the very final stage, after both extraction and conversion.

The estimate of the energy service is the first step. This estimate can be an exogenous input derived from the user, or correlations of GDP and population which reflect the socio\sphinxhyphen{}economic development of a region or country.


\section{Technologies}
\label{\detokenize{muse-components:technologies}}
Users are able to define any technology they wish for each of the energy sectors. Examples include power generators such as coal power plants, buses in the transport sector or lighting in the residential sector.

Each of the technologies are placed in their regions of interest, such as the USA or India. They are then defined by the following, but not limited to, variables:
\begin{itemize}
\item {} 
Capital costs

\item {} 
Fixed costs

\item {} 
Maximum capacity limit

\item {} 
Maximum capaicty growth

\item {} 
Lifetime of the technology

\item {} 
Utilization factor

\item {} 
Interest rate

\end{itemize}

Technologies, and their parameters are defined in the Technodata.csv file. For a full description of the input files, please refer to the {\hyperref[\detokenize{inputs/technodata:inputs-technodata}]{\sphinxcrossref{\DUrole{std,std-ref}{Techno\sphinxhyphen{}data}}}} file.


\section{Sectors}
\label{\detokenize{muse-components:sectors}}
Sectors typically group areas of economic activity together, such as the residential sector, which might include all energy conusming activies of households. Possible examples of sectors are:
\begin{itemize}
\item {} 
Gas sector

\item {} 
Power sector

\item {} 
Residential sector

\item {} 
Industrial sector

\end{itemize}

Each of these sectors contain their respective technologies which consume energy commodities. For example, the residential sector may consume electricity, gas or oil for a variety of different energy demands such as lighting, cooking and heating.

Each of the technologies, which consume a commodity, also output a different commodity or service demands. For example, a gas boiler consumes gas, but outputs heat and hot water.


\section{Agents}
\label{\detokenize{muse-components:agents}}
Agents represent the investment decision makers in an energy system, for example consumers or companies. They invest in technologies that meet service demands, like heating, or produce other needed energy commodities, like electricity. These agents can be heterogenous, meaning that their investment priorities have the ability to differ.

As an example, a generation company could compare potential power generators based on their levelized cost of electricity, their net present value, by minimising the total capital cost, a mixture of these and/or any user\sphinxhyphen{}defined approach. This approach more closely matches the behaviour of real\sphinxhyphen{}life agents in the energy market, where companies, or people, have different priorities and constraints.


\section{Market Clearing Algorithm}
\label{\detokenize{muse-components:market-clearing-algorithm}}
The market clearing algorithm (MCA) is the central component between the different supplies and demands of the energy system in question. The MCA iterates between the demand and supply of each of these sectors. Its role is to govern the endogenous price of commodities over the course of a simulation.

For a hypothetical example, the price of electricity is set to \$70/MWh. However, at this price, the majority of residential agents prefer to heat their homes using gas. As a result of this, residential agents consume less electricity and more gas. This reduction in demand reduces the electricity price to \$50/MWh. However, at this lower electricity price, some agents decide to invest in electric heating as opposed to gas. Eventually, the price converges on \$60/MWh, where supply and demand for both electricity and gas are equal.

This is the principle of the MCA. It finds an equilibrium by iterating through each of the different sectors until an overall equilibrium is reached for each of the commodities. It is possible to run the MCA in a carbon budget mode, as well as exogenous mode. The carbon budget mode ensures that a carbon price limits the amount of carbon produce by the market. Whereas, the exogenous mode allows the carbon price to be set by the user.


\chapter{Customising MUSE Tutorials}
\label{\detokenize{user-guide/index:customising-muse-tutorials}}\label{\detokenize{user-guide/index::doc}}
Next, we show you how to customise MUSE to create your own scenarios.

We recommend following the tutorials step by step, as the files build on the previous example. If you prefer to jump straight in, your results may be different to the ones presented. To help you, we have provided the code to generate the various examples, in case you want to compare your code to ours.


\section{Adding a new technology}
\label{\detokenize{user-guide/add-solar:Adding-a-new-technology}}\label{\detokenize{user-guide/add-solar::doc}}

\subsection{Input Files}
\label{\detokenize{user-guide/add-solar:Input-Files}}
MUSE is made up of a number of different {\hyperref[\detokenize{inputs/index::doc}]{\sphinxcrossref{\DUrole{doc}{input files}}}}. These, however, can be broadly split into two:
\begin{itemize}
\item {} 
{\hyperref[\detokenize{inputs/toml::doc}]{\sphinxcrossref{\DUrole{doc}{Simulation settings}}}}

\item {} 
{\hyperref[\detokenize{inputs/inputs_csv::doc}]{\sphinxcrossref{\DUrole{doc}{Simulation data}}}}

\end{itemize}

Simulation settings specify how a simulation should be run. For example, which sectors to run, for how many years and what to output.

Whereas, simulation data parametrises the technologies involved in the simulation, or the number and kinds of agents.

To create a customised case study it is necessary to edit both of these file types.

Simulation settings are specified in a TOML file. \sphinxhref{https://github.com/toml-lang/toml}{TOML} is a simple, extensible and intuitive file format well suited for specifying small sets of complex data.

Simulation data is specified in \sphinxhref{https://en.wikipedia.org/wiki/Comma-separated\_values}{CSV}. This is a common format used for larger datasets, and is made up of columns and rows, with a comma used to differentiate between entries.

MUSE requires at least the following files to successfully run:
\begin{itemize}
\item {} 
a single {\hyperref[\detokenize{inputs/toml::doc}]{\sphinxcrossref{\DUrole{doc}{simulation settings TOML file}}}} for the simulation as a whole

\item {} 
a file indicating initial market price {\hyperref[\detokenize{inputs/projections::doc}]{\sphinxcrossref{\DUrole{doc}{projections}}}}

\item {} 
a file describing the {\hyperref[\detokenize{inputs/commodities::doc}]{\sphinxcrossref{\DUrole{doc}{commodities in the simulation}}}}

\item {} 
for generalized sectors:
\begin{itemize}
\item {} 
a file descring the {\hyperref[\detokenize{inputs/agents::doc}]{\sphinxcrossref{\DUrole{doc}{agents}}}}

\item {} 
a file descring the {\hyperref[\detokenize{inputs/technodata::doc}]{\sphinxcrossref{\DUrole{doc}{technologies}}}}

\item {} 
a file descring the {\hyperref[\detokenize{inputs/commodities_io::doc}]{\sphinxcrossref{\DUrole{doc}{input commodities}}}} for each technology

\item {} 
a file descring the {\hyperref[\detokenize{inputs/commodities_io::doc}]{\sphinxcrossref{\DUrole{doc}{output commodities}}}} for each technology

\item {} 
a file descring the {\hyperref[\detokenize{inputs/existing_capacity::doc}]{\sphinxcrossref{\DUrole{doc}{existing capacity}}}} of a given sector

\end{itemize}

\item {} 
for each preset sector:
\begin{itemize}
\item {} 
a csv file describing consumption for the duration of the simulation

\end{itemize}

\end{itemize}

For a full description of these files see the {\hyperref[\detokenize{inputs/index::doc}]{\sphinxcrossref{\DUrole{doc}{input files section}}}}. To see how to customise an example, continue on this page.


\subsection{Addition of solar PV}
\label{\detokenize{user-guide/add-solar:Addition-of-solar-PV}}
In this section, we will add solar photovoltaics to the default model seen in the {\hyperref[\detokenize{running-muse-example::doc}]{\sphinxcrossref{\DUrole{doc}{example page}}}}. To achieve this, we must modify some of the input files shown in the above section. These files can be found in the \sphinxcode{\sphinxupquote{StarMuse}} folder at the following location:

\sphinxcode{\sphinxupquote{\{muse\_install\_location\}/src/muse/data/example/default}}

Change \sphinxcode{\sphinxupquote{\{muse\_install\_location\}}} to the location where you installed MUSE using your file browser. You can modify the files in your favourite spreadsheet editor or text editor such as Excel, Numbers, Notepad or TextEdit.


\subsubsection{Technodata Input}
\label{\detokenize{user-guide/add-solar:Technodata-Input}}
Within the default folder there is the \sphinxcode{\sphinxupquote{settings.toml}} file, input folder and technodata folder. To add a technology within the power sector, we must open the \sphinxcode{\sphinxupquote{technodata}} folder followed by the \sphinxcode{\sphinxupquote{power}} folder.

Next, we will edit the \sphinxcode{\sphinxupquote{CommIn.csv}} file, which specifies the commodities consumed by solar photovoltaics.

The table below shows the original \sphinxcode{\sphinxupquote{CommIn.csv}} version in normal text, and the added column and row in \sphinxstylestrong{bold}.


\begin{savenotes}\sphinxattablestart
\centering
\begin{tabular}[t]{|*{10}{\X{1}{10}|}}
\hline
\sphinxstyletheadfamily 
ProcessName
&\sphinxstyletheadfamily 
RegionName
&\sphinxstyletheadfamily 
Time
&\sphinxstyletheadfamily 
Level
&\sphinxstyletheadfamily 
electricity
&\sphinxstyletheadfamily 
gas
&\sphinxstyletheadfamily 
heat
&\sphinxstyletheadfamily 
CO2f
&\sphinxstyletheadfamily 
wind
&\sphinxstyletheadfamily 
\sphinxstylestrong{solar}
\\
\hline
Unit
&\begin{itemize}
\item {} 
\end{itemize}
&
Year
&\begin{itemize}
\item {} 
\end{itemize}
&
PJ/PJ
&
PJ/PJ
&
PJ/PJ
&
kt/PJ
&
PJ/PJ
&
\sphinxstylestrong{PJ/PJ}
\\
\hline
gasCCGT
&
R1
&
2020
&
fixed
&
0
&
1.67
&
0
&
0
&
0
&
\sphinxstylestrong{0}
\\
\hline
windturbine
&
R1
&
2020
&
fixed
&
0
&
0
&
0
&
0
&
1
&
\sphinxstylestrong{0}
\\
\hline
\sphinxstylestrong{solarPV}
&
\sphinxstylestrong{R1}
&
\sphinxstylestrong{2020}
&
\sphinxstylestrong{fixed}
&
\sphinxstylestrong{0}
&
\sphinxstylestrong{0}
&
\sphinxstylestrong{0}
&
\sphinxstylestrong{0}
&
\sphinxstylestrong{0}
&
\sphinxstylestrong{1}
\\
\hline
\end{tabular}
\par
\sphinxattableend\end{savenotes}

We must first add a new row at the bottom of the file, to indicate the new solar photovoltaic technology:
\begin{itemize}
\item {} 
we call this technology \sphinxcode{\sphinxupquote{solarPV}}

\item {} 
place it in region \sphinxcode{\sphinxupquote{R1}}

\item {} 
the data in this row is associated to the year 2020

\item {} 
the input type is fixed

\item {} 
solarPV consumes solar

\end{itemize}

As the solar commodity has not been previously defined, we must define it by adding a column, which we will call solar. We fill out the entries in the solar column, ie. that neither \sphinxcode{\sphinxupquote{gasCCGT}} nor \sphinxcode{\sphinxupquote{windturbine}} consume solar.

We repeat this process for the file: \sphinxcode{\sphinxupquote{CommOut.csv}}. This file specifies the output of the technology. In our case, solar photovoltaics only output \sphinxcode{\sphinxupquote{electricity}}. This is unlike \sphinxcode{\sphinxupquote{gasCCGT}} which also outputs \sphinxcode{\sphinxupquote{CO2f}}, or carbon dioxide.


\begin{savenotes}\sphinxattablestart
\centering
\begin{tabular}[t]{|*{10}{\X{1}{10}|}}
\hline
\sphinxstyletheadfamily 
ProcessName
&\sphinxstyletheadfamily 
RegionName
&\sphinxstyletheadfamily 
Time
&\sphinxstyletheadfamily 
Level
&\sphinxstyletheadfamily 
electricity
&\sphinxstyletheadfamily 
gas
&\sphinxstyletheadfamily 
heat
&\sphinxstyletheadfamily 
CO2f
&\sphinxstyletheadfamily 
wind
&\sphinxstyletheadfamily 
\sphinxstylestrong{solar}
\\
\hline
Unit
&\begin{itemize}
\item {} 
\end{itemize}
&
Year
&\begin{itemize}
\item {} 
\end{itemize}
&
PJ/PJ
&
PJ/PJ
&
PJ/PJ
&
kt/PJ
&
PJ/PJ
&
\sphinxstylestrong{PJ/PJ}
\\
\hline
gasCCGT
&
R1
&
2020
&
fixed
&
1
&
0
&
0
&
91.67
&
0
&
\sphinxstylestrong{0}
\\
\hline
windturbine
&
R1
&
2020
&
fixed
&
1
&
0
&
0
&
0
&
0
&
\sphinxstylestrong{0}
\\
\hline
\sphinxstylestrong{solarPV}
&
\sphinxstylestrong{R1}
&
\sphinxstylestrong{2020}
&
\sphinxstylestrong{fixed}
&
\sphinxstylestrong{1}
&
\sphinxstylestrong{0}
&
\sphinxstylestrong{0}
&
\sphinxstylestrong{0}
&
\sphinxstylestrong{0}
&
\sphinxstylestrong{0}
\\
\hline
\end{tabular}
\par
\sphinxattableend\end{savenotes}

Similar to the the \sphinxcode{\sphinxupquote{CommIn.csv}}, we create a new row, and add in the solar commodity. We must ensure that we call our new commodity and technologies the same as the previous file for MUSE to successfully run. ie \sphinxcode{\sphinxupquote{solar}} and \sphinxcode{\sphinxupquote{solarPV}}

The next file to modify is the \sphinxcode{\sphinxupquote{ExistingCapacity.csv}} file. This file details the existing capacity of each technology, per year. For this example, we will set the existing capacity to be 0.


\begin{savenotes}\sphinxattablestart
\centering
\begin{tabulary}{\linewidth}[t]{|T|T|T|T|T|T|T|T|T|T|}
\hline
\sphinxstyletheadfamily 
ProcessName
&\sphinxstyletheadfamily 
RegionName
&\sphinxstyletheadfamily 
Unit
&\sphinxstyletheadfamily 
2020
&\sphinxstyletheadfamily 
2025
&\sphinxstyletheadfamily 
2030
&\sphinxstyletheadfamily 
2035
&\sphinxstyletheadfamily 
2040
&\sphinxstyletheadfamily 
2045
&\sphinxstyletheadfamily 
2050
\\
\hline
gasCCGT
&
R1
&
PJ/y
&
1
&
1
&
0
&
0
&
0
&
0
&
0
\\
\hline
windturbine
&
R1
&
PJ/y
&
0
&
0
&
0
&
0
&
0
&
0
&
0
\\
\hline
\sphinxstylestrong{solarPV}
&
\sphinxstylestrong{R1}
&
\sphinxstylestrong{PJ/y}
&
\sphinxstylestrong{0}
&
\sphinxstylestrong{0}
&
\sphinxstylestrong{0}
&
\sphinxstylestrong{0}
&
\sphinxstylestrong{0}
&
\sphinxstylestrong{0}
&
\sphinxstylestrong{0}
\\
\hline
\end{tabulary}
\par
\sphinxattableend\end{savenotes}

Finally, the \sphinxcode{\sphinxupquote{technodata.csv}} containts parametrisation data for the technology, such as the cost, growth constraints, lifetime of the power plant and fuel used. The technodata file is too long for it all to be displayed here, so we will truncate the full version.

Here, we will only define the parameters: \sphinxcode{\sphinxupquote{processName}}, \sphinxcode{\sphinxupquote{RegionName}}, \sphinxcode{\sphinxupquote{Time}}, \sphinxcode{\sphinxupquote{Level}},\sphinxcode{\sphinxupquote{cap\_par}}, \sphinxcode{\sphinxupquote{Fuel}},\sphinxcode{\sphinxupquote{EndUse}},\sphinxcode{\sphinxupquote{Agent2}} and \sphinxcode{\sphinxupquote{Agent1}}

We shall copy the existing parameters from the \sphinxcode{\sphinxupquote{windturbine}} technology for the remaining parameters that can be seen in the \sphinxcode{\sphinxupquote{technodata.csv}} file for brevity. You can see the full file \sphinxhref{here}{here INSERT LINK HERE}


\begin{savenotes}\sphinxattablestart
\centering
\begin{tabular}[t]{|*{11}{\X{1}{11}|}}
\hline
\sphinxstyletheadfamily 
ProcessName
&\sphinxstyletheadfamily 
RegionName
&\sphinxstyletheadfamily 
Time
&\sphinxstyletheadfamily 
Level
&\sphinxstyletheadfamily 
cap\_par
&\sphinxstyletheadfamily 
cap\_exp
&\sphinxstyletheadfamily 
…
&\sphinxstyletheadfamily 
Fuel
&\sphinxstyletheadfamily 
EndUse
&\sphinxstyletheadfamily 
Agent2
&\sphinxstyletheadfamily 
Agent1
\\
\hline
Unit
&\begin{itemize}
\item {} 
\end{itemize}
&
Year
&\begin{itemize}
\item {} 
\end{itemize}
&
MUS\$2010/PJ\_a
&\begin{itemize}
\item {} 
\end{itemize}
&
…
&\begin{itemize}
\item {} 
\end{itemize}
&\begin{itemize}
\item {} 
\end{itemize}
&
Retrofit
&
New
\\
\hline
gasCCGT
&
R1
&
2020
&
fixed
&
23.78234399
&
1
&
…
&
gas
&
electricity
&
1
&
0
\\
\hline
windturbine
&
R1
&
2020
&
fixed
&
36.30771182
&
1
&
…
&
wind
&
electricity
&
1
&
0
\\
\hline
\sphinxstylestrong{solarPV}
&
\sphinxstylestrong{R1}
&
\sphinxstylestrong{2020}
&
\sphinxstylestrong{fixed}
&
\sphinxstylestrong{30}
&
\sphinxstylestrong{1}
&
…
&
\sphinxstylestrong{solar}
&
\sphinxstylestrong{electricity}
&
\sphinxstylestrong{1}
&
\sphinxstylestrong{0}
\\
\hline
\end{tabular}
\par
\sphinxattableend\end{savenotes}


\subsubsection{Global inputs}
\label{\detokenize{user-guide/add-solar:Global-inputs}}
Next, navigate to the \sphinxcode{\sphinxupquote{input}} folder, found at \sphinxcode{\sphinxupquote{\{muse\_installation\_location\}src/muse/data/example/default/input}}.

We now must edit each of the files found here to add the new solar commodity. Due to space constraints we will not display all of the entries contained in the input files. The edited files can be viewed \sphinxhref{here}{here INSERT LINK HERE} however.

The \sphinxcode{\sphinxupquote{BaseYearExport.csv}} file defines the exports in the base year. For our example we add a column to indicate that there is no export for solar. However, it is important that a column exists for our new commodity.


\begin{savenotes}\sphinxattablestart
\centering
\begin{tabular}[t]{|*{9}{\X{1}{9}|}}
\hline
\sphinxstyletheadfamily 
RegionName
&\sphinxstyletheadfamily 
Attribute
&\sphinxstyletheadfamily 
Time
&\sphinxstyletheadfamily 
electricity
&\sphinxstyletheadfamily 
gas
&\sphinxstyletheadfamily 
heat
&\sphinxstyletheadfamily 
CO2f
&\sphinxstyletheadfamily 
wind
&\sphinxstyletheadfamily 
\sphinxstylestrong{solar}
\\
\hline
Unit
&\begin{itemize}
\item {} 
\end{itemize}
&
Year
&
PJ
&
PJ
&
PJ
&
kt
&
PJ
&
\sphinxstylestrong{PJ}
\\
\hline
R1
&
Exports
&
2010
&
0
&
0
&
0
&
0
&
0
&
\sphinxstylestrong{0}
\\
\hline
R1
&
Exports
&
2015
&
0
&
0
&
0
&
0
&
0
&
\sphinxstylestrong{0}
\\
\hline
…
&
…
&
…
&
…
&
…
&
…
&
…
&
…
&
\sphinxstylestrong{…}
\\
\hline
R1
&
Exports
&
2100
&
0
&
0
&
0
&
0
&
0
&
\sphinxstylestrong{0}
\\
\hline
\end{tabular}
\par
\sphinxattableend\end{savenotes}

The \sphinxcode{\sphinxupquote{BaseYearImport.csv}} file defines the imports in the base year. Similarly to \sphinxcode{\sphinxupquote{BaseYearExport.csv}}, we add a column for solar in the \sphinxcode{\sphinxupquote{BaseYearImport.csv}} file. Again, we indicate that solar has no imports.


\begin{savenotes}\sphinxattablestart
\centering
\begin{tabular}[t]{|*{9}{\X{1}{9}|}}
\hline
\sphinxstyletheadfamily 
RegionName
&\sphinxstyletheadfamily 
Attribute
&\sphinxstyletheadfamily 
Time
&\sphinxstyletheadfamily 
electricity
&\sphinxstyletheadfamily 
gas
&\sphinxstyletheadfamily 
heat
&\sphinxstyletheadfamily 
CO2f
&\sphinxstyletheadfamily 
wind
&\sphinxstyletheadfamily 
\sphinxstylestrong{solar}
\\
\hline
Unit
&\begin{itemize}
\item {} 
\end{itemize}
&
Year
&
PJ
&
PJ
&
PJ
&
kt
&
PJ
&
\sphinxstylestrong{PJ}
\\
\hline
R1
&
Imports
&
2010
&
0
&
0
&
0
&
0
&
0
&
\sphinxstylestrong{0}
\\
\hline
R1
&
Imports
&
2015
&
0
&
0
&
0
&
0
&
0
&
\sphinxstylestrong{0}
\\
\hline
…
&
…
&
…
&
…
&
…
&
…
&
…
&
…
&
\sphinxstylestrong{…}
\\
\hline
R1
&
Imports
&
2100
&
0
&
0
&
0
&
0
&
0
&
\sphinxstylestrong{0}
\\
\hline
\end{tabular}
\par
\sphinxattableend\end{savenotes}

The \sphinxcode{\sphinxupquote{GlobalCommodities.csv}} file is the file which defines the commodities. Here we give the commodities a commodity type, CO2 emissions factor and heat rate. For this file, we will add the solar commodity, with zero CO2 emissions factor and a heat rate of 1.


\begin{savenotes}\sphinxattablestart
\centering
\begin{tabulary}{\linewidth}[t]{|T|T|T|T|T|T|}
\hline
\sphinxstyletheadfamily 
Commodity
&\sphinxstyletheadfamily 
CommodityType
&\sphinxstyletheadfamily 
CommodityName
&\sphinxstyletheadfamily 
CommodityEmissionFactor\_CO2
&\sphinxstyletheadfamily 
HeatRate
&\sphinxstyletheadfamily 
Unit
\\
\hline
Electricity
&
Energy
&
electricity
&
0
&
1
&
PJ
\\
\hline
Gas
&
Energy
&
gas
&
56.1
&
1
&
PJ
\\
\hline
Heat
&
Energy
&
heat
&
0
&
1
&
PJ
\\
\hline
Wind
&
Energy
&
wind
&
0
&
1
&
PJ
\\
\hline
CO2fuelcomsbustion
&
Environmental
&
CO2f
&
0
&
1
&
kt
\\
\hline
\sphinxstylestrong{Solar}
&
\sphinxstylestrong{Energy}
&
\sphinxstylestrong{solar}
&
\sphinxstylestrong{0}
&
\sphinxstylestrong{1}
&
\sphinxstylestrong{PJ}
\\
\hline
\end{tabulary}
\par
\sphinxattableend\end{savenotes}

The \sphinxcode{\sphinxupquote{projections.csv}} file details the initial market prices for the commodities. The market clearing algorithm will update these throughout the simulation, however, an initial estimate is required to start the simulation. As solar energy is free, we will indicate this by adding a final column.


\begin{savenotes}\sphinxattablestart
\centering
\begin{tabular}[t]{|*{9}{\X{1}{9}|}}
\hline
\sphinxstyletheadfamily 
RegionName
&\sphinxstyletheadfamily 
Attribute
&\sphinxstyletheadfamily 
Time
&\sphinxstyletheadfamily 
electricity
&\sphinxstyletheadfamily 
gas
&\sphinxstyletheadfamily 
heat
&\sphinxstyletheadfamily 
CO2f
&\sphinxstyletheadfamily 
wind
&\sphinxstyletheadfamily 
\sphinxstylestrong{solar}
\\
\hline
Unit
&\begin{itemize}
\item {} 
\end{itemize}
&
Year
&
MUS\$2010/PJ
&
MUS\$2010/PJ
&
MUS\$2010/PJ
&
MUS\$2010/kt
&
MUS\$2010/kt
&
\sphinxstylestrong{MUS\$2010/kt}
\\
\hline
R1
&
CommodityPrice
&
2010
&
14.81481472
&
6.6759
&
100
&
0
&
0
&
\sphinxstylestrong{0}
\\
\hline
R1
&
CommodityPrice
&
2015
&
17.89814806
&
6.914325
&
100
&
0.052913851
&
0
&
\sphinxstylestrong{0}
\\
\hline
…
&
…
&
…
&
…
&
…
&
…
&
…
&
…
&
\sphinxstylestrong{…}
\\
\hline
R1
&
CommodityPrice
&
2100
&
21.39814806
&
7.373485819
&
100
&
1.871299697
&
0
&
\sphinxstylestrong{0}
\\
\hline
\end{tabular}
\par
\sphinxattableend\end{savenotes}


\subsubsection{Running our customised simulation}
\label{\detokenize{user-guide/add-solar:Running-our-customised-simulation}}
Now we are able to run our simulation, with the new solar power technology.

To do this we run the same run command as previously in the anaconda command prompt:

\sphinxcode{\sphinxupquote{python \sphinxhyphen{}m muse settings.toml}}

The output should be similar to the output here. However, expect the simulation to take slightly longer to run. This is due to the additional calculations made.

If the simulation has run successfully, you should now have a folder in the same location as your settings.toml file called Results. The next step is to visualise the results using the python visualisation package \sphinxcode{\sphinxupquote{seaborn}} as well as the data analysis library \sphinxcode{\sphinxupquote{pandas}}.

{
\sphinxsetup{VerbatimColor={named}{nbsphinx-code-bg}}
\sphinxsetup{VerbatimBorderColor={named}{nbsphinx-code-border}}
\begin{sphinxVerbatim}[commandchars=\\\{\}]
\llap{\color{nbsphinxin}[2]:\,\hspace{\fboxrule}\hspace{\fboxsep}}\PYG{k+kn}{import} \PYG{n+nn}{seaborn} \PYG{k}{as} \PYG{n+nn}{sns}
\PYG{k+kn}{import} \PYG{n+nn}{pandas} \PYG{k}{as} \PYG{n+nn}{pd}
\end{sphinxVerbatim}
}

Next, we will import the \sphinxcode{\sphinxupquote{MCACapacity.csv}} file into pandas and print the first 5 lines using the \sphinxcode{\sphinxupquote{head()}} command.

Make sure to change the file path of \sphinxcode{\sphinxupquote{"../Results/MCACapacity.csv"}} to where the \sphinxcode{\sphinxupquote{MCACapacity.csv}} is on your computer, otherwise you will receive an error when you import the csv file.

{
\sphinxsetup{VerbatimColor={named}{nbsphinx-code-bg}}
\sphinxsetup{VerbatimBorderColor={named}{nbsphinx-code-border}}
\begin{sphinxVerbatim}[commandchars=\\\{\}]
\llap{\color{nbsphinxin}[6]:\,\hspace{\fboxrule}\hspace{\fboxsep}}\PYG{n}{mca\PYGZus{}capacity} \PYG{o}{=} \PYG{n}{pd}\PYG{o}{.}\PYG{n}{read\PYGZus{}csv}\PYG{p}{(}\PYG{l+s+s2}{\PYGZdq{}}\PYG{l+s+s2}{../Results/MCACapacity.csv}\PYG{l+s+s2}{\PYGZdq{}}\PYG{p}{)}
\PYG{n}{mca\PYGZus{}capacity}\PYG{o}{.}\PYG{n}{head}\PYG{p}{(}\PYG{p}{)}
\end{sphinxVerbatim}
}

{

\kern-\sphinxverbatimsmallskipamount\kern-\baselineskip
\kern+\FrameHeightAdjust\kern-\fboxrule
\vspace{\nbsphinxcodecellspacing}

\sphinxsetup{VerbatimColor={named}{white}}
\sphinxsetup{VerbatimBorderColor={named}{nbsphinx-code-border}}
\begin{sphinxVerbatim}[commandchars=\\\{\}]
\llap{\color{nbsphinxout}[6]:\,\hspace{\fboxrule}\hspace{\fboxsep}}   technology region agent      type       sector  capacity  year
0   gasboiler     R1    A1  retrofit  residential      10.0  2020
1     gasCCGT     R1    A1  retrofit        power       1.0  2020
2  gassupply1     R1    A1  retrofit          gas      15.0  2020
3   gasboiler     R1    A1  retrofit  residential       5.0  2025
4    heatpump     R1    A1  retrofit  residential      19.0  2025
\end{sphinxVerbatim}
}

We will only visualise the power sector in this example, as this was the only sector we changed. We, therefore, filter for this sector, and then visualise it using \sphinxcode{\sphinxupquote{seaborn}}:

{
\sphinxsetup{VerbatimColor={named}{nbsphinx-code-bg}}
\sphinxsetup{VerbatimBorderColor={named}{nbsphinx-code-border}}
\begin{sphinxVerbatim}[commandchars=\\\{\}]
\llap{\color{nbsphinxin}[7]:\,\hspace{\fboxrule}\hspace{\fboxsep}}\PYG{n}{power\PYGZus{}capacity} \PYG{o}{=} \PYG{n}{mca\PYGZus{}capacity}\PYG{p}{[}\PYG{n}{mca\PYGZus{}capacity}\PYG{o}{.}\PYG{n}{sector}\PYG{o}{==}\PYG{l+s+s2}{\PYGZdq{}}\PYG{l+s+s2}{power}\PYG{l+s+s2}{\PYGZdq{}}\PYG{p}{]}
\PYG{n}{sns}\PYG{o}{.}\PYG{n}{lineplot}\PYG{p}{(}\PYG{n}{data}\PYG{o}{=}\PYG{n}{power\PYGZus{}capacity}\PYG{p}{,} \PYG{n}{x}\PYG{o}{=}\PYG{l+s+s1}{\PYGZsq{}}\PYG{l+s+s1}{year}\PYG{l+s+s1}{\PYGZsq{}}\PYG{p}{,} \PYG{n}{y}\PYG{o}{=}\PYG{l+s+s1}{\PYGZsq{}}\PYG{l+s+s1}{capacity}\PYG{l+s+s1}{\PYGZsq{}}\PYG{p}{,} \PYG{n}{hue}\PYG{o}{=}\PYG{l+s+s2}{\PYGZdq{}}\PYG{l+s+s2}{technology}\PYG{l+s+s2}{\PYGZdq{}}\PYG{p}{)}
\end{sphinxVerbatim}
}

{

\kern-\sphinxverbatimsmallskipamount\kern-\baselineskip
\kern+\FrameHeightAdjust\kern-\fboxrule
\vspace{\nbsphinxcodecellspacing}

\sphinxsetup{VerbatimColor={named}{white}}
\sphinxsetup{VerbatimBorderColor={named}{nbsphinx-code-border}}
\begin{sphinxVerbatim}[commandchars=\\\{\}]
\llap{\color{nbsphinxout}[7]:\,\hspace{\fboxrule}\hspace{\fboxsep}}<matplotlib.axes.\_subplots.AxesSubplot at 0x7fd544a6e760>
\end{sphinxVerbatim}
}

\hrule height -\fboxrule\relax
\vspace{\nbsphinxcodecellspacing}

\makeatletter\setbox\nbsphinxpromptbox\box\voidb@x\makeatother

\begin{nbsphinxfancyoutput}

\noindent\sphinxincludegraphics[width=382\sphinxpxdimen,height=262\sphinxpxdimen]{{user-guide_add-solar_25_1}.png}

\end{nbsphinxfancyoutput}

We can now see that there is solarPV in addition to windturbine and gasCCGT, when compared to the example {\hyperref[\detokenize{running-muse-example::doc}]{\sphinxcrossref{\DUrole{doc}{here}}}}! That’s great and means it worked!

The difference in uptake of \sphinxcode{\sphinxupquote{solarPV}} compared to \sphinxcode{\sphinxupquote{windturbine}} is due to the fact that \sphinxcode{\sphinxupquote{solarPV}} has a lower \sphinxcode{\sphinxupquote{cap\_par}} cost of 30, compared to the \sphinxcode{\sphinxupquote{windturbine}}. Meaning that \sphinxcode{\sphinxupquote{solarPV}} outcompetes both \sphinxcode{\sphinxupquote{windturbine}} and \sphinxcode{\sphinxupquote{gasCCGT}} in the electricity market.


\subsubsection{Change Solar Price}
\label{\detokenize{user-guide/add-solar:Change-Solar-Price}}
Now, we will observe what happens if we increase the price of solar to be more expensive than wind. To achieve, this we have to modify the \sphinxcode{\sphinxupquote{Technodata.csv}} file:


\begin{savenotes}\sphinxattablestart
\centering
\begin{tabular}[t]{|*{11}{\X{1}{11}|}}
\hline
\sphinxstyletheadfamily 
ProcessName
&\sphinxstyletheadfamily 
RegionName
&\sphinxstyletheadfamily 
Time
&\sphinxstyletheadfamily 
Level
&\sphinxstyletheadfamily 
cap\_par
&\sphinxstyletheadfamily 
cap\_exp
&\sphinxstyletheadfamily 
…
&\sphinxstyletheadfamily 
Fuel
&\sphinxstyletheadfamily 
EndUse
&\sphinxstyletheadfamily 
Agent2
&\sphinxstyletheadfamily 
Agent1
\\
\hline
Unit
&\begin{itemize}
\item {} 
\end{itemize}
&
Year
&\begin{itemize}
\item {} 
\end{itemize}
&
MUS\$2010/PJ\_a
&\begin{itemize}
\item {} 
\end{itemize}
&
…
&\begin{itemize}
\item {} 
\end{itemize}
&\begin{itemize}
\item {} 
\end{itemize}
&
Retrofit
&
New
\\
\hline
gasCCGT
&
R1
&
2020
&
fixed
&
23.78234399
&
1
&
…
&
gas
&
electricity
&
1
&
0
\\
\hline
windturbine
&
R1
&
2020
&
fixed
&
36.30771182
&
1
&
…
&
wind
&
electricity
&
1
&
0
\\
\hline
solarPV
&
R1
&
2020
&
fixed
&
\sphinxstylestrong{40}
&
1
&
…
&
solar
&
electricity
&
1
&
0
\\
\hline
\end{tabular}
\par
\sphinxattableend\end{savenotes}

Here, we increase the \sphinxcode{\sphinxupquote{cap\_par}} variable by 10, to be a total of 40. We will now rerun the simulation, using the same command as previously and visualise the new results.

We must import the new \sphinxcode{\sphinxupquote{MCACapacity.csv}} file again, and then visualise the results.

{
\sphinxsetup{VerbatimColor={named}{nbsphinx-code-bg}}
\sphinxsetup{VerbatimBorderColor={named}{nbsphinx-code-border}}
\begin{sphinxVerbatim}[commandchars=\\\{\}]
\llap{\color{nbsphinxin}[8]:\,\hspace{\fboxrule}\hspace{\fboxsep}}\PYG{n}{mca\PYGZus{}capacity} \PYG{o}{=} \PYG{n}{pd}\PYG{o}{.}\PYG{n}{read\PYGZus{}csv}\PYG{p}{(}\PYG{l+s+s2}{\PYGZdq{}}\PYG{l+s+s2}{../Results/MCACapacity.csv}\PYG{l+s+s2}{\PYGZdq{}}\PYG{p}{)}
\PYG{n}{power\PYGZus{}capacity} \PYG{o}{=} \PYG{n}{mca\PYGZus{}capacity}\PYG{p}{[}\PYG{n}{mca\PYGZus{}capacity}\PYG{o}{.}\PYG{n}{sector}\PYG{o}{==}\PYG{l+s+s2}{\PYGZdq{}}\PYG{l+s+s2}{power}\PYG{l+s+s2}{\PYGZdq{}}\PYG{p}{]}
\PYG{n}{sns}\PYG{o}{.}\PYG{n}{lineplot}\PYG{p}{(}\PYG{n}{data}\PYG{o}{=}\PYG{n}{power\PYGZus{}capacity}\PYG{p}{,} \PYG{n}{x}\PYG{o}{=}\PYG{l+s+s1}{\PYGZsq{}}\PYG{l+s+s1}{year}\PYG{l+s+s1}{\PYGZsq{}}\PYG{p}{,} \PYG{n}{y}\PYG{o}{=}\PYG{l+s+s1}{\PYGZsq{}}\PYG{l+s+s1}{capacity}\PYG{l+s+s1}{\PYGZsq{}}\PYG{p}{,} \PYG{n}{hue}\PYG{o}{=}\PYG{l+s+s2}{\PYGZdq{}}\PYG{l+s+s2}{technology}\PYG{l+s+s2}{\PYGZdq{}}\PYG{p}{)}
\end{sphinxVerbatim}
}

{

\kern-\sphinxverbatimsmallskipamount\kern-\baselineskip
\kern+\FrameHeightAdjust\kern-\fboxrule
\vspace{\nbsphinxcodecellspacing}

\sphinxsetup{VerbatimColor={named}{white}}
\sphinxsetup{VerbatimBorderColor={named}{nbsphinx-code-border}}
\begin{sphinxVerbatim}[commandchars=\\\{\}]
\llap{\color{nbsphinxout}[8]:\,\hspace{\fboxrule}\hspace{\fboxsep}}<matplotlib.axes.\_subplots.AxesSubplot at 0x7fd544ef7ee0>
\end{sphinxVerbatim}
}

\hrule height -\fboxrule\relax
\vspace{\nbsphinxcodecellspacing}

\makeatletter\setbox\nbsphinxpromptbox\box\voidb@x\makeatother

\begin{nbsphinxfancyoutput}

\noindent\sphinxincludegraphics[width=382\sphinxpxdimen,height=262\sphinxpxdimen]{{user-guide_add-solar_30_1}.png}

\end{nbsphinxfancyoutput}

Now, we can see that the technology \sphinxcode{\sphinxupquote{windturbine}} outcompetes \sphinxcode{\sphinxupquote{solarPV}} and \sphinxcode{\sphinxupquote{gasCCGT}} due to the difference in price. The possibilities for creating your own scenarios are infinite.

For the full example with the completed input files see \sphinxhref{dead-link}{here INSERT LINK HERE}


\subsection{Next steps}
\label{\detokenize{user-guide/add-solar:Next-steps}}
In the next section we will add a new agent to the simulation.


\section{Adding an agent}
\label{\detokenize{user-guide/add-agent:Adding-an-agent}}\label{\detokenize{user-guide/add-agent::doc}}
In this section, we will add a new agent called \sphinxcode{\sphinxupquote{A2}}. This agent will be slightly different to the other agents in the \sphinxcode{\sphinxupquote{default}} example, in that it will make investments based upon a mixture of \sphinxhref{https://en.wikipedia.org/wiki/Levelized\_cost\_of\_energy}{levelised cost of electricity (LCOE)} and \sphinxhref{https://en.wikipedia.org/wiki/Net\_present\_value}{net present value (NPV)}. These two objectives will be combined by calculating the mean of the two when comparing potential investment options.

To achieve this, we must modify the \sphinxcode{\sphinxupquote{Agents.csv}} file in the directory:

\begin{sphinxVerbatim}[commandchars=\\\{\}]
\PYG{p}{\PYGZob{}}\PYG{n}{muse\PYGZus{}install\PYGZus{}location}\PYG{p}{\PYGZcb{}}\PYG{o}{/}\PYG{n}{src}\PYG{o}{/}\PYG{n}{muse}\PYG{o}{/}\PYG{n}{data}\PYG{o}{/}\PYG{n}{example}\PYG{o}{/}\PYG{n}{default}\PYG{o}{/}\PYG{n}{technodata}\PYG{o}{/}\PYG{n}{Agents}\PYG{o}{.}\PYG{n}{csv}
\end{sphinxVerbatim}

To do this, we will add two new rows to the file. To simplify the process, we copy the data from the first two rows of agent \sphinxcode{\sphinxupquote{A1}}, changing only the rows: \sphinxcode{\sphinxupquote{Name}}, \sphinxcode{\sphinxupquote{Objective1}}, \sphinxcode{\sphinxupquote{Objective2}}, \sphinxcode{\sphinxupquote{ObjData1}}, \sphinxcode{\sphinxupquote{ObjData2}} and \sphinxcode{\sphinxupquote{DecisionMethod}}. The values we changed can be seen below. Again, we only show some of the rows due to space constraints, however see \sphinxhref{broken-link}{here} for the full file.


\begin{savenotes}\sphinxattablestart
\centering
\begin{tabulary}{\linewidth}[t]{|T|T|T|T|T|T|T|T|T|T|T|T|T|}
\hline
\sphinxstyletheadfamily 
AgentShare
&\sphinxstyletheadfamily 
Name
&\sphinxstyletheadfamily 
AgentNumber
&\sphinxstyletheadfamily 
RegionName
&\sphinxstyletheadfamily 
Objective1
&\sphinxstyletheadfamily 
Objective2
&\sphinxstyletheadfamily 
Objective3
&\sphinxstyletheadfamily 
ObjData1
&\sphinxstyletheadfamily 
ObjData2
&\sphinxstyletheadfamily 
…
&\sphinxstyletheadfamily 
DecisionMethod
&\sphinxstyletheadfamily 
…
&\sphinxstyletheadfamily 
Type
\\
\hline
Agent1
&
A1
&
1
&
R1
&
LCOE
&&&
1
&&
…
&
singleObj
&
…
&
New
\\
\hline
Agent2
&
A1
&
2
&
R1
&
LCOE
&&&
1
&&
…
&
singleObj
&
…
&
Retrofit
\\
\hline
\sphinxstylestrong{Agent1}
&
\sphinxstylestrong{A2}
&
\sphinxstylestrong{1}
&
\sphinxstylestrong{R1}
&
\sphinxstylestrong{LCOE}
&
\sphinxstylestrong{NPV}
&&
\sphinxstylestrong{1}
&
\sphinxstylestrong{1}
&
\sphinxstylestrong{…}
&
\sphinxstylestrong{mean}
&
\sphinxstylestrong{…}
&
\sphinxstylestrong{New}
\\
\hline
\sphinxstylestrong{Agent2}
&
\sphinxstylestrong{A2}
&
\sphinxstylestrong{2}
&
\sphinxstylestrong{R1}
&
\sphinxstylestrong{LCOE}
&
\sphinxstylestrong{NPV}
&&
\sphinxstylestrong{1}
&
\sphinxstylestrong{1}
&
\sphinxstylestrong{…}
&
\sphinxstylestrong{mean}
&
\sphinxstylestrong{…}
&
\sphinxstylestrong{Retrofit}
\\
\hline
\end{tabulary}
\par
\sphinxattableend\end{savenotes}

We will now save this file and run the new simulation model using the following command:

\begin{sphinxVerbatim}[commandchars=\\\{\}]
\PYG{n}{python} \PYG{o}{\PYGZhy{}}\PYG{n}{m} \PYG{n}{muse} \PYG{n}{settings}\PYG{o}{.}\PYG{n}{toml}
\end{sphinxVerbatim}

Again, we use seaborn and pandas to analyse the data in the \sphinxcode{\sphinxupquote{Results}} folder.

{
\sphinxsetup{VerbatimColor={named}{nbsphinx-code-bg}}
\sphinxsetup{VerbatimBorderColor={named}{nbsphinx-code-border}}
\begin{sphinxVerbatim}[commandchars=\\\{\}]
\llap{\color{nbsphinxin}[6]:\,\hspace{\fboxrule}\hspace{\fboxsep}}\PYG{k+kn}{import} \PYG{n+nn}{pandas} \PYG{k}{as} \PYG{n+nn}{pd}
\PYG{k+kn}{import} \PYG{n+nn}{seaborn} \PYG{k}{as} \PYG{n+nn}{sns}
\end{sphinxVerbatim}
}

{
\sphinxsetup{VerbatimColor={named}{nbsphinx-code-bg}}
\sphinxsetup{VerbatimBorderColor={named}{nbsphinx-code-border}}
\begin{sphinxVerbatim}[commandchars=\\\{\}]
\llap{\color{nbsphinxin}[19]:\,\hspace{\fboxrule}\hspace{\fboxsep}}\PYG{n}{mca\PYGZus{}capacity} \PYG{o}{=} \PYG{n}{pd}\PYG{o}{.}\PYG{n}{read\PYGZus{}csv}\PYG{p}{(}\PYG{l+s+s2}{\PYGZdq{}}\PYG{l+s+s2}{../Results/MCACapacity.csv}\PYG{l+s+s2}{\PYGZdq{}}\PYG{p}{)}
\PYG{n}{power\PYGZus{}sector} \PYG{o}{=} \PYG{n}{mca\PYGZus{}capacity}\PYG{p}{[}\PYG{n}{mca\PYGZus{}capacity}\PYG{o}{.}\PYG{n}{sector}\PYG{o}{==}\PYG{l+s+s2}{\PYGZdq{}}\PYG{l+s+s2}{power}\PYG{l+s+s2}{\PYGZdq{}}\PYG{p}{]}
\PYG{n}{power\PYGZus{}sector}\PYG{o}{.}\PYG{n}{head}\PYG{p}{(}\PYG{p}{)}
\end{sphinxVerbatim}
}

{

\kern-\sphinxverbatimsmallskipamount\kern-\baselineskip
\kern+\FrameHeightAdjust\kern-\fboxrule
\vspace{\nbsphinxcodecellspacing}

\sphinxsetup{VerbatimColor={named}{white}}
\sphinxsetup{VerbatimBorderColor={named}{nbsphinx-code-border}}
\begin{sphinxVerbatim}[commandchars=\\\{\}]
\llap{\color{nbsphinxout}[19]:\,\hspace{\fboxrule}\hspace{\fboxsep}}     technology region agent      type sector  capacity  year
2       gasCCGT     R1    A1  retrofit  power     1.000  2020
3       gasCCGT     R1    A2  retrofit  power     1.000  2020
10      gasCCGT     R1    A1  retrofit  power     1.000  2025
11  windturbine     R1    A1  retrofit  power     5.172  2025
12      gasCCGT     R1    A2  retrofit  power    11.000  2025
\end{sphinxVerbatim}
}

This time we can see that there is data for the new agent, \sphinxcode{\sphinxupquote{A2}}. Next, we will visualise the investments made by each of the agents using seaborn’s facetgrid command.

{
\sphinxsetup{VerbatimColor={named}{nbsphinx-code-bg}}
\sphinxsetup{VerbatimBorderColor={named}{nbsphinx-code-border}}
\begin{sphinxVerbatim}[commandchars=\\\{\}]
\llap{\color{nbsphinxin}[20]:\,\hspace{\fboxrule}\hspace{\fboxsep}}\PYG{n}{g}\PYG{o}{=}\PYG{n}{sns}\PYG{o}{.}\PYG{n}{FacetGrid}\PYG{p}{(}\PYG{n}{power\PYGZus{}sector}\PYG{p}{,} \PYG{n}{row}\PYG{o}{=}\PYG{l+s+s1}{\PYGZsq{}}\PYG{l+s+s1}{agent}\PYG{l+s+s1}{\PYGZsq{}}\PYG{p}{)}
\PYG{n}{g}\PYG{o}{.}\PYG{n}{map}\PYG{p}{(}\PYG{n}{sns}\PYG{o}{.}\PYG{n}{lineplot}\PYG{p}{,} \PYG{l+s+s2}{\PYGZdq{}}\PYG{l+s+s2}{year}\PYG{l+s+s2}{\PYGZdq{}}\PYG{p}{,} \PYG{l+s+s2}{\PYGZdq{}}\PYG{l+s+s2}{capacity}\PYG{l+s+s2}{\PYGZdq{}}\PYG{p}{,} \PYG{l+s+s2}{\PYGZdq{}}\PYG{l+s+s2}{technology}\PYG{l+s+s2}{\PYGZdq{}}\PYG{p}{)}
\PYG{n}{g}\PYG{o}{.}\PYG{n}{add\PYGZus{}legend}\PYG{p}{(}\PYG{p}{)}
\end{sphinxVerbatim}
}

{

\kern-\sphinxverbatimsmallskipamount\kern-\baselineskip
\kern+\FrameHeightAdjust\kern-\fboxrule
\vspace{\nbsphinxcodecellspacing}

\sphinxsetup{VerbatimColor={named}{white}}
\sphinxsetup{VerbatimBorderColor={named}{nbsphinx-code-border}}
\begin{sphinxVerbatim}[commandchars=\\\{\}]
\llap{\color{nbsphinxout}[20]:\,\hspace{\fboxrule}\hspace{\fboxsep}}<seaborn.axisgrid.FacetGrid at 0x7f95819b4730>
\end{sphinxVerbatim}
}

\hrule height -\fboxrule\relax
\vspace{\nbsphinxcodecellspacing}

\makeatletter\setbox\nbsphinxpromptbox\box\voidb@x\makeatother

\begin{nbsphinxfancyoutput}

\noindent\sphinxincludegraphics[width=290\sphinxpxdimen,height=424\sphinxpxdimen]{{user-guide_add-agent_7_1}.png}

\end{nbsphinxfancyoutput}

In this scenario, agent \sphinxcode{\sphinxupquote{A1}} is investing using LCOE, whereas agent \sphinxcode{\sphinxupquote{A2}} is investing based on the mean of the objectives: LCOE and NPV in the same region. A different strategy is employed by these agents with \sphinxcode{\sphinxupquote{A2}} investing in gasCCGT and windturbines, whereas \sphinxcode{\sphinxupquote{A1}} invests in solarPV.

Next, we will see what occurs if the agents invest based upon the same investment strategy, with both investing using NPV. This requires to edit the \sphinxcode{\sphinxupquote{Agents.csv}} file once more, to look like the following:


\begin{savenotes}\sphinxattablestart
\centering
\begin{tabulary}{\linewidth}[t]{|T|T|T|T|T|T|T|T|T|T|T|T|T|}
\hline
\sphinxstyletheadfamily 
AgentShare
&\sphinxstyletheadfamily 
Name
&\sphinxstyletheadfamily 
AgentNumber
&\sphinxstyletheadfamily 
RegionName
&\sphinxstyletheadfamily 
Objective1
&\sphinxstyletheadfamily 
Objective2
&\sphinxstyletheadfamily 
Objective3
&\sphinxstyletheadfamily 
ObjData1
&\sphinxstyletheadfamily 
ObjData2
&\sphinxstyletheadfamily 
…
&\sphinxstyletheadfamily 
DecisionMethod
&\sphinxstyletheadfamily 
…
&\sphinxstyletheadfamily 
Type
\\
\hline
Agent1
&
A1
&
1
&
R1
&
LCOE
&&&
1
&&
…
&
singleObj
&
…
&
New
\\
\hline
Agent2
&
A1
&
2
&
R1
&
LCOE
&&&
1
&&
…
&
singleObj
&
…
&
Retrofit
\\
\hline
Agent1
&
A2
&
1
&
R1
&
\sphinxstylestrong{LCOE}
&&&
\sphinxstylestrong{1}
&&
…
&
\sphinxstylestrong{singleObj}
&
…
&
New
\\
\hline
Agent2
&
A2
&
2
&
R1
&
\sphinxstylestrong{LCOE}
&&&
\sphinxstylestrong{1}
&&
…
&
\sphinxstylestrong{singleObj}
&
…
&
Retrofit
\\
\hline
\end{tabulary}
\par
\sphinxattableend\end{savenotes}

Again, this requires the re\sphinxhyphen{}running of the simulation, and visualisation like before:

{
\sphinxsetup{VerbatimColor={named}{nbsphinx-code-bg}}
\sphinxsetup{VerbatimBorderColor={named}{nbsphinx-code-border}}
\begin{sphinxVerbatim}[commandchars=\\\{\}]
\llap{\color{nbsphinxin}[21]:\,\hspace{\fboxrule}\hspace{\fboxsep}}\PYG{n}{mca\PYGZus{}capacity} \PYG{o}{=} \PYG{n}{pd}\PYG{o}{.}\PYG{n}{read\PYGZus{}csv}\PYG{p}{(}\PYG{l+s+s2}{\PYGZdq{}}\PYG{l+s+s2}{../Results/MCACapacity.csv}\PYG{l+s+s2}{\PYGZdq{}}\PYG{p}{)}
\PYG{n}{power\PYGZus{}sector} \PYG{o}{=} \PYG{n}{mca\PYGZus{}capacity}\PYG{p}{[}\PYG{n}{mca\PYGZus{}capacity}\PYG{o}{.}\PYG{n}{sector}\PYG{o}{==}\PYG{l+s+s2}{\PYGZdq{}}\PYG{l+s+s2}{power}\PYG{l+s+s2}{\PYGZdq{}}\PYG{p}{]}
\PYG{n}{g}\PYG{o}{=}\PYG{n}{sns}\PYG{o}{.}\PYG{n}{FacetGrid}\PYG{p}{(}\PYG{n}{power\PYGZus{}sector}\PYG{p}{,} \PYG{n}{row}\PYG{o}{=}\PYG{l+s+s1}{\PYGZsq{}}\PYG{l+s+s1}{agent}\PYG{l+s+s1}{\PYGZsq{}}\PYG{p}{)}
\PYG{n}{g}\PYG{o}{.}\PYG{n}{map}\PYG{p}{(}\PYG{n}{sns}\PYG{o}{.}\PYG{n}{lineplot}\PYG{p}{,} \PYG{l+s+s2}{\PYGZdq{}}\PYG{l+s+s2}{year}\PYG{l+s+s2}{\PYGZdq{}}\PYG{p}{,} \PYG{l+s+s2}{\PYGZdq{}}\PYG{l+s+s2}{capacity}\PYG{l+s+s2}{\PYGZdq{}}\PYG{p}{,} \PYG{l+s+s2}{\PYGZdq{}}\PYG{l+s+s2}{technology}\PYG{l+s+s2}{\PYGZdq{}}\PYG{p}{)}
\PYG{n}{g}\PYG{o}{.}\PYG{n}{add\PYGZus{}legend}\PYG{p}{(}\PYG{p}{)}
\end{sphinxVerbatim}
}

{

\kern-\sphinxverbatimsmallskipamount\kern-\baselineskip
\kern+\FrameHeightAdjust\kern-\fboxrule
\vspace{\nbsphinxcodecellspacing}

\sphinxsetup{VerbatimColor={named}{white}}
\sphinxsetup{VerbatimBorderColor={named}{nbsphinx-code-border}}
\begin{sphinxVerbatim}[commandchars=\\\{\}]
\llap{\color{nbsphinxout}[21]:\,\hspace{\fboxrule}\hspace{\fboxsep}}<seaborn.axisgrid.FacetGrid at 0x7f957e5f0970>
\end{sphinxVerbatim}
}

\hrule height -\fboxrule\relax
\vspace{\nbsphinxcodecellspacing}

\makeatletter\setbox\nbsphinxpromptbox\box\voidb@x\makeatother

\begin{nbsphinxfancyoutput}

\noindent\sphinxincludegraphics[width=290\sphinxpxdimen,height=424\sphinxpxdimen]{{user-guide_add-agent_11_1}.png}

\end{nbsphinxfancyoutput}

In this new scenario, with both agents running the same objective, very similar results can be seen, with a high investmwent in \sphinxcode{\sphinxupquote{windturbine}}, none in \sphinxcode{\sphinxupquote{solarPV}} and low \sphinxcode{\sphinxupquote{gasCCGT}}. Have a play around with the files to see if you can come up with different scenarios!


\subsection{Next steps}
\label{\detokenize{user-guide/add-agent:Next-steps}}
In the next section we will show you how to add a new region.


\section{Adding a region}
\label{\detokenize{user-guide/add-region:Adding-a-region}}\label{\detokenize{user-guide/add-region::doc}}
The next step is to add a region which we will call \sphinxcode{\sphinxupquote{R2}}, however, this could equally be called \sphinxcode{\sphinxupquote{USA}} or \sphinxcode{\sphinxupquote{India}}. This requires a similar process to before of modifying the input simulation data. However, we will also have to change the \sphinxcode{\sphinxupquote{settings.toml}} file to achieve this.

The process to change the \sphinxcode{\sphinxupquote{settings.toml}} file is relatively simple. We just have to add our new region to the \sphinxcode{\sphinxupquote{regions}} variable, in the 4th line of the \sphinxcode{\sphinxupquote{settings.toml}} file, like so:

\begin{sphinxVerbatim}[commandchars=\\\{\}]
\PYG{n}{regions} \PYG{o}{=} \PYG{p}{[}\PYG{l+s+s2}{\PYGZdq{}}\PYG{l+s+s2}{R1}\PYG{l+s+s2}{\PYGZdq{}}\PYG{p}{,} \PYG{l+s+s2}{\PYGZdq{}}\PYG{l+s+s2}{R2}\PYG{l+s+s2}{\PYGZdq{}}\PYG{p}{]}
\end{sphinxVerbatim}

The process to change the input files, however, takes a bit more time. To achieve this, there must be data for each of the sectors for the new region. This, therefore, requires the modification of every {\hyperref[\detokenize{inputs/index::doc}]{\sphinxcrossref{\DUrole{doc}{input file}}}}.

Due to space constraints, we will not show you how to edit all of the files. However, you can access the modified files \sphinxhref{github-link}{here INSERT LINK HERE}.

Effectively, for this example, we will copy and paste the results for each of the input files from region \sphinxcode{\sphinxupquote{R1}}, and change the name of the region for the new rows to \sphinxcode{\sphinxupquote{R2}}.

However, as we are increasing the demand by adding a region, as well as modifying the costs of technologies, it may be the case that a higher growth in technology is required. For example, there may be no possible solution to meet demand without increasing the \sphinxcode{\sphinxupquote{windturbine}} maximum allowed limit. We will therefore increase the allowed limits for \sphinxcode{\sphinxupquote{windturbine}} in region \sphinxcode{\sphinxupquote{R2}}.

We have placed two examples as to how to edit the residential sector below. Again, the edited data are highlighted in \sphinxstylestrong{bold}, with the original data in normal text.

The following file is the modified \sphinxcode{\sphinxupquote{/technodata/residential/CommIn.csv}} file:


\begin{savenotes}\sphinxattablestart
\centering
\begin{tabular}[t]{|*{9}{\X{1}{9}|}}
\hline
\sphinxstyletheadfamily 
ProcessName
&\sphinxstyletheadfamily 
RegionName
&\sphinxstyletheadfamily 
Time
&\sphinxstyletheadfamily 
Level
&\sphinxstyletheadfamily 
electricity
&\sphinxstyletheadfamily 
gas
&\sphinxstyletheadfamily 
heat
&\sphinxstyletheadfamily 
CO2f
&\sphinxstyletheadfamily 
wind
\\
\hline
Unit
&\begin{itemize}
\item {} 
\end{itemize}
&
Year
&\begin{itemize}
\item {} 
\end{itemize}
&
PJ/PJ
&
PJ/PJ
&
PJ/PJ
&
kt/PJ
&
PJ/PJ
\\
\hline
gasboiler
&
R1
&
2020
&
fixed
&
0
&
1.16
&
0
&
0
&
0
\\
\hline
heatpump
&
R1
&
2020
&
fixed
&
0.4
&
0
&
0
&
0
&
0
\\
\hline
\sphinxstylestrong{gasboiler}
&
\sphinxstylestrong{R2}
&
\sphinxstylestrong{2020}
&
\sphinxstylestrong{fixed}
&
\sphinxstylestrong{0}
&
\sphinxstylestrong{1.16}
&
\sphinxstylestrong{0}
&
\sphinxstylestrong{0}
&
\sphinxstylestrong{0}
\\
\hline
\sphinxstylestrong{heatpump}
&
\sphinxstylestrong{R2}
&
\sphinxstylestrong{2020}
&
\sphinxstylestrong{fixed}
&
\sphinxstylestrong{0.4}
&
\sphinxstylestrong{0}
&
\sphinxstylestrong{0}
&
\sphinxstylestrong{0}
&
\sphinxstylestrong{0}
\\
\hline
\end{tabular}
\par
\sphinxattableend\end{savenotes}

Whereas the following file is the modified \sphinxcode{\sphinxupquote{/technodata/residential/ExistingCapacity.csv}} file:


\begin{savenotes}\sphinxattablestart
\centering
\begin{tabulary}{\linewidth}[t]{|T|T|T|T|T|T|T|T|T|T|}
\hline
\sphinxstyletheadfamily 
ProcessName
&\sphinxstyletheadfamily 
RegionName
&\sphinxstyletheadfamily 
Unit
&\sphinxstyletheadfamily 
2020
&\sphinxstyletheadfamily 
2025
&\sphinxstyletheadfamily 
2030
&\sphinxstyletheadfamily 
2035
&\sphinxstyletheadfamily 
2040
&\sphinxstyletheadfamily 
2045
&\sphinxstyletheadfamily 
2050
\\
\hline
gasboiler
&
R1
&
PJ/y
&
10
&
5
&
0
&
0
&
0
&
0
&
0
\\
\hline
heatpump
&
R1
&
PJ/y
&
0
&
0
&
0
&
0
&
0
&
0
&
0
\\
\hline
\sphinxstylestrong{gasboiler}
&
\sphinxstylestrong{R2}
&
\sphinxstylestrong{PJ/y}
&
\sphinxstylestrong{10}
&
\sphinxstylestrong{5}
&
\sphinxstylestrong{0}
&
\sphinxstylestrong{0}
&
\sphinxstylestrong{0}
&
\sphinxstylestrong{0}
&
\sphinxstylestrong{0}
\\
\hline
\sphinxstylestrong{heatpump}
&
\sphinxstylestrong{R2}
&
\sphinxstylestrong{PJ/y}
&
\sphinxstylestrong{0}
&
\sphinxstylestrong{0}
&
\sphinxstylestrong{0}
&
\sphinxstylestrong{0}
&
\sphinxstylestrong{0}
&
\sphinxstylestrong{0}
&
\sphinxstylestrong{0}
\\
\hline
\end{tabulary}
\par
\sphinxattableend\end{savenotes}

Below is the reduced \sphinxcode{\sphinxupquote{/technodata/power/technodata.csv}} file, showing the increased capacity for \sphinxcode{\sphinxupquote{windturbine}} in \sphinxcode{\sphinxupquote{R2}}. For this, we highlight only the elements we changed from the rows in \sphinxcode{\sphinxupquote{R1}}. The rest of the elements are the same for \sphinxcode{\sphinxupquote{R1}} as they are for \sphinxcode{\sphinxupquote{R2}}.


\begin{savenotes}\sphinxattablestart
\centering
\begin{tabular}[t]{|*{9}{\X{1}{9}|}}
\hline
\sphinxstyletheadfamily 
ProcessName
&\sphinxstyletheadfamily 
RegionName
&\sphinxstyletheadfamily 
…
&\sphinxstyletheadfamily 
MaxCapacityAddition
&\sphinxstyletheadfamily 
MaxCapacityGrowth
&\sphinxstyletheadfamily 
TotalCapacityLimit
&\sphinxstyletheadfamily 
…
&\sphinxstyletheadfamily 
Agent2
&\sphinxstyletheadfamily 
Agent1
\\
\hline
Unit
&\begin{itemize}
\item {} 
\end{itemize}
&
…
&
PJ
&
\%
&
PJ
&
…
&
Retrofit
&
New
\\
\hline
gasCCGT
&
R1
&
…
&
2
&
0.02
&
60
&
…
&
1
&
0
\\
\hline
windturbine
&
R1
&
…
&
2
&
0.02
&
60
&
…
&
1
&
0
\\
\hline
solarPV
&
R1
&
…
&
2
&
0.02
&
60
&
…
&
1
&
0
\\
\hline
gasCCGT
&
R2
&
…
&
2
&
0.02
&
60
&
…
&
1
&
0
\\
\hline
windturbine
&
R2
&
…
&
\sphinxstylestrong{5}
&
\sphinxstylestrong{0.05}
&
\sphinxstylestrong{100}
&
…
&
1
&
0
\\
\hline
solarPV
&
R2
&
…
&
2
&
0.02
&
60
&
…
&
1
&
0
\\
\hline
\end{tabular}
\par
\sphinxattableend\end{savenotes}

Now, go ahead and amend all of the other input files for each of the sectors by copying and pasting the rows from \sphinxcode{\sphinxupquote{R1}} and replacing the \sphinxcode{\sphinxupquote{RegionName}} to \sphinxcode{\sphinxupquote{R2}} for the new rows. All of the edited input files can be seen \sphinxhref{dead-link}{here}.

Again, we will run the results using the \sphinxcode{\sphinxupquote{python \sphinxhyphen{}m pip muse settings.toml}} in anaconda prompt, and analyse the data as follows:

{
\sphinxsetup{VerbatimColor={named}{nbsphinx-code-bg}}
\sphinxsetup{VerbatimBorderColor={named}{nbsphinx-code-border}}
\begin{sphinxVerbatim}[commandchars=\\\{\}]
\llap{\color{nbsphinxin}[1]:\,\hspace{\fboxrule}\hspace{\fboxsep}}\PYG{k+kn}{import} \PYG{n+nn}{seaborn} \PYG{k}{as} \PYG{n+nn}{sns}
\PYG{k+kn}{import} \PYG{n+nn}{pandas} \PYG{k}{as} \PYG{n+nn}{pd}
\PYG{k+kn}{import} \PYG{n+nn}{matplotlib}\PYG{n+nn}{.}\PYG{n+nn}{pyplot} \PYG{k}{as} \PYG{n+nn}{plt}
\end{sphinxVerbatim}
}

{
\sphinxsetup{VerbatimColor={named}{nbsphinx-code-bg}}
\sphinxsetup{VerbatimBorderColor={named}{nbsphinx-code-border}}
\begin{sphinxVerbatim}[commandchars=\\\{\}]
\llap{\color{nbsphinxin}[2]:\,\hspace{\fboxrule}\hspace{\fboxsep}}\PYG{n}{mca\PYGZus{}capacity} \PYG{o}{=} \PYG{n}{pd}\PYG{o}{.}\PYG{n}{read\PYGZus{}csv}\PYG{p}{(}\PYG{l+s+s2}{\PYGZdq{}}\PYG{l+s+s2}{../tutorial\PYGZhy{}code/add\PYGZhy{}region/Results/MCACapacity.csv}\PYG{l+s+s2}{\PYGZdq{}}\PYG{p}{)}

\PYG{k}{for} \PYG{n}{name}\PYG{p}{,} \PYG{n}{sector} \PYG{o+ow}{in} \PYG{n}{mca\PYGZus{}capacity}\PYG{o}{.}\PYG{n}{groupby}\PYG{p}{(}\PYG{l+s+s2}{\PYGZdq{}}\PYG{l+s+s2}{sector}\PYG{l+s+s2}{\PYGZdq{}}\PYG{p}{)}\PYG{p}{:}
    \PYG{n+nb}{print}\PYG{p}{(}\PYG{l+s+s2}{\PYGZdq{}}\PYG{l+s+si}{\PYGZob{}\PYGZcb{}}\PYG{l+s+s2}{ sector:}\PYG{l+s+s2}{\PYGZdq{}}\PYG{o}{.}\PYG{n}{format}\PYG{p}{(}\PYG{n}{name}\PYG{p}{)}\PYG{p}{)}
    \PYG{n}{g} \PYG{o}{=} \PYG{n}{sns}\PYG{o}{.}\PYG{n}{FacetGrid}\PYG{p}{(}\PYG{n}{data}\PYG{o}{=}\PYG{n}{sector}\PYG{p}{,} \PYG{n}{col}\PYG{o}{=}\PYG{l+s+s2}{\PYGZdq{}}\PYG{l+s+s2}{region}\PYG{l+s+s2}{\PYGZdq{}}\PYG{p}{)}
    \PYG{n}{g}\PYG{o}{.}\PYG{n}{map}\PYG{p}{(}\PYG{n}{sns}\PYG{o}{.}\PYG{n}{lineplot}\PYG{p}{,} \PYG{l+s+s2}{\PYGZdq{}}\PYG{l+s+s2}{year}\PYG{l+s+s2}{\PYGZdq{}}\PYG{p}{,} \PYG{l+s+s2}{\PYGZdq{}}\PYG{l+s+s2}{capacity}\PYG{l+s+s2}{\PYGZdq{}}\PYG{p}{,} \PYG{l+s+s2}{\PYGZdq{}}\PYG{l+s+s2}{technology}\PYG{l+s+s2}{\PYGZdq{}}\PYG{p}{)}
    \PYG{n}{g}\PYG{o}{.}\PYG{n}{add\PYGZus{}legend}\PYG{p}{(}\PYG{p}{)}
    \PYG{n}{plt}\PYG{o}{.}\PYG{n}{show}\PYG{p}{(}\PYG{p}{)}
\end{sphinxVerbatim}
}

{

\kern-\sphinxverbatimsmallskipamount\kern-\baselineskip
\kern+\FrameHeightAdjust\kern-\fboxrule
\vspace{\nbsphinxcodecellspacing}

\sphinxsetup{VerbatimColor={named}{white}}
\sphinxsetup{VerbatimBorderColor={named}{nbsphinx-code-border}}
\begin{sphinxVerbatim}[commandchars=\\\{\}]
gas sector:
\end{sphinxVerbatim}
}

\hrule height -\fboxrule\relax
\vspace{\nbsphinxcodecellspacing}

\makeatletter\setbox\nbsphinxpromptbox\box\voidb@x\makeatother

\begin{nbsphinxfancyoutput}

\noindent\sphinxincludegraphics[width=514\sphinxpxdimen,height=208\sphinxpxdimen]{{user-guide_add-region_5_1}.png}

\end{nbsphinxfancyoutput}

{

\kern-\sphinxverbatimsmallskipamount\kern-\baselineskip
\kern+\FrameHeightAdjust\kern-\fboxrule
\vspace{\nbsphinxcodecellspacing}

\sphinxsetup{VerbatimColor={named}{white}}
\sphinxsetup{VerbatimBorderColor={named}{nbsphinx-code-border}}
\begin{sphinxVerbatim}[commandchars=\\\{\}]
power sector:
\end{sphinxVerbatim}
}

\hrule height -\fboxrule\relax
\vspace{\nbsphinxcodecellspacing}

\makeatletter\setbox\nbsphinxpromptbox\box\voidb@x\makeatother

\begin{nbsphinxfancyoutput}

\noindent\sphinxincludegraphics[width=516\sphinxpxdimen,height=208\sphinxpxdimen]{{user-guide_add-region_5_3}.png}

\end{nbsphinxfancyoutput}

{

\kern-\sphinxverbatimsmallskipamount\kern-\baselineskip
\kern+\FrameHeightAdjust\kern-\fboxrule
\vspace{\nbsphinxcodecellspacing}

\sphinxsetup{VerbatimColor={named}{white}}
\sphinxsetup{VerbatimBorderColor={named}{nbsphinx-code-border}}
\begin{sphinxVerbatim}[commandchars=\\\{\}]
residential sector:
\end{sphinxVerbatim}
}

\hrule height -\fboxrule\relax
\vspace{\nbsphinxcodecellspacing}

\makeatletter\setbox\nbsphinxpromptbox\box\voidb@x\makeatother

\begin{nbsphinxfancyoutput}

\noindent\sphinxincludegraphics[width=512\sphinxpxdimen,height=208\sphinxpxdimen]{{user-guide_add-region_5_5}.png}

\end{nbsphinxfancyoutput}

Due to the similar natures of the two regions, with the parameters effectively copied and pasted between them, the results are very similar in both \sphinxcode{\sphinxupquote{R1}} and \sphinxcode{\sphinxupquote{R2}}. \sphinxcode{\sphinxupquote{gassupply1}} drops significantly within the gas sector, due to the increasing demand of \sphinxcode{\sphinxupquote{heatpump}} and falling demand of \sphinxcode{\sphinxupquote{gasboiler}} in both region \sphinxcode{\sphinxupquote{R1}} and \sphinxcode{\sphinxupquote{R2}}. \sphinxcode{\sphinxupquote{windturbine}} increases significantly to match this \sphinxcode{\sphinxupquote{heatpump}} demand.

Have a play around with the various costs data in the technodata files for each of the sectors and technologies to see if different scenarios emerge. Although be careful. In some cases, the constraints on certain technologies will make it impossible for the demand to be met. Therefore you may have to relax these constraints.


\subsection{Next steps}
\label{\detokenize{user-guide/add-region:Next-steps}}
In the next section we modify the \sphinxcode{\sphinxupquote{settings.toml}} file to change the timeslicing arrangements as well as project until 2040, instead of 2050, in two year timeslices.


\section{Modification of time}
\label{\detokenize{user-guide/modify-timing-data:Modification-of-time}}\label{\detokenize{user-guide/modify-timing-data::doc}}
In this section we will show you how to modify the timeslicing arrangement as well as change the time horizon and year intervals by modifying the \sphinxcode{\sphinxupquote{settings.toml}} file.


\subsection{Modify timeslicing}
\label{\detokenize{user-guide/modify-timing-data:Modify-timeslicing}}
Timeslicing is the division of a single year into multiple different sections. For example, we could slice the year into different seasons, make a distinction between weekday and weekend or a distinction between morning and night. We do this as energy demand profiles can show a difference between these timeslices. eg. Electricity consumption is lower during the night than during the day.

To achieve this, we have to modify the \sphinxcode{\sphinxupquote{settings.toml}} file, as well as the files within the preset folder: \sphinxcode{\sphinxupquote{Residential2020Consumption.csv}} and \sphinxcode{\sphinxupquote{Residential2050Consumption.csv}}. This is so that we can edit the demand for the residential sector for the new timeslices.

First we edit the \sphinxcode{\sphinxupquote{settings.toml}} file to add two additional timeslices: early\sphinxhyphen{}morning and late\sphinxhyphen{}afternoon. We also rename afternoon to mid\sphinxhyphen{}afternoon. These settings can be found at the bottom of the \sphinxcode{\sphinxupquote{settings.toml}} file.

An example of the changes is shown below:

\begin{sphinxVerbatim}[commandchars=\\\{\}]
\PYG{p}{[}\PYG{n}{timeslices}\PYG{p}{]}
\PYG{n+nb}{all}\PYG{o}{\PYGZhy{}}\PYG{n}{year}\PYG{o}{.}\PYG{n}{all}\PYG{o}{\PYGZhy{}}\PYG{n}{week}\PYG{o}{.}\PYG{n}{night} \PYG{o}{=} \PYG{l+m+mi}{1095}
\PYG{n+nb}{all}\PYG{o}{\PYGZhy{}}\PYG{n}{year}\PYG{o}{.}\PYG{n}{all}\PYG{o}{\PYGZhy{}}\PYG{n}{week}\PYG{o}{.}\PYG{n}{morning} \PYG{o}{=} \PYG{l+m+mi}{1095}
\PYG{n+nb}{all}\PYG{o}{\PYGZhy{}}\PYG{n}{year}\PYG{o}{.}\PYG{n}{all}\PYG{o}{\PYGZhy{}}\PYG{n}{week}\PYG{o}{.}\PYG{n}{mid}\PYG{o}{\PYGZhy{}}\PYG{n}{afternoon} \PYG{o}{=} \PYG{l+m+mi}{1095}
\PYG{n+nb}{all}\PYG{o}{\PYGZhy{}}\PYG{n}{year}\PYG{o}{.}\PYG{n}{all}\PYG{o}{\PYGZhy{}}\PYG{n}{week}\PYG{o}{.}\PYG{n}{early}\PYG{o}{\PYGZhy{}}\PYG{n}{peak} \PYG{o}{=} \PYG{l+m+mi}{1095}
\PYG{n+nb}{all}\PYG{o}{\PYGZhy{}}\PYG{n}{year}\PYG{o}{.}\PYG{n}{all}\PYG{o}{\PYGZhy{}}\PYG{n}{week}\PYG{o}{.}\PYG{n}{late}\PYG{o}{\PYGZhy{}}\PYG{n}{peak} \PYG{o}{=} \PYG{l+m+mi}{1095}
\PYG{n+nb}{all}\PYG{o}{\PYGZhy{}}\PYG{n}{year}\PYG{o}{.}\PYG{n}{all}\PYG{o}{\PYGZhy{}}\PYG{n}{week}\PYG{o}{.}\PYG{n}{evening} \PYG{o}{=} \PYG{l+m+mi}{1095}
\PYG{n+nb}{all}\PYG{o}{\PYGZhy{}}\PYG{n}{year}\PYG{o}{.}\PYG{n}{all}\PYG{o}{\PYGZhy{}}\PYG{n}{week}\PYG{o}{.}\PYG{n}{early}\PYG{o}{\PYGZhy{}}\PYG{n}{morning} \PYG{o}{=} \PYG{l+m+mi}{1095}
\PYG{n+nb}{all}\PYG{o}{\PYGZhy{}}\PYG{n}{year}\PYG{o}{.}\PYG{n}{all}\PYG{o}{\PYGZhy{}}\PYG{n}{week}\PYG{o}{.}\PYG{n}{late}\PYG{o}{\PYGZhy{}}\PYG{n}{afternoon} \PYG{o}{=} \PYG{l+m+mi}{1095}
\PYG{n}{level\PYGZus{}names} \PYG{o}{=} \PYG{p}{[}\PYG{l+s+s2}{\PYGZdq{}}\PYG{l+s+s2}{month}\PYG{l+s+s2}{\PYGZdq{}}\PYG{p}{,} \PYG{l+s+s2}{\PYGZdq{}}\PYG{l+s+s2}{day}\PYG{l+s+s2}{\PYGZdq{}}\PYG{p}{,} \PYG{l+s+s2}{\PYGZdq{}}\PYG{l+s+s2}{hour}\PYG{l+s+s2}{\PYGZdq{}}\PYG{p}{]}
\end{sphinxVerbatim}

Next, we modify both Residential Consumption files. Again, we put the text in bold for the modified entries. We must add the demand for the two additional timelsices, which we call timeslice 7 and 8. We make the demand for heat to be 2 for both of the new timeslices.

Below is the modified \sphinxcode{\sphinxupquote{Residential2020Consumption.csv}} file:


\begin{savenotes}\sphinxattablestart
\centering
\begin{tabulary}{\linewidth}[t]{|T|T|T|T|T|T|T|T|T|}
\hline


&\sphinxstyletheadfamily 
RegionName
&\sphinxstyletheadfamily 
ProcessName
&\sphinxstyletheadfamily 
Timeslice
&\sphinxstyletheadfamily 
electricity
&\sphinxstyletheadfamily 
gas
&\sphinxstyletheadfamily 
heat
&\sphinxstyletheadfamily 
CO2f
&\sphinxstyletheadfamily 
wind
\\
\hline
0
&
R1
&
gasboiler
&
1
&
0
&
0
&
1
&
0
&
0
\\
\hline
1
&
R1
&
gasboiler
&
2
&
0
&
0
&
1.5
&
0
&
0
\\
\hline
2
&
R1
&
gasboiler
&
3
&
0
&
0
&
1
&
0
&
0
\\
\hline
3
&
R1
&
gasboiler
&
4
&
0
&
0
&
1.5
&
0
&
0
\\
\hline
4
&
R1
&
gasboiler
&
5
&
0
&
0
&
3
&
0
&
0
\\
\hline
5
&
R1
&
gasboiler
&
6
&
0
&
0
&
2
&
0
&
0
\\
\hline
\sphinxstylestrong{6}
&
\sphinxstylestrong{R1}
&
\sphinxstylestrong{gasboiler}
&
\sphinxstylestrong{7}
&
\sphinxstylestrong{0}
&
\sphinxstylestrong{0}
&
\sphinxstylestrong{2}
&
\sphinxstylestrong{0}
&
\sphinxstylestrong{0}
\\
\hline
\sphinxstylestrong{7}
&
\sphinxstylestrong{R1}
&
\sphinxstylestrong{gasboiler}
&
\sphinxstylestrong{8}
&
\sphinxstylestrong{0}
&
\sphinxstylestrong{0}
&
\sphinxstylestrong{2}
&
\sphinxstylestrong{0}
&
\sphinxstylestrong{0}
\\
\hline
0
&
R2
&
gasboiler
&
1
&
0
&
0
&
1
&
0
&
0
\\
\hline
1
&
R2
&
gasboiler
&
2
&
0
&
0
&
1.5
&
0
&
0
\\
\hline
2
&
R2
&
gasboiler
&
3
&
0
&
0
&
1
&
0
&
0
\\
\hline
3
&
R2
&
gasboiler
&
4
&
0
&
0
&
1.5
&
0
&
0
\\
\hline
4
&
R2
&
gasboiler
&
5
&
0
&
0
&
3
&
0
&
0
\\
\hline
5
&
R2
&
gasboiler
&
6
&
0
&
0
&
2
&
0
&
0
\\
\hline
\sphinxstylestrong{6}
&
\sphinxstylestrong{R2}
&
\sphinxstylestrong{gasboiler}
&
\sphinxstylestrong{7}
&
\sphinxstylestrong{0}
&
\sphinxstylestrong{0}
&
\sphinxstylestrong{2}
&
\sphinxstylestrong{0}
&
\sphinxstylestrong{0}
\\
\hline
\sphinxstylestrong{7}
&
\sphinxstylestrong{R2}
&
\sphinxstylestrong{gasboiler}
&
\sphinxstylestrong{8}
&
\sphinxstylestrong{0}
&
\sphinxstylestrong{0}
&
\sphinxstylestrong{2}
&
\sphinxstylestrong{0}
&
\sphinxstylestrong{0}
\\
\hline
\end{tabulary}
\par
\sphinxattableend\end{savenotes}

We do the same for the \sphinxcode{\sphinxupquote{Residential2050Consumption.csv}}, however this time we make the demand for heat in 2050 to both be 5 for the new timeslices. See \sphinxhref{github-residential2050}{here INSERT LINK HERE} for the full file.



Once the relevant files have been edited, we are able to run the simulation model using \sphinxcode{\sphinxupquote{python \sphinxhyphen{}m muse settings.toml}}.

Then, once run, we import the necessary packages:

{
\sphinxsetup{VerbatimColor={named}{nbsphinx-code-bg}}
\sphinxsetup{VerbatimBorderColor={named}{nbsphinx-code-border}}
\begin{sphinxVerbatim}[commandchars=\\\{\}]
\llap{\color{nbsphinxin}[1]:\,\hspace{\fboxrule}\hspace{\fboxsep}}\PYG{k+kn}{import} \PYG{n+nn}{pandas} \PYG{k}{as} \PYG{n+nn}{pd}
\PYG{k+kn}{import} \PYG{n+nn}{seaborn} \PYG{k}{as} \PYG{n+nn}{sns}
\PYG{k+kn}{import} \PYG{n+nn}{matplotlib}\PYG{n+nn}{.}\PYG{n+nn}{pyplot} \PYG{k}{as} \PYG{n+nn}{plt}
\end{sphinxVerbatim}
}

and visualise the relevant data:

{
\sphinxsetup{VerbatimColor={named}{nbsphinx-code-bg}}
\sphinxsetup{VerbatimBorderColor={named}{nbsphinx-code-border}}
\begin{sphinxVerbatim}[commandchars=\\\{\}]
\llap{\color{nbsphinxin}[2]:\,\hspace{\fboxrule}\hspace{\fboxsep}}\PYG{n}{mca\PYGZus{}capacity} \PYG{o}{=} \PYG{n}{pd}\PYG{o}{.}\PYG{n}{read\PYGZus{}csv}\PYG{p}{(}\PYG{l+s+s2}{\PYGZdq{}}\PYG{l+s+s2}{../tutorial\PYGZhy{}code/modify\PYGZhy{}timing\PYGZhy{}data/modify\PYGZhy{}time\PYGZhy{}framework/Results/MCACapacity.csv}\PYG{l+s+s2}{\PYGZdq{}}\PYG{p}{)}

\PYG{k}{for} \PYG{n}{name}\PYG{p}{,} \PYG{n}{sector} \PYG{o+ow}{in} \PYG{n}{mca\PYGZus{}capacity}\PYG{o}{.}\PYG{n}{groupby}\PYG{p}{(}\PYG{l+s+s2}{\PYGZdq{}}\PYG{l+s+s2}{sector}\PYG{l+s+s2}{\PYGZdq{}}\PYG{p}{)}\PYG{p}{:}
    \PYG{n+nb}{print}\PYG{p}{(}\PYG{l+s+s2}{\PYGZdq{}}\PYG{l+s+si}{\PYGZob{}\PYGZcb{}}\PYG{l+s+s2}{ sector:}\PYG{l+s+s2}{\PYGZdq{}}\PYG{o}{.}\PYG{n}{format}\PYG{p}{(}\PYG{n}{name}\PYG{p}{)}\PYG{p}{)}
    \PYG{n}{fig}\PYG{p}{,} \PYG{n}{ax} \PYG{o}{=}\PYG{n}{plt}\PYG{o}{.}\PYG{n}{subplots}\PYG{p}{(}\PYG{l+m+mi}{1}\PYG{p}{,}\PYG{l+m+mi}{2}\PYG{p}{)}
    \PYG{n}{sns}\PYG{o}{.}\PYG{n}{lineplot}\PYG{p}{(}\PYG{n}{data}\PYG{o}{=}\PYG{n}{sector}\PYG{p}{[}\PYG{n}{sector}\PYG{o}{.}\PYG{n}{region}\PYG{o}{==}\PYG{l+s+s2}{\PYGZdq{}}\PYG{l+s+s2}{R1}\PYG{l+s+s2}{\PYGZdq{}}\PYG{p}{]}\PYG{p}{,} \PYG{n}{x}\PYG{o}{=}\PYG{l+s+s2}{\PYGZdq{}}\PYG{l+s+s2}{year}\PYG{l+s+s2}{\PYGZdq{}}\PYG{p}{,} \PYG{n}{y}\PYG{o}{=}\PYG{l+s+s2}{\PYGZdq{}}\PYG{l+s+s2}{capacity}\PYG{l+s+s2}{\PYGZdq{}}\PYG{p}{,} \PYG{n}{hue}\PYG{o}{=}\PYG{l+s+s2}{\PYGZdq{}}\PYG{l+s+s2}{technology}\PYG{l+s+s2}{\PYGZdq{}}\PYG{p}{,} \PYG{n}{ax}\PYG{o}{=}\PYG{n}{ax}\PYG{p}{[}\PYG{l+m+mi}{0}\PYG{p}{]}\PYG{p}{)}
    \PYG{n}{sns}\PYG{o}{.}\PYG{n}{lineplot}\PYG{p}{(}\PYG{n}{data}\PYG{o}{=}\PYG{n}{sector}\PYG{p}{[}\PYG{n}{sector}\PYG{o}{.}\PYG{n}{region}\PYG{o}{==}\PYG{l+s+s2}{\PYGZdq{}}\PYG{l+s+s2}{R2}\PYG{l+s+s2}{\PYGZdq{}}\PYG{p}{]}\PYG{p}{,} \PYG{n}{x}\PYG{o}{=}\PYG{l+s+s2}{\PYGZdq{}}\PYG{l+s+s2}{year}\PYG{l+s+s2}{\PYGZdq{}}\PYG{p}{,} \PYG{n}{y}\PYG{o}{=}\PYG{l+s+s2}{\PYGZdq{}}\PYG{l+s+s2}{capacity}\PYG{l+s+s2}{\PYGZdq{}}\PYG{p}{,} \PYG{n}{hue}\PYG{o}{=}\PYG{l+s+s2}{\PYGZdq{}}\PYG{l+s+s2}{technology}\PYG{l+s+s2}{\PYGZdq{}}\PYG{p}{,} \PYG{n}{ax}\PYG{o}{=}\PYG{n}{ax}\PYG{p}{[}\PYG{l+m+mi}{1}\PYG{p}{]}\PYG{p}{)}
    \PYG{n}{plt}\PYG{o}{.}\PYG{n}{show}\PYG{p}{(}\PYG{p}{)}
    \PYG{n}{plt}\PYG{o}{.}\PYG{n}{close}\PYG{p}{(}\PYG{p}{)}
\end{sphinxVerbatim}
}

{

\kern-\sphinxverbatimsmallskipamount\kern-\baselineskip
\kern+\FrameHeightAdjust\kern-\fboxrule
\vspace{\nbsphinxcodecellspacing}

\sphinxsetup{VerbatimColor={named}{white}}
\sphinxsetup{VerbatimBorderColor={named}{nbsphinx-code-border}}
\begin{sphinxVerbatim}[commandchars=\\\{\}]
gas sector:
\end{sphinxVerbatim}
}

\hrule height -\fboxrule\relax
\vspace{\nbsphinxcodecellspacing}

\makeatletter\setbox\nbsphinxpromptbox\box\voidb@x\makeatother

\begin{nbsphinxfancyoutput}

\noindent\sphinxincludegraphics[width=388\sphinxpxdimen,height=262\sphinxpxdimen]{{user-guide_modify-timing-data_7_1}.png}

\end{nbsphinxfancyoutput}

{

\kern-\sphinxverbatimsmallskipamount\kern-\baselineskip
\kern+\FrameHeightAdjust\kern-\fboxrule
\vspace{\nbsphinxcodecellspacing}

\sphinxsetup{VerbatimColor={named}{white}}
\sphinxsetup{VerbatimBorderColor={named}{nbsphinx-code-border}}
\begin{sphinxVerbatim}[commandchars=\\\{\}]
power sector:
\end{sphinxVerbatim}
}

\hrule height -\fboxrule\relax
\vspace{\nbsphinxcodecellspacing}

\makeatletter\setbox\nbsphinxpromptbox\box\voidb@x\makeatother

\begin{nbsphinxfancyoutput}

\noindent\sphinxincludegraphics[width=395\sphinxpxdimen,height=262\sphinxpxdimen]{{user-guide_modify-timing-data_7_3}.png}

\end{nbsphinxfancyoutput}

{

\kern-\sphinxverbatimsmallskipamount\kern-\baselineskip
\kern+\FrameHeightAdjust\kern-\fboxrule
\vspace{\nbsphinxcodecellspacing}

\sphinxsetup{VerbatimColor={named}{white}}
\sphinxsetup{VerbatimBorderColor={named}{nbsphinx-code-border}}
\begin{sphinxVerbatim}[commandchars=\\\{\}]
residential sector:
\end{sphinxVerbatim}
}

\hrule height -\fboxrule\relax
\vspace{\nbsphinxcodecellspacing}

\makeatletter\setbox\nbsphinxpromptbox\box\voidb@x\makeatother

\begin{nbsphinxfancyoutput}

\noindent\sphinxincludegraphics[width=388\sphinxpxdimen,height=262\sphinxpxdimen]{{user-guide_modify-timing-data_7_5}.png}

\end{nbsphinxfancyoutput}

Compared to the scenario where we added a {\hyperref[\detokenize{user-guide/add-region::doc}]{\sphinxcrossref{\DUrole{doc}{region}}}}, there is a slight increase in solarPV in the power sector. However, the rest remains unchanged.


\subsection{Modify time horizon and time periods}
\label{\detokenize{user-guide/modify-timing-data:Modify-time-horizon-and-time-periods}}
For the previous examples, we have run the scenario from 2020 to 2050, in 5 year time steps. This has been set at the top of the \sphinxcode{\sphinxupquote{settings.toml}} file. However, we may want to run a more detailed scenario, with 2 year time steps, and up until the year 2040.

Making this change is quite simple as we only have two lines to change. We will modify line 2 and 3 of the \sphinxcode{\sphinxupquote{settings.toml}} file, as follows:

\begin{sphinxVerbatim}[commandchars=\\\{\}]
\PYG{c+c1}{\PYGZsh{} Global settings \PYGZhy{} most REQUIRED}
\PYG{n}{time\PYGZus{}framework} \PYG{o}{=} \PYG{p}{[}\PYG{l+m+mi}{2020}\PYG{p}{,} \PYG{l+m+mi}{2022}\PYG{p}{,} \PYG{l+m+mi}{2024}\PYG{p}{,} \PYG{l+m+mi}{2026}\PYG{p}{,} \PYG{l+m+mi}{2028}\PYG{p}{,} \PYG{l+m+mi}{2030}\PYG{p}{,} \PYG{l+m+mi}{2032}\PYG{p}{,} \PYG{l+m+mi}{2034}\PYG{p}{,} \PYG{l+m+mi}{2036}\PYG{p}{,} \PYG{l+m+mi}{2038}\PYG{p}{,} \PYG{l+m+mi}{2040}\PYG{p}{]}
\PYG{n}{foresight} \PYG{o}{=} \PYG{l+m+mi}{2}   \PYG{c+c1}{\PYGZsh{} Has to be a multiple of the minimum separation between the years in time}
\end{sphinxVerbatim}

The \sphinxcode{\sphinxupquote{time\_framework}} details each year in which we run the simulation. The \sphinxcode{\sphinxupquote{foresight}} variable details how much foresight an agent has when making investments.

As we have modified the timeslicing arrangements there will be a change in the underlying demand for heating. This may require more electricity to service this demand. Therefore, we relax the constraints for growth in the power sector for all technologies and constraints in the \sphinxcode{\sphinxupquote{technodata/power/technodata.csv}}, as is shown below:


\begin{savenotes}\sphinxattablestart
\centering
\begin{tabular}[t]{|*{8}{\X{1}{8}|}}
\hline
\sphinxstyletheadfamily 
ProcessName
&\sphinxstyletheadfamily 
RegionName
&\sphinxstyletheadfamily 
…
&\sphinxstyletheadfamily 
MaxCapacityAddition
&\sphinxstyletheadfamily 
MaxCapacityGrowth
&\sphinxstyletheadfamily 
TotalCapacityLimit
&\sphinxstyletheadfamily 
…
&\sphinxstyletheadfamily 
Agent1
\\
\hline
Unit
&\begin{itemize}
\item {} 
\end{itemize}
&
…
&
PJ
&
\%
&
PJ
&
…
&
New
\\
\hline
gasCCGT
&
R1
&
…
&
\sphinxstylestrong{40}
&
\sphinxstylestrong{0.2}
&
\sphinxstylestrong{120}
&
…
&
0
\\
\hline
windturbine
&
R1
&
…
&
\sphinxstylestrong{40}
&
\sphinxstylestrong{0.2}
&
\sphinxstylestrong{120}
&
…
&
0
\\
\hline
solarPV
&
R1
&
…
&
\sphinxstylestrong{40}
&
\sphinxstylestrong{0.2}
&
\sphinxstylestrong{120}
&
…
&
0
\\
\hline
gasCCGT
&
R2
&
…
&
\sphinxstylestrong{40}
&
\sphinxstylestrong{0.2}
&
\sphinxstylestrong{120}
&
…
&
0
\\
\hline
windturbine
&
R2
&
…
&
\sphinxstylestrong{40}
&
\sphinxstylestrong{0.2}
&
\sphinxstylestrong{120}
&
…
&
0
\\
\hline
solarPV
&
R2
&
…
&
\sphinxstylestrong{40}
&
\sphinxstylestrong{0.2}
&
\sphinxstylestrong{120}
&
…
&
0
\\
\hline
\end{tabular}
\par
\sphinxattableend\end{savenotes}

We also modify the constraints defined in the \sphinxcode{\sphinxupquote{technodata.csv}} file for the residential sector, as shown below:


\begin{savenotes}\sphinxattablestart
\centering
\begin{tabular}[t]{|*{8}{\X{1}{8}|}}
\hline
\sphinxstyletheadfamily 
ProcessName
&\sphinxstyletheadfamily 
RegionName
&\sphinxstyletheadfamily 
…
&\sphinxstyletheadfamily 
MaxCapacityAddition
&\sphinxstyletheadfamily 
MaxCapacityGrowth
&\sphinxstyletheadfamily 
TotalCapacityLimit
&\sphinxstyletheadfamily 
…
&\sphinxstyletheadfamily 
Agent1
\\
\hline
Unit
&\begin{itemize}
\item {} 
\end{itemize}
&
…
&
PJ
&
\%
&
PJ
&
…
&
New
\\
\hline
gasboiler
&
R1
&
…
&
\sphinxstylestrong{60}
&
\sphinxstylestrong{0.5}
&
\sphinxstylestrong{120}
&
…
&
0
\\
\hline
heatpump
&
R1
&
…
&
\sphinxstylestrong{60}
&
\sphinxstylestrong{0.5}
&
\sphinxstylestrong{120}
&
…
&
0
\\
\hline
gasboiler
&
R2
&
…
&
\sphinxstylestrong{60}
&
\sphinxstylestrong{0.5}
&
\sphinxstylestrong{120}
&
…
&
0
\\
\hline
heatpump
&
R2
&
…
&
\sphinxstylestrong{60}
&
\sphinxstylestrong{0.5}
&
\sphinxstylestrong{120}
&
…
&
0
\\
\hline
\end{tabular}
\par
\sphinxattableend\end{savenotes}

It must be noted, that this is a toy example. For modelling a real life scenario, data should be sought to ensure there remain realistic constriants.

For the full power sector \sphinxcode{\sphinxupquote{technodata.csv}} file click \sphinxhref{github-power-technodata}{here INSERT LINK HERE}, and for the full residential sector \sphinxcode{\sphinxupquote{technodata.csv}} file click \sphinxhref{github-residential-technodata}{here INSERT LINK HERE}.



{
\sphinxsetup{VerbatimColor={named}{nbsphinx-code-bg}}
\sphinxsetup{VerbatimBorderColor={named}{nbsphinx-code-border}}
\begin{sphinxVerbatim}[commandchars=\\\{\}]
\llap{\color{nbsphinxin}[3]:\,\hspace{\fboxrule}\hspace{\fboxsep}}\PYG{n}{mca\PYGZus{}capacity} \PYG{o}{=} \PYG{n}{pd}\PYG{o}{.}\PYG{n}{read\PYGZus{}csv}\PYG{p}{(}\PYG{l+s+s2}{\PYGZdq{}}\PYG{l+s+s2}{../tutorial\PYGZhy{}code/modify\PYGZhy{}timing\PYGZhy{}data/modify\PYGZhy{}time\PYGZhy{}framework/Results/MCACapacity.csv}\PYG{l+s+s2}{\PYGZdq{}}\PYG{p}{)}

\PYG{k}{for} \PYG{n}{name}\PYG{p}{,} \PYG{n}{sector} \PYG{o+ow}{in} \PYG{n}{mca\PYGZus{}capacity}\PYG{o}{.}\PYG{n}{groupby}\PYG{p}{(}\PYG{l+s+s2}{\PYGZdq{}}\PYG{l+s+s2}{sector}\PYG{l+s+s2}{\PYGZdq{}}\PYG{p}{)}\PYG{p}{:}
    \PYG{n+nb}{print}\PYG{p}{(}\PYG{l+s+s2}{\PYGZdq{}}\PYG{l+s+si}{\PYGZob{}\PYGZcb{}}\PYG{l+s+s2}{ sector:}\PYG{l+s+s2}{\PYGZdq{}}\PYG{o}{.}\PYG{n}{format}\PYG{p}{(}\PYG{n}{name}\PYG{p}{)}\PYG{p}{)}
    \PYG{n}{fig}\PYG{p}{,} \PYG{n}{ax} \PYG{o}{=}\PYG{n}{plt}\PYG{o}{.}\PYG{n}{subplots}\PYG{p}{(}\PYG{l+m+mi}{1}\PYG{p}{,}\PYG{l+m+mi}{2}\PYG{p}{)}
    \PYG{n}{sns}\PYG{o}{.}\PYG{n}{lineplot}\PYG{p}{(}\PYG{n}{data}\PYG{o}{=}\PYG{n}{sector}\PYG{p}{[}\PYG{n}{sector}\PYG{o}{.}\PYG{n}{region}\PYG{o}{==}\PYG{l+s+s2}{\PYGZdq{}}\PYG{l+s+s2}{R1}\PYG{l+s+s2}{\PYGZdq{}}\PYG{p}{]}\PYG{p}{,} \PYG{n}{x}\PYG{o}{=}\PYG{l+s+s2}{\PYGZdq{}}\PYG{l+s+s2}{year}\PYG{l+s+s2}{\PYGZdq{}}\PYG{p}{,} \PYG{n}{y}\PYG{o}{=}\PYG{l+s+s2}{\PYGZdq{}}\PYG{l+s+s2}{capacity}\PYG{l+s+s2}{\PYGZdq{}}\PYG{p}{,} \PYG{n}{hue}\PYG{o}{=}\PYG{l+s+s2}{\PYGZdq{}}\PYG{l+s+s2}{technology}\PYG{l+s+s2}{\PYGZdq{}}\PYG{p}{,} \PYG{n}{ax}\PYG{o}{=}\PYG{n}{ax}\PYG{p}{[}\PYG{l+m+mi}{0}\PYG{p}{]}\PYG{p}{)}
    \PYG{n}{sns}\PYG{o}{.}\PYG{n}{lineplot}\PYG{p}{(}\PYG{n}{data}\PYG{o}{=}\PYG{n}{sector}\PYG{p}{[}\PYG{n}{sector}\PYG{o}{.}\PYG{n}{region}\PYG{o}{==}\PYG{l+s+s2}{\PYGZdq{}}\PYG{l+s+s2}{R2}\PYG{l+s+s2}{\PYGZdq{}}\PYG{p}{]}\PYG{p}{,} \PYG{n}{x}\PYG{o}{=}\PYG{l+s+s2}{\PYGZdq{}}\PYG{l+s+s2}{year}\PYG{l+s+s2}{\PYGZdq{}}\PYG{p}{,} \PYG{n}{y}\PYG{o}{=}\PYG{l+s+s2}{\PYGZdq{}}\PYG{l+s+s2}{capacity}\PYG{l+s+s2}{\PYGZdq{}}\PYG{p}{,} \PYG{n}{hue}\PYG{o}{=}\PYG{l+s+s2}{\PYGZdq{}}\PYG{l+s+s2}{technology}\PYG{l+s+s2}{\PYGZdq{}}\PYG{p}{,} \PYG{n}{ax}\PYG{o}{=}\PYG{n}{ax}\PYG{p}{[}\PYG{l+m+mi}{1}\PYG{p}{]}\PYG{p}{)}
    \PYG{n}{plt}\PYG{o}{.}\PYG{n}{show}\PYG{p}{(}\PYG{p}{)}
    \PYG{n}{plt}\PYG{o}{.}\PYG{n}{close}\PYG{p}{(}\PYG{p}{)}
\end{sphinxVerbatim}
}

{

\kern-\sphinxverbatimsmallskipamount\kern-\baselineskip
\kern+\FrameHeightAdjust\kern-\fboxrule
\vspace{\nbsphinxcodecellspacing}

\sphinxsetup{VerbatimColor={named}{white}}
\sphinxsetup{VerbatimBorderColor={named}{nbsphinx-code-border}}
\begin{sphinxVerbatim}[commandchars=\\\{\}]
gas sector:
\end{sphinxVerbatim}
}

\hrule height -\fboxrule\relax
\vspace{\nbsphinxcodecellspacing}

\makeatletter\setbox\nbsphinxpromptbox\box\voidb@x\makeatother

\begin{nbsphinxfancyoutput}

\noindent\sphinxincludegraphics[width=388\sphinxpxdimen,height=262\sphinxpxdimen]{{user-guide_modify-timing-data_13_1}.png}

\end{nbsphinxfancyoutput}

{

\kern-\sphinxverbatimsmallskipamount\kern-\baselineskip
\kern+\FrameHeightAdjust\kern-\fboxrule
\vspace{\nbsphinxcodecellspacing}

\sphinxsetup{VerbatimColor={named}{white}}
\sphinxsetup{VerbatimBorderColor={named}{nbsphinx-code-border}}
\begin{sphinxVerbatim}[commandchars=\\\{\}]
power sector:
\end{sphinxVerbatim}
}

\hrule height -\fboxrule\relax
\vspace{\nbsphinxcodecellspacing}

\makeatletter\setbox\nbsphinxpromptbox\box\voidb@x\makeatother

\begin{nbsphinxfancyoutput}

\noindent\sphinxincludegraphics[width=395\sphinxpxdimen,height=262\sphinxpxdimen]{{user-guide_modify-timing-data_13_3}.png}

\end{nbsphinxfancyoutput}

{

\kern-\sphinxverbatimsmallskipamount\kern-\baselineskip
\kern+\FrameHeightAdjust\kern-\fboxrule
\vspace{\nbsphinxcodecellspacing}

\sphinxsetup{VerbatimColor={named}{white}}
\sphinxsetup{VerbatimBorderColor={named}{nbsphinx-code-border}}
\begin{sphinxVerbatim}[commandchars=\\\{\}]
residential sector:
\end{sphinxVerbatim}
}

\hrule height -\fboxrule\relax
\vspace{\nbsphinxcodecellspacing}

\makeatletter\setbox\nbsphinxpromptbox\box\voidb@x\makeatother

\begin{nbsphinxfancyoutput}

\noindent\sphinxincludegraphics[width=388\sphinxpxdimen,height=262\sphinxpxdimen]{{user-guide_modify-timing-data_13_5}.png}

\end{nbsphinxfancyoutput}


\subsection{Next steps}
\label{\detokenize{user-guide/modify-timing-data:Next-steps}}
In the next section we detail how to add an exogenous service demand, such as demand for heating or cooking.


\section{Adding a service demand}
\label{\detokenize{user-guide/addition-service-demand:Adding-a-service-demand}}\label{\detokenize{user-guide/addition-service-demand::doc}}
In this section, we will detail how to add a service demand to MUSE.

A service demand is an end\sphinxhyphen{}use demand. For example, in the residential sector, a service demand could be cooking. Houses require energy to cook food and a technology to service this demand, such as an electric stove.

This process consists of setting a demand, either through inputs derived from the user or correlations of GDP and population which reflect the socio\sphinxhyphen{}economic decvelopment of a region or country. In addition, a technology must be added to service the demand.


\subsection{Addition of cooking demand}
\label{\detokenize{user-guide/addition-service-demand:Addition-of-cooking-demand}}
Firstly, we must add the demand section. In this example, we will add a cooking preset demand. To achieve this, we will now edit the \sphinxcode{\sphinxupquote{Residential2020Consumption.csv}} and \sphinxcode{\sphinxupquote{Residential2050Consumption.csv}} files, found within the \sphinxcode{\sphinxupquote{technodata/preset/}} directory.

The \sphinxcode{\sphinxupquote{Residential2020Consumption.csv}} file allows us to specify the demand in 2020 for each region and technology per timeslice. The \sphinxcode{\sphinxupquote{Residential2050Consumption.csv}} file does the same but for the year 2050. The datapoints between these are interpolated.

Firstly, we must add the new service demand: \sphinxcode{\sphinxupquote{cook}} as a column in these two files. Next, we add the demand. Again, the modified entries are in bold:


\begin{savenotes}\sphinxattablestart
\centering
\begin{tabulary}{\linewidth}[t]{|T|T|T|T|T|T|T|T|T|T|}
\hline


&\sphinxstyletheadfamily 
RegionName
&\sphinxstyletheadfamily 
ProcessName
&\sphinxstyletheadfamily 
Timeslice
&\sphinxstyletheadfamily 
electricity
&\sphinxstyletheadfamily 
gas
&\sphinxstyletheadfamily 
heat
&\sphinxstyletheadfamily 
CO2f
&\sphinxstyletheadfamily 
wind
&\sphinxstyletheadfamily 
\sphinxstylestrong{cook}
\\
\hline
0
&
R1
&
gasboiler
&
1
&
0
&
0
&
1
&
0
&
0
&
\sphinxstylestrong{0}
\\
\hline
…
&
…
&
…
&
…
&
…
&
…
&
…
&
…
&
…
&
\sphinxstylestrong{…}
\\
\hline
15
&
R2
&
gasboiler
&
8
&
0
&
0
&
2
&
0
&
0
&
\sphinxstylestrong{0}
\\
\hline
\sphinxstylestrong{16}
&
\sphinxstylestrong{R1}
&
\sphinxstylestrong{electric\_stove}
&
\sphinxstylestrong{1}
&
\sphinxstylestrong{0}
&
\sphinxstylestrong{0}
&
\sphinxstylestrong{0}
&
\sphinxstylestrong{0}
&
\sphinxstylestrong{0}
&
\sphinxstylestrong{1}
\\
\hline
\sphinxstylestrong{17}
&
\sphinxstylestrong{R1}
&
\sphinxstylestrong{electric\_stove}
&
\sphinxstylestrong{2}
&
\sphinxstylestrong{0}
&
\sphinxstylestrong{0}
&
\sphinxstylestrong{0}
&
\sphinxstylestrong{0}
&
\sphinxstylestrong{0}
&
\sphinxstylestrong{2}
\\
\hline
\sphinxstylestrong{18}
&
\sphinxstylestrong{R1}
&
\sphinxstylestrong{electric\_stove}
&
\sphinxstylestrong{3}
&
\sphinxstylestrong{0}
&
\sphinxstylestrong{0}
&
\sphinxstylestrong{0}
&
\sphinxstylestrong{0}
&
\sphinxstylestrong{0}
&
\sphinxstylestrong{1}
\\
\hline
\sphinxstylestrong{19}
&
\sphinxstylestrong{R1}
&
\sphinxstylestrong{electric\_stove}
&
\sphinxstylestrong{4}
&
\sphinxstylestrong{0}
&
\sphinxstylestrong{0}
&
\sphinxstylestrong{0}
&
\sphinxstylestrong{0}
&
\sphinxstylestrong{0}
&
\sphinxstylestrong{1.5}
\\
\hline
\sphinxstylestrong{20}
&
\sphinxstylestrong{R1}
&
\sphinxstylestrong{electric\_stove}
&
\sphinxstylestrong{5}
&
\sphinxstylestrong{0}
&
\sphinxstylestrong{0}
&
\sphinxstylestrong{0}
&
\sphinxstylestrong{0}
&
\sphinxstylestrong{0}
&
\sphinxstylestrong{2}
\\
\hline
\sphinxstylestrong{21}
&
\sphinxstylestrong{R1}
&
\sphinxstylestrong{electric\_stove}
&
\sphinxstylestrong{6}
&
\sphinxstylestrong{0}
&
\sphinxstylestrong{0}
&
\sphinxstylestrong{0}
&
\sphinxstylestrong{0}
&
\sphinxstylestrong{0}
&
\sphinxstylestrong{3}
\\
\hline
\sphinxstylestrong{22}
&
\sphinxstylestrong{R1}
&
\sphinxstylestrong{electric\_stove}
&
\sphinxstylestrong{7}
&
\sphinxstylestrong{0}
&
\sphinxstylestrong{0}
&
\sphinxstylestrong{0}
&
\sphinxstylestrong{0}
&
\sphinxstylestrong{0}
&
\sphinxstylestrong{2}
\\
\hline
\sphinxstylestrong{23}
&
\sphinxstylestrong{R1}
&
\sphinxstylestrong{electric\_stove}
&
\sphinxstylestrong{8}
&
\sphinxstylestrong{0}
&
\sphinxstylestrong{0}
&
\sphinxstylestrong{0}
&
\sphinxstylestrong{0}
&
\sphinxstylestrong{0}
&
\sphinxstylestrong{3}
\\
\hline
\sphinxstylestrong{24}
&
\sphinxstylestrong{R2}
&
\sphinxstylestrong{electric\_stove}
&
\sphinxstylestrong{1}
&
\sphinxstylestrong{0}
&
\sphinxstylestrong{0}
&
\sphinxstylestrong{0}
&
\sphinxstylestrong{0}
&
\sphinxstylestrong{0}
&
\sphinxstylestrong{1}
\\
\hline
\sphinxstylestrong{25}
&
\sphinxstylestrong{R2}
&
\sphinxstylestrong{electric\_stove}
&
\sphinxstylestrong{2}
&
\sphinxstylestrong{0}
&
\sphinxstylestrong{0}
&
\sphinxstylestrong{0}
&
\sphinxstylestrong{0}
&
\sphinxstylestrong{0}
&
\sphinxstylestrong{1}
\\
\hline
\sphinxstylestrong{26}
&
\sphinxstylestrong{R2}
&
\sphinxstylestrong{electric\_stove}
&
\sphinxstylestrong{3}
&
\sphinxstylestrong{0}
&
\sphinxstylestrong{0}
&
\sphinxstylestrong{0}
&
\sphinxstylestrong{0}
&
\sphinxstylestrong{0}
&
\sphinxstylestrong{1}
\\
\hline
\sphinxstylestrong{27}
&
\sphinxstylestrong{R2}
&
\sphinxstylestrong{electric\_stove}
&
\sphinxstylestrong{4}
&
\sphinxstylestrong{0}
&
\sphinxstylestrong{0}
&
\sphinxstylestrong{0}
&
\sphinxstylestrong{0}
&
\sphinxstylestrong{0}
&
\sphinxstylestrong{1.5}
\\
\hline
\sphinxstylestrong{28}
&
\sphinxstylestrong{R2}
&
\sphinxstylestrong{electric\_stove}
&
\sphinxstylestrong{5}
&
\sphinxstylestrong{0}
&
\sphinxstylestrong{0}
&
\sphinxstylestrong{0}
&
\sphinxstylestrong{0}
&
\sphinxstylestrong{0}
&
\sphinxstylestrong{2}
\\
\hline
\sphinxstylestrong{29}
&
\sphinxstylestrong{R2}
&
\sphinxstylestrong{electric\_stove}
&
\sphinxstylestrong{6}
&
\sphinxstylestrong{0}
&
\sphinxstylestrong{0}
&
\sphinxstylestrong{0}
&
\sphinxstylestrong{0}
&
\sphinxstylestrong{0}
&
\sphinxstylestrong{2}
\\
\hline
\sphinxstylestrong{30}
&
\sphinxstylestrong{R2}
&
\sphinxstylestrong{electric\_stove}
&
\sphinxstylestrong{7}
&
\sphinxstylestrong{0}
&
\sphinxstylestrong{0}
&
\sphinxstylestrong{0}
&
\sphinxstylestrong{0}
&
\sphinxstylestrong{0}
&
\sphinxstylestrong{2.5}
\\
\hline
\sphinxstylestrong{31}
&
\sphinxstylestrong{R2}
&
\sphinxstylestrong{electric\_stove}
&
\sphinxstylestrong{8}
&
\sphinxstylestrong{0}
&
\sphinxstylestrong{0}
&
\sphinxstylestrong{0}
&
\sphinxstylestrong{0}
&
\sphinxstylestrong{0}
&
\sphinxstylestrong{2}
\\
\hline
\end{tabulary}
\par
\sphinxattableend\end{savenotes}

For the purposes of brevity, we omitted the majority of the \sphinxcode{\sphinxupquote{gasboiler}} entries. However, these remain unchanged, apart from a \sphinxcode{\sphinxupquote{0}} entry in the cook column to indicate that a \sphinxcode{\sphinxupquote{gasboiler}} does not meet \sphinxcode{\sphinxupquote{cook}} demand.

We added an \sphinxcode{\sphinxupquote{electric\_stove}} process for each of the timeslices, which meets the \sphinxcode{\sphinxupquote{cook}} demand. This can be seen through the addition of a positive number in the \sphinxcode{\sphinxupquote{cook}} column.

The process is very similar for the \sphinxcode{\sphinxupquote{Residential2050Consumption.csv}} file, however, for this example, we often placed larger numbers to indicate higher demand in 2050. For the complete file see the link \sphinxhref{total-addition-service-demand-github}{here INCLUDE LINK HERE}

Next, we must edit the files within the \sphinxcode{\sphinxupquote{input}} folder. For this, we must add the \sphinxcode{\sphinxupquote{cook}} service demand to each of these files.

First, we will amend the \sphinxcode{\sphinxupquote{BaseYearExport.csv}} and \sphinxcode{\sphinxupquote{BaseYearImport.csv}} files. For this, we say that there is no import or export of the \sphinxcode{\sphinxupquote{cook}} service demand. A brief example is outlined below for \sphinxcode{\sphinxupquote{BaseYearExport.csv}}:


\begin{savenotes}\sphinxattablestart
\centering
\begin{tabular}[t]{|*{10}{\X{1}{10}|}}
\hline
\sphinxstyletheadfamily 
RegionName
&\sphinxstyletheadfamily 
Attribute
&\sphinxstyletheadfamily 
Time
&\sphinxstyletheadfamily 
electricity
&\sphinxstyletheadfamily 
gas
&\sphinxstyletheadfamily 
heat
&\sphinxstyletheadfamily 
CO2f
&\sphinxstyletheadfamily 
wind
&\sphinxstyletheadfamily 
solar
&\sphinxstyletheadfamily 
\sphinxstylestrong{cook}
\\
\hline
Unit
&\begin{itemize}
\item {} 
\end{itemize}
&
Year
&
PJ
&
PJ
&
PJ
&
kt
&
PJ
&
PJ
&
\sphinxstylestrong{PJ}
\\
\hline
R1
&
Exports
&
2010
&
0
&
0
&
0
&
0
&
0
&
0
&
\sphinxstylestrong{0}
\\
\hline
…
&
…
&
…
&
…
&
…
&
…
&
…
&
…
&
…
&
\sphinxstylestrong{…}
\\
\hline
R2
&
Exports
&
2100
&
0
&
0
&
0
&
0
&
0
&
0
&
\sphinxstylestrong{0}
\\
\hline
\end{tabular}
\par
\sphinxattableend\end{savenotes}

The same is true for the \sphinxcode{\sphinxupquote{BaseYearImport.csv}} file:


\begin{savenotes}\sphinxattablestart
\centering
\begin{tabular}[t]{|*{10}{\X{1}{10}|}}
\hline
\sphinxstyletheadfamily 
RegionName
&\sphinxstyletheadfamily 
Attribute
&\sphinxstyletheadfamily 
Time
&\sphinxstyletheadfamily 
electricity
&\sphinxstyletheadfamily 
gas
&\sphinxstyletheadfamily 
heat
&\sphinxstyletheadfamily 
CO2f
&\sphinxstyletheadfamily 
wind
&\sphinxstyletheadfamily 
solar
&\sphinxstyletheadfamily 
\sphinxstylestrong{cook}
\\
\hline
Unit
&\begin{itemize}
\item {} 
\end{itemize}
&
Year
&
PJ
&
PJ
&
PJ
&
kt
&
PJ
&
PJ
&
\sphinxstylestrong{PJ}
\\
\hline
R1
&
Imports
&
2010
&
0
&
0
&
0
&
0
&
0
&
0
&
\sphinxstylestrong{0}
\\
\hline
…
&
…
&
…
&
…
&
…
&
…
&
…
&
…
&
…
&
\sphinxstylestrong{…}
\\
\hline
R2
&
Imports
&
2100
&
0
&
0
&
0
&
0
&
0
&
0
&
\sphinxstylestrong{0}
\\
\hline
\end{tabular}
\par
\sphinxattableend\end{savenotes}

Next, we must edit the \sphinxcode{\sphinxupquote{GlobalCommodities.csv}} file. This is where we define the new commodity \sphinxcode{\sphinxupquote{cook}}. It tells MUSE the commodity type, name, emissions factor of CO2 and heat rate, amongst other things.

The example used for this tutorial is below:


\begin{savenotes}\sphinxattablestart
\centering
\begin{tabulary}{\linewidth}[t]{|T|T|T|T|T|T|}
\hline
\sphinxstyletheadfamily 
Commodity
&\sphinxstyletheadfamily 
CommodityType
&\sphinxstyletheadfamily 
CommodityName
&\sphinxstyletheadfamily 
CommodityEmissionFactor\_CO2
&\sphinxstyletheadfamily 
HeatRate
&\sphinxstyletheadfamily 
Unit
\\
\hline
Electricity
&
Energy
&
electricity
&
0
&
1
&
PJ
\\
\hline
Gas
&
Energy
&
gas
&
56.1
&
1
&
PJ
\\
\hline
Heat
&
Energy
&
heat
&
0
&
1
&
PJ
\\
\hline
Wind
&
Energy
&
wind
&
0
&
1
&
PJ
\\
\hline
CO2fuelcomsbustion
&
Environmental
&
CO2f
&
0
&
1
&
kt
\\
\hline
Solar
&
Energy
&
solar
&
0
&
1
&
PJ
\\
\hline
\sphinxstylestrong{Cook}
&
\sphinxstylestrong{Energy}
&
\sphinxstylestrong{cook}
&
\sphinxstylestrong{0}
&
\sphinxstylestrong{1}
&
\sphinxstylestrong{PJ}
\\
\hline
\end{tabulary}
\par
\sphinxattableend\end{savenotes}

Finally, the \sphinxcode{\sphinxupquote{Projections.csv}} file must be changed. This is a large file which details the expected cost of the technology in the first year of the simulation. Due to its size, we will only show two rows of the new column \sphinxcode{\sphinxupquote{cook}}.


\begin{savenotes}\sphinxattablestart
\centering
\begin{tabular}[t]{|*{5}{\X{1}{5}|}}
\hline
\sphinxstyletheadfamily 
RegionName
&\sphinxstyletheadfamily 
Attribute
&\sphinxstyletheadfamily 
Time
&\sphinxstyletheadfamily 
…
&\sphinxstyletheadfamily 
\sphinxstylestrong{cook}
\\
\hline
Unit
&\begin{itemize}
\item {} 
\end{itemize}
&
Year
&
…
&
\sphinxstylestrong{MUS\$2010/kt}
\\
\hline
R1
&
CommodityPrice
&
2010
&
…
&
\sphinxstylestrong{100}
\\
\hline
…
&
…
&
…
&
…
&
\sphinxstylestrong{…}
\\
\hline
R2
&
CommodityPrice
&
2100
&
…
&
\sphinxstylestrong{100}
\\
\hline
\end{tabular}
\par
\sphinxattableend\end{savenotes}

We set every price of cook to be \sphinxcode{\sphinxupquote{100MUS\$2010/kt}}


\subsection{Addition of cooking technology}
\label{\detokenize{user-guide/addition-service-demand:Addition-of-cooking-technology}}
Next, we must add a technology to service this new demand. This is achieved through a similar process as the section in the {\hyperref[\detokenize{user-guide/add-solar::doc}]{\sphinxcrossref{\DUrole{doc}{“adding a new technology”}}}} section. However, we must be careful to specify the end\sphinxhyphen{}use of the technology as \sphinxcode{\sphinxupquote{cook}}.

For this example, we will add two competing technologies to service the cooking demand: \sphinxcode{\sphinxupquote{electric\_stove}} and \sphinxcode{\sphinxupquote{gas\_stove}} to the \sphinxcode{\sphinxupquote{Technodata.csv}} file in \sphinxcode{\sphinxupquote{/technodata/residential/Technodata.csv}}.

Again for the interests of space, we have omitted the existing \sphinxcode{\sphinxupquote{gasboiler}} and \sphinxcode{\sphinxupquote{heatpump}} technologies. But we copy the \sphinxcode{\sphinxupquote{gasboiler}} row for \sphinxcode{\sphinxupquote{R1}} and paste it for the new \sphinxcode{\sphinxupquote{electric\_stove}} for both \sphinxcode{\sphinxupquote{R1}} and \sphinxcode{\sphinxupquote{R2}}. For \sphinxcode{\sphinxupquote{gas\_stove}} we copy and paste the data for \sphinxcode{\sphinxupquote{heatpump}} from region \sphinxcode{\sphinxupquote{R1}} for both \sphinxcode{\sphinxupquote{R1}} and \sphinxcode{\sphinxupquote{R2}}.

An important modification, however, is specifying the end\sphinxhyphen{}use for these new technologies to be \sphinxcode{\sphinxupquote{cook}} and not \sphinxcode{\sphinxupquote{heat}}.


\begin{savenotes}\sphinxattablestart
\centering
\begin{tabular}[t]{|*{10}{\X{1}{10}|}}
\hline
\sphinxstyletheadfamily 
ProcessName
&\sphinxstyletheadfamily 
RegionName
&\sphinxstyletheadfamily 
Time
&\sphinxstyletheadfamily 
Level
&\sphinxstyletheadfamily 
cap\_par
&\sphinxstyletheadfamily 
…
&\sphinxstyletheadfamily 
Fuel
&\sphinxstyletheadfamily 
EndUse
&\sphinxstyletheadfamily 
Agent2
&\sphinxstyletheadfamily 
Agent1
\\
\hline
Unit
&\begin{itemize}
\item {} 
\end{itemize}
&
Year
&\begin{itemize}
\item {} 
\end{itemize}
&
MUS\$2010/PJ\_a
&
…
&\begin{itemize}
\item {} 
\end{itemize}
&\begin{itemize}
\item {} 
\end{itemize}
&
Retrofit
&
New
\\
\hline
gasboiler
&
R1
&
2020
&
fixed
&
3.8
&
…
&
gas
&
heat
&
1
&
0
\\
\hline
…
&
…
&
…
&
…
&
…
&
…
&
…
&
…
&
…
&
…
\\
\hline
\sphinxstylestrong{electric\_stove}
&
\sphinxstylestrong{R1}
&
\sphinxstylestrong{2020}
&
\sphinxstylestrong{fixed}
&
\sphinxstylestrong{3.8}
&
\sphinxstylestrong{…}
&
\sphinxstylestrong{electricity}
&
\sphinxstylestrong{cook}
&
\sphinxstylestrong{1}
&
\sphinxstylestrong{0}
\\
\hline
\sphinxstylestrong{electric\_stove}
&
\sphinxstylestrong{R2}
&
\sphinxstylestrong{2020}
&
\sphinxstylestrong{fixed}
&
\sphinxstylestrong{3.8}
&
\sphinxstylestrong{…}
&
\sphinxstylestrong{electricity}
&
\sphinxstylestrong{cook}
&
\sphinxstylestrong{1}
&
\sphinxstylestrong{0}
\\
\hline
\sphinxstylestrong{gas\_stove}
&
\sphinxstylestrong{R1}
&
\sphinxstylestrong{2020}
&
\sphinxstylestrong{fixed}
&
\sphinxstylestrong{8.8667}
&
\sphinxstylestrong{…}
&
\sphinxstylestrong{gas}
&
\sphinxstylestrong{cook}
&
\sphinxstylestrong{1}
&
\sphinxstylestrong{0}
\\
\hline
\sphinxstylestrong{gas\_stove}
&
\sphinxstylestrong{R2}
&
\sphinxstylestrong{2020}
&
\sphinxstylestrong{fixed}
&
\sphinxstylestrong{8.8667}
&
\sphinxstylestrong{…}
&
\sphinxstylestrong{gas}
&
\sphinxstylestrong{cook}
&
\sphinxstylestrong{1}
&
\sphinxstylestrong{0}
\\
\hline
\end{tabular}
\par
\sphinxattableend\end{savenotes}

As can be seen we have added two technologies, in the two regions with different \sphinxcode{\sphinxupquote{cap\_par}} costs. We specified their respective fules, and the enduse for both is \sphinxcode{\sphinxupquote{cook}}. For the full file please see \sphinxhref{here}{here INSERT LINK HERE}.

We must also add the data for these new technologies to the following files:
\begin{itemize}
\item {} 
\sphinxcode{\sphinxupquote{CommIn.csv}}

\item {} 
\sphinxcode{\sphinxupquote{CommOut.csv}}

\item {} 
\sphinxcode{\sphinxupquote{ExistingCapacity.csv}}

\end{itemize}

This is largely a similar process to the tutorial shown in {\hyperref[\detokenize{user-guide/add-solar::doc}]{\sphinxcrossref{\DUrole{doc}{“adding a new technology”}}}}. We must add the input to each of the technologies (gas and electricity for \sphinxcode{\sphinxupquote{gas\_stove}} and \sphinxcode{\sphinxupquote{electric\_stove}} respectively), outputs of \sphinxcode{\sphinxupquote{cook}} for both and the existing capacity for each technology in each region.

Due to the additional demand for gas and electricity brought on by the new \sphinxcode{\sphinxupquote{cook}} demand, it is necessary to relax the growth constraints for \sphinxcode{\sphinxupquote{gassupply1}} in the \sphinxcode{\sphinxupquote{technodata/gas/technodata.csv}} file. For this example, we set this file as follows:


\begin{savenotes}\sphinxattablestart
\centering
\begin{tabular}[t]{|*{9}{\X{1}{9}|}}
\hline
\sphinxstyletheadfamily 
ProcessName
&\sphinxstyletheadfamily 
RegionName
&\sphinxstyletheadfamily 
Time
&\sphinxstyletheadfamily 
…
&\sphinxstyletheadfamily 
MaxCapacityAddition
&\sphinxstyletheadfamily 
MaxCapacityGrowth
&\sphinxstyletheadfamily 
TotalCapacityLimit
&\sphinxstyletheadfamily 
…
&\sphinxstyletheadfamily 
Agent1
\\
\hline
Unit
&\begin{itemize}
\item {} 
\end{itemize}
&
Year
&
…
&
PJ
&
\%
&
PJ
&
…
&
New
\\
\hline
gassupply1
&
R1
&
2020
&
…
&
\sphinxstylestrong{100}
&
\sphinxstylestrong{5}
&
\sphinxstylestrong{500}
&
…
&
0
\\
\hline
gassupply1
&
R2
&
2020
&
…
&
\sphinxstylestrong{100}
&
\sphinxstylestrong{5}
&
\sphinxstylestrong{120}
&
…
&
0
\\
\hline
\end{tabular}
\par
\sphinxattableend\end{savenotes}

To prevent repetition of the {\hyperref[\detokenize{user-guide/add-solar::doc}]{\sphinxcrossref{\DUrole{doc}{“adding a new technology”}}}} section, we will leave the full files \sphinxhref{link-here}{here INSERT LINK HERE}.

Again, we run the simulation with our modified input files using the following command, in the relevant directory:

\begin{sphinxVerbatim}[commandchars=\\\{\}]
\PYG{n}{python} \PYG{o}{\PYGZhy{}}\PYG{n}{m} \PYG{n}{pip} \PYG{n}{muse} \PYG{n}{settings}\PYG{o}{.}\PYG{n}{toml}
\end{sphinxVerbatim}

Once this has run we are ready to visualise our results.

{
\sphinxsetup{VerbatimColor={named}{nbsphinx-code-bg}}
\sphinxsetup{VerbatimBorderColor={named}{nbsphinx-code-border}}
\begin{sphinxVerbatim}[commandchars=\\\{\}]
\llap{\color{nbsphinxin}[2]:\,\hspace{\fboxrule}\hspace{\fboxsep}}\PYG{k+kn}{import} \PYG{n+nn}{pandas} \PYG{k}{as} \PYG{n+nn}{pd}
\PYG{k+kn}{import} \PYG{n+nn}{seaborn} \PYG{k}{as} \PYG{n+nn}{sns}
\PYG{k+kn}{import} \PYG{n+nn}{matplotlib}\PYG{n+nn}{.}\PYG{n+nn}{pyplot} \PYG{k}{as} \PYG{n+nn}{plt}
\end{sphinxVerbatim}
}

{
\sphinxsetup{VerbatimColor={named}{nbsphinx-code-bg}}
\sphinxsetup{VerbatimBorderColor={named}{nbsphinx-code-border}}
\begin{sphinxVerbatim}[commandchars=\\\{\}]
\llap{\color{nbsphinxin}[3]:\,\hspace{\fboxrule}\hspace{\fboxsep}}\PYG{n}{mca\PYGZus{}capacity} \PYG{o}{=} \PYG{n}{pd}\PYG{o}{.}\PYG{n}{read\PYGZus{}csv}\PYG{p}{(}\PYG{l+s+s2}{\PYGZdq{}}\PYG{l+s+s2}{../tutorial\PYGZhy{}code/add\PYGZhy{}service\PYGZhy{}demand/Results/MCACapacity.csv}\PYG{l+s+s2}{\PYGZdq{}}\PYG{p}{)}
\PYG{n}{mca\PYGZus{}capacity}\PYG{o}{.}\PYG{n}{head}\PYG{p}{(}\PYG{p}{)}


\end{sphinxVerbatim}
}

{

\kern-\sphinxverbatimsmallskipamount\kern-\baselineskip
\kern+\FrameHeightAdjust\kern-\fboxrule
\vspace{\nbsphinxcodecellspacing}

\sphinxsetup{VerbatimColor={named}{white}}
\sphinxsetup{VerbatimBorderColor={named}{nbsphinx-code-border}}
\begin{sphinxVerbatim}[commandchars=\\\{\}]
\llap{\color{nbsphinxout}[3]:\,\hspace{\fboxrule}\hspace{\fboxsep}}  technology region agent      type       sector  capacity  year
0  gas\_stove     R1    A1  retrofit  residential      10.0  2020
1  gasboiler     R1    A1  retrofit  residential      10.0  2020
2  gas\_stove     R2    A1  retrofit  residential      10.0  2020
3  gasboiler     R2    A1  retrofit  residential      10.0  2020
4  gas\_stove     R1    A2  retrofit  residential      10.0  2020
\end{sphinxVerbatim}
}

{
\sphinxsetup{VerbatimColor={named}{nbsphinx-code-bg}}
\sphinxsetup{VerbatimBorderColor={named}{nbsphinx-code-border}}
\begin{sphinxVerbatim}[commandchars=\\\{\}]
\llap{\color{nbsphinxin}[4]:\,\hspace{\fboxrule}\hspace{\fboxsep}}\PYG{k}{for} \PYG{n}{name}\PYG{p}{,} \PYG{n}{sector} \PYG{o+ow}{in} \PYG{n}{mca\PYGZus{}capacity}\PYG{o}{.}\PYG{n}{groupby}\PYG{p}{(}\PYG{l+s+s2}{\PYGZdq{}}\PYG{l+s+s2}{sector}\PYG{l+s+s2}{\PYGZdq{}}\PYG{p}{)}\PYG{p}{:}
    \PYG{n+nb}{print}\PYG{p}{(}\PYG{l+s+s2}{\PYGZdq{}}\PYG{l+s+si}{\PYGZob{}\PYGZcb{}}\PYG{l+s+s2}{ sector:}\PYG{l+s+s2}{\PYGZdq{}}\PYG{o}{.}\PYG{n}{format}\PYG{p}{(}\PYG{n}{name}\PYG{p}{)}\PYG{p}{)}
    \PYG{n}{fig}\PYG{p}{,} \PYG{n}{ax} \PYG{o}{=}\PYG{n}{plt}\PYG{o}{.}\PYG{n}{subplots}\PYG{p}{(}\PYG{l+m+mi}{1}\PYG{p}{,}\PYG{l+m+mi}{2}\PYG{p}{)}
    \PYG{n}{sns}\PYG{o}{.}\PYG{n}{lineplot}\PYG{p}{(}\PYG{n}{data}\PYG{o}{=}\PYG{n}{sector}\PYG{p}{[}\PYG{n}{sector}\PYG{o}{.}\PYG{n}{region}\PYG{o}{==}\PYG{l+s+s2}{\PYGZdq{}}\PYG{l+s+s2}{R1}\PYG{l+s+s2}{\PYGZdq{}}\PYG{p}{]}\PYG{p}{,} \PYG{n}{x}\PYG{o}{=}\PYG{l+s+s2}{\PYGZdq{}}\PYG{l+s+s2}{year}\PYG{l+s+s2}{\PYGZdq{}}\PYG{p}{,} \PYG{n}{y}\PYG{o}{=}\PYG{l+s+s2}{\PYGZdq{}}\PYG{l+s+s2}{capacity}\PYG{l+s+s2}{\PYGZdq{}}\PYG{p}{,} \PYG{n}{hue}\PYG{o}{=}\PYG{l+s+s2}{\PYGZdq{}}\PYG{l+s+s2}{technology}\PYG{l+s+s2}{\PYGZdq{}}\PYG{p}{,} \PYG{n}{ax}\PYG{o}{=}\PYG{n}{ax}\PYG{p}{[}\PYG{l+m+mi}{0}\PYG{p}{]}\PYG{p}{)}
    \PYG{n}{sns}\PYG{o}{.}\PYG{n}{lineplot}\PYG{p}{(}\PYG{n}{data}\PYG{o}{=}\PYG{n}{sector}\PYG{p}{[}\PYG{n}{sector}\PYG{o}{.}\PYG{n}{region}\PYG{o}{==}\PYG{l+s+s2}{\PYGZdq{}}\PYG{l+s+s2}{R2}\PYG{l+s+s2}{\PYGZdq{}}\PYG{p}{]}\PYG{p}{,} \PYG{n}{x}\PYG{o}{=}\PYG{l+s+s2}{\PYGZdq{}}\PYG{l+s+s2}{year}\PYG{l+s+s2}{\PYGZdq{}}\PYG{p}{,} \PYG{n}{y}\PYG{o}{=}\PYG{l+s+s2}{\PYGZdq{}}\PYG{l+s+s2}{capacity}\PYG{l+s+s2}{\PYGZdq{}}\PYG{p}{,} \PYG{n}{hue}\PYG{o}{=}\PYG{l+s+s2}{\PYGZdq{}}\PYG{l+s+s2}{technology}\PYG{l+s+s2}{\PYGZdq{}}\PYG{p}{,} \PYG{n}{ax}\PYG{o}{=}\PYG{n}{ax}\PYG{p}{[}\PYG{l+m+mi}{1}\PYG{p}{]}\PYG{p}{)}
    \PYG{n}{plt}\PYG{o}{.}\PYG{n}{show}\PYG{p}{(}\PYG{p}{)}
    \PYG{n}{plt}\PYG{o}{.}\PYG{n}{close}\PYG{p}{(}\PYG{p}{)}
\end{sphinxVerbatim}
}

{

\kern-\sphinxverbatimsmallskipamount\kern-\baselineskip
\kern+\FrameHeightAdjust\kern-\fboxrule
\vspace{\nbsphinxcodecellspacing}

\sphinxsetup{VerbatimColor={named}{white}}
\sphinxsetup{VerbatimBorderColor={named}{nbsphinx-code-border}}
\begin{sphinxVerbatim}[commandchars=\\\{\}]
gas sector:
\end{sphinxVerbatim}
}

\hrule height -\fboxrule\relax
\vspace{\nbsphinxcodecellspacing}

\makeatletter\setbox\nbsphinxpromptbox\box\voidb@x\makeatother

\begin{nbsphinxfancyoutput}

\noindent\sphinxincludegraphics[width=388\sphinxpxdimen,height=262\sphinxpxdimen]{{user-guide_addition-service-demand_15_1}.png}

\end{nbsphinxfancyoutput}

{

\kern-\sphinxverbatimsmallskipamount\kern-\baselineskip
\kern+\FrameHeightAdjust\kern-\fboxrule
\vspace{\nbsphinxcodecellspacing}

\sphinxsetup{VerbatimColor={named}{white}}
\sphinxsetup{VerbatimBorderColor={named}{nbsphinx-code-border}}
\begin{sphinxVerbatim}[commandchars=\\\{\}]
power sector:
\end{sphinxVerbatim}
}

\hrule height -\fboxrule\relax
\vspace{\nbsphinxcodecellspacing}

\makeatletter\setbox\nbsphinxpromptbox\box\voidb@x\makeatother

\begin{nbsphinxfancyoutput}

\noindent\sphinxincludegraphics[width=395\sphinxpxdimen,height=262\sphinxpxdimen]{{user-guide_addition-service-demand_15_3}.png}

\end{nbsphinxfancyoutput}

{

\kern-\sphinxverbatimsmallskipamount\kern-\baselineskip
\kern+\FrameHeightAdjust\kern-\fboxrule
\vspace{\nbsphinxcodecellspacing}

\sphinxsetup{VerbatimColor={named}{white}}
\sphinxsetup{VerbatimBorderColor={named}{nbsphinx-code-border}}
\begin{sphinxVerbatim}[commandchars=\\\{\}]
residential sector:
\end{sphinxVerbatim}
}

\hrule height -\fboxrule\relax
\vspace{\nbsphinxcodecellspacing}

\makeatletter\setbox\nbsphinxpromptbox\box\voidb@x\makeatother

\begin{nbsphinxfancyoutput}

\noindent\sphinxincludegraphics[width=388\sphinxpxdimen,height=262\sphinxpxdimen]{{user-guide_addition-service-demand_15_5}.png}

\end{nbsphinxfancyoutput}

We can see our new technology, the \sphinxcode{\sphinxupquote{gas\_stove}} is used over the \sphinxcode{\sphinxupquote{electric\_stove}}. Therefore, there is an increase in \sphinxcode{\sphinxupquote{gassupply1}} to accommodate for this growth in demand. However, this is not enough to displace \sphinxcode{\sphinxupquote{windturbine}} by \sphinxcode{\sphinxupquote{gasCCGT}}.


\subsection{Next steps}
\label{\detokenize{user-guide/addition-service-demand:Next-steps}}
This brings us to the end of the user guide! Using the information explained in this tutorial, or following similar steps, you will be able to create complex scenarios of your choosing.

For the full code to generate the final results, see \sphinxhref{dead-link}{here INSERT LINK HERE}.


\chapter{Input Files}
\label{\detokenize{inputs/index:input-files}}\label{\detokenize{inputs/index:id1}}\label{\detokenize{inputs/index::doc}}
In this section we detail each of the files required to run MUSE. We include information based on how these files should be used, as well as the data that populates them.


\section{TOML primer}
\label{\detokenize{inputs/toml_primer:toml-primer}}\label{\detokenize{inputs/toml_primer:id1}}\label{\detokenize{inputs/toml_primer::doc}}
The full specification for TOML files can be found
\sphinxhref{https://github.com/toml-lang/toml}{here}.
A TOML file is separated into sections, with each section except the topmost
introduced by a name in square brackets. Sections can hold key\sphinxhyphen{}value pairs,
e.g. a name associated with a value. For instance:

\begin{sphinxVerbatim}[commandchars=\\\{\}]
\PYG{n}{general\PYGZus{}string\PYGZus{}attribute} \PYG{o}{=} \PYG{l+s}{\PYGZdq{}x\PYGZdq{}}

\PYG{k}{[some\PYGZus{}section]}
\PYG{n}{section\PYGZus{}attribute} \PYG{o}{=} \PYG{l+m+mi}{12}

\PYG{k}{[some\PYGZus{}section.subsection]}
\PYG{n}{subsetion\PYGZus{}attribute} \PYG{o}{=} \PYG{k+kc}{true}
\end{sphinxVerbatim}

TOML is quite flexible in how one can define sections and attributes. The following
three examples are equivalent:

\begin{sphinxVerbatim}[commandchars=\\\{\}]
\PYG{k}{[sectors.residential.production]}
\PYG{n}{name} \PYG{o}{=} \PYG{l+s}{\PYGZdq{}match\PYGZdq{}}
\PYG{n}{costing} \PYG{o}{=} \PYG{l+s}{\PYGZdq{}prices\PYGZdq{}}
\end{sphinxVerbatim}

\begin{sphinxVerbatim}[commandchars=\\\{\}]
\PYG{k}{[sectors.residential]}
\PYG{n}{production} \PYG{o}{=} \PYG{p}{\PYGZob{}}\PYG{l+s}{\PYGZdq{}name\PYGZdq{}}\PYG{p}{:} \PYG{l+s}{\PYGZdq{}match\PYGZdq{}}\PYG{p}{,} \PYG{l+s}{\PYGZdq{}costing\PYGZdq{}}\PYG{p}{:} \PYG{l+s}{\PYGZdq{}prices\PYGZdq{}}\PYG{p}{\PYGZcb{}}
\end{sphinxVerbatim}

\begin{sphinxVerbatim}[commandchars=\\\{\}]
\PYG{k}{[sectors.residential]}
\PYG{n}{production}\PYG{p}{.}\PYG{n}{name} \PYG{o}{=} \PYG{l+s}{\PYGZdq{}match\PYGZdq{}}
\PYG{n}{production}\PYG{p}{.}\PYG{n}{costing} \PYG{o}{=} \PYG{l+s}{\PYGZdq{}prices\PYGZdq{}}
\end{sphinxVerbatim}
\phantomsection\label{\detokenize{inputs/toml_primer:toml-array}}
Additionally, TOML files can contain tabular data, specified row\sphinxhyphen{}by\sphinxhyphen{}row using double
square bracket. For instance, below we define a table with two rows and a single
\sphinxstyleemphasis{column} called \sphinxtitleref{some\_table\_of\_data} (though column is not quite the right term, TOML tables are made more
flexible than most tabular formats. Rather, each row can be considered a
dictionary).

\begin{sphinxVerbatim}[commandchars=\\\{\}]
\PYG{k}{[[some\PYGZus{}table\PYGZus{}of\PYGZus{}data]]}
\PYG{n}{a\PYGZus{}key} \PYG{o}{=} \PYG{l+s}{\PYGZdq{}a value\PYGZdq{}}

\PYG{k}{[[some\PYGZus{}table\PYGZus{}of\PYGZus{}data]]}
\PYG{n}{a\PYGZus{}key} \PYG{o}{=} \PYG{l+s}{\PYGZdq{}another value\PYGZdq{}}
\end{sphinxVerbatim}

As MUSE requires a number of data file, paths to files can be formated in a flexible manner. Paths can be formatted with shorthands for specific directories and are defined with curly\sphinxhyphen{}brackets. For example:

\begin{sphinxVerbatim}[commandchars=\\\{\}]
\PYG{n}{projection} \PYG{o}{=} \PYG{l+s}{\PYGZsq{}\PYGZob{}path\PYGZcb{}/inputs/projection.csv\PYGZsq{}}
\PYG{n}{timeslices\PYGZus{}path} \PYG{o}{=} \PYG{l+s}{\PYGZsq{}\PYGZob{}cwd\PYGZcb{}/technodata/timeslices.csv\PYGZsq{}}
\PYG{n}{consumption\PYGZus{}path} \PYG{o}{=} \PYG{l+s}{\PYGZsq{}\PYGZob{}muse\PYGZus{}sectors\PYGZcb{}/technodata/timeslices.csv\PYGZsq{}}
\end{sphinxVerbatim}
\begin{description}
\item[{path}] \leavevmode
refers to the directory where the TOML file is located

\item[{cwd}] \leavevmode
refers to the directory from which the muse simulation is launched

\item[{muse\_sectors}] \leavevmode
refers to the directory where default sectoral data is located

\end{description}


\section{Simulation settings}
\label{\detokenize{inputs/toml:simulation-settings}}\label{\detokenize{inputs/toml:id1}}\label{\detokenize{inputs/toml::doc}}
This section details the TOML input for MUSE. The format for TOML files is
described in this {\hyperref[\detokenize{inputs/toml_primer:toml-primer}]{\sphinxcrossref{\DUrole{std,std-ref}{previous section}}}}. Here, however, we focus on sections and
attributes that are specific to MUSE.

The TOML file can be read using \sphinxcode{\sphinxupquote{read\_settings()}}. The resulting
data is used to construt the market clearing algorithm directly in the \sphinxcode{\sphinxupquote{MCA\textquotesingle{}s
factory function}}.


\subsection{Main section}
\label{\detokenize{inputs/toml:main-section}}
This is the topmost section. It contains settings relevant to the simulation as
a whole.

\begin{sphinxVerbatim}[commandchars=\\\{\}]
\PYG{n}{time\PYGZus{}framework} \PYG{o}{=} \PYG{k}{[2020, 2025, 2030, 2035, 2040, 2045, 2050]}
\PYG{n}{regions} \PYG{o}{=} \PYG{k}{[\PYGZdq{}USA\PYGZdq{}]}
\PYG{n}{interpolation\PYGZus{}mode} \PYG{o}{=} \PYG{l+s}{\PYGZsq{}Active\PYGZsq{}}
\PYG{n}{log\PYGZus{}level} \PYG{o}{=} \PYG{l+s}{\PYGZsq{}info\PYGZsq{}}

\PYG{n}{equilibrium\PYGZus{}variable} \PYG{o}{=} \PYG{l+s}{\PYGZsq{}demand\PYGZsq{}}
\PYG{n}{maximum\PYGZus{}iterations} \PYG{o}{=} \PYG{l+m+mi}{100}
\PYG{n}{tolerance} \PYG{o}{=} \PYG{l+m+mf}{0.1}
\PYG{n}{tolerance\PYGZus{}unmet\PYGZus{}demand} \PYG{o}{=} \PYG{l+m+mi}{\PYGZhy{}0}\PYG{l+m+mf}{.1}
\end{sphinxVerbatim}
\begin{description}
\item[{time\_framework}] \leavevmode
Required. List of years for which the simulation will run.

\item[{region}] \leavevmode
Subset of regions to consider. If not given, defaults to all regions found in the
simulation data.

\item[{interpolation\_mode}] \leavevmode
interpolation when reading the initial market. One of
\sphinxtitleref{linear}, \sphinxtitleref{nearest}, \sphinxtitleref{zero}, \sphinxtitleref{slinear}, \sphinxtitleref{quadratic}, \sphinxtitleref{cubic}. Defaults to \sphinxtitleref{linear}.

\item[{log\_level:}] \leavevmode
verbosity of the output.

\item[{equilibirum\_variable}] \leavevmode
whether equilibrium of \sphinxtitleref{demand} or \sphinxtitleref{prices} should be sought. Defaults to \sphinxtitleref{demand}.

\item[{maximum\_iterations}] \leavevmode
Maximum number of iterations when searching for equilibrium. Defaults to 3.

\item[{tolerance}] \leavevmode
Tolerance criteria when checking for equilibrium. Defaults to 0.1.

\item[{tolerance\_unmet\_demand}] \leavevmode
Criteria checking whether the demand has been met.  Defaults to \sphinxhyphen{}0.1.

\item[{excluded\_commodities}] \leavevmode
List of commodities excluded from the equilibrium considerations. Defaults to the
list \sphinxtitleref{{[}“CO2f”, “CO2r”, “CO2c”, “CO2s”, “CH4”, “N2O”, “f\sphinxhyphen{}gases”{]}}.

\item[{plugins}] \leavevmode
Path or list of paths to extra python plugins, i.e. files with registered functions
such as \sphinxcode{\sphinxupquote{register\_output\_quantity()}}.

\end{description}


\subsection{Carbon market}
\label{\detokenize{inputs/toml:carbon-market}}
This section containts the settings related to the modelling of the carbon market. If omitted, it defaults to not
including the carbon market in the simulation.

Example

\begin{sphinxVerbatim}[commandchars=\\\{\}]
\PYG{k}{[carbon\PYGZus{}budget\PYGZus{}control]}
\PYG{n}{budget} \PYG{o}{=} \PYG{k}{[]}
\end{sphinxVerbatim}
\begin{description}
\item[{budget}] \leavevmode
Yearly budget. There should be one item for each year the simulation will run. In
other words, if given and not empty, this is a list with the same length as
\sphinxtitleref{time\_framework} from the main section. If not given or an empty list, then the
carbon market feature is disabled. Defaults to an empty list.

\item[{method}] \leavevmode
Method used to equilibrate the carbon market. Defaults to a simple iterative scheme. {[}INSERT OPTIONS HERE{]}

\item[{commodities}] \leavevmode
Commodities that make up the carbon market. Defaults to an empty list.

\item[{control\_undershoot}] \leavevmode
Whether to control carbon budget undershoots. Defaults to True.

\item[{control\_overshoot}] \leavevmode
Whether to control carbon budget overshoots. Defaults to True.

\item[{method\_options:}] \leavevmode
Additional options for the specific carbon method.

\end{description}


\subsection{Global input files}
\label{\detokenize{inputs/toml:global-input-files}}
Defines the paths specific simulation data files. The paths can be formatted as
explained in the {\hyperref[\detokenize{inputs/toml_primer:toml-primer}]{\sphinxcrossref{\DUrole{std,std-ref}{TOML primer}}}}.

\begin{sphinxVerbatim}[commandchars=\\\{\}]
\PYG{k}{[global\PYGZus{}input\PYGZus{}files]}
\PYG{n}{projections} \PYG{o}{=} \PYG{l+s}{\PYGZsq{}\PYGZob{}path\PYGZcb{}/inputs/Projections.csv\PYGZsq{}}
\PYG{n}{regions} \PYG{o}{=} \PYG{l+s}{\PYGZsq{}\PYGZob{}path\PYGZcb{}/inputs/Regions.csv\PYGZsq{}}
\PYG{n}{global\PYGZus{}commodities} \PYG{o}{=} \PYG{l+s}{\PYGZsq{}\PYGZob{}path\PYGZcb{}/inputs/MUSEGlobalCommodities.csv\PYGZsq{}}
\end{sphinxVerbatim}
\begin{description}
\item[{projections:}] \leavevmode
Path to a csv file giving initial market projection. See {\hyperref[\detokenize{inputs/projections:inputs-projection}]{\sphinxcrossref{\DUrole{std,std-ref}{Initial Market Projection}}}}.

\item[{regions:}] \leavevmode
Path to a csv file describing the regions. See
{\hyperref[\detokenize{inputs/regions:regional-data}]{\sphinxcrossref{\DUrole{std,std-ref}{Regional data}}}}.

\item[{global\_commodities:}] \leavevmode
Path to a csv file describing the comodities in the simulation. See
{\hyperref[\detokenize{inputs/commodities:inputs-commodities}]{\sphinxcrossref{\DUrole{std,std-ref}{Commodity Description}}}}.

\end{description}


\subsection{Timeslices}
\label{\detokenize{inputs/toml:timeslices}}\label{\detokenize{inputs/toml:timeslices-toml}}
Time\sphinxhyphen{}slices represent a sub\sphinxhyphen{}year disaggregation of commodity demand. Generally,
timeslices are expected to introduce several levels, e.g. season, day, or hour. The
simplest is to show the TOML for the default timeslice:

\begin{sphinxVerbatim}[commandchars=\\\{\}]
\PYG{k}{[timeslices]}
\PYG{n}{winter}\PYG{p}{.}\PYG{n}{weekday}\PYG{p}{.}\PYG{n}{night} \PYG{o}{=} \PYG{l+m+mi}{396}
\PYG{n}{winter}\PYG{p}{.}\PYG{n}{weekday}\PYG{p}{.}\PYG{n}{morning} \PYG{o}{=} \PYG{l+m+mi}{396}
\PYG{n}{winter}\PYG{p}{.}\PYG{n}{weekday}\PYG{p}{.}\PYG{n}{afternoon} \PYG{o}{=} \PYG{l+m+mi}{264}
\PYG{n}{winter}\PYG{p}{.}\PYG{n}{weekday}\PYG{p}{.}\PYG{n}{early\PYGZhy{}peak} \PYG{o}{=} \PYG{l+m+mi}{66}
\PYG{n}{winter}\PYG{p}{.}\PYG{n}{weekday}\PYG{p}{.}\PYG{n}{late\PYGZhy{}peak} \PYG{o}{=} \PYG{l+m+mi}{66}
\PYG{n}{winter}\PYG{p}{.}\PYG{n}{weekday}\PYG{p}{.}\PYG{n}{evening} \PYG{o}{=} \PYG{l+m+mi}{396}
\PYG{n}{winter}\PYG{p}{.}\PYG{n}{weekend}\PYG{p}{.}\PYG{n}{night} \PYG{o}{=} \PYG{l+m+mi}{156}
\PYG{n}{winter}\PYG{p}{.}\PYG{n}{weekend}\PYG{p}{.}\PYG{n}{morning} \PYG{o}{=} \PYG{l+m+mi}{156}
\PYG{n}{winter}\PYG{p}{.}\PYG{n}{weekend}\PYG{p}{.}\PYG{n}{afternoon} \PYG{o}{=} \PYG{l+m+mi}{156}
\PYG{n}{winter}\PYG{p}{.}\PYG{n}{weekend}\PYG{p}{.}\PYG{n}{evening} \PYG{o}{=} \PYG{l+m+mi}{156}
\PYG{n}{spring\PYGZhy{}autumn}\PYG{p}{.}\PYG{n}{weekday}\PYG{p}{.}\PYG{n}{night} \PYG{o}{=} \PYG{l+m+mi}{792}
\PYG{n}{spring\PYGZhy{}autumn}\PYG{p}{.}\PYG{n}{weekday}\PYG{p}{.}\PYG{n}{morning} \PYG{o}{=} \PYG{l+m+mi}{792}
\PYG{n}{spring\PYGZhy{}autumn}\PYG{p}{.}\PYG{n}{weekday}\PYG{p}{.}\PYG{n}{afternoon} \PYG{o}{=} \PYG{l+m+mi}{528}
\PYG{n}{spring\PYGZhy{}autumn}\PYG{p}{.}\PYG{n}{weekday}\PYG{p}{.}\PYG{n}{early\PYGZhy{}peak} \PYG{o}{=} \PYG{l+m+mi}{132}
\PYG{n}{spring\PYGZhy{}autumn}\PYG{p}{.}\PYG{n}{weekday}\PYG{p}{.}\PYG{n}{late\PYGZhy{}peak} \PYG{o}{=} \PYG{l+m+mi}{132}
\PYG{n}{spring\PYGZhy{}autumn}\PYG{p}{.}\PYG{n}{weekday}\PYG{p}{.}\PYG{n}{evening} \PYG{o}{=} \PYG{l+m+mi}{792}
\PYG{n}{spring\PYGZhy{}autumn}\PYG{p}{.}\PYG{n}{weekend}\PYG{p}{.}\PYG{n}{night} \PYG{o}{=} \PYG{l+m+mi}{300}
\PYG{n}{spring\PYGZhy{}autumn}\PYG{p}{.}\PYG{n}{weekend}\PYG{p}{.}\PYG{n}{morning} \PYG{o}{=} \PYG{l+m+mi}{300}
\PYG{n}{spring\PYGZhy{}autumn}\PYG{p}{.}\PYG{n}{weekend}\PYG{p}{.}\PYG{n}{afternoon} \PYG{o}{=} \PYG{l+m+mi}{300}
\PYG{n}{spring\PYGZhy{}autumn}\PYG{p}{.}\PYG{n}{weekend}\PYG{p}{.}\PYG{n}{evening} \PYG{o}{=} \PYG{l+m+mi}{300}
\PYG{n}{summer}\PYG{p}{.}\PYG{n}{weekday}\PYG{p}{.}\PYG{n}{night} \PYG{o}{=} \PYG{l+m+mi}{396}
\PYG{n}{summer}\PYG{p}{.}\PYG{n}{weekday}\PYG{p}{.}\PYG{n}{morning}  \PYG{o}{=} \PYG{l+m+mi}{396}
\PYG{n}{summer}\PYG{p}{.}\PYG{n}{weekday}\PYG{p}{.}\PYG{n}{afternoon} \PYG{o}{=} \PYG{l+m+mi}{264}
\PYG{n}{summer}\PYG{p}{.}\PYG{n}{weekday}\PYG{p}{.}\PYG{n}{early\PYGZhy{}peak} \PYG{o}{=} \PYG{l+m+mi}{66}
\PYG{n}{summer}\PYG{p}{.}\PYG{n}{weekday}\PYG{p}{.}\PYG{n}{late\PYGZhy{}peak} \PYG{o}{=} \PYG{l+m+mi}{66}
\PYG{n}{summer}\PYG{p}{.}\PYG{n}{weekday}\PYG{p}{.}\PYG{n}{evening} \PYG{o}{=} \PYG{l+m+mi}{396}
\PYG{n}{summer}\PYG{p}{.}\PYG{n}{weekend}\PYG{p}{.}\PYG{n}{night} \PYG{o}{=} \PYG{l+m+mi}{150}
\PYG{n}{summer}\PYG{p}{.}\PYG{n}{weekend}\PYG{p}{.}\PYG{n}{morning} \PYG{o}{=} \PYG{l+m+mi}{150}
\PYG{n}{summer}\PYG{p}{.}\PYG{n}{weekend}\PYG{p}{.}\PYG{n}{afternoon} \PYG{o}{=} \PYG{l+m+mi}{150}
\PYG{n}{summer}\PYG{p}{.}\PYG{n}{weekend}\PYG{p}{.}\PYG{n}{evening} \PYG{o}{=} \PYG{l+m+mi}{150}
\PYG{n}{level\PYGZus{}names} \PYG{o}{=} \PYG{k}{[\PYGZdq{}month\PYGZdq{}, \PYGZdq{}day\PYGZdq{}, \PYGZdq{}hour\PYGZdq{}]}
\end{sphinxVerbatim}

This input introduces three levels, via \sphinxcode{\sphinxupquote{level\_names}}: \sphinxcode{\sphinxupquote{month}}, \sphinxcode{\sphinxupquote{day}}, \sphinxcode{\sphinxupquote{hours}}.
Other simulations may want fewer or more levels.  The \sphinxcode{\sphinxupquote{month}} level is split into
three points of data, \sphinxcode{\sphinxupquote{winter}}, \sphinxcode{\sphinxupquote{spring\sphinxhyphen{}autumn}}, \sphinxcode{\sphinxupquote{summer}}. Then \sphinxcode{\sphinxupquote{day}} splits out
weekdays from weekends, and so on. Each line indicates the number of hours for the
relevant slice. It should be noted that the slices are not a cartesian products of each
levels. For instance, there no \sphinxcode{\sphinxupquote{peak}} periods during weekends. All that matters is
that the relative weights (i.e. the number of hours) are consistent and sum up to a
year.

The input above defines the finest times slice in the code. In order to define rougher
timeslices we can introduce items in each levels that represent aggregates at that
level. By default, we have the following:

\begin{sphinxVerbatim}[commandchars=\\\{\}]
\PYG{k}{[timeslices.aggregates]}
\PYG{n}{all\PYGZhy{}day} \PYG{o}{=} \PYG{k}{[\PYGZdq{}night\PYGZdq{}, \PYGZdq{}morning\PYGZdq{}, \PYGZdq{}afternoon\PYGZdq{}, \PYGZdq{}early\PYGZhy{}peak\PYGZdq{}, \PYGZdq{}late\PYGZhy{}peak\PYGZdq{}, \PYGZdq{}evening\PYGZdq{}]}
\PYG{n}{all\PYGZhy{}week} \PYG{o}{=} \PYG{k}{[\PYGZdq{}weekday\PYGZdq{}, \PYGZdq{}weekend\PYGZdq{}]}
\PYG{n}{all\PYGZhy{}year} \PYG{o}{=} \PYG{k}{[\PYGZdq{}winter\PYGZdq{}, \PYGZdq{}summer\PYGZdq{}, \PYGZdq{}spring\PYGZhy{}autumn\PYGZdq{}]}
\end{sphinxVerbatim}

Here, \sphinxcode{\sphinxupquote{all\sphinxhyphen{}day}} aggregates the full day. However, one could potentially create
aggregates such as:

\begin{sphinxVerbatim}[commandchars=\\\{\}]
\PYG{k}{[timeslices.aggregates]}
\PYG{n}{daylight} \PYG{o}{=} \PYG{k}{[\PYGZdq{}morning\PYGZdq{}, \PYGZdq{}afternoon\PYGZdq{}, \PYGZdq{}early\PYGZhy{}peak\PYGZdq{}, \PYGZdq{}late\PYGZhy{}peak\PYGZdq{}]}
\PYG{n}{nightlife} \PYG{o}{=} \PYG{k}{[\PYGZdq{}evening\PYGZdq{}, \PYGZdq{}night\PYGZdq{}]}
\end{sphinxVerbatim}

Once the finest timeslice and its aggregates are given, it is  possible for each sector
to define the timeslice simply by refering to the slices it will use at each level.

\begin{sphinxVerbatim}[commandchars=\\\{\}]
\PYG{k}{[sectors.some\PYGZus{}sector.timeslice\PYGZus{}levels]}
\PYG{n}{day} \PYG{o}{=} \PYG{k}{[\PYGZdq{}daylight\PYGZdq{}, \PYGZdq{}nightlife\PYGZdq{}]}
\PYG{n}{month} \PYG{o}{=} \PYG{k}{[\PYGZdq{}all\PYGZhy{}year\PYGZdq{}]}
\end{sphinxVerbatim}

Above, \sphinxcode{\sphinxupquote{sectors.some\_sector.timeslice\_levels.week}} defaults its value in the finest
timeslice. Indeed, if the subsection \sphinxcode{\sphinxupquote{sectors.some\_sector.timeslice\_levels}} is not
given, then the sector will default to using the finest timeslices.

Similarly, it is possible to specify a timeslice for the mca by adding an
\sphinxtitleref{mca.timeslice\_levels} section. However, be aware that if the MCA uses a rougher
timeslice framework, the market will be expressed within it. Hence information from
sectors with a finer timeslice framework will be lost.


\subsection{Standard sectors}
\label{\detokenize{inputs/toml:standard-sectors}}
Sectors are declared in the TOML file by adding a subsection to the \sphinxtitleref{sectors} section:

\begin{sphinxVerbatim}[commandchars=\\\{\}]
\PYG{k}{[sectors.residential]}
\PYG{n}{type} \PYG{o}{=} \PYG{l+s}{\PYGZsq{}default\PYGZsq{}}
\PYG{k}{[sectors.power]}
\PYG{n}{type} \PYG{o}{=} \PYG{l+s}{\PYGZsq{}default\PYGZsq{}}
\end{sphinxVerbatim}

Above, we’ve added two sectors, residential and power. The name of the subsection is
only used for identification. In other words, it should be chosen to be meaningful to
the user, since it will not affect the model itself.

Sectors are defined in \sphinxcode{\sphinxupquote{Sector}}.

A sector accepts a number of attributes and subsections.

\phantomsection\label{\detokenize{inputs/toml:sector-type}}\begin{description}
\item[{type}] \leavevmode
Defines the kind of sector this is. \sphinxstyleemphasis{Standard} sectors are those with type
“default”. This value corresponds to the name with which a sector class is registerd
with MUSE, via \sphinxcode{\sphinxupquote{register\_sector()}}. {[}INSERT OTHER OPTIONS HERE{]}

\end{description}
\phantomsection\label{\detokenize{inputs/toml:sector-priority}}\begin{description}
\item[{priority}] \leavevmode
An integer denoting which sectors runs when. Lower values imply the sector will run
earlier. If two sectors share the same priority. Later sectors can depend on earlier
sectors for the their input. If two sectors share the same priority, then their
order is not defined. Indeed, it should indicate that they can run in parallel.
For simplicity, the keyword also accepts standard values:
\begin{itemize}
\item {} 
“preset”: 0

\item {} 
“demand”: 10

\item {} 
“conversion”: 20

\item {} 
“supply”: 30

\item {} 
“last”: 100

\end{itemize}

Defaults to “last”.

\item[{interpolation}] \leavevmode
Interpolation method user when filling in missing values. Available interpolation
methods depend on the underlying \sphinxhref{https://docs.scipy.org/doc/scipy/reference/generated/scipy.interpolate.interp1d.html}{scipy method’s kind attribute}.

\item[{investment\_production}] \leavevmode
In its simplest form, this is the name of a method to compute the production from a
sector, as used when splitting the demand across agents. In other words, this is the
computation of the production which affects future investments. In it’s more general
form, \sphinxstyleemphasis{production} can be a subsection of its own, with a “name” attribute. For
instance:

\begin{sphinxVerbatim}[commandchars=\\\{\}]
\PYG{k}{[sectors.residential.production]}
\PYG{n}{name} \PYG{o}{=} \PYG{l+s}{\PYGZdq{}match\PYGZdq{}}
\PYG{n}{costing} \PYG{o}{=} \PYG{l+s}{\PYGZdq{}prices\PYGZdq{}}
\end{sphinxVerbatim}

MUSE provides two methods in \sphinxcode{\sphinxupquote{muse.production}}:
\begin{itemize}
\item {} \begin{description}
\item[{share: the production is the maximum production for the existing capacity and}] \leavevmode
the technology’s utilization factor.
See \sphinxcode{\sphinxupquote{muse.production.maximum\_production()}}.

\end{description}

\item {} \begin{description}
\item[{match: production and demand are matched according to a given cost metric. The}] \leavevmode
cost metric defaults to “prices”. It can be modified by using the general form
given above, with a “costing” attribute. The latter can be “prices”,
“gross\_margin”, or “lcoe”.
See \sphinxcode{\sphinxupquote{muse.production.demand\_matched\_production()}}.

\end{description}

\end{itemize}

\sphinxstyleemphasis{production} can also refer to any custom production method registered with MUSE via
\sphinxcode{\sphinxupquote{muse.production.register\_production()}}.

Defaults to “share”.

\item[{dispatch\_production}] \leavevmode
The name of the production method used to compute the sector’s output, as returned
to the muse market clearing algorithm. In other words, this is computation of the
production method which will affect other sectors.

It has the same format and options as the \sphinxstyleemphasis{production} attribute above.

\item[{demand\_share}] \leavevmode
A method used to split the MCA demand into seperate parts to be serviced by specific
agents. There is currently only one option, “new\_and\_retro”, corresponding to \sphinxstyleemphasis{new}
and \sphinxstyleemphasis{retro} agents.

\item[{interactions}] \leavevmode
Defines interactions between agents. These interactions take place right before new
investments are computed. The interactions can be anything. They are expected to
modify the agents and their assets. MUSE provides a default set of interactions that
have \sphinxstyleemphasis{new} agents pass on their assets to the corresponding \sphinxstyleemphasis{retro} agent, and the
\sphinxstyleemphasis{retro} agents pass on the make\sphinxhyphen{}up of their assets to the corresponding \sphinxstyleemphasis{new}
agents.

\sphinxstyleemphasis{interactions} are specified as a {\hyperref[\detokenize{inputs/toml_primer:toml-array}]{\sphinxcrossref{\DUrole{std,std-ref}{TOML array}}}}, e.g. with double
brackets. Each sector can specify an arbitrary number of interactaction, simply by
adding an extra interaction row.

There are two orthogonal concepts to interactions:
\begin{itemize}
\item {} 
a \sphinxstyleemphasis{net} defines the set of agents that interact. A set can contain any
number of agents, whether zero, two, or all agents in a sector. See
\sphinxcode{\sphinxupquote{muse.interactions.register\_interaction\_net()}}.

\item {} 
an \sphinxstyleemphasis{interaction} defines how the net actually interacts.  See
\sphinxcode{\sphinxupquote{muse.interactions.register\_agent\_interaction()}}.

\end{itemize}

In practice, we always consider sequences of nets (i.e. more than one net) that
interact using the same interaction function.

Hence, the input looks something like the following:

\begin{sphinxVerbatim}[commandchars=\\\{\}]
\PYG{k}{[[sectors.commercial.interactions]]}
\PYG{n}{net} \PYG{o}{=} \PYG{l+s}{\PYGZsq{}new\PYGZus{}to\PYGZus{}retro\PYGZsq{}}
\PYG{n}{interaction} \PYG{o}{=} \PYG{l+s}{\PYGZsq{}transfer\PYGZsq{}}
\end{sphinxVerbatim}

“new\_to\_retro” is a function that figures out all “new/retro” pairs of agents.
Whereas “transfer” is a function that performs the transfer of assets and
information between each pair.

Furthermore, it is possible to pass parameters to either the net of the interaction
as follows:

\begin{sphinxVerbatim}[commandchars=\\\{\}]
\PYG{k}{[[sectors.commercial.interactions]]}
\PYG{n}{net} \PYG{o}{=} \PYG{p}{\PYGZob{}}\PYG{l+s}{\PYGZdq{}name\PYGZdq{}}\PYG{p}{:} \PYG{l+s}{\PYGZdq{}some\PYGZus{}net\PYGZdq{}}\PYG{p}{,} \PYG{l+s}{\PYGZdq{}param\PYGZdq{}}\PYG{p}{:} \PYG{l+s}{\PYGZdq{}some value\PYGZdq{}}\PYG{p}{\PYGZcb{}}
\PYG{n}{interaction} \PYG{o}{=} \PYG{p}{\PYGZob{}}\PYG{l+s}{\PYGZdq{}name\PYGZdq{}}\PYG{p}{:} \PYG{l+s}{\PYGZdq{}some\PYGZus{}interaction\PYGZdq{}}\PYG{p}{,} \PYG{l+s}{\PYGZdq{}param\PYGZdq{}}\PYG{p}{:} \PYG{l+s}{\PYGZdq{}some other value\PYGZdq{}}\PYG{p}{\PYGZcb{}}
\end{sphinxVerbatim}

The parameters will depend on the net and interaction functions. Neither
“new\_to\_retro” nor “transfer” take any arguments at this point. MUSE interaction
facilities are defined in \sphinxcode{\sphinxupquote{muse.interactions}}.

\item[{output}] \leavevmode
Outputs are made up of several components. MUSE is designed to allow users to
mix\sphinxhyphen{}and\sphinxhyphen{}match both how and what to save.

\sphinxstyleemphasis{output} is specified as a TOML array, e.g. with double brackets. Each sector can
specify an arbitrary number of outputs, simply by adding an extra output row.

A single row looks like this:

\begin{sphinxVerbatim}[commandchars=\\\{\}]
\PYG{k}{[[sectors.commercial.outputs]]}
\PYG{n}{filename} \PYG{o}{=} \PYG{l+s}{\PYGZsq{}\PYGZob{}cwd\PYGZcb{}/Results/\PYGZob{}Sector\PYGZcb{}/\PYGZob{}Quantity\PYGZcb{}/\PYGZob{}year\PYGZcb{}\PYGZob{}suffix\PYGZcb{}\PYGZsq{}}
\PYG{n}{quantity} \PYG{o}{=} \PYG{l+s}{\PYGZdq{}capacity\PYGZdq{}}
\PYG{n}{sink} \PYG{o}{=} \PYG{l+s}{\PYGZsq{}csv\PYGZsq{}}
\PYG{n}{overwrite} \PYG{o}{=} \PYG{k+kc}{true}
\end{sphinxVerbatim}

The following attributes are available:
\begin{itemize}
\item {} \begin{description}
\item[{quantity: Name of the quantity to save. Currently, only \sphinxtitleref{capacity} exists,}] \leavevmode
refering to \sphinxcode{\sphinxupquote{muse.outputs.capacity()}}. However, users can
customize and create further output quantities by registering with MUSE via
\sphinxcode{\sphinxupquote{muse.outputs.register\_output\_quantity()}}. See
\sphinxcode{\sphinxupquote{muse.outputs}} for more details.

\end{description}

\item {} \begin{description}
\item[{sink: the sink is the place (disk, cloud, database, etc…) and format with which}] \leavevmode
the computed quantity is saved. Currently only sinks that save to files are
implemented. The filename can specified via \sphinxtitleref{filename}, as given below. The
following sinks are available: “csv”, “netcfd”, “excel”. However, more sinks can
be added by interested users, and registered with MUSE via
\sphinxcode{\sphinxupquote{muse.outputs.register\_output\_sink()}}. See
\sphinxcode{\sphinxupquote{muse.outputs}} for more details.

\end{description}

\item {} \begin{description}
\item[{filename: defines the format of the file where to save the data. There are several}] \leavevmode
standard values that are automatically substituted:
\begin{itemize}
\item {} 
cwd: current working directory, where MUSE was started

\item {} 
path: directory where the TOML file resides

\item {} 
sector: name of the current sector (.e.g. “commercial” above)

\item {} 
Sector: capitalized name of the current sector

\item {} 
quantity: name of the quantity to save (as given by the quantity attribute)

\item {} 
Quantity: capitablized name of the quantity to save

\item {} 
year: current year

\item {} 
suffix: standard suffix/file extension of the sink

\end{itemize}

Defaults to \sphinxtitleref{\{cwd\}/\{default\_output\_dir\}/\{Sector\}/\{Quantity\}/\{year\}\{suffix\}}.

\end{description}

\item {} \begin{description}
\item[{overwrite: If \sphinxtitleref{False} MUSE will issue an error and abort, instead of}] \leavevmode
overwriting an existing file. Defaults to \sphinxtitleref{False}. This prevents important output files from being overwritten.

\end{description}

\end{itemize}

\item[{technodata}] \leavevmode
Path to a csv file containing the characterization of the technologies involved in
the sector, e.g. lifetime, capital costs, etc… See {\hyperref[\detokenize{inputs/technodata:inputs-technodata}]{\sphinxcrossref{\DUrole{std,std-ref}{Techno\sphinxhyphen{}data}}}}.

\item[{timeslice\_levels}] \leavevmode
Slices to consider in a level. If absent, defaults to the finest timeslices.  See
{\hyperref[\detokenize{inputs/toml:timeslices}]{\sphinxcrossref{Timeslices}}}

\item[{commodities\_in}] \leavevmode
Path to a csv file describing the inputs of each technology involved in the sector.
See {\hyperref[\detokenize{inputs/commodities_io:inputs-iocomms}]{\sphinxcrossref{\DUrole{std,std-ref}{General features}}}}.

\item[{commodities\_out}] \leavevmode
Path to a csv file describing the outputs of each technology involved in the sector.
See {\hyperref[\detokenize{inputs/commodities_io:inputs-ocomms}]{\sphinxcrossref{\DUrole{std,std-ref}{Output Commodities}}}}.

\item[{existing\_capacity}] \leavevmode
Path to a csv file describing the initial capacity of the sector.
See {\hyperref[\detokenize{inputs/existing_capacity:inputs-existing-capacity}]{\sphinxcrossref{\DUrole{std,std-ref}{Existing Sectoral Capacity}}}}.

\item[{agents}] \leavevmode
Path to a csv file describing the agents in the sector.
See {\hyperref[\detokenize{inputs/agents:inputs-agents}]{\sphinxcrossref{\DUrole{std,std-ref}{Agents}}}}.

\end{description}


\subsection{Preset sectors}
\label{\detokenize{inputs/toml:preset-sectors}}
The commodity production, commodity consumption and product prices of preset sectors are determined
exogeneously. They are know from the start of the simulation and are not affected by the
simulation.

Preset sectors are defined in \sphinxcode{\sphinxupquote{PresetSector}}.

The three components, production, consumption, and prices, can be set independantly and
not all three need to be set. Production and consumption default to zero, and prices
default to leaving things unchanged.

The following defines a standard preset sector where consumption is defined as a
function of macro\sphinxhyphen{}economic data, i.e. population and gdp.

\begin{sphinxVerbatim}[commandchars=\\\{\}]
\PYG{k}{[sectors.commercial\PYGZus{}presets]}
\PYG{n}{type} \PYG{o}{=} \PYG{l+s}{\PYGZsq{}presets\PYGZsq{}}
\PYG{n}{priority} \PYG{o}{=} \PYG{l+s}{\PYGZsq{}presets\PYGZsq{}}
\PYG{n}{timeslice\PYGZus{}shares\PYGZus{}path} \PYG{o}{=} \PYG{l+s}{\PYGZsq{}\PYGZob{}path\PYGZcb{}/technodata/TimesliceShareCommercial.csv\PYGZsq{}}
\PYG{n}{macrodrivers\PYGZus{}path} \PYG{o}{=} \PYG{l+s}{\PYGZsq{}\PYGZob{}path\PYGZcb{}/technodata/Macrodrivers.csv\PYGZsq{}}
\PYG{n}{regression\PYGZus{}path} \PYG{o}{=} \PYG{l+s}{\PYGZsq{}\PYGZob{}path\PYGZcb{}/technodata/regressionparameters.csv\PYGZsq{}}
\PYG{n}{timeslices\PYGZus{}levels} \PYG{o}{=} \PYG{p}{\PYGZob{}}\PYG{l+s}{\PYGZsq{}day\PYGZsq{}}\PYG{p}{:} \PYG{p}{[}\PYG{l+s}{\PYGZsq{}all\PYGZhy{}day\PYGZsq{}}\PYG{p}{]}\PYG{p}{\PYGZcb{}}
\PYG{n}{forecast} \PYG{o}{=} \PYG{k}{[0, 5]}
\end{sphinxVerbatim}

The following attributes are accepted:
\begin{description}
\item[{type:}] \leavevmode
See the attribute in the standard mode, {\hyperref[\detokenize{inputs/toml:sector-type}]{\sphinxcrossref{\DUrole{std,std-ref}{type}}}}. \sphinxstyleemphasis{Preset} sectors
are those with type “presets”.

\item[{priority}] \leavevmode
See the attribute in the standard mode, {\hyperref[\detokenize{inputs/toml:sector-priority}]{\sphinxcrossref{\DUrole{std,std-ref}{priority}}}}.

\item[{timeslices\_levels:}] \leavevmode
See the attribute in the standard mode, {\hyperref[\detokenize{inputs/toml:timeslices}]{\sphinxcrossref{Timeslices}}}.

\end{description}
\phantomsection\label{\detokenize{inputs/toml:preset-consumption}}\begin{description}
\item[{consumption\_path:}] \leavevmode
CSV output files, one per year. This attribute can include wild cards, i.e. ‘*’,
which can match anything. For instance: \sphinxtitleref{consumption\_path = “\{cwd\}/Consumtion*.csv”} will match any csv file starting with “Consumption” in the
current working directory. The file names must include the year for which it defines
the consumption, e.g. \sphinxtitleref{Consumption2015.csv}.

The CSV format should follow the following format:


\begin{savenotes}\sphinxattablestart
\centering
\sphinxcapstartof{table}
\sphinxthecaptionisattop
\sphinxcaption{Consumption}\label{\detokenize{inputs/toml:id2}}
\sphinxaftertopcaption
\begin{tabulary}{\linewidth}[t]{|T|T|T|T|T|T|T|}
\hline
\sphinxstyletheadfamily &\sphinxstyletheadfamily 
RegionName
&\sphinxstyletheadfamily 
ProcessName
&\sphinxstyletheadfamily 
TimeSlice
&\sphinxstyletheadfamily 
electricity
&\sphinxstyletheadfamily 
diesel
&\sphinxstyletheadfamily 
algae
\\
\hline\sphinxstyletheadfamily 
0
&\sphinxstyletheadfamily 
USA
&\sphinxstyletheadfamily 
fluorescent light
&\sphinxstyletheadfamily 
1
&
1.9
&
0
&
0
\\
\hline\sphinxstyletheadfamily 
1
&\sphinxstyletheadfamily 
USA
&\sphinxstyletheadfamily 
fluorescent light
&\sphinxstyletheadfamily 
2
&
1.8
&
0
&
0
\\
\hline
\end{tabulary}
\par
\sphinxattableend\end{savenotes}

The index column as well as “RegionName”, “ProcessName”, and “TimeSlice” must be
present. Further columns are reserved for commodities. “TimeSlice” refers to the
index of the timeslice.

\item[{supply\_path:}] \leavevmode
CSV file, one per year, indicating the amount of a commodities produced. It follows
the same format as {\hyperref[\detokenize{inputs/toml:preset-consumption}]{\sphinxcrossref{\DUrole{std,std-ref}{consumption\_path}}}}.

\item[{supply\_path:}] \leavevmode
CSV file, one per year, indicating the amount of a commodities produced. It follows
the same format as {\hyperref[\detokenize{inputs/toml:preset-consumption}]{\sphinxcrossref{\DUrole{std,std-ref}{consumption\_path}}}}.

\item[{prices\_path:}] \leavevmode
CSV file indicating the amount of a commodities produced. The format of the CSV files
follows that of {\hyperref[\detokenize{inputs/projections:inputs-projection}]{\sphinxcrossref{\DUrole{std,std-ref}{Initial Market Projection}}}}.

\end{description}
\phantomsection\label{\detokenize{inputs/toml:preset-demand}}\begin{description}
\item[{demand\_path:}] \leavevmode
Incompatible with {\hyperref[\detokenize{inputs/toml:preset-consumption}]{\sphinxcrossref{\DUrole{std,std-ref}{consumption\_path}}}} or
{\hyperref[\detokenize{inputs/toml:preset-macro}]{\sphinxcrossref{\DUrole{std,std-ref}{macrodrivers\_path}}}}. A CSV file containing the consumption in the
same format as {\hyperref[\detokenize{inputs/projections:inputs-projection}]{\sphinxcrossref{\DUrole{std,std-ref}{Initial Market Projection}}}}.

\end{description}
\phantomsection\label{\detokenize{inputs/toml:preset-macro}}\begin{description}
\item[{macrodrivers\_path:}] \leavevmode
Incompatible with {\hyperref[\detokenize{inputs/toml:preset-consumption}]{\sphinxcrossref{\DUrole{std,std-ref}{consumption\_path}}}} or
{\hyperref[\detokenize{inputs/toml:preset-demand}]{\sphinxcrossref{\DUrole{std,std-ref}{demand\_path}}}}. Path to a CSV file giving the profile of the
macrodrivers. Also requires {\hyperref[\detokenize{inputs/toml:preset-regression}]{\sphinxcrossref{\DUrole{std,std-ref}{regression\_path}}}}.

\end{description}
\phantomsection\label{\detokenize{inputs/toml:preset-regression}}\begin{description}
\item[{regression\_path:}] \leavevmode
Incompatible with {\hyperref[\detokenize{inputs/toml:preset-consumption}]{\sphinxcrossref{\DUrole{std,std-ref}{consumption\_path}}}} or
{\hyperref[\detokenize{inputs/toml:preset-demand}]{\sphinxcrossref{\DUrole{std,std-ref}{demand\_path}}}}. Path to a CSV file giving the regression
parameters with respect to the macrodrivers.
Also requires {\hyperref[\detokenize{inputs/toml:preset-macro}]{\sphinxcrossref{\DUrole{std,std-ref}{macrodrivers\_path}}}}.

\item[{timeslice\_shares\_path}] \leavevmode
Optional csv file giving shares per timeslice. Requires
{\hyperref[\detokenize{inputs/toml:preset-consumption}]{\sphinxcrossref{\DUrole{std,std-ref}{macrodrivers\_path}}}}.

\item[{filters:}] \leavevmode
Optional dictionary of entries by which to filter the consumption.  Requires
{\hyperref[\detokenize{inputs/toml:preset-consumption}]{\sphinxcrossref{\DUrole{std,std-ref}{macrodrivers\_path}}}}. For instance,

\begin{sphinxVerbatim}[commandchars=\\\{\}]
\PYG{n}{filters}\PYG{o}{.}\PYG{n}{region} \PYG{o}{=} \PYG{p}{[}\PYG{l+s+s2}{\PYGZdq{}}\PYG{l+s+s2}{USA}\PYG{l+s+s2}{\PYGZdq{}}\PYG{p}{,} \PYG{l+s+s2}{\PYGZdq{}}\PYG{l+s+s2}{ASEA}\PYG{l+s+s2}{\PYGZdq{}}\PYG{p}{]}
\PYG{n}{filters}\PYG{o}{.}\PYG{n}{commodity} \PYG{o}{=} \PYG{p}{[}\PYG{l+s+s2}{\PYGZdq{}}\PYG{l+s+s2}{algae}\PYG{l+s+s2}{\PYGZdq{}}\PYG{p}{,} \PYG{l+s+s2}{\PYGZdq{}}\PYG{l+s+s2}{fluorescent light}\PYG{l+s+s2}{\PYGZdq{}}\PYG{p}{]}
\end{sphinxVerbatim}

\end{description}


\subsection{Legacy Sectors}
\label{\detokenize{inputs/toml:legacy-sectors}}
Legacy sectors wrap sectors developed for a previous version of MUSE to the open\sphinxhyphen{}source
version.

Preset sectors are defined in \sphinxcode{\sphinxupquote{PresetSector}}.

The can be defined in the TOML file as follows:

\begin{sphinxVerbatim}[commandchars=\\\{\}]
\PYG{k}{[global\PYGZus{}input\PYGZus{}files]}
\PYG{n}{macrodrivers} \PYG{o}{=} \PYG{l+s}{\PYGZsq{}\PYGZob{}path\PYGZcb{}/input/Macrodrivers.csv\PYGZsq{}}
\PYG{n}{regions} \PYG{o}{=} \PYG{l+s}{\PYGZsq{}\PYGZob{}path\PYGZcb{}/input/Regions.csv\PYGZsq{}}
\PYG{n}{global\PYGZus{}commodities} \PYG{o}{=} \PYG{l+s}{\PYGZsq{}\PYGZob{}path\PYGZcb{}/input/MUSEGlobalCommodities.csv\PYGZsq{}}

\PYG{k}{[sectors.Industry]}
\PYG{n}{type} \PYG{o}{=} \PYG{l+s}{\PYGZsq{}legacy\PYGZsq{}}
\PYG{n}{priority} \PYG{o}{=} \PYG{l+s}{\PYGZsq{}demand\PYGZsq{}}
\PYG{n}{agregation\PYGZus{}level} \PYG{o}{=} \PYG{l+s}{\PYGZsq{}month\PYGZsq{}}
\PYG{n}{excess} \PYG{o}{=} \PYG{l+m+mi}{0}

\PYG{n}{userdata\PYGZus{}path} \PYG{o}{=} \PYG{l+s}{\PYGZsq{}\PYGZob{}muse\PYGZus{}sectors\PYGZcb{}/Industry\PYGZsq{}}
\PYG{n}{technodata\PYGZus{}path} \PYG{o}{=} \PYG{l+s}{\PYGZsq{}\PYGZob{}muse\PYGZus{}sectors\PYGZcb{}/Industry\PYGZsq{}}
\PYG{n}{timeslices\PYGZus{}path} \PYG{o}{=} \PYG{l+s}{\PYGZsq{}\PYGZob{}muse\PYGZus{}sectors\PYGZcb{}/Industry/TimeslicesIndustry.csv\PYGZsq{}}
\PYG{n}{output\PYGZus{}path} \PYG{o}{=} \PYG{l+s}{\PYGZsq{}\PYGZob{}path\PYGZcb{}/output\PYGZsq{}}
\end{sphinxVerbatim}

For historical reasons, the three \sphinxtitleref{global\_input\_files} above are required. The sector
itself can use the following attributes.
\begin{description}
\item[{type:}] \leavevmode
See the attribute in the standard mode, {\hyperref[\detokenize{inputs/toml:sector-type}]{\sphinxcrossref{\DUrole{std,std-ref}{type}}}}. \sphinxstyleemphasis{Legacy} sectors
are those with type “legacy”.

\item[{priority}] \leavevmode
See the attribute in the standard mode, {\hyperref[\detokenize{inputs/toml:sector-priority}]{\sphinxcrossref{\DUrole{std,std-ref}{priority}}}}.

\item[{agregation\_level:}] \leavevmode
Information relevant to the sector’s timeslice.

\item[{excess:}] \leavevmode
Excess factor used to model early obsolescence.

\item[{timeslices\_path:}] \leavevmode
Path to a timeslice  {\hyperref[\detokenize{inputs/timeslices:inputs-legacy-timeslices}]{\sphinxcrossref{\DUrole{std,std-ref}{time\_slices}}}}.

\item[{userdata\_path:}] \leavevmode
Path to a directory with sector\sphinxhyphen{}specific data files.

\item[{technodata\_path:}] \leavevmode
Path to a technodata CSV file. See. {\hyperref[\detokenize{inputs/technodata:inputs-technodata}]{\sphinxcrossref{\DUrole{std,std-ref}{Techno\sphinxhyphen{}data}}}}.

\item[{output\_path:}] \leavevmode
Path to a diretory where the sector will write output files.

\end{description}


\section{Input Files}
\label{\detokenize{inputs/inputs_csv:input-files}}\label{\detokenize{inputs/inputs_csv::doc}}

\subsection{Initial Market Projection}
\label{\detokenize{inputs/projections:initial-market-projection}}\label{\detokenize{inputs/projections:inputs-projection}}\label{\detokenize{inputs/projections::doc}}
MUSE needs an initial projection of the market prices for each period of the simulation.
\begin{itemize}
\item {} 
The price trajectory is needed if the MCA works in \sphinxstyleemphasis{equilibrium} mode as an initial
trajectory for the base year of the simulation. The market will override the
calculated prices obtained from each commodity equilibrium for all the future periods
following the base year

\item {} 
Similarly, if the market works in a \sphinxstyleemphasis{carbon budget} mode, the prices are used as a
starting point. The only difference from the previous case is given by the fact that
the MCA will be calculating an additional global market price for carbon dioxide (and
additional pollutants if required)

\item {} 
If the MCA works in an \sphinxstyleemphasis{exogenous} mode, it will use the initial market projection as
the projection for the the base year and all the future periods of the simulation

\end{itemize}

The forward price trajectory should follow the structure reported in the table below.


\begin{savenotes}\sphinxattablestart
\centering
\sphinxcapstartof{table}
\sphinxthecaptionisattop
\sphinxcaption{Initial market projections}\label{\detokenize{inputs/projections:id1}}
\sphinxaftertopcaption
\begin{tabular}[t]{|*{6}{\X{1}{6}|}}
\hline
\sphinxstyletheadfamily 
RegionName
&\sphinxstyletheadfamily 
Attribute
&\sphinxstyletheadfamily 
Time
&\sphinxstyletheadfamily 
com1
&\sphinxstyletheadfamily 
com2
&\sphinxstyletheadfamily 
com3
\\
\hline
Unit
&\begin{itemize}
\item {} 
\end{itemize}
&
Year
&
MUS\$2010/PJ
&
MUS\$2010/PJ
&
MUS\$2010/PJ
\\
\hline
region1
&
CommodityPrice
&
2010
&
20
&
1.9583
&
2
\\
\hline
region1
&
CommodityPrice
&
2015
&
20
&
1.9583
&
2
\\
\hline
region1
&
CommodityPrice
&
2020
&
20.38518042
&
1.996014941
&
2.038518042
\\
\hline
region1
&
CommodityPrice
&
2025
&
20.77777903
&
2.034456234
&
2.077777903
\\
\hline
region1
&
CommodityPrice
&
2030
&
21.17793872
&
2.073637869
&
2.117793872
\\
\hline
region1
&
CommodityPrice
&
2035
&
21.58580508
&
2.113574105
&
2.158580508
\\
\hline
region1
&
CommodityPrice
&
2040
&
22.00152655
&
2.154279472
&
2.200152655
\\
\hline
region1
&
CommodityPrice
&
2045
&
22.42525441
&
2.195768786
&
2.242525441
\\
\hline
region1
&
CommodityPrice
&
2050
&
22.85714286
&
2.238057143
&
2.285714286
\\
\hline
\end{tabular}
\par
\sphinxattableend\end{savenotes}
\begin{description}
\item[{RegionName}] \leavevmode
represents the region ID and needs to be consistent across all the data inputs

\item[{Attribute}] \leavevmode
defines the attribute type. In this case it refers to the CommodityPrice; it is
relevant only for internal use

\item[{Time}] \leavevmode
corresponds to the time periods of the simulation; the simulated time framework in
the example goes from 2010 through to 2050 with a 5\sphinxhyphen{}year time step

\item[{com1, …, comN}] \leavevmode
Any further columns represent the commodities modelled, as defined in the global
commodities the row Unit reports the unit in which the technology consumption is
defined; it is for the user internal reference only. The names \sphinxstyleemphasis{comX} should be
replaced with the names of the commodities.

\end{description}


\subsection{Regional data}
\label{\detokenize{inputs/regions:regional-data}}\label{\detokenize{inputs/regions:id1}}\label{\detokenize{inputs/regions::doc}}
MUSE requires the definition of the methodology used for investment and dispatch and alias
demand matching. The methodology has to be defined by region and subregion, meant as a
geographical subdivision in a region. Currently, the methodology definition is
important for the legacy sectors only.

Below the generic structure of the input commodity file for the electric
heater is shown:


\begin{savenotes}\sphinxattablestart
\centering
\sphinxcapstartof{table}
\sphinxthecaptionisattop
\sphinxcaption{Methodology used in investment and demand matching}\label{\detokenize{inputs/regions:id2}}
\sphinxaftertopcaption
\begin{tabulary}{\linewidth}[t]{|T|T|T|T|T|}
\hline
\sphinxstyletheadfamily 
SectorName
&\sphinxstyletheadfamily 
RegionName
&\sphinxstyletheadfamily 
Subregion
&\sphinxstyletheadfamily 
sMethodologyPlanning
&\sphinxstyletheadfamily 
sMethodologyDispatch
\\
\hline
Agriculture
&
region1
&
region1
&
NPV
&
DCF
\\
\hline
Bioenergy
&
region1
&
region1
&
NPV
&
DCF
\\
\hline
Industry
&
region1
&
region1
&
NPV
&
DCF
\\
\hline
Residential
&
region1
&
region1
&
EAC
&
EAC
\\
\hline
Commercial
&
region1
&
region1
&
EAC
&
EAC
\\
\hline
Transport
&
region1
&
region1
&
LCOE
&
LCOE
\\
\hline
Power
&
region1
&
region1
&
LCOE
&
LCOE
\\
\hline
Refinery
&
region1
&
region1
&
LCOE
&
LCOE
\\
\hline
Supply
&
region1
&
region1
&
LCOE
&
LCOE
\\
\hline
\end{tabulary}
\par
\sphinxattableend\end{savenotes}
\begin{description}
\item[{SectorName}] \leavevmode
represents the sector\_ID and needs to be consistent across the data input files

\item[{RegionName}] \leavevmode
represents the region ID and needs to be consistent across all the data inputs

\item[{Subregion}] \leavevmode
represents the subregion ID and needs to be consistent across all the data inputs

\item[{sMethodologyPlanning}] \leavevmode
reports the cost quantity used for making investments in new technologies in each
sector (e.g. NPV stands for net present value, EAC stands for equivalent annual
costs, LCOE stands for levelised cost of energy)

\item[{sMethodologyDispatch}] \leavevmode
reports the cost quantity used for the demand matching using existing technologies in
each sector (e.g. DCF stands for discounted cash flow, EAC stands for equivalent
annual cost, LCOE stands for levelised cost of energy)

\end{description}


\subsection{Commodity Description}
\label{\detokenize{inputs/commodities:commodity-description}}\label{\detokenize{inputs/commodities:inputs-commodities}}\label{\detokenize{inputs/commodities::doc}}
MUSE handles a configurable number and type of commodities which are primarily used to
represent energy, services, pollutants/emissions. The commodities for the simulation as
a whole are defined in a csv file with the following structure.


\begin{savenotes}\sphinxattablestart
\centering
\sphinxcapstartof{table}
\sphinxthecaptionisattop
\sphinxcaption{Global commodities}\label{\detokenize{inputs/commodities:id1}}
\sphinxaftertopcaption
\begin{tabulary}{\linewidth}[t]{|T|T|T|T|T|T|}
\hline
\sphinxstyletheadfamily 
Commodity
&\sphinxstyletheadfamily 
CommodityType
&\sphinxstyletheadfamily 
CommodityName
&\sphinxstyletheadfamily 
CommodityEmissionFactor\_CO2
&\sphinxstyletheadfamily 
HeatRate
&\sphinxstyletheadfamily 
Unit
\\
\hline
Coal
&
Energy
&
hardcoal
&
94.6
&
29
&
PJ
\\
\hline
Agricultural\sphinxhyphen{}residues
&
Energy
&
agrires
&
112
&
15.4
&
PJ
\\
\hline
\end{tabulary}
\par
\sphinxattableend\end{savenotes}
\begin{description}
\item[{Commodity}] \leavevmode
represents the extended name of a commodity

\item[{CommodityType}] \leavevmode
defines the type of a commodity (i.e. energy, material or environmental)

\item[{CommodityName}] \leavevmode
is the internal name used for a commodity inside the model.

\item[{CommodityEmissionFactor\_CO2}] \leavevmode
is CO2 emission per unit of commodity flow

\item[{HeatRate}] \leavevmode
represents the lower heating value of an energy commodity

\item[{Unit}] \leavevmode
is the unit used as a basis for all the input data. More specifically the model allows
a totally flexible way of defining the commodities. CommodityName is currently the
only column used internally as it defines the names of commodities and needs to be
used consistently across all the input data files. The remaining columns of the file
are only relevant for the user internal reference for the original sets of
assumptions used.

\end{description}


\subsection{Techno\sphinxhyphen{}data}
\label{\detokenize{inputs/technodata:techno-data}}\label{\detokenize{inputs/technodata:inputs-technodata}}\label{\detokenize{inputs/technodata::doc}}
The techno\sphinxhyphen{}data includes the techno\sphinxhyphen{}economic characteristics of each technology such
as capital, fixed and variable cost, lifetime, utilisation factor.
The techno\sphinxhyphen{}data should follow the structure reported in the table. The column order
is not important and additional input data can alsobe read in this format. In the table,
the electric boiler used in households is taken as an example for a generic region, region1.


\begin{savenotes}\sphinxattablestart
\centering
\sphinxcapstartof{table}
\sphinxthecaptionisattop
\sphinxcaption{Techno\sphinxhyphen{}data}\label{\detokenize{inputs/technodata:id1}}
\sphinxaftertopcaption
\begin{tabulary}{\linewidth}[t]{|T|T|T|T|T|T|T|T|}
\hline
\sphinxstyletheadfamily 
ProcessName
&\sphinxstyletheadfamily 
RegionName
&\sphinxstyletheadfamily 
Time
&\sphinxstyletheadfamily 
Level
&\sphinxstyletheadfamily 
cap\_par
&\sphinxstyletheadfamily 
cap\_exp
&\sphinxstyletheadfamily 
fix\_par
&\sphinxstyletheadfamily 
…
\\
\hline
resBoilerElectric
&
region1
&
2010
&
fixed
&
3.81
&
1.00
&
0.38
&
…
\\
\hline
resBoilerElectric
&
region1
&
2030
&
fixed
&
3.81
&
1.00
&
0.38
&
…
\\
\hline
\end{tabulary}
\par
\sphinxattableend\end{savenotes}
\begin{description}
\item[{ProcessName}] \leavevmode
represents the technology ID and needs to be consistent across all the data inputs

\item[{RegionName}] \leavevmode
represents the region ID and needs to be consistent across all the data inputs

\item[{Time}] \leavevmode
represents the period of the simulation to which the value applies; it needs to
contain at least the base year of the simulation

\item[{Level}] \leavevmode
characterises either a fixed or a flexible input type

\item[{cap\_par, cap\_exp}] \leavevmode
are used in the capital cost estimation. Capital costs are calculated as:
\begin{equation*}
\begin{split}\text{CAPEX} = \text{cap\_par} * \text{(Capacity)}^\text{cap\_exp}\end{split}
\end{equation*}
where the parameter cap\_par is estimated at a selected reference size (i.e. Capref),
such as:
\begin{equation*}
\begin{split}\text{cap\_par} = \left(
   \frac{\text{CAPEXref}}{\text{Capref}}
\right)^{\text{cap\_exp}}\end{split}
\end{equation*}
Capref is decided by the modeller before filling the input data files.

This allows the model to take into account economies of scale. ie. As \sphinxtitleref{Capacity} increases, the price of the technology decreases.

\end{description}

fix\_par, fix\_exp
\begin{quote}

are used in the fixed cost estimation. Fixed costs are calculated as:
\begin{equation*}
\begin{split}\text{FOM} = \text{fix\_par} * (\text{Capacity})^\text{fix\_exp}\end{split}
\end{equation*}
where the parameter fix\_par is estimated at a selected reference size (i.e. Capref),
such as:
\begin{equation*}
\begin{split}\text{fix\_par} = \left(
   \frac{\text{FOMref}}{\text{Capref}}
\right)^{\text{fix\_exp}}\end{split}
\end{equation*}
Capref is decided by the modeller before filling the input data files.
\end{quote}
\begin{description}
\item[{var\_par, var\_exp}] \leavevmode
are used in the variable costs estimation. These variable costs are capacity
dependent Variable costs are calculated as:
\begin{equation*}
\begin{split}\text{VAREX} = \text{cap\_par} * \text{(Capacity)}^{\text{cap\_exp}}\end{split}
\end{equation*}
where the parameter var\_par is estimated at a selected reference size (i.e. Capref),
such as:
\begin{equation*}
\begin{split}\text{var\_par} = \left(
   \frac{\text{VARref}}{\text{Capref}}
\right)^{\text{var\_exp}}\end{split}
\end{equation*}
Capref is decided by the modeller before filling the input data files.

\item[{MaxCapacityAddition}] \leavevmode
represents the maximum addition of installed capacity per technology, region, year.

\item[{MaxCapacityGrowth}] \leavevmode
represents the maximum growth in capacity as a fraction of the installed capacity per
technology, region and year.

\item[{TotalCapacityLimit}] \leavevmode
represents the total capacity limit per technology, region and year.

\item[{TechnicalLife}] \leavevmode
represents the number of years that a technology operates before it is decommissioned.

\item[{UtilizationFactor}] \leavevmode
is the number of operating hours of a process over the maximum number of hours in a year.

\item[{ScalingSize}] \leavevmode
represents the minimum size of a technology to be installed.

\item[{efficiency}] \leavevmode
is calculated as the ratio between the total output commodities and the input commodities.

\item[{AvailabiliyYear}] \leavevmode
defines the starting year of a technology; for example the value equals 1 when a
technology would be available or 0 when a technology would not be available.

\item[{Type}] \leavevmode
defines the type of a technology.

\item[{Fuel}] \leavevmode
defines the fuel used by a technology.

\item[{EndUse}] \leavevmode
defines the end use of a technology.

\item[{InterestRate}] \leavevmode
is the technology interest rate.

\item[{Agent\_0, …, Agent\_N}] \leavevmode
represent the allocation of the initial capacity to the each agent.

\end{description}

The input data has to be provided for the base year. Additional years within the time
framework of the overall simulation can be defined. In this case, MUSE would interpolate
the values between the provided periods and assume a constant value afterwards.


\subsection{Time\sphinxhyphen{}slices}
\label{\detokenize{inputs/timeslices:time-slices}}\label{\detokenize{inputs/timeslices:inputs-legacy-timeslices}}\label{\detokenize{inputs/timeslices::doc}}
\begin{sphinxadmonition}{note}{Note:}
This input file is only for legacy sectors. For anything else, please see {\hyperref[\detokenize{inputs/toml:simulation-settings}]{\sphinxcrossref{\DUrole{std,std-ref}{Simulation settings}}}}.
\end{sphinxadmonition}

Time\sphinxhyphen{}slices represent a sub\sphinxhyphen{}year disaggregation of commodity demand. They are fully
flexible in number and names as to serve the specific representation of the commodity
demand, supply, and supply cost profile in each energy sector.  Each time slice is
independent in terms of the number of represent hours, as long as it is meaningful for the
users and their data inputs. 1 is the minimum number of time\sphinxhyphen{}slice as this would
correspond to a full year.  The time\sphinxhyphen{}slice definition of a sector affects the commodity
price profile and the supply cost profile.

The csv file for the time\sphinxhyphen{}slice definition would report the length (in hours) of each
time slice as characteristic to the selected sector to represent diurnal, weekly and
seasonal variation of energy commodities, demand and supply, as shown in the table for
30 time\sphinxhyphen{}slices.


\begin{savenotes}\sphinxatlongtablestart\begin{longtable}[c]{|l|l|l|l|l|l|}
\sphinxthelongtablecaptionisattop
\caption{Time\sphinxhyphen{}slices\strut}\label{\detokenize{inputs/timeslices:id1}}\\*[\sphinxlongtablecapskipadjust]
\hline
\sphinxstyletheadfamily 
AgLevel
&\sphinxstyletheadfamily 
SN
&\sphinxstyletheadfamily 
Month
&\sphinxstyletheadfamily 
Day
&\sphinxstyletheadfamily 
Hour
&\sphinxstyletheadfamily 
RepresentHours
\\
\hline
\endfirsthead

\multicolumn{6}{c}%
{\makebox[0pt]{\sphinxtablecontinued{\tablename\ \thetable{} \textendash{} continued from previous page}}}\\
\hline
\sphinxstyletheadfamily 
AgLevel
&\sphinxstyletheadfamily 
SN
&\sphinxstyletheadfamily 
Month
&\sphinxstyletheadfamily 
Day
&\sphinxstyletheadfamily 
Hour
&\sphinxstyletheadfamily 
RepresentHours
\\
\hline
\endhead

\hline
\multicolumn{6}{r}{\makebox[0pt][r]{\sphinxtablecontinued{continues on next page}}}\\
\endfoot

\endlastfoot
\sphinxstyletheadfamily 
Hour
&\sphinxstyletheadfamily 
1
&\sphinxstyletheadfamily 
Winter
&\sphinxstyletheadfamily 
Weekday
&
Night
&
396
\\
\hline\sphinxstyletheadfamily 
Hour
&\sphinxstyletheadfamily 
2
&\sphinxstyletheadfamily 
Winter
&\sphinxstyletheadfamily 
Weekday
&
Morning
&
396
\\
\hline\sphinxstyletheadfamily 
Hour
&\sphinxstyletheadfamily 
3
&\sphinxstyletheadfamily 
Winter
&\sphinxstyletheadfamily 
Weekday
&
Afternoon
&
264
\\
\hline\sphinxstyletheadfamily 
Hour
&\sphinxstyletheadfamily 
4
&\sphinxstyletheadfamily 
Winter
&\sphinxstyletheadfamily 
Weekday
&
EarlyPeak
&
66
\\
\hline\sphinxstyletheadfamily 
Hour
&\sphinxstyletheadfamily 
5
&\sphinxstyletheadfamily 
Winter
&\sphinxstyletheadfamily 
Weekday
&
LatePeak
&
66
\\
\hline\sphinxstyletheadfamily 
Hour
&\sphinxstyletheadfamily 
6
&\sphinxstyletheadfamily 
Winter
&\sphinxstyletheadfamily 
Weekday
&
Evening
&
396
\\
\hline\sphinxstyletheadfamily 
Hour
&\sphinxstyletheadfamily 
7
&\sphinxstyletheadfamily 
Winter
&\sphinxstyletheadfamily 
Weekend
&
Night
&
156
\\
\hline\sphinxstyletheadfamily 
Hour
&\sphinxstyletheadfamily 
8
&\sphinxstyletheadfamily 
Winter
&\sphinxstyletheadfamily 
Weekend
&
Morning
&
156
\\
\hline\sphinxstyletheadfamily 
Hour
&\sphinxstyletheadfamily 
9
&\sphinxstyletheadfamily 
Winter
&\sphinxstyletheadfamily 
Weekend
&
Afternoon
&
156
\\
\hline\sphinxstyletheadfamily 
Hour
&\sphinxstyletheadfamily 
10
&\sphinxstyletheadfamily 
Winter
&\sphinxstyletheadfamily 
Weekend
&
Evening
&
156
\\
\hline\sphinxstyletheadfamily 
Hour
&\sphinxstyletheadfamily 
11
&\sphinxstyletheadfamily 
SpringAutumn
&\sphinxstyletheadfamily 
Weekday
&
Night
&
792
\\
\hline\sphinxstyletheadfamily 
Hour
&\sphinxstyletheadfamily 
12
&\sphinxstyletheadfamily 
SpringAutumn
&\sphinxstyletheadfamily 
Weekday
&
Morning
&
792
\\
\hline\sphinxstyletheadfamily 
Hour
&\sphinxstyletheadfamily 
13
&\sphinxstyletheadfamily 
SpringAutumn
&\sphinxstyletheadfamily 
Weekday
&
Afternoon
&
528
\\
\hline\sphinxstyletheadfamily 
Hour
&\sphinxstyletheadfamily 
14
&\sphinxstyletheadfamily 
SpringAutumn
&\sphinxstyletheadfamily 
Weekday
&
EarlyPeak
&
132
\\
\hline\sphinxstyletheadfamily 
Hour
&\sphinxstyletheadfamily 
15
&\sphinxstyletheadfamily 
SpringAutumn
&\sphinxstyletheadfamily 
Weekday
&
LatePeak
&
132
\\
\hline\sphinxstyletheadfamily 
Hour
&\sphinxstyletheadfamily 
16
&\sphinxstyletheadfamily 
SpringAutumn
&\sphinxstyletheadfamily 
Weekday
&
Evening
&
792
\\
\hline\sphinxstyletheadfamily 
Hour
&\sphinxstyletheadfamily 
17
&\sphinxstyletheadfamily 
SpringAutumn
&\sphinxstyletheadfamily 
Weekend
&
Night
&
300
\\
\hline\sphinxstyletheadfamily 
Hour
&\sphinxstyletheadfamily 
18
&\sphinxstyletheadfamily 
SpringAutumn
&\sphinxstyletheadfamily 
Weekend
&
Morning
&
300
\\
\hline\sphinxstyletheadfamily 
Hour
&\sphinxstyletheadfamily 
19
&\sphinxstyletheadfamily 
SpringAutumn
&\sphinxstyletheadfamily 
Weekend
&
Afternoon
&
300
\\
\hline\sphinxstyletheadfamily 
Hour
&\sphinxstyletheadfamily 
20
&\sphinxstyletheadfamily 
SpringAutumn
&\sphinxstyletheadfamily 
Weekend
&
Evening
&
300
\\
\hline\sphinxstyletheadfamily 
Hour
&\sphinxstyletheadfamily 
21
&\sphinxstyletheadfamily 
Summer
&\sphinxstyletheadfamily 
Weekday
&
Night
&
396
\\
\hline\sphinxstyletheadfamily 
Hour
&\sphinxstyletheadfamily 
22
&\sphinxstyletheadfamily 
Summer
&\sphinxstyletheadfamily 
Weekday
&
Morning
&
396
\\
\hline\sphinxstyletheadfamily 
Hour
&\sphinxstyletheadfamily 
23
&\sphinxstyletheadfamily 
Summer
&\sphinxstyletheadfamily 
Weekday
&
Afternoon
&
264
\\
\hline\sphinxstyletheadfamily 
Hour
&\sphinxstyletheadfamily 
24
&\sphinxstyletheadfamily 
Summer
&\sphinxstyletheadfamily 
Weekday
&
EarlyPeak
&
66
\\
\hline\sphinxstyletheadfamily 
Hour
&\sphinxstyletheadfamily 
25
&\sphinxstyletheadfamily 
Summer
&\sphinxstyletheadfamily 
Weekday
&
LatePeak
&
66
\\
\hline\sphinxstyletheadfamily 
Hour
&\sphinxstyletheadfamily 
26
&\sphinxstyletheadfamily 
Summer
&\sphinxstyletheadfamily 
Weekday
&
Evening
&
396
\\
\hline\sphinxstyletheadfamily 
Hour
&\sphinxstyletheadfamily 
27
&\sphinxstyletheadfamily 
Summer
&\sphinxstyletheadfamily 
Weekend
&
Night
&
150
\\
\hline\sphinxstyletheadfamily 
Hour
&\sphinxstyletheadfamily 
28
&\sphinxstyletheadfamily 
Summer
&\sphinxstyletheadfamily 
Weekend
&
Morning
&
150
\\
\hline\sphinxstyletheadfamily 
Hour
&\sphinxstyletheadfamily 
29
&\sphinxstyletheadfamily 
Summer
&\sphinxstyletheadfamily 
Weekend
&
Afternoon
&
150
\\
\hline\sphinxstyletheadfamily 
Hour
&\sphinxstyletheadfamily 
30
&\sphinxstyletheadfamily 
Summer
&\sphinxstyletheadfamily 
Weekend
&
Evening
&
150
\\
\hline
\end{longtable}\sphinxatlongtableend\end{savenotes}

It reports the aggregation level of the sector time\sphinxhyphen{}slices (AgLevel), slice number (SN),
seasonal time slices (Month), weekly time slices (Day), hourly profile (Hour), the
amount of hours associated to each time slice (RepresentHours).


\subsection{Input Commodities}
\label{\detokenize{inputs/commodities_io:input-commodities}}\label{\detokenize{inputs/commodities_io:inputs-icomms}}\label{\detokenize{inputs/commodities_io::doc}}
Input commodities are the commodities consumed (also called consumables in MUSE) by each
technology.  They are defined in a csv file which describes the commodity inputs to each
technology, calculated per unit of technology activity. See {\hyperref[\detokenize{inputs/commodities_io:inputs-iocomms}]{\sphinxcrossref{\DUrole{std,std-ref}{below}}}} for a description.


\subsection{Output Commodities}
\label{\detokenize{inputs/commodities_io:output-commodities}}\label{\detokenize{inputs/commodities_io:inputs-ocomms}}
Output commodities are the commodities produced (also called products in MUSE) by each
technology.  They are defined in a csv file which describes the commodity outputs from
each technology, defined per unit of technology activity. Emissions, such as CO2
(produced from fuel combustion and reactions), CH4, N2O, F\sphinxhyphen{}gases, can also be accounted
for in this file. See {\hyperref[\detokenize{inputs/commodities_io:inputs-iocomms}]{\sphinxcrossref{\DUrole{std,std-ref}{below}}}} for a description.


\subsection{General features}
\label{\detokenize{inputs/commodities_io:general-features}}\label{\detokenize{inputs/commodities_io:inputs-iocomms}}
To illustrate the data required for a generic technology in MUSE, the \sphinxstyleemphasis{electric boiler
technology} is used as an example. The commodity flow for the electric boiler, capable
to cover space heating and water heating energy service demands.

\begin{figure}[htbp]
\centering
\capstart

\noindent\sphinxincludegraphics[width=400\sphinxpxdimen]{{commodities_io}.png}
\caption{The table below shows the basic data requirements for a typical technology, the
electric boiler.}\label{\detokenize{inputs/commodities_io:id1}}\end{figure}

\noindent\sphinxincludegraphics[width=400\sphinxpxdimen]{{commodities_io_table}.png}

Below it is shown the generic structure of the input commodity file for the electric
heater.


\begin{savenotes}\sphinxattablestart
\centering
\sphinxcapstartof{table}
\sphinxthecaptionisattop
\sphinxcaption{Commodities used as consumables \sphinxhyphen{} Input commodities}\label{\detokenize{inputs/commodities_io:id2}}
\sphinxaftertopcaption
\begin{tabular}[t]{|*{5}{\X{1}{5}|}}
\hline
\sphinxstyletheadfamily 
ProcessName
&\sphinxstyletheadfamily 
RegionName
&\sphinxstyletheadfamily 
Time
&\sphinxstyletheadfamily 
Level
&\sphinxstyletheadfamily 
electricity
\\
\hline
Unit
&\begin{itemize}
\item {} 
\end{itemize}
&
Year
&\begin{itemize}
\item {} 
\end{itemize}
&
GWh/PJ
\\
\hline
resBoilerElectric
&
region1
&
2010
&
fixed
&
300
\\
\hline
resBoilerElectric
&
region1
&
2030
&
fixed
&
290
\\
\hline
\end{tabular}
\par
\sphinxattableend\end{savenotes}
\begin{description}
\item[{ProcessName}] \leavevmode
represents the technology ID and needs to be consistent across all the data inputs.

\item[{RegionName}] \leavevmode
represents the region ID and needs to be consistent across all the data inputs.

\item[{Time}] \leavevmode
represents the period of the simulation to which the value applies; it needs to
contain at least the base year of the simulation.

\item[{Level}] \leavevmode
characterises either a fixed or a flexible input type the following columns should
contain the list of commodities the row.

\item[{Unit}] \leavevmode
reports the unit in which the technology consumption is defined; it is for the user
internal reference only.

\end{description}

The same structure for the csv file would also apply for the output commodity file. The
input data has to be provided for the base year. Additional years within the time
framework of the overall simulation can be defined. In this case, MUSE would interpolate
the values between the provided periods and assume a constant value afterwards.


\subsection{Existing Sectoral Capacity}
\label{\detokenize{inputs/existing_capacity:existing-sectoral-capacity}}\label{\detokenize{inputs/existing_capacity:inputs-existing-capacity}}\label{\detokenize{inputs/existing_capacity::doc}}
For each technology, the decommissioning profile should be given to MUSE.

The csv file which provides the installed capacity in base year and the decommissioning
profile in the future periods for each technology in a sector, in each region, should
follow the structure reported in the table.


\begin{savenotes}\sphinxattablestart
\centering
\sphinxcapstartof{table}
\sphinxthecaptionisattop
\sphinxcaption{Existing capacity of technologies: the residential boiler example}\label{\detokenize{inputs/existing_capacity:id1}}
\sphinxaftertopcaption
\begin{tabulary}{\linewidth}[t]{|T|T|T|T|T|T|T|T|}
\hline
\sphinxstyletheadfamily 
ProcessName
&\sphinxstyletheadfamily 
RegionName
&\sphinxstyletheadfamily 
Unit
&\sphinxstyletheadfamily 
2010
&\sphinxstyletheadfamily 
2020
&\sphinxstyletheadfamily 
2030
&\sphinxstyletheadfamily 
2040
&\sphinxstyletheadfamily 
2050
\\
\hline
resBoilerElectric
&
region1
&
PJ/y
&
5
&
0.5
&
0
&
0
&
0
\\
\hline
resBoilerElectric
&
region2
&
PJ/y
&
39
&
3.5
&
1
&
0.3
&
0
\\
\hline
\end{tabulary}
\par
\sphinxattableend\end{savenotes}
\begin{description}
\item[{ProcessName}] \leavevmode
represents the technology ID and needs to be consistent across all the data inputs.

\item[{RegionName}] \leavevmode
represents the region ID and needs to be consistent across all the data inputs.

\item[{Unit}] \leavevmode
reports the unit of the technology capacity; it is for the user internal reference only.

\item[{2010,…, 2050}] \leavevmode
represent the simulated periods.

\end{description}


\subsection{Agents}
\label{\detokenize{inputs/agents:agents}}\label{\detokenize{inputs/agents:inputs-agents}}\label{\detokenize{inputs/agents::doc}}
In MUSE, an agent\sphinxhyphen{}based formulation was originally introduced for the residential and
commercial building sectors {\color{red}\bfseries{}:cite:\textasciigrave{}2019:sachs\textasciigrave{}}.  Agents are defined using a CSV file, with
one agent per row, using a somewhat historical format meant specifically for retrofit
and new\sphinxhyphen{}capacity agent pairs. This CSV file can be read using
\sphinxcode{\sphinxupquote{read\_csv\_agent\_parameters()}}. The data is also
interpreted to some degree in the factory functions
\sphinxcode{\sphinxupquote{create\_retrofit\_agent()}} and
\sphinxcode{\sphinxupquote{create\_newcapa\_agent()}}.

For instance, we have the following CSV table:


\begin{savenotes}\sphinxattablestart
\centering
\begin{tabulary}{\linewidth}[t]{|T|T|T|T|T|T|T|T|}
\hline
\sphinxstyletheadfamily 
Name
&\sphinxstyletheadfamily 
Type
&\sphinxstyletheadfamily 
AgentShare
&\sphinxstyletheadfamily 
RegionName
&\sphinxstyletheadfamily 
Objective1
&\sphinxstyletheadfamily 
SearchRule
&\sphinxstyletheadfamily 
DecisionMethod
&\sphinxstyletheadfamily 
…
\\
\hline
A1
&
New
&
Agent5
&
ASEAN
&
EAC
&
all
&
epsilonCon
&
…
\\
\hline
A4
&
New
&
Agent6
&
ASEAN
&
CapitalCosts
&
existing
&
weightedSum
&
…
\\
\hline
A1
&
Retrofit
&
Agent1
&
ASEAN
&
efficiency
&
all
&
epsilonCon
&
…
\\
\hline
A2
&
Retrofit
&
Agent2
&
ASEAN
&
Emissions
&
similar
&
weightedSum
&
…
\\
\hline
\end{tabulary}
\par
\sphinxattableend\end{savenotes}

For simplicity, not all columns are included in the example above. Though all column
listed below are currently required.

The columns have the following meaning:

\phantomsection\label{\detokenize{inputs/agents:name}}\begin{description}
\item[{Name}] \leavevmode
Name shared by a retrofit and new\sphinxhyphen{}capacity agent pair.

\item[{Type}] \leavevmode
One of “New” or “Retrofit”. “New” and “Retrofit” agents make up a pair with a given
{\hyperref[\detokenize{inputs/agents:name}]{\sphinxcrossref{\DUrole{std,std-ref}{name}}}}. The demand is split into two, with one part coming from
decommissioned assets, and the other coming from everything else. “Retrofit” agents
invest only to make up for decommissioned assets. They are often limited in the
technologies they can consider (by {\hyperref[\detokenize{inputs/agents:searchrule}]{\sphinxcrossref{\DUrole{std,std-ref}{SearchRule}}}}). “New” agents
invest on the rest of the demand, and can often consider more general sets of
technologies.

\item[{AgentShare}] \leavevmode
Name of the share of the existing capacity assigned to this agent. Only meaningful
for retrofit agents. The actual share itself can be found in
{\hyperref[\detokenize{inputs/technodata:inputs-technodata}]{\sphinxcrossref{\DUrole{std,std-ref}{Techno\sphinxhyphen{}data}}}}.

\item[{RegionName}] \leavevmode
Region where an agent operates.

\end{description}
\phantomsection\label{\detokenize{inputs/agents:objective1}}\begin{description}
\item[{Objective1}] \leavevmode
First objective that an agent will try and maximize or minimize during investment.
This objective should be one registered with
\sphinxcode{\sphinxupquote{@register\_objective}}. The following objectives are
available with MUSE:
\begin{itemize}
\item {} 
\sphinxcode{\sphinxupquote{comfort}}: Comfort provided by a given technology. Comfort does
not change during the simulation. It is obtained straightforwardly from
{\hyperref[\detokenize{inputs/technodata:inputs-technodata}]{\sphinxcrossref{\DUrole{std,std-ref}{Techno\sphinxhyphen{}data}}}}.

\item {} 
\sphinxcode{\sphinxupquote{efficiency}}: Efficiency of the technologies. Efficiency does
not change during the simulation. It is obtained straightforwardly from
{\hyperref[\detokenize{inputs/technodata:inputs-technodata}]{\sphinxcrossref{\DUrole{std,std-ref}{Techno\sphinxhyphen{}data}}}}.

\item {} 
\sphinxcode{\sphinxupquote{fixed\_costs}}: The fixed maintenance costs incurred by a
technology. The costs are a function of the capacity required to fulfil the current
demand.

\item {} 
\sphinxcode{\sphinxupquote{capital\_costs}}: The capital cost incurred by a
technology. The capital cost does not change during the simulation. It is obtained
as a function of parameters found in {\hyperref[\detokenize{inputs/technodata:inputs-technodata}]{\sphinxcrossref{\DUrole{std,std-ref}{Techno\sphinxhyphen{}data}}}}.

\item {} 
\sphinxcode{\sphinxupquote{emission\_cost}}: The costs associated for emissions for a
technology. The costs is a function both of the amount produced (equated to the
total demand in this case) and of the prices associated with each pollutant.
Aliased to “emission” for simplicity.

\item {} 
\sphinxcode{\sphinxupquote{fuel\_consumption\_cost}}: Costs of the fuels for
each technology, where each technology is used to fulfil the whole demand.

\item {} 
\sphinxcode{\sphinxupquote{lifetime\_levelized\_cost\_of\_energy}}:
LCOE over the lifetime of a technology. Aliased to “LCOE” for simplicity.

\item {} 
\sphinxcode{\sphinxupquote{net\_present\_value}}: Present value of all the costs of
installing and operating a technology, minus its revenues, of the course of its
lifetime. Aliased to “NPV” for simplicity.

\item {} 
\sphinxcode{\sphinxupquote{equivalent\_annual\_cost}}: Annualized form of the
net present value. Aliased to “EAC” for simplicity.

\end{itemize}

The weight associated with this objective can be changed using {\hyperref[\detokenize{inputs/agents:objdata1}]{\sphinxcrossref{\DUrole{std,std-ref}{ObjData1}}}}.  Whether the objective should be minimized or maximized depends on
{\hyperref[\detokenize{inputs/agents:objsort1}]{\sphinxcrossref{\DUrole{std,std-ref}{Objsort1}}}}. Multiple objectives are combined using the
{\hyperref[\detokenize{inputs/agents:decisionmethod}]{\sphinxcrossref{\DUrole{std,std-ref}{DecisionMethod}}}}

\end{description}
\phantomsection\label{\detokenize{inputs/agents:objective2}}\begin{description}
\item[{Objective2}] \leavevmode
Second objective. See {\hyperref[\detokenize{inputs/agents:objective1}]{\sphinxcrossref{\DUrole{std,std-ref}{Objective1}}}}.

\end{description}
\phantomsection\label{\detokenize{inputs/agents:objective3}}\begin{description}
\item[{Objective3:}] \leavevmode
Third objective. See {\hyperref[\detokenize{inputs/agents:objective1}]{\sphinxcrossref{\DUrole{std,std-ref}{Objective1}}}}.

\end{description}
\phantomsection\label{\detokenize{inputs/agents:objdata1}}\begin{description}
\item[{ObjData1}] \leavevmode
A weight associated with the {\hyperref[\detokenize{inputs/agents:objective1}]{\sphinxcrossref{\DUrole{std,std-ref}{first objective}}}}. Whether it is used
will depend in large part on the {\hyperref[\detokenize{inputs/agents:decisionmethod}]{\sphinxcrossref{\DUrole{std,std-ref}{decision method}}}}.

\item[{ObjData2}] \leavevmode
A weight associated with the {\hyperref[\detokenize{inputs/agents:objective2}]{\sphinxcrossref{\DUrole{std,std-ref}{second objective}}}}. See {\hyperref[\detokenize{inputs/agents:objdata1}]{\sphinxcrossref{\DUrole{std,std-ref}{ObjData1}}}}.

\item[{ObjData3}] \leavevmode
A weight associated with the {\hyperref[\detokenize{inputs/agents:objective3}]{\sphinxcrossref{\DUrole{std,std-ref}{third objective}}}}. See {\hyperref[\detokenize{inputs/agents:objdata1}]{\sphinxcrossref{\DUrole{std,std-ref}{ObjData1}}}}.

\end{description}
\phantomsection\label{\detokenize{inputs/agents:objsort1}}\begin{description}
\item[{Objsort1}] \leavevmode
Whether to maximize (\sphinxtitleref{True}) or minimize (\sphinxtitleref{False}) the {\hyperref[\detokenize{inputs/agents:objective1}]{\sphinxcrossref{\DUrole{std,std-ref}{first objective}}}}.

\item[{Objsort2}] \leavevmode
Whether to maximize (\sphinxtitleref{True}) or minimize (\sphinxtitleref{False}) the {\hyperref[\detokenize{inputs/agents:objective2}]{\sphinxcrossref{\DUrole{std,std-ref}{second objective}}}}.

\item[{Objsort3}] \leavevmode
Whether to maximize (\sphinxtitleref{True}) or minimize (\sphinxtitleref{False}) the {\hyperref[\detokenize{inputs/agents:objective3}]{\sphinxcrossref{\DUrole{std,std-ref}{third objective}}}}.

\end{description}
\phantomsection\label{\detokenize{inputs/agents:searchrule}}\begin{description}
\item[{SearchRule}] \leavevmode
The search rule allows users to par down the search space of technologies to those an
agent is likely to consider.
The search rule is any function with a given signature, and registered with MUSE via
\sphinxcode{\sphinxupquote{@register\_filter}}. The following search rules, defined
in \sphinxcode{\sphinxupquote{filters}}, are available with MUSE:
\begin{itemize}
\item {} 
\sphinxcode{\sphinxupquote{same\_enduse}}: Only allow technologies that provide the same
enduse as the current set of technologies owned by the agent.

\item {} 
\sphinxcode{\sphinxupquote{identity}}: Allows all current technologies. E.g. disables
filtering. Aliased to “all”.

\item {} 
\sphinxcode{\sphinxupquote{similar\_technology}}: Only allows technologies that
have the same type as current crop of technologies in the agent, as determined by
“tech\_type” in {\hyperref[\detokenize{inputs/technodata:inputs-technodata}]{\sphinxcrossref{\DUrole{std,std-ref}{Techno\sphinxhyphen{}data}}}}. Aliased to “similar”.

\item {} 
\sphinxcode{\sphinxupquote{same\_fuels}}: Only allows technologies that consume the same
fuels as the current crop of technologies in the agent. Aliased to
“fueltype”.

\item {} 
\sphinxcode{\sphinxupquote{currently\_existing\_tech}}: Only allows
technologies that the agent already owns. Aliased to “existing”.

\item {} 
\sphinxcode{\sphinxupquote{currently\_referenced\_tech}}: Only allows
technologies that are currently present in the market with non\sphinxhyphen{}zero capacity.

\item {} 
\sphinxcode{\sphinxupquote{maturity}}: Only allows technologies that have achieved a given
market share.

\end{itemize}

The implementation allows for combining these filters. However, the CSV data format
described here does not.

\end{description}
\phantomsection\label{\detokenize{inputs/agents:decisionmethod}}\begin{description}
\item[{DecisionMethod}] \leavevmode
Decision methods reduce multiple objectives into a single scalar objective per
replacement technology. They allow combining several objectives into a single metric
through which replacement technologies can be ranked.

Decision methods are any function which follow a given signature and are registered
via the decorator \sphinxcode{\sphinxupquote{@register\_decision}}. The following
decision methods are available with MUSE, as implemented in
\sphinxcode{\sphinxupquote{decisions}}:
\begin{itemize}
\item {} 
\sphinxcode{\sphinxupquote{mean}}: Computes the average across several objectives.

\item {} 
\sphinxcode{\sphinxupquote{weighted\_sum}}: Computes a weighted average across several
objectives.

\item {} 
\sphinxcode{\sphinxupquote{lexical\_comparion}}: Compares objectives using a
binned lexical comparison operator. Aliased to “lexo”.

\item {} 
\sphinxcode{\sphinxupquote{retro\_lexical\_comparion}}: A binned lexical
comparison function where the bin size is adjusted to ensure the current crop of
technologies are competitive. Aliased to “retro\_lexo”.

\item {} 
\sphinxcode{\sphinxupquote{epsilon\_constraints}}: A comparison method which
ensures that first selects technologies following constraints on objectives 2 and
higher, before actually ranking them using objective 1. Aliased to “epsilon” ad
“epsilon\_con”.

\item {} 
\sphinxcode{\sphinxupquote{retro\_epsilon\_constraints}}: A variation on
epsilon constraints which ensures that the current crop of technologies are not
deselected by the constraints. Aliased to “retro\_epsilon”.

\item {} 
\sphinxcode{\sphinxupquote{single\_objective}}: A decision method to allow
ranking via a single objective.

\end{itemize}

The functions allow for any number of objectives. However, the format described here
allows only for three.

\item[{Quantity}] \leavevmode
A factor used to determine the demand share of “New” agents.

\item[{MaturityThreshold}] \leavevmode
Parameter for the search rule \sphinxcode{\sphinxupquote{maturity}}.

\end{description}


\subsection{Indices and tables}
\label{\detokenize{inputs/inputs_csv:indices-and-tables}}\begin{itemize}
\item {} 
\DUrole{xref,std,std-ref}{genindex}

\item {} 
\DUrole{xref,std,std-ref}{modindex}

\item {} 
\DUrole{xref,std,std-ref}{search}

\end{itemize}


\section{Indices and tables}
\label{\detokenize{inputs/index:indices-and-tables}}\begin{itemize}
\item {} 
\DUrole{xref,std,std-ref}{genindex}

\item {} 
\DUrole{xref,std,std-ref}{modindex}

\item {} 
\DUrole{xref,std,std-ref}{search}

\end{itemize}


\chapter{Advanced developer guide}
\label{\detokenize{advanced-guide/index:advanced-developer-guide}}\label{\detokenize{advanced-guide/index::doc}}

\section{Hooks}
\label{\detokenize{advanced-guide/hooks:hooks}}\label{\detokenize{advanced-guide/hooks::doc}}
This is where we explain hooks.


\section{Indices and tables}
\label{\detokenize{advanced-guide/index:indices-and-tables}}\begin{itemize}
\item {} 
\DUrole{xref,std,std-ref}{genindex}

\item {} 
\DUrole{xref,std,std-ref}{modindex}

\item {} 
\DUrole{xref,std,std-ref}{search}

\end{itemize}


\chapter{Indices and tables}
\label{\detokenize{index:indices-and-tables}}\begin{itemize}
\item {} 
\DUrole{xref,std,std-ref}{genindex}

\item {} 
\DUrole{xref,std,std-ref}{modindex}

\item {} 
\DUrole{xref,std,std-ref}{search}

\end{itemize}



\renewcommand{\indexname}{Index}
\printindex
\end{document}