%% Generated by Sphinx.
\def\sphinxdocclass{report}
\documentclass[letterpaper,10pt,english]{sphinxmanual}
\ifdefined\pdfpxdimen
   \let\sphinxpxdimen\pdfpxdimen\else\newdimen\sphinxpxdimen
\fi \sphinxpxdimen=.75bp\relax

\PassOptionsToPackage{warn}{textcomp}
\usepackage[utf8]{inputenc}
\ifdefined\DeclareUnicodeCharacter
% support both utf8 and utf8x syntaxes
  \ifdefined\DeclareUnicodeCharacterAsOptional
    \def\sphinxDUC#1{\DeclareUnicodeCharacter{"#1}}
  \else
    \let\sphinxDUC\DeclareUnicodeCharacter
  \fi
  \sphinxDUC{00A0}{\nobreakspace}
  \sphinxDUC{2500}{\sphinxunichar{2500}}
  \sphinxDUC{2502}{\sphinxunichar{2502}}
  \sphinxDUC{2514}{\sphinxunichar{2514}}
  \sphinxDUC{251C}{\sphinxunichar{251C}}
  \sphinxDUC{2572}{\textbackslash}
\fi
\usepackage{cmap}
\usepackage[T1]{fontenc}
\usepackage{amsmath,amssymb,amstext}
\usepackage{babel}



\usepackage{times}
\expandafter\ifx\csname T@LGR\endcsname\relax
\else
% LGR was declared as font encoding
  \substitutefont{LGR}{\rmdefault}{cmr}
  \substitutefont{LGR}{\sfdefault}{cmss}
  \substitutefont{LGR}{\ttdefault}{cmtt}
\fi
\expandafter\ifx\csname T@X2\endcsname\relax
  \expandafter\ifx\csname T@T2A\endcsname\relax
  \else
  % T2A was declared as font encoding
    \substitutefont{T2A}{\rmdefault}{cmr}
    \substitutefont{T2A}{\sfdefault}{cmss}
    \substitutefont{T2A}{\ttdefault}{cmtt}
  \fi
\else
% X2 was declared as font encoding
  \substitutefont{X2}{\rmdefault}{cmr}
  \substitutefont{X2}{\sfdefault}{cmss}
  \substitutefont{X2}{\ttdefault}{cmtt}
\fi


\usepackage[Bjarne]{fncychap}
\usepackage{sphinx}

\fvset{fontsize=\small}
\usepackage{geometry}


% Include hyperref last.
\usepackage{hyperref}
% Fix anchor placement for figures with captions.
\usepackage{hypcap}% it must be loaded after hyperref.
% Set up styles of URL: it should be placed after hyperref.
\urlstyle{same}

\addto\captionsenglish{\renewcommand{\contentsname}{Contents:}}

\usepackage{sphinxmessages}
\setcounter{tocdepth}{1}


% Jupyter Notebook code cell colors
\definecolor{nbsphinxin}{HTML}{307FC1}
\definecolor{nbsphinxout}{HTML}{BF5B3D}
\definecolor{nbsphinx-code-bg}{HTML}{F5F5F5}
\definecolor{nbsphinx-code-border}{HTML}{E0E0E0}
\definecolor{nbsphinx-stderr}{HTML}{FFDDDD}
% ANSI colors for output streams and traceback highlighting
\definecolor{ansi-black}{HTML}{3E424D}
\definecolor{ansi-black-intense}{HTML}{282C36}
\definecolor{ansi-red}{HTML}{E75C58}
\definecolor{ansi-red-intense}{HTML}{B22B31}
\definecolor{ansi-green}{HTML}{00A250}
\definecolor{ansi-green-intense}{HTML}{007427}
\definecolor{ansi-yellow}{HTML}{DDB62B}
\definecolor{ansi-yellow-intense}{HTML}{B27D12}
\definecolor{ansi-blue}{HTML}{208FFB}
\definecolor{ansi-blue-intense}{HTML}{0065CA}
\definecolor{ansi-magenta}{HTML}{D160C4}
\definecolor{ansi-magenta-intense}{HTML}{A03196}
\definecolor{ansi-cyan}{HTML}{60C6C8}
\definecolor{ansi-cyan-intense}{HTML}{258F8F}
\definecolor{ansi-white}{HTML}{C5C1B4}
\definecolor{ansi-white-intense}{HTML}{A1A6B2}
\definecolor{ansi-default-inverse-fg}{HTML}{FFFFFF}
\definecolor{ansi-default-inverse-bg}{HTML}{000000}

% Define an environment for non-plain-text code cell outputs (e.g. images)
\makeatletter
\newenvironment{nbsphinxfancyoutput}{%
    % Avoid fatal error with framed.sty if graphics too long to fit on one page
    \let\sphinxincludegraphics\nbsphinxincludegraphics
    \nbsphinx@image@maxheight\textheight
    \advance\nbsphinx@image@maxheight -2\fboxsep   % default \fboxsep 3pt
    \advance\nbsphinx@image@maxheight -2\fboxrule  % default \fboxrule 0.4pt
    \advance\nbsphinx@image@maxheight -\baselineskip
\def\nbsphinxfcolorbox{\spx@fcolorbox{nbsphinx-code-border}{white}}%
\def\FrameCommand{\nbsphinxfcolorbox\nbsphinxfancyaddprompt\@empty}%
\def\FirstFrameCommand{\nbsphinxfcolorbox\nbsphinxfancyaddprompt\sphinxVerbatim@Continues}%
\def\MidFrameCommand{\nbsphinxfcolorbox\sphinxVerbatim@Continued\sphinxVerbatim@Continues}%
\def\LastFrameCommand{\nbsphinxfcolorbox\sphinxVerbatim@Continued\@empty}%
\MakeFramed{\advance\hsize-\width\@totalleftmargin\z@\linewidth\hsize\@setminipage}%
\lineskip=1ex\lineskiplimit=1ex\raggedright%
}{\par\unskip\@minipagefalse\endMakeFramed}
\makeatother
\newbox\nbsphinxpromptbox
\def\nbsphinxfancyaddprompt{\ifvoid\nbsphinxpromptbox\else
    \kern\fboxrule\kern\fboxsep
    \copy\nbsphinxpromptbox
    \kern-\ht\nbsphinxpromptbox\kern-\dp\nbsphinxpromptbox
    \kern-\fboxsep\kern-\fboxrule\nointerlineskip
    \fi}
\newlength\nbsphinxcodecellspacing
\setlength{\nbsphinxcodecellspacing}{0pt}

% Define support macros for attaching opening and closing lines to notebooks
\newsavebox\nbsphinxbox
\makeatletter
\newcommand{\nbsphinxstartnotebook}[1]{%
    \par
    % measure needed space
    \setbox\nbsphinxbox\vtop{{#1\par}}
    % reserve some space at bottom of page, else start new page
    \needspace{\dimexpr2.5\baselineskip+\ht\nbsphinxbox+\dp\nbsphinxbox}
    % mimick vertical spacing from \section command
      \addpenalty\@secpenalty
      \@tempskipa 3.5ex \@plus 1ex \@minus .2ex\relax
      \addvspace\@tempskipa
      {\Large\@tempskipa\baselineskip
             \advance\@tempskipa-\prevdepth
             \advance\@tempskipa-\ht\nbsphinxbox
             \ifdim\@tempskipa>\z@
               \vskip \@tempskipa
             \fi}
    \unvbox\nbsphinxbox
    % if notebook starts with a \section, prevent it from adding extra space
    \@nobreaktrue\everypar{\@nobreakfalse\everypar{}}%
    % compensate the parskip which will get inserted by next paragraph
    \nobreak\vskip-\parskip
    % do not break here
    \nobreak
}% end of \nbsphinxstartnotebook

\newcommand{\nbsphinxstopnotebook}[1]{%
    \par
    % measure needed space
    \setbox\nbsphinxbox\vbox{{#1\par}}
    \nobreak % it updates page totals
    \dimen@\pagegoal
    \advance\dimen@-\pagetotal \advance\dimen@-\pagedepth
    \advance\dimen@-\ht\nbsphinxbox \advance\dimen@-\dp\nbsphinxbox
    \ifdim\dimen@<\z@
      % little space left
      \unvbox\nbsphinxbox
      \kern-.8\baselineskip
      \nobreak\vskip\z@\@plus1fil
      \penalty100
      \vskip\z@\@plus-1fil
      \kern.8\baselineskip
    \else
      \unvbox\nbsphinxbox
    \fi
}% end of \nbsphinxstopnotebook

% Ensure height of an included graphics fits in nbsphinxfancyoutput frame
\newdimen\nbsphinx@image@maxheight % set in nbsphinxfancyoutput environment
\newcommand*{\nbsphinxincludegraphics}[2][]{%
    \gdef\spx@includegraphics@options{#1}%
    \setbox\spx@image@box\hbox{\includegraphics[#1,draft]{#2}}%
    \in@false
    \ifdim \wd\spx@image@box>\linewidth
      \g@addto@macro\spx@includegraphics@options{,width=\linewidth}%
      \in@true
    \fi
    % no rotation, no need to worry about depth
    \ifdim \ht\spx@image@box>\nbsphinx@image@maxheight
      \g@addto@macro\spx@includegraphics@options{,height=\nbsphinx@image@maxheight}%
      \in@true
    \fi
    \ifin@
      \g@addto@macro\spx@includegraphics@options{,keepaspectratio}%
    \fi
    \setbox\spx@image@box\box\voidb@x % clear memory
    \expandafter\includegraphics\expandafter[\spx@includegraphics@options]{#2}%
}% end of "\MakeFrame"-safe variant of \sphinxincludegraphics
\makeatother

\makeatletter
\renewcommand*\sphinx@verbatim@nolig@list{\do\'\do\`}
\begingroup
\catcode`'=\active
\let\nbsphinx@noligs\@noligs
\g@addto@macro\nbsphinx@noligs{\let'\PYGZsq}
\endgroup
\makeatother
\renewcommand*\sphinxbreaksbeforeactivelist{\do\<\do\"\do\'}
\renewcommand*\sphinxbreaksafteractivelist{\do\.\do\,\do\:\do\;\do\?\do\!\do\/\do\>\do\-}
\makeatletter
\fvset{codes*=\sphinxbreaksattexescapedchars\do\^\^\let\@noligs\nbsphinx@noligs}
\makeatother



\title{MUSE: ModUlar energy system Simulation Environment Documentation}
\date{Nov 10, 2020}
\release{0.8}
\author{Sustainable Gas Institute}
\newcommand{\sphinxlogo}{\vbox{}}
\renewcommand{\releasename}{Release}
\makeindex
\begin{document}

\pagestyle{empty}
\sphinxmaketitle
\pagestyle{plain}
\sphinxtableofcontents
\pagestyle{normal}
\phantomsection\label{\detokenize{index::doc}}



\chapter{Installation}
\label{\detokenize{installing-muse:installation}}\label{\detokenize{installing-muse::doc}}
There are two ways to install MUSE: one for users who do not wish to modify the source code of MUSE, and another for developers who do.

\begin{sphinxadmonition}{note}{Note:}
Windows users and developers may need to install \sphinxhref{https://visualstudio.microsoft.com/downloads/\#build-tools-for-visual-studio-2019}{Windows Build Tools}. These tools include C/C++ compilers which are needed to build some python dependencies.

MacOS includes compilers by default, hence no action is needed for Mac users.

Linux users may need to install a C compiler, whether GNU gcc or Clang, as well python development packages, depending on their distribution.
\begin{enumerate}
\sphinxsetlistlabels{\arabic}{enumi}{enumii}{}{.}%
\item {} 
Download Microsoft Visual C++ Build Tools from this link: \sphinxurl{https://visualstudio.microsoft.com/downloads/}

\item {} 
Select your preferred edition. The “Community” is free and contains what is required.

\item {} 
Run the installer

\item {} 
Select: Workloads \(\rightarrow\) Visual C++ build tools.

\item {} 
Install options: select only the “Windows 10 SDK” (assuming the computer is Windows 10){]}

\end{enumerate}

For further information, see this link: \sphinxurl{https://www.scivision.dev/python-windows-visual-c-14-required}
\end{sphinxadmonition}


\section{For users}
\label{\detokenize{installing-muse:for-users}}
MUSE is developed using python, an open\sphinxhyphen{}source programming language, which means that there are two steps to the installation process. First, python should be installed. Then so should MUSE.

The simplest method to install python is by downloading the \sphinxhref{https://www.anaconda.com/distribution/\#download-section}{Anaconda distribution}. Make sure to choose the appropriate operating system (e.g. windows), python version 3.7, and the 64 bit installer. Once this has been done follow the steps for the anaconda installer, as prompted.

After python is installed we can install MUSE. MUSE can be installed via the \sphinxhref{https://docs.anaconda.com/anaconda/user-guide/getting-started/\#write-a-python-program-using-anaconda-prompt-or-terminal}{Anaconda Prompt} (or any terminal on Mac and Linux). This is a command\sphinxhyphen{}line interface to python and the python eco\sphinxhyphen{}system. In the anaconda prompt, run:

\begin{sphinxVerbatim}[commandchars=\\\{\}]
python \PYGZhy{}m pip install \PYGZhy{}\PYGZhy{}user git+https://github.com/SGIModel/StarMuse
\end{sphinxVerbatim}

It should now be possible to run muse. Again, this can be done in the anaconda prompt as follows:

\begin{sphinxVerbatim}[commandchars=\\\{\}]
python \PYGZhy{}m muse \PYGZhy{}\PYGZhy{}help
\end{sphinxVerbatim}

\begin{sphinxadmonition}{note}{Note:}
Although not strictly necessary, users are encouraged to create an \sphinxhref{https://www.anaconda.com/what-is-anaconda/}{Anaconda virtual environment} and install MUSE there, as shown in {\hyperref[\detokenize{installing-muse:installation-devs}]{\sphinxcrossref{\DUrole{std,std-ref}{For developers}}}}.
\end{sphinxadmonition}


\section{For developers}
\label{\detokenize{installing-muse:for-developers}}\label{\detokenize{installing-muse:installation-devs}}
Although not strictly necessary, creating an \sphinxhref{https://www.anaconda.com/what-is-anaconda/}{Anaconda virtual environment} is highly
recommended. Anaconda will isolate users and developers from changes occuring on their
operating system, and from conflicts between python packages. It also ensures reproducibility
from day to day.

Create a virtual env including python with:

\begin{sphinxVerbatim}[commandchars=\\\{\}]
conda create \PYGZhy{}n muse \PYG{n+nv}{python}\PYG{o}{=}\PYG{l+m}{3}.7
\end{sphinxVerbatim}

Activate the environment with:

\begin{sphinxVerbatim}[commandchars=\\\{\}]
conda activate muse
\end{sphinxVerbatim}

Later, to recover the system\sphinxhyphen{}wide “normal” python, deactivate the environment with:

\begin{sphinxVerbatim}[commandchars=\\\{\}]
conda deactivate
\end{sphinxVerbatim}

The simplest approach is to first download the muse code with \sphinxhref{https://git-scm.com/}{git}:

\begin{sphinxVerbatim}[commandchars=\\\{\}]
git clone https://github.com/SGIModel/StarMuse.git muse
\end{sphinxVerbatim}

For interested users, there are plenty of \sphinxhref{http://try.github.io/}{good} tutorials for \sphinxhref{https://git-scm.com/}{git}.
Next, it is possible to install the working directory into the conda environment:

\begin{sphinxVerbatim}[commandchars=\\\{\}]
\PYG{c+c1}{\PYGZsh{} On Linux and Mac}
\PYG{n+nb}{cd} muse
conda activate muse
python \PYGZhy{}m pip install \PYGZhy{}e \PYG{l+s+s2}{\PYGZdq{}.[dev,docs]\PYGZdq{}}

\PYG{c+c1}{\PYGZsh{} On Windows}
dir muse
conda activate muse
python \PYGZhy{}m pip install \PYGZhy{}e \PYG{l+s+s2}{\PYGZdq{}.[dev,docs]\PYGZdq{}}
\end{sphinxVerbatim}

The quotation marks are needed on some systems or shells, and do not hurt on any. The
downloaded code can then be modified. The changes will be automatically reflected in the
conda environment.

Tests can be run with the command \sphinxhref{https://docs.pytest.org/en/latest/}{pytest}, from the testing framework of the same name.

The documentation can be built with:

\begin{sphinxVerbatim}[commandchars=\\\{\}]
python setup.py docs
\end{sphinxVerbatim}

The main page for the documentation can then be found at
\sphinxtitleref{build\textbackslash{}sphinx\textbackslash{}html\textbackslash{}index.html} (or \sphinxtitleref{build/sphinx/html/index.html} on Mac and Linux).
The file can viewed from any web browser.


\chapter{Running your first example}
\label{\detokenize{running-muse-example:Running-your-first-example}}\label{\detokenize{running-muse-example::doc}}
In this section we run an example simulation of MUSE and visualise the results. There are a number of different examples in the source code, which can be found \sphinxhref{dead-link}{INSERT LINK HERE}.

Once python and MUSE have been installed, we can run an example. To do this open anaconda prompt. Then change directory to where you have downloaded the MUSE source code.

Navigate to the following link for MacOS or Linux based operating systems:

\sphinxcode{\sphinxupquote{\{MUSE\_download\_location\}/StarMuse/run/example/default/}}

Change \sphinxcode{\sphinxupquote{\{MUSE\_download\_location\}}} to the location you downloaded MUSE to, for example \sphinxcode{\sphinxupquote{Users/\{my\_name\}/Documents/}} using the \sphinxcode{\sphinxupquote{cd}} command, or “change directory” command. Once we have navigated to the directory containing the example settings \sphinxcode{\sphinxupquote{settings.toml}} we can run the simulation using the following command in the anaconda prompt or terminal:

\sphinxcode{\sphinxupquote{python \sphinxhyphen{}m muse settings.toml}}

If running correctly, your prompt should output text similar to that which can be found here.

It is also possible to run MUSE directly in python using the following code:

{
\sphinxsetup{VerbatimColor={named}{nbsphinx-code-bg}}
\sphinxsetup{VerbatimBorderColor={named}{nbsphinx-code-border}}
\begin{sphinxVerbatim}[commandchars=\\\{\}]
\llap{\color{nbsphinxin}[ ]:\,\hspace{\fboxrule}\hspace{\fboxsep}}\PYG{k+kn}{from} \PYG{n+nn}{muse} \PYG{k+kn}{import} \PYG{n}{examples}
\PYG{n}{model} \PYG{o}{=} \PYG{n}{examples}\PYG{o}{.}\PYG{n}{model}\PYG{p}{(}\PYG{l+s+s2}{\PYGZdq{}}\PYG{l+s+s2}{default}\PYG{l+s+s2}{\PYGZdq{}}\PYG{p}{)}
\PYG{n}{model}\PYG{o}{.}\PYG{n}{run}\PYG{p}{(}\PYG{p}{)}
\end{sphinxVerbatim}
}


\section{Results}
\label{\detokenize{running-muse-example:Results}}
If the default MUSE example has run successfully, you should now have a folder called \sphinxcode{\sphinxupquote{Results}} in the same directory as \sphinxcode{\sphinxupquote{settings.toml}}.

This directory should contain results for each sector (\sphinxcode{\sphinxupquote{Gas}},\sphinxcode{\sphinxupquote{Power}} and \sphinxcode{\sphinxupquote{Residential}}) as well as results for the entire simulation in the form of \sphinxcode{\sphinxupquote{MCACapacity.csv}} and \sphinxcode{\sphinxupquote{MCAPrices.csv}}.
\begin{itemize}
\item {} 
\sphinxcode{\sphinxupquote{MCACapacity.csv}} contains information about the capacity each agent has for each technology per year.

\item {} 
\sphinxcode{\sphinxupquote{MCAPrices.csv}} has the price of each commodity per year and timeslice. eg. the cost of electricity at night for electricity in 2020.

\end{itemize}

Within each of the sector result folders, there is an output for \sphinxcode{\sphinxupquote{Capacity}} for each commodity in each year. The years into the future, which the simulation has not run to, refers to the capacity as it retires. Within the \sphinxcode{\sphinxupquote{Residential}} folder there is also a folder for \sphinxcode{\sphinxupquote{Supply}} within each year. This refers to how much end\sphinxhyphen{}use commodity was output.

The output can be fully configurable, as shown in the developer guide {\hyperref[\detokenize{advanced-guide/extending-muse::doc}]{\sphinxcrossref{\DUrole{doc}{here}}}}.


\section{Visualisation}
\label{\detokenize{running-muse-example:Visualisation}}
{
\sphinxsetup{VerbatimColor={named}{nbsphinx-code-bg}}
\sphinxsetup{VerbatimBorderColor={named}{nbsphinx-code-border}}
\begin{sphinxVerbatim}[commandchars=\\\{\}]
\llap{\color{nbsphinxin}[1]:\,\hspace{\fboxrule}\hspace{\fboxsep}}\PYG{k+kn}{import} \PYG{n+nn}{pandas} \PYG{k}{as} \PYG{n+nn}{pd}
\PYG{k+kn}{import} \PYG{n+nn}{matplotlib}\PYG{n+nn}{.}\PYG{n+nn}{pyplot} \PYG{k}{as} \PYG{n+nn}{plt}
\PYG{k+kn}{import} \PYG{n+nn}{seaborn} \PYG{k}{as} \PYG{n+nn}{sns}
\end{sphinxVerbatim}
}

Next, we load the dataset of interest to us for this example: the \sphinxcode{\sphinxupquote{MCACapacity.csv}} file. We do this using pandas.

{
\sphinxsetup{VerbatimColor={named}{nbsphinx-code-bg}}
\sphinxsetup{VerbatimBorderColor={named}{nbsphinx-code-border}}
\begin{sphinxVerbatim}[commandchars=\\\{\}]
\llap{\color{nbsphinxin}[2]:\,\hspace{\fboxrule}\hspace{\fboxsep}}\PYG{n}{capacity\PYGZus{}results} \PYG{o}{=} \PYG{n}{pd}\PYG{o}{.}\PYG{n}{read\PYGZus{}csv}\PYG{p}{(}\PYG{l+s+s2}{\PYGZdq{}}\PYG{l+s+s2}{Results/MCAcapacity.csv}\PYG{l+s+s2}{\PYGZdq{}}\PYG{p}{)}
\PYG{n}{capacity\PYGZus{}results}\PYG{o}{.}\PYG{n}{head}\PYG{p}{(}\PYG{p}{)}
\end{sphinxVerbatim}
}

{

\kern-\sphinxverbatimsmallskipamount\kern-\baselineskip
\kern+\FrameHeightAdjust\kern-\fboxrule
\vspace{\nbsphinxcodecellspacing}

\sphinxsetup{VerbatimColor={named}{white}}
\sphinxsetup{VerbatimBorderColor={named}{nbsphinx-code-border}}
\begin{sphinxVerbatim}[commandchars=\\\{\}]
\llap{\color{nbsphinxout}[2]:\,\hspace{\fboxrule}\hspace{\fboxsep}}   technology region agent      type       sector  capacity  year
0   gasboiler     R1    A1  retrofit  residential      10.0  2020
1     gasCCGT     R1    A1  retrofit        power       1.0  2020
2  gassupply1     R1    A1  retrofit          gas      15.0  2020
3   gasboiler     R1    A1  retrofit  residential       5.0  2025
4    heatpump     R1    A1  retrofit  residential      19.0  2025
\end{sphinxVerbatim}
}

Using the \sphinxcode{\sphinxupquote{head}} command we print the first five rows of our dataset. Next, we will visualise each of the sectors, with capacity on the y\sphinxhyphen{}axis and year on the x\sphinxhyphen{}axis.

Don’t worry too much about the code if some of it is unfamiliar. We effectively split the data into each sector and then plot a line plot for each.

{
\sphinxsetup{VerbatimColor={named}{nbsphinx-code-bg}}
\sphinxsetup{VerbatimBorderColor={named}{nbsphinx-code-border}}
\begin{sphinxVerbatim}[commandchars=\\\{\}]
\llap{\color{nbsphinxin}[3]:\,\hspace{\fboxrule}\hspace{\fboxsep}}\PYG{k}{for} \PYG{n}{sector\PYGZus{}name}\PYG{p}{,} \PYG{n}{results} \PYG{o+ow}{in} \PYG{n}{capacity\PYGZus{}results}\PYG{o}{.}\PYG{n}{groupby}\PYG{p}{(}\PYG{l+s+s2}{\PYGZdq{}}\PYG{l+s+s2}{sector}\PYG{l+s+s2}{\PYGZdq{}}\PYG{p}{)}\PYG{p}{:}
    \PYG{n+nb}{print}\PYG{p}{(}\PYG{l+s+s2}{\PYGZdq{}}\PYG{l+s+si}{\PYGZob{}\PYGZcb{}}\PYG{l+s+s2}{ sector}\PYG{l+s+s2}{\PYGZdq{}}\PYG{o}{.}\PYG{n}{format}\PYG{p}{(}\PYG{n}{sector\PYGZus{}name}\PYG{p}{)}\PYG{p}{)}
    \PYG{n}{sns}\PYG{o}{.}\PYG{n}{lineplot}\PYG{p}{(}\PYG{n}{data}\PYG{o}{=}\PYG{n}{results}\PYG{p}{,} \PYG{n}{x}\PYG{o}{=}\PYG{l+s+s2}{\PYGZdq{}}\PYG{l+s+s2}{year}\PYG{l+s+s2}{\PYGZdq{}}\PYG{p}{,} \PYG{n}{y}\PYG{o}{=}\PYG{l+s+s2}{\PYGZdq{}}\PYG{l+s+s2}{capacity}\PYG{l+s+s2}{\PYGZdq{}}\PYG{p}{,} \PYG{n}{hue}\PYG{o}{=}\PYG{l+s+s2}{\PYGZdq{}}\PYG{l+s+s2}{technology}\PYG{l+s+s2}{\PYGZdq{}}\PYG{p}{)}
    \PYG{n}{plt}\PYG{o}{.}\PYG{n}{ylabel}\PYG{p}{(}\PYG{l+s+s2}{\PYGZdq{}}\PYG{l+s+s2}{capacity (PJ)}\PYG{l+s+s2}{\PYGZdq{}}\PYG{p}{)}
    \PYG{n}{plt}\PYG{o}{.}\PYG{n}{show}\PYG{p}{(}\PYG{p}{)}
    \PYG{n}{plt}\PYG{o}{.}\PYG{n}{close}\PYG{p}{(}\PYG{p}{)}
\end{sphinxVerbatim}
}

{

\kern-\sphinxverbatimsmallskipamount\kern-\baselineskip
\kern+\FrameHeightAdjust\kern-\fboxrule
\vspace{\nbsphinxcodecellspacing}

\sphinxsetup{VerbatimColor={named}{white}}
\sphinxsetup{VerbatimBorderColor={named}{nbsphinx-code-border}}
\begin{sphinxVerbatim}[commandchars=\\\{\}]
gas sector
\end{sphinxVerbatim}
}

\hrule height -\fboxrule\relax
\vspace{\nbsphinxcodecellspacing}

\makeatletter\setbox\nbsphinxpromptbox\box\voidb@x\makeatother

\begin{nbsphinxfancyoutput}

\noindent\sphinxincludegraphics[width=382\sphinxpxdimen,height=262\sphinxpxdimen]{{running-muse-example_11_1}.png}

\end{nbsphinxfancyoutput}

{

\kern-\sphinxverbatimsmallskipamount\kern-\baselineskip
\kern+\FrameHeightAdjust\kern-\fboxrule
\vspace{\nbsphinxcodecellspacing}

\sphinxsetup{VerbatimColor={named}{white}}
\sphinxsetup{VerbatimBorderColor={named}{nbsphinx-code-border}}
\begin{sphinxVerbatim}[commandchars=\\\{\}]
power sector
\end{sphinxVerbatim}
}

\hrule height -\fboxrule\relax
\vspace{\nbsphinxcodecellspacing}

\makeatletter\setbox\nbsphinxpromptbox\box\voidb@x\makeatother

\begin{nbsphinxfancyoutput}

\noindent\sphinxincludegraphics[width=382\sphinxpxdimen,height=262\sphinxpxdimen]{{running-muse-example_11_3}.png}

\end{nbsphinxfancyoutput}

{

\kern-\sphinxverbatimsmallskipamount\kern-\baselineskip
\kern+\FrameHeightAdjust\kern-\fboxrule
\vspace{\nbsphinxcodecellspacing}

\sphinxsetup{VerbatimColor={named}{white}}
\sphinxsetup{VerbatimBorderColor={named}{nbsphinx-code-border}}
\begin{sphinxVerbatim}[commandchars=\\\{\}]
residential sector
\end{sphinxVerbatim}
}

\hrule height -\fboxrule\relax
\vspace{\nbsphinxcodecellspacing}

\makeatletter\setbox\nbsphinxpromptbox\box\voidb@x\makeatother

\begin{nbsphinxfancyoutput}

\noindent\sphinxincludegraphics[width=382\sphinxpxdimen,height=262\sphinxpxdimen]{{running-muse-example_11_5}.png}

\end{nbsphinxfancyoutput}

In this toy example, we can see that the end\sphinxhyphen{}use technology of choice in the residential sector becomes a heatpump. The heatpump displaces the gas boiler. Therefore, the supply of gas crashes due to a reduced demand. To account for the increase in demand for electricity, the agent invests heavily in wind turbines.

Note, that the units are in petajoules (PJ). MUSE requires consistent units across each of the sectors, and each of the input files (which we will see later). The model does not make any unit conversion internally.


\section{Next steps}
\label{\detokenize{running-muse-example:Next-steps}}
If you want to jump straight into customising your own example scenarios, head to the link {\hyperref[\detokenize{user-guide/index::doc}]{\sphinxcrossref{\DUrole{doc}{here}}}}. If you would like a little bit of background based on how MUSE works first, head to the next section!


\chapter{MUSE Overview}
\label{\detokenize{overview:muse-overview}}\label{\detokenize{overview::doc}}
\begin{sphinxadmonition}{note}{Note:}
TODO: Potentially find introductory image to place here.
\end{sphinxadmonition}

MUSE is an open source agent\sphinxhyphen{}based modelling environment that can be used to simulate change in an energy system over time. An example of the type of question MUSE can help in answering is:
\begin{itemize}
\item {} 
How may a carbon budget affect investments made in the power sector over the next 30 years?

\end{itemize}

MUSE can incorporate residential, power, industrial and conversion sectors, meaning many questions can be explored using MUSE, as per the wishes of the user.

MUSE is an agent\sphinxhyphen{}based modelling environment, where the agents are investors and consumers. In MUSE, this means that investment decisions are made from the point of view of the investor and consumer. These agents can be heterogenous, enabling for differering investment strategies between agents, as in the real world.

MUSE is technology rich and can model energy production, conversion and end\sphinxhyphen{}use technologies. So, for example, MUSE can enable the user to develop a power sector with solar photovoltaics, wind turbines and gas power plants which produce energy for appliances like electric stoves, heaters and lighting in the residential sector. Agents invest within these sectors, investing in technologies such as electric stoves in the residential sector or gas power plants in the power sectors. The investments made depend on the agent’s investment strategies.

Every sector is a user configurable module. This means that a user can configure any number of sectors, cointaining custom, user\sphinxhyphen{}defined technologies and commodities. MUSE is fully data\sphinxhyphen{}driven, meaning that the configuration of the model is carried out using a selection of {\hyperref[\detokenize{inputs/index:input-files}]{\sphinxcrossref{\DUrole{std,std-ref}{Input Files}}}}. This means that you are able to customise MUSE to your wishes by modifying these input files. In addition, MUSE can model any geographical region around the world and over any time scale, from a single year through to 100 years or more. Within a year, MUSE allows for a user\sphinxhyphen{}defined temporal granularity. This allows for the year to be split into different seasons and times, where energy demand may differ. Thus allowing us to model diurnal peaks in the demand, varying weekly and seasonally.

MUSE differs from the vast majority of energy systems models, which are intertemporal optimisation, by allowing agents to have “limited foresight”. This enables these agents to invest under uncertainty of the future, as in the real world. In addition, MUSE is a “partial equilibrium” model, in the sense that it balances supply and demand of each energy commodity in the system.


\section{What questions can MUSE answer?}
\label{\detokenize{overview:what-questions-can-muse-answer}}
MUSE allows for users to investigate how an energy system may evolve over a time period, based upon investors using different decision metrics or objectives such as the \sphinxhref{https://en.wikipedia.org/wiki/Net\_present\_value}{net present value}, \sphinxhref{https://en.wikipedia.org/wiki/Levelized\_cost\_of\_energy}{levelized cost of electricity} or a custom\sphinxhyphen{}defined function. In addition to this, it can simulate how investors search for technology options, and how different objectives are combined to reach an investment decision.

The search for new technologies can depend on several factors such as agents’ budgets, technology maturity or preferences on the fuel\sphinxhyphen{}type. For instance, an investor in the power sector may decide that they want to focus on renewable energy, whereas another may prefer the perceived most profitable option.

Examples of the questions MUSE can answer include:
\begin{itemize}
\item {} 
\sphinxhref{https://www.sciencedirect.com/science/article/pii/S0306261920308072}{How may India’s steel industry decarbonise?}

\item {} 
\sphinxhref{https://www.sciencedirect.com/science/article/pii/S036054421930177X}{How might residential consumers change their investment decisions over time?}

\item {} 
How might a carbon tax impact investments made in the power sector?

\end{itemize}


\section{How to use MUSE}
\label{\detokenize{overview:how-to-use-muse}}
There are a huge number of ways that MUSE could be used. The energy field is varied and diverse, and many different scenarios can be explored. Users can model the impact of changes in technology prices, demand, policy instruments, sector interactions and much, much more. People are always thinking of new ways that MUSE can be used. So, get creative!

A simulation model of a geographical region or world can be developed and is made up of the following features:
\begin{enumerate}
\sphinxsetlistlabels{\arabic}{enumi}{enumii}{}{.}%
\item {} 
\sphinxstylestrong{Sectors} such as the power sector, gas production sector and the residential sector.

\item {} 
\sphinxstylestrong{Agents} such as a high\sphinxhyphen{}income subsection of the population in the UK or a risk\sphinxhyphen{}averse generation company. These agents are responsible for making investments in energy technologies.

\item {} 
\sphinxstylestrong{Technologies} which the agents choose to adopt. Technologies either produce an energy commodity (e.g. electricity), or a service demand (e.g. building space heating).

\item {} 
\sphinxstylestrong{Service demands} are demands that must be serviced such as lighting, heating or steel production.

\item {} 
\sphinxstylestrong{Market clearing algorithm} is the algorithm which determines global commodity prices based upon the balancing of supply and demand from each of the sectors. It must be noted, however, that only the conversion and supply sectors are able to modify prices; the demand sectors are price\sphinxhyphen{}takers, and so do not modify prices.

\item {} 
\sphinxstylestrong{Equilibrium prices} are the prices determined by the market clearing algorithm and can determine the investments made by agents in various sectors. This allows for the model to project how the system may develop over a time period.

\end{enumerate}

These features are described in more detail in the rest of this documentation.


\section{What are MUSE’s unique features?}
\label{\detokenize{overview:what-are-muse-s-unique-features}}
MUSE is a generalisable agent\sphinxhyphen{}based modelling environment and simulates energy transitions from the point of view of the investor and consumer agents. This means that users can define their own agents based upon their needs. The fact that MUSE is an agent\sphinxhyphen{}based model means that each of these agents can have different investment behaviours.

Additionally, agent\sphinxhyphen{}based models allow for agents to model imperfect information and limited foresight. An example of this is the ability to model the uncertainty residential users face when predicting the price of gas over the next 25 years. This is a unique feature to agent\sphinxhyphen{}based models when compared to intertemporal optimisation models and more closely models the real world. Many energy systems models are intertemporal optimisation models, which consider the viewpoint of a single benevolent decision maker, with perfect foresight and knowledge. These models optimise energy system investment and operation.

Whilst such intertemporal optimisation models are certainly useful, MUSE is different in that it models the incentives and challenges faced by investors. It can, therefore, be used to investigate different research questions, from the point of view of the investor and consumer. These questions are up to you, so impress us!

MUSE is completely open source, and ready for development.


\section{Visualisation of MUSE}
\label{\detokenize{overview:visualisation-of-muse}}
\noindent{\hspace*{\fill}\sphinxincludegraphics[width=550\sphinxpxdimen]{{MUSE-diagram-carbon-budget-colour}.png}\hspace*{\fill}}

The figure above displays the key sectors of MUSE:
\begin{itemize}
\item {} 
Primary supply sectors; this allows to model diurnal peaks in the demand, varying weekly and seasonally.

\item {} 
Conversion sectors

\item {} 
Demand sectors

\item {} 
Climate model (in the current model this is simplified by the use of a carbon budget.)

\item {} 
Market clearing algorithm (MCA)

\end{itemize}


\section{How MUSE works}
\label{\detokenize{overview:how-muse-works}}
MUSE works by iterating between sectors shown above to ensure that energy service demands are met by the technologies chosen by the agents. Next, we detail the calculations made by MUSE throughout the simulation.
\begin{enumerate}
\sphinxsetlistlabels{\arabic}{enumi}{enumii}{}{.}%
\item {} 
The energy service demand is calculated. For example, how much electricity, gas and oil demand is there for cooking, building space heating and lighting in the residential sector? It must be noted, that this is only known after the energy service demand sector is solved and the technologies invested in are decided.

\item {} 
A demand sector is solved. That is, agents choose which end\sphinxhyphen{}use technologies to serve the demands in the sector. For example, electric stoves are compared to gas stoves to meet demand for cooking. These technologies are chosen based upon their:
\begin{enumerate}
\sphinxsetlistlabels{\roman}{enumii}{enumiii}{}{.}%
\item {} 
Search space (which technologies are they willing to consider?)

\item {} 
Their objectives (which metrics do they consider important?)

\item {} 
Their decision rules (how do they choose to combine their metrics if they have multiple?)

\end{enumerate}

\item {} 
The decisions made by the agents in the demand sectors then leads to a certain level of demand for energy commodities, such as electricity, gas and oil, as a whole. This demand is then passed to the MCA.

\item {} 
The MCA then sends these demands to the sectors that supply these energy commodities (supply or conversion sectors).

\item {} 
The supply and conversion sectors are solved: agents in these sectors use the same approach (i.e. search space, objectives, decision rules) to decide which technologies to investment in to serve the energy commodity demand. For example, agents in the power sector may decide to invest in solar photovoltaics, wind turbines and gas power plants to service the electricity demand.

\item {} 
As a result of these decisions a price for each energy commodity is formed based upon supply and demand. This is passed to the MCA.

\item {} 
The MCA then sends these prices back to the demand sectors, which are solved again as above.

\item {} 
This process repeats itself until commodity supply and demand converges for each energy commodity. Once these converge, the model has found a “partial equilibrium” and it moves forward to the next time period.

\end{enumerate}


\chapter{Key MUSE Components}
\label{\detokenize{muse-components:key-muse-components}}\label{\detokenize{muse-components::doc}}
MUSE is made up of five key components:
\begin{itemize}
\item {} 
Service Demand

\item {} 
Technologies

\item {} 
Sectors

\item {} 
Agents

\item {} 
Market Clearing Algorithm

\end{itemize}

In this section we will briefly explore what these components do and and how they interact.


\section{Service Demand}
\label{\detokenize{muse-components:service-demand}}
The service demand is a user input which defines the demand that an end\sphinxhyphen{}use sector has. An example of this is the service demand commodity of heat or cooling that the residential sector requires. End\sphinxhyphen{}use in this case, refers to the energy which is utilised at the very final stage, after both extraction and conversion.

The estimate of the energy service is the first step. This estimate can be an exogenous input derived from the user, or correlations of GDP and population which reflect the socio\sphinxhyphen{}economic development of a region or country.


\section{Technologies}
\label{\detokenize{muse-components:technologies}}
Users are able to define any technology they wish for each of the energy sectors. Examples include power generators such as coal power plants, buses in the transport sector or lighting in the residential sector.

Each of the technologies are placed in their regions of interest, such as the USA or India. They are then defined by the following, but not limited to, variables:
\begin{itemize}
\item {} 
Capital costs

\item {} 
Fixed costs

\item {} 
Maximum capacity limit

\item {} 
Maximum capaicty growth

\item {} 
Lifetime of the technology

\item {} 
Utilization factor

\item {} 
Interest rate

\end{itemize}

Technologies, and their parameters are defined in the Technodata.csv file. For a full description of the input files, please refer to the {\hyperref[\detokenize{inputs/technodata:inputs-technodata}]{\sphinxcrossref{\DUrole{std,std-ref}{Techno\sphinxhyphen{}data}}}} file.


\section{Sectors}
\label{\detokenize{muse-components:sectors}}
Sectors typically group areas of economic activity together, such as the residential sector, which might include all energy conusming activies of households. Possible examples of sectors are:
\begin{itemize}
\item {} 
Gas sector

\item {} 
Power sector

\item {} 
Residential sector

\item {} 
Industrial sector

\end{itemize}

Each of these sectors contain their respective technologies which consume energy commodities. For example, the residential sector may consume electricity, gas or oil for a variety of different energy demands such as lighting, cooking and heating.

Each of the technologies, which consume a commodity, also output a different commodity or service demands. For example, a gas boiler consumes gas, but outputs heat and hot water.


\section{Agents}
\label{\detokenize{muse-components:agents}}
Agents represent the investment decision makers in an energy system, for example consumers or companies. They invest in technologies that meet service demands, like heating, or produce other needed energy commodities, like electricity. These agents can be heterogenous, meaning that their investment priorities have the ability to differ.

As an example, a generation company could compare potential power generators based on their levelized cost of electricity, their net present value, by minimising the total capital cost, a mixture of these and/or any user\sphinxhyphen{}defined approach. This approach more closely matches the behaviour of real\sphinxhyphen{}life agents in the energy market, where companies, or people, have different priorities and constraints.


\section{Market Clearing Algorithm}
\label{\detokenize{muse-components:market-clearing-algorithm}}
The market clearing algorithm (MCA) is the central component between the different supplies and demands of the energy system in question. The MCA iterates between the demand and supply of each of these sectors. Its role is to govern the endogenous price of commodities over the course of a simulation.

For a hypothetical example, the price of electricity is set to \$70/MWh. However, at this price, the majority of residential agents prefer to heat their homes using gas. As a result of this, residential agents consume less electricity and more gas. This reduction in demand reduces the electricity price to \$50/MWh. However, at this lower electricity price, some agents decide to invest in electric heating as opposed to gas. Eventually, the price converges on \$60/MWh, where supply and demand for both electricity and gas are equal.

This is the principle of the MCA. It finds an equilibrium by iterating through each of the different sectors until an overall equilibrium is reached for each of the commodities. It is possible to run the MCA in a carbon budget mode, as well as exogenous mode. The carbon budget mode ensures that a carbon price limits the amount of carbon produce by the market. Whereas, the exogenous mode allows the carbon price to be set by the user.


\chapter{Customising MUSE Tutorials}
\label{\detokenize{user-guide/index:customising-muse-tutorials}}\label{\detokenize{user-guide/index::doc}}
Next, we show you how to customise MUSE to create your own scenarios.

We recommend following the tutorials step by step, as the files build on the previous example. If you prefer to jump straight in, your results may be different to the ones presented. To help you, we have provided the code to generate the various examples, in case you want to compare your code to ours.


\section{Adding a new technology}
\label{\detokenize{user-guide/add-solar:Adding-a-new-technology}}\label{\detokenize{user-guide/add-solar::doc}}

\subsection{Input Files}
\label{\detokenize{user-guide/add-solar:Input-Files}}
MUSE is made up of a number of different {\hyperref[\detokenize{inputs/index::doc}]{\sphinxcrossref{\DUrole{doc}{input files}}}}. These, however, can be broadly split into two:
\begin{itemize}
\item {} 
{\hyperref[\detokenize{inputs/toml::doc}]{\sphinxcrossref{\DUrole{doc}{Simulation settings}}}}

\item {} 
{\hyperref[\detokenize{inputs/inputs_csv::doc}]{\sphinxcrossref{\DUrole{doc}{Simulation data}}}}

\end{itemize}

Simulation settings specify how a simulation should be run. For example, which sectors to run, for how many years and what to output.

Whereas, simulation data parametrises the technologies involved in the simulation, or the number and kinds of agents.

To create a customised case study it is necessary to edit both of these file types.

Simulation settings are specified in a TOML file. \sphinxhref{https://github.com/toml-lang/toml}{TOML} is a simple, extensible and intuitive file format well suited for specifying small sets of complex data.

Simulation data is specified in \sphinxhref{https://en.wikipedia.org/wiki/Comma-separated\_values}{CSV}. This is a common format used for larger datasets, and is made up of columns and rows, with a comma used to differentiate between entries.

MUSE requires at least the following files to successfully run:
\begin{itemize}
\item {} 
a single {\hyperref[\detokenize{inputs/toml::doc}]{\sphinxcrossref{\DUrole{doc}{simulation settings TOML file}}}} for the simulation as a whole

\item {} 
a file indicating initial market price {\hyperref[\detokenize{inputs/projections::doc}]{\sphinxcrossref{\DUrole{doc}{projections}}}}

\item {} 
a file describing the {\hyperref[\detokenize{inputs/commodities::doc}]{\sphinxcrossref{\DUrole{doc}{commodities in the simulation}}}}

\item {} 
for generalized sectors:
\begin{itemize}
\item {} 
a file descring the {\hyperref[\detokenize{inputs/agents::doc}]{\sphinxcrossref{\DUrole{doc}{agents}}}}

\item {} 
a file descring the {\hyperref[\detokenize{inputs/technodata::doc}]{\sphinxcrossref{\DUrole{doc}{technologies}}}}

\item {} 
a file descring the {\hyperref[\detokenize{inputs/commodities_io::doc}]{\sphinxcrossref{\DUrole{doc}{input commodities}}}} for each technology

\item {} 
a file descring the {\hyperref[\detokenize{inputs/commodities_io::doc}]{\sphinxcrossref{\DUrole{doc}{output commodities}}}} for each technology

\item {} 
a file descring the {\hyperref[\detokenize{inputs/existing_capacity::doc}]{\sphinxcrossref{\DUrole{doc}{existing capacity}}}} of a given sector

\end{itemize}

\item {} 
for each preset sector:
\begin{itemize}
\item {} 
a csv file describing consumption for the duration of the simulation

\end{itemize}

\end{itemize}

For a full description of these files see the {\hyperref[\detokenize{inputs/index::doc}]{\sphinxcrossref{\DUrole{doc}{input files section}}}}. To see how to customise an example, continue on this page.


\subsection{Addition of solar PV}
\label{\detokenize{user-guide/add-solar:Addition-of-solar-PV}}
In this section, we will add solar photovoltaics to the default model seen in the {\hyperref[\detokenize{running-muse-example::doc}]{\sphinxcrossref{\DUrole{doc}{example page}}}}. To achieve this, we must modify some of the input files shown in the above section. These files can be found in the \sphinxcode{\sphinxupquote{StarMuse}} folder at the following location:

\sphinxcode{\sphinxupquote{\{muse\_install\_location\}/src/muse/data/example/default}}

Change \sphinxcode{\sphinxupquote{\{muse\_install\_location\}}} to the location where you installed MUSE using your file browser. You can modify the files in your favourite spreadsheet editor or text editor such as Excel, Numbers, Notepad or TextEdit.


\subsubsection{Technodata Input}
\label{\detokenize{user-guide/add-solar:Technodata-Input}}
Within the default folder there is the \sphinxcode{\sphinxupquote{settings.toml}} file, input folder and technodata folder. To add a technology within the power sector, we must open the \sphinxcode{\sphinxupquote{technodata}} folder followed by the \sphinxcode{\sphinxupquote{power}} folder.

Next, we will edit the \sphinxcode{\sphinxupquote{CommIn.csv}} file, which specifies the commodities consumed by solar photovoltaics.

The table below shows the original \sphinxcode{\sphinxupquote{CommIn.csv}} version in normal text, and the added column and row in \sphinxstylestrong{bold}.


\begin{savenotes}\sphinxattablestart
\centering
\begin{tabular}[t]{|*{10}{\X{1}{10}|}}
\hline
\sphinxstyletheadfamily 
ProcessName
&\sphinxstyletheadfamily 
RegionName
&\sphinxstyletheadfamily 
Time
&\sphinxstyletheadfamily 
Level
&\sphinxstyletheadfamily 
electricity
&\sphinxstyletheadfamily 
gas
&\sphinxstyletheadfamily 
heat
&\sphinxstyletheadfamily 
CO2f
&\sphinxstyletheadfamily 
wind
&\sphinxstyletheadfamily 
\sphinxstylestrong{solar}
\\
\hline
Unit
&\begin{itemize}
\item {} 
\end{itemize}
&
Year
&\begin{itemize}
\item {} 
\end{itemize}
&
PJ/PJ
&
PJ/PJ
&
PJ/PJ
&
kt/PJ
&
PJ/PJ
&
\sphinxstylestrong{PJ/PJ}
\\
\hline
gasCCGT
&
R1
&
2020
&
fixed
&
0
&
1.67
&
0
&
0
&
0
&
\sphinxstylestrong{0}
\\
\hline
windturbine
&
R1
&
2020
&
fixed
&
0
&
0
&
0
&
0
&
1
&
\sphinxstylestrong{0}
\\
\hline
\sphinxstylestrong{solarPV}
&
\sphinxstylestrong{R1}
&
\sphinxstylestrong{2020}
&
\sphinxstylestrong{fixed}
&
\sphinxstylestrong{0}
&
\sphinxstylestrong{0}
&
\sphinxstylestrong{0}
&
\sphinxstylestrong{0}
&
\sphinxstylestrong{0}
&
\sphinxstylestrong{1}
\\
\hline
\end{tabular}
\par
\sphinxattableend\end{savenotes}

We must first add a new row at the bottom of the file, to indicate the new solar photovoltaic technology:
\begin{itemize}
\item {} 
we call this technology \sphinxcode{\sphinxupquote{solarPV}}

\item {} 
place it in region \sphinxcode{\sphinxupquote{R1}}

\item {} 
the data in this row is associated to the year 2020

\item {} 
the input type is fixed

\item {} 
solarPV consumes solar

\end{itemize}

As the solar commodity has not been previously defined, we must define it by adding a column, which we will call solar. We fill out the entries in the solar column, ie. that neither \sphinxcode{\sphinxupquote{gasCCGT}} nor \sphinxcode{\sphinxupquote{windturbine}} consume solar.

We repeat this process for the file: \sphinxcode{\sphinxupquote{CommOut.csv}}. This file specifies the output of the technology. In our case, solar photovoltaics only output \sphinxcode{\sphinxupquote{electricity}}. This is unlike \sphinxcode{\sphinxupquote{gasCCGT}} which also outputs \sphinxcode{\sphinxupquote{CO2f}}, or carbon dioxide.


\begin{savenotes}\sphinxattablestart
\centering
\begin{tabular}[t]{|*{10}{\X{1}{10}|}}
\hline
\sphinxstyletheadfamily 
ProcessName
&\sphinxstyletheadfamily 
RegionName
&\sphinxstyletheadfamily 
Time
&\sphinxstyletheadfamily 
Level
&\sphinxstyletheadfamily 
electricity
&\sphinxstyletheadfamily 
gas
&\sphinxstyletheadfamily 
heat
&\sphinxstyletheadfamily 
CO2f
&\sphinxstyletheadfamily 
wind
&\sphinxstyletheadfamily 
\sphinxstylestrong{solar}
\\
\hline
Unit
&\begin{itemize}
\item {} 
\end{itemize}
&
Year
&\begin{itemize}
\item {} 
\end{itemize}
&
PJ/PJ
&
PJ/PJ
&
PJ/PJ
&
kt/PJ
&
PJ/PJ
&
\sphinxstylestrong{PJ/PJ}
\\
\hline
gasCCGT
&
R1
&
2020
&
fixed
&
1
&
0
&
0
&
91.67
&
0
&
\sphinxstylestrong{0}
\\
\hline
windturbine
&
R1
&
2020
&
fixed
&
1
&
0
&
0
&
0
&
0
&
\sphinxstylestrong{0}
\\
\hline
\sphinxstylestrong{solarPV}
&
\sphinxstylestrong{R1}
&
\sphinxstylestrong{2020}
&
\sphinxstylestrong{fixed}
&
\sphinxstylestrong{1}
&
\sphinxstylestrong{0}
&
\sphinxstylestrong{0}
&
\sphinxstylestrong{0}
&
\sphinxstylestrong{0}
&
\sphinxstylestrong{0}
\\
\hline
\end{tabular}
\par
\sphinxattableend\end{savenotes}

Similar to the the \sphinxcode{\sphinxupquote{CommIn.csv}}, we create a new row, and add in the solar commodity. We must ensure that we call our new commodity and technologies the same as the previous file for MUSE to successfully run. ie \sphinxcode{\sphinxupquote{solar}} and \sphinxcode{\sphinxupquote{solarPV}}

The next file to modify is the \sphinxcode{\sphinxupquote{ExistingCapacity.csv}} file. This file details the existing capacity of each technology, per year. For this example, we will set the existing capacity to be 0.


\begin{savenotes}\sphinxattablestart
\centering
\begin{tabulary}{\linewidth}[t]{|T|T|T|T|T|T|T|T|T|T|}
\hline
\sphinxstyletheadfamily 
ProcessName
&\sphinxstyletheadfamily 
RegionName
&\sphinxstyletheadfamily 
Unit
&\sphinxstyletheadfamily 
2020
&\sphinxstyletheadfamily 
2025
&\sphinxstyletheadfamily 
2030
&\sphinxstyletheadfamily 
2035
&\sphinxstyletheadfamily 
2040
&\sphinxstyletheadfamily 
2045
&\sphinxstyletheadfamily 
2050
\\
\hline
gasCCGT
&
R1
&
PJ/y
&
1
&
1
&
0
&
0
&
0
&
0
&
0
\\
\hline
windturbine
&
R1
&
PJ/y
&
0
&
0
&
0
&
0
&
0
&
0
&
0
\\
\hline
\sphinxstylestrong{solarPV}
&
\sphinxstylestrong{R1}
&
\sphinxstylestrong{PJ/y}
&
\sphinxstylestrong{0}
&
\sphinxstylestrong{0}
&
\sphinxstylestrong{0}
&
\sphinxstylestrong{0}
&
\sphinxstylestrong{0}
&
\sphinxstylestrong{0}
&
\sphinxstylestrong{0}
\\
\hline
\end{tabulary}
\par
\sphinxattableend\end{savenotes}

Finally, the \sphinxcode{\sphinxupquote{technodata.csv}} containts parametrisation data for the technology, such as the cost, growth constraints, lifetime of the power plant and fuel used. The technodata file is too long for it all to be displayed here, so we will truncate the full version.

Here, we will only define the parameters: \sphinxcode{\sphinxupquote{processName}}, \sphinxcode{\sphinxupquote{RegionName}}, \sphinxcode{\sphinxupquote{Time}}, \sphinxcode{\sphinxupquote{Level}},\sphinxcode{\sphinxupquote{cap\_par}}, \sphinxcode{\sphinxupquote{Fuel}},\sphinxcode{\sphinxupquote{EndUse}},\sphinxcode{\sphinxupquote{Agent2}} and \sphinxcode{\sphinxupquote{Agent1}}

We shall copy the existing parameters from the \sphinxcode{\sphinxupquote{windturbine}} technology for the remaining parameters that can be seen in the \sphinxcode{\sphinxupquote{technodata.csv}} file for brevity. You can see the full file \sphinxhref{here}{here INSERT LINK HERE}


\begin{savenotes}\sphinxattablestart
\centering
\begin{tabular}[t]{|*{11}{\X{1}{11}|}}
\hline
\sphinxstyletheadfamily 
ProcessName
&\sphinxstyletheadfamily 
RegionName
&\sphinxstyletheadfamily 
Time
&\sphinxstyletheadfamily 
Level
&\sphinxstyletheadfamily 
cap\_par
&\sphinxstyletheadfamily 
cap\_exp
&\sphinxstyletheadfamily 
…
&\sphinxstyletheadfamily 
Fuel
&\sphinxstyletheadfamily 
EndUse
&\sphinxstyletheadfamily 
Agent2
&\sphinxstyletheadfamily 
Agent1
\\
\hline
Unit
&\begin{itemize}
\item {} 
\end{itemize}
&
Year
&\begin{itemize}
\item {} 
\end{itemize}
&
MUS\$2010/PJ\_a
&\begin{itemize}
\item {} 
\end{itemize}
&
…
&\begin{itemize}
\item {} 
\end{itemize}
&\begin{itemize}
\item {} 
\end{itemize}
&
Retrofit
&
New
\\
\hline
gasCCGT
&
R1
&
2020
&
fixed
&
23.78234399
&
1
&
…
&
gas
&
electricity
&
1
&
0
\\
\hline
windturbine
&
R1
&
2020
&
fixed
&
36.30771182
&
1
&
…
&
wind
&
electricity
&
1
&
0
\\
\hline
\sphinxstylestrong{solarPV}
&
\sphinxstylestrong{R1}
&
\sphinxstylestrong{2020}
&
\sphinxstylestrong{fixed}
&
\sphinxstylestrong{30}
&
\sphinxstylestrong{1}
&
…
&
\sphinxstylestrong{solar}
&
\sphinxstylestrong{electricity}
&
\sphinxstylestrong{1}
&
\sphinxstylestrong{0}
\\
\hline
\end{tabular}
\par
\sphinxattableend\end{savenotes}


\subsubsection{Global inputs}
\label{\detokenize{user-guide/add-solar:Global-inputs}}
Next, navigate to the \sphinxcode{\sphinxupquote{input}} folder, found at \sphinxcode{\sphinxupquote{\{muse\_installation\_location\}src/muse/data/example/default/input}}.

We now must edit each of the files found here to add the new solar commodity. Due to space constraints we will not display all of the entries contained in the input files. The edited files can be viewed \sphinxhref{here}{here INSERT LINK HERE} however.

The \sphinxcode{\sphinxupquote{BaseYearExport.csv}} file defines the exports in the base year. For our example we add a column to indicate that there is no export for solar. However, it is important that a column exists for our new commodity.


\begin{savenotes}\sphinxattablestart
\centering
\begin{tabular}[t]{|*{9}{\X{1}{9}|}}
\hline
\sphinxstyletheadfamily 
RegionName
&\sphinxstyletheadfamily 
Attribute
&\sphinxstyletheadfamily 
Time
&\sphinxstyletheadfamily 
electricity
&\sphinxstyletheadfamily 
gas
&\sphinxstyletheadfamily 
heat
&\sphinxstyletheadfamily 
CO2f
&\sphinxstyletheadfamily 
wind
&\sphinxstyletheadfamily 
\sphinxstylestrong{solar}
\\
\hline
Unit
&\begin{itemize}
\item {} 
\end{itemize}
&
Year
&
PJ
&
PJ
&
PJ
&
kt
&
PJ
&
\sphinxstylestrong{PJ}
\\
\hline
R1
&
Exports
&
2010
&
0
&
0
&
0
&
0
&
0
&
\sphinxstylestrong{0}
\\
\hline
R1
&
Exports
&
2015
&
0
&
0
&
0
&
0
&
0
&
\sphinxstylestrong{0}
\\
\hline
…
&
…
&
…
&
…
&
…
&
…
&
…
&
…
&
\sphinxstylestrong{…}
\\
\hline
R1
&
Exports
&
2100
&
0
&
0
&
0
&
0
&
0
&
\sphinxstylestrong{0}
\\
\hline
\end{tabular}
\par
\sphinxattableend\end{savenotes}

The \sphinxcode{\sphinxupquote{BaseYearImport.csv}} file defines the imports in the base year. Similarly to \sphinxcode{\sphinxupquote{BaseYearExport.csv}}, we add a column for solar in the \sphinxcode{\sphinxupquote{BaseYearImport.csv}} file. Again, we indicate that solar has no imports.


\begin{savenotes}\sphinxattablestart
\centering
\begin{tabular}[t]{|*{9}{\X{1}{9}|}}
\hline
\sphinxstyletheadfamily 
RegionName
&\sphinxstyletheadfamily 
Attribute
&\sphinxstyletheadfamily 
Time
&\sphinxstyletheadfamily 
electricity
&\sphinxstyletheadfamily 
gas
&\sphinxstyletheadfamily 
heat
&\sphinxstyletheadfamily 
CO2f
&\sphinxstyletheadfamily 
wind
&\sphinxstyletheadfamily 
\sphinxstylestrong{solar}
\\
\hline
Unit
&\begin{itemize}
\item {} 
\end{itemize}
&
Year
&
PJ
&
PJ
&
PJ
&
kt
&
PJ
&
\sphinxstylestrong{PJ}
\\
\hline
R1
&
Imports
&
2010
&
0
&
0
&
0
&
0
&
0
&
\sphinxstylestrong{0}
\\
\hline
R1
&
Imports
&
2015
&
0
&
0
&
0
&
0
&
0
&
\sphinxstylestrong{0}
\\
\hline
…
&
…
&
…
&
…
&
…
&
…
&
…
&
…
&
\sphinxstylestrong{…}
\\
\hline
R1
&
Imports
&
2100
&
0
&
0
&
0
&
0
&
0
&
\sphinxstylestrong{0}
\\
\hline
\end{tabular}
\par
\sphinxattableend\end{savenotes}

The \sphinxcode{\sphinxupquote{GlobalCommodities.csv}} file is the file which defines the commodities. Here we give the commodities a commodity type, CO2 emissions factor and heat rate. For this file, we will add the solar commodity, with zero CO2 emissions factor and a heat rate of 1.


\begin{savenotes}\sphinxattablestart
\centering
\begin{tabulary}{\linewidth}[t]{|T|T|T|T|T|T|}
\hline
\sphinxstyletheadfamily 
Commodity
&\sphinxstyletheadfamily 
CommodityType
&\sphinxstyletheadfamily 
CommodityName
&\sphinxstyletheadfamily 
CommodityEmissionFactor\_CO2
&\sphinxstyletheadfamily 
HeatRate
&\sphinxstyletheadfamily 
Unit
\\
\hline
Electricity
&
Energy
&
electricity
&
0
&
1
&
PJ
\\
\hline
Gas
&
Energy
&
gas
&
56.1
&
1
&
PJ
\\
\hline
Heat
&
Energy
&
heat
&
0
&
1
&
PJ
\\
\hline
Wind
&
Energy
&
wind
&
0
&
1
&
PJ
\\
\hline
CO2fuelcomsbustion
&
Environmental
&
CO2f
&
0
&
1
&
kt
\\
\hline
\sphinxstylestrong{Solar}
&
\sphinxstylestrong{Energy}
&
\sphinxstylestrong{solar}
&
\sphinxstylestrong{0}
&
\sphinxstylestrong{1}
&
\sphinxstylestrong{PJ}
\\
\hline
\end{tabulary}
\par
\sphinxattableend\end{savenotes}

The \sphinxcode{\sphinxupquote{projections.csv}} file details the initial market prices for the commodities. The market clearing algorithm will update these throughout the simulation, however, an initial estimate is required to start the simulation. As solar energy is free, we will indicate this by adding a final column.


\begin{savenotes}\sphinxattablestart
\centering
\begin{tabular}[t]{|*{9}{\X{1}{9}|}}
\hline
\sphinxstyletheadfamily 
RegionName
&\sphinxstyletheadfamily 
Attribute
&\sphinxstyletheadfamily 
Time
&\sphinxstyletheadfamily 
electricity
&\sphinxstyletheadfamily 
gas
&\sphinxstyletheadfamily 
heat
&\sphinxstyletheadfamily 
CO2f
&\sphinxstyletheadfamily 
wind
&\sphinxstyletheadfamily 
\sphinxstylestrong{solar}
\\
\hline
Unit
&\begin{itemize}
\item {} 
\end{itemize}
&
Year
&
MUS\$2010/PJ
&
MUS\$2010/PJ
&
MUS\$2010/PJ
&
MUS\$2010/kt
&
MUS\$2010/kt
&
\sphinxstylestrong{MUS\$2010/kt}
\\
\hline
R1
&
CommodityPrice
&
2010
&
14.81481472
&
6.6759
&
100
&
0
&
0
&
\sphinxstylestrong{0}
\\
\hline
R1
&
CommodityPrice
&
2015
&
17.89814806
&
6.914325
&
100
&
0.052913851
&
0
&
\sphinxstylestrong{0}
\\
\hline
…
&
…
&
…
&
…
&
…
&
…
&
…
&
…
&
\sphinxstylestrong{…}
\\
\hline
R1
&
CommodityPrice
&
2100
&
21.39814806
&
7.373485819
&
100
&
1.871299697
&
0
&
\sphinxstylestrong{0}
\\
\hline
\end{tabular}
\par
\sphinxattableend\end{savenotes}


\subsubsection{Running our customised simulation}
\label{\detokenize{user-guide/add-solar:Running-our-customised-simulation}}
Now we are able to run our simulation, with the new solar power technology.

To do this we run the same run command as previously in the anaconda command prompt:

\sphinxcode{\sphinxupquote{python \sphinxhyphen{}m muse settings.toml}}

The output should be similar to the output here. However, expect the simulation to take slightly longer to run. This is due to the additional calculations made.

If the simulation has run successfully, you should now have a folder in the same location as your settings.toml file called Results. The next step is to visualise the results using the python visualisation package \sphinxcode{\sphinxupquote{seaborn}} as well as the data analysis library \sphinxcode{\sphinxupquote{pandas}}.

{
\sphinxsetup{VerbatimColor={named}{nbsphinx-code-bg}}
\sphinxsetup{VerbatimBorderColor={named}{nbsphinx-code-border}}
\begin{sphinxVerbatim}[commandchars=\\\{\}]
\llap{\color{nbsphinxin}[2]:\,\hspace{\fboxrule}\hspace{\fboxsep}}\PYG{k+kn}{import} \PYG{n+nn}{seaborn} \PYG{k}{as} \PYG{n+nn}{sns}
\PYG{k+kn}{import} \PYG{n+nn}{pandas} \PYG{k}{as} \PYG{n+nn}{pd}
\end{sphinxVerbatim}
}

Next, we will import the \sphinxcode{\sphinxupquote{MCACapacity.csv}} file into pandas and print the first 5 lines using the \sphinxcode{\sphinxupquote{head()}} command.

Make sure to change the file path of \sphinxcode{\sphinxupquote{"../Results/MCACapacity.csv"}} to where the \sphinxcode{\sphinxupquote{MCACapacity.csv}} is on your computer, otherwise you will receive an error when you import the csv file.

{
\sphinxsetup{VerbatimColor={named}{nbsphinx-code-bg}}
\sphinxsetup{VerbatimBorderColor={named}{nbsphinx-code-border}}
\begin{sphinxVerbatim}[commandchars=\\\{\}]
\llap{\color{nbsphinxin}[6]:\,\hspace{\fboxrule}\hspace{\fboxsep}}\PYG{n}{mca\PYGZus{}capacity} \PYG{o}{=} \PYG{n}{pd}\PYG{o}{.}\PYG{n}{read\PYGZus{}csv}\PYG{p}{(}\PYG{l+s+s2}{\PYGZdq{}}\PYG{l+s+s2}{../Results/MCACapacity.csv}\PYG{l+s+s2}{\PYGZdq{}}\PYG{p}{)}
\PYG{n}{mca\PYGZus{}capacity}\PYG{o}{.}\PYG{n}{head}\PYG{p}{(}\PYG{p}{)}
\end{sphinxVerbatim}
}

{

\kern-\sphinxverbatimsmallskipamount\kern-\baselineskip
\kern+\FrameHeightAdjust\kern-\fboxrule
\vspace{\nbsphinxcodecellspacing}

\sphinxsetup{VerbatimColor={named}{white}}
\sphinxsetup{VerbatimBorderColor={named}{nbsphinx-code-border}}
\begin{sphinxVerbatim}[commandchars=\\\{\}]
\llap{\color{nbsphinxout}[6]:\,\hspace{\fboxrule}\hspace{\fboxsep}}   technology region agent      type       sector  capacity  year
0   gasboiler     R1    A1  retrofit  residential      10.0  2020
1     gasCCGT     R1    A1  retrofit        power       1.0  2020
2  gassupply1     R1    A1  retrofit          gas      15.0  2020
3   gasboiler     R1    A1  retrofit  residential       5.0  2025
4    heatpump     R1    A1  retrofit  residential      19.0  2025
\end{sphinxVerbatim}
}

We will only visualise the power sector in this example, as this was the only sector we changed. We, therefore, filter for this sector, and then visualise it using \sphinxcode{\sphinxupquote{seaborn}}:

{
\sphinxsetup{VerbatimColor={named}{nbsphinx-code-bg}}
\sphinxsetup{VerbatimBorderColor={named}{nbsphinx-code-border}}
\begin{sphinxVerbatim}[commandchars=\\\{\}]
\llap{\color{nbsphinxin}[7]:\,\hspace{\fboxrule}\hspace{\fboxsep}}\PYG{n}{power\PYGZus{}capacity} \PYG{o}{=} \PYG{n}{mca\PYGZus{}capacity}\PYG{p}{[}\PYG{n}{mca\PYGZus{}capacity}\PYG{o}{.}\PYG{n}{sector}\PYG{o}{==}\PYG{l+s+s2}{\PYGZdq{}}\PYG{l+s+s2}{power}\PYG{l+s+s2}{\PYGZdq{}}\PYG{p}{]}
\PYG{n}{sns}\PYG{o}{.}\PYG{n}{lineplot}\PYG{p}{(}\PYG{n}{data}\PYG{o}{=}\PYG{n}{power\PYGZus{}capacity}\PYG{p}{,} \PYG{n}{x}\PYG{o}{=}\PYG{l+s+s1}{\PYGZsq{}}\PYG{l+s+s1}{year}\PYG{l+s+s1}{\PYGZsq{}}\PYG{p}{,} \PYG{n}{y}\PYG{o}{=}\PYG{l+s+s1}{\PYGZsq{}}\PYG{l+s+s1}{capacity}\PYG{l+s+s1}{\PYGZsq{}}\PYG{p}{,} \PYG{n}{hue}\PYG{o}{=}\PYG{l+s+s2}{\PYGZdq{}}\PYG{l+s+s2}{technology}\PYG{l+s+s2}{\PYGZdq{}}\PYG{p}{)}
\end{sphinxVerbatim}
}

{

\kern-\sphinxverbatimsmallskipamount\kern-\baselineskip
\kern+\FrameHeightAdjust\kern-\fboxrule
\vspace{\nbsphinxcodecellspacing}

\sphinxsetup{VerbatimColor={named}{white}}
\sphinxsetup{VerbatimBorderColor={named}{nbsphinx-code-border}}
\begin{sphinxVerbatim}[commandchars=\\\{\}]
\llap{\color{nbsphinxout}[7]:\,\hspace{\fboxrule}\hspace{\fboxsep}}<matplotlib.axes.\_subplots.AxesSubplot at 0x7fd544a6e760>
\end{sphinxVerbatim}
}

\hrule height -\fboxrule\relax
\vspace{\nbsphinxcodecellspacing}

\makeatletter\setbox\nbsphinxpromptbox\box\voidb@x\makeatother

\begin{nbsphinxfancyoutput}

\noindent\sphinxincludegraphics[width=382\sphinxpxdimen,height=262\sphinxpxdimen]{{user-guide_add-solar_25_1}.png}

\end{nbsphinxfancyoutput}

We can now see that there is solarPV in addition to windturbine and gasCCGT, when compared to the example {\hyperref[\detokenize{running-muse-example::doc}]{\sphinxcrossref{\DUrole{doc}{here}}}}! That’s great and means it worked!

The difference in uptake of \sphinxcode{\sphinxupquote{solarPV}} compared to \sphinxcode{\sphinxupquote{windturbine}} is due to the fact that \sphinxcode{\sphinxupquote{solarPV}} has a lower \sphinxcode{\sphinxupquote{cap\_par}} cost of 30, compared to the \sphinxcode{\sphinxupquote{windturbine}}. Meaning that \sphinxcode{\sphinxupquote{solarPV}} outcompetes both \sphinxcode{\sphinxupquote{windturbine}} and \sphinxcode{\sphinxupquote{gasCCGT}} in the electricity market.


\subsubsection{Change Solar Price}
\label{\detokenize{user-guide/add-solar:Change-Solar-Price}}
Now, we will observe what happens if we increase the price of solar to be more expensive than wind. To achieve, this we have to modify the \sphinxcode{\sphinxupquote{Technodata.csv}} file:


\begin{savenotes}\sphinxattablestart
\centering
\begin{tabular}[t]{|*{11}{\X{1}{11}|}}
\hline
\sphinxstyletheadfamily 
ProcessName
&\sphinxstyletheadfamily 
RegionName
&\sphinxstyletheadfamily 
Time
&\sphinxstyletheadfamily 
Level
&\sphinxstyletheadfamily 
cap\_par
&\sphinxstyletheadfamily 
cap\_exp
&\sphinxstyletheadfamily 
…
&\sphinxstyletheadfamily 
Fuel
&\sphinxstyletheadfamily 
EndUse
&\sphinxstyletheadfamily 
Agent2
&\sphinxstyletheadfamily 
Agent1
\\
\hline
Unit
&\begin{itemize}
\item {} 
\end{itemize}
&
Year
&\begin{itemize}
\item {} 
\end{itemize}
&
MUS\$2010/PJ\_a
&\begin{itemize}
\item {} 
\end{itemize}
&
…
&\begin{itemize}
\item {} 
\end{itemize}
&\begin{itemize}
\item {} 
\end{itemize}
&
Retrofit
&
New
\\
\hline
gasCCGT
&
R1
&
2020
&
fixed
&
23.78234399
&
1
&
…
&
gas
&
electricity
&
1
&
0
\\
\hline
windturbine
&
R1
&
2020
&
fixed
&
36.30771182
&
1
&
…
&
wind
&
electricity
&
1
&
0
\\
\hline
solarPV
&
R1
&
2020
&
fixed
&
\sphinxstylestrong{40}
&
1
&
…
&
solar
&
electricity
&
1
&
0
\\
\hline
\end{tabular}
\par
\sphinxattableend\end{savenotes}

Here, we increase the \sphinxcode{\sphinxupquote{cap\_par}} variable by 10, to be a total of 40. We will now rerun the simulation, using the same command as previously and visualise the new results.

We must import the new \sphinxcode{\sphinxupquote{MCACapacity.csv}} file again, and then visualise the results.

{
\sphinxsetup{VerbatimColor={named}{nbsphinx-code-bg}}
\sphinxsetup{VerbatimBorderColor={named}{nbsphinx-code-border}}
\begin{sphinxVerbatim}[commandchars=\\\{\}]
\llap{\color{nbsphinxin}[8]:\,\hspace{\fboxrule}\hspace{\fboxsep}}\PYG{n}{mca\PYGZus{}capacity} \PYG{o}{=} \PYG{n}{pd}\PYG{o}{.}\PYG{n}{read\PYGZus{}csv}\PYG{p}{(}\PYG{l+s+s2}{\PYGZdq{}}\PYG{l+s+s2}{../Results/MCACapacity.csv}\PYG{l+s+s2}{\PYGZdq{}}\PYG{p}{)}
\PYG{n}{power\PYGZus{}capacity} \PYG{o}{=} \PYG{n}{mca\PYGZus{}capacity}\PYG{p}{[}\PYG{n}{mca\PYGZus{}capacity}\PYG{o}{.}\PYG{n}{sector}\PYG{o}{==}\PYG{l+s+s2}{\PYGZdq{}}\PYG{l+s+s2}{power}\PYG{l+s+s2}{\PYGZdq{}}\PYG{p}{]}
\PYG{n}{sns}\PYG{o}{.}\PYG{n}{lineplot}\PYG{p}{(}\PYG{n}{data}\PYG{o}{=}\PYG{n}{power\PYGZus{}capacity}\PYG{p}{,} \PYG{n}{x}\PYG{o}{=}\PYG{l+s+s1}{\PYGZsq{}}\PYG{l+s+s1}{year}\PYG{l+s+s1}{\PYGZsq{}}\PYG{p}{,} \PYG{n}{y}\PYG{o}{=}\PYG{l+s+s1}{\PYGZsq{}}\PYG{l+s+s1}{capacity}\PYG{l+s+s1}{\PYGZsq{}}\PYG{p}{,} \PYG{n}{hue}\PYG{o}{=}\PYG{l+s+s2}{\PYGZdq{}}\PYG{l+s+s2}{technology}\PYG{l+s+s2}{\PYGZdq{}}\PYG{p}{)}
\end{sphinxVerbatim}
}

{

\kern-\sphinxverbatimsmallskipamount\kern-\baselineskip
\kern+\FrameHeightAdjust\kern-\fboxrule
\vspace{\nbsphinxcodecellspacing}

\sphinxsetup{VerbatimColor={named}{white}}
\sphinxsetup{VerbatimBorderColor={named}{nbsphinx-code-border}}
\begin{sphinxVerbatim}[commandchars=\\\{\}]
\llap{\color{nbsphinxout}[8]:\,\hspace{\fboxrule}\hspace{\fboxsep}}<matplotlib.axes.\_subplots.AxesSubplot at 0x7fd544ef7ee0>
\end{sphinxVerbatim}
}

\hrule height -\fboxrule\relax
\vspace{\nbsphinxcodecellspacing}

\makeatletter\setbox\nbsphinxpromptbox\box\voidb@x\makeatother

\begin{nbsphinxfancyoutput}

\noindent\sphinxincludegraphics[width=382\sphinxpxdimen,height=262\sphinxpxdimen]{{user-guide_add-solar_30_1}.png}

\end{nbsphinxfancyoutput}

Now, we can see that the technology \sphinxcode{\sphinxupquote{windturbine}} outcompetes \sphinxcode{\sphinxupquote{solarPV}} and \sphinxcode{\sphinxupquote{gasCCGT}} due to the difference in price. The possibilities for creating your own scenarios are infinite.

For the full example with the completed input files see \sphinxhref{dead-link}{here INSERT LINK HERE}


\subsection{Next steps}
\label{\detokenize{user-guide/add-solar:Next-steps}}
In the next section we will add a new agent to the simulation.


\section{Adding an agent}
\label{\detokenize{user-guide/add-agent:Adding-an-agent}}\label{\detokenize{user-guide/add-agent::doc}}
In this section, we will add a new agent called \sphinxcode{\sphinxupquote{A2}}. This agent will be slightly different to the other agents in the \sphinxcode{\sphinxupquote{default}} example, in that it will make investments based upon a mixture of \sphinxhref{https://en.wikipedia.org/wiki/Levelized\_cost\_of\_energy}{levelised cost of electricity (LCOE)} and \sphinxhref{https://en.wikipedia.org/wiki/Net\_present\_value}{net present value (NPV)}. These two objectives will be combined by calculating the mean of the two when comparing potential investment options.

To achieve this, we must modify the \sphinxcode{\sphinxupquote{Agents.csv}} file in the directory:

\begin{sphinxVerbatim}[commandchars=\\\{\}]
\PYG{p}{\PYGZob{}}\PYG{n}{muse\PYGZus{}install\PYGZus{}location}\PYG{p}{\PYGZcb{}}\PYG{o}{/}\PYG{n}{src}\PYG{o}{/}\PYG{n}{muse}\PYG{o}{/}\PYG{n}{data}\PYG{o}{/}\PYG{n}{example}\PYG{o}{/}\PYG{n}{default}\PYG{o}{/}\PYG{n}{technodata}\PYG{o}{/}\PYG{n}{Agents}\PYG{o}{.}\PYG{n}{csv}
\end{sphinxVerbatim}

To do this, we will add two new rows to the file. To simplify the process, we copy the data from the first two rows of agent \sphinxcode{\sphinxupquote{A1}}, changing only the rows: \sphinxcode{\sphinxupquote{Name}}, \sphinxcode{\sphinxupquote{Objective1}}, \sphinxcode{\sphinxupquote{Objective2}}, \sphinxcode{\sphinxupquote{ObjData1}}, \sphinxcode{\sphinxupquote{ObjData2}} and \sphinxcode{\sphinxupquote{DecisionMethod}}. The values we changed can be seen below. Again, we only show some of the rows due to space constraints, however see \sphinxhref{broken-link}{here} for the full file.


\begin{savenotes}\sphinxattablestart
\centering
\begin{tabulary}{\linewidth}[t]{|T|T|T|T|T|T|T|T|T|T|T|T|T|}
\hline
\sphinxstyletheadfamily 
AgentShare
&\sphinxstyletheadfamily 
Name
&\sphinxstyletheadfamily 
AgentNumber
&\sphinxstyletheadfamily 
RegionName
&\sphinxstyletheadfamily 
Objective1
&\sphinxstyletheadfamily 
Objective2
&\sphinxstyletheadfamily 
Objective3
&\sphinxstyletheadfamily 
ObjData1
&\sphinxstyletheadfamily 
ObjData2
&\sphinxstyletheadfamily 
…
&\sphinxstyletheadfamily 
DecisionMethod
&\sphinxstyletheadfamily 
…
&\sphinxstyletheadfamily 
Type
\\
\hline
Agent1
&
A1
&
1
&
R1
&
LCOE
&&&
1
&&
…
&
singleObj
&
…
&
New
\\
\hline
Agent2
&
A1
&
2
&
R1
&
LCOE
&&&
1
&&
…
&
singleObj
&
…
&
Retrofit
\\
\hline
\sphinxstylestrong{Agent1}
&
\sphinxstylestrong{A2}
&
\sphinxstylestrong{1}
&
\sphinxstylestrong{R1}
&
\sphinxstylestrong{LCOE}
&
\sphinxstylestrong{NPV}
&&
\sphinxstylestrong{1}
&
\sphinxstylestrong{1}
&
\sphinxstylestrong{…}
&
\sphinxstylestrong{mean}
&
\sphinxstylestrong{…}
&
\sphinxstylestrong{New}
\\
\hline
\sphinxstylestrong{Agent2}
&
\sphinxstylestrong{A2}
&
\sphinxstylestrong{2}
&
\sphinxstylestrong{R1}
&
\sphinxstylestrong{LCOE}
&
\sphinxstylestrong{NPV}
&&
\sphinxstylestrong{1}
&
\sphinxstylestrong{1}
&
\sphinxstylestrong{…}
&
\sphinxstylestrong{mean}
&
\sphinxstylestrong{…}
&
\sphinxstylestrong{Retrofit}
\\
\hline
\end{tabulary}
\par
\sphinxattableend\end{savenotes}

We will now save this file and run the new simulation model using the following command:

\begin{sphinxVerbatim}[commandchars=\\\{\}]
\PYG{n}{python} \PYG{o}{\PYGZhy{}}\PYG{n}{m} \PYG{n}{muse} \PYG{n}{settings}\PYG{o}{.}\PYG{n}{toml}
\end{sphinxVerbatim}

Again, we use seaborn and pandas to analyse the data in the \sphinxcode{\sphinxupquote{Results}} folder.

{
\sphinxsetup{VerbatimColor={named}{nbsphinx-code-bg}}
\sphinxsetup{VerbatimBorderColor={named}{nbsphinx-code-border}}
\begin{sphinxVerbatim}[commandchars=\\\{\}]
\llap{\color{nbsphinxin}[6]:\,\hspace{\fboxrule}\hspace{\fboxsep}}\PYG{k+kn}{import} \PYG{n+nn}{pandas} \PYG{k}{as} \PYG{n+nn}{pd}
\PYG{k+kn}{import} \PYG{n+nn}{seaborn} \PYG{k}{as} \PYG{n+nn}{sns}
\end{sphinxVerbatim}
}

{
\sphinxsetup{VerbatimColor={named}{nbsphinx-code-bg}}
\sphinxsetup{VerbatimBorderColor={named}{nbsphinx-code-border}}
\begin{sphinxVerbatim}[commandchars=\\\{\}]
\llap{\color{nbsphinxin}[19]:\,\hspace{\fboxrule}\hspace{\fboxsep}}\PYG{n}{mca\PYGZus{}capacity} \PYG{o}{=} \PYG{n}{pd}\PYG{o}{.}\PYG{n}{read\PYGZus{}csv}\PYG{p}{(}\PYG{l+s+s2}{\PYGZdq{}}\PYG{l+s+s2}{../Results/MCACapacity.csv}\PYG{l+s+s2}{\PYGZdq{}}\PYG{p}{)}
\PYG{n}{power\PYGZus{}sector} \PYG{o}{=} \PYG{n}{mca\PYGZus{}capacity}\PYG{p}{[}\PYG{n}{mca\PYGZus{}capacity}\PYG{o}{.}\PYG{n}{sector}\PYG{o}{==}\PYG{l+s+s2}{\PYGZdq{}}\PYG{l+s+s2}{power}\PYG{l+s+s2}{\PYGZdq{}}\PYG{p}{]}
\PYG{n}{power\PYGZus{}sector}\PYG{o}{.}\PYG{n}{head}\PYG{p}{(}\PYG{p}{)}
\end{sphinxVerbatim}
}

{

\kern-\sphinxverbatimsmallskipamount\kern-\baselineskip
\kern+\FrameHeightAdjust\kern-\fboxrule
\vspace{\nbsphinxcodecellspacing}

\sphinxsetup{VerbatimColor={named}{white}}
\sphinxsetup{VerbatimBorderColor={named}{nbsphinx-code-border}}
\begin{sphinxVerbatim}[commandchars=\\\{\}]
\llap{\color{nbsphinxout}[19]:\,\hspace{\fboxrule}\hspace{\fboxsep}}     technology region agent      type sector  capacity  year
2       gasCCGT     R1    A1  retrofit  power     1.000  2020
3       gasCCGT     R1    A2  retrofit  power     1.000  2020
10      gasCCGT     R1    A1  retrofit  power     1.000  2025
11  windturbine     R1    A1  retrofit  power     5.172  2025
12      gasCCGT     R1    A2  retrofit  power    11.000  2025
\end{sphinxVerbatim}
}

This time we can see that there is data for the new agent, \sphinxcode{\sphinxupquote{A2}}. Next, we will visualise the investments made by each of the agents using seaborn’s facetgrid command.

{
\sphinxsetup{VerbatimColor={named}{nbsphinx-code-bg}}
\sphinxsetup{VerbatimBorderColor={named}{nbsphinx-code-border}}
\begin{sphinxVerbatim}[commandchars=\\\{\}]
\llap{\color{nbsphinxin}[20]:\,\hspace{\fboxrule}\hspace{\fboxsep}}\PYG{n}{g}\PYG{o}{=}\PYG{n}{sns}\PYG{o}{.}\PYG{n}{FacetGrid}\PYG{p}{(}\PYG{n}{power\PYGZus{}sector}\PYG{p}{,} \PYG{n}{row}\PYG{o}{=}\PYG{l+s+s1}{\PYGZsq{}}\PYG{l+s+s1}{agent}\PYG{l+s+s1}{\PYGZsq{}}\PYG{p}{)}
\PYG{n}{g}\PYG{o}{.}\PYG{n}{map}\PYG{p}{(}\PYG{n}{sns}\PYG{o}{.}\PYG{n}{lineplot}\PYG{p}{,} \PYG{l+s+s2}{\PYGZdq{}}\PYG{l+s+s2}{year}\PYG{l+s+s2}{\PYGZdq{}}\PYG{p}{,} \PYG{l+s+s2}{\PYGZdq{}}\PYG{l+s+s2}{capacity}\PYG{l+s+s2}{\PYGZdq{}}\PYG{p}{,} \PYG{l+s+s2}{\PYGZdq{}}\PYG{l+s+s2}{technology}\PYG{l+s+s2}{\PYGZdq{}}\PYG{p}{)}
\PYG{n}{g}\PYG{o}{.}\PYG{n}{add\PYGZus{}legend}\PYG{p}{(}\PYG{p}{)}
\end{sphinxVerbatim}
}

{

\kern-\sphinxverbatimsmallskipamount\kern-\baselineskip
\kern+\FrameHeightAdjust\kern-\fboxrule
\vspace{\nbsphinxcodecellspacing}

\sphinxsetup{VerbatimColor={named}{white}}
\sphinxsetup{VerbatimBorderColor={named}{nbsphinx-code-border}}
\begin{sphinxVerbatim}[commandchars=\\\{\}]
\llap{\color{nbsphinxout}[20]:\,\hspace{\fboxrule}\hspace{\fboxsep}}<seaborn.axisgrid.FacetGrid at 0x7f95819b4730>
\end{sphinxVerbatim}
}

\hrule height -\fboxrule\relax
\vspace{\nbsphinxcodecellspacing}

\makeatletter\setbox\nbsphinxpromptbox\box\voidb@x\makeatother

\begin{nbsphinxfancyoutput}

\noindent\sphinxincludegraphics[width=290\sphinxpxdimen,height=424\sphinxpxdimen]{{user-guide_add-agent_7_1}.png}

\end{nbsphinxfancyoutput}

In this scenario, agent \sphinxcode{\sphinxupquote{A1}} is investing using LCOE, whereas agent \sphinxcode{\sphinxupquote{A2}} is investing based on the mean of the objectives: LCOE and NPV in the same region. A different strategy is employed by these agents with \sphinxcode{\sphinxupquote{A2}} investing in gasCCGT and windturbines, whereas \sphinxcode{\sphinxupquote{A1}} invests in solarPV.

Next, we will see what occurs if the agents invest based upon the same investment strategy, with both investing using NPV. This requires to edit the \sphinxcode{\sphinxupquote{Agents.csv}} file once more, to look like the following:


\begin{savenotes}\sphinxattablestart
\centering
\begin{tabulary}{\linewidth}[t]{|T|T|T|T|T|T|T|T|T|T|T|T|T|}
\hline
\sphinxstyletheadfamily 
AgentShare
&\sphinxstyletheadfamily 
Name
&\sphinxstyletheadfamily 
AgentNumber
&\sphinxstyletheadfamily 
RegionName
&\sphinxstyletheadfamily 
Objective1
&\sphinxstyletheadfamily 
Objective2
&\sphinxstyletheadfamily 
Objective3
&\sphinxstyletheadfamily 
ObjData1
&\sphinxstyletheadfamily 
ObjData2
&\sphinxstyletheadfamily 
…
&\sphinxstyletheadfamily 
DecisionMethod
&\sphinxstyletheadfamily 
…
&\sphinxstyletheadfamily 
Type
\\
\hline
Agent1
&
A1
&
1
&
R1
&
LCOE
&&&
1
&&
…
&
singleObj
&
…
&
New
\\
\hline
Agent2
&
A1
&
2
&
R1
&
LCOE
&&&
1
&&
…
&
singleObj
&
…
&
Retrofit
\\
\hline
Agent1
&
A2
&
1
&
R1
&
\sphinxstylestrong{LCOE}
&&&
\sphinxstylestrong{1}
&&
…
&
\sphinxstylestrong{singleObj}
&
…
&
New
\\
\hline
Agent2
&
A2
&
2
&
R1
&
\sphinxstylestrong{LCOE}
&&&
\sphinxstylestrong{1}
&&
…
&
\sphinxstylestrong{singleObj}
&
…
&
Retrofit
\\
\hline
\end{tabulary}
\par
\sphinxattableend\end{savenotes}

Again, this requires the re\sphinxhyphen{}running of the simulation, and visualisation like before:

{
\sphinxsetup{VerbatimColor={named}{nbsphinx-code-bg}}
\sphinxsetup{VerbatimBorderColor={named}{nbsphinx-code-border}}
\begin{sphinxVerbatim}[commandchars=\\\{\}]
\llap{\color{nbsphinxin}[21]:\,\hspace{\fboxrule}\hspace{\fboxsep}}\PYG{n}{mca\PYGZus{}capacity} \PYG{o}{=} \PYG{n}{pd}\PYG{o}{.}\PYG{n}{read\PYGZus{}csv}\PYG{p}{(}\PYG{l+s+s2}{\PYGZdq{}}\PYG{l+s+s2}{../Results/MCACapacity.csv}\PYG{l+s+s2}{\PYGZdq{}}\PYG{p}{)}
\PYG{n}{power\PYGZus{}sector} \PYG{o}{=} \PYG{n}{mca\PYGZus{}capacity}\PYG{p}{[}\PYG{n}{mca\PYGZus{}capacity}\PYG{o}{.}\PYG{n}{sector}\PYG{o}{==}\PYG{l+s+s2}{\PYGZdq{}}\PYG{l+s+s2}{power}\PYG{l+s+s2}{\PYGZdq{}}\PYG{p}{]}
\PYG{n}{g}\PYG{o}{=}\PYG{n}{sns}\PYG{o}{.}\PYG{n}{FacetGrid}\PYG{p}{(}\PYG{n}{power\PYGZus{}sector}\PYG{p}{,} \PYG{n}{row}\PYG{o}{=}\PYG{l+s+s1}{\PYGZsq{}}\PYG{l+s+s1}{agent}\PYG{l+s+s1}{\PYGZsq{}}\PYG{p}{)}
\PYG{n}{g}\PYG{o}{.}\PYG{n}{map}\PYG{p}{(}\PYG{n}{sns}\PYG{o}{.}\PYG{n}{lineplot}\PYG{p}{,} \PYG{l+s+s2}{\PYGZdq{}}\PYG{l+s+s2}{year}\PYG{l+s+s2}{\PYGZdq{}}\PYG{p}{,} \PYG{l+s+s2}{\PYGZdq{}}\PYG{l+s+s2}{capacity}\PYG{l+s+s2}{\PYGZdq{}}\PYG{p}{,} \PYG{l+s+s2}{\PYGZdq{}}\PYG{l+s+s2}{technology}\PYG{l+s+s2}{\PYGZdq{}}\PYG{p}{)}
\PYG{n}{g}\PYG{o}{.}\PYG{n}{add\PYGZus{}legend}\PYG{p}{(}\PYG{p}{)}
\end{sphinxVerbatim}
}

{

\kern-\sphinxverbatimsmallskipamount\kern-\baselineskip
\kern+\FrameHeightAdjust\kern-\fboxrule
\vspace{\nbsphinxcodecellspacing}

\sphinxsetup{VerbatimColor={named}{white}}
\sphinxsetup{VerbatimBorderColor={named}{nbsphinx-code-border}}
\begin{sphinxVerbatim}[commandchars=\\\{\}]
\llap{\color{nbsphinxout}[21]:\,\hspace{\fboxrule}\hspace{\fboxsep}}<seaborn.axisgrid.FacetGrid at 0x7f957e5f0970>
\end{sphinxVerbatim}
}

\hrule height -\fboxrule\relax
\vspace{\nbsphinxcodecellspacing}

\makeatletter\setbox\nbsphinxpromptbox\box\voidb@x\makeatother

\begin{nbsphinxfancyoutput}

\noindent\sphinxincludegraphics[width=290\sphinxpxdimen,height=424\sphinxpxdimen]{{user-guide_add-agent_11_1}.png}

\end{nbsphinxfancyoutput}

In this new scenario, with both agents running the same objective, very similar results can be seen, with a high investmwent in \sphinxcode{\sphinxupquote{windturbine}}, none in \sphinxcode{\sphinxupquote{solarPV}} and low \sphinxcode{\sphinxupquote{gasCCGT}}. Have a play around with the files to see if you can come up with different scenarios!


\subsection{Next steps}
\label{\detokenize{user-guide/add-agent:Next-steps}}
In the next section we will show you how to add a new region.


\section{Adding a region}
\label{\detokenize{user-guide/add-region:Adding-a-region}}\label{\detokenize{user-guide/add-region::doc}}
The next step is to add a region which we will call \sphinxcode{\sphinxupquote{R2}}, however, this could equally be called \sphinxcode{\sphinxupquote{USA}} or \sphinxcode{\sphinxupquote{India}}. This requires a similar process to before of modifying the input simulation data. However, we will also have to change the \sphinxcode{\sphinxupquote{settings.toml}} file to achieve this.

The process to change the \sphinxcode{\sphinxupquote{settings.toml}} file is relatively simple. We just have to add our new region to the \sphinxcode{\sphinxupquote{regions}} variable, in the 4th line of the \sphinxcode{\sphinxupquote{settings.toml}} file, like so:

\begin{sphinxVerbatim}[commandchars=\\\{\}]
\PYG{n}{regions} \PYG{o}{=} \PYG{p}{[}\PYG{l+s+s2}{\PYGZdq{}}\PYG{l+s+s2}{R1}\PYG{l+s+s2}{\PYGZdq{}}\PYG{p}{,} \PYG{l+s+s2}{\PYGZdq{}}\PYG{l+s+s2}{R2}\PYG{l+s+s2}{\PYGZdq{}}\PYG{p}{]}
\end{sphinxVerbatim}

The process to change the input files, however, takes a bit more time. To achieve this, there must be data for each of the sectors for the new region. This, therefore, requires the modification of every {\hyperref[\detokenize{inputs/index::doc}]{\sphinxcrossref{\DUrole{doc}{input file}}}}.

Due to space constraints, we will not show you how to edit all of the files. However, you can access the modified files \sphinxhref{github-link}{here INSERT LINK HERE}.

Effectively, for this example, we will copy and paste the results for each of the input files from region \sphinxcode{\sphinxupquote{R1}}, and change the name of the region for the new rows to \sphinxcode{\sphinxupquote{R2}}.

However, as we are increasing the demand by adding a region, as well as modifying the costs of technologies, it may be the case that a higher growth in technology is required. For example, there may be no possible solution to meet demand without increasing the \sphinxcode{\sphinxupquote{windturbine}} maximum allowed limit. We will therefore increase the allowed limits for \sphinxcode{\sphinxupquote{windturbine}} in region \sphinxcode{\sphinxupquote{R2}}.

We have placed two examples as to how to edit the residential sector below. Again, the edited data are highlighted in \sphinxstylestrong{bold}, with the original data in normal text.

The following file is the modified \sphinxcode{\sphinxupquote{/technodata/residential/CommIn.csv}} file:


\begin{savenotes}\sphinxattablestart
\centering
\begin{tabular}[t]{|*{9}{\X{1}{9}|}}
\hline
\sphinxstyletheadfamily 
ProcessName
&\sphinxstyletheadfamily 
RegionName
&\sphinxstyletheadfamily 
Time
&\sphinxstyletheadfamily 
Level
&\sphinxstyletheadfamily 
electricity
&\sphinxstyletheadfamily 
gas
&\sphinxstyletheadfamily 
heat
&\sphinxstyletheadfamily 
CO2f
&\sphinxstyletheadfamily 
wind
\\
\hline
Unit
&\begin{itemize}
\item {} 
\end{itemize}
&
Year
&\begin{itemize}
\item {} 
\end{itemize}
&
PJ/PJ
&
PJ/PJ
&
PJ/PJ
&
kt/PJ
&
PJ/PJ
\\
\hline
gasboiler
&
R1
&
2020
&
fixed
&
0
&
1.16
&
0
&
0
&
0
\\
\hline
heatpump
&
R1
&
2020
&
fixed
&
0.4
&
0
&
0
&
0
&
0
\\
\hline
\sphinxstylestrong{gasboiler}
&
\sphinxstylestrong{R2}
&
\sphinxstylestrong{2020}
&
\sphinxstylestrong{fixed}
&
\sphinxstylestrong{0}
&
\sphinxstylestrong{1.16}
&
\sphinxstylestrong{0}
&
\sphinxstylestrong{0}
&
\sphinxstylestrong{0}
\\
\hline
\sphinxstylestrong{heatpump}
&
\sphinxstylestrong{R2}
&
\sphinxstylestrong{2020}
&
\sphinxstylestrong{fixed}
&
\sphinxstylestrong{0.4}
&
\sphinxstylestrong{0}
&
\sphinxstylestrong{0}
&
\sphinxstylestrong{0}
&
\sphinxstylestrong{0}
\\
\hline
\end{tabular}
\par
\sphinxattableend\end{savenotes}

Whereas the following file is the modified \sphinxcode{\sphinxupquote{/technodata/residential/ExistingCapacity.csv}} file:


\begin{savenotes}\sphinxattablestart
\centering
\begin{tabulary}{\linewidth}[t]{|T|T|T|T|T|T|T|T|T|T|}
\hline
\sphinxstyletheadfamily 
ProcessName
&\sphinxstyletheadfamily 
RegionName
&\sphinxstyletheadfamily 
Unit
&\sphinxstyletheadfamily 
2020
&\sphinxstyletheadfamily 
2025
&\sphinxstyletheadfamily 
2030
&\sphinxstyletheadfamily 
2035
&\sphinxstyletheadfamily 
2040
&\sphinxstyletheadfamily 
2045
&\sphinxstyletheadfamily 
2050
\\
\hline
gasboiler
&
R1
&
PJ/y
&
10
&
5
&
0
&
0
&
0
&
0
&
0
\\
\hline
heatpump
&
R1
&
PJ/y
&
0
&
0
&
0
&
0
&
0
&
0
&
0
\\
\hline
\sphinxstylestrong{gasboiler}
&
\sphinxstylestrong{R2}
&
\sphinxstylestrong{PJ/y}
&
\sphinxstylestrong{10}
&
\sphinxstylestrong{5}
&
\sphinxstylestrong{0}
&
\sphinxstylestrong{0}
&
\sphinxstylestrong{0}
&
\sphinxstylestrong{0}
&
\sphinxstylestrong{0}
\\
\hline
\sphinxstylestrong{heatpump}
&
\sphinxstylestrong{R2}
&
\sphinxstylestrong{PJ/y}
&
\sphinxstylestrong{0}
&
\sphinxstylestrong{0}
&
\sphinxstylestrong{0}
&
\sphinxstylestrong{0}
&
\sphinxstylestrong{0}
&
\sphinxstylestrong{0}
&
\sphinxstylestrong{0}
\\
\hline
\end{tabulary}
\par
\sphinxattableend\end{savenotes}

Below is the reduced \sphinxcode{\sphinxupquote{/technodata/power/technodata.csv}} file, showing the increased capacity for \sphinxcode{\sphinxupquote{windturbine}} in \sphinxcode{\sphinxupquote{R2}}. For this, we highlight only the elements we changed from the rows in \sphinxcode{\sphinxupquote{R1}}. The rest of the elements are the same for \sphinxcode{\sphinxupquote{R1}} as they are for \sphinxcode{\sphinxupquote{R2}}.


\begin{savenotes}\sphinxattablestart
\centering
\begin{tabular}[t]{|*{9}{\X{1}{9}|}}
\hline
\sphinxstyletheadfamily 
ProcessName
&\sphinxstyletheadfamily 
RegionName
&\sphinxstyletheadfamily 
…
&\sphinxstyletheadfamily 
MaxCapacityAddition
&\sphinxstyletheadfamily 
MaxCapacityGrowth
&\sphinxstyletheadfamily 
TotalCapacityLimit
&\sphinxstyletheadfamily 
…
&\sphinxstyletheadfamily 
Agent2
&\sphinxstyletheadfamily 
Agent1
\\
\hline
Unit
&\begin{itemize}
\item {} 
\end{itemize}
&
…
&
PJ
&
\%
&
PJ
&
…
&
Retrofit
&
New
\\
\hline
gasCCGT
&
R1
&
…
&
2
&
0.02
&
60
&
…
&
1
&
0
\\
\hline
windturbine
&
R1
&
…
&
2
&
0.02
&
60
&
…
&
1
&
0
\\
\hline
solarPV
&
R1
&
…
&
2
&
0.02
&
60
&
…
&
1
&
0
\\
\hline
gasCCGT
&
R2
&
…
&
2
&
0.02
&
60
&
…
&
1
&
0
\\
\hline
windturbine
&
R2
&
…
&
\sphinxstylestrong{5}
&
\sphinxstylestrong{0.05}
&
\sphinxstylestrong{100}
&
…
&
1
&
0
\\
\hline
solarPV
&
R2
&
…
&
2
&
0.02
&
60
&
…
&
1
&
0
\\
\hline
\end{tabular}
\par
\sphinxattableend\end{savenotes}

Now, go ahead and amend all of the other input files for each of the sectors by copying and pasting the rows from \sphinxcode{\sphinxupquote{R1}} and replacing the \sphinxcode{\sphinxupquote{RegionName}} to \sphinxcode{\sphinxupquote{R2}} for the new rows. All of the edited input files can be seen \sphinxhref{dead-link}{here}.

Again, we will run the results using the \sphinxcode{\sphinxupquote{python \sphinxhyphen{}m pip muse settings.toml}} in anaconda prompt, and analyse the data as follows:

{
\sphinxsetup{VerbatimColor={named}{nbsphinx-code-bg}}
\sphinxsetup{VerbatimBorderColor={named}{nbsphinx-code-border}}
\begin{sphinxVerbatim}[commandchars=\\\{\}]
\llap{\color{nbsphinxin}[1]:\,\hspace{\fboxrule}\hspace{\fboxsep}}\PYG{k+kn}{import} \PYG{n+nn}{seaborn} \PYG{k}{as} \PYG{n+nn}{sns}
\PYG{k+kn}{import} \PYG{n+nn}{pandas} \PYG{k}{as} \PYG{n+nn}{pd}
\PYG{k+kn}{import} \PYG{n+nn}{matplotlib}\PYG{n+nn}{.}\PYG{n+nn}{pyplot} \PYG{k}{as} \PYG{n+nn}{plt}
\end{sphinxVerbatim}
}

{
\sphinxsetup{VerbatimColor={named}{nbsphinx-code-bg}}
\sphinxsetup{VerbatimBorderColor={named}{nbsphinx-code-border}}
\begin{sphinxVerbatim}[commandchars=\\\{\}]
\llap{\color{nbsphinxin}[2]:\,\hspace{\fboxrule}\hspace{\fboxsep}}\PYG{n}{mca\PYGZus{}capacity} \PYG{o}{=} \PYG{n}{pd}\PYG{o}{.}\PYG{n}{read\PYGZus{}csv}\PYG{p}{(}\PYG{l+s+s2}{\PYGZdq{}}\PYG{l+s+s2}{../tutorial\PYGZhy{}code/add\PYGZhy{}region/Results/MCACapacity.csv}\PYG{l+s+s2}{\PYGZdq{}}\PYG{p}{)}

\PYG{k}{for} \PYG{n}{name}\PYG{p}{,} \PYG{n}{sector} \PYG{o+ow}{in} \PYG{n}{mca\PYGZus{}capacity}\PYG{o}{.}\PYG{n}{groupby}\PYG{p}{(}\PYG{l+s+s2}{\PYGZdq{}}\PYG{l+s+s2}{sector}\PYG{l+s+s2}{\PYGZdq{}}\PYG{p}{)}\PYG{p}{:}
    \PYG{n+nb}{print}\PYG{p}{(}\PYG{l+s+s2}{\PYGZdq{}}\PYG{l+s+si}{\PYGZob{}\PYGZcb{}}\PYG{l+s+s2}{ sector:}\PYG{l+s+s2}{\PYGZdq{}}\PYG{o}{.}\PYG{n}{format}\PYG{p}{(}\PYG{n}{name}\PYG{p}{)}\PYG{p}{)}
    \PYG{n}{g} \PYG{o}{=} \PYG{n}{sns}\PYG{o}{.}\PYG{n}{FacetGrid}\PYG{p}{(}\PYG{n}{data}\PYG{o}{=}\PYG{n}{sector}\PYG{p}{,} \PYG{n}{col}\PYG{o}{=}\PYG{l+s+s2}{\PYGZdq{}}\PYG{l+s+s2}{region}\PYG{l+s+s2}{\PYGZdq{}}\PYG{p}{)}
    \PYG{n}{g}\PYG{o}{.}\PYG{n}{map}\PYG{p}{(}\PYG{n}{sns}\PYG{o}{.}\PYG{n}{lineplot}\PYG{p}{,} \PYG{l+s+s2}{\PYGZdq{}}\PYG{l+s+s2}{year}\PYG{l+s+s2}{\PYGZdq{}}\PYG{p}{,} \PYG{l+s+s2}{\PYGZdq{}}\PYG{l+s+s2}{capacity}\PYG{l+s+s2}{\PYGZdq{}}\PYG{p}{,} \PYG{l+s+s2}{\PYGZdq{}}\PYG{l+s+s2}{technology}\PYG{l+s+s2}{\PYGZdq{}}\PYG{p}{)}
    \PYG{n}{g}\PYG{o}{.}\PYG{n}{add\PYGZus{}legend}\PYG{p}{(}\PYG{p}{)}
    \PYG{n}{plt}\PYG{o}{.}\PYG{n}{show}\PYG{p}{(}\PYG{p}{)}
\end{sphinxVerbatim}
}

{

\kern-\sphinxverbatimsmallskipamount\kern-\baselineskip
\kern+\FrameHeightAdjust\kern-\fboxrule
\vspace{\nbsphinxcodecellspacing}

\sphinxsetup{VerbatimColor={named}{white}}
\sphinxsetup{VerbatimBorderColor={named}{nbsphinx-code-border}}
\begin{sphinxVerbatim}[commandchars=\\\{\}]
gas sector:
\end{sphinxVerbatim}
}

\hrule height -\fboxrule\relax
\vspace{\nbsphinxcodecellspacing}

\makeatletter\setbox\nbsphinxpromptbox\box\voidb@x\makeatother

\begin{nbsphinxfancyoutput}

\noindent\sphinxincludegraphics[width=514\sphinxpxdimen,height=208\sphinxpxdimen]{{user-guide_add-region_5_1}.png}

\end{nbsphinxfancyoutput}

{

\kern-\sphinxverbatimsmallskipamount\kern-\baselineskip
\kern+\FrameHeightAdjust\kern-\fboxrule
\vspace{\nbsphinxcodecellspacing}

\sphinxsetup{VerbatimColor={named}{white}}
\sphinxsetup{VerbatimBorderColor={named}{nbsphinx-code-border}}
\begin{sphinxVerbatim}[commandchars=\\\{\}]
power sector:
\end{sphinxVerbatim}
}

\hrule height -\fboxrule\relax
\vspace{\nbsphinxcodecellspacing}

\makeatletter\setbox\nbsphinxpromptbox\box\voidb@x\makeatother

\begin{nbsphinxfancyoutput}

\noindent\sphinxincludegraphics[width=516\sphinxpxdimen,height=208\sphinxpxdimen]{{user-guide_add-region_5_3}.png}

\end{nbsphinxfancyoutput}

{

\kern-\sphinxverbatimsmallskipamount\kern-\baselineskip
\kern+\FrameHeightAdjust\kern-\fboxrule
\vspace{\nbsphinxcodecellspacing}

\sphinxsetup{VerbatimColor={named}{white}}
\sphinxsetup{VerbatimBorderColor={named}{nbsphinx-code-border}}
\begin{sphinxVerbatim}[commandchars=\\\{\}]
residential sector:
\end{sphinxVerbatim}
}

\hrule height -\fboxrule\relax
\vspace{\nbsphinxcodecellspacing}

\makeatletter\setbox\nbsphinxpromptbox\box\voidb@x\makeatother

\begin{nbsphinxfancyoutput}

\noindent\sphinxincludegraphics[width=512\sphinxpxdimen,height=208\sphinxpxdimen]{{user-guide_add-region_5_5}.png}

\end{nbsphinxfancyoutput}

Due to the similar natures of the two regions, with the parameters effectively copied and pasted between them, the results are very similar in both \sphinxcode{\sphinxupquote{R1}} and \sphinxcode{\sphinxupquote{R2}}. \sphinxcode{\sphinxupquote{gassupply1}} drops significantly within the gas sector, due to the increasing demand of \sphinxcode{\sphinxupquote{heatpump}} and falling demand of \sphinxcode{\sphinxupquote{gasboiler}} in both region \sphinxcode{\sphinxupquote{R1}} and \sphinxcode{\sphinxupquote{R2}}. \sphinxcode{\sphinxupquote{windturbine}} increases significantly to match this \sphinxcode{\sphinxupquote{heatpump}} demand.

Have a play around with the various costs data in the technodata files for each of the sectors and technologies to see if different scenarios emerge. Although be careful. In some cases, the constraints on certain technologies will make it impossible for the demand to be met. Therefore you may have to relax these constraints.


\subsection{Next steps}
\label{\detokenize{user-guide/add-region:Next-steps}}
In the next section we modify the \sphinxcode{\sphinxupquote{settings.toml}} file to change the timeslicing arrangements as well as project until 2040, instead of 2050, in two year timeslices.


\section{Modification of time}
\label{\detokenize{user-guide/modify-timing-data:Modification-of-time}}\label{\detokenize{user-guide/modify-timing-data::doc}}
In this section we will show you how to modify the timeslicing arrangement as well as change the time horizon and year intervals by modifying the \sphinxcode{\sphinxupquote{settings.toml}} file.


\subsection{Modify timeslicing}
\label{\detokenize{user-guide/modify-timing-data:Modify-timeslicing}}
Timeslicing is the division of a single year into multiple different sections. For example, we could slice the year into different seasons, make a distinction between weekday and weekend or a distinction between morning and night. We do this as energy demand profiles can show a difference between these timeslices. eg. Electricity consumption is lower during the night than during the day.

To achieve this, we have to modify the \sphinxcode{\sphinxupquote{settings.toml}} file, as well as the files within the preset folder: \sphinxcode{\sphinxupquote{Residential2020Consumption.csv}} and \sphinxcode{\sphinxupquote{Residential2050Consumption.csv}}. This is so that we can edit the demand for the residential sector for the new timeslices.

First we edit the \sphinxcode{\sphinxupquote{settings.toml}} file to add two additional timeslices: early\sphinxhyphen{}morning and late\sphinxhyphen{}afternoon. We also rename afternoon to mid\sphinxhyphen{}afternoon. These settings can be found at the bottom of the \sphinxcode{\sphinxupquote{settings.toml}} file.

An example of the changes is shown below:

\begin{sphinxVerbatim}[commandchars=\\\{\}]
\PYG{p}{[}\PYG{n}{timeslices}\PYG{p}{]}
\PYG{n+nb}{all}\PYG{o}{\PYGZhy{}}\PYG{n}{year}\PYG{o}{.}\PYG{n}{all}\PYG{o}{\PYGZhy{}}\PYG{n}{week}\PYG{o}{.}\PYG{n}{night} \PYG{o}{=} \PYG{l+m+mi}{1095}
\PYG{n+nb}{all}\PYG{o}{\PYGZhy{}}\PYG{n}{year}\PYG{o}{.}\PYG{n}{all}\PYG{o}{\PYGZhy{}}\PYG{n}{week}\PYG{o}{.}\PYG{n}{morning} \PYG{o}{=} \PYG{l+m+mi}{1095}
\PYG{n+nb}{all}\PYG{o}{\PYGZhy{}}\PYG{n}{year}\PYG{o}{.}\PYG{n}{all}\PYG{o}{\PYGZhy{}}\PYG{n}{week}\PYG{o}{.}\PYG{n}{mid}\PYG{o}{\PYGZhy{}}\PYG{n}{afternoon} \PYG{o}{=} \PYG{l+m+mi}{1095}
\PYG{n+nb}{all}\PYG{o}{\PYGZhy{}}\PYG{n}{year}\PYG{o}{.}\PYG{n}{all}\PYG{o}{\PYGZhy{}}\PYG{n}{week}\PYG{o}{.}\PYG{n}{early}\PYG{o}{\PYGZhy{}}\PYG{n}{peak} \PYG{o}{=} \PYG{l+m+mi}{1095}
\PYG{n+nb}{all}\PYG{o}{\PYGZhy{}}\PYG{n}{year}\PYG{o}{.}\PYG{n}{all}\PYG{o}{\PYGZhy{}}\PYG{n}{week}\PYG{o}{.}\PYG{n}{late}\PYG{o}{\PYGZhy{}}\PYG{n}{peak} \PYG{o}{=} \PYG{l+m+mi}{1095}
\PYG{n+nb}{all}\PYG{o}{\PYGZhy{}}\PYG{n}{year}\PYG{o}{.}\PYG{n}{all}\PYG{o}{\PYGZhy{}}\PYG{n}{week}\PYG{o}{.}\PYG{n}{evening} \PYG{o}{=} \PYG{l+m+mi}{1095}
\PYG{n+nb}{all}\PYG{o}{\PYGZhy{}}\PYG{n}{year}\PYG{o}{.}\PYG{n}{all}\PYG{o}{\PYGZhy{}}\PYG{n}{week}\PYG{o}{.}\PYG{n}{early}\PYG{o}{\PYGZhy{}}\PYG{n}{morning} \PYG{o}{=} \PYG{l+m+mi}{1095}
\PYG{n+nb}{all}\PYG{o}{\PYGZhy{}}\PYG{n}{year}\PYG{o}{.}\PYG{n}{all}\PYG{o}{\PYGZhy{}}\PYG{n}{week}\PYG{o}{.}\PYG{n}{late}\PYG{o}{\PYGZhy{}}\PYG{n}{afternoon} \PYG{o}{=} \PYG{l+m+mi}{1095}
\PYG{n}{level\PYGZus{}names} \PYG{o}{=} \PYG{p}{[}\PYG{l+s+s2}{\PYGZdq{}}\PYG{l+s+s2}{month}\PYG{l+s+s2}{\PYGZdq{}}\PYG{p}{,} \PYG{l+s+s2}{\PYGZdq{}}\PYG{l+s+s2}{day}\PYG{l+s+s2}{\PYGZdq{}}\PYG{p}{,} \PYG{l+s+s2}{\PYGZdq{}}\PYG{l+s+s2}{hour}\PYG{l+s+s2}{\PYGZdq{}}\PYG{p}{]}
\end{sphinxVerbatim}

Next, we modify both Residential Consumption files. Again, we put the text in bold for the modified entries. We must add the demand for the two additional timelsices, which we call timeslice 7 and 8. We make the demand for heat to be 2 for both of the new timeslices.

Below is the modified \sphinxcode{\sphinxupquote{Residential2020Consumption.csv}} file:


\begin{savenotes}\sphinxattablestart
\centering
\begin{tabulary}{\linewidth}[t]{|T|T|T|T|T|T|T|T|T|}
\hline


&\sphinxstyletheadfamily 
RegionName
&\sphinxstyletheadfamily 
ProcessName
&\sphinxstyletheadfamily 
Timeslice
&\sphinxstyletheadfamily 
electricity
&\sphinxstyletheadfamily 
gas
&\sphinxstyletheadfamily 
heat
&\sphinxstyletheadfamily 
CO2f
&\sphinxstyletheadfamily 
wind
\\
\hline
0
&
R1
&
gasboiler
&
1
&
0
&
0
&
1
&
0
&
0
\\
\hline
1
&
R1
&
gasboiler
&
2
&
0
&
0
&
1.5
&
0
&
0
\\
\hline
2
&
R1
&
gasboiler
&
3
&
0
&
0
&
1
&
0
&
0
\\
\hline
3
&
R1
&
gasboiler
&
4
&
0
&
0
&
1.5
&
0
&
0
\\
\hline
4
&
R1
&
gasboiler
&
5
&
0
&
0
&
3
&
0
&
0
\\
\hline
5
&
R1
&
gasboiler
&
6
&
0
&
0
&
2
&
0
&
0
\\
\hline
\sphinxstylestrong{6}
&
\sphinxstylestrong{R1}
&
\sphinxstylestrong{gasboiler}
&
\sphinxstylestrong{7}
&
\sphinxstylestrong{0}
&
\sphinxstylestrong{0}
&
\sphinxstylestrong{2}
&
\sphinxstylestrong{0}
&
\sphinxstylestrong{0}
\\
\hline
\sphinxstylestrong{7}
&
\sphinxstylestrong{R1}
&
\sphinxstylestrong{gasboiler}
&
\sphinxstylestrong{8}
&
\sphinxstylestrong{0}
&
\sphinxstylestrong{0}
&
\sphinxstylestrong{2}
&
\sphinxstylestrong{0}
&
\sphinxstylestrong{0}
\\
\hline
0
&
R2
&
gasboiler
&
1
&
0
&
0
&
1
&
0
&
0
\\
\hline
1
&
R2
&
gasboiler
&
2
&
0
&
0
&
1.5
&
0
&
0
\\
\hline
2
&
R2
&
gasboiler
&
3
&
0
&
0
&
1
&
0
&
0
\\
\hline
3
&
R2
&
gasboiler
&
4
&
0
&
0
&
1.5
&
0
&
0
\\
\hline
4
&
R2
&
gasboiler
&
5
&
0
&
0
&
3
&
0
&
0
\\
\hline
5
&
R2
&
gasboiler
&
6
&
0
&
0
&
2
&
0
&
0
\\
\hline
\sphinxstylestrong{6}
&
\sphinxstylestrong{R2}
&
\sphinxstylestrong{gasboiler}
&
\sphinxstylestrong{7}
&
\sphinxstylestrong{0}
&
\sphinxstylestrong{0}
&
\sphinxstylestrong{2}
&
\sphinxstylestrong{0}
&
\sphinxstylestrong{0}
\\
\hline
\sphinxstylestrong{7}
&
\sphinxstylestrong{R2}
&
\sphinxstylestrong{gasboiler}
&
\sphinxstylestrong{8}
&
\sphinxstylestrong{0}
&
\sphinxstylestrong{0}
&
\sphinxstylestrong{2}
&
\sphinxstylestrong{0}
&
\sphinxstylestrong{0}
\\
\hline
\end{tabulary}
\par
\sphinxattableend\end{savenotes}

We do the same for the \sphinxcode{\sphinxupquote{Residential2050Consumption.csv}}, however this time we make the demand for heat in 2050 to both be 5 for the new timeslices. See \sphinxhref{github-residential2050}{here INSERT LINK HERE} for the full file.



Once the relevant files have been edited, we are able to run the simulation model using \sphinxcode{\sphinxupquote{python \sphinxhyphen{}m muse settings.toml}}.

Then, once run, we import the necessary packages:

{
\sphinxsetup{VerbatimColor={named}{nbsphinx-code-bg}}
\sphinxsetup{VerbatimBorderColor={named}{nbsphinx-code-border}}
\begin{sphinxVerbatim}[commandchars=\\\{\}]
\llap{\color{nbsphinxin}[1]:\,\hspace{\fboxrule}\hspace{\fboxsep}}\PYG{k+kn}{import} \PYG{n+nn}{pandas} \PYG{k}{as} \PYG{n+nn}{pd}
\PYG{k+kn}{import} \PYG{n+nn}{seaborn} \PYG{k}{as} \PYG{n+nn}{sns}
\PYG{k+kn}{import} \PYG{n+nn}{matplotlib}\PYG{n+nn}{.}\PYG{n+nn}{pyplot} \PYG{k}{as} \PYG{n+nn}{plt}
\end{sphinxVerbatim}
}

and visualise the relevant data:

{
\sphinxsetup{VerbatimColor={named}{nbsphinx-code-bg}}
\sphinxsetup{VerbatimBorderColor={named}{nbsphinx-code-border}}
\begin{sphinxVerbatim}[commandchars=\\\{\}]
\llap{\color{nbsphinxin}[2]:\,\hspace{\fboxrule}\hspace{\fboxsep}}\PYG{n}{mca\PYGZus{}capacity} \PYG{o}{=} \PYG{n}{pd}\PYG{o}{.}\PYG{n}{read\PYGZus{}csv}\PYG{p}{(}\PYG{l+s+s2}{\PYGZdq{}}\PYG{l+s+s2}{../tutorial\PYGZhy{}code/modify\PYGZhy{}timing\PYGZhy{}data/modify\PYGZhy{}time\PYGZhy{}framework/Results/MCACapacity.csv}\PYG{l+s+s2}{\PYGZdq{}}\PYG{p}{)}

\PYG{k}{for} \PYG{n}{name}\PYG{p}{,} \PYG{n}{sector} \PYG{o+ow}{in} \PYG{n}{mca\PYGZus{}capacity}\PYG{o}{.}\PYG{n}{groupby}\PYG{p}{(}\PYG{l+s+s2}{\PYGZdq{}}\PYG{l+s+s2}{sector}\PYG{l+s+s2}{\PYGZdq{}}\PYG{p}{)}\PYG{p}{:}
    \PYG{n+nb}{print}\PYG{p}{(}\PYG{l+s+s2}{\PYGZdq{}}\PYG{l+s+si}{\PYGZob{}\PYGZcb{}}\PYG{l+s+s2}{ sector:}\PYG{l+s+s2}{\PYGZdq{}}\PYG{o}{.}\PYG{n}{format}\PYG{p}{(}\PYG{n}{name}\PYG{p}{)}\PYG{p}{)}
    \PYG{n}{fig}\PYG{p}{,} \PYG{n}{ax} \PYG{o}{=}\PYG{n}{plt}\PYG{o}{.}\PYG{n}{subplots}\PYG{p}{(}\PYG{l+m+mi}{1}\PYG{p}{,}\PYG{l+m+mi}{2}\PYG{p}{)}
    \PYG{n}{sns}\PYG{o}{.}\PYG{n}{lineplot}\PYG{p}{(}\PYG{n}{data}\PYG{o}{=}\PYG{n}{sector}\PYG{p}{[}\PYG{n}{sector}\PYG{o}{.}\PYG{n}{region}\PYG{o}{==}\PYG{l+s+s2}{\PYGZdq{}}\PYG{l+s+s2}{R1}\PYG{l+s+s2}{\PYGZdq{}}\PYG{p}{]}\PYG{p}{,} \PYG{n}{x}\PYG{o}{=}\PYG{l+s+s2}{\PYGZdq{}}\PYG{l+s+s2}{year}\PYG{l+s+s2}{\PYGZdq{}}\PYG{p}{,} \PYG{n}{y}\PYG{o}{=}\PYG{l+s+s2}{\PYGZdq{}}\PYG{l+s+s2}{capacity}\PYG{l+s+s2}{\PYGZdq{}}\PYG{p}{,} \PYG{n}{hue}\PYG{o}{=}\PYG{l+s+s2}{\PYGZdq{}}\PYG{l+s+s2}{technology}\PYG{l+s+s2}{\PYGZdq{}}\PYG{p}{,} \PYG{n}{ax}\PYG{o}{=}\PYG{n}{ax}\PYG{p}{[}\PYG{l+m+mi}{0}\PYG{p}{]}\PYG{p}{)}
    \PYG{n}{sns}\PYG{o}{.}\PYG{n}{lineplot}\PYG{p}{(}\PYG{n}{data}\PYG{o}{=}\PYG{n}{sector}\PYG{p}{[}\PYG{n}{sector}\PYG{o}{.}\PYG{n}{region}\PYG{o}{==}\PYG{l+s+s2}{\PYGZdq{}}\PYG{l+s+s2}{R2}\PYG{l+s+s2}{\PYGZdq{}}\PYG{p}{]}\PYG{p}{,} \PYG{n}{x}\PYG{o}{=}\PYG{l+s+s2}{\PYGZdq{}}\PYG{l+s+s2}{year}\PYG{l+s+s2}{\PYGZdq{}}\PYG{p}{,} \PYG{n}{y}\PYG{o}{=}\PYG{l+s+s2}{\PYGZdq{}}\PYG{l+s+s2}{capacity}\PYG{l+s+s2}{\PYGZdq{}}\PYG{p}{,} \PYG{n}{hue}\PYG{o}{=}\PYG{l+s+s2}{\PYGZdq{}}\PYG{l+s+s2}{technology}\PYG{l+s+s2}{\PYGZdq{}}\PYG{p}{,} \PYG{n}{ax}\PYG{o}{=}\PYG{n}{ax}\PYG{p}{[}\PYG{l+m+mi}{1}\PYG{p}{]}\PYG{p}{)}
    \PYG{n}{plt}\PYG{o}{.}\PYG{n}{show}\PYG{p}{(}\PYG{p}{)}
    \PYG{n}{plt}\PYG{o}{.}\PYG{n}{close}\PYG{p}{(}\PYG{p}{)}
\end{sphinxVerbatim}
}

{

\kern-\sphinxverbatimsmallskipamount\kern-\baselineskip
\kern+\FrameHeightAdjust\kern-\fboxrule
\vspace{\nbsphinxcodecellspacing}

\sphinxsetup{VerbatimColor={named}{white}}
\sphinxsetup{VerbatimBorderColor={named}{nbsphinx-code-border}}
\begin{sphinxVerbatim}[commandchars=\\\{\}]
gas sector:
\end{sphinxVerbatim}
}

\hrule height -\fboxrule\relax
\vspace{\nbsphinxcodecellspacing}

\makeatletter\setbox\nbsphinxpromptbox\box\voidb@x\makeatother

\begin{nbsphinxfancyoutput}

\noindent\sphinxincludegraphics[width=388\sphinxpxdimen,height=262\sphinxpxdimen]{{user-guide_modify-timing-data_7_1}.png}

\end{nbsphinxfancyoutput}

{

\kern-\sphinxverbatimsmallskipamount\kern-\baselineskip
\kern+\FrameHeightAdjust\kern-\fboxrule
\vspace{\nbsphinxcodecellspacing}

\sphinxsetup{VerbatimColor={named}{white}}
\sphinxsetup{VerbatimBorderColor={named}{nbsphinx-code-border}}
\begin{sphinxVerbatim}[commandchars=\\\{\}]
power sector:
\end{sphinxVerbatim}
}

\hrule height -\fboxrule\relax
\vspace{\nbsphinxcodecellspacing}

\makeatletter\setbox\nbsphinxpromptbox\box\voidb@x\makeatother

\begin{nbsphinxfancyoutput}

\noindent\sphinxincludegraphics[width=395\sphinxpxdimen,height=262\sphinxpxdimen]{{user-guide_modify-timing-data_7_3}.png}

\end{nbsphinxfancyoutput}

{

\kern-\sphinxverbatimsmallskipamount\kern-\baselineskip
\kern+\FrameHeightAdjust\kern-\fboxrule
\vspace{\nbsphinxcodecellspacing}

\sphinxsetup{VerbatimColor={named}{white}}
\sphinxsetup{VerbatimBorderColor={named}{nbsphinx-code-border}}
\begin{sphinxVerbatim}[commandchars=\\\{\}]
residential sector:
\end{sphinxVerbatim}
}

\hrule height -\fboxrule\relax
\vspace{\nbsphinxcodecellspacing}

\makeatletter\setbox\nbsphinxpromptbox\box\voidb@x\makeatother

\begin{nbsphinxfancyoutput}

\noindent\sphinxincludegraphics[width=388\sphinxpxdimen,height=262\sphinxpxdimen]{{user-guide_modify-timing-data_7_5}.png}

\end{nbsphinxfancyoutput}

Compared to the scenario where we added a {\hyperref[\detokenize{user-guide/add-region::doc}]{\sphinxcrossref{\DUrole{doc}{region}}}}, there is a slight increase in solarPV in the power sector. However, the rest remains unchanged.


\subsection{Modify time horizon and time periods}
\label{\detokenize{user-guide/modify-timing-data:Modify-time-horizon-and-time-periods}}
For the previous examples, we have run the scenario from 2020 to 2050, in 5 year time steps. This has been set at the top of the \sphinxcode{\sphinxupquote{settings.toml}} file. However, we may want to run a more detailed scenario, with 2 year time steps, and up until the year 2040.

Making this change is quite simple as we only have two lines to change. We will modify line 2 and 3 of the \sphinxcode{\sphinxupquote{settings.toml}} file, as follows:

\begin{sphinxVerbatim}[commandchars=\\\{\}]
\PYG{c+c1}{\PYGZsh{} Global settings \PYGZhy{} most REQUIRED}
\PYG{n}{time\PYGZus{}framework} \PYG{o}{=} \PYG{p}{[}\PYG{l+m+mi}{2020}\PYG{p}{,} \PYG{l+m+mi}{2022}\PYG{p}{,} \PYG{l+m+mi}{2024}\PYG{p}{,} \PYG{l+m+mi}{2026}\PYG{p}{,} \PYG{l+m+mi}{2028}\PYG{p}{,} \PYG{l+m+mi}{2030}\PYG{p}{,} \PYG{l+m+mi}{2032}\PYG{p}{,} \PYG{l+m+mi}{2034}\PYG{p}{,} \PYG{l+m+mi}{2036}\PYG{p}{,} \PYG{l+m+mi}{2038}\PYG{p}{,} \PYG{l+m+mi}{2040}\PYG{p}{]}
\PYG{n}{foresight} \PYG{o}{=} \PYG{l+m+mi}{2}   \PYG{c+c1}{\PYGZsh{} Has to be a multiple of the minimum separation between the years in time}
\end{sphinxVerbatim}

The \sphinxcode{\sphinxupquote{time\_framework}} details each year in which we run the simulation. The \sphinxcode{\sphinxupquote{foresight}} variable details how much foresight an agent has when making investments.

As we have modified the timeslicing arrangements there will be a change in the underlying demand for heating. This may require more electricity to service this demand. Therefore, we relax the constraints for growth in the power sector for all technologies and constraints in the \sphinxcode{\sphinxupquote{technodata/power/technodata.csv}}, as is shown below:


\begin{savenotes}\sphinxattablestart
\centering
\begin{tabular}[t]{|*{8}{\X{1}{8}|}}
\hline
\sphinxstyletheadfamily 
ProcessName
&\sphinxstyletheadfamily 
RegionName
&\sphinxstyletheadfamily 
…
&\sphinxstyletheadfamily 
MaxCapacityAddition
&\sphinxstyletheadfamily 
MaxCapacityGrowth
&\sphinxstyletheadfamily 
TotalCapacityLimit
&\sphinxstyletheadfamily 
…
&\sphinxstyletheadfamily 
Agent1
\\
\hline
Unit
&\begin{itemize}
\item {} 
\end{itemize}
&
…
&
PJ
&
\%
&
PJ
&
…
&
New
\\
\hline
gasCCGT
&
R1
&
…
&
\sphinxstylestrong{40}
&
\sphinxstylestrong{0.2}
&
\sphinxstylestrong{120}
&
…
&
0
\\
\hline
windturbine
&
R1
&
…
&
\sphinxstylestrong{40}
&
\sphinxstylestrong{0.2}
&
\sphinxstylestrong{120}
&
…
&
0
\\
\hline
solarPV
&
R1
&
…
&
\sphinxstylestrong{40}
&
\sphinxstylestrong{0.2}
&
\sphinxstylestrong{120}
&
…
&
0
\\
\hline
gasCCGT
&
R2
&
…
&
\sphinxstylestrong{40}
&
\sphinxstylestrong{0.2}
&
\sphinxstylestrong{120}
&
…
&
0
\\
\hline
windturbine
&
R2
&
…
&
\sphinxstylestrong{40}
&
\sphinxstylestrong{0.2}
&
\sphinxstylestrong{120}
&
…
&
0
\\
\hline
solarPV
&
R2
&
…
&
\sphinxstylestrong{40}
&
\sphinxstylestrong{0.2}
&
\sphinxstylestrong{120}
&
…
&
0
\\
\hline
\end{tabular}
\par
\sphinxattableend\end{savenotes}

We also modify the constraints defined in the \sphinxcode{\sphinxupquote{technodata.csv}} file for the residential sector, as shown below:


\begin{savenotes}\sphinxattablestart
\centering
\begin{tabular}[t]{|*{8}{\X{1}{8}|}}
\hline
\sphinxstyletheadfamily 
ProcessName
&\sphinxstyletheadfamily 
RegionName
&\sphinxstyletheadfamily 
…
&\sphinxstyletheadfamily 
MaxCapacityAddition
&\sphinxstyletheadfamily 
MaxCapacityGrowth
&\sphinxstyletheadfamily 
TotalCapacityLimit
&\sphinxstyletheadfamily 
…
&\sphinxstyletheadfamily 
Agent1
\\
\hline
Unit
&\begin{itemize}
\item {} 
\end{itemize}
&
…
&
PJ
&
\%
&
PJ
&
…
&
New
\\
\hline
gasboiler
&
R1
&
…
&
\sphinxstylestrong{60}
&
\sphinxstylestrong{0.5}
&
\sphinxstylestrong{120}
&
…
&
0
\\
\hline
heatpump
&
R1
&
…
&
\sphinxstylestrong{60}
&
\sphinxstylestrong{0.5}
&
\sphinxstylestrong{120}
&
…
&
0
\\
\hline
gasboiler
&
R2
&
…
&
\sphinxstylestrong{60}
&
\sphinxstylestrong{0.5}
&
\sphinxstylestrong{120}
&
…
&
0
\\
\hline
heatpump
&
R2
&
…
&
\sphinxstylestrong{60}
&
\sphinxstylestrong{0.5}
&
\sphinxstylestrong{120}
&
…
&
0
\\
\hline
\end{tabular}
\par
\sphinxattableend\end{savenotes}

It must be noted, that this is a toy example. For modelling a real life scenario, data should be sought to ensure there remain realistic constriants.

For the full power sector \sphinxcode{\sphinxupquote{technodata.csv}} file click \sphinxhref{github-power-technodata}{here INSERT LINK HERE}, and for the full residential sector \sphinxcode{\sphinxupquote{technodata.csv}} file click \sphinxhref{github-residential-technodata}{here INSERT LINK HERE}.



{
\sphinxsetup{VerbatimColor={named}{nbsphinx-code-bg}}
\sphinxsetup{VerbatimBorderColor={named}{nbsphinx-code-border}}
\begin{sphinxVerbatim}[commandchars=\\\{\}]
\llap{\color{nbsphinxin}[3]:\,\hspace{\fboxrule}\hspace{\fboxsep}}\PYG{n}{mca\PYGZus{}capacity} \PYG{o}{=} \PYG{n}{pd}\PYG{o}{.}\PYG{n}{read\PYGZus{}csv}\PYG{p}{(}\PYG{l+s+s2}{\PYGZdq{}}\PYG{l+s+s2}{../tutorial\PYGZhy{}code/modify\PYGZhy{}timing\PYGZhy{}data/modify\PYGZhy{}time\PYGZhy{}framework/Results/MCACapacity.csv}\PYG{l+s+s2}{\PYGZdq{}}\PYG{p}{)}

\PYG{k}{for} \PYG{n}{name}\PYG{p}{,} \PYG{n}{sector} \PYG{o+ow}{in} \PYG{n}{mca\PYGZus{}capacity}\PYG{o}{.}\PYG{n}{groupby}\PYG{p}{(}\PYG{l+s+s2}{\PYGZdq{}}\PYG{l+s+s2}{sector}\PYG{l+s+s2}{\PYGZdq{}}\PYG{p}{)}\PYG{p}{:}
    \PYG{n+nb}{print}\PYG{p}{(}\PYG{l+s+s2}{\PYGZdq{}}\PYG{l+s+si}{\PYGZob{}\PYGZcb{}}\PYG{l+s+s2}{ sector:}\PYG{l+s+s2}{\PYGZdq{}}\PYG{o}{.}\PYG{n}{format}\PYG{p}{(}\PYG{n}{name}\PYG{p}{)}\PYG{p}{)}
    \PYG{n}{fig}\PYG{p}{,} \PYG{n}{ax} \PYG{o}{=}\PYG{n}{plt}\PYG{o}{.}\PYG{n}{subplots}\PYG{p}{(}\PYG{l+m+mi}{1}\PYG{p}{,}\PYG{l+m+mi}{2}\PYG{p}{)}
    \PYG{n}{sns}\PYG{o}{.}\PYG{n}{lineplot}\PYG{p}{(}\PYG{n}{data}\PYG{o}{=}\PYG{n}{sector}\PYG{p}{[}\PYG{n}{sector}\PYG{o}{.}\PYG{n}{region}\PYG{o}{==}\PYG{l+s+s2}{\PYGZdq{}}\PYG{l+s+s2}{R1}\PYG{l+s+s2}{\PYGZdq{}}\PYG{p}{]}\PYG{p}{,} \PYG{n}{x}\PYG{o}{=}\PYG{l+s+s2}{\PYGZdq{}}\PYG{l+s+s2}{year}\PYG{l+s+s2}{\PYGZdq{}}\PYG{p}{,} \PYG{n}{y}\PYG{o}{=}\PYG{l+s+s2}{\PYGZdq{}}\PYG{l+s+s2}{capacity}\PYG{l+s+s2}{\PYGZdq{}}\PYG{p}{,} \PYG{n}{hue}\PYG{o}{=}\PYG{l+s+s2}{\PYGZdq{}}\PYG{l+s+s2}{technology}\PYG{l+s+s2}{\PYGZdq{}}\PYG{p}{,} \PYG{n}{ax}\PYG{o}{=}\PYG{n}{ax}\PYG{p}{[}\PYG{l+m+mi}{0}\PYG{p}{]}\PYG{p}{)}
    \PYG{n}{sns}\PYG{o}{.}\PYG{n}{lineplot}\PYG{p}{(}\PYG{n}{data}\PYG{o}{=}\PYG{n}{sector}\PYG{p}{[}\PYG{n}{sector}\PYG{o}{.}\PYG{n}{region}\PYG{o}{==}\PYG{l+s+s2}{\PYGZdq{}}\PYG{l+s+s2}{R2}\PYG{l+s+s2}{\PYGZdq{}}\PYG{p}{]}\PYG{p}{,} \PYG{n}{x}\PYG{o}{=}\PYG{l+s+s2}{\PYGZdq{}}\PYG{l+s+s2}{year}\PYG{l+s+s2}{\PYGZdq{}}\PYG{p}{,} \PYG{n}{y}\PYG{o}{=}\PYG{l+s+s2}{\PYGZdq{}}\PYG{l+s+s2}{capacity}\PYG{l+s+s2}{\PYGZdq{}}\PYG{p}{,} \PYG{n}{hue}\PYG{o}{=}\PYG{l+s+s2}{\PYGZdq{}}\PYG{l+s+s2}{technology}\PYG{l+s+s2}{\PYGZdq{}}\PYG{p}{,} \PYG{n}{ax}\PYG{o}{=}\PYG{n}{ax}\PYG{p}{[}\PYG{l+m+mi}{1}\PYG{p}{]}\PYG{p}{)}
    \PYG{n}{plt}\PYG{o}{.}\PYG{n}{show}\PYG{p}{(}\PYG{p}{)}
    \PYG{n}{plt}\PYG{o}{.}\PYG{n}{close}\PYG{p}{(}\PYG{p}{)}
\end{sphinxVerbatim}
}

{

\kern-\sphinxverbatimsmallskipamount\kern-\baselineskip
\kern+\FrameHeightAdjust\kern-\fboxrule
\vspace{\nbsphinxcodecellspacing}

\sphinxsetup{VerbatimColor={named}{white}}
\sphinxsetup{VerbatimBorderColor={named}{nbsphinx-code-border}}
\begin{sphinxVerbatim}[commandchars=\\\{\}]
gas sector:
\end{sphinxVerbatim}
}

\hrule height -\fboxrule\relax
\vspace{\nbsphinxcodecellspacing}

\makeatletter\setbox\nbsphinxpromptbox\box\voidb@x\makeatother

\begin{nbsphinxfancyoutput}

\noindent\sphinxincludegraphics[width=388\sphinxpxdimen,height=262\sphinxpxdimen]{{user-guide_modify-timing-data_13_1}.png}

\end{nbsphinxfancyoutput}

{

\kern-\sphinxverbatimsmallskipamount\kern-\baselineskip
\kern+\FrameHeightAdjust\kern-\fboxrule
\vspace{\nbsphinxcodecellspacing}

\sphinxsetup{VerbatimColor={named}{white}}
\sphinxsetup{VerbatimBorderColor={named}{nbsphinx-code-border}}
\begin{sphinxVerbatim}[commandchars=\\\{\}]
power sector:
\end{sphinxVerbatim}
}

\hrule height -\fboxrule\relax
\vspace{\nbsphinxcodecellspacing}

\makeatletter\setbox\nbsphinxpromptbox\box\voidb@x\makeatother

\begin{nbsphinxfancyoutput}

\noindent\sphinxincludegraphics[width=395\sphinxpxdimen,height=262\sphinxpxdimen]{{user-guide_modify-timing-data_13_3}.png}

\end{nbsphinxfancyoutput}

{

\kern-\sphinxverbatimsmallskipamount\kern-\baselineskip
\kern+\FrameHeightAdjust\kern-\fboxrule
\vspace{\nbsphinxcodecellspacing}

\sphinxsetup{VerbatimColor={named}{white}}
\sphinxsetup{VerbatimBorderColor={named}{nbsphinx-code-border}}
\begin{sphinxVerbatim}[commandchars=\\\{\}]
residential sector:
\end{sphinxVerbatim}
}

\hrule height -\fboxrule\relax
\vspace{\nbsphinxcodecellspacing}

\makeatletter\setbox\nbsphinxpromptbox\box\voidb@x\makeatother

\begin{nbsphinxfancyoutput}

\noindent\sphinxincludegraphics[width=388\sphinxpxdimen,height=262\sphinxpxdimen]{{user-guide_modify-timing-data_13_5}.png}

\end{nbsphinxfancyoutput}


\subsection{Next steps}
\label{\detokenize{user-guide/modify-timing-data:Next-steps}}
In the next section we detail how to add an exogenous service demand, such as demand for heating or cooking.


\section{Adding a service demand}
\label{\detokenize{user-guide/addition-service-demand:Adding-a-service-demand}}\label{\detokenize{user-guide/addition-service-demand::doc}}
In this section, we will detail how to add a service demand to MUSE.

A service demand is an end\sphinxhyphen{}use demand. For example, in the residential sector, a service demand could be cooking. Houses require energy to cook food and a technology to service this demand, such as an electric stove.

This process consists of setting a demand, either through inputs derived from the user or correlations of GDP and population which reflect the socio\sphinxhyphen{}economic decvelopment of a region or country. In addition, a technology must be added to service the demand.


\subsection{Addition of cooking demand}
\label{\detokenize{user-guide/addition-service-demand:Addition-of-cooking-demand}}
Firstly, we must add the demand section. In this example, we will add a cooking preset demand. To achieve this, we will now edit the \sphinxcode{\sphinxupquote{Residential2020Consumption.csv}} and \sphinxcode{\sphinxupquote{Residential2050Consumption.csv}} files, found within the \sphinxcode{\sphinxupquote{technodata/preset/}} directory.

The \sphinxcode{\sphinxupquote{Residential2020Consumption.csv}} file allows us to specify the demand in 2020 for each region and technology per timeslice. The \sphinxcode{\sphinxupquote{Residential2050Consumption.csv}} file does the same but for the year 2050. The datapoints between these are interpolated.

Firstly, we must add the new service demand: \sphinxcode{\sphinxupquote{cook}} as a column in these two files. Next, we add the demand. Again, the modified entries are in bold:


\begin{savenotes}\sphinxattablestart
\centering
\begin{tabulary}{\linewidth}[t]{|T|T|T|T|T|T|T|T|T|T|}
\hline


&\sphinxstyletheadfamily 
RegionName
&\sphinxstyletheadfamily 
ProcessName
&\sphinxstyletheadfamily 
Timeslice
&\sphinxstyletheadfamily 
electricity
&\sphinxstyletheadfamily 
gas
&\sphinxstyletheadfamily 
heat
&\sphinxstyletheadfamily 
CO2f
&\sphinxstyletheadfamily 
wind
&\sphinxstyletheadfamily 
\sphinxstylestrong{cook}
\\
\hline
0
&
R1
&
gasboiler
&
1
&
0
&
0
&
1
&
0
&
0
&
\sphinxstylestrong{0}
\\
\hline
…
&
…
&
…
&
…
&
…
&
…
&
…
&
…
&
…
&
\sphinxstylestrong{…}
\\
\hline
15
&
R2
&
gasboiler
&
8
&
0
&
0
&
2
&
0
&
0
&
\sphinxstylestrong{0}
\\
\hline
\sphinxstylestrong{16}
&
\sphinxstylestrong{R1}
&
\sphinxstylestrong{electric\_stove}
&
\sphinxstylestrong{1}
&
\sphinxstylestrong{0}
&
\sphinxstylestrong{0}
&
\sphinxstylestrong{0}
&
\sphinxstylestrong{0}
&
\sphinxstylestrong{0}
&
\sphinxstylestrong{1}
\\
\hline
\sphinxstylestrong{17}
&
\sphinxstylestrong{R1}
&
\sphinxstylestrong{electric\_stove}
&
\sphinxstylestrong{2}
&
\sphinxstylestrong{0}
&
\sphinxstylestrong{0}
&
\sphinxstylestrong{0}
&
\sphinxstylestrong{0}
&
\sphinxstylestrong{0}
&
\sphinxstylestrong{2}
\\
\hline
\sphinxstylestrong{18}
&
\sphinxstylestrong{R1}
&
\sphinxstylestrong{electric\_stove}
&
\sphinxstylestrong{3}
&
\sphinxstylestrong{0}
&
\sphinxstylestrong{0}
&
\sphinxstylestrong{0}
&
\sphinxstylestrong{0}
&
\sphinxstylestrong{0}
&
\sphinxstylestrong{1}
\\
\hline
\sphinxstylestrong{19}
&
\sphinxstylestrong{R1}
&
\sphinxstylestrong{electric\_stove}
&
\sphinxstylestrong{4}
&
\sphinxstylestrong{0}
&
\sphinxstylestrong{0}
&
\sphinxstylestrong{0}
&
\sphinxstylestrong{0}
&
\sphinxstylestrong{0}
&
\sphinxstylestrong{1.5}
\\
\hline
\sphinxstylestrong{20}
&
\sphinxstylestrong{R1}
&
\sphinxstylestrong{electric\_stove}
&
\sphinxstylestrong{5}
&
\sphinxstylestrong{0}
&
\sphinxstylestrong{0}
&
\sphinxstylestrong{0}
&
\sphinxstylestrong{0}
&
\sphinxstylestrong{0}
&
\sphinxstylestrong{2}
\\
\hline
\sphinxstylestrong{21}
&
\sphinxstylestrong{R1}
&
\sphinxstylestrong{electric\_stove}
&
\sphinxstylestrong{6}
&
\sphinxstylestrong{0}
&
\sphinxstylestrong{0}
&
\sphinxstylestrong{0}
&
\sphinxstylestrong{0}
&
\sphinxstylestrong{0}
&
\sphinxstylestrong{3}
\\
\hline
\sphinxstylestrong{22}
&
\sphinxstylestrong{R1}
&
\sphinxstylestrong{electric\_stove}
&
\sphinxstylestrong{7}
&
\sphinxstylestrong{0}
&
\sphinxstylestrong{0}
&
\sphinxstylestrong{0}
&
\sphinxstylestrong{0}
&
\sphinxstylestrong{0}
&
\sphinxstylestrong{2}
\\
\hline
\sphinxstylestrong{23}
&
\sphinxstylestrong{R1}
&
\sphinxstylestrong{electric\_stove}
&
\sphinxstylestrong{8}
&
\sphinxstylestrong{0}
&
\sphinxstylestrong{0}
&
\sphinxstylestrong{0}
&
\sphinxstylestrong{0}
&
\sphinxstylestrong{0}
&
\sphinxstylestrong{3}
\\
\hline
\sphinxstylestrong{24}
&
\sphinxstylestrong{R2}
&
\sphinxstylestrong{electric\_stove}
&
\sphinxstylestrong{1}
&
\sphinxstylestrong{0}
&
\sphinxstylestrong{0}
&
\sphinxstylestrong{0}
&
\sphinxstylestrong{0}
&
\sphinxstylestrong{0}
&
\sphinxstylestrong{1}
\\
\hline
\sphinxstylestrong{25}
&
\sphinxstylestrong{R2}
&
\sphinxstylestrong{electric\_stove}
&
\sphinxstylestrong{2}
&
\sphinxstylestrong{0}
&
\sphinxstylestrong{0}
&
\sphinxstylestrong{0}
&
\sphinxstylestrong{0}
&
\sphinxstylestrong{0}
&
\sphinxstylestrong{1}
\\
\hline
\sphinxstylestrong{26}
&
\sphinxstylestrong{R2}
&
\sphinxstylestrong{electric\_stove}
&
\sphinxstylestrong{3}
&
\sphinxstylestrong{0}
&
\sphinxstylestrong{0}
&
\sphinxstylestrong{0}
&
\sphinxstylestrong{0}
&
\sphinxstylestrong{0}
&
\sphinxstylestrong{1}
\\
\hline
\sphinxstylestrong{27}
&
\sphinxstylestrong{R2}
&
\sphinxstylestrong{electric\_stove}
&
\sphinxstylestrong{4}
&
\sphinxstylestrong{0}
&
\sphinxstylestrong{0}
&
\sphinxstylestrong{0}
&
\sphinxstylestrong{0}
&
\sphinxstylestrong{0}
&
\sphinxstylestrong{1.5}
\\
\hline
\sphinxstylestrong{28}
&
\sphinxstylestrong{R2}
&
\sphinxstylestrong{electric\_stove}
&
\sphinxstylestrong{5}
&
\sphinxstylestrong{0}
&
\sphinxstylestrong{0}
&
\sphinxstylestrong{0}
&
\sphinxstylestrong{0}
&
\sphinxstylestrong{0}
&
\sphinxstylestrong{2}
\\
\hline
\sphinxstylestrong{29}
&
\sphinxstylestrong{R2}
&
\sphinxstylestrong{electric\_stove}
&
\sphinxstylestrong{6}
&
\sphinxstylestrong{0}
&
\sphinxstylestrong{0}
&
\sphinxstylestrong{0}
&
\sphinxstylestrong{0}
&
\sphinxstylestrong{0}
&
\sphinxstylestrong{2}
\\
\hline
\sphinxstylestrong{30}
&
\sphinxstylestrong{R2}
&
\sphinxstylestrong{electric\_stove}
&
\sphinxstylestrong{7}
&
\sphinxstylestrong{0}
&
\sphinxstylestrong{0}
&
\sphinxstylestrong{0}
&
\sphinxstylestrong{0}
&
\sphinxstylestrong{0}
&
\sphinxstylestrong{2.5}
\\
\hline
\sphinxstylestrong{31}
&
\sphinxstylestrong{R2}
&
\sphinxstylestrong{electric\_stove}
&
\sphinxstylestrong{8}
&
\sphinxstylestrong{0}
&
\sphinxstylestrong{0}
&
\sphinxstylestrong{0}
&
\sphinxstylestrong{0}
&
\sphinxstylestrong{0}
&
\sphinxstylestrong{2}
\\
\hline
\end{tabulary}
\par
\sphinxattableend\end{savenotes}

For the purposes of brevity, we omitted the majority of the \sphinxcode{\sphinxupquote{gasboiler}} entries. However, these remain unchanged, apart from a \sphinxcode{\sphinxupquote{0}} entry in the cook column to indicate that a \sphinxcode{\sphinxupquote{gasboiler}} does not meet \sphinxcode{\sphinxupquote{cook}} demand.

We added an \sphinxcode{\sphinxupquote{electric\_stove}} process for each of the timeslices, which meets the \sphinxcode{\sphinxupquote{cook}} demand. This can be seen through the addition of a positive number in the \sphinxcode{\sphinxupquote{cook}} column.

The process is very similar for the \sphinxcode{\sphinxupquote{Residential2050Consumption.csv}} file, however, for this example, we often placed larger numbers to indicate higher demand in 2050. For the complete file see the link \sphinxhref{total-addition-service-demand-github}{here INCLUDE LINK HERE}

Next, we must edit the files within the \sphinxcode{\sphinxupquote{input}} folder. For this, we must add the \sphinxcode{\sphinxupquote{cook}} service demand to each of these files.

First, we will amend the \sphinxcode{\sphinxupquote{BaseYearExport.csv}} and \sphinxcode{\sphinxupquote{BaseYearImport.csv}} files. For this, we say that there is no import or export of the \sphinxcode{\sphinxupquote{cook}} service demand. A brief example is outlined below for \sphinxcode{\sphinxupquote{BaseYearExport.csv}}:


\begin{savenotes}\sphinxattablestart
\centering
\begin{tabular}[t]{|*{10}{\X{1}{10}|}}
\hline
\sphinxstyletheadfamily 
RegionName
&\sphinxstyletheadfamily 
Attribute
&\sphinxstyletheadfamily 
Time
&\sphinxstyletheadfamily 
electricity
&\sphinxstyletheadfamily 
gas
&\sphinxstyletheadfamily 
heat
&\sphinxstyletheadfamily 
CO2f
&\sphinxstyletheadfamily 
wind
&\sphinxstyletheadfamily 
solar
&\sphinxstyletheadfamily 
\sphinxstylestrong{cook}
\\
\hline
Unit
&\begin{itemize}
\item {} 
\end{itemize}
&
Year
&
PJ
&
PJ
&
PJ
&
kt
&
PJ
&
PJ
&
\sphinxstylestrong{PJ}
\\
\hline
R1
&
Exports
&
2010
&
0
&
0
&
0
&
0
&
0
&
0
&
\sphinxstylestrong{0}
\\
\hline
…
&
…
&
…
&
…
&
…
&
…
&
…
&
…
&
…
&
\sphinxstylestrong{…}
\\
\hline
R2
&
Exports
&
2100
&
0
&
0
&
0
&
0
&
0
&
0
&
\sphinxstylestrong{0}
\\
\hline
\end{tabular}
\par
\sphinxattableend\end{savenotes}

The same is true for the \sphinxcode{\sphinxupquote{BaseYearImport.csv}} file:


\begin{savenotes}\sphinxattablestart
\centering
\begin{tabular}[t]{|*{10}{\X{1}{10}|}}
\hline
\sphinxstyletheadfamily 
RegionName
&\sphinxstyletheadfamily 
Attribute
&\sphinxstyletheadfamily 
Time
&\sphinxstyletheadfamily 
electricity
&\sphinxstyletheadfamily 
gas
&\sphinxstyletheadfamily 
heat
&\sphinxstyletheadfamily 
CO2f
&\sphinxstyletheadfamily 
wind
&\sphinxstyletheadfamily 
solar
&\sphinxstyletheadfamily 
\sphinxstylestrong{cook}
\\
\hline
Unit
&\begin{itemize}
\item {} 
\end{itemize}
&
Year
&
PJ
&
PJ
&
PJ
&
kt
&
PJ
&
PJ
&
\sphinxstylestrong{PJ}
\\
\hline
R1
&
Imports
&
2010
&
0
&
0
&
0
&
0
&
0
&
0
&
\sphinxstylestrong{0}
\\
\hline
…
&
…
&
…
&
…
&
…
&
…
&
…
&
…
&
…
&
\sphinxstylestrong{…}
\\
\hline
R2
&
Imports
&
2100
&
0
&
0
&
0
&
0
&
0
&
0
&
\sphinxstylestrong{0}
\\
\hline
\end{tabular}
\par
\sphinxattableend\end{savenotes}

Next, we must edit the \sphinxcode{\sphinxupquote{GlobalCommodities.csv}} file. This is where we define the new commodity \sphinxcode{\sphinxupquote{cook}}. It tells MUSE the commodity type, name, emissions factor of CO2 and heat rate, amongst other things.

The example used for this tutorial is below:


\begin{savenotes}\sphinxattablestart
\centering
\begin{tabulary}{\linewidth}[t]{|T|T|T|T|T|T|}
\hline
\sphinxstyletheadfamily 
Commodity
&\sphinxstyletheadfamily 
CommodityType
&\sphinxstyletheadfamily 
CommodityName
&\sphinxstyletheadfamily 
CommodityEmissionFactor\_CO2
&\sphinxstyletheadfamily 
HeatRate
&\sphinxstyletheadfamily 
Unit
\\
\hline
Electricity
&
Energy
&
electricity
&
0
&
1
&
PJ
\\
\hline
Gas
&
Energy
&
gas
&
56.1
&
1
&
PJ
\\
\hline
Heat
&
Energy
&
heat
&
0
&
1
&
PJ
\\
\hline
Wind
&
Energy
&
wind
&
0
&
1
&
PJ
\\
\hline
CO2fuelcomsbustion
&
Environmental
&
CO2f
&
0
&
1
&
kt
\\
\hline
Solar
&
Energy
&
solar
&
0
&
1
&
PJ
\\
\hline
\sphinxstylestrong{Cook}
&
\sphinxstylestrong{Energy}
&
\sphinxstylestrong{cook}
&
\sphinxstylestrong{0}
&
\sphinxstylestrong{1}
&
\sphinxstylestrong{PJ}
\\
\hline
\end{tabulary}
\par
\sphinxattableend\end{savenotes}

Finally, the \sphinxcode{\sphinxupquote{Projections.csv}} file must be changed. This is a large file which details the expected cost of the technology in the first year of the simulation. Due to its size, we will only show two rows of the new column \sphinxcode{\sphinxupquote{cook}}.


\begin{savenotes}\sphinxattablestart
\centering
\begin{tabular}[t]{|*{5}{\X{1}{5}|}}
\hline
\sphinxstyletheadfamily 
RegionName
&\sphinxstyletheadfamily 
Attribute
&\sphinxstyletheadfamily 
Time
&\sphinxstyletheadfamily 
…
&\sphinxstyletheadfamily 
\sphinxstylestrong{cook}
\\
\hline
Unit
&\begin{itemize}
\item {} 
\end{itemize}
&
Year
&
…
&
\sphinxstylestrong{MUS\$2010/kt}
\\
\hline
R1
&
CommodityPrice
&
2010
&
…
&
\sphinxstylestrong{100}
\\
\hline
…
&
…
&
…
&
…
&
\sphinxstylestrong{…}
\\
\hline
R2
&
CommodityPrice
&
2100
&
…
&
\sphinxstylestrong{100}
\\
\hline
\end{tabular}
\par
\sphinxattableend\end{savenotes}

We set every price of cook to be \sphinxcode{\sphinxupquote{100MUS\$2010/kt}}


\subsection{Addition of cooking technology}
\label{\detokenize{user-guide/addition-service-demand:Addition-of-cooking-technology}}
Next, we must add a technology to service this new demand. This is achieved through a similar process as the section in the {\hyperref[\detokenize{user-guide/add-solar::doc}]{\sphinxcrossref{\DUrole{doc}{“adding a new technology”}}}} section. However, we must be careful to specify the end\sphinxhyphen{}use of the technology as \sphinxcode{\sphinxupquote{cook}}.

For this example, we will add two competing technologies to service the cooking demand: \sphinxcode{\sphinxupquote{electric\_stove}} and \sphinxcode{\sphinxupquote{gas\_stove}} to the \sphinxcode{\sphinxupquote{Technodata.csv}} file in \sphinxcode{\sphinxupquote{/technodata/residential/Technodata.csv}}.

Again for the interests of space, we have omitted the existing \sphinxcode{\sphinxupquote{gasboiler}} and \sphinxcode{\sphinxupquote{heatpump}} technologies. But we copy the \sphinxcode{\sphinxupquote{gasboiler}} row for \sphinxcode{\sphinxupquote{R1}} and paste it for the new \sphinxcode{\sphinxupquote{electric\_stove}} for both \sphinxcode{\sphinxupquote{R1}} and \sphinxcode{\sphinxupquote{R2}}. For \sphinxcode{\sphinxupquote{gas\_stove}} we copy and paste the data for \sphinxcode{\sphinxupquote{heatpump}} from region \sphinxcode{\sphinxupquote{R1}} for both \sphinxcode{\sphinxupquote{R1}} and \sphinxcode{\sphinxupquote{R2}}.

An important modification, however, is specifying the end\sphinxhyphen{}use for these new technologies to be \sphinxcode{\sphinxupquote{cook}} and not \sphinxcode{\sphinxupquote{heat}}.


\begin{savenotes}\sphinxattablestart
\centering
\begin{tabular}[t]{|*{10}{\X{1}{10}|}}
\hline
\sphinxstyletheadfamily 
ProcessName
&\sphinxstyletheadfamily 
RegionName
&\sphinxstyletheadfamily 
Time
&\sphinxstyletheadfamily 
Level
&\sphinxstyletheadfamily 
cap\_par
&\sphinxstyletheadfamily 
…
&\sphinxstyletheadfamily 
Fuel
&\sphinxstyletheadfamily 
EndUse
&\sphinxstyletheadfamily 
Agent2
&\sphinxstyletheadfamily 
Agent1
\\
\hline
Unit
&\begin{itemize}
\item {} 
\end{itemize}
&
Year
&\begin{itemize}
\item {} 
\end{itemize}
&
MUS\$2010/PJ\_a
&
…
&\begin{itemize}
\item {} 
\end{itemize}
&\begin{itemize}
\item {} 
\end{itemize}
&
Retrofit
&
New
\\
\hline
gasboiler
&
R1
&
2020
&
fixed
&
3.8
&
…
&
gas
&
heat
&
1
&
0
\\
\hline
…
&
…
&
…
&
…
&
…
&
…
&
…
&
…
&
…
&
…
\\
\hline
\sphinxstylestrong{electric\_stove}
&
\sphinxstylestrong{R1}
&
\sphinxstylestrong{2020}
&
\sphinxstylestrong{fixed}
&
\sphinxstylestrong{3.8}
&
\sphinxstylestrong{…}
&
\sphinxstylestrong{electricity}
&
\sphinxstylestrong{cook}
&
\sphinxstylestrong{1}
&
\sphinxstylestrong{0}
\\
\hline
\sphinxstylestrong{electric\_stove}
&
\sphinxstylestrong{R2}
&
\sphinxstylestrong{2020}
&
\sphinxstylestrong{fixed}
&
\sphinxstylestrong{3.8}
&
\sphinxstylestrong{…}
&
\sphinxstylestrong{electricity}
&
\sphinxstylestrong{cook}
&
\sphinxstylestrong{1}
&
\sphinxstylestrong{0}
\\
\hline
\sphinxstylestrong{gas\_stove}
&
\sphinxstylestrong{R1}
&
\sphinxstylestrong{2020}
&
\sphinxstylestrong{fixed}
&
\sphinxstylestrong{8.8667}
&
\sphinxstylestrong{…}
&
\sphinxstylestrong{gas}
&
\sphinxstylestrong{cook}
&
\sphinxstylestrong{1}
&
\sphinxstylestrong{0}
\\
\hline
\sphinxstylestrong{gas\_stove}
&
\sphinxstylestrong{R2}
&
\sphinxstylestrong{2020}
&
\sphinxstylestrong{fixed}
&
\sphinxstylestrong{8.8667}
&
\sphinxstylestrong{…}
&
\sphinxstylestrong{gas}
&
\sphinxstylestrong{cook}
&
\sphinxstylestrong{1}
&
\sphinxstylestrong{0}
\\
\hline
\end{tabular}
\par
\sphinxattableend\end{savenotes}

As can be seen we have added two technologies, in the two regions with different \sphinxcode{\sphinxupquote{cap\_par}} costs. We specified their respective fules, and the enduse for both is \sphinxcode{\sphinxupquote{cook}}. For the full file please see \sphinxhref{here}{here INSERT LINK HERE}.

We must also add the data for these new technologies to the following files:
\begin{itemize}
\item {} 
\sphinxcode{\sphinxupquote{CommIn.csv}}

\item {} 
\sphinxcode{\sphinxupquote{CommOut.csv}}

\item {} 
\sphinxcode{\sphinxupquote{ExistingCapacity.csv}}

\end{itemize}

This is largely a similar process to the tutorial shown in {\hyperref[\detokenize{user-guide/add-solar::doc}]{\sphinxcrossref{\DUrole{doc}{“adding a new technology”}}}}. We must add the input to each of the technologies (gas and electricity for \sphinxcode{\sphinxupquote{gas\_stove}} and \sphinxcode{\sphinxupquote{electric\_stove}} respectively), outputs of \sphinxcode{\sphinxupquote{cook}} for both and the existing capacity for each technology in each region.

Due to the additional demand for gas and electricity brought on by the new \sphinxcode{\sphinxupquote{cook}} demand, it is necessary to relax the growth constraints for \sphinxcode{\sphinxupquote{gassupply1}} in the \sphinxcode{\sphinxupquote{technodata/gas/technodata.csv}} file. For this example, we set this file as follows:


\begin{savenotes}\sphinxattablestart
\centering
\begin{tabular}[t]{|*{9}{\X{1}{9}|}}
\hline
\sphinxstyletheadfamily 
ProcessName
&\sphinxstyletheadfamily 
RegionName
&\sphinxstyletheadfamily 
Time
&\sphinxstyletheadfamily 
…
&\sphinxstyletheadfamily 
MaxCapacityAddition
&\sphinxstyletheadfamily 
MaxCapacityGrowth
&\sphinxstyletheadfamily 
TotalCapacityLimit
&\sphinxstyletheadfamily 
…
&\sphinxstyletheadfamily 
Agent1
\\
\hline
Unit
&\begin{itemize}
\item {} 
\end{itemize}
&
Year
&
…
&
PJ
&
\%
&
PJ
&
…
&
New
\\
\hline
gassupply1
&
R1
&
2020
&
…
&
\sphinxstylestrong{100}
&
\sphinxstylestrong{5}
&
\sphinxstylestrong{500}
&
…
&
0
\\
\hline
gassupply1
&
R2
&
2020
&
…
&
\sphinxstylestrong{100}
&
\sphinxstylestrong{5}
&
\sphinxstylestrong{120}
&
…
&
0
\\
\hline
\end{tabular}
\par
\sphinxattableend\end{savenotes}

To prevent repetition of the {\hyperref[\detokenize{user-guide/add-solar::doc}]{\sphinxcrossref{\DUrole{doc}{“adding a new technology”}}}} section, we will leave the full files \sphinxhref{link-here}{here INSERT LINK HERE}.

Again, we run the simulation with our modified input files using the following command, in the relevant directory:

\begin{sphinxVerbatim}[commandchars=\\\{\}]
\PYG{n}{python} \PYG{o}{\PYGZhy{}}\PYG{n}{m} \PYG{n}{pip} \PYG{n}{muse} \PYG{n}{settings}\PYG{o}{.}\PYG{n}{toml}
\end{sphinxVerbatim}

Once this has run we are ready to visualise our results.

{
\sphinxsetup{VerbatimColor={named}{nbsphinx-code-bg}}
\sphinxsetup{VerbatimBorderColor={named}{nbsphinx-code-border}}
\begin{sphinxVerbatim}[commandchars=\\\{\}]
\llap{\color{nbsphinxin}[2]:\,\hspace{\fboxrule}\hspace{\fboxsep}}\PYG{k+kn}{import} \PYG{n+nn}{pandas} \PYG{k}{as} \PYG{n+nn}{pd}
\PYG{k+kn}{import} \PYG{n+nn}{seaborn} \PYG{k}{as} \PYG{n+nn}{sns}
\PYG{k+kn}{import} \PYG{n+nn}{matplotlib}\PYG{n+nn}{.}\PYG{n+nn}{pyplot} \PYG{k}{as} \PYG{n+nn}{plt}
\end{sphinxVerbatim}
}

{
\sphinxsetup{VerbatimColor={named}{nbsphinx-code-bg}}
\sphinxsetup{VerbatimBorderColor={named}{nbsphinx-code-border}}
\begin{sphinxVerbatim}[commandchars=\\\{\}]
\llap{\color{nbsphinxin}[3]:\,\hspace{\fboxrule}\hspace{\fboxsep}}\PYG{n}{mca\PYGZus{}capacity} \PYG{o}{=} \PYG{n}{pd}\PYG{o}{.}\PYG{n}{read\PYGZus{}csv}\PYG{p}{(}\PYG{l+s+s2}{\PYGZdq{}}\PYG{l+s+s2}{../tutorial\PYGZhy{}code/add\PYGZhy{}service\PYGZhy{}demand/Results/MCACapacity.csv}\PYG{l+s+s2}{\PYGZdq{}}\PYG{p}{)}
\PYG{n}{mca\PYGZus{}capacity}\PYG{o}{.}\PYG{n}{head}\PYG{p}{(}\PYG{p}{)}


\end{sphinxVerbatim}
}

{

\kern-\sphinxverbatimsmallskipamount\kern-\baselineskip
\kern+\FrameHeightAdjust\kern-\fboxrule
\vspace{\nbsphinxcodecellspacing}

\sphinxsetup{VerbatimColor={named}{white}}
\sphinxsetup{VerbatimBorderColor={named}{nbsphinx-code-border}}
\begin{sphinxVerbatim}[commandchars=\\\{\}]
\llap{\color{nbsphinxout}[3]:\,\hspace{\fboxrule}\hspace{\fboxsep}}  technology region agent      type       sector  capacity  year
0  gas\_stove     R1    A1  retrofit  residential      10.0  2020
1  gasboiler     R1    A1  retrofit  residential      10.0  2020
2  gas\_stove     R2    A1  retrofit  residential      10.0  2020
3  gasboiler     R2    A1  retrofit  residential      10.0  2020
4  gas\_stove     R1    A2  retrofit  residential      10.0  2020
\end{sphinxVerbatim}
}

{
\sphinxsetup{VerbatimColor={named}{nbsphinx-code-bg}}
\sphinxsetup{VerbatimBorderColor={named}{nbsphinx-code-border}}
\begin{sphinxVerbatim}[commandchars=\\\{\}]
\llap{\color{nbsphinxin}[4]:\,\hspace{\fboxrule}\hspace{\fboxsep}}\PYG{k}{for} \PYG{n}{name}\PYG{p}{,} \PYG{n}{sector} \PYG{o+ow}{in} \PYG{n}{mca\PYGZus{}capacity}\PYG{o}{.}\PYG{n}{groupby}\PYG{p}{(}\PYG{l+s+s2}{\PYGZdq{}}\PYG{l+s+s2}{sector}\PYG{l+s+s2}{\PYGZdq{}}\PYG{p}{)}\PYG{p}{:}
    \PYG{n+nb}{print}\PYG{p}{(}\PYG{l+s+s2}{\PYGZdq{}}\PYG{l+s+si}{\PYGZob{}\PYGZcb{}}\PYG{l+s+s2}{ sector:}\PYG{l+s+s2}{\PYGZdq{}}\PYG{o}{.}\PYG{n}{format}\PYG{p}{(}\PYG{n}{name}\PYG{p}{)}\PYG{p}{)}
    \PYG{n}{fig}\PYG{p}{,} \PYG{n}{ax} \PYG{o}{=}\PYG{n}{plt}\PYG{o}{.}\PYG{n}{subplots}\PYG{p}{(}\PYG{l+m+mi}{1}\PYG{p}{,}\PYG{l+m+mi}{2}\PYG{p}{)}
    \PYG{n}{sns}\PYG{o}{.}\PYG{n}{lineplot}\PYG{p}{(}\PYG{n}{data}\PYG{o}{=}\PYG{n}{sector}\PYG{p}{[}\PYG{n}{sector}\PYG{o}{.}\PYG{n}{region}\PYG{o}{==}\PYG{l+s+s2}{\PYGZdq{}}\PYG{l+s+s2}{R1}\PYG{l+s+s2}{\PYGZdq{}}\PYG{p}{]}\PYG{p}{,} \PYG{n}{x}\PYG{o}{=}\PYG{l+s+s2}{\PYGZdq{}}\PYG{l+s+s2}{year}\PYG{l+s+s2}{\PYGZdq{}}\PYG{p}{,} \PYG{n}{y}\PYG{o}{=}\PYG{l+s+s2}{\PYGZdq{}}\PYG{l+s+s2}{capacity}\PYG{l+s+s2}{\PYGZdq{}}\PYG{p}{,} \PYG{n}{hue}\PYG{o}{=}\PYG{l+s+s2}{\PYGZdq{}}\PYG{l+s+s2}{technology}\PYG{l+s+s2}{\PYGZdq{}}\PYG{p}{,} \PYG{n}{ax}\PYG{o}{=}\PYG{n}{ax}\PYG{p}{[}\PYG{l+m+mi}{0}\PYG{p}{]}\PYG{p}{)}
    \PYG{n}{sns}\PYG{o}{.}\PYG{n}{lineplot}\PYG{p}{(}\PYG{n}{data}\PYG{o}{=}\PYG{n}{sector}\PYG{p}{[}\PYG{n}{sector}\PYG{o}{.}\PYG{n}{region}\PYG{o}{==}\PYG{l+s+s2}{\PYGZdq{}}\PYG{l+s+s2}{R2}\PYG{l+s+s2}{\PYGZdq{}}\PYG{p}{]}\PYG{p}{,} \PYG{n}{x}\PYG{o}{=}\PYG{l+s+s2}{\PYGZdq{}}\PYG{l+s+s2}{year}\PYG{l+s+s2}{\PYGZdq{}}\PYG{p}{,} \PYG{n}{y}\PYG{o}{=}\PYG{l+s+s2}{\PYGZdq{}}\PYG{l+s+s2}{capacity}\PYG{l+s+s2}{\PYGZdq{}}\PYG{p}{,} \PYG{n}{hue}\PYG{o}{=}\PYG{l+s+s2}{\PYGZdq{}}\PYG{l+s+s2}{technology}\PYG{l+s+s2}{\PYGZdq{}}\PYG{p}{,} \PYG{n}{ax}\PYG{o}{=}\PYG{n}{ax}\PYG{p}{[}\PYG{l+m+mi}{1}\PYG{p}{]}\PYG{p}{)}
    \PYG{n}{plt}\PYG{o}{.}\PYG{n}{show}\PYG{p}{(}\PYG{p}{)}
    \PYG{n}{plt}\PYG{o}{.}\PYG{n}{close}\PYG{p}{(}\PYG{p}{)}
\end{sphinxVerbatim}
}

{

\kern-\sphinxverbatimsmallskipamount\kern-\baselineskip
\kern+\FrameHeightAdjust\kern-\fboxrule
\vspace{\nbsphinxcodecellspacing}

\sphinxsetup{VerbatimColor={named}{white}}
\sphinxsetup{VerbatimBorderColor={named}{nbsphinx-code-border}}
\begin{sphinxVerbatim}[commandchars=\\\{\}]
gas sector:
\end{sphinxVerbatim}
}

\hrule height -\fboxrule\relax
\vspace{\nbsphinxcodecellspacing}

\makeatletter\setbox\nbsphinxpromptbox\box\voidb@x\makeatother

\begin{nbsphinxfancyoutput}

\noindent\sphinxincludegraphics[width=388\sphinxpxdimen,height=262\sphinxpxdimen]{{user-guide_addition-service-demand_15_1}.png}

\end{nbsphinxfancyoutput}

{

\kern-\sphinxverbatimsmallskipamount\kern-\baselineskip
\kern+\FrameHeightAdjust\kern-\fboxrule
\vspace{\nbsphinxcodecellspacing}

\sphinxsetup{VerbatimColor={named}{white}}
\sphinxsetup{VerbatimBorderColor={named}{nbsphinx-code-border}}
\begin{sphinxVerbatim}[commandchars=\\\{\}]
power sector:
\end{sphinxVerbatim}
}

\hrule height -\fboxrule\relax
\vspace{\nbsphinxcodecellspacing}

\makeatletter\setbox\nbsphinxpromptbox\box\voidb@x\makeatother

\begin{nbsphinxfancyoutput}

\noindent\sphinxincludegraphics[width=395\sphinxpxdimen,height=262\sphinxpxdimen]{{user-guide_addition-service-demand_15_3}.png}

\end{nbsphinxfancyoutput}

{

\kern-\sphinxverbatimsmallskipamount\kern-\baselineskip
\kern+\FrameHeightAdjust\kern-\fboxrule
\vspace{\nbsphinxcodecellspacing}

\sphinxsetup{VerbatimColor={named}{white}}
\sphinxsetup{VerbatimBorderColor={named}{nbsphinx-code-border}}
\begin{sphinxVerbatim}[commandchars=\\\{\}]
residential sector:
\end{sphinxVerbatim}
}

\hrule height -\fboxrule\relax
\vspace{\nbsphinxcodecellspacing}

\makeatletter\setbox\nbsphinxpromptbox\box\voidb@x\makeatother

\begin{nbsphinxfancyoutput}

\noindent\sphinxincludegraphics[width=388\sphinxpxdimen,height=262\sphinxpxdimen]{{user-guide_addition-service-demand_15_5}.png}

\end{nbsphinxfancyoutput}

We can see our new technology, the \sphinxcode{\sphinxupquote{gas\_stove}} is used over the \sphinxcode{\sphinxupquote{electric\_stove}}. Therefore, there is an increase in \sphinxcode{\sphinxupquote{gassupply1}} to accommodate for this growth in demand. However, this is not enough to displace \sphinxcode{\sphinxupquote{windturbine}} by \sphinxcode{\sphinxupquote{gasCCGT}}.


\subsection{Next steps}
\label{\detokenize{user-guide/addition-service-demand:Next-steps}}
This brings us to the end of the user guide! Using the information explained in this tutorial, or following similar steps, you will be able to create complex scenarios of your choosing.

For the full code to generate the final results, see \sphinxhref{dead-link}{here INSERT LINK HERE}.


\chapter{Input Files}
\label{\detokenize{inputs/index:input-files}}\label{\detokenize{inputs/index:id1}}\label{\detokenize{inputs/index::doc}}
In this section we detail each of the files required to run MUSE. We include information based on how these files should be used, as well as the data that populates them.


\section{TOML primer}
\label{\detokenize{inputs/toml_primer:toml-primer}}\label{\detokenize{inputs/toml_primer:id1}}\label{\detokenize{inputs/toml_primer::doc}}
The full specification for TOML files can be found
\sphinxhref{https://github.com/toml-lang/toml}{here}.
A TOML file is separated into sections, with each section except the topmost
introduced by a name in square brackets. Sections can hold key\sphinxhyphen{}value pairs,
e.g. a name associated with a value. For instance:

\begin{sphinxVerbatim}[commandchars=\\\{\}]
\PYG{n}{general\PYGZus{}string\PYGZus{}attribute} \PYG{o}{=} \PYG{l+s}{\PYGZdq{}x\PYGZdq{}}

\PYG{k}{[some\PYGZus{}section]}
\PYG{n}{section\PYGZus{}attribute} \PYG{o}{=} \PYG{l+m+mi}{12}

\PYG{k}{[some\PYGZus{}section.subsection]}
\PYG{n}{subsetion\PYGZus{}attribute} \PYG{o}{=} \PYG{k+kc}{true}
\end{sphinxVerbatim}

TOML is quite flexible in how one can define sections and attributes. The following
three examples are equivalent:

\begin{sphinxVerbatim}[commandchars=\\\{\}]
\PYG{k}{[sectors.residential.production]}
\PYG{n}{name} \PYG{o}{=} \PYG{l+s}{\PYGZdq{}match\PYGZdq{}}
\PYG{n}{costing} \PYG{o}{=} \PYG{l+s}{\PYGZdq{}prices\PYGZdq{}}
\end{sphinxVerbatim}

\begin{sphinxVerbatim}[commandchars=\\\{\}]
\PYG{k}{[sectors.residential]}
\PYG{n}{production} \PYG{o}{=} \PYG{p}{\PYGZob{}}\PYG{l+s}{\PYGZdq{}name\PYGZdq{}}\PYG{p}{:} \PYG{l+s}{\PYGZdq{}match\PYGZdq{}}\PYG{p}{,} \PYG{l+s}{\PYGZdq{}costing\PYGZdq{}}\PYG{p}{:} \PYG{l+s}{\PYGZdq{}prices\PYGZdq{}}\PYG{p}{\PYGZcb{}}
\end{sphinxVerbatim}

\begin{sphinxVerbatim}[commandchars=\\\{\}]
\PYG{k}{[sectors.residential]}
\PYG{n}{production}\PYG{p}{.}\PYG{n}{name} \PYG{o}{=} \PYG{l+s}{\PYGZdq{}match\PYGZdq{}}
\PYG{n}{production}\PYG{p}{.}\PYG{n}{costing} \PYG{o}{=} \PYG{l+s}{\PYGZdq{}prices\PYGZdq{}}
\end{sphinxVerbatim}
\phantomsection\label{\detokenize{inputs/toml_primer:toml-array}}
Additionally, TOML files can contain tabular data, specified row\sphinxhyphen{}by\sphinxhyphen{}row using double
square bracket. For instance, below we define a table with two rows and a single
\sphinxstyleemphasis{column} called \sphinxtitleref{some\_table\_of\_data} (though column is not quite the right term, TOML tables are made more
flexible than most tabular formats. Rather, each row can be considered a
dictionary).

\begin{sphinxVerbatim}[commandchars=\\\{\}]
\PYG{k}{[[some\PYGZus{}table\PYGZus{}of\PYGZus{}data]]}
\PYG{n}{a\PYGZus{}key} \PYG{o}{=} \PYG{l+s}{\PYGZdq{}a value\PYGZdq{}}

\PYG{k}{[[some\PYGZus{}table\PYGZus{}of\PYGZus{}data]]}
\PYG{n}{a\PYGZus{}key} \PYG{o}{=} \PYG{l+s}{\PYGZdq{}another value\PYGZdq{}}
\end{sphinxVerbatim}

As MUSE requires a number of data file, paths to files can be formated in a flexible manner. Paths can be formatted with shorthands for specific directories and are defined with curly\sphinxhyphen{}brackets. For example:

\begin{sphinxVerbatim}[commandchars=\\\{\}]
\PYG{n}{projection} \PYG{o}{=} \PYG{l+s}{\PYGZsq{}\PYGZob{}path\PYGZcb{}/inputs/projection.csv\PYGZsq{}}
\PYG{n}{timeslices\PYGZus{}path} \PYG{o}{=} \PYG{l+s}{\PYGZsq{}\PYGZob{}cwd\PYGZcb{}/technodata/timeslices.csv\PYGZsq{}}
\PYG{n}{consumption\PYGZus{}path} \PYG{o}{=} \PYG{l+s}{\PYGZsq{}\PYGZob{}muse\PYGZus{}sectors\PYGZcb{}/technodata/timeslices.csv\PYGZsq{}}
\end{sphinxVerbatim}
\begin{description}
\item[{path}] \leavevmode
refers to the directory where the TOML file is located

\item[{cwd}] \leavevmode
refers to the directory from which the muse simulation is launched

\item[{muse\_sectors}] \leavevmode
refers to the directory where default sectoral data is located

\end{description}


\section{Simulation settings}
\label{\detokenize{inputs/toml:simulation-settings}}\label{\detokenize{inputs/toml:id1}}\label{\detokenize{inputs/toml::doc}}
This section details the TOML input for MUSE. The format for TOML files is
described in this {\hyperref[\detokenize{inputs/toml_primer:toml-primer}]{\sphinxcrossref{\DUrole{std,std-ref}{previous section}}}}. Here, however, we focus on sections and
attributes that are specific to MUSE.

The TOML file can be read using \sphinxcode{\sphinxupquote{read\_settings()}}. The resulting
data is used to construt the market clearing algorithm directly in the \sphinxcode{\sphinxupquote{MCA\textquotesingle{}s
factory function}}.


\subsection{Main section}
\label{\detokenize{inputs/toml:main-section}}
This is the topmost section. It contains settings relevant to the simulation as
a whole.

\begin{sphinxVerbatim}[commandchars=\\\{\}]
\PYG{n}{time\PYGZus{}framework} \PYG{o}{=} \PYG{k}{[2020, 2025, 2030, 2035, 2040, 2045, 2050]}
\PYG{n}{regions} \PYG{o}{=} \PYG{k}{[\PYGZdq{}USA\PYGZdq{}]}
\PYG{n}{interpolation\PYGZus{}mode} \PYG{o}{=} \PYG{l+s}{\PYGZsq{}Active\PYGZsq{}}
\PYG{n}{log\PYGZus{}level} \PYG{o}{=} \PYG{l+s}{\PYGZsq{}info\PYGZsq{}}

\PYG{n}{equilibrium\PYGZus{}variable} \PYG{o}{=} \PYG{l+s}{\PYGZsq{}demand\PYGZsq{}}
\PYG{n}{maximum\PYGZus{}iterations} \PYG{o}{=} \PYG{l+m+mi}{100}
\PYG{n}{tolerance} \PYG{o}{=} \PYG{l+m+mf}{0.1}
\PYG{n}{tolerance\PYGZus{}unmet\PYGZus{}demand} \PYG{o}{=} \PYG{l+m+mi}{\PYGZhy{}0}\PYG{l+m+mf}{.1}
\end{sphinxVerbatim}
\begin{description}
\item[{time\_framework}] \leavevmode
Required. List of years for which the simulation will run.

\item[{region}] \leavevmode
Subset of regions to consider. If not given, defaults to all regions found in the
simulation data.

\item[{interpolation\_mode}] \leavevmode
interpolation when reading the initial market. One of
\sphinxtitleref{linear}, \sphinxtitleref{nearest}, \sphinxtitleref{zero}, \sphinxtitleref{slinear}, \sphinxtitleref{quadratic}, \sphinxtitleref{cubic}. Defaults to \sphinxtitleref{linear}.

\item[{log\_level:}] \leavevmode
verbosity of the output.

\item[{equilibirum\_variable}] \leavevmode
whether equilibrium of \sphinxtitleref{demand} or \sphinxtitleref{prices} should be sought. Defaults to \sphinxtitleref{demand}.

\item[{maximum\_iterations}] \leavevmode
Maximum number of iterations when searching for equilibrium. Defaults to 3.

\item[{tolerance}] \leavevmode
Tolerance criteria when checking for equilibrium. Defaults to 0.1.

\item[{tolerance\_unmet\_demand}] \leavevmode
Criteria checking whether the demand has been met.  Defaults to \sphinxhyphen{}0.1.

\item[{excluded\_commodities}] \leavevmode
List of commodities excluded from the equilibrium considerations. Defaults to the
list \sphinxtitleref{{[}“CO2f”, “CO2r”, “CO2c”, “CO2s”, “CH4”, “N2O”, “f\sphinxhyphen{}gases”{]}}.

\item[{plugins}] \leavevmode
Path or list of paths to extra python plugins, i.e. files with registered functions
such as \sphinxcode{\sphinxupquote{register\_output\_quantity()}}.

\end{description}


\subsection{Carbon market}
\label{\detokenize{inputs/toml:carbon-market}}
This section containts the settings related to the modelling of the carbon market. If omitted, it defaults to not
including the carbon market in the simulation.

Example

\begin{sphinxVerbatim}[commandchars=\\\{\}]
\PYG{k}{[carbon\PYGZus{}budget\PYGZus{}control]}
\PYG{n}{budget} \PYG{o}{=} \PYG{k}{[]}
\end{sphinxVerbatim}
\begin{description}
\item[{budget}] \leavevmode
Yearly budget. There should be one item for each year the simulation will run. In
other words, if given and not empty, this is a list with the same length as
\sphinxtitleref{time\_framework} from the main section. If not given or an empty list, then the
carbon market feature is disabled. Defaults to an empty list.

\item[{method}] \leavevmode
Method used to equilibrate the carbon market. Defaults to a simple iterative scheme. {[}INSERT OPTIONS HERE{]}

\item[{commodities}] \leavevmode
Commodities that make up the carbon market. Defaults to an empty list.

\item[{control\_undershoot}] \leavevmode
Whether to control carbon budget undershoots. Defaults to True.

\item[{control\_overshoot}] \leavevmode
Whether to control carbon budget overshoots. Defaults to True.

\item[{method\_options:}] \leavevmode
Additional options for the specific carbon method.

\end{description}


\subsection{Global input files}
\label{\detokenize{inputs/toml:global-input-files}}
Defines the paths specific simulation data files. The paths can be formatted as
explained in the {\hyperref[\detokenize{inputs/toml_primer:toml-primer}]{\sphinxcrossref{\DUrole{std,std-ref}{TOML primer}}}}.

\begin{sphinxVerbatim}[commandchars=\\\{\}]
\PYG{k}{[global\PYGZus{}input\PYGZus{}files]}
\PYG{n}{projections} \PYG{o}{=} \PYG{l+s}{\PYGZsq{}\PYGZob{}path\PYGZcb{}/inputs/Projections.csv\PYGZsq{}}
\PYG{n}{regions} \PYG{o}{=} \PYG{l+s}{\PYGZsq{}\PYGZob{}path\PYGZcb{}/inputs/Regions.csv\PYGZsq{}}
\PYG{n}{global\PYGZus{}commodities} \PYG{o}{=} \PYG{l+s}{\PYGZsq{}\PYGZob{}path\PYGZcb{}/inputs/MUSEGlobalCommodities.csv\PYGZsq{}}
\end{sphinxVerbatim}
\begin{description}
\item[{projections:}] \leavevmode
Path to a csv file giving initial market projection. See {\hyperref[\detokenize{inputs/projections:inputs-projection}]{\sphinxcrossref{\DUrole{std,std-ref}{Initial Market Projection}}}}.

\item[{regions:}] \leavevmode
Path to a csv file describing the regions. See
{\hyperref[\detokenize{inputs/regions:regional-data}]{\sphinxcrossref{\DUrole{std,std-ref}{Regional data}}}}.

\item[{global\_commodities:}] \leavevmode
Path to a csv file describing the comodities in the simulation. See
{\hyperref[\detokenize{inputs/commodities:inputs-commodities}]{\sphinxcrossref{\DUrole{std,std-ref}{Commodity Description}}}}.

\end{description}


\subsection{Timeslices}
\label{\detokenize{inputs/toml:timeslices}}\label{\detokenize{inputs/toml:timeslices-toml}}
Time\sphinxhyphen{}slices represent a sub\sphinxhyphen{}year disaggregation of commodity demand. Generally,
timeslices are expected to introduce several levels, e.g. season, day, or hour. The
simplest is to show the TOML for the default timeslice:

\begin{sphinxVerbatim}[commandchars=\\\{\}]
\PYG{k}{[timeslices]}
\PYG{n}{winter}\PYG{p}{.}\PYG{n}{weekday}\PYG{p}{.}\PYG{n}{night} \PYG{o}{=} \PYG{l+m+mi}{396}
\PYG{n}{winter}\PYG{p}{.}\PYG{n}{weekday}\PYG{p}{.}\PYG{n}{morning} \PYG{o}{=} \PYG{l+m+mi}{396}
\PYG{n}{winter}\PYG{p}{.}\PYG{n}{weekday}\PYG{p}{.}\PYG{n}{afternoon} \PYG{o}{=} \PYG{l+m+mi}{264}
\PYG{n}{winter}\PYG{p}{.}\PYG{n}{weekday}\PYG{p}{.}\PYG{n}{early\PYGZhy{}peak} \PYG{o}{=} \PYG{l+m+mi}{66}
\PYG{n}{winter}\PYG{p}{.}\PYG{n}{weekday}\PYG{p}{.}\PYG{n}{late\PYGZhy{}peak} \PYG{o}{=} \PYG{l+m+mi}{66}
\PYG{n}{winter}\PYG{p}{.}\PYG{n}{weekday}\PYG{p}{.}\PYG{n}{evening} \PYG{o}{=} \PYG{l+m+mi}{396}
\PYG{n}{winter}\PYG{p}{.}\PYG{n}{weekend}\PYG{p}{.}\PYG{n}{night} \PYG{o}{=} \PYG{l+m+mi}{156}
\PYG{n}{winter}\PYG{p}{.}\PYG{n}{weekend}\PYG{p}{.}\PYG{n}{morning} \PYG{o}{=} \PYG{l+m+mi}{156}
\PYG{n}{winter}\PYG{p}{.}\PYG{n}{weekend}\PYG{p}{.}\PYG{n}{afternoon} \PYG{o}{=} \PYG{l+m+mi}{156}
\PYG{n}{winter}\PYG{p}{.}\PYG{n}{weekend}\PYG{p}{.}\PYG{n}{evening} \PYG{o}{=} \PYG{l+m+mi}{156}
\PYG{n}{spring\PYGZhy{}autumn}\PYG{p}{.}\PYG{n}{weekday}\PYG{p}{.}\PYG{n}{night} \PYG{o}{=} \PYG{l+m+mi}{792}
\PYG{n}{spring\PYGZhy{}autumn}\PYG{p}{.}\PYG{n}{weekday}\PYG{p}{.}\PYG{n}{morning} \PYG{o}{=} \PYG{l+m+mi}{792}
\PYG{n}{spring\PYGZhy{}autumn}\PYG{p}{.}\PYG{n}{weekday}\PYG{p}{.}\PYG{n}{afternoon} \PYG{o}{=} \PYG{l+m+mi}{528}
\PYG{n}{spring\PYGZhy{}autumn}\PYG{p}{.}\PYG{n}{weekday}\PYG{p}{.}\PYG{n}{early\PYGZhy{}peak} \PYG{o}{=} \PYG{l+m+mi}{132}
\PYG{n}{spring\PYGZhy{}autumn}\PYG{p}{.}\PYG{n}{weekday}\PYG{p}{.}\PYG{n}{late\PYGZhy{}peak} \PYG{o}{=} \PYG{l+m+mi}{132}
\PYG{n}{spring\PYGZhy{}autumn}\PYG{p}{.}\PYG{n}{weekday}\PYG{p}{.}\PYG{n}{evening} \PYG{o}{=} \PYG{l+m+mi}{792}
\PYG{n}{spring\PYGZhy{}autumn}\PYG{p}{.}\PYG{n}{weekend}\PYG{p}{.}\PYG{n}{night} \PYG{o}{=} \PYG{l+m+mi}{300}
\PYG{n}{spring\PYGZhy{}autumn}\PYG{p}{.}\PYG{n}{weekend}\PYG{p}{.}\PYG{n}{morning} \PYG{o}{=} \PYG{l+m+mi}{300}
\PYG{n}{spring\PYGZhy{}autumn}\PYG{p}{.}\PYG{n}{weekend}\PYG{p}{.}\PYG{n}{afternoon} \PYG{o}{=} \PYG{l+m+mi}{300}
\PYG{n}{spring\PYGZhy{}autumn}\PYG{p}{.}\PYG{n}{weekend}\PYG{p}{.}\PYG{n}{evening} \PYG{o}{=} \PYG{l+m+mi}{300}
\PYG{n}{summer}\PYG{p}{.}\PYG{n}{weekday}\PYG{p}{.}\PYG{n}{night} \PYG{o}{=} \PYG{l+m+mi}{396}
\PYG{n}{summer}\PYG{p}{.}\PYG{n}{weekday}\PYG{p}{.}\PYG{n}{morning}  \PYG{o}{=} \PYG{l+m+mi}{396}
\PYG{n}{summer}\PYG{p}{.}\PYG{n}{weekday}\PYG{p}{.}\PYG{n}{afternoon} \PYG{o}{=} \PYG{l+m+mi}{264}
\PYG{n}{summer}\PYG{p}{.}\PYG{n}{weekday}\PYG{p}{.}\PYG{n}{early\PYGZhy{}peak} \PYG{o}{=} \PYG{l+m+mi}{66}
\PYG{n}{summer}\PYG{p}{.}\PYG{n}{weekday}\PYG{p}{.}\PYG{n}{late\PYGZhy{}peak} \PYG{o}{=} \PYG{l+m+mi}{66}
\PYG{n}{summer}\PYG{p}{.}\PYG{n}{weekday}\PYG{p}{.}\PYG{n}{evening} \PYG{o}{=} \PYG{l+m+mi}{396}
\PYG{n}{summer}\PYG{p}{.}\PYG{n}{weekend}\PYG{p}{.}\PYG{n}{night} \PYG{o}{=} \PYG{l+m+mi}{150}
\PYG{n}{summer}\PYG{p}{.}\PYG{n}{weekend}\PYG{p}{.}\PYG{n}{morning} \PYG{o}{=} \PYG{l+m+mi}{150}
\PYG{n}{summer}\PYG{p}{.}\PYG{n}{weekend}\PYG{p}{.}\PYG{n}{afternoon} \PYG{o}{=} \PYG{l+m+mi}{150}
\PYG{n}{summer}\PYG{p}{.}\PYG{n}{weekend}\PYG{p}{.}\PYG{n}{evening} \PYG{o}{=} \PYG{l+m+mi}{150}
\PYG{n}{level\PYGZus{}names} \PYG{o}{=} \PYG{k}{[\PYGZdq{}month\PYGZdq{}, \PYGZdq{}day\PYGZdq{}, \PYGZdq{}hour\PYGZdq{}]}
\end{sphinxVerbatim}

This input introduces three levels, via \sphinxcode{\sphinxupquote{level\_names}}: \sphinxcode{\sphinxupquote{month}}, \sphinxcode{\sphinxupquote{day}}, \sphinxcode{\sphinxupquote{hours}}.
Other simulations may want fewer or more levels.  The \sphinxcode{\sphinxupquote{month}} level is split into
three points of data, \sphinxcode{\sphinxupquote{winter}}, \sphinxcode{\sphinxupquote{spring\sphinxhyphen{}autumn}}, \sphinxcode{\sphinxupquote{summer}}. Then \sphinxcode{\sphinxupquote{day}} splits out
weekdays from weekends, and so on. Each line indicates the number of hours for the
relevant slice. It should be noted that the slices are not a cartesian products of each
levels. For instance, there no \sphinxcode{\sphinxupquote{peak}} periods during weekends. All that matters is
that the relative weights (i.e. the number of hours) are consistent and sum up to a
year.

The input above defines the finest times slice in the code. In order to define rougher
timeslices we can introduce items in each levels that represent aggregates at that
level. By default, we have the following:

\begin{sphinxVerbatim}[commandchars=\\\{\}]
\PYG{k}{[timeslices.aggregates]}
\PYG{n}{all\PYGZhy{}day} \PYG{o}{=} \PYG{k}{[\PYGZdq{}night\PYGZdq{}, \PYGZdq{}morning\PYGZdq{}, \PYGZdq{}afternoon\PYGZdq{}, \PYGZdq{}early\PYGZhy{}peak\PYGZdq{}, \PYGZdq{}late\PYGZhy{}peak\PYGZdq{}, \PYGZdq{}evening\PYGZdq{}]}
\PYG{n}{all\PYGZhy{}week} \PYG{o}{=} \PYG{k}{[\PYGZdq{}weekday\PYGZdq{}, \PYGZdq{}weekend\PYGZdq{}]}
\PYG{n}{all\PYGZhy{}year} \PYG{o}{=} \PYG{k}{[\PYGZdq{}winter\PYGZdq{}, \PYGZdq{}summer\PYGZdq{}, \PYGZdq{}spring\PYGZhy{}autumn\PYGZdq{}]}
\end{sphinxVerbatim}

Here, \sphinxcode{\sphinxupquote{all\sphinxhyphen{}day}} aggregates the full day. However, one could potentially create
aggregates such as:

\begin{sphinxVerbatim}[commandchars=\\\{\}]
\PYG{k}{[timeslices.aggregates]}
\PYG{n}{daylight} \PYG{o}{=} \PYG{k}{[\PYGZdq{}morning\PYGZdq{}, \PYGZdq{}afternoon\PYGZdq{}, \PYGZdq{}early\PYGZhy{}peak\PYGZdq{}, \PYGZdq{}late\PYGZhy{}peak\PYGZdq{}]}
\PYG{n}{nightlife} \PYG{o}{=} \PYG{k}{[\PYGZdq{}evening\PYGZdq{}, \PYGZdq{}night\PYGZdq{}]}
\end{sphinxVerbatim}

Once the finest timeslice and its aggregates are given, it is  possible for each sector
to define the timeslice simply by refering to the slices it will use at each level.

\begin{sphinxVerbatim}[commandchars=\\\{\}]
\PYG{k}{[sectors.some\PYGZus{}sector.timeslice\PYGZus{}levels]}
\PYG{n}{day} \PYG{o}{=} \PYG{k}{[\PYGZdq{}daylight\PYGZdq{}, \PYGZdq{}nightlife\PYGZdq{}]}
\PYG{n}{month} \PYG{o}{=} \PYG{k}{[\PYGZdq{}all\PYGZhy{}year\PYGZdq{}]}
\end{sphinxVerbatim}

Above, \sphinxcode{\sphinxupquote{sectors.some\_sector.timeslice\_levels.week}} defaults its value in the finest
timeslice. Indeed, if the subsection \sphinxcode{\sphinxupquote{sectors.some\_sector.timeslice\_levels}} is not
given, then the sector will default to using the finest timeslices.

Similarly, it is possible to specify a timeslice for the mca by adding an
\sphinxtitleref{mca.timeslice\_levels} section. However, be aware that if the MCA uses a rougher
timeslice framework, the market will be expressed within it. Hence information from
sectors with a finer timeslice framework will be lost.


\subsection{Standard sectors}
\label{\detokenize{inputs/toml:standard-sectors}}
Sectors are declared in the TOML file by adding a subsection to the \sphinxtitleref{sectors} section:

\begin{sphinxVerbatim}[commandchars=\\\{\}]
\PYG{k}{[sectors.residential]}
\PYG{n}{type} \PYG{o}{=} \PYG{l+s}{\PYGZsq{}default\PYGZsq{}}
\PYG{k}{[sectors.power]}
\PYG{n}{type} \PYG{o}{=} \PYG{l+s}{\PYGZsq{}default\PYGZsq{}}
\end{sphinxVerbatim}

Above, we’ve added two sectors, residential and power. The name of the subsection is
only used for identification. In other words, it should be chosen to be meaningful to
the user, since it will not affect the model itself.

Sectors are defined in \sphinxcode{\sphinxupquote{Sector}}.

A sector accepts a number of attributes and subsections.

\phantomsection\label{\detokenize{inputs/toml:sector-type}}\begin{description}
\item[{type}] \leavevmode
Defines the kind of sector this is. \sphinxstyleemphasis{Standard} sectors are those with type
“default”. This value corresponds to the name with which a sector class is registerd
with MUSE, via \sphinxcode{\sphinxupquote{register\_sector()}}. {[}INSERT OTHER OPTIONS HERE{]}

\end{description}
\phantomsection\label{\detokenize{inputs/toml:sector-priority}}\begin{description}
\item[{priority}] \leavevmode
An integer denoting which sectors runs when. Lower values imply the sector will run
earlier. If two sectors share the same priority. Later sectors can depend on earlier
sectors for the their input. If two sectors share the same priority, then their
order is not defined. Indeed, it should indicate that they can run in parallel.
For simplicity, the keyword also accepts standard values:
\begin{itemize}
\item {} 
“preset”: 0

\item {} 
“demand”: 10

\item {} 
“conversion”: 20

\item {} 
“supply”: 30

\item {} 
“last”: 100

\end{itemize}

Defaults to “last”.

\item[{interpolation}] \leavevmode
Interpolation method user when filling in missing values. Available interpolation
methods depend on the underlying \sphinxhref{https://docs.scipy.org/doc/scipy/reference/generated/scipy.interpolate.interp1d.html}{scipy method’s kind attribute}.

\item[{investment\_production}] \leavevmode
In its simplest form, this is the name of a method to compute the production from a
sector, as used when splitting the demand across agents. In other words, this is the
computation of the production which affects future investments. In it’s more general
form, \sphinxstyleemphasis{production} can be a subsection of its own, with a “name” attribute. For
instance:

\begin{sphinxVerbatim}[commandchars=\\\{\}]
\PYG{k}{[sectors.residential.production]}
\PYG{n}{name} \PYG{o}{=} \PYG{l+s}{\PYGZdq{}match\PYGZdq{}}
\PYG{n}{costing} \PYG{o}{=} \PYG{l+s}{\PYGZdq{}prices\PYGZdq{}}
\end{sphinxVerbatim}

MUSE provides two methods in \sphinxcode{\sphinxupquote{muse.production}}:
\begin{itemize}
\item {} \begin{description}
\item[{share: the production is the maximum production for the existing capacity and}] \leavevmode
the technology’s utilization factor.
See \sphinxcode{\sphinxupquote{muse.production.maximum\_production()}}.

\end{description}

\item {} \begin{description}
\item[{match: production and demand are matched according to a given cost metric. The}] \leavevmode
cost metric defaults to “prices”. It can be modified by using the general form
given above, with a “costing” attribute. The latter can be “prices”,
“gross\_margin”, or “lcoe”.
See \sphinxcode{\sphinxupquote{muse.production.demand\_matched\_production()}}.

\end{description}

\end{itemize}

\sphinxstyleemphasis{production} can also refer to any custom production method registered with MUSE via
\sphinxcode{\sphinxupquote{muse.production.register\_production()}}.

Defaults to “share”.

\item[{dispatch\_production}] \leavevmode
The name of the production method used to compute the sector’s output, as returned
to the muse market clearing algorithm. In other words, this is computation of the
production method which will affect other sectors.

It has the same format and options as the \sphinxstyleemphasis{production} attribute above.

\item[{demand\_share}] \leavevmode
A method used to split the MCA demand into seperate parts to be serviced by specific
agents. There is currently only one option, “new\_and\_retro”, corresponding to \sphinxstyleemphasis{new}
and \sphinxstyleemphasis{retro} agents.

\item[{interactions}] \leavevmode
Defines interactions between agents. These interactions take place right before new
investments are computed. The interactions can be anything. They are expected to
modify the agents and their assets. MUSE provides a default set of interactions that
have \sphinxstyleemphasis{new} agents pass on their assets to the corresponding \sphinxstyleemphasis{retro} agent, and the
\sphinxstyleemphasis{retro} agents pass on the make\sphinxhyphen{}up of their assets to the corresponding \sphinxstyleemphasis{new}
agents.

\sphinxstyleemphasis{interactions} are specified as a {\hyperref[\detokenize{inputs/toml_primer:toml-array}]{\sphinxcrossref{\DUrole{std,std-ref}{TOML array}}}}, e.g. with double
brackets. Each sector can specify an arbitrary number of interactaction, simply by
adding an extra interaction row.

There are two orthogonal concepts to interactions:
\begin{itemize}
\item {} 
a \sphinxstyleemphasis{net} defines the set of agents that interact. A set can contain any
number of agents, whether zero, two, or all agents in a sector. See
\sphinxcode{\sphinxupquote{muse.interactions.register\_interaction\_net()}}.

\item {} 
an \sphinxstyleemphasis{interaction} defines how the net actually interacts.  See
\sphinxcode{\sphinxupquote{muse.interactions.register\_agent\_interaction()}}.

\end{itemize}

In practice, we always consider sequences of nets (i.e. more than one net) that
interact using the same interaction function.

Hence, the input looks something like the following:

\begin{sphinxVerbatim}[commandchars=\\\{\}]
\PYG{k}{[[sectors.commercial.interactions]]}
\PYG{n}{net} \PYG{o}{=} \PYG{l+s}{\PYGZsq{}new\PYGZus{}to\PYGZus{}retro\PYGZsq{}}
\PYG{n}{interaction} \PYG{o}{=} \PYG{l+s}{\PYGZsq{}transfer\PYGZsq{}}
\end{sphinxVerbatim}

“new\_to\_retro” is a function that figures out all “new/retro” pairs of agents.
Whereas “transfer” is a function that performs the transfer of assets and
information between each pair.

Furthermore, it is possible to pass parameters to either the net of the interaction
as follows:

\begin{sphinxVerbatim}[commandchars=\\\{\}]
\PYG{k}{[[sectors.commercial.interactions]]}
\PYG{n}{net} \PYG{o}{=} \PYG{p}{\PYGZob{}}\PYG{l+s}{\PYGZdq{}name\PYGZdq{}}\PYG{p}{:} \PYG{l+s}{\PYGZdq{}some\PYGZus{}net\PYGZdq{}}\PYG{p}{,} \PYG{l+s}{\PYGZdq{}param\PYGZdq{}}\PYG{p}{:} \PYG{l+s}{\PYGZdq{}some value\PYGZdq{}}\PYG{p}{\PYGZcb{}}
\PYG{n}{interaction} \PYG{o}{=} \PYG{p}{\PYGZob{}}\PYG{l+s}{\PYGZdq{}name\PYGZdq{}}\PYG{p}{:} \PYG{l+s}{\PYGZdq{}some\PYGZus{}interaction\PYGZdq{}}\PYG{p}{,} \PYG{l+s}{\PYGZdq{}param\PYGZdq{}}\PYG{p}{:} \PYG{l+s}{\PYGZdq{}some other value\PYGZdq{}}\PYG{p}{\PYGZcb{}}
\end{sphinxVerbatim}

The parameters will depend on the net and interaction functions. Neither
“new\_to\_retro” nor “transfer” take any arguments at this point. MUSE interaction
facilities are defined in \sphinxcode{\sphinxupquote{muse.interactions}}.

\item[{output}] \leavevmode
Outputs are made up of several components. MUSE is designed to allow users to
mix\sphinxhyphen{}and\sphinxhyphen{}match both how and what to save.

\sphinxstyleemphasis{output} is specified as a TOML array, e.g. with double brackets. Each sector can
specify an arbitrary number of outputs, simply by adding an extra output row.

A single row looks like this:

\begin{sphinxVerbatim}[commandchars=\\\{\}]
\PYG{k}{[[sectors.commercial.outputs]]}
\PYG{n}{filename} \PYG{o}{=} \PYG{l+s}{\PYGZsq{}\PYGZob{}cwd\PYGZcb{}/Results/\PYGZob{}Sector\PYGZcb{}/\PYGZob{}Quantity\PYGZcb{}/\PYGZob{}year\PYGZcb{}\PYGZob{}suffix\PYGZcb{}\PYGZsq{}}
\PYG{n}{quantity} \PYG{o}{=} \PYG{l+s}{\PYGZdq{}capacity\PYGZdq{}}
\PYG{n}{sink} \PYG{o}{=} \PYG{l+s}{\PYGZsq{}csv\PYGZsq{}}
\PYG{n}{overwrite} \PYG{o}{=} \PYG{k+kc}{true}
\end{sphinxVerbatim}

The following attributes are available:
\begin{itemize}
\item {} \begin{description}
\item[{quantity: Name of the quantity to save. Currently, only \sphinxtitleref{capacity} exists,}] \leavevmode
refering to \sphinxcode{\sphinxupquote{muse.outputs.capacity()}}. However, users can
customize and create further output quantities by registering with MUSE via
\sphinxcode{\sphinxupquote{muse.outputs.register\_output\_quantity()}}. See
\sphinxcode{\sphinxupquote{muse.outputs}} for more details.

\end{description}

\item {} \begin{description}
\item[{sink: the sink is the place (disk, cloud, database, etc…) and format with which}] \leavevmode
the computed quantity is saved. Currently only sinks that save to files are
implemented. The filename can specified via \sphinxtitleref{filename}, as given below. The
following sinks are available: “csv”, “netcfd”, “excel”. However, more sinks can
be added by interested users, and registered with MUSE via
\sphinxcode{\sphinxupquote{muse.outputs.register\_output\_sink()}}. See
\sphinxcode{\sphinxupquote{muse.outputs}} for more details.

\end{description}

\item {} \begin{description}
\item[{filename: defines the format of the file where to save the data. There are several}] \leavevmode
standard values that are automatically substituted:
\begin{itemize}
\item {} 
cwd: current working directory, where MUSE was started

\item {} 
path: directory where the TOML file resides

\item {} 
sector: name of the current sector (.e.g. “commercial” above)

\item {} 
Sector: capitalized name of the current sector

\item {} 
quantity: name of the quantity to save (as given by the quantity attribute)

\item {} 
Quantity: capitablized name of the quantity to save

\item {} 
year: current year

\item {} 
suffix: standard suffix/file extension of the sink

\end{itemize}

Defaults to \sphinxtitleref{\{cwd\}/\{default\_output\_dir\}/\{Sector\}/\{Quantity\}/\{year\}\{suffix\}}.

\end{description}

\item {} \begin{description}
\item[{overwrite: If \sphinxtitleref{False} MUSE will issue an error and abort, instead of}] \leavevmode
overwriting an existing file. Defaults to \sphinxtitleref{False}. This prevents important output files from being overwritten.

\end{description}

\end{itemize}

\item[{technodata}] \leavevmode
Path to a csv file containing the characterization of the technologies involved in
the sector, e.g. lifetime, capital costs, etc… See {\hyperref[\detokenize{inputs/technodata:inputs-technodata}]{\sphinxcrossref{\DUrole{std,std-ref}{Techno\sphinxhyphen{}data}}}}.

\item[{timeslice\_levels}] \leavevmode
Slices to consider in a level. If absent, defaults to the finest timeslices.  See
{\hyperref[\detokenize{inputs/toml:timeslices}]{\sphinxcrossref{Timeslices}}}

\item[{commodities\_in}] \leavevmode
Path to a csv file describing the inputs of each technology involved in the sector.
See {\hyperref[\detokenize{inputs/commodities_io:inputs-iocomms}]{\sphinxcrossref{\DUrole{std,std-ref}{General features}}}}.

\item[{commodities\_out}] \leavevmode
Path to a csv file describing the outputs of each technology involved in the sector.
See {\hyperref[\detokenize{inputs/commodities_io:inputs-ocomms}]{\sphinxcrossref{\DUrole{std,std-ref}{Output Commodities}}}}.

\item[{existing\_capacity}] \leavevmode
Path to a csv file describing the initial capacity of the sector.
See {\hyperref[\detokenize{inputs/existing_capacity:inputs-existing-capacity}]{\sphinxcrossref{\DUrole{std,std-ref}{Existing Sectoral Capacity}}}}.

\item[{agents}] \leavevmode
Path to a csv file describing the agents in the sector.
See {\hyperref[\detokenize{inputs/agents:inputs-agents}]{\sphinxcrossref{\DUrole{std,std-ref}{Agents}}}}.

\end{description}


\subsection{Preset sectors}
\label{\detokenize{inputs/toml:preset-sectors}}
The commodity production, commodity consumption and product prices of preset sectors are determined
exogeneously. They are know from the start of the simulation and are not affected by the
simulation.

Preset sectors are defined in \sphinxcode{\sphinxupquote{PresetSector}}.

The three components, production, consumption, and prices, can be set independantly and
not all three need to be set. Production and consumption default to zero, and prices
default to leaving things unchanged.

The following defines a standard preset sector where consumption is defined as a
function of macro\sphinxhyphen{}economic data, i.e. population and gdp.

\begin{sphinxVerbatim}[commandchars=\\\{\}]
\PYG{k}{[sectors.commercial\PYGZus{}presets]}
\PYG{n}{type} \PYG{o}{=} \PYG{l+s}{\PYGZsq{}presets\PYGZsq{}}
\PYG{n}{priority} \PYG{o}{=} \PYG{l+s}{\PYGZsq{}presets\PYGZsq{}}
\PYG{n}{timeslice\PYGZus{}shares\PYGZus{}path} \PYG{o}{=} \PYG{l+s}{\PYGZsq{}\PYGZob{}path\PYGZcb{}/technodata/TimesliceShareCommercial.csv\PYGZsq{}}
\PYG{n}{macrodrivers\PYGZus{}path} \PYG{o}{=} \PYG{l+s}{\PYGZsq{}\PYGZob{}path\PYGZcb{}/technodata/Macrodrivers.csv\PYGZsq{}}
\PYG{n}{regression\PYGZus{}path} \PYG{o}{=} \PYG{l+s}{\PYGZsq{}\PYGZob{}path\PYGZcb{}/technodata/regressionparameters.csv\PYGZsq{}}
\PYG{n}{timeslices\PYGZus{}levels} \PYG{o}{=} \PYG{p}{\PYGZob{}}\PYG{l+s}{\PYGZsq{}day\PYGZsq{}}\PYG{p}{:} \PYG{p}{[}\PYG{l+s}{\PYGZsq{}all\PYGZhy{}day\PYGZsq{}}\PYG{p}{]}\PYG{p}{\PYGZcb{}}
\PYG{n}{forecast} \PYG{o}{=} \PYG{k}{[0, 5]}
\end{sphinxVerbatim}

The following attributes are accepted:
\begin{description}
\item[{type:}] \leavevmode
See the attribute in the standard mode, \DUrole{xref,std,std-ref}{type}. \sphinxstyleemphasis{Preset} sectors
are those with type “presets”.

\item[{priority}] \leavevmode
See the attribute in the standard mode, \DUrole{xref,std,std-ref}{priority}.

\item[{timeslices\_levels:}] \leavevmode
See the attribute in the standard mode, {\hyperref[\detokenize{inputs/toml:timeslices}]{\sphinxcrossref{Timeslices}}}.

\end{description}
\phantomsection\label{\detokenize{inputs/toml:preset-consumption}}\begin{description}
\item[{consumption\_path:}] \leavevmode
CSV output files, one per year. This attribute can include wild cards, i.e. ‘*’,
which can match anything. For instance: \sphinxtitleref{consumption\_path = “\{cwd\}/Consumtion*.csv”} will match any csv file starting with “Consumption” in the
current working directory. The file names must include the year for which it defines
the consumption, e.g. \sphinxtitleref{Consumption2015.csv}.

The CSV format should follow the following format:


\begin{savenotes}\sphinxattablestart
\centering
\sphinxcapstartof{table}
\sphinxthecaptionisattop
\sphinxcaption{Consumption}\label{\detokenize{inputs/toml:id2}}
\sphinxaftertopcaption
\begin{tabulary}{\linewidth}[t]{|T|T|T|T|T|T|T|}
\hline
\sphinxstyletheadfamily &\sphinxstyletheadfamily 
RegionName
&\sphinxstyletheadfamily 
ProcessName
&\sphinxstyletheadfamily 
TimeSlice
&\sphinxstyletheadfamily 
electricity
&\sphinxstyletheadfamily 
diesel
&\sphinxstyletheadfamily 
algae
\\
\hline\sphinxstyletheadfamily 
0
&\sphinxstyletheadfamily 
USA
&\sphinxstyletheadfamily 
fluorescent light
&\sphinxstyletheadfamily 
1
&
1.9
&
0
&
0
\\
\hline\sphinxstyletheadfamily 
1
&\sphinxstyletheadfamily 
USA
&\sphinxstyletheadfamily 
fluorescent light
&\sphinxstyletheadfamily 
2
&
1.8
&
0
&
0
\\
\hline
\end{tabulary}
\par
\sphinxattableend\end{savenotes}

The index column as well as “RegionName”, “ProcessName”, and “TimeSlice” must be
present. Further columns are reserved for commodities. “TimeSlice” refers to the
index of the timeslice.

\item[{supply\_path:}] \leavevmode
CSV file, one per year, indicating the amount of a commodities produced. It follows
the same format as \DUrole{xref,std,std-ref}{consumption\_path}.

\item[{supply\_path:}] \leavevmode
CSV file, one per year, indicating the amount of a commodities produced. It follows
the same format as \DUrole{xref,std,std-ref}{consumption\_path}.

\item[{prices\_path:}] \leavevmode
CSV file indicating the amount of a commodities produced. The format of the CSV files
follows that of {\hyperref[\detokenize{inputs/projections:inputs-projection}]{\sphinxcrossref{\DUrole{std,std-ref}{Initial Market Projection}}}}.

\end{description}
\phantomsection\label{\detokenize{inputs/toml:preset-demand}}\begin{description}
\item[{demand\_path:}] \leavevmode
Incompatible with \DUrole{xref,std,std-ref}{consumption\_path} or
\DUrole{xref,std,std-ref}{macrodrivers\_path}. A CSV file containing the consumption in the
same format as {\hyperref[\detokenize{inputs/projections:inputs-projection}]{\sphinxcrossref{\DUrole{std,std-ref}{Initial Market Projection}}}}.

\end{description}
\phantomsection\label{\detokenize{inputs/toml:preset-macro}}\begin{description}
\item[{macrodrivers\_path:}] \leavevmode
Incompatible with \DUrole{xref,std,std-ref}{consumption\_path} or
\DUrole{xref,std,std-ref}{demand\_path}. Path to a CSV file giving the profile of the
macrodrivers. Also requires \DUrole{xref,std,std-ref}{regression\_path}.

\end{description}
\phantomsection\label{\detokenize{inputs/toml:preset-regression}}\begin{description}
\item[{regression\_path:}] \leavevmode
Incompatible with \DUrole{xref,std,std-ref}{consumption\_path} or
\DUrole{xref,std,std-ref}{demand\_path}. Path to a CSV file giving the regression
parameters with respect to the macrodrivers.
Also requires \DUrole{xref,std,std-ref}{macrodrivers\_path}.

\item[{timeslice\_shares\_path}] \leavevmode
Optional csv file giving shares per timeslice. Requires
\DUrole{xref,std,std-ref}{macrodrivers\_path}.

\item[{filters:}] \leavevmode
Optional dictionary of entries by which to filter the consumption.  Requires
\DUrole{xref,std,std-ref}{macrodrivers\_path}. For instance,

\begin{sphinxVerbatim}[commandchars=\\\{\}]
\PYG{n}{filters}\PYG{o}{.}\PYG{n}{region} \PYG{o}{=} \PYG{p}{[}\PYG{l+s+s2}{\PYGZdq{}}\PYG{l+s+s2}{USA}\PYG{l+s+s2}{\PYGZdq{}}\PYG{p}{,} \PYG{l+s+s2}{\PYGZdq{}}\PYG{l+s+s2}{ASEA}\PYG{l+s+s2}{\PYGZdq{}}\PYG{p}{]}
\PYG{n}{filters}\PYG{o}{.}\PYG{n}{commodity} \PYG{o}{=} \PYG{p}{[}\PYG{l+s+s2}{\PYGZdq{}}\PYG{l+s+s2}{algae}\PYG{l+s+s2}{\PYGZdq{}}\PYG{p}{,} \PYG{l+s+s2}{\PYGZdq{}}\PYG{l+s+s2}{fluorescent light}\PYG{l+s+s2}{\PYGZdq{}}\PYG{p}{]}
\end{sphinxVerbatim}

\end{description}


\subsection{Legacy Sectors}
\label{\detokenize{inputs/toml:legacy-sectors}}
Legacy sectors wrap sectors developed for a previous version of MUSE to the open\sphinxhyphen{}source
version.

Preset sectors are defined in \sphinxcode{\sphinxupquote{PresetSector}}.

The can be defined in the TOML file as follows:

\begin{sphinxVerbatim}[commandchars=\\\{\}]
\PYG{k}{[global\PYGZus{}input\PYGZus{}files]}
\PYG{n}{macrodrivers} \PYG{o}{=} \PYG{l+s}{\PYGZsq{}\PYGZob{}path\PYGZcb{}/input/Macrodrivers.csv\PYGZsq{}}
\PYG{n}{regions} \PYG{o}{=} \PYG{l+s}{\PYGZsq{}\PYGZob{}path\PYGZcb{}/input/Regions.csv\PYGZsq{}}
\PYG{n}{global\PYGZus{}commodities} \PYG{o}{=} \PYG{l+s}{\PYGZsq{}\PYGZob{}path\PYGZcb{}/input/MUSEGlobalCommodities.csv\PYGZsq{}}

\PYG{k}{[sectors.Industry]}
\PYG{n}{type} \PYG{o}{=} \PYG{l+s}{\PYGZsq{}legacy\PYGZsq{}}
\PYG{n}{priority} \PYG{o}{=} \PYG{l+s}{\PYGZsq{}demand\PYGZsq{}}
\PYG{n}{agregation\PYGZus{}level} \PYG{o}{=} \PYG{l+s}{\PYGZsq{}month\PYGZsq{}}
\PYG{n}{excess} \PYG{o}{=} \PYG{l+m+mi}{0}

\PYG{n}{userdata\PYGZus{}path} \PYG{o}{=} \PYG{l+s}{\PYGZsq{}\PYGZob{}muse\PYGZus{}sectors\PYGZcb{}/Industry\PYGZsq{}}
\PYG{n}{technodata\PYGZus{}path} \PYG{o}{=} \PYG{l+s}{\PYGZsq{}\PYGZob{}muse\PYGZus{}sectors\PYGZcb{}/Industry\PYGZsq{}}
\PYG{n}{timeslices\PYGZus{}path} \PYG{o}{=} \PYG{l+s}{\PYGZsq{}\PYGZob{}muse\PYGZus{}sectors\PYGZcb{}/Industry/TimeslicesIndustry.csv\PYGZsq{}}
\PYG{n}{output\PYGZus{}path} \PYG{o}{=} \PYG{l+s}{\PYGZsq{}\PYGZob{}path\PYGZcb{}/output\PYGZsq{}}
\end{sphinxVerbatim}

For historical reasons, the three \sphinxtitleref{global\_input\_files} above are required. The sector
itself can use the following attributes.
\begin{description}
\item[{type:}] \leavevmode
See the attribute in the standard mode, \DUrole{xref,std,std-ref}{type}. \sphinxstyleemphasis{Legacy} sectors
are those with type “legacy”.

\item[{priority}] \leavevmode
See the attribute in the standard mode, \DUrole{xref,std,std-ref}{priority}.

\item[{agregation\_level:}] \leavevmode
Information relevant to the sector’s timeslice.

\item[{excess:}] \leavevmode
Excess factor used to model early obsolescence.

\item[{timeslices\_path:}] \leavevmode
Path to a timeslice  {\hyperref[\detokenize{inputs/timeslices:inputs-legacy-timeslices}]{\sphinxcrossref{\DUrole{std,std-ref}{time\_slices}}}}.

\item[{userdata\_path:}] \leavevmode
Path to a directory with sector\sphinxhyphen{}specific data files.

\item[{technodata\_path:}] \leavevmode
Path to a technodata CSV file. See. {\hyperref[\detokenize{inputs/technodata:inputs-technodata}]{\sphinxcrossref{\DUrole{std,std-ref}{Techno\sphinxhyphen{}data}}}}.

\item[{output\_path:}] \leavevmode
Path to a diretory where the sector will write output files.

\end{description}


\section{Input Files}
\label{\detokenize{inputs/inputs_csv:input-files}}\label{\detokenize{inputs/inputs_csv::doc}}

\subsection{Initial Market Projection}
\label{\detokenize{inputs/projections:initial-market-projection}}\label{\detokenize{inputs/projections:inputs-projection}}\label{\detokenize{inputs/projections::doc}}
MUSE needs an initial projection of the market prices for each period of the simulation.
\begin{itemize}
\item {} 
The price trajectory is needed if the MCA works in \sphinxstyleemphasis{equilibrium} mode as an initial
trajectory for the base year of the simulation. The market will override the
calculated prices obtained from each commodity equilibrium for all the future periods
following the base year

\item {} 
Similarly, if the market works in a \sphinxstyleemphasis{carbon budget} mode, the prices are used as a
starting point. The only difference from the previous case is given by the fact that
the MCA will be calculating an additional global market price for carbon dioxide (and
additional pollutants if required)

\item {} 
If the MCA works in an \sphinxstyleemphasis{exogenous} mode, it will use the initial market projection as
the projection for the the base year and all the future periods of the simulation

\end{itemize}

The forward price trajectory should follow the structure reported in the table below.


\begin{savenotes}\sphinxattablestart
\centering
\sphinxcapstartof{table}
\sphinxthecaptionisattop
\sphinxcaption{Initial market projections}\label{\detokenize{inputs/projections:id1}}
\sphinxaftertopcaption
\begin{tabular}[t]{|*{6}{\X{1}{6}|}}
\hline
\sphinxstyletheadfamily 
RegionName
&\sphinxstyletheadfamily 
Attribute
&\sphinxstyletheadfamily 
Time
&\sphinxstyletheadfamily 
com1
&\sphinxstyletheadfamily 
com2
&\sphinxstyletheadfamily 
com3
\\
\hline
Unit
&\begin{itemize}
\item {} 
\end{itemize}
&
Year
&
MUS\$2010/PJ
&
MUS\$2010/PJ
&
MUS\$2010/PJ
\\
\hline
region1
&
CommodityPrice
&
2010
&
20
&
1.9583
&
2
\\
\hline
region1
&
CommodityPrice
&
2015
&
20
&
1.9583
&
2
\\
\hline
region1
&
CommodityPrice
&
2020
&
20.38518042
&
1.996014941
&
2.038518042
\\
\hline
region1
&
CommodityPrice
&
2025
&
20.77777903
&
2.034456234
&
2.077777903
\\
\hline
region1
&
CommodityPrice
&
2030
&
21.17793872
&
2.073637869
&
2.117793872
\\
\hline
region1
&
CommodityPrice
&
2035
&
21.58580508
&
2.113574105
&
2.158580508
\\
\hline
region1
&
CommodityPrice
&
2040
&
22.00152655
&
2.154279472
&
2.200152655
\\
\hline
region1
&
CommodityPrice
&
2045
&
22.42525441
&
2.195768786
&
2.242525441
\\
\hline
region1
&
CommodityPrice
&
2050
&
22.85714286
&
2.238057143
&
2.285714286
\\
\hline
\end{tabular}
\par
\sphinxattableend\end{savenotes}
\begin{description}
\item[{RegionName}] \leavevmode
represents the region ID and needs to be consistent across all the data inputs

\item[{Attribute}] \leavevmode
defines the attribute type. In this case it refers to the CommodityPrice; it is
relevant only for internal use

\item[{Time}] \leavevmode
corresponds to the time periods of the simulation; the simulated time framework in
the example goes from 2010 through to 2050 with a 5\sphinxhyphen{}year time step

\item[{com1, …, comN}] \leavevmode
Any further columns represent the commodities modelled, as defined in the global
commodities the row Unit reports the unit in which the technology consumption is
defined; it is for the user internal reference only. The names \sphinxstyleemphasis{comX} should be
replaced with the names of the commodities.

\end{description}


\subsection{Regional data}
\label{\detokenize{inputs/regions:regional-data}}\label{\detokenize{inputs/regions:id1}}\label{\detokenize{inputs/regions::doc}}
MUSE requires the definition of the methodology used for investment and dispatch and alias
demand matching. The methodology has to be defined by region and subregion, meant as a
geographical subdivision in a region. Currently, the methodology definition is
important for the legacy sectors only.

Below the generic structure of the input commodity file for the electric
heater is shown:


\begin{savenotes}\sphinxattablestart
\centering
\sphinxcapstartof{table}
\sphinxthecaptionisattop
\sphinxcaption{Methodology used in investment and demand matching}\label{\detokenize{inputs/regions:id2}}
\sphinxaftertopcaption
\begin{tabulary}{\linewidth}[t]{|T|T|T|T|T|}
\hline
\sphinxstyletheadfamily 
SectorName
&\sphinxstyletheadfamily 
RegionName
&\sphinxstyletheadfamily 
Subregion
&\sphinxstyletheadfamily 
sMethodologyPlanning
&\sphinxstyletheadfamily 
sMethodologyDispatch
\\
\hline
Agriculture
&
region1
&
region1
&
NPV
&
DCF
\\
\hline
Bioenergy
&
region1
&
region1
&
NPV
&
DCF
\\
\hline
Industry
&
region1
&
region1
&
NPV
&
DCF
\\
\hline
Residential
&
region1
&
region1
&
EAC
&
EAC
\\
\hline
Commercial
&
region1
&
region1
&
EAC
&
EAC
\\
\hline
Transport
&
region1
&
region1
&
LCOE
&
LCOE
\\
\hline
Power
&
region1
&
region1
&
LCOE
&
LCOE
\\
\hline
Refinery
&
region1
&
region1
&
LCOE
&
LCOE
\\
\hline
Supply
&
region1
&
region1
&
LCOE
&
LCOE
\\
\hline
\end{tabulary}
\par
\sphinxattableend\end{savenotes}
\begin{description}
\item[{SectorName}] \leavevmode
represents the sector\_ID and needs to be consistent across the data input files

\item[{RegionName}] \leavevmode
represents the region ID and needs to be consistent across all the data inputs

\item[{Subregion}] \leavevmode
represents the subregion ID and needs to be consistent across all the data inputs

\item[{sMethodologyPlanning}] \leavevmode
reports the cost quantity used for making investments in new technologies in each
sector (e.g. NPV stands for net present value, EAC stands for equivalent annual
costs, LCOE stands for levelised cost of energy)

\item[{sMethodologyDispatch}] \leavevmode
reports the cost quantity used for the demand matching using existing technologies in
each sector (e.g. DCF stands for discounted cash flow, EAC stands for equivalent
annual cost, LCOE stands for levelised cost of energy)

\end{description}


\subsection{Commodity Description}
\label{\detokenize{inputs/commodities:commodity-description}}\label{\detokenize{inputs/commodities:inputs-commodities}}\label{\detokenize{inputs/commodities::doc}}
MUSE handles a configurable number and type of commodities which are primarily used to
represent energy, services, pollutants/emissions. The commodities for the simulation as
a whole are defined in a csv file with the following structure.


\begin{savenotes}\sphinxattablestart
\centering
\sphinxcapstartof{table}
\sphinxthecaptionisattop
\sphinxcaption{Global commodities}\label{\detokenize{inputs/commodities:id1}}
\sphinxaftertopcaption
\begin{tabulary}{\linewidth}[t]{|T|T|T|T|T|T|}
\hline
\sphinxstyletheadfamily 
Commodity
&\sphinxstyletheadfamily 
CommodityType
&\sphinxstyletheadfamily 
CommodityName
&\sphinxstyletheadfamily 
CommodityEmissionFactor\_CO2
&\sphinxstyletheadfamily 
HeatRate
&\sphinxstyletheadfamily 
Unit
\\
\hline
Coal
&
Energy
&
hardcoal
&
94.6
&
29
&
PJ
\\
\hline
Agricultural\sphinxhyphen{}residues
&
Energy
&
agrires
&
112
&
15.4
&
PJ
\\
\hline
\end{tabulary}
\par
\sphinxattableend\end{savenotes}
\begin{description}
\item[{Commodity}] \leavevmode
represents the extended name of a commodity

\item[{CommodityType}] \leavevmode
defines the type of a commodity (i.e. energy, material or environmental)

\item[{CommodityName}] \leavevmode
is the internal name used for a commodity inside the model.

\item[{CommodityEmissionFactor\_CO2}] \leavevmode
is CO2 emission per unit of commodity flow

\item[{HeatRate}] \leavevmode
represents the lower heating value of an energy commodity

\item[{Unit}] \leavevmode
is the unit used as a basis for all the input data. More specifically the model allows
a totally flexible way of defining the commodities. CommodityName is currently the
only column used internally as it defines the names of commodities and needs to be
used consistently across all the input data files. The remaining columns of the file
are only relevant for the user internal reference for the original sets of
assumptions used.

\end{description}


\subsection{Techno\sphinxhyphen{}data}
\label{\detokenize{inputs/technodata:techno-data}}\label{\detokenize{inputs/technodata:inputs-technodata}}\label{\detokenize{inputs/technodata::doc}}
The techno\sphinxhyphen{}data includes the techno\sphinxhyphen{}economic characteristics of each technology such
as capital, fixed and variable cost, lifetime, utilisation factor.
The techno\sphinxhyphen{}data should follow the structure reported in the table. The column order
is not important and additional input data can alsobe read in this format. In the table,
the electric boiler used in households is taken as an example for a generic region, region1.


\begin{savenotes}\sphinxattablestart
\centering
\sphinxcapstartof{table}
\sphinxthecaptionisattop
\sphinxcaption{Techno\sphinxhyphen{}data}\label{\detokenize{inputs/technodata:id1}}
\sphinxaftertopcaption
\begin{tabulary}{\linewidth}[t]{|T|T|T|T|T|T|T|T|}
\hline
\sphinxstyletheadfamily 
ProcessName
&\sphinxstyletheadfamily 
RegionName
&\sphinxstyletheadfamily 
Time
&\sphinxstyletheadfamily 
Level
&\sphinxstyletheadfamily 
cap\_par
&\sphinxstyletheadfamily 
cap\_exp
&\sphinxstyletheadfamily 
fix\_par
&\sphinxstyletheadfamily 
…
\\
\hline
resBoilerElectric
&
region1
&
2010
&
fixed
&
3.81
&
1.00
&
0.38
&
…
\\
\hline
resBoilerElectric
&
region1
&
2030
&
fixed
&
3.81
&
1.00
&
0.38
&
…
\\
\hline
\end{tabulary}
\par
\sphinxattableend\end{savenotes}
\begin{description}
\item[{ProcessName}] \leavevmode
represents the technology ID and needs to be consistent across all the data inputs

\item[{RegionName}] \leavevmode
represents the region ID and needs to be consistent across all the data inputs

\item[{Time}] \leavevmode
represents the period of the simulation to which the value applies; it needs to
contain at least the base year of the simulation

\item[{Level}] \leavevmode
characterises either a fixed or a flexible input type

\item[{cap\_par, cap\_exp}] \leavevmode
are used in the capital cost estimation. Capital costs are calculated as:
\begin{equation*}
\begin{split}\text{CAPEX} = \text{cap\_par} * \text{(Capacity)}^\text{cap\_exp}\end{split}
\end{equation*}
where the parameter cap\_par is estimated at a selected reference size (i.e. Capref),
such as:
\begin{equation*}
\begin{split}\text{cap\_par} = \left(
   \frac{\text{CAPEXref}}{\text{Capref}}
\right)^{\text{cap\_exp}}\end{split}
\end{equation*}
Capref is decided by the modeller before filling the input data files.

This allows the model to take into account economies of scale. ie. As \sphinxtitleref{Capacity} increases, the price of the technology decreases.

\end{description}

fix\_par, fix\_exp
\begin{quote}

are used in the fixed cost estimation. Fixed costs are calculated as:
\begin{equation*}
\begin{split}\text{FOM} = \text{fix\_par} * (\text{Capacity})^\text{fix\_exp}\end{split}
\end{equation*}
where the parameter fix\_par is estimated at a selected reference size (i.e. Capref),
such as:
\begin{equation*}
\begin{split}\text{fix\_par} = \left(
   \frac{\text{FOMref}}{\text{Capref}}
\right)^{\text{fix\_exp}}\end{split}
\end{equation*}
Capref is decided by the modeller before filling the input data files.
\end{quote}
\begin{description}
\item[{var\_par, var\_exp}] \leavevmode
are used in the variable costs estimation. These variable costs are capacity
dependent Variable costs are calculated as:
\begin{equation*}
\begin{split}\text{VAREX} = \text{cap\_par} * \text{(Capacity)}^{\text{cap\_exp}}\end{split}
\end{equation*}
where the parameter var\_par is estimated at a selected reference size (i.e. Capref),
such as:
\begin{equation*}
\begin{split}\text{var\_par} = \left(
   \frac{\text{VARref}}{\text{Capref}}
\right)^{\text{var\_exp}}\end{split}
\end{equation*}
Capref is decided by the modeller before filling the input data files.

\item[{MaxCapacityAddition}] \leavevmode
represents the maximum addition of installed capacity per technology, region, year.

\item[{MaxCapacityGrowth}] \leavevmode
represents the maximum growth in capacity as a fraction of the installed capacity per
technology, region and year.

\item[{TotalCapacityLimit}] \leavevmode
represents the total capacity limit per technology, region and year.

\item[{TechnicalLife}] \leavevmode
represents the number of years that a technology operates before it is decommissioned.

\item[{UtilizationFactor}] \leavevmode
is the number of operating hours of a process over the maximum number of hours in a year.

\item[{ScalingSize}] \leavevmode
represents the minimum size of a technology to be installed.

\item[{efficiency}] \leavevmode
is calculated as the ratio between the total output commodities and the input commodities.

\item[{AvailabiliyYear}] \leavevmode
defines the starting year of a technology; for example the value equals 1 when a
technology would be available or 0 when a technology would not be available.

\item[{Type}] \leavevmode
defines the type of a technology.

\item[{Fuel}] \leavevmode
defines the fuel used by a technology.

\item[{EndUse}] \leavevmode
defines the end use of a technology.

\item[{InterestRate}] \leavevmode
is the technology interest rate.

\item[{Agent\_0, …, Agent\_N}] \leavevmode
represent the allocation of the initial capacity to the each agent.

\end{description}

The input data has to be provided for the base year. Additional years within the time
framework of the overall simulation can be defined. In this case, MUSE would interpolate
the values between the provided periods and assume a constant value afterwards.


\subsection{Time\sphinxhyphen{}slices}
\label{\detokenize{inputs/timeslices:time-slices}}\label{\detokenize{inputs/timeslices:inputs-legacy-timeslices}}\label{\detokenize{inputs/timeslices::doc}}
\begin{sphinxadmonition}{note}{Note:}
This input file is only for legacy sectors. For anything else, please see \DUrole{xref,std,std-ref}{simulation\sphinxhyphen{}settings}.
\end{sphinxadmonition}

Time\sphinxhyphen{}slices represent a sub\sphinxhyphen{}year disaggregation of commodity demand. They are fully
flexible in number and names as to serve the specific representation of the commodity
demand, supply, and supply cost profile in each energy sector.  Each time slice is
independent in terms of the number of represent hours, as long as it is meaningful for the
users and their data inputs. 1 is the minimum number of time\sphinxhyphen{}slice as this would
correspond to a full year.  The time\sphinxhyphen{}slice definition of a sector affects the commodity
price profile and the supply cost profile.

The csv file for the time\sphinxhyphen{}slice definition would report the length (in hours) of each
time slice as characteristic to the selected sector to represent diurnal, weekly and
seasonal variation of energy commodities, demand and supply, as shown in the table for
30 time\sphinxhyphen{}slices.


\begin{savenotes}\sphinxatlongtablestart\begin{longtable}[c]{|l|l|l|l|l|l|}
\sphinxthelongtablecaptionisattop
\caption{Time\sphinxhyphen{}slices\strut}\label{\detokenize{inputs/timeslices:id1}}\\*[\sphinxlongtablecapskipadjust]
\hline
\sphinxstyletheadfamily 
AgLevel
&\sphinxstyletheadfamily 
SN
&\sphinxstyletheadfamily 
Month
&\sphinxstyletheadfamily 
Day
&\sphinxstyletheadfamily 
Hour
&\sphinxstyletheadfamily 
RepresentHours
\\
\hline
\endfirsthead

\multicolumn{6}{c}%
{\makebox[0pt]{\sphinxtablecontinued{\tablename\ \thetable{} \textendash{} continued from previous page}}}\\
\hline
\sphinxstyletheadfamily 
AgLevel
&\sphinxstyletheadfamily 
SN
&\sphinxstyletheadfamily 
Month
&\sphinxstyletheadfamily 
Day
&\sphinxstyletheadfamily 
Hour
&\sphinxstyletheadfamily 
RepresentHours
\\
\hline
\endhead

\hline
\multicolumn{6}{r}{\makebox[0pt][r]{\sphinxtablecontinued{continues on next page}}}\\
\endfoot

\endlastfoot
\sphinxstyletheadfamily 
Hour
&\sphinxstyletheadfamily 
1
&\sphinxstyletheadfamily 
Winter
&\sphinxstyletheadfamily 
Weekday
&
Night
&
396
\\
\hline\sphinxstyletheadfamily 
Hour
&\sphinxstyletheadfamily 
2
&\sphinxstyletheadfamily 
Winter
&\sphinxstyletheadfamily 
Weekday
&
Morning
&
396
\\
\hline\sphinxstyletheadfamily 
Hour
&\sphinxstyletheadfamily 
3
&\sphinxstyletheadfamily 
Winter
&\sphinxstyletheadfamily 
Weekday
&
Afternoon
&
264
\\
\hline\sphinxstyletheadfamily 
Hour
&\sphinxstyletheadfamily 
4
&\sphinxstyletheadfamily 
Winter
&\sphinxstyletheadfamily 
Weekday
&
EarlyPeak
&
66
\\
\hline\sphinxstyletheadfamily 
Hour
&\sphinxstyletheadfamily 
5
&\sphinxstyletheadfamily 
Winter
&\sphinxstyletheadfamily 
Weekday
&
LatePeak
&
66
\\
\hline\sphinxstyletheadfamily 
Hour
&\sphinxstyletheadfamily 
6
&\sphinxstyletheadfamily 
Winter
&\sphinxstyletheadfamily 
Weekday
&
Evening
&
396
\\
\hline\sphinxstyletheadfamily 
Hour
&\sphinxstyletheadfamily 
7
&\sphinxstyletheadfamily 
Winter
&\sphinxstyletheadfamily 
Weekend
&
Night
&
156
\\
\hline\sphinxstyletheadfamily 
Hour
&\sphinxstyletheadfamily 
8
&\sphinxstyletheadfamily 
Winter
&\sphinxstyletheadfamily 
Weekend
&
Morning
&
156
\\
\hline\sphinxstyletheadfamily 
Hour
&\sphinxstyletheadfamily 
9
&\sphinxstyletheadfamily 
Winter
&\sphinxstyletheadfamily 
Weekend
&
Afternoon
&
156
\\
\hline\sphinxstyletheadfamily 
Hour
&\sphinxstyletheadfamily 
10
&\sphinxstyletheadfamily 
Winter
&\sphinxstyletheadfamily 
Weekend
&
Evening
&
156
\\
\hline\sphinxstyletheadfamily 
Hour
&\sphinxstyletheadfamily 
11
&\sphinxstyletheadfamily 
SpringAutumn
&\sphinxstyletheadfamily 
Weekday
&
Night
&
792
\\
\hline\sphinxstyletheadfamily 
Hour
&\sphinxstyletheadfamily 
12
&\sphinxstyletheadfamily 
SpringAutumn
&\sphinxstyletheadfamily 
Weekday
&
Morning
&
792
\\
\hline\sphinxstyletheadfamily 
Hour
&\sphinxstyletheadfamily 
13
&\sphinxstyletheadfamily 
SpringAutumn
&\sphinxstyletheadfamily 
Weekday
&
Afternoon
&
528
\\
\hline\sphinxstyletheadfamily 
Hour
&\sphinxstyletheadfamily 
14
&\sphinxstyletheadfamily 
SpringAutumn
&\sphinxstyletheadfamily 
Weekday
&
EarlyPeak
&
132
\\
\hline\sphinxstyletheadfamily 
Hour
&\sphinxstyletheadfamily 
15
&\sphinxstyletheadfamily 
SpringAutumn
&\sphinxstyletheadfamily 
Weekday
&
LatePeak
&
132
\\
\hline\sphinxstyletheadfamily 
Hour
&\sphinxstyletheadfamily 
16
&\sphinxstyletheadfamily 
SpringAutumn
&\sphinxstyletheadfamily 
Weekday
&
Evening
&
792
\\
\hline\sphinxstyletheadfamily 
Hour
&\sphinxstyletheadfamily 
17
&\sphinxstyletheadfamily 
SpringAutumn
&\sphinxstyletheadfamily 
Weekend
&
Night
&
300
\\
\hline\sphinxstyletheadfamily 
Hour
&\sphinxstyletheadfamily 
18
&\sphinxstyletheadfamily 
SpringAutumn
&\sphinxstyletheadfamily 
Weekend
&
Morning
&
300
\\
\hline\sphinxstyletheadfamily 
Hour
&\sphinxstyletheadfamily 
19
&\sphinxstyletheadfamily 
SpringAutumn
&\sphinxstyletheadfamily 
Weekend
&
Afternoon
&
300
\\
\hline\sphinxstyletheadfamily 
Hour
&\sphinxstyletheadfamily 
20
&\sphinxstyletheadfamily 
SpringAutumn
&\sphinxstyletheadfamily 
Weekend
&
Evening
&
300
\\
\hline\sphinxstyletheadfamily 
Hour
&\sphinxstyletheadfamily 
21
&\sphinxstyletheadfamily 
Summer
&\sphinxstyletheadfamily 
Weekday
&
Night
&
396
\\
\hline\sphinxstyletheadfamily 
Hour
&\sphinxstyletheadfamily 
22
&\sphinxstyletheadfamily 
Summer
&\sphinxstyletheadfamily 
Weekday
&
Morning
&
396
\\
\hline\sphinxstyletheadfamily 
Hour
&\sphinxstyletheadfamily 
23
&\sphinxstyletheadfamily 
Summer
&\sphinxstyletheadfamily 
Weekday
&
Afternoon
&
264
\\
\hline\sphinxstyletheadfamily 
Hour
&\sphinxstyletheadfamily 
24
&\sphinxstyletheadfamily 
Summer
&\sphinxstyletheadfamily 
Weekday
&
EarlyPeak
&
66
\\
\hline\sphinxstyletheadfamily 
Hour
&\sphinxstyletheadfamily 
25
&\sphinxstyletheadfamily 
Summer
&\sphinxstyletheadfamily 
Weekday
&
LatePeak
&
66
\\
\hline\sphinxstyletheadfamily 
Hour
&\sphinxstyletheadfamily 
26
&\sphinxstyletheadfamily 
Summer
&\sphinxstyletheadfamily 
Weekday
&
Evening
&
396
\\
\hline\sphinxstyletheadfamily 
Hour
&\sphinxstyletheadfamily 
27
&\sphinxstyletheadfamily 
Summer
&\sphinxstyletheadfamily 
Weekend
&
Night
&
150
\\
\hline\sphinxstyletheadfamily 
Hour
&\sphinxstyletheadfamily 
28
&\sphinxstyletheadfamily 
Summer
&\sphinxstyletheadfamily 
Weekend
&
Morning
&
150
\\
\hline\sphinxstyletheadfamily 
Hour
&\sphinxstyletheadfamily 
29
&\sphinxstyletheadfamily 
Summer
&\sphinxstyletheadfamily 
Weekend
&
Afternoon
&
150
\\
\hline\sphinxstyletheadfamily 
Hour
&\sphinxstyletheadfamily 
30
&\sphinxstyletheadfamily 
Summer
&\sphinxstyletheadfamily 
Weekend
&
Evening
&
150
\\
\hline
\end{longtable}\sphinxatlongtableend\end{savenotes}

It reports the aggregation level of the sector time\sphinxhyphen{}slices (AgLevel), slice number (SN),
seasonal time slices (Month), weekly time slices (Day), hourly profile (Hour), the
amount of hours associated to each time slice (RepresentHours).


\subsection{Input Commodities}
\label{\detokenize{inputs/commodities_io:input-commodities}}\label{\detokenize{inputs/commodities_io:inputs-icomms}}\label{\detokenize{inputs/commodities_io::doc}}
Input commodities are the commodities consumed (also called consumables in MUSE) by each
technology.  They are defined in a csv file which describes the commodity inputs to each
technology, calculated per unit of technology activity. See {\hyperref[\detokenize{inputs/commodities_io:inputs-iocomms}]{\sphinxcrossref{\DUrole{std,std-ref}{below}}}} for a description.


\subsection{Output Commodities}
\label{\detokenize{inputs/commodities_io:output-commodities}}\label{\detokenize{inputs/commodities_io:inputs-ocomms}}
Output commodities are the commodities produced (also called products in MUSE) by each
technology.  They are defined in a csv file which describes the commodity outputs from
each technology, defined per unit of technology activity. Emissions, such as CO2
(produced from fuel combustion and reactions), CH4, N2O, F\sphinxhyphen{}gases, can also be accounted
for in this file. See {\hyperref[\detokenize{inputs/commodities_io:inputs-iocomms}]{\sphinxcrossref{\DUrole{std,std-ref}{below}}}} for a description.


\subsection{General features}
\label{\detokenize{inputs/commodities_io:general-features}}\label{\detokenize{inputs/commodities_io:inputs-iocomms}}
To illustrate the data required for a generic technology in MUSE, the \sphinxstyleemphasis{electric boiler
technology} is used as an example. The commodity flow for the electric boiler, capable
to cover space heating and water heating energy service demands.

\begin{figure}[htbp]
\centering
\capstart

\noindent\sphinxincludegraphics[width=400\sphinxpxdimen]{{commodities_io}.png}
\caption{The table below shows the basic data requirements for a typical technology, the
electric boiler.}\label{\detokenize{inputs/commodities_io:id1}}\end{figure}

\noindent\sphinxincludegraphics[width=400\sphinxpxdimen]{{commodities_io_table}.png}

Below it is shown the generic structure of the input commodity file for the electric
heater.


\begin{savenotes}\sphinxattablestart
\centering
\sphinxcapstartof{table}
\sphinxthecaptionisattop
\sphinxcaption{Commodities used as consumables \sphinxhyphen{} Input commodities}\label{\detokenize{inputs/commodities_io:id2}}
\sphinxaftertopcaption
\begin{tabular}[t]{|*{5}{\X{1}{5}|}}
\hline
\sphinxstyletheadfamily 
ProcessName
&\sphinxstyletheadfamily 
RegionName
&\sphinxstyletheadfamily 
Time
&\sphinxstyletheadfamily 
Level
&\sphinxstyletheadfamily 
electricity
\\
\hline
Unit
&\begin{itemize}
\item {} 
\end{itemize}
&
Year
&\begin{itemize}
\item {} 
\end{itemize}
&
GWh/PJ
\\
\hline
resBoilerElectric
&
region1
&
2010
&
fixed
&
300
\\
\hline
resBoilerElectric
&
region1
&
2030
&
fixed
&
290
\\
\hline
\end{tabular}
\par
\sphinxattableend\end{savenotes}
\begin{description}
\item[{ProcessName}] \leavevmode
represents the technology ID and needs to be consistent across all the data inputs.

\item[{RegionName}] \leavevmode
represents the region ID and needs to be consistent across all the data inputs.

\item[{Time}] \leavevmode
represents the period of the simulation to which the value applies; it needs to
contain at least the base year of the simulation.

\item[{Level}] \leavevmode
characterises either a fixed or a flexible input type the following columns should
contain the list of commodities the row.

\item[{Unit}] \leavevmode
reports the unit in which the technology consumption is defined; it is for the user
internal reference only.

\end{description}

The same structure for the csv file would also apply for the output commodity file. The
input data has to be provided for the base year. Additional years within the time
framework of the overall simulation can be defined. In this case, MUSE would interpolate
the values between the provided periods and assume a constant value afterwards.


\subsection{Existing Sectoral Capacity}
\label{\detokenize{inputs/existing_capacity:existing-sectoral-capacity}}\label{\detokenize{inputs/existing_capacity:inputs-existing-capacity}}\label{\detokenize{inputs/existing_capacity::doc}}
For each technology, the decommissioning profile should be given to MUSE.

The csv file which provides the installed capacity in base year and the decommissioning
profile in the future periods for each technology in a sector, in each region, should
follow the structure reported in the table.


\begin{savenotes}\sphinxattablestart
\centering
\sphinxcapstartof{table}
\sphinxthecaptionisattop
\sphinxcaption{Existing capacity of technologies: the residential boiler example}\label{\detokenize{inputs/existing_capacity:id1}}
\sphinxaftertopcaption
\begin{tabulary}{\linewidth}[t]{|T|T|T|T|T|T|T|T|}
\hline
\sphinxstyletheadfamily 
ProcessName
&\sphinxstyletheadfamily 
RegionName
&\sphinxstyletheadfamily 
Unit
&\sphinxstyletheadfamily 
2010
&\sphinxstyletheadfamily 
2020
&\sphinxstyletheadfamily 
2030
&\sphinxstyletheadfamily 
2040
&\sphinxstyletheadfamily 
2050
\\
\hline
resBoilerElectric
&
region1
&
PJ/y
&
5
&
0.5
&
0
&
0
&
0
\\
\hline
resBoilerElectric
&
region2
&
PJ/y
&
39
&
3.5
&
1
&
0.3
&
0
\\
\hline
\end{tabulary}
\par
\sphinxattableend\end{savenotes}
\begin{description}
\item[{ProcessName}] \leavevmode
represents the technology ID and needs to be consistent across all the data inputs.

\item[{RegionName}] \leavevmode
represents the region ID and needs to be consistent across all the data inputs.

\item[{Unit}] \leavevmode
reports the unit of the technology capacity; it is for the user internal reference only.

\item[{2010,…, 2050}] \leavevmode
represent the simulated periods.

\end{description}


\subsection{Agents}
\label{\detokenize{inputs/agents:agents}}\label{\detokenize{inputs/agents:inputs-agents}}\label{\detokenize{inputs/agents::doc}}
In MUSE, an agent\sphinxhyphen{}based formulation was originally introduced for the residential and
commercial building sectors \DUrole{bibtex}{{[}2019:sachs{]}}.  Agents are defined using a CSV file, with
one agent per row, using a somewhat historical format meant specifically for retrofit
and new\sphinxhyphen{}capacity agent pairs. This CSV file can be read using
\sphinxcode{\sphinxupquote{read\_csv\_agent\_parameters()}}. The data is also
interpreted to some degree in the factory functions
\sphinxcode{\sphinxupquote{create\_retrofit\_agent()}} and
\sphinxcode{\sphinxupquote{create\_newcapa\_agent()}}.

For instance, we have the following CSV table:


\begin{savenotes}\sphinxattablestart
\centering
\begin{tabulary}{\linewidth}[t]{|T|T|T|T|T|T|T|T|}
\hline
\sphinxstyletheadfamily 
Name
&\sphinxstyletheadfamily 
Type
&\sphinxstyletheadfamily 
AgentShare
&\sphinxstyletheadfamily 
RegionName
&\sphinxstyletheadfamily 
Objective1
&\sphinxstyletheadfamily 
SearchRule
&\sphinxstyletheadfamily 
DecisionMethod
&\sphinxstyletheadfamily 
…
\\
\hline
A1
&
New
&
Agent5
&
ASEAN
&
EAC
&
all
&
epsilonCon
&
…
\\
\hline
A4
&
New
&
Agent6
&
ASEAN
&
CapitalCosts
&
existing
&
weightedSum
&
…
\\
\hline
A1
&
Retrofit
&
Agent1
&
ASEAN
&
efficiency
&
all
&
epsilonCon
&
…
\\
\hline
A2
&
Retrofit
&
Agent2
&
ASEAN
&
Emissions
&
similar
&
weightedSum
&
…
\\
\hline
\end{tabulary}
\par
\sphinxattableend\end{savenotes}

For simplicity, not all columns are included in the example above. Though all column
listed below are currently required.

The columns have the following meaning:

\phantomsection\label{\detokenize{inputs/agents:name}}\begin{description}
\item[{Name}] \leavevmode
Name shared by a retrofit and new\sphinxhyphen{}capacity agent pair.

\item[{Type}] \leavevmode
One of “New” or “Retrofit”. “New” and “Retrofit” agents make up a pair with a given
{\hyperref[\detokenize{inputs/agents:name}]{\sphinxcrossref{\DUrole{std,std-ref}{name}}}}. The demand is split into two, with one part coming from
decommissioned assets, and the other coming from everything else. “Retrofit” agents
invest only to make up for decommissioned assets. They are often limited in the
technologies they can consider (by {\hyperref[\detokenize{inputs/agents:searchrule}]{\sphinxcrossref{\DUrole{std,std-ref}{SearchRule}}}}). “New” agents
invest on the rest of the demand, and can often consider more general sets of
technologies.

\item[{AgentShare}] \leavevmode
Name of the share of the existing capacity assigned to this agent. Only meaningful
for retrofit agents. The actual share itself can be found in
{\hyperref[\detokenize{inputs/technodata:inputs-technodata}]{\sphinxcrossref{\DUrole{std,std-ref}{Techno\sphinxhyphen{}data}}}}.

\item[{RegionName}] \leavevmode
Region where an agent operates.

\end{description}
\phantomsection\label{\detokenize{inputs/agents:objective1}}\begin{description}
\item[{Objective1}] \leavevmode
First objective that an agent will try and maximize or minimize during investment.
This objective should be one registered with
\sphinxcode{\sphinxupquote{@register\_objective}}. The following objectives are
available with MUSE:
\begin{itemize}
\item {} 
\sphinxcode{\sphinxupquote{comfort}}: Comfort provided by a given technology. Comfort does
not change during the simulation. It is obtained straightforwardly from
{\hyperref[\detokenize{inputs/technodata:inputs-technodata}]{\sphinxcrossref{\DUrole{std,std-ref}{Techno\sphinxhyphen{}data}}}}.

\item {} 
\sphinxcode{\sphinxupquote{efficiency}}: Efficiency of the technologies. Efficiency does
not change during the simulation. It is obtained straightforwardly from
{\hyperref[\detokenize{inputs/technodata:inputs-technodata}]{\sphinxcrossref{\DUrole{std,std-ref}{Techno\sphinxhyphen{}data}}}}.

\item {} 
\sphinxcode{\sphinxupquote{fixed\_costs}}: The fixed maintenance costs incurred by a
technology. The costs are a function of the capacity required to fulfil the current
demand.

\item {} 
\sphinxcode{\sphinxupquote{capital\_costs}}: The capital cost incurred by a
technology. The capital cost does not change during the simulation. It is obtained
as a function of parameters found in {\hyperref[\detokenize{inputs/technodata:inputs-technodata}]{\sphinxcrossref{\DUrole{std,std-ref}{Techno\sphinxhyphen{}data}}}}.

\item {} 
\sphinxcode{\sphinxupquote{emission\_cost}}: The costs associated for emissions for a
technology. The costs is a function both of the amount produced (equated to the
total demand in this case) and of the prices associated with each pollutant.
Aliased to “emission” for simplicity.

\item {} 
\sphinxcode{\sphinxupquote{fuel\_consumption\_cost}}: Costs of the fuels for
each technology, where each technology is used to fulfil the whole demand.

\item {} 
\sphinxcode{\sphinxupquote{lifetime\_levelized\_cost\_of\_energy}}:
LCOE over the lifetime of a technology. Aliased to “LCOE” for simplicity.

\item {} 
\sphinxcode{\sphinxupquote{net\_present\_value}}: Present value of all the costs of
installing and operating a technology, minus its revenues, of the course of its
lifetime. Aliased to “NPV” for simplicity.

\item {} 
\sphinxcode{\sphinxupquote{equivalent\_annual\_cost}}: Annualized form of the
net present value. Aliased to “EAC” for simplicity.

\end{itemize}

The weight associated with this objective can be changed using {\hyperref[\detokenize{inputs/agents:objdata1}]{\sphinxcrossref{\DUrole{std,std-ref}{ObjData1}}}}.  Whether the objective should be minimized or maximized depends on
{\hyperref[\detokenize{inputs/agents:objsort1}]{\sphinxcrossref{\DUrole{std,std-ref}{Objsort1}}}}. Multiple objectives are combined using the
{\hyperref[\detokenize{inputs/agents:decisionmethod}]{\sphinxcrossref{\DUrole{std,std-ref}{DecisionMethod}}}}

\end{description}
\phantomsection\label{\detokenize{inputs/agents:objective2}}\begin{description}
\item[{Objective2}] \leavevmode
Second objective. See {\hyperref[\detokenize{inputs/agents:objective1}]{\sphinxcrossref{\DUrole{std,std-ref}{Objective1}}}}.

\end{description}
\phantomsection\label{\detokenize{inputs/agents:objective3}}\begin{description}
\item[{Objective3:}] \leavevmode
Third objective. See {\hyperref[\detokenize{inputs/agents:objective1}]{\sphinxcrossref{\DUrole{std,std-ref}{Objective1}}}}.

\end{description}
\phantomsection\label{\detokenize{inputs/agents:objdata1}}\begin{description}
\item[{ObjData1}] \leavevmode
A weight associated with the {\hyperref[\detokenize{inputs/agents:objective1}]{\sphinxcrossref{\DUrole{std,std-ref}{first objective}}}}. Whether it is used
will depend in large part on the {\hyperref[\detokenize{inputs/agents:decisionmethod}]{\sphinxcrossref{\DUrole{std,std-ref}{decision method}}}}.

\item[{ObjData2}] \leavevmode
A weight associated with the {\hyperref[\detokenize{inputs/agents:objective2}]{\sphinxcrossref{\DUrole{std,std-ref}{second objective}}}}. See {\hyperref[\detokenize{inputs/agents:objdata1}]{\sphinxcrossref{\DUrole{std,std-ref}{ObjData1}}}}.

\item[{ObjData3}] \leavevmode
A weight associated with the {\hyperref[\detokenize{inputs/agents:objective3}]{\sphinxcrossref{\DUrole{std,std-ref}{third objective}}}}. See {\hyperref[\detokenize{inputs/agents:objdata1}]{\sphinxcrossref{\DUrole{std,std-ref}{ObjData1}}}}.

\end{description}
\phantomsection\label{\detokenize{inputs/agents:objsort1}}\begin{description}
\item[{Objsort1}] \leavevmode
Whether to maximize (\sphinxtitleref{True}) or minimize (\sphinxtitleref{False}) the {\hyperref[\detokenize{inputs/agents:objective1}]{\sphinxcrossref{\DUrole{std,std-ref}{first objective}}}}.

\item[{Objsort2}] \leavevmode
Whether to maximize (\sphinxtitleref{True}) or minimize (\sphinxtitleref{False}) the {\hyperref[\detokenize{inputs/agents:objective2}]{\sphinxcrossref{\DUrole{std,std-ref}{second objective}}}}.

\item[{Objsort3}] \leavevmode
Whether to maximize (\sphinxtitleref{True}) or minimize (\sphinxtitleref{False}) the {\hyperref[\detokenize{inputs/agents:objective3}]{\sphinxcrossref{\DUrole{std,std-ref}{third objective}}}}.

\end{description}
\phantomsection\label{\detokenize{inputs/agents:searchrule}}\begin{description}
\item[{SearchRule}] \leavevmode
The search rule allows users to par down the search space of technologies to those an
agent is likely to consider.
The search rule is any function with a given signature, and registered with MUSE via
\sphinxcode{\sphinxupquote{@register\_filter}}. The following search rules, defined
in \sphinxcode{\sphinxupquote{filters}}, are available with MUSE:
\begin{itemize}
\item {} 
\sphinxcode{\sphinxupquote{same\_enduse}}: Only allow technologies that provide the same
enduse as the current set of technologies owned by the agent.

\item {} 
\sphinxcode{\sphinxupquote{identity}}: Allows all current technologies. E.g. disables
filtering. Aliased to “all”.

\item {} 
\sphinxcode{\sphinxupquote{similar\_technology}}: Only allows technologies that
have the same type as current crop of technologies in the agent, as determined by
“tech\_type” in {\hyperref[\detokenize{inputs/technodata:inputs-technodata}]{\sphinxcrossref{\DUrole{std,std-ref}{Techno\sphinxhyphen{}data}}}}. Aliased to “similar”.

\item {} 
\sphinxcode{\sphinxupquote{same\_fuels}}: Only allows technologies that consume the same
fuels as the current crop of technologies in the agent. Aliased to
“fueltype”.

\item {} 
\sphinxcode{\sphinxupquote{currently\_existing\_tech}}: Only allows
technologies that the agent already owns. Aliased to “existing”.

\item {} 
\sphinxcode{\sphinxupquote{currently\_referenced\_tech}}: Only allows
technologies that are currently present in the market with non\sphinxhyphen{}zero capacity.

\item {} 
\sphinxcode{\sphinxupquote{maturity}}: Only allows technologies that have achieved a given
market share.

\end{itemize}

The implementation allows for combining these filters. However, the CSV data format
described here does not.

\end{description}
\phantomsection\label{\detokenize{inputs/agents:decisionmethod}}\begin{description}
\item[{DecisionMethod}] \leavevmode
Decision methods reduce multiple objectives into a single scalar objective per
replacement technology. They allow combining several objectives into a single metric
through which replacement technologies can be ranked.

Decision methods are any function which follow a given signature and are registered
via the decorator \sphinxcode{\sphinxupquote{@register\_decision}}. The following
decision methods are available with MUSE, as implemented in
\sphinxcode{\sphinxupquote{decisions}}:
\begin{itemize}
\item {} 
\sphinxcode{\sphinxupquote{mean}}: Computes the average across several objectives.

\item {} 
\sphinxcode{\sphinxupquote{weighted\_sum}}: Computes a weighted average across several
objectives.

\item {} 
\sphinxcode{\sphinxupquote{lexical\_comparion}}: Compares objectives using a
binned lexical comparison operator. Aliased to “lexo”.

\item {} 
\sphinxcode{\sphinxupquote{retro\_lexical\_comparion}}: A binned lexical
comparison function where the bin size is adjusted to ensure the current crop of
technologies are competitive. Aliased to “retro\_lexo”.

\item {} 
\sphinxcode{\sphinxupquote{epsilon\_constraints}}: A comparison method which
ensures that first selects technologies following constraints on objectives 2 and
higher, before actually ranking them using objective 1. Aliased to “epsilon” ad
“epsilon\_con”.

\item {} 
\sphinxcode{\sphinxupquote{retro\_epsilon\_constraints}}: A variation on
epsilon constraints which ensures that the current crop of technologies are not
deselected by the constraints. Aliased to “retro\_epsilon”.

\item {} 
\sphinxcode{\sphinxupquote{single\_objective}}: A decision method to allow
ranking via a single objective.

\end{itemize}

The functions allow for any number of objectives. However, the format described here
allows only for three.

\item[{Quantity}] \leavevmode
A factor used to determine the demand share of “New” agents.

\item[{MaturityThreshold}] \leavevmode
Parameter for the search rule \sphinxcode{\sphinxupquote{maturity}}.

\end{description}


\subsection{Indices and tables}
\label{\detokenize{inputs/inputs_csv:indices-and-tables}}\begin{itemize}
\item {} 
\DUrole{xref,std,std-ref}{genindex}

\item {} 
\DUrole{xref,std,std-ref}{modindex}

\item {} 
\DUrole{xref,std,std-ref}{search}

\end{itemize}


\section{Indices and tables}
\label{\detokenize{inputs/index:indices-and-tables}}\begin{itemize}
\item {} 
\DUrole{xref,std,std-ref}{genindex}

\item {} 
\DUrole{xref,std,std-ref}{modindex}

\item {} 
\DUrole{xref,std,std-ref}{search}

\end{itemize}


\chapter{Advanced guide}
\label{\detokenize{advanced-guide/index:advanced-guide}}\label{\detokenize{advanced-guide/index::doc}}

\section{Extending MUSE}
\label{\detokenize{advanced-guide/extending-muse:Extending-MUSE}}\label{\detokenize{advanced-guide/extending-muse::doc}}
One key feature of the generalized sector’s implementation is that it should be easy to extend. As such, MUSE can be made to run custom python functions, as long as these inputs and output of the function follow a standard specific to each step. We will look at a few here.

Below is a list of possible hooks, referenced by their implementation in the MUSE model:
\begin{itemize}
\item {} 
\sphinxcode{\sphinxupquote{register\_interaction\_net}} in \sphinxcode{\sphinxupquote{muse.interactions}}: a list of lists of agents that interact together.

\item {} 
\sphinxcode{\sphinxupquote{register\_agent\_interaction}} in \sphinxcode{\sphinxupquote{muse.interactions}}: Given a list of interacting agents, perform the interaction.

\item {} 
\sphinxcode{\sphinxupquote{register\_production}} in \sphinxcode{\sphinxupquote{muse.production}}: A method to compute the production from a sector, given the demand and the capacity.

\item {} 
\sphinxcode{\sphinxupquote{register\_initial\_asset\_transform}} in \sphinxcode{\sphinxupquote{muse.hooks}}: Allows any kind of transformation to be applied to the assets of an agent, prior to investing.

\item {} 
\sphinxcode{\sphinxupquote{register\_final\_asset\_transform}} in \sphinxcode{\sphinxupquote{muse.hooks}}: After computing the investment, this sets the assets that will be owned by the agents.

\item {} 
\sphinxcode{\sphinxupquote{register\_demand\_share}} in \sphinxcode{\sphinxupquote{muse.demand\_share}}: During agent investment, this is the share of the demand that an agent will try and satisfy.

\item {} 
\sphinxcode{\sphinxupquote{register\_filter}} in \sphinxcode{\sphinxupquote{muse.filters}}: A filter to remove technologies from consideration, during agent investment.

\item {} 
\sphinxcode{\sphinxupquote{register\_objective}} in \sphinxcode{\sphinxupquote{muse.objectives}}: A quantity which allows an agent to compare technologies during investment.

\item {} 
\sphinxcode{\sphinxupquote{register\_decision}} in \sphinxcode{\sphinxupquote{muse.decisions}}: A transformation applied to aggregate multiple objectives into a single objective during agent investment, e.g. via a weighted sum.

\item {} 
\sphinxcode{\sphinxupquote{register\_investment}} in \sphinxcode{\sphinxupquote{muse.investment}}: During agent investment, matches the demand for future investment using the decision metric above.

\item {} 
\sphinxcode{\sphinxupquote{register\_output\_quantity}} in \sphinxcode{\sphinxupquote{muse.output.sector}}: A sectorial quantity to output for postmortem analysis.

\item {} 
\sphinxcode{\sphinxupquote{register\_output\_sink}} in \sphinxcode{\sphinxupquote{muse.outputs}}: A \sphinxstyleemphasis{place} to store an output quantity, e.g. a file with a given format, a database on premise or on the cloud, etc…

\item {} 
\sphinxcode{\sphinxupquote{register\_carbon\_budget\_fitter}} in \sphinxcode{\sphinxupquote{muse.carbon\_budget}}

\item {} 
\sphinxcode{\sphinxupquote{register\_carbon\_budget\_method}} in \sphinxcode{\sphinxupquote{muse.carbon\_budget}}

\item {} 
\sphinxcode{\sphinxupquote{register\_sector}}: Registers a function that can create a sector from a muse configuration object.

\end{itemize}


\subsection{Extending outputs}
\label{\detokenize{advanced-guide/extending-muse:Extending-outputs}}
MUSE can be used to save custom quantities as well as data for analysis. There are two steps to this process:
\begin{itemize}
\item {} 
Computing the quantity of interest

\item {} 
Store the quantity of interest in a sink

\end{itemize}

In practice, this means that we can compute any quantity, such as capacity or consumption of an energy source and save it to a csv file, or a netcdf file.


\subsubsection{Output extension}
\label{\detokenize{advanced-guide/extending-muse:Output-extension}}
To demonstrate this, we will compute a new edited quantity of consumption, then save it as a text file.

The current implementation of the quantity of consumption found in \sphinxcode{\sphinxupquote{muse.outputs.sector}} filters out values of 0. In this example, we would like to maintain the values of 0, but do not want to edit the source code of MUSE.

This is rather simple to do using MUSE’s hooks.

First we create a new function called \sphinxcode{\sphinxupquote{consumption\_zero}} as follows:

{
\sphinxsetup{VerbatimColor={named}{nbsphinx-code-bg}}
\sphinxsetup{VerbatimBorderColor={named}{nbsphinx-code-border}}
\begin{sphinxVerbatim}[commandchars=\\\{\}]
\llap{\color{nbsphinxin}[1]:\,\hspace{\fboxrule}\hspace{\fboxsep}}\PYG{k+kn}{from} \PYG{n+nn}{muse}\PYG{n+nn}{.}\PYG{n+nn}{outputs} \PYG{k+kn}{import} \PYG{n}{register\PYGZus{}output\PYGZus{}quantity}
\PYG{k+kn}{from} \PYG{n+nn}{muse}\PYG{n+nn}{.}\PYG{n+nn}{outputs}\PYG{n+nn}{.}\PYG{n+nn}{sector} \PYG{k+kn}{import} \PYG{n}{market\PYGZus{}quantity}
\PYG{k+kn}{from} \PYG{n+nn}{xarray} \PYG{k+kn}{import} \PYG{n}{Dataset}\PYG{p}{,} \PYG{n}{DataArray}
\PYG{k+kn}{from} \PYG{n+nn}{typing} \PYG{k+kn}{import} \PYG{n}{Optional}\PYG{p}{,} \PYG{n}{List}\PYG{p}{,} \PYG{n}{Text}

\PYG{n+nd}{@register\PYGZus{}output\PYGZus{}quantity}
\PYG{k}{def} \PYG{n+nf}{consumption\PYGZus{}zero}\PYG{p}{(}
    \PYG{n}{market}\PYG{p}{:} \PYG{n}{Dataset}\PYG{p}{,}
    \PYG{n}{capacity}\PYG{p}{:} \PYG{n}{DataArray}\PYG{p}{,}
    \PYG{n}{technologies}\PYG{p}{:} \PYG{n}{Dataset}\PYG{p}{,}
\PYG{p}{)}\PYG{p}{:}
    \PYG{l+s+sd}{\PYGZdq{}\PYGZdq{}\PYGZdq{}Current consumption.\PYGZdq{}\PYGZdq{}\PYGZdq{}}
    \PYG{n}{result} \PYG{o}{=} \PYG{p}{(}
        \PYG{n}{market\PYGZus{}quantity}\PYG{p}{(}\PYG{n}{market}\PYG{o}{.}\PYG{n}{consumption}\PYG{p}{,} \PYG{n}{sum\PYGZus{}over}\PYG{o}{=}\PYG{l+s+s2}{\PYGZdq{}}\PYG{l+s+s2}{timeslice}\PYG{l+s+s2}{\PYGZdq{}}\PYG{p}{,} \PYG{n}{drop}\PYG{o}{=}\PYG{k+kc}{None}\PYG{p}{)}
        \PYG{o}{.}\PYG{n}{rename}\PYG{p}{(}\PYG{l+s+s2}{\PYGZdq{}}\PYG{l+s+s2}{consumption}\PYG{l+s+s2}{\PYGZdq{}}\PYG{p}{)}
        \PYG{o}{.}\PYG{n}{to\PYGZus{}dataframe}\PYG{p}{(}\PYG{p}{)}
        \PYG{o}{.}\PYG{n}{round}\PYG{p}{(}\PYG{l+m+mi}{4}\PYG{p}{)}
    \PYG{p}{)}
    \PYG{k}{return} \PYG{n}{result}
\end{sphinxVerbatim}
}

The function we created takes three arguments. These arguments (\sphinxcode{\sphinxupquote{market}}, \sphinxcode{\sphinxupquote{capacity}} and \sphinxcode{\sphinxupquote{technology}}) are mandatory for the \sphinxcode{\sphinxupquote{@register\_output\_quantity}} hook. Other hooks require different arguments.

Whilst this function is very similar to the \sphinxcode{\sphinxupquote{consumption}} function in \sphinxcode{\sphinxupquote{muse.outputs.sector}}, we have modified it slightly by allowing for values of \sphinxcode{\sphinxupquote{0}}.

The important part of this function is the \sphinxcode{\sphinxupquote{@register\_output\_quantity}} decorator. This decorator ensures that this new quantity is addressable in the TOML file. Notice that we did not need to edit the source code to create our new function.

Next, we can create a sink to save the output quantity previously registered. For this example, this sink will simply dump the quantity it is given to a file, with the “Hello world!” message:

{
\sphinxsetup{VerbatimColor={named}{nbsphinx-code-bg}}
\sphinxsetup{VerbatimBorderColor={named}{nbsphinx-code-border}}
\begin{sphinxVerbatim}[commandchars=\\\{\}]
\llap{\color{nbsphinxin}[2]:\,\hspace{\fboxrule}\hspace{\fboxsep}}\PYG{k+kn}{from} \PYG{n+nn}{typing} \PYG{k+kn}{import} \PYG{n}{Any}\PYG{p}{,} \PYG{n}{Text}
\PYG{k+kn}{from} \PYG{n+nn}{muse}\PYG{n+nn}{.}\PYG{n+nn}{outputs}\PYG{n+nn}{.}\PYG{n+nn}{sinks} \PYG{k+kn}{import} \PYG{n}{register\PYGZus{}output\PYGZus{}sink}\PYG{p}{,} \PYG{n}{sink\PYGZus{}to\PYGZus{}file}

\PYG{n+nd}{@register\PYGZus{}output\PYGZus{}sink}\PYG{p}{(}\PYG{n}{name}\PYG{o}{=}\PYG{l+s+s2}{\PYGZdq{}}\PYG{l+s+s2}{txt}\PYG{l+s+s2}{\PYGZdq{}}\PYG{p}{)}
\PYG{n+nd}{@sink\PYGZus{}to\PYGZus{}file}\PYG{p}{(}\PYG{l+s+s2}{\PYGZdq{}}\PYG{l+s+s2}{.txt}\PYG{l+s+s2}{\PYGZdq{}}\PYG{p}{)}
\PYG{k}{def} \PYG{n+nf}{text\PYGZus{}dump}\PYG{p}{(}\PYG{n}{data}\PYG{p}{:} \PYG{n}{Any}\PYG{p}{,} \PYG{n}{filename}\PYG{p}{:} \PYG{n}{Text}\PYG{p}{)} \PYG{o}{\PYGZhy{}}\PYG{o}{\PYGZgt{}} \PYG{k+kc}{None}\PYG{p}{:}
    \PYG{k+kn}{from} \PYG{n+nn}{pathlib} \PYG{k+kn}{import} \PYG{n}{Path}
    \PYG{n}{Path}\PYG{p}{(}\PYG{n}{filename}\PYG{p}{)}\PYG{o}{.}\PYG{n}{write\PYGZus{}text}\PYG{p}{(}\PYG{l+s+sa}{f}\PYG{l+s+s2}{\PYGZdq{}}\PYG{l+s+s2}{Hello world!}\PYG{l+s+se}{\PYGZbs{}n}\PYG{l+s+se}{\PYGZbs{}n}\PYG{l+s+si}{\PYGZob{}}\PYG{n}{data}\PYG{l+s+si}{\PYGZcb{}}\PYG{l+s+s2}{\PYGZdq{}}\PYG{p}{)}
\end{sphinxVerbatim}
}

The code above makes use of two dectorators: \sphinxcode{\sphinxupquote{@register\_output\_sink}} and \sphinxcode{\sphinxupquote{@sink\_to\_file}}.

\sphinxcode{\sphinxupquote{@register\_output\_sink}} registers the function with MUSE, so that the sink is addressable from a TOML file. The second one, \sphinxcode{\sphinxupquote{@sink\_to\_file}}, is optional. This adds some nice\sphinxhyphen{}to\sphinxhyphen{}have features to sinks that are files. For example, a way to specify filenames and check that files cannot be overwritten, unless explicitly allowed to.

Next, we want to modify the TOML file to actually use this output type. To do this, we add a section to the output table:

\begin{sphinxVerbatim}[commandchars=\\\{\}]
\PYG{k}{[[sectors.residential.outputs]]}
\PYG{n}{quantity} \PYG{o}{=} \PYG{l+s}{\PYGZdq{}consumption\PYGZus{}zero\PYGZdq{}}
\PYG{n}{sink} \PYG{o}{=} \PYG{l+s}{\PYGZdq{}txt\PYGZdq{}}
\PYG{n}{filename} \PYG{o}{=} \PYG{l+s}{\PYGZdq{}\PYGZob{}cwd\PYGZcb{}/\PYGZob{}default\PYGZus{}output\PYGZus{}dir\PYGZcb{}/\PYGZob{}Sector\PYGZcb{}\PYGZob{}Quantity\PYGZcb{}\PYGZob{}year\PYGZcb{}\PYGZob{}suffix\PYGZcb{}\PYGZdq{}}
\end{sphinxVerbatim}

The last line above allows us to specify the name of the file. We could also use \sphinxcode{\sphinxupquote{sector}} above or \sphinxcode{\sphinxupquote{quantity}}.

There can be as many sections of this kind as we like in the TOML file, which allow for multiple outputs.

Next, we first copy the default model provided with muse to a local subfolder called “model”. Then we read the \sphinxcode{\sphinxupquote{settings.toml}} file and modify it using python. You may prefer to modify the \sphinxcode{\sphinxupquote{settings.toml}} file using your favorite text editor. However, modifying the file programmatically allows us to routinely run this notebook as part of MUSE’s test suite and check that the tutorial it is still up to date.

{
\sphinxsetup{VerbatimColor={named}{nbsphinx-code-bg}}
\sphinxsetup{VerbatimBorderColor={named}{nbsphinx-code-border}}
\begin{sphinxVerbatim}[commandchars=\\\{\}]
\llap{\color{nbsphinxin}[3]:\,\hspace{\fboxrule}\hspace{\fboxsep}}\PYG{k+kn}{from} \PYG{n+nn}{pathlib} \PYG{k+kn}{import} \PYG{n}{Path}
\PYG{k+kn}{from} \PYG{n+nn}{toml} \PYG{k+kn}{import} \PYG{n}{load}\PYG{p}{,} \PYG{n}{dump}
\PYG{k+kn}{from} \PYG{n+nn}{muse} \PYG{k+kn}{import} \PYG{n}{examples}

\PYG{n}{model\PYGZus{}path} \PYG{o}{=} \PYG{n}{examples}\PYG{o}{.}\PYG{n}{copy\PYGZus{}model}\PYG{p}{(}\PYG{n}{overwrite}\PYG{o}{=}\PYG{k+kc}{True}\PYG{p}{)}
\PYG{n}{settings} \PYG{o}{=} \PYG{n}{load}\PYG{p}{(}\PYG{n}{model\PYGZus{}path} \PYG{o}{/} \PYG{l+s+s2}{\PYGZdq{}}\PYG{l+s+s2}{settings.toml}\PYG{l+s+s2}{\PYGZdq{}}\PYG{p}{)}
\PYG{n}{new\PYGZus{}output} \PYG{o}{=} \PYG{p}{\PYGZob{}}
    \PYG{l+s+s2}{\PYGZdq{}}\PYG{l+s+s2}{quantity}\PYG{l+s+s2}{\PYGZdq{}}\PYG{p}{:} \PYG{l+s+s2}{\PYGZdq{}}\PYG{l+s+s2}{consumption\PYGZus{}zero}\PYG{l+s+s2}{\PYGZdq{}}\PYG{p}{,}
    \PYG{l+s+s2}{\PYGZdq{}}\PYG{l+s+s2}{sink}\PYG{l+s+s2}{\PYGZdq{}}\PYG{p}{:}  \PYG{l+s+s2}{\PYGZdq{}}\PYG{l+s+s2}{txt}\PYG{l+s+s2}{\PYGZdq{}}\PYG{p}{,}
    \PYG{l+s+s2}{\PYGZdq{}}\PYG{l+s+s2}{overwrite}\PYG{l+s+s2}{\PYGZdq{}}\PYG{p}{:} \PYG{k+kc}{True}\PYG{p}{,}
    \PYG{l+s+s2}{\PYGZdq{}}\PYG{l+s+s2}{filename}\PYG{l+s+s2}{\PYGZdq{}}\PYG{p}{:} \PYG{l+s+s2}{\PYGZdq{}}\PYG{l+s+si}{\PYGZob{}cwd\PYGZcb{}}\PYG{l+s+s2}{/}\PYG{l+s+si}{\PYGZob{}default\PYGZus{}output\PYGZus{}dir\PYGZcb{}}\PYG{l+s+s2}{/}\PYG{l+s+si}{\PYGZob{}Sector\PYGZcb{}}\PYG{l+s+si}{\PYGZob{}Quantity\PYGZcb{}}\PYG{l+s+si}{\PYGZob{}year\PYGZcb{}}\PYG{l+s+si}{\PYGZob{}suffix\PYGZcb{}}\PYG{l+s+s2}{\PYGZdq{}}\PYG{p}{,}
\PYG{p}{\PYGZcb{}}
\PYG{n}{settings}\PYG{p}{[}\PYG{l+s+s2}{\PYGZdq{}}\PYG{l+s+s2}{sectors}\PYG{l+s+s2}{\PYGZdq{}}\PYG{p}{]}\PYG{p}{[}\PYG{l+s+s2}{\PYGZdq{}}\PYG{l+s+s2}{residential}\PYG{l+s+s2}{\PYGZdq{}}\PYG{p}{]}\PYG{p}{[}\PYG{l+s+s2}{\PYGZdq{}}\PYG{l+s+s2}{outputs}\PYG{l+s+s2}{\PYGZdq{}}\PYG{p}{]}\PYG{o}{.}\PYG{n}{append}\PYG{p}{(}\PYG{n}{new\PYGZus{}output}\PYG{p}{)}
\PYG{n}{dump}\PYG{p}{(}\PYG{n}{settings}\PYG{p}{,} \PYG{p}{(}\PYG{n}{model\PYGZus{}path} \PYG{o}{/} \PYG{l+s+s2}{\PYGZdq{}}\PYG{l+s+s2}{modified\PYGZus{}settings.toml}\PYG{l+s+s2}{\PYGZdq{}}\PYG{p}{)}\PYG{o}{.}\PYG{n}{open}\PYG{p}{(}\PYG{l+s+s2}{\PYGZdq{}}\PYG{l+s+s2}{w}\PYG{l+s+s2}{\PYGZdq{}}\PYG{p}{)}\PYG{p}{)}
\PYG{n}{settings}
\end{sphinxVerbatim}
}

{

\kern-\sphinxverbatimsmallskipamount\kern-\baselineskip
\kern+\FrameHeightAdjust\kern-\fboxrule
\vspace{\nbsphinxcodecellspacing}

\sphinxsetup{VerbatimColor={named}{nbsphinx-stderr}}
\sphinxsetup{VerbatimBorderColor={named}{nbsphinx-code-border}}
\begin{sphinxVerbatim}[commandchars=\\\{\}]
-- 2020-11-09 11:19:48 - muse.sectors.register - INFO
Sector legacy registered.

-- 2020-11-09 11:19:48 - muse.sectors.register - INFO
Sector preset registered, with alias presets.

-- 2020-11-09 11:19:48 - muse.sectors.register - INFO
Sector default registered.

\end{sphinxVerbatim}
}

{

\kern-\sphinxverbatimsmallskipamount\kern-\baselineskip
\kern+\FrameHeightAdjust\kern-\fboxrule
\vspace{\nbsphinxcodecellspacing}

\sphinxsetup{VerbatimColor={named}{white}}
\sphinxsetup{VerbatimBorderColor={named}{nbsphinx-code-border}}
\begin{sphinxVerbatim}[commandchars=\\\{\}]
\llap{\color{nbsphinxout}[3]:\,\hspace{\fboxrule}\hspace{\fboxsep}}\{'time\_framework': [2020, 2025, 2030, 2035, 2040, 2045, 2050],
 'foresight': 5,
 'regions': ['R1'],
 'interest\_rate': 0.1,
 'interpolation\_mode': 'Active',
 'log\_level': 'info',
 'equilibrium\_variable': 'demand',
 'maximum\_iterations': 100,
 'tolerance': 0.1,
 'tolerance\_unmet\_demand': -0.1,
 'outputs': [\{'quantity': 'prices',
   'sink': 'aggregate',
   'filename': '\{cwd\}/\{default\_output\_dir\}/MCA\{Quantity\}.csv'\},
  \{'quantity': 'capacity',
   'sink': 'aggregate',
   'filename': '\{cwd\}/\{default\_output\_dir\}/MCA\{Quantity\}.csv'\}],
 'carbon\_budget\_control': \{'budget': []\},
 'global\_input\_files': \{'projections': '\{path\}/input/Projections.csv',
  'global\_commodities': '\{path\}/input/GlobalCommodities.csv'\},
 'sectors': \{'residential': \{'type': 'default',
   'priority': 1,
   'dispatch\_production': 'share',
   'technodata': '\{path\}/technodata/residential/Technodata.csv',
   'commodities\_in': '\{path\}/technodata/residential/CommIn.csv',
   'commodities\_out': '\{path\}/technodata/residential/CommOut.csv',
   'subsectors': \{'retro\_and\_new': \{'agents': '\{path\}/technodata/Agents.csv',
     'existing\_capacity': '\{path\}/technodata/residential/ExistingCapacity.csv',
     'lpsolver': 'scipy',
     'constraints': ['max\_production',
      'max\_capacity\_expansion',
      'demand',
      'search\_space'],
     'demand\_share': 'new\_and\_retro',
     'forecast': 5\}\},
   'outputs': [\{'filename': '\{cwd\}/\{default\_output\_dir\}/\{Sector\}/\{Quantity\}/\{year\}\{suffix\}',
     'quantity': 'capacity',
     'sink': 'csv',
     'overwrite': True\},
    \{'filename': '\{cwd\}/\{default\_output\_dir\}/\{Sector\}/\{Quantity\}/\{year\}\{suffix\}',
     'quantity': \{'name': 'supply',
      'sum\_over': 'timeslice',
      'drop': ['comm\_usage', 'units\_prices']\},
     'sink': 'csv',
     'overwrite': True\},
    \{'quantity': 'consumption\_zero',
     'sink': 'txt',
     'overwrite': True,
     'filename': '\{cwd\}/\{default\_output\_dir\}/\{Sector\}\{Quantity\}\{year\}\{suffix\}'\}],
   'interactions': [\{'net': 'new\_to\_retro', 'interaction': 'transfer'\}]\},
  'power': \{'type': 'default',
   'priority': 2,
   'dispatch\_production': 'share',
   'technodata': '\{path\}/technodata/power/Technodata.csv',
   'commodities\_in': '\{path\}/technodata/power/CommIn.csv',
   'commodities\_out': '\{path\}/technodata/power/CommOut.csv',
   'subsectors': \{'retro\_and\_new': \{'agents': '\{path\}/technodata/Agents.csv',
     'existing\_capacity': '\{path\}/technodata/power/ExistingCapacity.csv',
     'lpsolver': 'scipy'\}\},
   'outputs': [\{'filename': '\{cwd\}/\{default\_output\_dir\}/\{Sector\}/\{Quantity\}/\{year\}\{suffix\}',
     'quantity': 'capacity',
     'sink': 'csv',
     'overwrite': True\}],
   'interactions': [\{'net': 'new\_to\_retro', 'interaction': 'transfer'\}]\},
  'gas': \{'type': 'default',
   'priority': 3,
   'dispatch\_production': 'share',
   'technodata': '\{path\}/technodata/gas/Technodata.csv',
   'commodities\_in': '\{path\}/technodata/gas/CommIn.csv',
   'commodities\_out': '\{path\}/technodata/gas/CommOut.csv',
   'subsectors': \{'retro\_and\_new': \{'agents': '\{path\}/technodata/Agents.csv',
     'existing\_capacity': '\{path\}/technodata/gas/ExistingCapacity.csv',
     'lpsolver': 'scipy'\}\},
   'outputs': [\{'filename': '\{cwd\}/\{default\_output\_dir\}/\{Sector\}/\{Quantity\}/\{year\}\{suffix\}',
     'quantity': 'capacity',
     'sink': 'csv',
     'overwrite': True\}],
   'interactions': [\{'net': 'new\_to\_retro', 'interaction': 'transfer'\}]\},
  'residential\_presets': \{'type': 'presets',
   'priority': 0,
   'consumption\_path': '\{path\}/technodata/preset/*Consumption.csv'\}\},
 'timeslices': \{'all-year': \{'all-week': \{'night': 1460,
    'morning': 1460,
    'afternoon': 1460,
    'early-peak': 1460,
    'late-peak': 1460,
    'evening': 1460\}\},
  'level\_names': ['month', 'day', 'hour']\}\}
\end{sphinxVerbatim}
}

We can now run the simulation. There are two ways to do this. From the command\sphinxhyphen{}line, where we can do:

\begin{sphinxVerbatim}[commandchars=\\\{\}]
\PYG{n}{python3} \PYG{o}{\PYGZhy{}}\PYG{n}{m} \PYG{n}{muse} \PYG{n}{data}\PYG{o}{/}\PYG{n}{commercial}\PYG{o}{/}\PYG{n}{modified\PYGZus{}settings}\PYG{o}{.}\PYG{n}{toml}
\end{sphinxVerbatim}

(note that slashes may be the other way on Windows). Or directly from the notebook:

{
\sphinxsetup{VerbatimColor={named}{nbsphinx-code-bg}}
\sphinxsetup{VerbatimBorderColor={named}{nbsphinx-code-border}}
\begin{sphinxVerbatim}[commandchars=\\\{\}]
\llap{\color{nbsphinxin}[4]:\,\hspace{\fboxrule}\hspace{\fboxsep}}\PYG{k+kn}{import} \PYG{n+nn}{logging}
\PYG{k+kn}{from} \PYG{n+nn}{muse}\PYG{n+nn}{.}\PYG{n+nn}{mca} \PYG{k+kn}{import} \PYG{n}{MCA}
\PYG{n}{logging}\PYG{o}{.}\PYG{n}{getLogger}\PYG{p}{(}\PYG{l+s+s2}{\PYGZdq{}}\PYG{l+s+s2}{muse}\PYG{l+s+s2}{\PYGZdq{}}\PYG{p}{)}\PYG{o}{.}\PYG{n}{setLevel}\PYG{p}{(}\PYG{l+m+mi}{0}\PYG{p}{)}
\PYG{n}{mca} \PYG{o}{=} \PYG{n}{MCA}\PYG{o}{.}\PYG{n}{factory}\PYG{p}{(}\PYG{n}{model\PYGZus{}path} \PYG{o}{/} \PYG{l+s+s2}{\PYGZdq{}}\PYG{l+s+s2}{modified\PYGZus{}settings.toml}\PYG{l+s+s2}{\PYGZdq{}}\PYG{p}{)}
\PYG{n}{mca}\PYG{o}{.}\PYG{n}{run}\PYG{p}{(}\PYG{p}{)}\PYG{p}{;}
\end{sphinxVerbatim}
}

{

\kern-\sphinxverbatimsmallskipamount\kern-\baselineskip
\kern+\FrameHeightAdjust\kern-\fboxrule
\vspace{\nbsphinxcodecellspacing}

\sphinxsetup{VerbatimColor={named}{white}}
\sphinxsetup{VerbatimBorderColor={named}{nbsphinx-code-border}}
\begin{sphinxVerbatim}[commandchars=\\\{\}]
Primal Feasibility  Dual Feasibility    Duality Gap         Step             Path Parameter      Objective
1.0                 1.0                 1.0                 -                1.0                 148.9679256735
0.2598249156018     0.2598249156018     0.2598249156018     0.7495200004432  0.2598249156018     120.5733849622
0.02399956829695    0.02399956829695    0.02399956829695    0.9210391224498  0.02399956829695    4.780663765494
0.0181364461758     0.0181364461758     0.0181364461758     0.2509588065043  0.0181364461758     7.107141691547
0.01499350833129    0.01499350833129    0.01499350833129    0.1921973185437  0.01499350833129    70.77614035582
0.004968295711366   0.004968295711367   0.004968295711366   0.6857131120066  0.004968295711366   164.7472224003
0.0006443120819652  0.0006443120819642  0.000644312081964   0.8804718592549  0.0006443120819672  289.7109372802
2.427431365313e-06  2.427431365276e-06  2.42743136527e-06   0.9976309182175  2.427431365399e-06  310.7437190082
1.214286379284e-10  1.214286245985e-10  1.214286217581e-10  0.9999499778566  1.214286284187e-10  310.7859704917
Optimization terminated successfully.
         Current function value: 310.785970
         Iterations: 8
Primal Feasibility  Dual Feasibility    Duality Gap         Step             Path Parameter      Objective
1.0                 1.0                 1.0                 -                1.0                 148.9679256735
0.1806451257467     0.1806451257467     0.1806451257467     0.8281847212959  0.1806451257467     169.6037982942
0.02684129624378    0.02684129624378    0.02684129624378    0.8635116331789  0.02684129624378    123.2904723181
0.01081107082374    0.01081107082373    0.01081107082373    0.6279145428497  0.01081107082373    293.3552576002
0.00150335333755    0.001503353337531   0.001503353337531   0.8785435425234  0.001503353337582   618.5754737903
4.126548293386e-06  4.126548293452e-06  4.126548293469e-06  0.99729939758    4.126548293473e-06  673.2429034259
2.063813940498e-10  2.063814559148e-10  2.063814538311e-10  0.9999499869317  2.06381853248e-10   673.369600975
Optimization terminated successfully.
         Current function value: 673.369601
         Iterations: 6
Primal Feasibility  Dual Feasibility    Duality Gap         Step             Path Parameter      Objective
1.0                 1.0                 1.0                 -                1.0                 225.7506433631
0.07675788061753    0.07675788061753    0.07675788061751    0.9271368109475  0.07675788061753    1.307751786207
0.01889464099889    0.01889464099889    0.01889464099888    0.7970949255449  0.01889464099889    19.4989221165
0.007543783963398   0.007543783963394   0.007543783963392   0.6153162486386  0.007543783963394   16.52048983639
0.002504946781023   0.002504946781021   0.00250494678102    0.7004074675406  0.002504946781021   72.319459457
0.0004445444355394  0.000444544435539   0.0004445444355388  0.8689995047748  0.000444544435539   423.758860583
1.214501095331e-05  1.214501095325e-05  1.214501095324e-05  0.9820606335229  1.214501095322e-05  675.1646942955
1.070439776445e-09  1.07043978707e-09   1.070439766673e-09  0.9999120772687  1.070439801419e-09  681.9179209593
5.353929135063e-14  5.353592310399e-14  5.35373239097e-14   0.9999499866071  5.352230256347e-14  681.9185224492
Optimization terminated successfully.
         Current function value: 681.918522
         Iterations: 8
\end{sphinxVerbatim}
}

{

\kern-\sphinxverbatimsmallskipamount\kern-\baselineskip
\kern+\FrameHeightAdjust\kern-\fboxrule
\vspace{\nbsphinxcodecellspacing}

\sphinxsetup{VerbatimColor={named}{nbsphinx-stderr}}
\sphinxsetup{VerbatimBorderColor={named}{nbsphinx-code-border}}
\begin{sphinxVerbatim}[commandchars=\\\{\}]
-- 2020-11-09 11:19:59 - muse.mca - WARNING
Check growth constraints for wind.

\end{sphinxVerbatim}
}

{

\kern-\sphinxverbatimsmallskipamount\kern-\baselineskip
\kern+\FrameHeightAdjust\kern-\fboxrule
\vspace{\nbsphinxcodecellspacing}

\sphinxsetup{VerbatimColor={named}{white}}
\sphinxsetup{VerbatimBorderColor={named}{nbsphinx-code-border}}
\begin{sphinxVerbatim}[commandchars=\\\{\}]
Primal Feasibility  Dual Feasibility    Duality Gap         Step             Path Parameter      Objective
1.0                 1.0                 1.0                 -                1.0                 359.2443189825
0.2131118149695     0.2131118149695     0.2131118149695     0.7960343319409  0.2131118149695     262.2885392799
0.05758094728718    0.05758094728718    0.05758094728718    0.7523513573813  0.05758094728718    13.16175379893
0.01048349585899    0.01048349585899    0.01048349585899    0.8203940076989  0.01048349585899    9.036399448741
0.009005598049604   0.009005598049604   0.009005598049604   0.1488155054953  0.009005598049604   19.40296370843
0.002317604003518   0.002317604003518   0.002317604003518   0.8098673571979  0.002317604003518   226.5492459765
0.0009784310820962  0.0009784310820963  0.0009784310820963  0.5991550275447  0.0009784310820971  289.4684048776
0.0001326875071986  0.0001326875071986  0.0001326875071986  0.8836343167337  0.0001326875071987  358.6919881748
6.688262823007e-08  6.688262823276e-08  6.688262823391e-08  0.9995454959391  6.688262822145e-08  365.8223849509
3.344208692489e-12  3.34420597683e-12   3.344204125008e-12  0.9999499989235  3.344177862259e-12  365.8250744356
Optimization terminated successfully.
         Current function value: 365.825074
         Iterations: 9
Primal Feasibility  Dual Feasibility    Duality Gap         Step             Path Parameter      Objective
1.0                 1.0                 1.0                 -                1.0                 179.6221594912
0.06792119055695    0.06792119055695    0.06792119055695    0.9362017234058  0.06792119055695    76.26287556551
0.009746786382626   0.009746786382626   0.009746786382626   0.8719786940257  0.009746786382625   34.28929487798
0.00504956460969    0.005049564609689   0.005049564609689   0.5134005827998  0.005049564609689   115.6145513358
0.003132107070668   0.003132107070663   0.003132107070663   0.3922116719677  0.003132107070663   175.0444435627
0.0005331490663204  0.000533149066321   0.000533149066321   0.8631478238108  0.0005331490663195  430.7265923665
6.809758501168e-06  6.809758501084e-06  6.809758501069e-06  0.9896962642248  6.809758501216e-06  482.9615875673
3.566453823927e-10  3.56645399101e-10   3.566453982014e-10  0.9999476448412  3.566453742836e-10  483.66367672
Optimization terminated successfully.
         Current function value: 483.663677
         Iterations: 7
Primal Feasibility  Dual Feasibility    Duality Gap         Step             Path Parameter      Objective
1.0                 1.0                 1.0                 -                1.0                 547.5170704708
0.06099397574481    0.06099397574481    0.06099397574482    0.944387525785   0.06099397574482    192.4316112656
0.01386491315737    0.01386491315737    0.01386491315737    0.7896145976147  0.01386491315737    3.157754203839
0.00713143943229    0.00713143943229    0.007131439432292   0.5116778496497  0.007131439432291   6.913227303095
0.0007854492963609  0.0007854492963609  0.0007854492963611  0.9057153968575  0.000785449296361   6.591304425235
0.0003968642772996  0.0003968642772997  0.0003968642772998  0.5195625901748  0.0003968642772996  5.218772647071
5.276614006577e-06  5.276614006577e-06  5.276614006576e-06  0.9941140946865  5.276614006565e-06  5.272614338781
3.679893198137e-10  3.679893242575e-10  3.679893250277e-10  0.9999303117912  3.679893357516e-10  5.266601980833
1.790517523351e-14  1.842347640038e-14  1.841384711876e-14  0.9999499611495  1.844295319492e-14  5.266601636509
Optimization terminated successfully.
         Current function value: 5.266602
         Iterations: 8
Primal Feasibility  Dual Feasibility    Duality Gap         Step             Path Parameter      Objective
1.0                 1.0                 1.0                 -                1.0                 273.7585352354
0.06117571447458    0.06117571447458    0.06117571447455    0.9425787965419  0.06117571447457    204.7212184456
0.009942269802024   0.009942269802025   0.00994226980202    0.8403658088806  0.009942269802024   0.7760264390675
0.002147318631478   0.002147318631478   0.002147318631477   0.8277733790617  0.002147318631478   3.894705138429
0.0009649606635229  0.0009649606635227  0.0009649606635223  0.5635818341147  0.0009649606635226  2.700821086695
0.0002350111516202  0.0002350111516201  0.00023501115162    0.7866567764877  0.0002350111516201  14.62347819316
5.857294640779e-05  5.857294640777e-05  5.857294640774e-05  0.8105927228255  5.857294640777e-05  123.50656776
1.165473006364e-06  1.165473006407e-06  1.165473006407e-06  0.9835504451632  1.165473006409e-06  151.6654340598
8.362816702708e-11  8.362816612134e-11  8.362816648317e-11  0.9999285085203  8.36281608009e-11   152.3843167925
4.185697343433e-15  4.184037681176e-15  4.18416825892e-15   0.9999499678098  4.181415889613e-15  152.3843678125
Optimization terminated successfully.
         Current function value: 152.384368
         Iterations: 9
Primal Feasibility  Dual Feasibility    Duality Gap         Step             Path Parameter      Objective
1.0                 1.0                 1.0                 -                1.0                 24.59260538017
0.01918847234907    0.01918847234907    0.01918847234907    0.9908439847925  0.01918847234907    0.01449776485458
0.01221126742797    0.01221126742797    0.01221126742797    0.3735738645022  0.01221126742797    0.1603023022985
0.009831906616685   0.009831906616685   0.009831906616685   0.2161492677935  0.009831906616685   20.38072882765
0.001212407727123   0.001212407727023   0.001212407727023   0.8886055956883  0.001212407727023   51.40232805869
4.730913157321e-06  4.73091315438e-06   4.730913154368e-06  0.9976178802846  4.730913155107e-06  59.83474944668
2.366017130877e-10  2.366016942176e-10  2.366016879444e-10  0.9999499881554  2.366016905478e-10  59.84200873085
Optimization terminated successfully.
         Current function value: 59.842009
         Iterations: 6
\end{sphinxVerbatim}
}

{

\kern-\sphinxverbatimsmallskipamount\kern-\baselineskip
\kern+\FrameHeightAdjust\kern-\fboxrule
\vspace{\nbsphinxcodecellspacing}

\sphinxsetup{VerbatimColor={named}{nbsphinx-stderr}}
\sphinxsetup{VerbatimBorderColor={named}{nbsphinx-code-border}}
\begin{sphinxVerbatim}[commandchars=\\\{\}]
-- 2020-11-09 11:20:09 - muse.mca - WARNING
Check growth constraints for wind.

\end{sphinxVerbatim}
}

{

\kern-\sphinxverbatimsmallskipamount\kern-\baselineskip
\kern+\FrameHeightAdjust\kern-\fboxrule
\vspace{\nbsphinxcodecellspacing}

\sphinxsetup{VerbatimColor={named}{white}}
\sphinxsetup{VerbatimBorderColor={named}{nbsphinx-code-border}}
\begin{sphinxVerbatim}[commandchars=\\\{\}]
Primal Feasibility  Dual Feasibility    Duality Gap         Step             Path Parameter      Objective
1.0                 1.0                 1.0                 -                1.0                 359.2443189825
0.2131118149695     0.2131118149695     0.2131118149695     0.7960343319409  0.2131118149695     262.2885392799
0.05758094728718    0.05758094728718    0.05758094728718    0.7523513573813  0.05758094728718    13.16175379893
0.01048349585899    0.01048349585899    0.01048349585899    0.8203940076989  0.01048349585899    9.036399448741
0.009005598049604   0.009005598049604   0.009005598049604   0.1488155054953  0.009005598049604   19.40296370843
0.002317604003518   0.002317604003518   0.002317604003518   0.8098673571979  0.002317604003518   226.5492459765
0.0009784310820962  0.0009784310820963  0.0009784310820963  0.5991550275447  0.0009784310820971  289.4684048776
0.0001326875071986  0.0001326875071986  0.0001326875071986  0.8836343167337  0.0001326875071987  358.6919881748
6.688262823007e-08  6.688262823276e-08  6.688262823391e-08  0.9995454959391  6.688262822145e-08  365.8223849509
3.344208692489e-12  3.34420597683e-12   3.344204125008e-12  0.9999499989235  3.344177862259e-12  365.8250744356
Optimization terminated successfully.
         Current function value: 365.825074
         Iterations: 9
Primal Feasibility  Dual Feasibility    Duality Gap         Step             Path Parameter      Objective
1.0                 1.0                 1.0                 -                1.0                 179.6221594912
0.06792119055695    0.06792119055695    0.06792119055695    0.9362017234058  0.06792119055695    76.26287556551
0.009746786382626   0.009746786382626   0.009746786382626   0.8719786940257  0.009746786382625   34.28929487798
0.00504956460969    0.005049564609689   0.005049564609689   0.5134005827998  0.005049564609689   115.6145513358
0.003132107070668   0.003132107070663   0.003132107070663   0.3922116719677  0.003132107070663   175.0444435627
0.0005331490663204  0.000533149066321   0.000533149066321   0.8631478238108  0.0005331490663195  430.7265923665
6.809758501168e-06  6.809758501084e-06  6.809758501069e-06  0.9896962642248  6.809758501216e-06  482.9615875673
3.566453823927e-10  3.56645399101e-10   3.566453982014e-10  0.9999476448412  3.566453742836e-10  483.66367672
Optimization terminated successfully.
         Current function value: 483.663677
         Iterations: 7
Primal Feasibility  Dual Feasibility    Duality Gap         Step             Path Parameter      Objective
1.0                 1.0                 1.0                 -                1.0                 547.5170704708
0.06099397574481    0.06099397574481    0.06099397574482    0.944387525785   0.06099397574482    192.4316112656
0.01386491315737    0.01386491315737    0.01386491315737    0.7896145976147  0.01386491315737    3.157754203839
0.00713143943229    0.00713143943229    0.007131439432292   0.5116778496497  0.007131439432291   6.913227303095
0.0007854492963609  0.0007854492963609  0.0007854492963611  0.9057153968575  0.000785449296361   6.591304425235
0.0003968642772996  0.0003968642772997  0.0003968642772998  0.5195625901748  0.0003968642772996  5.218772647071
5.276614006577e-06  5.276614006577e-06  5.276614006576e-06  0.9941140946865  5.276614006565e-06  5.272614338781
3.679893198137e-10  3.679893242575e-10  3.679893250277e-10  0.9999303117912  3.679893357516e-10  5.266601980833
1.790517523351e-14  1.842347640038e-14  1.841384711876e-14  0.9999499611495  1.844295319492e-14  5.266601636509
Optimization terminated successfully.
         Current function value: 5.266602
         Iterations: 8
Primal Feasibility  Dual Feasibility    Duality Gap         Step             Path Parameter      Objective
1.0                 1.0                 1.0                 -                1.0                 273.7585352354
0.06117571447458    0.06117571447458    0.06117571447455    0.9425787965419  0.06117571447457    204.7212184456
0.009942269802024   0.009942269802025   0.00994226980202    0.8403658088806  0.009942269802024   0.7760264390675
0.002147318631478   0.002147318631478   0.002147318631477   0.8277733790617  0.002147318631478   3.894705138429
0.0009649606635229  0.0009649606635227  0.0009649606635223  0.5635818341147  0.0009649606635226  2.700821086695
0.0002350111516202  0.0002350111516201  0.00023501115162    0.7866567764877  0.0002350111516201  14.62347819316
5.857294640779e-05  5.857294640777e-05  5.857294640774e-05  0.8105927228255  5.857294640777e-05  123.50656776
1.165473006364e-06  1.165473006407e-06  1.165473006407e-06  0.9835504451632  1.165473006409e-06  151.6654340598
8.362816702708e-11  8.362816612134e-11  8.362816648317e-11  0.9999285085203  8.36281608009e-11   152.3843167925
4.185697343433e-15  4.184037681176e-15  4.18416825892e-15   0.9999499678098  4.181415889613e-15  152.3843678125
Optimization terminated successfully.
         Current function value: 152.384368
         Iterations: 9
Primal Feasibility  Dual Feasibility    Duality Gap         Step             Path Parameter      Objective
1.0                 1.0                 1.0                 -                1.0                 24.59260538017
0.01918847234907    0.01918847234907    0.01918847234907    0.9908439847925  0.01918847234907    0.01449776485458
0.01221126742797    0.01221126742797    0.01221126742797    0.3735738645022  0.01221126742797    0.1603023022985
0.009831906616685   0.009831906616685   0.009831906616685   0.2161492677935  0.009831906616685   20.38072882765
0.001212407727123   0.001212407727023   0.001212407727023   0.8886055956883  0.001212407727023   51.40232805869
4.730913157321e-06  4.73091315438e-06   4.730913154368e-06  0.9976178802846  4.730913155107e-06  59.83474944668
2.366017130877e-10  2.366016942176e-10  2.366016879444e-10  0.9999499881554  2.366016905478e-10  59.84200873085
Optimization terminated successfully.
         Current function value: 59.842009
         Iterations: 6
\end{sphinxVerbatim}
}

{

\kern-\sphinxverbatimsmallskipamount\kern-\baselineskip
\kern+\FrameHeightAdjust\kern-\fboxrule
\vspace{\nbsphinxcodecellspacing}

\sphinxsetup{VerbatimColor={named}{nbsphinx-stderr}}
\sphinxsetup{VerbatimBorderColor={named}{nbsphinx-code-border}}
\begin{sphinxVerbatim}[commandchars=\\\{\}]
-- 2020-11-09 11:20:18 - muse.mca - WARNING
Check growth constraints for wind.

\end{sphinxVerbatim}
}

{

\kern-\sphinxverbatimsmallskipamount\kern-\baselineskip
\kern+\FrameHeightAdjust\kern-\fboxrule
\vspace{\nbsphinxcodecellspacing}

\sphinxsetup{VerbatimColor={named}{white}}
\sphinxsetup{VerbatimBorderColor={named}{nbsphinx-code-border}}
\begin{sphinxVerbatim}[commandchars=\\\{\}]
Primal Feasibility  Dual Feasibility    Duality Gap         Step             Path Parameter      Objective
1.0                 1.0                 1.0                 -                1.0                 414.316476678
0.1943112264485     0.1943112264485     0.1943112264485     0.8115414580695  0.1943112264485     293.1180738751
0.05737071259203    0.05737071259203    0.05737071259203    0.7291321197238  0.05737071259203    17.07106219013
0.01062277327644    0.01062277327644    0.01062277327644    0.8173814867435  0.01062277327644    12.44885046687
0.009064514495371   0.009064514495372   0.009064514495372   0.1550649500025  0.009064514495372   28.00868491133
0.002334119807294   0.002334119807293   0.002334119807293   0.8051326751218  0.002334119807295   275.8552661516
0.0006946146456239  0.0006946146456238  0.0006946146456238  0.7097377670861  0.0006946146456242  332.7560201847
0.000253341221249   0.0002533412212489  0.0002533412212489  0.6718481634042  0.0002533412212491  377.2933943083
1.33722226265e-06   1.337222262655e-06  1.337222262657e-06  0.9952598213341  1.337222262651e-06  392.3599718847
6.775876566156e-11  6.775875940049e-11  6.775875524037e-11  0.9999493313269  6.775868905564e-11  392.4568216146
Optimization terminated successfully.
         Current function value: 392.456822
         Iterations: 9
Primal Feasibility  Dual Feasibility    Duality Gap         Step             Path Parameter      Objective
1.0                 1.0                 1.0                 -                1.0                 414.316476678
0.1030590385675     0.1030590385675     0.1030590385675     0.9011154818883  0.1030590385675     219.2748221074
0.02654054666621    0.02654054666621    0.02654054666621    0.7752562655553  0.02654054666621    276.7551091027
0.01427394633699    0.01427394633699    0.01427394633699    0.4912526709581  0.01427394633699    607.4391713228
0.001218580492651   0.001218580492601   0.001218580492601   0.9248974186534  0.001218580492676   1166.289755796
1.888403761715e-05  1.888403761622e-05  1.888403761624e-05  0.985551099478   1.888403761735e-05  1241.740132447
1.545745837373e-09  1.545745372552e-09  1.545745362755e-09  0.999922927438   1.545745648845e-09  1242.779847889
1.692223225044e-10  1.873153075245e-10  1.873152955153e-10  0.8801597673128  1.817506825368e-10  1242.779937345
Optimization terminated successfully.
         Current function value: 1242.779937
         Iterations: 7
Primal Feasibility  Dual Feasibility    Duality Gap         Step             Path Parameter      Objective
1.0                 1.0                 1.0                 -                1.0                 652.1069903706
0.06097571429344    0.06097571429346    0.06097571429347    0.9444902662603  0.06097571429346    285.2043200632
0.01533833555931    0.01533833555931    0.01533833555932    0.7609599230413  0.01533833555932    2.7262007615
0.006316196361705   0.006316196361706   0.006316196361708   0.6163829523723  0.006316196361707   8.634999065661
0.001219580332373   0.001219580332374   0.001219580332374   0.8323480537452  0.001219580332374   8.307794697355
0.0005645791470893  0.0005645791470901  0.0005645791470902  0.5500511910892  0.0005645791470901  6.632034269378
0.0001849432525583  0.0001849432525588  0.0001849432525588  0.7137620753374  0.0001849432525588  4.483601470858
2.517759634035e-05  2.517759634041e-05  2.517759634042e-05  0.9034269505801  2.517759634041e-05  4.138476382855
3.098182398421e-08  3.098182439526e-08  3.09818244027e-08   0.9988322072353  3.098182439405e-08  4.000023749419
1.548986338477e-12  1.549101875116e-12  1.549109003245e-12  0.9999499994251  1.549106268784e-12  3.999950652469
Optimization terminated successfully.
         Current function value: 3.999951
         Iterations: 9
Primal Feasibility  Dual Feasibility    Duality Gap         Step             Path Parameter      Objective
1.0                 1.0                 1.0                 -                1.0                 24.59260538017
0.02944420803734    0.02944420803734    0.02944420803733    0.9778743187132  0.02944420803734    27.12355968787
0.008204037532537   0.008204037532537   0.008204037532536   0.7626654784836  0.008204037532537   110.4823556367
0.0004115277845134  0.0004115277844911  0.000411527784491   0.9587152757388  0.0004115277845217  181.7019444318
2.842168973263e-08  2.842168968894e-08  2.84216896431e-08   0.9999314505927  2.84216896475e-08   184.4443098122
1.421064808692e-12  1.421032025856e-12  1.420975506573e-12  0.999950002033   1.421084481186e-12  184.444539173
Optimization terminated successfully.
         Current function value: 184.444539
         Iterations: 5
\end{sphinxVerbatim}
}

{

\kern-\sphinxverbatimsmallskipamount\kern-\baselineskip
\kern+\FrameHeightAdjust\kern-\fboxrule
\vspace{\nbsphinxcodecellspacing}

\sphinxsetup{VerbatimColor={named}{nbsphinx-stderr}}
\sphinxsetup{VerbatimBorderColor={named}{nbsphinx-code-border}}
\begin{sphinxVerbatim}[commandchars=\\\{\}]
-- 2020-11-09 11:20:25 - muse.mca - WARNING
Check growth constraints for wind.

\end{sphinxVerbatim}
}

{

\kern-\sphinxverbatimsmallskipamount\kern-\baselineskip
\kern+\FrameHeightAdjust\kern-\fboxrule
\vspace{\nbsphinxcodecellspacing}

\sphinxsetup{VerbatimColor={named}{white}}
\sphinxsetup{VerbatimBorderColor={named}{nbsphinx-code-border}}
\begin{sphinxVerbatim}[commandchars=\\\{\}]
Primal Feasibility  Dual Feasibility    Duality Gap         Step             Path Parameter      Objective
1.0                 1.0                 1.0                 -                1.0                 414.316476678
0.1943112264485     0.1943112264485     0.1943112264485     0.8115414580695  0.1943112264485     293.1180738751
0.05737071259203    0.05737071259203    0.05737071259203    0.7291321197238  0.05737071259203    17.07106219013
0.01062277327644    0.01062277327644    0.01062277327644    0.8173814867435  0.01062277327644    12.44885046687
0.009064514495371   0.009064514495372   0.009064514495372   0.1550649500025  0.009064514495372   28.00868491133
0.002334119807294   0.002334119807293   0.002334119807293   0.8051326751218  0.002334119807295   275.8552661516
0.0006946146456239  0.0006946146456238  0.0006946146456238  0.7097377670861  0.0006946146456242  332.7560201847
0.000253341221249   0.0002533412212489  0.0002533412212489  0.6718481634042  0.0002533412212491  377.2933943083
1.33722226265e-06   1.337222262655e-06  1.337222262657e-06  0.9952598213341  1.337222262651e-06  392.3599718847
6.775876566156e-11  6.775875940049e-11  6.775875524037e-11  0.9999493313269  6.775868905564e-11  392.4568216146
Optimization terminated successfully.
         Current function value: 392.456822
         Iterations: 9
Primal Feasibility  Dual Feasibility    Duality Gap         Step             Path Parameter      Objective
1.0                 1.0                 1.0                 -                1.0                 414.316476678
0.1030590385675     0.1030590385675     0.1030590385675     0.9011154818883  0.1030590385675     219.2748221074
0.02654054666621    0.02654054666621    0.02654054666621    0.7752562655553  0.02654054666621    276.7551091027
0.01427394633699    0.01427394633699    0.01427394633699    0.4912526709581  0.01427394633699    607.4391713228
0.001218580492651   0.001218580492601   0.001218580492601   0.9248974186534  0.001218580492676   1166.289755796
1.888403761715e-05  1.888403761622e-05  1.888403761624e-05  0.985551099478   1.888403761735e-05  1241.740132447
1.545745837373e-09  1.545745372552e-09  1.545745362755e-09  0.999922927438   1.545745648845e-09  1242.779847889
1.692223225044e-10  1.873153075245e-10  1.873152955153e-10  0.8801597673128  1.817506825368e-10  1242.779937345
Optimization terminated successfully.
         Current function value: 1242.779937
         Iterations: 7
Primal Feasibility  Dual Feasibility    Duality Gap         Step             Path Parameter      Objective
1.0                 1.0                 1.0                 -                1.0                 652.1069903706
0.06097571429344    0.06097571429346    0.06097571429347    0.9444902662603  0.06097571429346    285.2043200632
0.01533833555931    0.01533833555931    0.01533833555932    0.7609599230413  0.01533833555932    2.7262007615
0.006316196361705   0.006316196361706   0.006316196361708   0.6163829523723  0.006316196361707   8.634999065661
0.001219580332373   0.001219580332374   0.001219580332374   0.8323480537452  0.001219580332374   8.307794697355
0.0005645791470893  0.0005645791470901  0.0005645791470902  0.5500511910892  0.0005645791470901  6.632034269378
0.0001849432525583  0.0001849432525588  0.0001849432525588  0.7137620753374  0.0001849432525588  4.483601470858
2.517759634035e-05  2.517759634041e-05  2.517759634042e-05  0.9034269505801  2.517759634041e-05  4.138476382855
3.098182398421e-08  3.098182439526e-08  3.09818244027e-08   0.9988322072353  3.098182439405e-08  4.000023749419
1.548986338477e-12  1.549101875116e-12  1.549109003245e-12  0.9999499994251  1.549106268784e-12  3.999950652469
Optimization terminated successfully.
         Current function value: 3.999951
         Iterations: 9
Primal Feasibility  Dual Feasibility    Duality Gap         Step             Path Parameter      Objective
1.0                 1.0                 1.0                 -                1.0                 24.59260538017
0.02944420803734    0.02944420803734    0.02944420803733    0.9778743187132  0.02944420803734    27.12355968787
0.008204037532537   0.008204037532537   0.008204037532536   0.7626654784836  0.008204037532537   110.4823556367
0.0004115277845134  0.0004115277844911  0.000411527784491   0.9587152757388  0.0004115277845217  181.7019444318
2.842168973263e-08  2.842168968894e-08  2.84216896431e-08   0.9999314505927  2.84216896475e-08   184.4443098122
1.421064808692e-12  1.421032025856e-12  1.420975506573e-12  0.999950002033   1.421084481186e-12  184.444539173
Optimization terminated successfully.
         Current function value: 184.444539
         Iterations: 5
\end{sphinxVerbatim}
}

{

\kern-\sphinxverbatimsmallskipamount\kern-\baselineskip
\kern+\FrameHeightAdjust\kern-\fboxrule
\vspace{\nbsphinxcodecellspacing}

\sphinxsetup{VerbatimColor={named}{nbsphinx-stderr}}
\sphinxsetup{VerbatimBorderColor={named}{nbsphinx-code-border}}
\begin{sphinxVerbatim}[commandchars=\\\{\}]
-- 2020-11-09 11:20:33 - muse.mca - WARNING
Check growth constraints for wind.

\end{sphinxVerbatim}
}

{

\kern-\sphinxverbatimsmallskipamount\kern-\baselineskip
\kern+\FrameHeightAdjust\kern-\fboxrule
\vspace{\nbsphinxcodecellspacing}

\sphinxsetup{VerbatimColor={named}{white}}
\sphinxsetup{VerbatimBorderColor={named}{nbsphinx-code-border}}
\begin{sphinxVerbatim}[commandchars=\\\{\}]
Primal Feasibility  Dual Feasibility    Duality Gap         Step             Path Parameter      Objective
1.0                 1.0                 1.0                 -                1.0                 454.6784294315
0.1862232334687     0.1862232334687     0.1862232334687     0.8174567476576  0.1862232334687     282.4632604722
0.02870747975048    0.02870747975048    0.02870747975048    0.8875534078522  0.02870747975048    46.09791383177
0.009705128410004   0.009705128410004   0.009705128410004   0.6723641007507  0.009705128410004   35.13402203751
0.007693915131578   0.007693915131578   0.007693915131578   0.2197837130033  0.007693915131578   80.93881539206
0.001699044794839   0.001699044794837   0.001699044794837   0.8250841182677  0.00169904479484    298.2653882331
0.0005650008980062  0.0005650008980057  0.0005650008980057  0.678565948553   0.0005650008980065  327.975809321
0.0002196033953225  0.0002196033953223  0.0002196033953223  0.6469207178836  0.0002196033953226  358.4708363196
2.579861478512e-06  2.579861478501e-06  2.579861478502e-06  0.9887803221269  2.579861478497e-06  368.5514208652
1.310502062448e-10  1.310502049574e-10  1.31050204896e-10   0.9999492039366  1.310502128569e-10  368.6632645094
Optimization terminated successfully.
         Current function value: 368.663265
         Iterations: 9
Primal Feasibility  Dual Feasibility    Duality Gap         Step             Path Parameter      Objective
1.0                 1.0                 1.0                 -                1.0                 454.6784294315
0.06717002148527    0.06717002148527    0.06717002148527    0.939336263039   0.06717002148527    183.2225626364
0.0335506482529     0.0335506482529     0.0335506482529     0.5230681149682  0.0335506482529     256.2098343218
0.01040420231972    0.01040420231972    0.01040420231972    0.6975119886622  0.01040420231972    178.3301859411
0.004349337378734   0.004349337378734   0.004349337378734   0.6235886580387  0.004349337378734   396.8309868051
0.002928936786464   0.002928936786463   0.002928936786463   0.3467952576409  0.002928936786463   488.976382689
0.0001458208281248  0.0001458208281499  0.0001458208281499  0.9601001354313  0.0001458208281409  665.3317152907
1.593588680453e-07  1.593588684743e-07  1.593588684755e-07  0.9989479829423  1.593588661973e-07  675.8691141106
7.991316370683e-12  7.990632294946e-12  7.990631661844e-12  0.999949860274   7.990813072693e-12  675.8826828943
Optimization terminated successfully.
         Current function value: 675.882683
         Iterations: 8
Primal Feasibility  Dual Feasibility    Duality Gap         Step             Path Parameter      Objective
1.0                 1.0                 1.0                 -                1.0                 759.5666413636
0.06104093664241    0.06104093664241    0.0610409366424     0.9444854714867  0.06104093664241    386.9519025016
0.01481511454072    0.01481511454072    0.01481511454072    0.7670183371838  0.01481511454072    2.643589195991
0.005238876768342   0.005238876768342   0.005238876768341   0.6766638370173  0.005238876768342   11.29316821413
0.001567745187159   0.001567745187159   0.001567745187159   0.7265977484523  0.001567745187159   12.32846209581
0.0001184080214619  0.0001184080214619  0.0001184080214619  0.9328477549533  0.0001184080214619  6.401954889258
1.02649122158e-05   1.026491221577e-05  1.026491221577e-05  0.959846524846   1.026491221577e-05  6.140180339771
2.564197796788e-08  2.564197949727e-08  2.564197949173e-08  0.9988486248561  2.56419794805e-08   6.06660177389
1.282243371032e-12  1.282166817908e-12  1.282163560611e-12  0.9999499974762  1.282163552715e-12  6.066591804033
Optimization terminated successfully.
         Current function value: 6.066592
         Iterations: 8
\end{sphinxVerbatim}
}

{

\kern-\sphinxverbatimsmallskipamount\kern-\baselineskip
\kern+\FrameHeightAdjust\kern-\fboxrule
\vspace{\nbsphinxcodecellspacing}

\sphinxsetup{VerbatimColor={named}{nbsphinx-stderr}}
\sphinxsetup{VerbatimBorderColor={named}{nbsphinx-code-border}}
\begin{sphinxVerbatim}[commandchars=\\\{\}]
-- 2020-11-09 11:20:40 - muse.mca - WARNING
Check growth constraints for wind.

\end{sphinxVerbatim}
}

{

\kern-\sphinxverbatimsmallskipamount\kern-\baselineskip
\kern+\FrameHeightAdjust\kern-\fboxrule
\vspace{\nbsphinxcodecellspacing}

\sphinxsetup{VerbatimColor={named}{white}}
\sphinxsetup{VerbatimBorderColor={named}{nbsphinx-code-border}}
\begin{sphinxVerbatim}[commandchars=\\\{\}]
Primal Feasibility  Dual Feasibility    Duality Gap         Step             Path Parameter      Objective
1.0                 1.0                 1.0                 -                1.0                 454.6784294315
0.1862232334687     0.1862232334687     0.1862232334687     0.8174567476576  0.1862232334687     282.4632604722
0.02870747975048    0.02870747975048    0.02870747975048    0.8875534078522  0.02870747975048    46.09791383177
0.009705128410004   0.009705128410004   0.009705128410004   0.6723641007507  0.009705128410004   35.13402203751
0.007693915131578   0.007693915131578   0.007693915131578   0.2197837130033  0.007693915131578   80.93881539206
0.001699044794839   0.001699044794837   0.001699044794837   0.8250841182677  0.00169904479484    298.2653882331
0.0005650008980062  0.0005650008980057  0.0005650008980057  0.678565948553   0.0005650008980065  327.975809321
0.0002196033953225  0.0002196033953223  0.0002196033953223  0.6469207178836  0.0002196033953226  358.4708363196
2.579861478512e-06  2.579861478501e-06  2.579861478502e-06  0.9887803221269  2.579861478497e-06  368.5514208652
1.310502062448e-10  1.310502049574e-10  1.31050204896e-10   0.9999492039366  1.310502128569e-10  368.6632645094
Optimization terminated successfully.
         Current function value: 368.663265
         Iterations: 9
Primal Feasibility  Dual Feasibility    Duality Gap         Step             Path Parameter      Objective
1.0                 1.0                 1.0                 -                1.0                 454.6784294315
0.06717002148527    0.06717002148527    0.06717002148527    0.939336263039   0.06717002148527    183.2225626364
0.0335506482529     0.0335506482529     0.0335506482529     0.5230681149682  0.0335506482529     256.2098343218
0.01040420231972    0.01040420231972    0.01040420231972    0.6975119886622  0.01040420231972    178.3301859411
0.004349337378734   0.004349337378734   0.004349337378734   0.6235886580387  0.004349337378734   396.8309868051
0.002928936786464   0.002928936786463   0.002928936786463   0.3467952576409  0.002928936786463   488.976382689
0.0001458208281248  0.0001458208281499  0.0001458208281499  0.9601001354313  0.0001458208281409  665.3317152907
1.593588680453e-07  1.593588684743e-07  1.593588684755e-07  0.9989479829423  1.593588661973e-07  675.8691141106
7.991316370683e-12  7.990632294946e-12  7.990631661844e-12  0.999949860274   7.990813072693e-12  675.8826828943
Optimization terminated successfully.
         Current function value: 675.882683
         Iterations: 8
Primal Feasibility  Dual Feasibility    Duality Gap         Step             Path Parameter      Objective
1.0                 1.0                 1.0                 -                1.0                 759.5666413636
0.06104093664241    0.06104093664241    0.0610409366424     0.9444854714867  0.06104093664241    386.9519025016
0.01481511454072    0.01481511454072    0.01481511454072    0.7670183371838  0.01481511454072    2.643589195991
0.005238876768342   0.005238876768342   0.005238876768341   0.6766638370173  0.005238876768342   11.29316821413
0.001567745187159   0.001567745187159   0.001567745187159   0.7265977484523  0.001567745187159   12.32846209581
0.0001184080214619  0.0001184080214619  0.0001184080214619  0.9328477549533  0.0001184080214619  6.401954889258
1.02649122158e-05   1.026491221577e-05  1.026491221577e-05  0.959846524846   1.026491221577e-05  6.140180339771
2.564197796788e-08  2.564197949727e-08  2.564197949173e-08  0.9988486248561  2.56419794805e-08   6.06660177389
1.282243371032e-12  1.282166817908e-12  1.282163560611e-12  0.9999499974762  1.282163552715e-12  6.066591804033
Optimization terminated successfully.
         Current function value: 6.066592
         Iterations: 8
\end{sphinxVerbatim}
}

{

\kern-\sphinxverbatimsmallskipamount\kern-\baselineskip
\kern+\FrameHeightAdjust\kern-\fboxrule
\vspace{\nbsphinxcodecellspacing}

\sphinxsetup{VerbatimColor={named}{nbsphinx-stderr}}
\sphinxsetup{VerbatimBorderColor={named}{nbsphinx-code-border}}
\begin{sphinxVerbatim}[commandchars=\\\{\}]
-- 2020-11-09 11:20:46 - muse.mca - WARNING
Check growth constraints for wind.

\end{sphinxVerbatim}
}

{

\kern-\sphinxverbatimsmallskipamount\kern-\baselineskip
\kern+\FrameHeightAdjust\kern-\fboxrule
\vspace{\nbsphinxcodecellspacing}

\sphinxsetup{VerbatimColor={named}{white}}
\sphinxsetup{VerbatimBorderColor={named}{nbsphinx-code-border}}
\begin{sphinxVerbatim}[commandchars=\\\{\}]
Primal Feasibility  Dual Feasibility    Duality Gap         Step             Path Parameter      Objective
1.0                 1.0                 1.0                 -                1.0                 506.4464461439
0.1717574295871     0.1717574295871     0.1717574295871     0.8319963473544  0.1717574295871     268.7406800963
0.06638577511032    0.06638577511032    0.06638577511033    0.6490768395246  0.06638577511033    376.5617800925
0.0108119991109     0.01081199911089    0.01081199911089    0.8457086649906  0.01081199911089    156.44831956
0.006166341998836   0.006166341998834   0.006166341998835   0.4575243211097  0.006166341998835   243.5697308399
0.0005157722017999  0.0005157722017955  0.0005157722017955  0.9303097249327  0.0005157722018036  353.933860755
7.752032087913e-08  7.75203208986e-08   7.752032089586e-08  0.9998584907938  7.75203208902e-08   354.2607816225
1.875610793654e-11  1.875609893145e-11  1.875610020548e-11  0.9997580492617  1.875613913502e-11  354.2617798007
Optimization terminated successfully.
         Current function value: 354.261780
         Iterations: 7
Primal Feasibility  Dual Feasibility    Duality Gap         Step             Path Parameter      Objective
1.0                 1.0                 1.0                 -                1.0                 506.4464461439
0.1079826576971     0.1079826576971     0.1079826576971     0.8981297430914  0.1079826576971     201.9066839097
0.04065945390789    0.04065945390789    0.04065945390789    0.6593890232973  0.04065945390789    434.9407529189
0.01889246737726    0.01889246737727    0.01889246737727    0.5585864512458  0.01889246737727    712.8967272059
0.001989080291767   0.001989080291864   0.001989080291864   0.9101975479096  0.001989080291929   1386.629312789
2.319893722758e-06  2.31989372302e-06   2.319893722997e-06  0.9988740155508  2.319893723235e-06  1475.934762389
2.096073130738e-10  2.096071489352e-10  2.096071783066e-10  0.9999096479509  2.096074566229e-10  1476.090721225
Optimization terminated successfully.
         Current function value: 1476.090721
         Iterations: 6
Primal Feasibility  Dual Feasibility    Duality Gap         Step             Path Parameter      Objective
1.0                 1.0                 1.0                 -                1.0                 889.6273141593
0.0609484122893     0.06094841228931    0.0609484122893     0.9446306809213  0.06094841228931    507.5981904767
0.01403712586721    0.01403712586721    0.01403712586721    0.7774983036897  0.01403712586721    2.560790425808
0.004916790985586   0.004916790985586   0.004916790985586   0.6812990636392  0.004916790985587   14.46599029197
0.001087493288514   0.001087493288517   0.001087493288517   0.8126332143986  0.001087493288517   17.14374653765
0.0004935333934202  0.0004935333934214  0.0004935333934214  0.5558253633026  0.0004935333934214  10.54120251124
0.0001121697661281  0.000112169766129   0.000112169766129   0.817643306912   0.000112169766129   4.9896729167
2.712089127881e-05  2.712089127903e-05  2.712089127903e-05  0.7970501144483  2.712089127903e-05  4.324997088558
1.122718419375e-07  1.122718402016e-07  1.122718401965e-07  0.99597588406    1.122718401941e-07  4.000800850778
5.614947446286e-12  5.615029242581e-12  5.615032519692e-12  0.9999499875898  5.615033454619e-12  3.999950693022
2.903000170299e-16  2.886166226175e-16  2.808194717321e-16  0.9999499790221  2.807865903642e-16  3.999950650496
Optimization terminated successfully.
         Current function value: 3.999951
         Iterations: 10
Primal Feasibility  Dual Feasibility    Duality Gap         Step             Path Parameter      Objective
1.0                 1.0                 1.0                 -                1.0                 889.6273141593
0.06073185177466    0.06073185177467    0.06073185177466    0.9450179496586  0.06073185177466    760.7133777158
0.00617243956643    0.006172439566431   0.006172439566431   0.8994093979063  0.006172439566431   1.525116036565
0.002290102597025   0.002290102597025   0.002290102597025   0.6678934240013  0.002290102597025   13.23877514086
0.0007463324176618  0.0007463324176611  0.0007463324176611  0.7077017235652  0.0007463324176608  10.86935158651
0.0003923306880488  0.0003923306880484  0.0003923306880484  0.488114542318   0.0003923306880483  6.236673613644
2.432640260798e-05  2.432640260791e-05  2.432640260791e-05  1.0              2.43264026079e-05   0.3563903987643
6.281868977076e-08  6.281868977357e-08  6.281868977105e-08  0.9991491899851  6.281868977102e-08  0.0001119022044274
3.378349078595e-12  3.378347740195e-12  3.378348155213e-12  0.9999462206524  3.378348155213e-12  6.056769044982e-09
2.306615405556e-13  2.306653965292e-13  2.306614266427e-13  0.9320769632577  2.306614266427e-13  4.595744934239e-10
Optimization terminated successfully.
         Current function value: 0.000000
         Iterations: 9
\end{sphinxVerbatim}
}

{

\kern-\sphinxverbatimsmallskipamount\kern-\baselineskip
\kern+\FrameHeightAdjust\kern-\fboxrule
\vspace{\nbsphinxcodecellspacing}

\sphinxsetup{VerbatimColor={named}{nbsphinx-stderr}}
\sphinxsetup{VerbatimBorderColor={named}{nbsphinx-code-border}}
\begin{sphinxVerbatim}[commandchars=\\\{\}]
-- 2020-11-09 11:20:54 - muse.mca - WARNING
Check growth constraints for wind.

\end{sphinxVerbatim}
}

{

\kern-\sphinxverbatimsmallskipamount\kern-\baselineskip
\kern+\FrameHeightAdjust\kern-\fboxrule
\vspace{\nbsphinxcodecellspacing}

\sphinxsetup{VerbatimColor={named}{white}}
\sphinxsetup{VerbatimBorderColor={named}{nbsphinx-code-border}}
\begin{sphinxVerbatim}[commandchars=\\\{\}]
Primal Feasibility  Dual Feasibility    Duality Gap         Step             Path Parameter      Objective
1.0                 1.0                 1.0                 -                1.0                 506.4464461439
0.1717574295871     0.1717574295871     0.1717574295871     0.8319963473544  0.1717574295871     268.7406800963
0.06638577511032    0.06638577511032    0.06638577511033    0.6490768395246  0.06638577511033    376.5617800925
0.0108119991109     0.01081199911089    0.01081199911089    0.8457086649906  0.01081199911089    156.44831956
0.006166341998836   0.006166341998834   0.006166341998835   0.4575243211097  0.006166341998835   243.5697308399
0.0005157722017999  0.0005157722017955  0.0005157722017955  0.9303097249327  0.0005157722018036  353.933860755
7.752032087913e-08  7.75203208986e-08   7.752032089586e-08  0.9998584907938  7.75203208902e-08   354.2607816225
1.875610793654e-11  1.875609893145e-11  1.875610020548e-11  0.9997580492617  1.875613913502e-11  354.2617798007
Optimization terminated successfully.
         Current function value: 354.261780
         Iterations: 7
Primal Feasibility  Dual Feasibility    Duality Gap         Step             Path Parameter      Objective
1.0                 1.0                 1.0                 -                1.0                 506.4464461439
0.1079826576971     0.1079826576971     0.1079826576971     0.8981297430914  0.1079826576971     201.9066839097
0.04065945390789    0.04065945390789    0.04065945390789    0.6593890232973  0.04065945390789    434.9407529189
0.01889246737726    0.01889246737727    0.01889246737727    0.5585864512458  0.01889246737727    712.8967272059
0.001989080291767   0.001989080291864   0.001989080291864   0.9101975479096  0.001989080291929   1386.629312789
2.319893722758e-06  2.31989372302e-06   2.319893722997e-06  0.9988740155508  2.319893723235e-06  1475.934762389
2.096073130738e-10  2.096071489352e-10  2.096071783066e-10  0.9999096479509  2.096074566229e-10  1476.090721225
Optimization terminated successfully.
         Current function value: 1476.090721
         Iterations: 6
Primal Feasibility  Dual Feasibility    Duality Gap         Step             Path Parameter      Objective
1.0                 1.0                 1.0                 -                1.0                 889.6273141593
0.0609484122893     0.06094841228931    0.0609484122893     0.9446306809213  0.06094841228931    507.5981904767
0.01403712586721    0.01403712586721    0.01403712586721    0.7774983036897  0.01403712586721    2.560790425808
0.004916790985586   0.004916790985586   0.004916790985586   0.6812990636392  0.004916790985587   14.46599029197
0.001087493288514   0.001087493288517   0.001087493288517   0.8126332143986  0.001087493288517   17.14374653765
0.0004935333934202  0.0004935333934214  0.0004935333934214  0.5558253633026  0.0004935333934214  10.54120251124
0.0001121697661281  0.000112169766129   0.000112169766129   0.817643306912   0.000112169766129   4.9896729167
2.712089127881e-05  2.712089127903e-05  2.712089127903e-05  0.7970501144483  2.712089127903e-05  4.324997088558
1.122718419375e-07  1.122718402016e-07  1.122718401965e-07  0.99597588406    1.122718401941e-07  4.000800850778
5.614947446286e-12  5.615029242581e-12  5.615032519692e-12  0.9999499875898  5.615033454619e-12  3.999950693022
2.903000170299e-16  2.886166226175e-16  2.808194717321e-16  0.9999499790221  2.807865903642e-16  3.999950650496
Optimization terminated successfully.
         Current function value: 3.999951
         Iterations: 10
Primal Feasibility  Dual Feasibility    Duality Gap         Step             Path Parameter      Objective
1.0                 1.0                 1.0                 -                1.0                 889.6273141593
0.06073185177466    0.06073185177467    0.06073185177466    0.9450179496586  0.06073185177466    760.7133777158
0.00617243956643    0.006172439566431   0.006172439566431   0.8994093979063  0.006172439566431   1.525116036565
0.002290102597025   0.002290102597025   0.002290102597025   0.6678934240013  0.002290102597025   13.23877514086
0.0007463324176618  0.0007463324176611  0.0007463324176611  0.7077017235652  0.0007463324176608  10.86935158651
0.0003923306880488  0.0003923306880484  0.0003923306880484  0.488114542318   0.0003923306880483  6.236673613644
2.432640260798e-05  2.432640260791e-05  2.432640260791e-05  1.0              2.43264026079e-05   0.3563903987643
6.281868977076e-08  6.281868977357e-08  6.281868977105e-08  0.9991491899851  6.281868977102e-08  0.0001119022044274
3.378349078595e-12  3.378347740195e-12  3.378348155213e-12  0.9999462206524  3.378348155213e-12  6.056769044982e-09
2.306615405556e-13  2.306653965292e-13  2.306614266427e-13  0.9320769632577  2.306614266427e-13  4.595744934239e-10
Optimization terminated successfully.
         Current function value: 0.000000
         Iterations: 9
\end{sphinxVerbatim}
}

{

\kern-\sphinxverbatimsmallskipamount\kern-\baselineskip
\kern+\FrameHeightAdjust\kern-\fboxrule
\vspace{\nbsphinxcodecellspacing}

\sphinxsetup{VerbatimColor={named}{nbsphinx-stderr}}
\sphinxsetup{VerbatimBorderColor={named}{nbsphinx-code-border}}
\begin{sphinxVerbatim}[commandchars=\\\{\}]
-- 2020-11-09 11:21:02 - muse.mca - WARNING
Check growth constraints for wind.

\end{sphinxVerbatim}
}

{

\kern-\sphinxverbatimsmallskipamount\kern-\baselineskip
\kern+\FrameHeightAdjust\kern-\fboxrule
\vspace{\nbsphinxcodecellspacing}

\sphinxsetup{VerbatimColor={named}{white}}
\sphinxsetup{VerbatimBorderColor={named}{nbsphinx-code-border}}
\begin{sphinxVerbatim}[commandchars=\\\{\}]
Primal Feasibility  Dual Feasibility    Duality Gap         Step             Path Parameter      Objective
1.0                 1.0                 1.0                 -                1.0                 568.8949309456
0.1574999940443     0.1574999940443     0.1574999940443     0.8462995490748  0.1574999940443     259.3692323549
0.07523425397025    0.07523425397025    0.07523425397025    0.5547720211262  0.07523425397025    453.5904211877
0.01015218722862    0.01015218722862    0.01015218722862    0.8831324835902  0.01015218722862    172.5109988087
0.005597873343747   0.005597873343747   0.005597873343747   0.4782155730769  0.005597873343747   257.1812475882
0.0004682435100659  0.0004682435100634  0.0004682435100634  0.9301215087302  0.000468243510069   357.0084554784
6.611502592752e-08  6.611502591615e-08  6.611502591873e-08  0.999863385888   6.611502594636e-08  355.7877874388
7.555502313764e-12  7.555508723405e-12  7.555505999397e-12  0.9998857217433  7.555493074812e-12  355.7884668282
Optimization terminated successfully.
         Current function value: 355.788467
         Iterations: 7
Primal Feasibility  Dual Feasibility    Duality Gap         Step             Path Parameter      Objective
1.0                 1.0                 1.0                 -                1.0                 568.8949309456
0.07655552550188    0.07655552550188    0.07655552550188    0.9326533055761  0.07655552550188    158.9571104913
0.02915610316818    0.02915610316819    0.02915610316819    0.6583861263339  0.02915610316818    366.3114833866
0.01377238842158    0.01377238842158    0.01377238842158    0.5494198953479  0.01377238842158    509.1861831748
0.002113311162461   0.002113311162461   0.002113311162461   0.8672288939288  0.002113311162461   954.4881944043
1.802801647426e-06  1.802801647524e-06  1.802801647519e-06  0.9992160883437  1.802801646136e-06  1007.960176974
1.128093610417e-10  1.128094830078e-10  1.128094936008e-10  0.9999374254592  1.128097428918e-10  1008.067349764
Optimization terminated successfully.
         Current function value: 1008.067350
         Iterations: 6
Primal Feasibility  Dual Feasibility    Duality Gap         Step             Path Parameter      Objective
1.0                 1.0                 1.0                 -                1.0                 1027.252465794
0.06092848901518    0.06092848901523    0.06092848901522    0.9446907062507  0.06092848901524    639.5911028223
0.01301865096831    0.01301865096832    0.01301865096832    0.7926600108064  0.01301865096833    2.590461755952
0.00456587722928    0.004565877229283   0.004565877229283   0.6815721939632  0.004565877229284   18.42445056578
0.001192620177961   0.001192620177954   0.001192620177954   0.7720888734274  0.001192620177954   21.2766378272
0.0004150762782539  0.0004150762782515  0.0004150762782515  0.6551231023534  0.0004150762782516  10.90714854956
8.30397207983e-05   8.303972079784e-05  8.303972079783e-05  0.8551088375548  8.303972079786e-05  4.883155784925
2.345269716473e-05  2.345269716461e-05  2.34526971646e-05   0.7548298132573  2.345269716461e-05  4.346885094868
1.371159312911e-07  1.371159321168e-07  1.371159321194e-07  0.994331926271   1.371159321123e-07  4.001183377585
6.859845575537e-12  6.859856824223e-12  6.859863126594e-12  0.9999499723157  6.85986409236e-12   3.999950686436
6.215051942361e-16  3.43521130371e-16   3.430265101783e-16  0.9999499985889  3.431205195905e-16  3.999950624744
Optimization terminated successfully.
         Current function value: 3.999951
         Iterations: 10
Primal Feasibility  Dual Feasibility    Duality Gap         Step             Path Parameter      Objective
1.0                 1.0                 1.0                 -                1.0                 513.6262328969
0.06112045297154    0.06112045297154    0.06112045297151    0.9426671718798  0.06112045297154    286.961037666
0.0186024953335     0.0186024953335     0.01860249533349    0.7059824756003  0.0186024953335     1.911789377935
0.006446621823764   0.006446621823763   0.00644662182376    0.6910883581126  0.006446621823763   10.97980543166
0.0005501537566849  0.0005501537566955  0.0005501537566952  0.9377765193638  0.0005501537566949  10.00501917633
4.007383875704e-06  4.007383875795e-06  4.007383875804e-06  0.9951127586026  4.007383875797e-06  6.687770636962
2.007342726257e-10  2.007343535262e-10  2.007343506552e-10  0.9999499088827  2.007343503785e-10  6.666585422706
2.161448037787e-10  2.007343535257e-10  2.007343507631e-10  2.456113364069e-461.961302857585e-10  6.666585422564
6.009588090216e-12  2.723883580422e-12  2.723885134159e-12  0.9866729468426  2.685910112544e-12  6.666584374146
Optimization terminated successfully.
         Current function value: 6.666584
         Iterations: 8
\end{sphinxVerbatim}
}

{

\kern-\sphinxverbatimsmallskipamount\kern-\baselineskip
\kern+\FrameHeightAdjust\kern-\fboxrule
\vspace{\nbsphinxcodecellspacing}

\sphinxsetup{VerbatimColor={named}{nbsphinx-stderr}}
\sphinxsetup{VerbatimBorderColor={named}{nbsphinx-code-border}}
\begin{sphinxVerbatim}[commandchars=\\\{\}]
/Users/alexkell/Documents/SGI/1-examples/example\_model/model/Results/muse/src/muse/investments.py:325: OptimizeWarning: Solving system with option 'cholesky':True failed. It is normal for this to happen occasionally, especially as the solution is approached. However, if you see this frequently, consider setting option 'cholesky' to False.
  res = linprog(**adapter.kwargs, options=dict(disp=True))
/Users/alexkell/Documents/SGI/1-examples/example\_model/model/Results/muse/src/muse/investments.py:325: OptimizeWarning: Solving system with option 'sym\_pos':True failed. It is normal for this to happen occasionally, especially as the solution is approached. However, if you see this frequently, consider setting option 'sym\_pos' to False.
  res = linprog(**adapter.kwargs, options=dict(disp=True))
/Users/alexkell/anaconda3/lib/python3.8/site-packages/scipy/optimize/\_linprog\_ip.py:116: LinAlgWarning: Ill-conditioned matrix (rcond=1.62544e-36): result may not be accurate.
  return sp.linalg.solve(M, r, sym\_pos=sym\_pos)
-- 2020-11-09 11:21:10 - muse.mca - WARNING
Check growth constraints for wind.

\end{sphinxVerbatim}
}

{

\kern-\sphinxverbatimsmallskipamount\kern-\baselineskip
\kern+\FrameHeightAdjust\kern-\fboxrule
\vspace{\nbsphinxcodecellspacing}

\sphinxsetup{VerbatimColor={named}{white}}
\sphinxsetup{VerbatimBorderColor={named}{nbsphinx-code-border}}
\begin{sphinxVerbatim}[commandchars=\\\{\}]
Primal Feasibility  Dual Feasibility    Duality Gap         Step             Path Parameter      Objective
1.0                 1.0                 1.0                 -                1.0                 568.8949309456
0.1574999940443     0.1574999940443     0.1574999940443     0.8462995490748  0.1574999940443     259.3692323549
0.07523425397025    0.07523425397025    0.07523425397025    0.5547720211262  0.07523425397025    453.5904211877
0.01015218722862    0.01015218722862    0.01015218722862    0.8831324835902  0.01015218722862    172.5109988087
0.005597873343747   0.005597873343747   0.005597873343747   0.4782155730769  0.005597873343747   257.1812475882
0.0004682435100659  0.0004682435100634  0.0004682435100634  0.9301215087302  0.000468243510069   357.0084554784
6.611502592752e-08  6.611502591615e-08  6.611502591873e-08  0.999863385888   6.611502594636e-08  355.7877874388
7.555502313764e-12  7.555508723405e-12  7.555505999397e-12  0.9998857217433  7.555493074812e-12  355.7884668282
Optimization terminated successfully.
         Current function value: 355.788467
         Iterations: 7
Primal Feasibility  Dual Feasibility    Duality Gap         Step             Path Parameter      Objective
1.0                 1.0                 1.0                 -                1.0                 568.8949309456
0.07655552550188    0.07655552550188    0.07655552550188    0.9326533055761  0.07655552550188    158.9571104913
0.02915610316818    0.02915610316819    0.02915610316819    0.6583861263339  0.02915610316818    366.3114833866
0.01377238842158    0.01377238842158    0.01377238842158    0.5494198953479  0.01377238842158    509.1861831748
0.002113311162461   0.002113311162461   0.002113311162461   0.8672288939288  0.002113311162461   954.4881944043
1.802801647426e-06  1.802801647524e-06  1.802801647519e-06  0.9992160883437  1.802801646136e-06  1007.960176974
1.128093610417e-10  1.128094830078e-10  1.128094936008e-10  0.9999374254592  1.128097428918e-10  1008.067349764
Optimization terminated successfully.
         Current function value: 1008.067350
         Iterations: 6
Primal Feasibility  Dual Feasibility    Duality Gap         Step             Path Parameter      Objective
1.0                 1.0                 1.0                 -                1.0                 1027.252465794
0.06092848901518    0.06092848901523    0.06092848901522    0.9446907062507  0.06092848901524    639.5911028223
0.01301865096831    0.01301865096832    0.01301865096832    0.7926600108064  0.01301865096833    2.590461755952
0.00456587722928    0.004565877229283   0.004565877229283   0.6815721939632  0.004565877229284   18.42445056578
0.001192620177961   0.001192620177954   0.001192620177954   0.7720888734274  0.001192620177954   21.2766378272
0.0004150762782539  0.0004150762782515  0.0004150762782515  0.6551231023534  0.0004150762782516  10.90714854956
8.30397207983e-05   8.303972079784e-05  8.303972079783e-05  0.8551088375548  8.303972079786e-05  4.883155784925
2.345269716473e-05  2.345269716461e-05  2.34526971646e-05   0.7548298132573  2.345269716461e-05  4.346885094868
1.371159312911e-07  1.371159321168e-07  1.371159321194e-07  0.994331926271   1.371159321123e-07  4.001183377585
6.859845575537e-12  6.859856824223e-12  6.859863126594e-12  0.9999499723157  6.85986409236e-12   3.999950686436
6.215051942361e-16  3.43521130371e-16   3.430265101783e-16  0.9999499985889  3.431205195905e-16  3.999950624744
Optimization terminated successfully.
         Current function value: 3.999951
         Iterations: 10
Primal Feasibility  Dual Feasibility    Duality Gap         Step             Path Parameter      Objective
1.0                 1.0                 1.0                 -                1.0                 513.6262328969
0.06112045297154    0.06112045297154    0.06112045297151    0.9426671718798  0.06112045297154    286.961037666
0.0186024953335     0.0186024953335     0.01860249533349    0.7059824756003  0.0186024953335     1.911789377935
0.006446621823764   0.006446621823763   0.00644662182376    0.6910883581126  0.006446621823763   10.97980543166
0.0005501537566849  0.0005501537566955  0.0005501537566952  0.9377765193638  0.0005501537566949  10.00501917633
4.007383875704e-06  4.007383875795e-06  4.007383875804e-06  0.9951127586026  4.007383875797e-06  6.687770636962
2.007342726257e-10  2.007343535262e-10  2.007343506552e-10  0.9999499088827  2.007343503785e-10  6.666585422706
2.161448037787e-10  2.007343535257e-10  2.007343507631e-10  2.456113364069e-461.961302857585e-10  6.666585422564
6.009588090216e-12  2.723883580422e-12  2.723885134159e-12  0.9866729468426  2.685910112544e-12  6.666584374146
Optimization terminated successfully.
         Current function value: 6.666584
         Iterations: 8
\end{sphinxVerbatim}
}

{

\kern-\sphinxverbatimsmallskipamount\kern-\baselineskip
\kern+\FrameHeightAdjust\kern-\fboxrule
\vspace{\nbsphinxcodecellspacing}

\sphinxsetup{VerbatimColor={named}{nbsphinx-stderr}}
\sphinxsetup{VerbatimBorderColor={named}{nbsphinx-code-border}}
\begin{sphinxVerbatim}[commandchars=\\\{\}]
/Users/alexkell/Documents/SGI/1-examples/example\_model/model/Results/muse/src/muse/investments.py:325: OptimizeWarning: Solving system with option 'cholesky':True failed. It is normal for this to happen occasionally, especially as the solution is approached. However, if you see this frequently, consider setting option 'cholesky' to False.
  res = linprog(**adapter.kwargs, options=dict(disp=True))
/Users/alexkell/Documents/SGI/1-examples/example\_model/model/Results/muse/src/muse/investments.py:325: OptimizeWarning: Solving system with option 'sym\_pos':True failed. It is normal for this to happen occasionally, especially as the solution is approached. However, if you see this frequently, consider setting option 'sym\_pos' to False.
  res = linprog(**adapter.kwargs, options=dict(disp=True))
/Users/alexkell/anaconda3/lib/python3.8/site-packages/scipy/optimize/\_linprog\_ip.py:116: LinAlgWarning: Ill-conditioned matrix (rcond=1.62544e-36): result may not be accurate.
  return sp.linalg.solve(M, r, sym\_pos=sym\_pos)
-- 2020-11-09 11:21:16 - muse.mca - WARNING
Check growth constraints for wind.

\end{sphinxVerbatim}
}

{

\kern-\sphinxverbatimsmallskipamount\kern-\baselineskip
\kern+\FrameHeightAdjust\kern-\fboxrule
\vspace{\nbsphinxcodecellspacing}

\sphinxsetup{VerbatimColor={named}{white}}
\sphinxsetup{VerbatimBorderColor={named}{nbsphinx-code-border}}
\begin{sphinxVerbatim}[commandchars=\\\{\}]
Primal Feasibility  Dual Feasibility    Duality Gap         Step             Path Parameter      Objective
1.0                 1.0                 1.0                 -                1.0                 647.4402248427
0.1042702416317     0.1042702416317     0.1042702416317     0.9044651903652  0.1042702416317     197.7436068274
0.04330783922975    0.04330783922975    0.04330783922975    0.6213338928986  0.04330783922975    554.8458695929
0.02029339907042    0.02029339907042    0.02029339907042    0.5546164251713  0.02029339907042    887.503418141
0.002059630009005   0.002059630008646   0.002059630008646   0.9123945707354  0.002059630008855   1773.497308596
1.96164830749e-06   1.961648306624e-06  1.961648306623e-06  0.9995475315428  1.96164830693e-06   1899.795955905
1.183904657193e-10  1.183907029537e-10  1.183907180056e-10  0.9999396473343  1.183903667831e-10  1900.055708261
Optimization terminated successfully.
         Current function value: 1900.055708
         Iterations: 6
Primal Feasibility  Dual Feasibility    Duality Gap         Step             Path Parameter      Objective
1.0                 1.0                 1.0                 -                1.0                 1199.542590081
0.06080970203684    0.06080970203684    0.06080970203685    0.9448463893022  0.06080970203684    802.3521760538
0.01206790508619    0.01206790508619    0.01206790508619    0.8067148881887  0.01206790508619    2.242531350203
0.004235304009507   0.004235304009507   0.004235304009507   0.6829892140296  0.004235304009507   22.93309435174
0.001076107298274   0.001076107298272   0.001076107298272   0.7811880414723  0.001076107298272   27.34472817154
0.0001381467026759  0.0001381467026756  0.0001381467026756  0.9534954297498  0.0001381467026756  2.164474733514
0.0001283068196922  0.0001283068196919  0.0001283068196919  0.07280278667804 0.0001283068196919  2.01175576967
1.709535377769e-06  1.709535377765e-06  1.709535377765e-06  0.9870423816225  1.709535377765e-06  0.04804313560572
9.48541628326e-11   9.485418648369e-11  9.485418526931e-11  0.9999445934679  9.485418526932e-11  2.66445492392e-06
8.795947527601e-12  8.795912717956e-12  8.795907275833e-12  0.9086823994024  8.795907275834e-12  2.473302892196e-07
8.361396439476e-12  8.361363544519e-12  8.361358288329e-12  0.05441434596209 8.361358288329e-12  2.360712463483e-07
1.014674559722e-12  1.014680661495e-12  1.014676648034e-12  0.8808930741304  1.014676648034e-12  2.943765656853e-08
7.649517191012e-13  7.649501663219e-13  7.649518441425e-13  0.266685424728   7.649518441426e-13  2.292855842523e-08
2.251056337644e-13  2.251034423868e-13  2.250976823208e-13  0.7199831364733  2.250976823208e-13  9.225506253453e-09
Optimization terminated successfully.
         Current function value: 0.000000
         Iterations: 13
Primal Feasibility  Dual Feasibility    Duality Gap         Step             Path Parameter      Objective
1.0                 1.0                 1.0                 -                1.0                 599.7712950406
0.06108522397884    0.06108522397884    0.06108522397884    0.9426921917133  0.06108522397884    370.5338700223
0.01718424163858    0.01718424163858    0.01718424163858    0.7269400839096  0.01718424163858    1.816590685562
0.006448030539302   0.006448030539302   0.006448030539302   0.6618055567806  0.006448030539302   13.33233954634
0.0007451641812888  0.0007451641813056  0.0007451641813056  0.9128367304104  0.0007451641813056  13.46011423965
0.0002051849835872  0.0002051849835913  0.0002051849835913  0.726248232922   0.0002051849835914  8.092503300126
3.421589486985e-06  3.421589487287e-06  3.421589487279e-06  1.0              3.421589487245e-06  5.943322696325
2.883371020626e-10  2.88337212738e-10   2.883372076584e-10  0.999916005146   2.883372071457e-10  5.933260704906
1.437342489421e-14  1.441626748189e-14  1.441700316141e-14  0.999949999407   1.441794244359e-14  5.933260075848
Optimization terminated successfully.
         Current function value: 5.933260
         Iterations: 8
\end{sphinxVerbatim}
}

{

\kern-\sphinxverbatimsmallskipamount\kern-\baselineskip
\kern+\FrameHeightAdjust\kern-\fboxrule
\vspace{\nbsphinxcodecellspacing}

\sphinxsetup{VerbatimColor={named}{nbsphinx-stderr}}
\sphinxsetup{VerbatimBorderColor={named}{nbsphinx-code-border}}
\begin{sphinxVerbatim}[commandchars=\\\{\}]
-- 2020-11-09 11:21:23 - muse.mca - WARNING
Check growth constraints for wind.

\end{sphinxVerbatim}
}

{

\kern-\sphinxverbatimsmallskipamount\kern-\baselineskip
\kern+\FrameHeightAdjust\kern-\fboxrule
\vspace{\nbsphinxcodecellspacing}

\sphinxsetup{VerbatimColor={named}{white}}
\sphinxsetup{VerbatimBorderColor={named}{nbsphinx-code-border}}
\begin{sphinxVerbatim}[commandchars=\\\{\}]
Primal Feasibility  Dual Feasibility    Duality Gap         Step             Path Parameter      Objective
1.0                 1.0                 1.0                 -                1.0                 647.4402248427
0.1042702416317     0.1042702416317     0.1042702416317     0.9044651903652  0.1042702416317     197.7436068274
0.04330783922975    0.04330783922975    0.04330783922975    0.6213338928986  0.04330783922975    554.8458695929
0.02029339907042    0.02029339907042    0.02029339907042    0.5546164251713  0.02029339907042    887.503418141
0.002059630009005   0.002059630008646   0.002059630008646   0.9123945707354  0.002059630008855   1773.497308596
1.96164830749e-06   1.961648306624e-06  1.961648306623e-06  0.9995475315428  1.96164830693e-06   1899.795955905
1.183904657193e-10  1.183907029537e-10  1.183907180056e-10  0.9999396473343  1.183903667831e-10  1900.055708261
Optimization terminated successfully.
         Current function value: 1900.055708
         Iterations: 6
Primal Feasibility  Dual Feasibility    Duality Gap         Step             Path Parameter      Objective
1.0                 1.0                 1.0                 -                1.0                 1199.542590081
0.06080970203684    0.06080970203684    0.06080970203685    0.9448463893022  0.06080970203684    802.3521760538
0.01206790508619    0.01206790508619    0.01206790508619    0.8067148881887  0.01206790508619    2.242531350203
0.004235304009507   0.004235304009507   0.004235304009507   0.6829892140296  0.004235304009507   22.93309435174
0.001076107298274   0.001076107298272   0.001076107298272   0.7811880414723  0.001076107298272   27.34472817154
0.0001381467026759  0.0001381467026756  0.0001381467026756  0.9534954297498  0.0001381467026756  2.164474733514
0.0001283068196922  0.0001283068196919  0.0001283068196919  0.07280278667804 0.0001283068196919  2.01175576967
1.709535377769e-06  1.709535377765e-06  1.709535377765e-06  0.9870423816225  1.709535377765e-06  0.04804313560572
9.48541628326e-11   9.485418648369e-11  9.485418526931e-11  0.9999445934679  9.485418526932e-11  2.66445492392e-06
8.795947527601e-12  8.795912717956e-12  8.795907275833e-12  0.9086823994024  8.795907275834e-12  2.473302892196e-07
8.361396439476e-12  8.361363544519e-12  8.361358288329e-12  0.05441434596209 8.361358288329e-12  2.360712463483e-07
1.014674559722e-12  1.014680661495e-12  1.014676648034e-12  0.8808930741304  1.014676648034e-12  2.943765656853e-08
7.649517191012e-13  7.649501663219e-13  7.649518441425e-13  0.266685424728   7.649518441426e-13  2.292855842523e-08
2.251056337644e-13  2.251034423868e-13  2.250976823208e-13  0.7199831364733  2.250976823208e-13  9.225506253453e-09
Optimization terminated successfully.
         Current function value: 0.000000
         Iterations: 13
Primal Feasibility  Dual Feasibility    Duality Gap         Step             Path Parameter      Objective
1.0                 1.0                 1.0                 -                1.0                 599.7712950406
0.06108522397884    0.06108522397884    0.06108522397884    0.9426921917133  0.06108522397884    370.5338700223
0.01718424163858    0.01718424163858    0.01718424163858    0.7269400839096  0.01718424163858    1.816590685562
0.006448030539302   0.006448030539302   0.006448030539302   0.6618055567806  0.006448030539302   13.33233954634
0.0007451641812888  0.0007451641813056  0.0007451641813056  0.9128367304104  0.0007451641813056  13.46011423965
0.0002051849835872  0.0002051849835913  0.0002051849835913  0.726248232922   0.0002051849835914  8.092503300126
3.421589486985e-06  3.421589487287e-06  3.421589487279e-06  1.0              3.421589487245e-06  5.943322696325
2.883371020626e-10  2.88337212738e-10   2.883372076584e-10  0.999916005146   2.883372071457e-10  5.933260704906
1.437342489421e-14  1.441626748189e-14  1.441700316141e-14  0.999949999407   1.441794244359e-14  5.933260075848
Optimization terminated successfully.
         Current function value: 5.933260
         Iterations: 8
\end{sphinxVerbatim}
}

{

\kern-\sphinxverbatimsmallskipamount\kern-\baselineskip
\kern+\FrameHeightAdjust\kern-\fboxrule
\vspace{\nbsphinxcodecellspacing}

\sphinxsetup{VerbatimColor={named}{nbsphinx-stderr}}
\sphinxsetup{VerbatimBorderColor={named}{nbsphinx-code-border}}
\begin{sphinxVerbatim}[commandchars=\\\{\}]
-- 2020-11-09 11:21:29 - muse.mca - WARNING
Check growth constraints for wind.

\end{sphinxVerbatim}
}

We can now check that the simulation has created the files that we expect. We also check that our “Hello, world!” message has printed:

{
\sphinxsetup{VerbatimColor={named}{nbsphinx-code-bg}}
\sphinxsetup{VerbatimBorderColor={named}{nbsphinx-code-border}}
\begin{sphinxVerbatim}[commandchars=\\\{\}]
\llap{\color{nbsphinxin}[5]:\,\hspace{\fboxrule}\hspace{\fboxsep}}\PYG{n}{all\PYGZus{}txt\PYGZus{}files} \PYG{o}{=} \PYG{n+nb}{sorted}\PYG{p}{(}\PYG{p}{(}\PYG{n}{Path}\PYG{p}{(}\PYG{p}{)} \PYG{o}{/} \PYG{l+s+s2}{\PYGZdq{}}\PYG{l+s+s2}{Results}\PYG{l+s+s2}{\PYGZdq{}}\PYG{p}{)}\PYG{o}{.}\PYG{n}{glob}\PYG{p}{(}\PYG{l+s+s2}{\PYGZdq{}}\PYG{l+s+s2}{Residential*.txt}\PYG{l+s+s2}{\PYGZdq{}}\PYG{p}{)}\PYG{p}{)}
\PYG{k}{assert} \PYG{l+s+s2}{\PYGZdq{}}\PYG{l+s+s2}{Hello world!}\PYG{l+s+s2}{\PYGZdq{}} \PYG{o+ow}{in} \PYG{n}{all\PYGZus{}txt\PYGZus{}files}\PYG{p}{[}\PYG{l+m+mi}{0}\PYG{p}{]}\PYG{o}{.}\PYG{n}{read\PYGZus{}text}\PYG{p}{(}\PYG{p}{)}
\PYG{n}{all\PYGZus{}txt\PYGZus{}files}
\end{sphinxVerbatim}
}

{

\kern-\sphinxverbatimsmallskipamount\kern-\baselineskip
\kern+\FrameHeightAdjust\kern-\fboxrule
\vspace{\nbsphinxcodecellspacing}

\sphinxsetup{VerbatimColor={named}{white}}
\sphinxsetup{VerbatimBorderColor={named}{nbsphinx-code-border}}
\begin{sphinxVerbatim}[commandchars=\\\{\}]
\llap{\color{nbsphinxout}[5]:\,\hspace{\fboxrule}\hspace{\fboxsep}}[PosixPath('Results/ResidentialConsumption\_Zero2020.txt'),
 PosixPath('Results/ResidentialConsumption\_Zero2025.txt'),
 PosixPath('Results/ResidentialConsumption\_Zero2030.txt'),
 PosixPath('Results/ResidentialConsumption\_Zero2035.txt'),
 PosixPath('Results/ResidentialConsumption\_Zero2040.txt'),
 PosixPath('Results/ResidentialConsumption\_Zero2045.txt'),
 PosixPath('Results/ResidentialConsumption\_Zero2050.txt')]
\end{sphinxVerbatim}
}

Our model output the files we were expecting and passed the \sphinxcode{\sphinxupquote{assert}} statement, meaning that it could find the “Hello world!” messages in the outputs.


\subsection{Adding TOML parameters to the outputs}
\label{\detokenize{advanced-guide/extending-muse:Adding-TOML-parameters-to-the-outputs}}
It would be useful if we could pass parameters from the TOML file to our new functions \sphinxcode{\sphinxupquote{consumption\_zero}} and \sphinxcode{\sphinxupquote{text\_dump}}. For example, in our previous iteration the consumption output was aggregating the data by \sphinxcode{\sphinxupquote{"timeslice"}}, by hardcoding the variable. We can pass a parameter which could do this by setting the \sphinxcode{\sphinxupquote{sum\_over}} parameter to be \sphinxcode{\sphinxupquote{True}}. In addition, we could change the message output by a new \sphinxcode{\sphinxupquote{text\_dump}} function.

Not all hooks are this flexible (for historical reasons, rather than any intrinsic difficulty). However, for outputs, we can do this as follows:

{
\sphinxsetup{VerbatimColor={named}{nbsphinx-code-bg}}
\sphinxsetup{VerbatimBorderColor={named}{nbsphinx-code-border}}
\begin{sphinxVerbatim}[commandchars=\\\{\}]
\llap{\color{nbsphinxin}[6]:\,\hspace{\fboxrule}\hspace{\fboxsep}}\PYG{n+nd}{@register\PYGZus{}output\PYGZus{}quantity}\PYG{p}{(}\PYG{n}{overwrite}\PYG{o}{=}\PYG{k+kc}{True}\PYG{p}{)}
\PYG{k}{def} \PYG{n+nf}{consumption\PYGZus{}zero}\PYG{p}{(}
    \PYG{n}{market}\PYG{p}{:} \PYG{n}{Dataset}\PYG{p}{,}
    \PYG{n}{capacity}\PYG{p}{:} \PYG{n}{DataArray}\PYG{p}{,}
    \PYG{n}{technologies}\PYG{p}{:} \PYG{n}{Dataset}\PYG{p}{,}
    \PYG{n}{sum\PYGZus{}over}\PYG{p}{:} \PYG{n}{Optional}\PYG{p}{[}\PYG{n}{List}\PYG{p}{[}\PYG{n}{Text}\PYG{p}{]}\PYG{p}{]} \PYG{o}{=} \PYG{k+kc}{None}\PYG{p}{,}
    \PYG{n}{drop}\PYG{p}{:} \PYG{n}{Optional}\PYG{p}{[}\PYG{n}{List}\PYG{p}{[}\PYG{n}{Text}\PYG{p}{]}\PYG{p}{]} \PYG{o}{=} \PYG{k+kc}{None}\PYG{p}{,}
    \PYG{n}{rounding}\PYG{p}{:} \PYG{n+nb}{int} \PYG{o}{=} \PYG{l+m+mi}{4}\PYG{p}{,}
\PYG{p}{)}\PYG{p}{:}
    \PYG{l+s+sd}{\PYGZdq{}\PYGZdq{}\PYGZdq{}Current consumption.\PYGZdq{}\PYGZdq{}\PYGZdq{}}
    \PYG{n}{result} \PYG{o}{=} \PYG{p}{(}
        \PYG{n}{market\PYGZus{}quantity}\PYG{p}{(}\PYG{n}{market}\PYG{o}{.}\PYG{n}{consumption}\PYG{p}{,} \PYG{n}{sum\PYGZus{}over}\PYG{o}{=}\PYG{n}{sum\PYGZus{}over}\PYG{p}{,} \PYG{n}{drop}\PYG{o}{=}\PYG{n}{drop}\PYG{p}{)}
        \PYG{o}{.}\PYG{n}{rename}\PYG{p}{(}\PYG{l+s+s2}{\PYGZdq{}}\PYG{l+s+s2}{consumption}\PYG{l+s+s2}{\PYGZdq{}}\PYG{p}{)}
        \PYG{o}{.}\PYG{n}{to\PYGZus{}dataframe}\PYG{p}{(}\PYG{p}{)}
        \PYG{o}{.}\PYG{n}{round}\PYG{p}{(}\PYG{n}{rounding}\PYG{p}{)}
    \PYG{p}{)}
    \PYG{k}{return} \PYG{n}{result}


\PYG{n+nd}{@register\PYGZus{}output\PYGZus{}sink}\PYG{p}{(}\PYG{n}{name}\PYG{o}{=}\PYG{l+s+s2}{\PYGZdq{}}\PYG{l+s+s2}{txt}\PYG{l+s+s2}{\PYGZdq{}}\PYG{p}{,} \PYG{n}{overwrite}\PYG{o}{=}\PYG{k+kc}{True}\PYG{p}{)}
\PYG{n+nd}{@sink\PYGZus{}to\PYGZus{}file}\PYG{p}{(}\PYG{l+s+s2}{\PYGZdq{}}\PYG{l+s+s2}{.txt}\PYG{l+s+s2}{\PYGZdq{}}\PYG{p}{)}
\PYG{k}{def} \PYG{n+nf}{text\PYGZus{}dump}\PYG{p}{(}
    \PYG{n}{data}\PYG{p}{:} \PYG{n}{Any}\PYG{p}{,}
    \PYG{n}{filename}\PYG{p}{:} \PYG{n}{Text}\PYG{p}{,}
    \PYG{n}{msg} \PYG{p}{:} \PYG{n}{Optional}\PYG{p}{[}\PYG{n}{Text}\PYG{p}{]} \PYG{o}{=} \PYG{l+s+s2}{\PYGZdq{}}\PYG{l+s+s2}{Hello, world!}\PYG{l+s+s2}{\PYGZdq{}}
\PYG{p}{)} \PYG{o}{\PYGZhy{}}\PYG{o}{\PYGZgt{}} \PYG{k+kc}{None}\PYG{p}{:}
    \PYG{k+kn}{from} \PYG{n+nn}{pathlib} \PYG{k+kn}{import} \PYG{n}{Path}
    \PYG{n}{Path}\PYG{p}{(}\PYG{n}{filename}\PYG{p}{)}\PYG{o}{.}\PYG{n}{write\PYGZus{}text}\PYG{p}{(}\PYG{l+s+sa}{f}\PYG{l+s+s2}{\PYGZdq{}}\PYG{l+s+si}{\PYGZob{}}\PYG{n}{msg}\PYG{l+s+si}{\PYGZcb{}}\PYG{l+s+se}{\PYGZbs{}n}\PYG{l+s+se}{\PYGZbs{}n}\PYG{l+s+si}{\PYGZob{}}\PYG{n}{data}\PYG{l+s+si}{\PYGZcb{}}\PYG{l+s+s2}{\PYGZdq{}}\PYG{p}{)}
\end{sphinxVerbatim}
}

We simply added parameters as arguments to both of our functions: \sphinxcode{\sphinxupquote{consumption\_zero}} and \sphinxcode{\sphinxupquote{text\_dump}}.

Note: The overwrite argument allows us to overwrite previously defined registered functions. This is useful in a notebook such as this. But it should not be used in general. If overwrite were false, then the code would issue a warning and it would leave the TOML to refer to the original functions at the beginning of the notebook. This is useful when using custom modules.

Now we can modify the output section to take additional arguments:

\begin{sphinxVerbatim}[commandchars=\\\{\}]
\PYG{p}{[}\PYG{p}{[}\PYG{n}{sectors}\PYG{o}{.}\PYG{n}{commercial}\PYG{o}{.}\PYG{n}{outputs}\PYG{p}{]}\PYG{p}{]}
\PYG{n}{quantity}\PYG{o}{.}\PYG{n}{name} \PYG{o}{=} \PYG{l+s+s2}{\PYGZdq{}}\PYG{l+s+s2}{consumption\PYGZus{}zero}\PYG{l+s+s2}{\PYGZdq{}}
\PYG{n}{quantity}\PYG{o}{.}\PYG{n}{sum\PYGZus{}over} \PYG{o}{=} \PYG{l+s+s2}{\PYGZdq{}}\PYG{l+s+s2}{timeslice}\PYG{l+s+s2}{\PYGZdq{}}
\PYG{n}{sink}\PYG{o}{.}\PYG{n}{name} \PYG{o}{=} \PYG{l+s+s2}{\PYGZdq{}}\PYG{l+s+s2}{txt}\PYG{l+s+s2}{\PYGZdq{}}
\PYG{n}{sink}\PYG{o}{.}\PYG{n}{filename} \PYG{o}{=} \PYG{l+s+s2}{\PYGZdq{}}\PYG{l+s+si}{\PYGZob{}cwd\PYGZcb{}}\PYG{l+s+s2}{/}\PYG{l+s+si}{\PYGZob{}default\PYGZus{}output\PYGZus{}dir\PYGZcb{}}\PYG{l+s+s2}{/}\PYG{l+s+si}{\PYGZob{}Sector\PYGZcb{}}\PYG{l+s+si}{\PYGZob{}Quantity\PYGZcb{}}\PYG{l+s+si}{\PYGZob{}year\PYGZcb{}}\PYG{l+s+si}{\PYGZob{}suffix\PYGZcb{}}\PYG{l+s+s2}{\PYGZdq{}}
\PYG{n}{sink}\PYG{o}{.}\PYG{n}{msg} \PYG{o}{=} \PYG{l+s+s2}{\PYGZdq{}}\PYG{l+s+s2}{Hello, you!}\PYG{l+s+s2}{\PYGZdq{}}
\PYG{n}{sink}\PYG{o}{.}\PYG{n}{overwrite} \PYG{o}{=} \PYG{k+kc}{True}
\end{sphinxVerbatim}

Here, we still want to use the \sphinxcode{\sphinxupquote{consumption\_zero}} function and the \sphinxcode{\sphinxupquote{txt}} sink. But we would like to change the message from “Hello world!” to “Hello you!” within the \sphinxcode{\sphinxupquote{TOML}} file.

Now, both sink and quantity are dictionaries which can take any number of arguments. Previously, we were using a shorthand for convenience. Again, we create a new settings file, and run this with our new parameters, which interface with our new functions.

{
\sphinxsetup{VerbatimColor={named}{nbsphinx-code-bg}}
\sphinxsetup{VerbatimBorderColor={named}{nbsphinx-code-border}}
\begin{sphinxVerbatim}[commandchars=\\\{\}]
\llap{\color{nbsphinxin}[7]:\,\hspace{\fboxrule}\hspace{\fboxsep}}\PYG{k+kn}{from} \PYG{n+nn}{pathlib} \PYG{k+kn}{import} \PYG{n}{Path}
\PYG{k+kn}{from} \PYG{n+nn}{toml} \PYG{k+kn}{import} \PYG{n}{load}\PYG{p}{,} \PYG{n}{dump}
\PYG{k+kn}{from} \PYG{n+nn}{muse} \PYG{k+kn}{import} \PYG{n}{examples}

\PYG{n}{model\PYGZus{}path} \PYG{o}{=} \PYG{n}{examples}\PYG{o}{.}\PYG{n}{copy\PYGZus{}model}\PYG{p}{(}\PYG{n}{overwrite}\PYG{o}{=}\PYG{k+kc}{True}\PYG{p}{)}
\PYG{n}{settings} \PYG{o}{=} \PYG{n}{load}\PYG{p}{(}\PYG{n}{model\PYGZus{}path} \PYG{o}{/} \PYG{l+s+s2}{\PYGZdq{}}\PYG{l+s+s2}{settings.toml}\PYG{l+s+s2}{\PYGZdq{}}\PYG{p}{)}
\PYG{n}{settings}\PYG{p}{[}\PYG{l+s+s2}{\PYGZdq{}}\PYG{l+s+s2}{sectors}\PYG{l+s+s2}{\PYGZdq{}}\PYG{p}{]}\PYG{p}{[}\PYG{l+s+s2}{\PYGZdq{}}\PYG{l+s+s2}{residential}\PYG{l+s+s2}{\PYGZdq{}}\PYG{p}{]}\PYG{p}{[}\PYG{l+s+s2}{\PYGZdq{}}\PYG{l+s+s2}{outputs}\PYG{l+s+s2}{\PYGZdq{}}\PYG{p}{]} \PYG{o}{=} \PYG{p}{[}
    \PYG{p}{\PYGZob{}}
        \PYG{l+s+s2}{\PYGZdq{}}\PYG{l+s+s2}{quantity}\PYG{l+s+s2}{\PYGZdq{}}\PYG{p}{:}\PYG{p}{\PYGZob{}}
            \PYG{l+s+s2}{\PYGZdq{}}\PYG{l+s+s2}{name}\PYG{l+s+s2}{\PYGZdq{}}\PYG{p}{:} \PYG{l+s+s2}{\PYGZdq{}}\PYG{l+s+s2}{consumption\PYGZus{}zero}\PYG{l+s+s2}{\PYGZdq{}}\PYG{p}{,}
            \PYG{l+s+s2}{\PYGZdq{}}\PYG{l+s+s2}{sum\PYGZus{}over}\PYG{l+s+s2}{\PYGZdq{}}\PYG{p}{:} \PYG{l+s+s2}{\PYGZdq{}}\PYG{l+s+s2}{timeslice}\PYG{l+s+s2}{\PYGZdq{}}
        \PYG{p}{\PYGZcb{}}\PYG{p}{,}
        \PYG{l+s+s2}{\PYGZdq{}}\PYG{l+s+s2}{sink}\PYG{l+s+s2}{\PYGZdq{}}\PYG{p}{:}\PYG{p}{\PYGZob{}}
            \PYG{l+s+s2}{\PYGZdq{}}\PYG{l+s+s2}{name}\PYG{l+s+s2}{\PYGZdq{}}\PYG{p}{:} \PYG{l+s+s2}{\PYGZdq{}}\PYG{l+s+s2}{txt}\PYG{l+s+s2}{\PYGZdq{}}\PYG{p}{,}
            \PYG{l+s+s2}{\PYGZdq{}}\PYG{l+s+s2}{filename}\PYG{l+s+s2}{\PYGZdq{}}\PYG{p}{:} \PYG{l+s+s2}{\PYGZdq{}}\PYG{l+s+si}{\PYGZob{}cwd\PYGZcb{}}\PYG{l+s+s2}{/}\PYG{l+s+si}{\PYGZob{}default\PYGZus{}output\PYGZus{}dir\PYGZcb{}}\PYG{l+s+s2}{/}\PYG{l+s+si}{\PYGZob{}Sector\PYGZcb{}}\PYG{l+s+si}{\PYGZob{}Quantity\PYGZcb{}}\PYG{l+s+si}{\PYGZob{}year\PYGZcb{}}\PYG{l+s+si}{\PYGZob{}suffix\PYGZcb{}}\PYG{l+s+s2}{\PYGZdq{}}\PYG{p}{,}
            \PYG{l+s+s2}{\PYGZdq{}}\PYG{l+s+s2}{msg}\PYG{l+s+s2}{\PYGZdq{}}\PYG{p}{:} \PYG{l+s+s2}{\PYGZdq{}}\PYG{l+s+s2}{Hello, you!}\PYG{l+s+s2}{\PYGZdq{}}\PYG{p}{,}
            \PYG{l+s+s2}{\PYGZdq{}}\PYG{l+s+s2}{overwrite}\PYG{l+s+s2}{\PYGZdq{}}\PYG{p}{:} \PYG{k+kc}{True}\PYG{p}{,}
        \PYG{p}{\PYGZcb{}}

    \PYG{p}{\PYGZcb{}}
\PYG{p}{]}

\PYG{n}{dump}\PYG{p}{(}\PYG{n}{settings}\PYG{p}{,} \PYG{p}{(}\PYG{n}{model\PYGZus{}path} \PYG{o}{/} \PYG{l+s+s2}{\PYGZdq{}}\PYG{l+s+s2}{modified\PYGZus{}settings\PYGZus{}2.toml}\PYG{l+s+s2}{\PYGZdq{}}\PYG{p}{)}\PYG{o}{.}\PYG{n}{open}\PYG{p}{(}\PYG{l+s+s2}{\PYGZdq{}}\PYG{l+s+s2}{w}\PYG{l+s+s2}{\PYGZdq{}}\PYG{p}{)}\PYG{p}{)}
\PYG{n}{settings}
\end{sphinxVerbatim}
}

{

\kern-\sphinxverbatimsmallskipamount\kern-\baselineskip
\kern+\FrameHeightAdjust\kern-\fboxrule
\vspace{\nbsphinxcodecellspacing}

\sphinxsetup{VerbatimColor={named}{white}}
\sphinxsetup{VerbatimBorderColor={named}{nbsphinx-code-border}}
\begin{sphinxVerbatim}[commandchars=\\\{\}]
\llap{\color{nbsphinxout}[7]:\,\hspace{\fboxrule}\hspace{\fboxsep}}\{'time\_framework': [2020, 2025, 2030, 2035, 2040, 2045, 2050],
 'foresight': 5,
 'regions': ['R1'],
 'interest\_rate': 0.1,
 'interpolation\_mode': 'Active',
 'log\_level': 'info',
 'equilibrium\_variable': 'demand',
 'maximum\_iterations': 100,
 'tolerance': 0.1,
 'tolerance\_unmet\_demand': -0.1,
 'outputs': [\{'quantity': 'prices',
   'sink': 'aggregate',
   'filename': '\{cwd\}/\{default\_output\_dir\}/MCA\{Quantity\}.csv'\},
  \{'quantity': 'capacity',
   'sink': 'aggregate',
   'filename': '\{cwd\}/\{default\_output\_dir\}/MCA\{Quantity\}.csv'\}],
 'carbon\_budget\_control': \{'budget': []\},
 'global\_input\_files': \{'projections': '\{path\}/input/Projections.csv',
  'global\_commodities': '\{path\}/input/GlobalCommodities.csv'\},
 'sectors': \{'residential': \{'type': 'default',
   'priority': 1,
   'dispatch\_production': 'share',
   'technodata': '\{path\}/technodata/residential/Technodata.csv',
   'commodities\_in': '\{path\}/technodata/residential/CommIn.csv',
   'commodities\_out': '\{path\}/technodata/residential/CommOut.csv',
   'subsectors': \{'retro\_and\_new': \{'agents': '\{path\}/technodata/Agents.csv',
     'existing\_capacity': '\{path\}/technodata/residential/ExistingCapacity.csv',
     'lpsolver': 'scipy',
     'constraints': ['max\_production',
      'max\_capacity\_expansion',
      'demand',
      'search\_space'],
     'demand\_share': 'new\_and\_retro',
     'forecast': 5\}\},
   'outputs': [\{'quantity': \{'name': 'consumption\_zero',
      'sum\_over': 'timeslice'\},
     'sink': \{'name': 'txt',
      'filename': '\{cwd\}/\{default\_output\_dir\}/\{Sector\}\{Quantity\}\{year\}\{suffix\}',
      'msg': 'Hello, you!',
      'overwrite': True\}\}],
   'interactions': [\{'net': 'new\_to\_retro', 'interaction': 'transfer'\}]\},
  'power': \{'type': 'default',
   'priority': 2,
   'dispatch\_production': 'share',
   'technodata': '\{path\}/technodata/power/Technodata.csv',
   'commodities\_in': '\{path\}/technodata/power/CommIn.csv',
   'commodities\_out': '\{path\}/technodata/power/CommOut.csv',
   'subsectors': \{'retro\_and\_new': \{'agents': '\{path\}/technodata/Agents.csv',
     'existing\_capacity': '\{path\}/technodata/power/ExistingCapacity.csv',
     'lpsolver': 'scipy'\}\},
   'outputs': [\{'filename': '\{cwd\}/\{default\_output\_dir\}/\{Sector\}/\{Quantity\}/\{year\}\{suffix\}',
     'quantity': 'capacity',
     'sink': 'csv',
     'overwrite': True\}],
   'interactions': [\{'net': 'new\_to\_retro', 'interaction': 'transfer'\}]\},
  'gas': \{'type': 'default',
   'priority': 3,
   'dispatch\_production': 'share',
   'technodata': '\{path\}/technodata/gas/Technodata.csv',
   'commodities\_in': '\{path\}/technodata/gas/CommIn.csv',
   'commodities\_out': '\{path\}/technodata/gas/CommOut.csv',
   'subsectors': \{'retro\_and\_new': \{'agents': '\{path\}/technodata/Agents.csv',
     'existing\_capacity': '\{path\}/technodata/gas/ExistingCapacity.csv',
     'lpsolver': 'scipy'\}\},
   'outputs': [\{'filename': '\{cwd\}/\{default\_output\_dir\}/\{Sector\}/\{Quantity\}/\{year\}\{suffix\}',
     'quantity': 'capacity',
     'sink': 'csv',
     'overwrite': True\}],
   'interactions': [\{'net': 'new\_to\_retro', 'interaction': 'transfer'\}]\},
  'residential\_presets': \{'type': 'presets',
   'priority': 0,
   'consumption\_path': '\{path\}/technodata/preset/*Consumption.csv'\}\},
 'timeslices': \{'all-year': \{'all-week': \{'night': 1460,
    'morning': 1460,
    'afternoon': 1460,
    'early-peak': 1460,
    'late-peak': 1460,
    'evening': 1460\}\},
  'level\_names': ['month', 'day', 'hour']\}\}
\end{sphinxVerbatim}
}

We then run the simulation again:

{
\sphinxsetup{VerbatimColor={named}{nbsphinx-code-bg}}
\sphinxsetup{VerbatimBorderColor={named}{nbsphinx-code-border}}
\begin{sphinxVerbatim}[commandchars=\\\{\}]
\llap{\color{nbsphinxin}[10]:\,\hspace{\fboxrule}\hspace{\fboxsep}}\PYG{n}{mca} \PYG{o}{=} \PYG{n}{MCA}\PYG{o}{.}\PYG{n}{factory}\PYG{p}{(}\PYG{n}{model\PYGZus{}path} \PYG{o}{/} \PYG{l+s+s2}{\PYGZdq{}}\PYG{l+s+s2}{modified\PYGZus{}settings\PYGZus{}2.toml}\PYG{l+s+s2}{\PYGZdq{}}\PYG{p}{)}
\PYG{n}{mca}\PYG{o}{.}\PYG{n}{run}\PYG{p}{(}\PYG{p}{)}\PYG{p}{;}
\end{sphinxVerbatim}
}

{

\kern-\sphinxverbatimsmallskipamount\kern-\baselineskip
\kern+\FrameHeightAdjust\kern-\fboxrule
\vspace{\nbsphinxcodecellspacing}

\sphinxsetup{VerbatimColor={named}{white}}
\sphinxsetup{VerbatimBorderColor={named}{nbsphinx-code-border}}
\begin{sphinxVerbatim}[commandchars=\\\{\}]
Primal Feasibility  Dual Feasibility    Duality Gap         Step             Path Parameter      Objective
1.0                 1.0                 1.0                 -                1.0                 148.9679256735
0.2598249156018     0.2598249156018     0.2598249156018     0.7495200004432  0.2598249156018     120.5733849622
0.02399956829695    0.02399956829695    0.02399956829695    0.9210391224498  0.02399956829695    4.780663765494
0.0181364461758     0.0181364461758     0.0181364461758     0.2509588065043  0.0181364461758     7.107141691547
0.01499350833129    0.01499350833129    0.01499350833129    0.1921973185437  0.01499350833129    70.77614035582
0.004968295711366   0.004968295711367   0.004968295711366   0.6857131120066  0.004968295711366   164.7472224003
0.0006443120819652  0.0006443120819642  0.000644312081964   0.8804718592549  0.0006443120819672  289.7109372802
2.427431365313e-06  2.427431365276e-06  2.42743136527e-06   0.9976309182175  2.427431365399e-06  310.7437190082
1.214286379284e-10  1.214286245985e-10  1.214286217581e-10  0.9999499778566  1.214286284187e-10  310.7859704917
Optimization terminated successfully.
         Current function value: 310.785970
         Iterations: 8
Primal Feasibility  Dual Feasibility    Duality Gap         Step             Path Parameter      Objective
1.0                 1.0                 1.0                 -                1.0                 148.9679256735
0.1806451257467     0.1806451257467     0.1806451257467     0.8281847212959  0.1806451257467     169.6037982942
0.02684129624378    0.02684129624378    0.02684129624378    0.8635116331789  0.02684129624378    123.2904723181
0.01081107082374    0.01081107082373    0.01081107082373    0.6279145428497  0.01081107082373    293.3552576002
0.00150335333755    0.001503353337531   0.001503353337531   0.8785435425234  0.001503353337582   618.5754737903
4.126548293386e-06  4.126548293452e-06  4.126548293469e-06  0.99729939758    4.126548293473e-06  673.2429034259
2.063813940498e-10  2.063814559148e-10  2.063814538311e-10  0.9999499869317  2.06381853248e-10   673.369600975
Optimization terminated successfully.
         Current function value: 673.369601
         Iterations: 6
Primal Feasibility  Dual Feasibility    Duality Gap         Step             Path Parameter      Objective
1.0                 1.0                 1.0                 -                1.0                 225.7506433631
0.07675788061753    0.07675788061753    0.07675788061751    0.9271368109475  0.07675788061753    1.307751786207
0.01889464099889    0.01889464099889    0.01889464099888    0.7970949255449  0.01889464099889    19.4989221165
0.007543783963398   0.007543783963394   0.007543783963392   0.6153162486386  0.007543783963394   16.52048983639
0.002504946781023   0.002504946781021   0.00250494678102    0.7004074675406  0.002504946781021   72.319459457
0.0004445444355394  0.000444544435539   0.0004445444355388  0.8689995047748  0.000444544435539   423.758860583
1.214501095331e-05  1.214501095325e-05  1.214501095324e-05  0.9820606335229  1.214501095322e-05  675.1646942955
1.070439776445e-09  1.07043978707e-09   1.070439766673e-09  0.9999120772687  1.070439801419e-09  681.9179209593
5.353929135063e-14  5.353592310399e-14  5.35373239097e-14   0.9999499866071  5.352230256347e-14  681.9185224492
Optimization terminated successfully.
         Current function value: 681.918522
         Iterations: 8
\end{sphinxVerbatim}
}

{

\kern-\sphinxverbatimsmallskipamount\kern-\baselineskip
\kern+\FrameHeightAdjust\kern-\fboxrule
\vspace{\nbsphinxcodecellspacing}

\sphinxsetup{VerbatimColor={named}{nbsphinx-stderr}}
\sphinxsetup{VerbatimBorderColor={named}{nbsphinx-code-border}}
\begin{sphinxVerbatim}[commandchars=\\\{\}]
-- 2020-11-09 11:32:00 - muse.mca - WARNING
Check growth constraints for wind.

\end{sphinxVerbatim}
}

{

\kern-\sphinxverbatimsmallskipamount\kern-\baselineskip
\kern+\FrameHeightAdjust\kern-\fboxrule
\vspace{\nbsphinxcodecellspacing}

\sphinxsetup{VerbatimColor={named}{white}}
\sphinxsetup{VerbatimBorderColor={named}{nbsphinx-code-border}}
\begin{sphinxVerbatim}[commandchars=\\\{\}]
Primal Feasibility  Dual Feasibility    Duality Gap         Step             Path Parameter      Objective
1.0                 1.0                 1.0                 -                1.0                 359.2443189825
0.2131118149695     0.2131118149695     0.2131118149695     0.7960343319409  0.2131118149695     262.2885392799
0.05758094728718    0.05758094728718    0.05758094728718    0.7523513573813  0.05758094728718    13.16175379893
0.01048349585899    0.01048349585899    0.01048349585899    0.8203940076989  0.01048349585899    9.036399448741
0.009005598049604   0.009005598049604   0.009005598049604   0.1488155054953  0.009005598049604   19.40296370843
0.002317604003518   0.002317604003518   0.002317604003518   0.8098673571979  0.002317604003518   226.5492459765
0.0009784310820962  0.0009784310820963  0.0009784310820963  0.5991550275447  0.0009784310820971  289.4684048776
0.0001326875071986  0.0001326875071986  0.0001326875071986  0.8836343167337  0.0001326875071987  358.6919881748
6.688262823007e-08  6.688262823276e-08  6.688262823391e-08  0.9995454959391  6.688262822145e-08  365.8223849509
3.344208692489e-12  3.34420597683e-12   3.344204125008e-12  0.9999499989235  3.344177862259e-12  365.8250744356
Optimization terminated successfully.
         Current function value: 365.825074
         Iterations: 9
Primal Feasibility  Dual Feasibility    Duality Gap         Step             Path Parameter      Objective
1.0                 1.0                 1.0                 -                1.0                 179.6221594912
0.06792119055695    0.06792119055695    0.06792119055695    0.9362017234058  0.06792119055695    76.26287556551
0.009746786382626   0.009746786382626   0.009746786382626   0.8719786940257  0.009746786382625   34.28929487798
0.00504956460969    0.005049564609689   0.005049564609689   0.5134005827998  0.005049564609689   115.6145513358
0.003132107070668   0.003132107070663   0.003132107070663   0.3922116719677  0.003132107070663   175.0444435627
0.0005331490663204  0.000533149066321   0.000533149066321   0.8631478238108  0.0005331490663195  430.7265923665
6.809758501168e-06  6.809758501084e-06  6.809758501069e-06  0.9896962642248  6.809758501216e-06  482.9615875673
3.566453823927e-10  3.56645399101e-10   3.566453982014e-10  0.9999476448412  3.566453742836e-10  483.66367672
Optimization terminated successfully.
         Current function value: 483.663677
         Iterations: 7
Primal Feasibility  Dual Feasibility    Duality Gap         Step             Path Parameter      Objective
1.0                 1.0                 1.0                 -                1.0                 547.5170704708
0.06099397574481    0.06099397574481    0.06099397574482    0.944387525785   0.06099397574482    192.4316112656
0.01386491315737    0.01386491315737    0.01386491315737    0.7896145976147  0.01386491315737    3.157754203839
0.00713143943229    0.00713143943229    0.007131439432292   0.5116778496497  0.007131439432291   6.913227303095
0.0007854492963609  0.0007854492963609  0.0007854492963611  0.9057153968575  0.000785449296361   6.591304425235
0.0003968642772996  0.0003968642772997  0.0003968642772998  0.5195625901748  0.0003968642772996  5.218772647071
5.276614006577e-06  5.276614006577e-06  5.276614006576e-06  0.9941140946865  5.276614006565e-06  5.272614338781
3.679893198137e-10  3.679893242575e-10  3.679893250277e-10  0.9999303117912  3.679893357516e-10  5.266601980833
1.790517523351e-14  1.842347640038e-14  1.841384711876e-14  0.9999499611495  1.844295319492e-14  5.266601636509
Optimization terminated successfully.
         Current function value: 5.266602
         Iterations: 8
Primal Feasibility  Dual Feasibility    Duality Gap         Step             Path Parameter      Objective
1.0                 1.0                 1.0                 -                1.0                 273.7585352354
0.06117571447458    0.06117571447458    0.06117571447455    0.9425787965419  0.06117571447457    204.7212184456
0.009942269802024   0.009942269802025   0.00994226980202    0.8403658088806  0.009942269802024   0.7760264390675
0.002147318631478   0.002147318631478   0.002147318631477   0.8277733790617  0.002147318631478   3.894705138429
0.0009649606635229  0.0009649606635227  0.0009649606635223  0.5635818341147  0.0009649606635226  2.700821086695
0.0002350111516202  0.0002350111516201  0.00023501115162    0.7866567764877  0.0002350111516201  14.62347819316
5.857294640779e-05  5.857294640777e-05  5.857294640774e-05  0.8105927228255  5.857294640777e-05  123.50656776
1.165473006364e-06  1.165473006407e-06  1.165473006407e-06  0.9835504451632  1.165473006409e-06  151.6654340598
8.362816702708e-11  8.362816612134e-11  8.362816648317e-11  0.9999285085203  8.36281608009e-11   152.3843167925
4.185697343433e-15  4.184037681176e-15  4.18416825892e-15   0.9999499678098  4.181415889613e-15  152.3843678125
Optimization terminated successfully.
         Current function value: 152.384368
         Iterations: 9
Primal Feasibility  Dual Feasibility    Duality Gap         Step             Path Parameter      Objective
1.0                 1.0                 1.0                 -                1.0                 24.59260538017
0.01918847234907    0.01918847234907    0.01918847234907    0.9908439847925  0.01918847234907    0.01449776485458
0.01221126742797    0.01221126742797    0.01221126742797    0.3735738645022  0.01221126742797    0.1603023022985
0.009831906616685   0.009831906616685   0.009831906616685   0.2161492677935  0.009831906616685   20.38072882765
0.001212407727123   0.001212407727023   0.001212407727023   0.8886055956883  0.001212407727023   51.40232805869
4.730913157321e-06  4.73091315438e-06   4.730913154368e-06  0.9976178802846  4.730913155107e-06  59.83474944668
2.366017130877e-10  2.366016942176e-10  2.366016879444e-10  0.9999499881554  2.366016905478e-10  59.84200873085
Optimization terminated successfully.
         Current function value: 59.842009
         Iterations: 6
\end{sphinxVerbatim}
}

{

\kern-\sphinxverbatimsmallskipamount\kern-\baselineskip
\kern+\FrameHeightAdjust\kern-\fboxrule
\vspace{\nbsphinxcodecellspacing}

\sphinxsetup{VerbatimColor={named}{nbsphinx-stderr}}
\sphinxsetup{VerbatimBorderColor={named}{nbsphinx-code-border}}
\begin{sphinxVerbatim}[commandchars=\\\{\}]
-- 2020-11-09 11:32:10 - muse.mca - WARNING
Check growth constraints for wind.

\end{sphinxVerbatim}
}

{

\kern-\sphinxverbatimsmallskipamount\kern-\baselineskip
\kern+\FrameHeightAdjust\kern-\fboxrule
\vspace{\nbsphinxcodecellspacing}

\sphinxsetup{VerbatimColor={named}{white}}
\sphinxsetup{VerbatimBorderColor={named}{nbsphinx-code-border}}
\begin{sphinxVerbatim}[commandchars=\\\{\}]
Primal Feasibility  Dual Feasibility    Duality Gap         Step             Path Parameter      Objective
1.0                 1.0                 1.0                 -                1.0                 359.2443189825
0.2131118149695     0.2131118149695     0.2131118149695     0.7960343319409  0.2131118149695     262.2885392799
0.05758094728718    0.05758094728718    0.05758094728718    0.7523513573813  0.05758094728718    13.16175379893
0.01048349585899    0.01048349585899    0.01048349585899    0.8203940076989  0.01048349585899    9.036399448741
0.009005598049604   0.009005598049604   0.009005598049604   0.1488155054953  0.009005598049604   19.40296370843
0.002317604003518   0.002317604003518   0.002317604003518   0.8098673571979  0.002317604003518   226.5492459765
0.0009784310820962  0.0009784310820963  0.0009784310820963  0.5991550275447  0.0009784310820971  289.4684048776
0.0001326875071986  0.0001326875071986  0.0001326875071986  0.8836343167337  0.0001326875071987  358.6919881748
6.688262823007e-08  6.688262823276e-08  6.688262823391e-08  0.9995454959391  6.688262822145e-08  365.8223849509
3.344208692489e-12  3.34420597683e-12   3.344204125008e-12  0.9999499989235  3.344177862259e-12  365.8250744356
Optimization terminated successfully.
         Current function value: 365.825074
         Iterations: 9
Primal Feasibility  Dual Feasibility    Duality Gap         Step             Path Parameter      Objective
1.0                 1.0                 1.0                 -                1.0                 179.6221594912
0.06792119055695    0.06792119055695    0.06792119055695    0.9362017234058  0.06792119055695    76.26287556551
0.009746786382626   0.009746786382626   0.009746786382626   0.8719786940257  0.009746786382625   34.28929487798
0.00504956460969    0.005049564609689   0.005049564609689   0.5134005827998  0.005049564609689   115.6145513358
0.003132107070668   0.003132107070663   0.003132107070663   0.3922116719677  0.003132107070663   175.0444435627
0.0005331490663204  0.000533149066321   0.000533149066321   0.8631478238108  0.0005331490663195  430.7265923665
6.809758501168e-06  6.809758501084e-06  6.809758501069e-06  0.9896962642248  6.809758501216e-06  482.9615875673
3.566453823927e-10  3.56645399101e-10   3.566453982014e-10  0.9999476448412  3.566453742836e-10  483.66367672
Optimization terminated successfully.
         Current function value: 483.663677
         Iterations: 7
Primal Feasibility  Dual Feasibility    Duality Gap         Step             Path Parameter      Objective
1.0                 1.0                 1.0                 -                1.0                 547.5170704708
0.06099397574481    0.06099397574481    0.06099397574482    0.944387525785   0.06099397574482    192.4316112656
0.01386491315737    0.01386491315737    0.01386491315737    0.7896145976147  0.01386491315737    3.157754203839
0.00713143943229    0.00713143943229    0.007131439432292   0.5116778496497  0.007131439432291   6.913227303095
0.0007854492963609  0.0007854492963609  0.0007854492963611  0.9057153968575  0.000785449296361   6.591304425235
0.0003968642772996  0.0003968642772997  0.0003968642772998  0.5195625901748  0.0003968642772996  5.218772647071
5.276614006577e-06  5.276614006577e-06  5.276614006576e-06  0.9941140946865  5.276614006565e-06  5.272614338781
3.679893198137e-10  3.679893242575e-10  3.679893250277e-10  0.9999303117912  3.679893357516e-10  5.266601980833
1.790517523351e-14  1.842347640038e-14  1.841384711876e-14  0.9999499611495  1.844295319492e-14  5.266601636509
Optimization terminated successfully.
         Current function value: 5.266602
         Iterations: 8
Primal Feasibility  Dual Feasibility    Duality Gap         Step             Path Parameter      Objective
1.0                 1.0                 1.0                 -                1.0                 273.7585352354
0.06117571447458    0.06117571447458    0.06117571447455    0.9425787965419  0.06117571447457    204.7212184456
0.009942269802024   0.009942269802025   0.00994226980202    0.8403658088806  0.009942269802024   0.7760264390675
0.002147318631478   0.002147318631478   0.002147318631477   0.8277733790617  0.002147318631478   3.894705138429
0.0009649606635229  0.0009649606635227  0.0009649606635223  0.5635818341147  0.0009649606635226  2.700821086695
0.0002350111516202  0.0002350111516201  0.00023501115162    0.7866567764877  0.0002350111516201  14.62347819316
5.857294640779e-05  5.857294640777e-05  5.857294640774e-05  0.8105927228255  5.857294640777e-05  123.50656776
1.165473006364e-06  1.165473006407e-06  1.165473006407e-06  0.9835504451632  1.165473006409e-06  151.6654340598
8.362816702708e-11  8.362816612134e-11  8.362816648317e-11  0.9999285085203  8.36281608009e-11   152.3843167925
4.185697343433e-15  4.184037681176e-15  4.18416825892e-15   0.9999499678098  4.181415889613e-15  152.3843678125
Optimization terminated successfully.
         Current function value: 152.384368
         Iterations: 9
Primal Feasibility  Dual Feasibility    Duality Gap         Step             Path Parameter      Objective
1.0                 1.0                 1.0                 -                1.0                 24.59260538017
0.01918847234907    0.01918847234907    0.01918847234907    0.9908439847925  0.01918847234907    0.01449776485458
0.01221126742797    0.01221126742797    0.01221126742797    0.3735738645022  0.01221126742797    0.1603023022985
0.009831906616685   0.009831906616685   0.009831906616685   0.2161492677935  0.009831906616685   20.38072882765
0.001212407727123   0.001212407727023   0.001212407727023   0.8886055956883  0.001212407727023   51.40232805869
4.730913157321e-06  4.73091315438e-06   4.730913154368e-06  0.9976178802846  4.730913155107e-06  59.83474944668
2.366017130877e-10  2.366016942176e-10  2.366016879444e-10  0.9999499881554  2.366016905478e-10  59.84200873085
Optimization terminated successfully.
         Current function value: 59.842009
         Iterations: 6
\end{sphinxVerbatim}
}

{

\kern-\sphinxverbatimsmallskipamount\kern-\baselineskip
\kern+\FrameHeightAdjust\kern-\fboxrule
\vspace{\nbsphinxcodecellspacing}

\sphinxsetup{VerbatimColor={named}{nbsphinx-stderr}}
\sphinxsetup{VerbatimBorderColor={named}{nbsphinx-code-border}}
\begin{sphinxVerbatim}[commandchars=\\\{\}]
-- 2020-11-09 11:32:19 - muse.mca - WARNING
Check growth constraints for wind.

\end{sphinxVerbatim}
}

{

\kern-\sphinxverbatimsmallskipamount\kern-\baselineskip
\kern+\FrameHeightAdjust\kern-\fboxrule
\vspace{\nbsphinxcodecellspacing}

\sphinxsetup{VerbatimColor={named}{white}}
\sphinxsetup{VerbatimBorderColor={named}{nbsphinx-code-border}}
\begin{sphinxVerbatim}[commandchars=\\\{\}]
Primal Feasibility  Dual Feasibility    Duality Gap         Step             Path Parameter      Objective
1.0                 1.0                 1.0                 -                1.0                 414.316476678
0.1943112264485     0.1943112264485     0.1943112264485     0.8115414580695  0.1943112264485     293.1180738751
0.05737071259203    0.05737071259203    0.05737071259203    0.7291321197238  0.05737071259203    17.07106219013
0.01062277327644    0.01062277327644    0.01062277327644    0.8173814867435  0.01062277327644    12.44885046687
0.009064514495371   0.009064514495372   0.009064514495372   0.1550649500025  0.009064514495372   28.00868491133
0.002334119807294   0.002334119807293   0.002334119807293   0.8051326751218  0.002334119807295   275.8552661516
0.0006946146456239  0.0006946146456238  0.0006946146456238  0.7097377670861  0.0006946146456242  332.7560201847
0.000253341221249   0.0002533412212489  0.0002533412212489  0.6718481634042  0.0002533412212491  377.2933943083
1.33722226265e-06   1.337222262655e-06  1.337222262657e-06  0.9952598213341  1.337222262651e-06  392.3599718847
6.775876566156e-11  6.775875940049e-11  6.775875524037e-11  0.9999493313269  6.775868905564e-11  392.4568216146
Optimization terminated successfully.
         Current function value: 392.456822
         Iterations: 9
Primal Feasibility  Dual Feasibility    Duality Gap         Step             Path Parameter      Objective
1.0                 1.0                 1.0                 -                1.0                 414.316476678
0.1030590385675     0.1030590385675     0.1030590385675     0.9011154818883  0.1030590385675     219.2748221074
0.02654054666621    0.02654054666621    0.02654054666621    0.7752562655553  0.02654054666621    276.7551091027
0.01427394633699    0.01427394633699    0.01427394633699    0.4912526709581  0.01427394633699    607.4391713228
0.001218580492651   0.001218580492601   0.001218580492601   0.9248974186534  0.001218580492676   1166.289755796
1.888403761715e-05  1.888403761622e-05  1.888403761624e-05  0.985551099478   1.888403761735e-05  1241.740132447
1.545745837373e-09  1.545745372552e-09  1.545745362755e-09  0.999922927438   1.545745648845e-09  1242.779847889
1.692223225044e-10  1.873153075245e-10  1.873152955153e-10  0.8801597673128  1.817506825368e-10  1242.779937345
Optimization terminated successfully.
         Current function value: 1242.779937
         Iterations: 7
Primal Feasibility  Dual Feasibility    Duality Gap         Step             Path Parameter      Objective
1.0                 1.0                 1.0                 -                1.0                 652.1069903706
0.06097571429344    0.06097571429346    0.06097571429347    0.9444902662603  0.06097571429346    285.2043200632
0.01533833555931    0.01533833555931    0.01533833555932    0.7609599230413  0.01533833555932    2.7262007615
0.006316196361705   0.006316196361706   0.006316196361708   0.6163829523723  0.006316196361707   8.634999065661
0.001219580332373   0.001219580332374   0.001219580332374   0.8323480537452  0.001219580332374   8.307794697355
0.0005645791470893  0.0005645791470901  0.0005645791470902  0.5500511910892  0.0005645791470901  6.632034269378
0.0001849432525583  0.0001849432525588  0.0001849432525588  0.7137620753374  0.0001849432525588  4.483601470858
2.517759634035e-05  2.517759634041e-05  2.517759634042e-05  0.9034269505801  2.517759634041e-05  4.138476382855
3.098182398421e-08  3.098182439526e-08  3.09818244027e-08   0.9988322072353  3.098182439405e-08  4.000023749419
1.548986338477e-12  1.549101875116e-12  1.549109003245e-12  0.9999499994251  1.549106268784e-12  3.999950652469
Optimization terminated successfully.
         Current function value: 3.999951
         Iterations: 9
Primal Feasibility  Dual Feasibility    Duality Gap         Step             Path Parameter      Objective
1.0                 1.0                 1.0                 -                1.0                 24.59260538017
0.02944420803734    0.02944420803734    0.02944420803733    0.9778743187132  0.02944420803734    27.12355968787
0.008204037532537   0.008204037532537   0.008204037532536   0.7626654784836  0.008204037532537   110.4823556367
0.0004115277845134  0.0004115277844911  0.000411527784491   0.9587152757388  0.0004115277845217  181.7019444318
2.842168973263e-08  2.842168968894e-08  2.84216896431e-08   0.9999314505927  2.84216896475e-08   184.4443098122
1.421064808692e-12  1.421032025856e-12  1.420975506573e-12  0.999950002033   1.421084481186e-12  184.444539173
Optimization terminated successfully.
         Current function value: 184.444539
         Iterations: 5
\end{sphinxVerbatim}
}

{

\kern-\sphinxverbatimsmallskipamount\kern-\baselineskip
\kern+\FrameHeightAdjust\kern-\fboxrule
\vspace{\nbsphinxcodecellspacing}

\sphinxsetup{VerbatimColor={named}{nbsphinx-stderr}}
\sphinxsetup{VerbatimBorderColor={named}{nbsphinx-code-border}}
\begin{sphinxVerbatim}[commandchars=\\\{\}]
-- 2020-11-09 11:32:28 - muse.mca - WARNING
Check growth constraints for wind.

\end{sphinxVerbatim}
}

{

\kern-\sphinxverbatimsmallskipamount\kern-\baselineskip
\kern+\FrameHeightAdjust\kern-\fboxrule
\vspace{\nbsphinxcodecellspacing}

\sphinxsetup{VerbatimColor={named}{white}}
\sphinxsetup{VerbatimBorderColor={named}{nbsphinx-code-border}}
\begin{sphinxVerbatim}[commandchars=\\\{\}]
Primal Feasibility  Dual Feasibility    Duality Gap         Step             Path Parameter      Objective
1.0                 1.0                 1.0                 -                1.0                 414.316476678
0.1943112264485     0.1943112264485     0.1943112264485     0.8115414580695  0.1943112264485     293.1180738751
0.05737071259203    0.05737071259203    0.05737071259203    0.7291321197238  0.05737071259203    17.07106219013
0.01062277327644    0.01062277327644    0.01062277327644    0.8173814867435  0.01062277327644    12.44885046687
0.009064514495371   0.009064514495372   0.009064514495372   0.1550649500025  0.009064514495372   28.00868491133
0.002334119807294   0.002334119807293   0.002334119807293   0.8051326751218  0.002334119807295   275.8552661516
0.0006946146456239  0.0006946146456238  0.0006946146456238  0.7097377670861  0.0006946146456242  332.7560201847
0.000253341221249   0.0002533412212489  0.0002533412212489  0.6718481634042  0.0002533412212491  377.2933943083
1.33722226265e-06   1.337222262655e-06  1.337222262657e-06  0.9952598213341  1.337222262651e-06  392.3599718847
6.775876566156e-11  6.775875940049e-11  6.775875524037e-11  0.9999493313269  6.775868905564e-11  392.4568216146
Optimization terminated successfully.
         Current function value: 392.456822
         Iterations: 9
Primal Feasibility  Dual Feasibility    Duality Gap         Step             Path Parameter      Objective
1.0                 1.0                 1.0                 -                1.0                 414.316476678
0.1030590385675     0.1030590385675     0.1030590385675     0.9011154818883  0.1030590385675     219.2748221074
0.02654054666621    0.02654054666621    0.02654054666621    0.7752562655553  0.02654054666621    276.7551091027
0.01427394633699    0.01427394633699    0.01427394633699    0.4912526709581  0.01427394633699    607.4391713228
0.001218580492651   0.001218580492601   0.001218580492601   0.9248974186534  0.001218580492676   1166.289755796
1.888403761715e-05  1.888403761622e-05  1.888403761624e-05  0.985551099478   1.888403761735e-05  1241.740132447
1.545745837373e-09  1.545745372552e-09  1.545745362755e-09  0.999922927438   1.545745648845e-09  1242.779847889
1.692223225044e-10  1.873153075245e-10  1.873152955153e-10  0.8801597673128  1.817506825368e-10  1242.779937345
Optimization terminated successfully.
         Current function value: 1242.779937
         Iterations: 7
Primal Feasibility  Dual Feasibility    Duality Gap         Step             Path Parameter      Objective
1.0                 1.0                 1.0                 -                1.0                 652.1069903706
0.06097571429344    0.06097571429346    0.06097571429347    0.9444902662603  0.06097571429346    285.2043200632
0.01533833555931    0.01533833555931    0.01533833555932    0.7609599230413  0.01533833555932    2.7262007615
0.006316196361705   0.006316196361706   0.006316196361708   0.6163829523723  0.006316196361707   8.634999065661
0.001219580332373   0.001219580332374   0.001219580332374   0.8323480537452  0.001219580332374   8.307794697355
0.0005645791470893  0.0005645791470901  0.0005645791470902  0.5500511910892  0.0005645791470901  6.632034269378
0.0001849432525583  0.0001849432525588  0.0001849432525588  0.7137620753374  0.0001849432525588  4.483601470858
2.517759634035e-05  2.517759634041e-05  2.517759634042e-05  0.9034269505801  2.517759634041e-05  4.138476382855
3.098182398421e-08  3.098182439526e-08  3.09818244027e-08   0.9988322072353  3.098182439405e-08  4.000023749419
1.548986338477e-12  1.549101875116e-12  1.549109003245e-12  0.9999499994251  1.549106268784e-12  3.999950652469
Optimization terminated successfully.
         Current function value: 3.999951
         Iterations: 9
Primal Feasibility  Dual Feasibility    Duality Gap         Step             Path Parameter      Objective
1.0                 1.0                 1.0                 -                1.0                 24.59260538017
0.02944420803734    0.02944420803734    0.02944420803733    0.9778743187132  0.02944420803734    27.12355968787
0.008204037532537   0.008204037532537   0.008204037532536   0.7626654784836  0.008204037532537   110.4823556367
0.0004115277845134  0.0004115277844911  0.000411527784491   0.9587152757388  0.0004115277845217  181.7019444318
2.842168973263e-08  2.842168968894e-08  2.84216896431e-08   0.9999314505927  2.84216896475e-08   184.4443098122
1.421064808692e-12  1.421032025856e-12  1.420975506573e-12  0.999950002033   1.421084481186e-12  184.444539173
Optimization terminated successfully.
         Current function value: 184.444539
         Iterations: 5
\end{sphinxVerbatim}
}

{

\kern-\sphinxverbatimsmallskipamount\kern-\baselineskip
\kern+\FrameHeightAdjust\kern-\fboxrule
\vspace{\nbsphinxcodecellspacing}

\sphinxsetup{VerbatimColor={named}{nbsphinx-stderr}}
\sphinxsetup{VerbatimBorderColor={named}{nbsphinx-code-border}}
\begin{sphinxVerbatim}[commandchars=\\\{\}]
-- 2020-11-09 11:32:37 - muse.mca - WARNING
Check growth constraints for wind.

\end{sphinxVerbatim}
}

{

\kern-\sphinxverbatimsmallskipamount\kern-\baselineskip
\kern+\FrameHeightAdjust\kern-\fboxrule
\vspace{\nbsphinxcodecellspacing}

\sphinxsetup{VerbatimColor={named}{white}}
\sphinxsetup{VerbatimBorderColor={named}{nbsphinx-code-border}}
\begin{sphinxVerbatim}[commandchars=\\\{\}]
Primal Feasibility  Dual Feasibility    Duality Gap         Step             Path Parameter      Objective
1.0                 1.0                 1.0                 -                1.0                 454.6784294315
0.1862232334687     0.1862232334687     0.1862232334687     0.8174567476576  0.1862232334687     282.4632604722
0.02870747975048    0.02870747975048    0.02870747975048    0.8875534078522  0.02870747975048    46.09791383177
0.009705128410004   0.009705128410004   0.009705128410004   0.6723641007507  0.009705128410004   35.13402203751
0.007693915131578   0.007693915131578   0.007693915131578   0.2197837130033  0.007693915131578   80.93881539206
0.001699044794839   0.001699044794837   0.001699044794837   0.8250841182677  0.00169904479484    298.2653882331
0.0005650008980062  0.0005650008980057  0.0005650008980057  0.678565948553   0.0005650008980065  327.975809321
0.0002196033953225  0.0002196033953223  0.0002196033953223  0.6469207178836  0.0002196033953226  358.4708363196
2.579861478512e-06  2.579861478501e-06  2.579861478502e-06  0.9887803221269  2.579861478497e-06  368.5514208652
1.310502062448e-10  1.310502049574e-10  1.31050204896e-10   0.9999492039366  1.310502128569e-10  368.6632645094
Optimization terminated successfully.
         Current function value: 368.663265
         Iterations: 9
Primal Feasibility  Dual Feasibility    Duality Gap         Step             Path Parameter      Objective
1.0                 1.0                 1.0                 -                1.0                 454.6784294315
0.06717002148527    0.06717002148527    0.06717002148527    0.939336263039   0.06717002148527    183.2225626364
0.0335506482529     0.0335506482529     0.0335506482529     0.5230681149682  0.0335506482529     256.2098343218
0.01040420231972    0.01040420231972    0.01040420231972    0.6975119886622  0.01040420231972    178.3301859411
0.004349337378734   0.004349337378734   0.004349337378734   0.6235886580387  0.004349337378734   396.8309868051
0.002928936786464   0.002928936786463   0.002928936786463   0.3467952576409  0.002928936786463   488.976382689
0.0001458208281248  0.0001458208281499  0.0001458208281499  0.9601001354313  0.0001458208281409  665.3317152907
1.593588680453e-07  1.593588684743e-07  1.593588684755e-07  0.9989479829423  1.593588661973e-07  675.8691141106
7.991316370683e-12  7.990632294946e-12  7.990631661844e-12  0.999949860274   7.990813072693e-12  675.8826828943
Optimization terminated successfully.
         Current function value: 675.882683
         Iterations: 8
Primal Feasibility  Dual Feasibility    Duality Gap         Step             Path Parameter      Objective
1.0                 1.0                 1.0                 -                1.0                 759.5666413636
0.06104093664241    0.06104093664241    0.0610409366424     0.9444854714867  0.06104093664241    386.9519025016
0.01481511454072    0.01481511454072    0.01481511454072    0.7670183371838  0.01481511454072    2.643589195991
0.005238876768342   0.005238876768342   0.005238876768341   0.6766638370173  0.005238876768342   11.29316821413
0.001567745187159   0.001567745187159   0.001567745187159   0.7265977484523  0.001567745187159   12.32846209581
0.0001184080214619  0.0001184080214619  0.0001184080214619  0.9328477549533  0.0001184080214619  6.401954889258
1.02649122158e-05   1.026491221577e-05  1.026491221577e-05  0.959846524846   1.026491221577e-05  6.140180339771
2.564197796788e-08  2.564197949727e-08  2.564197949173e-08  0.9988486248561  2.56419794805e-08   6.06660177389
1.282243371032e-12  1.282166817908e-12  1.282163560611e-12  0.9999499974762  1.282163552715e-12  6.066591804033
Optimization terminated successfully.
         Current function value: 6.066592
         Iterations: 8
\end{sphinxVerbatim}
}

{

\kern-\sphinxverbatimsmallskipamount\kern-\baselineskip
\kern+\FrameHeightAdjust\kern-\fboxrule
\vspace{\nbsphinxcodecellspacing}

\sphinxsetup{VerbatimColor={named}{nbsphinx-stderr}}
\sphinxsetup{VerbatimBorderColor={named}{nbsphinx-code-border}}
\begin{sphinxVerbatim}[commandchars=\\\{\}]
-- 2020-11-09 11:32:46 - muse.mca - WARNING
Check growth constraints for wind.

\end{sphinxVerbatim}
}

{

\kern-\sphinxverbatimsmallskipamount\kern-\baselineskip
\kern+\FrameHeightAdjust\kern-\fboxrule
\vspace{\nbsphinxcodecellspacing}

\sphinxsetup{VerbatimColor={named}{white}}
\sphinxsetup{VerbatimBorderColor={named}{nbsphinx-code-border}}
\begin{sphinxVerbatim}[commandchars=\\\{\}]
Primal Feasibility  Dual Feasibility    Duality Gap         Step             Path Parameter      Objective
1.0                 1.0                 1.0                 -                1.0                 454.6784294315
0.1862232334687     0.1862232334687     0.1862232334687     0.8174567476576  0.1862232334687     282.4632604722
0.02870747975048    0.02870747975048    0.02870747975048    0.8875534078522  0.02870747975048    46.09791383177
0.009705128410004   0.009705128410004   0.009705128410004   0.6723641007507  0.009705128410004   35.13402203751
0.007693915131578   0.007693915131578   0.007693915131578   0.2197837130033  0.007693915131578   80.93881539206
0.001699044794839   0.001699044794837   0.001699044794837   0.8250841182677  0.00169904479484    298.2653882331
0.0005650008980062  0.0005650008980057  0.0005650008980057  0.678565948553   0.0005650008980065  327.975809321
0.0002196033953225  0.0002196033953223  0.0002196033953223  0.6469207178836  0.0002196033953226  358.4708363196
2.579861478512e-06  2.579861478501e-06  2.579861478502e-06  0.9887803221269  2.579861478497e-06  368.5514208652
1.310502062448e-10  1.310502049574e-10  1.31050204896e-10   0.9999492039366  1.310502128569e-10  368.6632645094
Optimization terminated successfully.
         Current function value: 368.663265
         Iterations: 9
Primal Feasibility  Dual Feasibility    Duality Gap         Step             Path Parameter      Objective
1.0                 1.0                 1.0                 -                1.0                 454.6784294315
0.06717002148527    0.06717002148527    0.06717002148527    0.939336263039   0.06717002148527    183.2225626364
0.0335506482529     0.0335506482529     0.0335506482529     0.5230681149682  0.0335506482529     256.2098343218
0.01040420231972    0.01040420231972    0.01040420231972    0.6975119886622  0.01040420231972    178.3301859411
0.004349337378734   0.004349337378734   0.004349337378734   0.6235886580387  0.004349337378734   396.8309868051
0.002928936786464   0.002928936786463   0.002928936786463   0.3467952576409  0.002928936786463   488.976382689
0.0001458208281248  0.0001458208281499  0.0001458208281499  0.9601001354313  0.0001458208281409  665.3317152907
1.593588680453e-07  1.593588684743e-07  1.593588684755e-07  0.9989479829423  1.593588661973e-07  675.8691141106
7.991316370683e-12  7.990632294946e-12  7.990631661844e-12  0.999949860274   7.990813072693e-12  675.8826828943
Optimization terminated successfully.
         Current function value: 675.882683
         Iterations: 8
Primal Feasibility  Dual Feasibility    Duality Gap         Step             Path Parameter      Objective
1.0                 1.0                 1.0                 -                1.0                 759.5666413636
0.06104093664241    0.06104093664241    0.0610409366424     0.9444854714867  0.06104093664241    386.9519025016
0.01481511454072    0.01481511454072    0.01481511454072    0.7670183371838  0.01481511454072    2.643589195991
0.005238876768342   0.005238876768342   0.005238876768341   0.6766638370173  0.005238876768342   11.29316821413
0.001567745187159   0.001567745187159   0.001567745187159   0.7265977484523  0.001567745187159   12.32846209581
0.0001184080214619  0.0001184080214619  0.0001184080214619  0.9328477549533  0.0001184080214619  6.401954889258
1.02649122158e-05   1.026491221577e-05  1.026491221577e-05  0.959846524846   1.026491221577e-05  6.140180339771
2.564197796788e-08  2.564197949727e-08  2.564197949173e-08  0.9988486248561  2.56419794805e-08   6.06660177389
1.282243371032e-12  1.282166817908e-12  1.282163560611e-12  0.9999499974762  1.282163552715e-12  6.066591804033
Optimization terminated successfully.
         Current function value: 6.066592
         Iterations: 8
\end{sphinxVerbatim}
}

{

\kern-\sphinxverbatimsmallskipamount\kern-\baselineskip
\kern+\FrameHeightAdjust\kern-\fboxrule
\vspace{\nbsphinxcodecellspacing}

\sphinxsetup{VerbatimColor={named}{nbsphinx-stderr}}
\sphinxsetup{VerbatimBorderColor={named}{nbsphinx-code-border}}
\begin{sphinxVerbatim}[commandchars=\\\{\}]
-- 2020-11-09 11:32:55 - muse.mca - WARNING
Check growth constraints for wind.

\end{sphinxVerbatim}
}

{

\kern-\sphinxverbatimsmallskipamount\kern-\baselineskip
\kern+\FrameHeightAdjust\kern-\fboxrule
\vspace{\nbsphinxcodecellspacing}

\sphinxsetup{VerbatimColor={named}{white}}
\sphinxsetup{VerbatimBorderColor={named}{nbsphinx-code-border}}
\begin{sphinxVerbatim}[commandchars=\\\{\}]
Primal Feasibility  Dual Feasibility    Duality Gap         Step             Path Parameter      Objective
1.0                 1.0                 1.0                 -                1.0                 506.4464461439
0.1717574295871     0.1717574295871     0.1717574295871     0.8319963473544  0.1717574295871     268.7406800963
0.06638577511032    0.06638577511032    0.06638577511033    0.6490768395246  0.06638577511033    376.5617800925
0.0108119991109     0.01081199911089    0.01081199911089    0.8457086649906  0.01081199911089    156.44831956
0.006166341998836   0.006166341998834   0.006166341998835   0.4575243211097  0.006166341998835   243.5697308399
0.0005157722017999  0.0005157722017955  0.0005157722017955  0.9303097249327  0.0005157722018036  353.933860755
7.752032087913e-08  7.75203208986e-08   7.752032089586e-08  0.9998584907938  7.75203208902e-08   354.2607816225
1.875610793654e-11  1.875609893145e-11  1.875610020548e-11  0.9997580492617  1.875613913502e-11  354.2617798007
Optimization terminated successfully.
         Current function value: 354.261780
         Iterations: 7
Primal Feasibility  Dual Feasibility    Duality Gap         Step             Path Parameter      Objective
1.0                 1.0                 1.0                 -                1.0                 506.4464461439
0.1079826576971     0.1079826576971     0.1079826576971     0.8981297430914  0.1079826576971     201.9066839097
0.04065945390789    0.04065945390789    0.04065945390789    0.6593890232973  0.04065945390789    434.9407529189
0.01889246737726    0.01889246737727    0.01889246737727    0.5585864512458  0.01889246737727    712.8967272059
0.001989080291767   0.001989080291864   0.001989080291864   0.9101975479096  0.001989080291929   1386.629312789
2.319893722758e-06  2.31989372302e-06   2.319893722997e-06  0.9988740155508  2.319893723235e-06  1475.934762389
2.096073130738e-10  2.096071489352e-10  2.096071783066e-10  0.9999096479509  2.096074566229e-10  1476.090721225
Optimization terminated successfully.
         Current function value: 1476.090721
         Iterations: 6
Primal Feasibility  Dual Feasibility    Duality Gap         Step             Path Parameter      Objective
1.0                 1.0                 1.0                 -                1.0                 889.6273141593
0.0609484122893     0.06094841228931    0.0609484122893     0.9446306809213  0.06094841228931    507.5981904767
0.01403712586721    0.01403712586721    0.01403712586721    0.7774983036897  0.01403712586721    2.560790425808
0.004916790985586   0.004916790985586   0.004916790985586   0.6812990636392  0.004916790985587   14.46599029197
0.001087493288514   0.001087493288517   0.001087493288517   0.8126332143986  0.001087493288517   17.14374653765
0.0004935333934202  0.0004935333934214  0.0004935333934214  0.5558253633026  0.0004935333934214  10.54120251124
0.0001121697661281  0.000112169766129   0.000112169766129   0.817643306912   0.000112169766129   4.9896729167
2.712089127881e-05  2.712089127903e-05  2.712089127903e-05  0.7970501144483  2.712089127903e-05  4.324997088558
1.122718419375e-07  1.122718402016e-07  1.122718401965e-07  0.99597588406    1.122718401941e-07  4.000800850778
5.614947446286e-12  5.615029242581e-12  5.615032519692e-12  0.9999499875898  5.615033454619e-12  3.999950693022
2.903000170299e-16  2.886166226175e-16  2.808194717321e-16  0.9999499790221  2.807865903642e-16  3.999950650496
Optimization terminated successfully.
         Current function value: 3.999951
         Iterations: 10
Primal Feasibility  Dual Feasibility    Duality Gap         Step             Path Parameter      Objective
1.0                 1.0                 1.0                 -                1.0                 889.6273141593
0.06073185177466    0.06073185177467    0.06073185177466    0.9450179496586  0.06073185177466    760.7133777158
0.00617243956643    0.006172439566431   0.006172439566431   0.8994093979063  0.006172439566431   1.525116036565
0.002290102597025   0.002290102597025   0.002290102597025   0.6678934240013  0.002290102597025   13.23877514086
0.0007463324176618  0.0007463324176611  0.0007463324176611  0.7077017235652  0.0007463324176608  10.86935158651
0.0003923306880488  0.0003923306880484  0.0003923306880484  0.488114542318   0.0003923306880483  6.236673613644
2.432640260798e-05  2.432640260791e-05  2.432640260791e-05  1.0              2.43264026079e-05   0.3563903987643
6.281868977076e-08  6.281868977357e-08  6.281868977105e-08  0.9991491899851  6.281868977102e-08  0.0001119022044274
3.378349078595e-12  3.378347740195e-12  3.378348155213e-12  0.9999462206524  3.378348155213e-12  6.056769044982e-09
2.306615405556e-13  2.306653965292e-13  2.306614266427e-13  0.9320769632577  2.306614266427e-13  4.595744934239e-10
Optimization terminated successfully.
         Current function value: 0.000000
         Iterations: 9
\end{sphinxVerbatim}
}

{

\kern-\sphinxverbatimsmallskipamount\kern-\baselineskip
\kern+\FrameHeightAdjust\kern-\fboxrule
\vspace{\nbsphinxcodecellspacing}

\sphinxsetup{VerbatimColor={named}{nbsphinx-stderr}}
\sphinxsetup{VerbatimBorderColor={named}{nbsphinx-code-border}}
\begin{sphinxVerbatim}[commandchars=\\\{\}]
-- 2020-11-09 11:33:05 - muse.mca - WARNING
Check growth constraints for wind.

\end{sphinxVerbatim}
}

{

\kern-\sphinxverbatimsmallskipamount\kern-\baselineskip
\kern+\FrameHeightAdjust\kern-\fboxrule
\vspace{\nbsphinxcodecellspacing}

\sphinxsetup{VerbatimColor={named}{white}}
\sphinxsetup{VerbatimBorderColor={named}{nbsphinx-code-border}}
\begin{sphinxVerbatim}[commandchars=\\\{\}]
Primal Feasibility  Dual Feasibility    Duality Gap         Step             Path Parameter      Objective
1.0                 1.0                 1.0                 -                1.0                 506.4464461439
0.1717574295871     0.1717574295871     0.1717574295871     0.8319963473544  0.1717574295871     268.7406800963
0.06638577511032    0.06638577511032    0.06638577511033    0.6490768395246  0.06638577511033    376.5617800925
0.0108119991109     0.01081199911089    0.01081199911089    0.8457086649906  0.01081199911089    156.44831956
0.006166341998836   0.006166341998834   0.006166341998835   0.4575243211097  0.006166341998835   243.5697308399
0.0005157722017999  0.0005157722017955  0.0005157722017955  0.9303097249327  0.0005157722018036  353.933860755
7.752032087913e-08  7.75203208986e-08   7.752032089586e-08  0.9998584907938  7.75203208902e-08   354.2607816225
1.875610793654e-11  1.875609893145e-11  1.875610020548e-11  0.9997580492617  1.875613913502e-11  354.2617798007
Optimization terminated successfully.
         Current function value: 354.261780
         Iterations: 7
Primal Feasibility  Dual Feasibility    Duality Gap         Step             Path Parameter      Objective
1.0                 1.0                 1.0                 -                1.0                 506.4464461439
0.1079826576971     0.1079826576971     0.1079826576971     0.8981297430914  0.1079826576971     201.9066839097
0.04065945390789    0.04065945390789    0.04065945390789    0.6593890232973  0.04065945390789    434.9407529189
0.01889246737726    0.01889246737727    0.01889246737727    0.5585864512458  0.01889246737727    712.8967272059
0.001989080291767   0.001989080291864   0.001989080291864   0.9101975479096  0.001989080291929   1386.629312789
2.319893722758e-06  2.31989372302e-06   2.319893722997e-06  0.9988740155508  2.319893723235e-06  1475.934762389
2.096073130738e-10  2.096071489352e-10  2.096071783066e-10  0.9999096479509  2.096074566229e-10  1476.090721225
Optimization terminated successfully.
         Current function value: 1476.090721
         Iterations: 6
Primal Feasibility  Dual Feasibility    Duality Gap         Step             Path Parameter      Objective
1.0                 1.0                 1.0                 -                1.0                 889.6273141593
0.0609484122893     0.06094841228931    0.0609484122893     0.9446306809213  0.06094841228931    507.5981904767
0.01403712586721    0.01403712586721    0.01403712586721    0.7774983036897  0.01403712586721    2.560790425808
0.004916790985586   0.004916790985586   0.004916790985586   0.6812990636392  0.004916790985587   14.46599029197
0.001087493288514   0.001087493288517   0.001087493288517   0.8126332143986  0.001087493288517   17.14374653765
0.0004935333934202  0.0004935333934214  0.0004935333934214  0.5558253633026  0.0004935333934214  10.54120251124
0.0001121697661281  0.000112169766129   0.000112169766129   0.817643306912   0.000112169766129   4.9896729167
2.712089127881e-05  2.712089127903e-05  2.712089127903e-05  0.7970501144483  2.712089127903e-05  4.324997088558
1.122718419375e-07  1.122718402016e-07  1.122718401965e-07  0.99597588406    1.122718401941e-07  4.000800850778
5.614947446286e-12  5.615029242581e-12  5.615032519692e-12  0.9999499875898  5.615033454619e-12  3.999950693022
2.903000170299e-16  2.886166226175e-16  2.808194717321e-16  0.9999499790221  2.807865903642e-16  3.999950650496
Optimization terminated successfully.
         Current function value: 3.999951
         Iterations: 10
Primal Feasibility  Dual Feasibility    Duality Gap         Step             Path Parameter      Objective
1.0                 1.0                 1.0                 -                1.0                 889.6273141593
0.06073185177466    0.06073185177467    0.06073185177466    0.9450179496586  0.06073185177466    760.7133777158
0.00617243956643    0.006172439566431   0.006172439566431   0.8994093979063  0.006172439566431   1.525116036565
0.002290102597025   0.002290102597025   0.002290102597025   0.6678934240013  0.002290102597025   13.23877514086
0.0007463324176618  0.0007463324176611  0.0007463324176611  0.7077017235652  0.0007463324176608  10.86935158651
0.0003923306880488  0.0003923306880484  0.0003923306880484  0.488114542318   0.0003923306880483  6.236673613644
2.432640260798e-05  2.432640260791e-05  2.432640260791e-05  1.0              2.43264026079e-05   0.3563903987643
6.281868977076e-08  6.281868977357e-08  6.281868977105e-08  0.9991491899851  6.281868977102e-08  0.0001119022044274
3.378349078595e-12  3.378347740195e-12  3.378348155213e-12  0.9999462206524  3.378348155213e-12  6.056769044982e-09
2.306615405556e-13  2.306653965292e-13  2.306614266427e-13  0.9320769632577  2.306614266427e-13  4.595744934239e-10
Optimization terminated successfully.
         Current function value: 0.000000
         Iterations: 9
\end{sphinxVerbatim}
}

{

\kern-\sphinxverbatimsmallskipamount\kern-\baselineskip
\kern+\FrameHeightAdjust\kern-\fboxrule
\vspace{\nbsphinxcodecellspacing}

\sphinxsetup{VerbatimColor={named}{nbsphinx-stderr}}
\sphinxsetup{VerbatimBorderColor={named}{nbsphinx-code-border}}
\begin{sphinxVerbatim}[commandchars=\\\{\}]
-- 2020-11-09 11:33:13 - muse.mca - WARNING
Check growth constraints for wind.

\end{sphinxVerbatim}
}

{

\kern-\sphinxverbatimsmallskipamount\kern-\baselineskip
\kern+\FrameHeightAdjust\kern-\fboxrule
\vspace{\nbsphinxcodecellspacing}

\sphinxsetup{VerbatimColor={named}{white}}
\sphinxsetup{VerbatimBorderColor={named}{nbsphinx-code-border}}
\begin{sphinxVerbatim}[commandchars=\\\{\}]
Primal Feasibility  Dual Feasibility    Duality Gap         Step             Path Parameter      Objective
1.0                 1.0                 1.0                 -                1.0                 568.8949309456
0.1574999940443     0.1574999940443     0.1574999940443     0.8462995490748  0.1574999940443     259.3692323549
0.07523425397025    0.07523425397025    0.07523425397025    0.5547720211262  0.07523425397025    453.5904211877
0.01015218722862    0.01015218722862    0.01015218722862    0.8831324835902  0.01015218722862    172.5109988087
0.005597873343747   0.005597873343747   0.005597873343747   0.4782155730769  0.005597873343747   257.1812475882
0.0004682435100659  0.0004682435100634  0.0004682435100634  0.9301215087302  0.000468243510069   357.0084554784
6.611502592752e-08  6.611502591615e-08  6.611502591873e-08  0.999863385888   6.611502594636e-08  355.7877874388
7.555502313764e-12  7.555508723405e-12  7.555505999397e-12  0.9998857217433  7.555493074812e-12  355.7884668282
Optimization terminated successfully.
         Current function value: 355.788467
         Iterations: 7
Primal Feasibility  Dual Feasibility    Duality Gap         Step             Path Parameter      Objective
1.0                 1.0                 1.0                 -                1.0                 568.8949309456
0.07655552550188    0.07655552550188    0.07655552550188    0.9326533055761  0.07655552550188    158.9571104913
0.02915610316818    0.02915610316819    0.02915610316819    0.6583861263339  0.02915610316818    366.3114833866
0.01377238842158    0.01377238842158    0.01377238842158    0.5494198953479  0.01377238842158    509.1861831748
0.002113311162461   0.002113311162461   0.002113311162461   0.8672288939288  0.002113311162461   954.4881944043
1.802801647426e-06  1.802801647524e-06  1.802801647519e-06  0.9992160883437  1.802801646136e-06  1007.960176974
1.128093610417e-10  1.128094830078e-10  1.128094936008e-10  0.9999374254592  1.128097428918e-10  1008.067349764
Optimization terminated successfully.
         Current function value: 1008.067350
         Iterations: 6
Primal Feasibility  Dual Feasibility    Duality Gap         Step             Path Parameter      Objective
1.0                 1.0                 1.0                 -                1.0                 1027.252465794
0.06092848901518    0.06092848901523    0.06092848901522    0.9446907062507  0.06092848901524    639.5911028223
0.01301865096831    0.01301865096832    0.01301865096832    0.7926600108064  0.01301865096833    2.590461755952
0.00456587722928    0.004565877229283   0.004565877229283   0.6815721939632  0.004565877229284   18.42445056578
0.001192620177961   0.001192620177954   0.001192620177954   0.7720888734274  0.001192620177954   21.2766378272
0.0004150762782539  0.0004150762782515  0.0004150762782515  0.6551231023534  0.0004150762782516  10.90714854956
8.30397207983e-05   8.303972079784e-05  8.303972079783e-05  0.8551088375548  8.303972079786e-05  4.883155784925
2.345269716473e-05  2.345269716461e-05  2.34526971646e-05   0.7548298132573  2.345269716461e-05  4.346885094868
1.371159312911e-07  1.371159321168e-07  1.371159321194e-07  0.994331926271   1.371159321123e-07  4.001183377585
6.859845575537e-12  6.859856824223e-12  6.859863126594e-12  0.9999499723157  6.85986409236e-12   3.999950686436
6.215051942361e-16  3.43521130371e-16   3.430265101783e-16  0.9999499985889  3.431205195905e-16  3.999950624744
Optimization terminated successfully.
         Current function value: 3.999951
         Iterations: 10
Primal Feasibility  Dual Feasibility    Duality Gap         Step             Path Parameter      Objective
1.0                 1.0                 1.0                 -                1.0                 513.6262328969
0.06112045297154    0.06112045297154    0.06112045297151    0.9426671718798  0.06112045297154    286.961037666
0.0186024953335     0.0186024953335     0.01860249533349    0.7059824756003  0.0186024953335     1.911789377935
0.006446621823764   0.006446621823763   0.00644662182376    0.6910883581126  0.006446621823763   10.97980543166
0.0005501537566849  0.0005501537566955  0.0005501537566952  0.9377765193638  0.0005501537566949  10.00501917633
4.007383875704e-06  4.007383875795e-06  4.007383875804e-06  0.9951127586026  4.007383875797e-06  6.687770636962
2.007342726257e-10  2.007343535262e-10  2.007343506552e-10  0.9999499088827  2.007343503785e-10  6.666585422706
2.161448037787e-10  2.007343535257e-10  2.007343507631e-10  2.456113364069e-461.961302857585e-10  6.666585422564
6.009588090216e-12  2.723883580422e-12  2.723885134159e-12  0.9866729468426  2.685910112544e-12  6.666584374146
Optimization terminated successfully.
         Current function value: 6.666584
         Iterations: 8
\end{sphinxVerbatim}
}

{

\kern-\sphinxverbatimsmallskipamount\kern-\baselineskip
\kern+\FrameHeightAdjust\kern-\fboxrule
\vspace{\nbsphinxcodecellspacing}

\sphinxsetup{VerbatimColor={named}{nbsphinx-stderr}}
\sphinxsetup{VerbatimBorderColor={named}{nbsphinx-code-border}}
\begin{sphinxVerbatim}[commandchars=\\\{\}]
/Users/alexkell/Documents/SGI/1-examples/example\_model/model/Results/muse/src/muse/investments.py:325: OptimizeWarning: Solving system with option 'cholesky':True failed. It is normal for this to happen occasionally, especially as the solution is approached. However, if you see this frequently, consider setting option 'cholesky' to False.
  res = linprog(**adapter.kwargs, options=dict(disp=True))
/Users/alexkell/Documents/SGI/1-examples/example\_model/model/Results/muse/src/muse/investments.py:325: OptimizeWarning: Solving system with option 'sym\_pos':True failed. It is normal for this to happen occasionally, especially as the solution is approached. However, if you see this frequently, consider setting option 'sym\_pos' to False.
  res = linprog(**adapter.kwargs, options=dict(disp=True))
/Users/alexkell/anaconda3/lib/python3.8/site-packages/scipy/optimize/\_linprog\_ip.py:116: LinAlgWarning: Ill-conditioned matrix (rcond=1.62544e-36): result may not be accurate.
  return sp.linalg.solve(M, r, sym\_pos=sym\_pos)
-- 2020-11-09 11:33:25 - muse.mca - WARNING
Check growth constraints for wind.

\end{sphinxVerbatim}
}

{

\kern-\sphinxverbatimsmallskipamount\kern-\baselineskip
\kern+\FrameHeightAdjust\kern-\fboxrule
\vspace{\nbsphinxcodecellspacing}

\sphinxsetup{VerbatimColor={named}{white}}
\sphinxsetup{VerbatimBorderColor={named}{nbsphinx-code-border}}
\begin{sphinxVerbatim}[commandchars=\\\{\}]
Primal Feasibility  Dual Feasibility    Duality Gap         Step             Path Parameter      Objective
1.0                 1.0                 1.0                 -                1.0                 568.8949309456
0.1574999940443     0.1574999940443     0.1574999940443     0.8462995490748  0.1574999940443     259.3692323549
0.07523425397025    0.07523425397025    0.07523425397025    0.5547720211262  0.07523425397025    453.5904211877
0.01015218722862    0.01015218722862    0.01015218722862    0.8831324835902  0.01015218722862    172.5109988087
0.005597873343747   0.005597873343747   0.005597873343747   0.4782155730769  0.005597873343747   257.1812475882
0.0004682435100659  0.0004682435100634  0.0004682435100634  0.9301215087302  0.000468243510069   357.0084554784
6.611502592752e-08  6.611502591615e-08  6.611502591873e-08  0.999863385888   6.611502594636e-08  355.7877874388
7.555502313764e-12  7.555508723405e-12  7.555505999397e-12  0.9998857217433  7.555493074812e-12  355.7884668282
Optimization terminated successfully.
         Current function value: 355.788467
         Iterations: 7
Primal Feasibility  Dual Feasibility    Duality Gap         Step             Path Parameter      Objective
1.0                 1.0                 1.0                 -                1.0                 568.8949309456
0.07655552550188    0.07655552550188    0.07655552550188    0.9326533055761  0.07655552550188    158.9571104913
0.02915610316818    0.02915610316819    0.02915610316819    0.6583861263339  0.02915610316818    366.3114833866
0.01377238842158    0.01377238842158    0.01377238842158    0.5494198953479  0.01377238842158    509.1861831748
0.002113311162461   0.002113311162461   0.002113311162461   0.8672288939288  0.002113311162461   954.4881944043
1.802801647426e-06  1.802801647524e-06  1.802801647519e-06  0.9992160883437  1.802801646136e-06  1007.960176974
1.128093610417e-10  1.128094830078e-10  1.128094936008e-10  0.9999374254592  1.128097428918e-10  1008.067349764
Optimization terminated successfully.
         Current function value: 1008.067350
         Iterations: 6
Primal Feasibility  Dual Feasibility    Duality Gap         Step             Path Parameter      Objective
1.0                 1.0                 1.0                 -                1.0                 1027.252465794
0.06092848901518    0.06092848901523    0.06092848901522    0.9446907062507  0.06092848901524    639.5911028223
0.01301865096831    0.01301865096832    0.01301865096832    0.7926600108064  0.01301865096833    2.590461755952
0.00456587722928    0.004565877229283   0.004565877229283   0.6815721939632  0.004565877229284   18.42445056578
0.001192620177961   0.001192620177954   0.001192620177954   0.7720888734274  0.001192620177954   21.2766378272
0.0004150762782539  0.0004150762782515  0.0004150762782515  0.6551231023534  0.0004150762782516  10.90714854956
8.30397207983e-05   8.303972079784e-05  8.303972079783e-05  0.8551088375548  8.303972079786e-05  4.883155784925
2.345269716473e-05  2.345269716461e-05  2.34526971646e-05   0.7548298132573  2.345269716461e-05  4.346885094868
1.371159312911e-07  1.371159321168e-07  1.371159321194e-07  0.994331926271   1.371159321123e-07  4.001183377585
6.859845575537e-12  6.859856824223e-12  6.859863126594e-12  0.9999499723157  6.85986409236e-12   3.999950686436
6.215051942361e-16  3.43521130371e-16   3.430265101783e-16  0.9999499985889  3.431205195905e-16  3.999950624744
Optimization terminated successfully.
         Current function value: 3.999951
         Iterations: 10
Primal Feasibility  Dual Feasibility    Duality Gap         Step             Path Parameter      Objective
1.0                 1.0                 1.0                 -                1.0                 513.6262328969
0.06112045297154    0.06112045297154    0.06112045297151    0.9426671718798  0.06112045297154    286.961037666
0.0186024953335     0.0186024953335     0.01860249533349    0.7059824756003  0.0186024953335     1.911789377935
0.006446621823764   0.006446621823763   0.00644662182376    0.6910883581126  0.006446621823763   10.97980543166
0.0005501537566849  0.0005501537566955  0.0005501537566952  0.9377765193638  0.0005501537566949  10.00501917633
4.007383875704e-06  4.007383875795e-06  4.007383875804e-06  0.9951127586026  4.007383875797e-06  6.687770636962
2.007342726257e-10  2.007343535262e-10  2.007343506552e-10  0.9999499088827  2.007343503785e-10  6.666585422706
2.161448037787e-10  2.007343535257e-10  2.007343507631e-10  2.456113364069e-461.961302857585e-10  6.666585422564
6.009588090216e-12  2.723883580422e-12  2.723885134159e-12  0.9866729468426  2.685910112544e-12  6.666584374146
Optimization terminated successfully.
         Current function value: 6.666584
         Iterations: 8
\end{sphinxVerbatim}
}

{

\kern-\sphinxverbatimsmallskipamount\kern-\baselineskip
\kern+\FrameHeightAdjust\kern-\fboxrule
\vspace{\nbsphinxcodecellspacing}

\sphinxsetup{VerbatimColor={named}{nbsphinx-stderr}}
\sphinxsetup{VerbatimBorderColor={named}{nbsphinx-code-border}}
\begin{sphinxVerbatim}[commandchars=\\\{\}]
/Users/alexkell/Documents/SGI/1-examples/example\_model/model/Results/muse/src/muse/investments.py:325: OptimizeWarning: Solving system with option 'cholesky':True failed. It is normal for this to happen occasionally, especially as the solution is approached. However, if you see this frequently, consider setting option 'cholesky' to False.
  res = linprog(**adapter.kwargs, options=dict(disp=True))
/Users/alexkell/Documents/SGI/1-examples/example\_model/model/Results/muse/src/muse/investments.py:325: OptimizeWarning: Solving system with option 'sym\_pos':True failed. It is normal for this to happen occasionally, especially as the solution is approached. However, if you see this frequently, consider setting option 'sym\_pos' to False.
  res = linprog(**adapter.kwargs, options=dict(disp=True))
/Users/alexkell/anaconda3/lib/python3.8/site-packages/scipy/optimize/\_linprog\_ip.py:116: LinAlgWarning: Ill-conditioned matrix (rcond=1.62544e-36): result may not be accurate.
  return sp.linalg.solve(M, r, sym\_pos=sym\_pos)
-- 2020-11-09 11:33:37 - muse.mca - WARNING
Check growth constraints for wind.

\end{sphinxVerbatim}
}

{

\kern-\sphinxverbatimsmallskipamount\kern-\baselineskip
\kern+\FrameHeightAdjust\kern-\fboxrule
\vspace{\nbsphinxcodecellspacing}

\sphinxsetup{VerbatimColor={named}{white}}
\sphinxsetup{VerbatimBorderColor={named}{nbsphinx-code-border}}
\begin{sphinxVerbatim}[commandchars=\\\{\}]
Primal Feasibility  Dual Feasibility    Duality Gap         Step             Path Parameter      Objective
1.0                 1.0                 1.0                 -                1.0                 647.4402248427
0.1042702416317     0.1042702416317     0.1042702416317     0.9044651903652  0.1042702416317     197.7436068274
0.04330783922975    0.04330783922975    0.04330783922975    0.6213338928986  0.04330783922975    554.8458695929
0.02029339907042    0.02029339907042    0.02029339907042    0.5546164251713  0.02029339907042    887.503418141
0.002059630009005   0.002059630008646   0.002059630008646   0.9123945707354  0.002059630008855   1773.497308596
1.96164830749e-06   1.961648306624e-06  1.961648306623e-06  0.9995475315428  1.96164830693e-06   1899.795955905
1.183904657193e-10  1.183907029537e-10  1.183907180056e-10  0.9999396473343  1.183903667831e-10  1900.055708261
Optimization terminated successfully.
         Current function value: 1900.055708
         Iterations: 6
Primal Feasibility  Dual Feasibility    Duality Gap         Step             Path Parameter      Objective
1.0                 1.0                 1.0                 -                1.0                 1199.542590081
0.06080970203684    0.06080970203684    0.06080970203685    0.9448463893022  0.06080970203684    802.3521760538
0.01206790508619    0.01206790508619    0.01206790508619    0.8067148881887  0.01206790508619    2.242531350203
0.004235304009507   0.004235304009507   0.004235304009507   0.6829892140296  0.004235304009507   22.93309435174
0.001076107298274   0.001076107298272   0.001076107298272   0.7811880414723  0.001076107298272   27.34472817154
0.0001381467026759  0.0001381467026756  0.0001381467026756  0.9534954297498  0.0001381467026756  2.164474733514
0.0001283068196922  0.0001283068196919  0.0001283068196919  0.07280278667804 0.0001283068196919  2.01175576967
1.709535377769e-06  1.709535377765e-06  1.709535377765e-06  0.9870423816225  1.709535377765e-06  0.04804313560572
9.48541628326e-11   9.485418648369e-11  9.485418526931e-11  0.9999445934679  9.485418526932e-11  2.66445492392e-06
8.795947527601e-12  8.795912717956e-12  8.795907275833e-12  0.9086823994024  8.795907275834e-12  2.473302892196e-07
8.361396439476e-12  8.361363544519e-12  8.361358288329e-12  0.05441434596209 8.361358288329e-12  2.360712463483e-07
1.014674559722e-12  1.014680661495e-12  1.014676648034e-12  0.8808930741304  1.014676648034e-12  2.943765656853e-08
7.649517191012e-13  7.649501663219e-13  7.649518441425e-13  0.266685424728   7.649518441426e-13  2.292855842523e-08
2.251056337644e-13  2.251034423868e-13  2.250976823208e-13  0.7199831364733  2.250976823208e-13  9.225506253453e-09
Optimization terminated successfully.
         Current function value: 0.000000
         Iterations: 13
Primal Feasibility  Dual Feasibility    Duality Gap         Step             Path Parameter      Objective
1.0                 1.0                 1.0                 -                1.0                 599.7712950406
0.06108522397884    0.06108522397884    0.06108522397884    0.9426921917133  0.06108522397884    370.5338700223
0.01718424163858    0.01718424163858    0.01718424163858    0.7269400839096  0.01718424163858    1.816590685562
0.006448030539302   0.006448030539302   0.006448030539302   0.6618055567806  0.006448030539302   13.33233954634
0.0007451641812888  0.0007451641813056  0.0007451641813056  0.9128367304104  0.0007451641813056  13.46011423965
0.0002051849835872  0.0002051849835913  0.0002051849835913  0.726248232922   0.0002051849835914  8.092503300126
3.421589486985e-06  3.421589487287e-06  3.421589487279e-06  1.0              3.421589487245e-06  5.943322696325
2.883371020626e-10  2.88337212738e-10   2.883372076584e-10  0.999916005146   2.883372071457e-10  5.933260704906
1.437342489421e-14  1.441626748189e-14  1.441700316141e-14  0.999949999407   1.441794244359e-14  5.933260075848
Optimization terminated successfully.
         Current function value: 5.933260
         Iterations: 8
\end{sphinxVerbatim}
}

{

\kern-\sphinxverbatimsmallskipamount\kern-\baselineskip
\kern+\FrameHeightAdjust\kern-\fboxrule
\vspace{\nbsphinxcodecellspacing}

\sphinxsetup{VerbatimColor={named}{nbsphinx-stderr}}
\sphinxsetup{VerbatimBorderColor={named}{nbsphinx-code-border}}
\begin{sphinxVerbatim}[commandchars=\\\{\}]
-- 2020-11-09 11:33:46 - muse.mca - WARNING
Check growth constraints for wind.

\end{sphinxVerbatim}
}

{

\kern-\sphinxverbatimsmallskipamount\kern-\baselineskip
\kern+\FrameHeightAdjust\kern-\fboxrule
\vspace{\nbsphinxcodecellspacing}

\sphinxsetup{VerbatimColor={named}{white}}
\sphinxsetup{VerbatimBorderColor={named}{nbsphinx-code-border}}
\begin{sphinxVerbatim}[commandchars=\\\{\}]
Primal Feasibility  Dual Feasibility    Duality Gap         Step             Path Parameter      Objective
1.0                 1.0                 1.0                 -                1.0                 647.4402248427
0.1042702416317     0.1042702416317     0.1042702416317     0.9044651903652  0.1042702416317     197.7436068274
0.04330783922975    0.04330783922975    0.04330783922975    0.6213338928986  0.04330783922975    554.8458695929
0.02029339907042    0.02029339907042    0.02029339907042    0.5546164251713  0.02029339907042    887.503418141
0.002059630009005   0.002059630008646   0.002059630008646   0.9123945707354  0.002059630008855   1773.497308596
1.96164830749e-06   1.961648306624e-06  1.961648306623e-06  0.9995475315428  1.96164830693e-06   1899.795955905
1.183904657193e-10  1.183907029537e-10  1.183907180056e-10  0.9999396473343  1.183903667831e-10  1900.055708261
Optimization terminated successfully.
         Current function value: 1900.055708
         Iterations: 6
Primal Feasibility  Dual Feasibility    Duality Gap         Step             Path Parameter      Objective
1.0                 1.0                 1.0                 -                1.0                 1199.542590081
0.06080970203684    0.06080970203684    0.06080970203685    0.9448463893022  0.06080970203684    802.3521760538
0.01206790508619    0.01206790508619    0.01206790508619    0.8067148881887  0.01206790508619    2.242531350203
0.004235304009507   0.004235304009507   0.004235304009507   0.6829892140296  0.004235304009507   22.93309435174
0.001076107298274   0.001076107298272   0.001076107298272   0.7811880414723  0.001076107298272   27.34472817154
0.0001381467026759  0.0001381467026756  0.0001381467026756  0.9534954297498  0.0001381467026756  2.164474733514
0.0001283068196922  0.0001283068196919  0.0001283068196919  0.07280278667804 0.0001283068196919  2.01175576967
1.709535377769e-06  1.709535377765e-06  1.709535377765e-06  0.9870423816225  1.709535377765e-06  0.04804313560572
9.48541628326e-11   9.485418648369e-11  9.485418526931e-11  0.9999445934679  9.485418526932e-11  2.66445492392e-06
8.795947527601e-12  8.795912717956e-12  8.795907275833e-12  0.9086823994024  8.795907275834e-12  2.473302892196e-07
8.361396439476e-12  8.361363544519e-12  8.361358288329e-12  0.05441434596209 8.361358288329e-12  2.360712463483e-07
1.014674559722e-12  1.014680661495e-12  1.014676648034e-12  0.8808930741304  1.014676648034e-12  2.943765656853e-08
7.649517191012e-13  7.649501663219e-13  7.649518441425e-13  0.266685424728   7.649518441426e-13  2.292855842523e-08
2.251056337644e-13  2.251034423868e-13  2.250976823208e-13  0.7199831364733  2.250976823208e-13  9.225506253453e-09
Optimization terminated successfully.
         Current function value: 0.000000
         Iterations: 13
Primal Feasibility  Dual Feasibility    Duality Gap         Step             Path Parameter      Objective
1.0                 1.0                 1.0                 -                1.0                 599.7712950406
0.06108522397884    0.06108522397884    0.06108522397884    0.9426921917133  0.06108522397884    370.5338700223
0.01718424163858    0.01718424163858    0.01718424163858    0.7269400839096  0.01718424163858    1.816590685562
0.006448030539302   0.006448030539302   0.006448030539302   0.6618055567806  0.006448030539302   13.33233954634
0.0007451641812888  0.0007451641813056  0.0007451641813056  0.9128367304104  0.0007451641813056  13.46011423965
0.0002051849835872  0.0002051849835913  0.0002051849835913  0.726248232922   0.0002051849835914  8.092503300126
3.421589486985e-06  3.421589487287e-06  3.421589487279e-06  1.0              3.421589487245e-06  5.943322696325
2.883371020626e-10  2.88337212738e-10   2.883372076584e-10  0.999916005146   2.883372071457e-10  5.933260704906
1.437342489421e-14  1.441626748189e-14  1.441700316141e-14  0.999949999407   1.441794244359e-14  5.933260075848
Optimization terminated successfully.
         Current function value: 5.933260
         Iterations: 8
\end{sphinxVerbatim}
}

{

\kern-\sphinxverbatimsmallskipamount\kern-\baselineskip
\kern+\FrameHeightAdjust\kern-\fboxrule
\vspace{\nbsphinxcodecellspacing}

\sphinxsetup{VerbatimColor={named}{nbsphinx-stderr}}
\sphinxsetup{VerbatimBorderColor={named}{nbsphinx-code-border}}
\begin{sphinxVerbatim}[commandchars=\\\{\}]
-- 2020-11-09 11:33:58 - muse.mca - WARNING
Check growth constraints for wind.

\end{sphinxVerbatim}
}

And we can check the parameters were used accordingly:

{
\sphinxsetup{VerbatimColor={named}{nbsphinx-code-bg}}
\sphinxsetup{VerbatimBorderColor={named}{nbsphinx-code-border}}
\begin{sphinxVerbatim}[commandchars=\\\{\}]
\llap{\color{nbsphinxin}[9]:\,\hspace{\fboxrule}\hspace{\fboxsep}}\PYG{n}{all\PYGZus{}txt\PYGZus{}files} \PYG{o}{=} \PYG{n+nb}{sorted}\PYG{p}{(}\PYG{p}{(}\PYG{n}{Path}\PYG{p}{(}\PYG{p}{)} \PYG{o}{/} \PYG{l+s+s2}{\PYGZdq{}}\PYG{l+s+s2}{Results}\PYG{l+s+s2}{\PYGZdq{}}\PYG{p}{)}\PYG{o}{.}\PYG{n}{glob}\PYG{p}{(}\PYG{l+s+s2}{\PYGZdq{}}\PYG{l+s+s2}{Residential*.txt}\PYG{l+s+s2}{\PYGZdq{}}\PYG{p}{)}\PYG{p}{)}
\PYG{k}{assert} \PYG{n+nb}{len}\PYG{p}{(}\PYG{n}{all\PYGZus{}txt\PYGZus{}files}\PYG{p}{)} \PYG{o}{==} \PYG{l+m+mi}{7}
\PYG{k}{assert} \PYG{l+s+s2}{\PYGZdq{}}\PYG{l+s+s2}{Hello, you!}\PYG{l+s+s2}{\PYGZdq{}} \PYG{o+ow}{in} \PYG{n}{all\PYGZus{}txt\PYGZus{}files}\PYG{p}{[}\PYG{l+m+mi}{0}\PYG{p}{]}\PYG{o}{.}\PYG{n}{read\PYGZus{}text}\PYG{p}{(}\PYG{p}{)}
\PYG{n}{all\PYGZus{}txt\PYGZus{}files}
\end{sphinxVerbatim}
}

{

\kern-\sphinxverbatimsmallskipamount\kern-\baselineskip
\kern+\FrameHeightAdjust\kern-\fboxrule
\vspace{\nbsphinxcodecellspacing}

\sphinxsetup{VerbatimColor={named}{white}}
\sphinxsetup{VerbatimBorderColor={named}{nbsphinx-code-border}}
\begin{sphinxVerbatim}[commandchars=\\\{\}]
\llap{\color{nbsphinxout}[9]:\,\hspace{\fboxrule}\hspace{\fboxsep}}[PosixPath('Results/ResidentialConsumption\_Zero2020.txt'),
 PosixPath('Results/ResidentialConsumption\_Zero2025.txt'),
 PosixPath('Results/ResidentialConsumption\_Zero2030.txt'),
 PosixPath('Results/ResidentialConsumption\_Zero2035.txt'),
 PosixPath('Results/ResidentialConsumption\_Zero2040.txt'),
 PosixPath('Results/ResidentialConsumption\_Zero2045.txt'),
 PosixPath('Results/ResidentialConsumption\_Zero2050.txt')]
\end{sphinxVerbatim}
}

Again, we can see that the number of output files generated were as we expected and that our new message “Hello, you!” was found within these files. This means that our output and sink functions worked as expected.


\subsection{Next steps}
\label{\detokenize{advanced-guide/extending-muse:Next-steps}}
In the next section we will output a technology filter, to stop agents from investing in a certain technology, and a new metric to combine multiple objectives.


\section{Further extending MUSE}
\label{\detokenize{advanced-guide/further-extending-muse:Further-extending-MUSE}}\label{\detokenize{advanced-guide/further-extending-muse::doc}}
{
\sphinxsetup{VerbatimColor={named}{nbsphinx-code-bg}}
\sphinxsetup{VerbatimBorderColor={named}{nbsphinx-code-border}}
\begin{sphinxVerbatim}[commandchars=\\\{\}]
\llap{\color{nbsphinxin}[1]:\,\hspace{\fboxrule}\hspace{\fboxsep}}\PYG{k+kn}{from} \PYG{n+nn}{xarray} \PYG{k+kn}{import} \PYG{n}{Dataset}\PYG{p}{,} \PYG{n}{DataArray}

\PYG{k+kn}{from} \PYG{n+nn}{muse}\PYG{n+nn}{.}\PYG{n+nn}{agents} \PYG{k+kn}{import} \PYG{n}{Agent}
\PYG{k+kn}{from} \PYG{n+nn}{muse}\PYG{n+nn}{.}\PYG{n+nn}{filters} \PYG{k+kn}{import} \PYG{n}{register\PYGZus{}filter}

\PYG{n+nd}{@register\PYGZus{}filter}
\PYG{k}{def} \PYG{n+nf}{no\PYGZus{}ccgt\PYGZus{}filter}\PYG{p}{(}
    \PYG{n}{agent}\PYG{p}{:} \PYG{n}{Agent}\PYG{p}{,}
    \PYG{n}{search\PYGZus{}space}\PYG{p}{:} \PYG{n}{DataArray}\PYG{p}{,}
    \PYG{n}{technologies}\PYG{p}{:} \PYG{n}{Dataset}\PYG{p}{,}
    \PYG{n}{market}\PYG{p}{:} \PYG{n}{Dataset}
\PYG{p}{)} \PYG{o}{\PYGZhy{}}\PYG{o}{\PYGZgt{}} \PYG{n}{DataArray}\PYG{p}{:}
    \PYG{l+s+sd}{\PYGZdq{}\PYGZdq{}\PYGZdq{}Excludes gasCCGT.\PYGZdq{}\PYGZdq{}\PYGZdq{}}
    \PYG{n}{dropped\PYGZus{}tech} \PYG{o}{=} \PYG{n}{search\PYGZus{}space}\PYG{o}{.}\PYG{n}{where}\PYG{p}{(}\PYG{n}{search\PYGZus{}space}\PYG{o}{.}\PYG{n}{replacement} \PYG{o}{!=} \PYG{l+s+s2}{\PYGZdq{}}\PYG{l+s+s2}{windturbine}\PYG{l+s+s2}{\PYGZdq{}}\PYG{p}{,} \PYG{n}{drop}\PYG{o}{=}\PYG{k+kc}{True}\PYG{p}{)}
    \PYG{k}{return} \PYG{n}{search\PYGZus{}space} \PYG{o}{\PYGZam{}} \PYG{n}{search\PYGZus{}space}\PYG{o}{.}\PYG{n}{replacement}\PYG{o}{.}\PYG{n}{isin}\PYG{p}{(}\PYG{n}{dropped\PYGZus{}tech}\PYG{o}{.}\PYG{n}{replacement}\PYG{p}{)}
\end{sphinxVerbatim}
}

{
\sphinxsetup{VerbatimColor={named}{nbsphinx-code-bg}}
\sphinxsetup{VerbatimBorderColor={named}{nbsphinx-code-border}}
\begin{sphinxVerbatim}[commandchars=\\\{\}]
\llap{\color{nbsphinxin}[6]:\,\hspace{\fboxrule}\hspace{\fboxsep}}\PYG{k+kn}{import} \PYG{n+nn}{logging}
\PYG{k+kn}{from} \PYG{n+nn}{muse}\PYG{n+nn}{.}\PYG{n+nn}{mca} \PYG{k+kn}{import} \PYG{n}{MCA}
\PYG{k+kn}{from} \PYG{n+nn}{muse} \PYG{k+kn}{import} \PYG{n}{examples}

\PYG{c+c1}{\PYGZsh{} model\PYGZus{}path = examples.copy\PYGZus{}model(overwrite=True)}
\PYG{n}{logging}\PYG{o}{.}\PYG{n}{getLogger}\PYG{p}{(}\PYG{l+s+s2}{\PYGZdq{}}\PYG{l+s+s2}{muse}\PYG{l+s+s2}{\PYGZdq{}}\PYG{p}{)}\PYG{o}{.}\PYG{n}{setLevel}\PYG{p}{(}\PYG{l+m+mi}{0}\PYG{p}{)}
\PYG{n}{mca} \PYG{o}{=} \PYG{n}{MCA}\PYG{o}{.}\PYG{n}{factory}\PYG{p}{(}\PYG{l+s+s2}{\PYGZdq{}}\PYG{l+s+s2}{model/settings.toml}\PYG{l+s+s2}{\PYGZdq{}}\PYG{p}{)}
\PYG{n}{mca}\PYG{o}{.}\PYG{n}{run}\PYG{p}{(}\PYG{p}{)}\PYG{p}{;}
\end{sphinxVerbatim}
}

{

\kern-\sphinxverbatimsmallskipamount\kern-\baselineskip
\kern+\FrameHeightAdjust\kern-\fboxrule
\vspace{\nbsphinxcodecellspacing}

\sphinxsetup{VerbatimColor={named}{white}}
\sphinxsetup{VerbatimBorderColor={named}{nbsphinx-code-border}}
\begin{sphinxVerbatim}[commandchars=\\\{\}]
Primal Feasibility  Dual Feasibility    Duality Gap         Step             Path Parameter      Objective
1.0                 1.0                 1.0                 -                1.0                 148.9679256735
0.2598249156018     0.2598249156018     0.2598249156018     0.7495200004432  0.2598249156018     120.5733849622
0.02399956829695    0.02399956829695    0.02399956829695    0.9210391224498  0.02399956829695    4.780663765494
0.0181364461758     0.0181364461758     0.0181364461758     0.2509588065043  0.0181364461758     7.107141691547
0.01499350833129    0.01499350833129    0.01499350833129    0.1921973185437  0.01499350833129    70.77614035582
0.004968295711366   0.004968295711367   0.004968295711366   0.6857131120066  0.004968295711366   164.7472224003
0.0006443120819652  0.0006443120819642  0.000644312081964   0.8804718592549  0.0006443120819672  289.7109372802
2.427431365313e-06  2.427431365276e-06  2.42743136527e-06   0.9976309182175  2.427431365399e-06  310.7437190082
1.214286379284e-10  1.214286245985e-10  1.214286217581e-10  0.9999499778566  1.214286284187e-10  310.7859704917
Optimization terminated successfully.
         Current function value: 310.785970
         Iterations: 8
HIHIHIHIHI
Primal Feasibility  Dual Feasibility    Duality Gap         Step             Path Parameter      Objective
1.0                 1.0                 1.0                 -                1.0                 148.9679256735
0.1806451257467     0.1806451257467     0.1806451257467     0.8281847212959  0.1806451257467     169.6037982942
0.02684129624378    0.02684129624378    0.02684129624378    0.8635116331789  0.02684129624378    123.2904723181
0.01081107082374    0.01081107082373    0.01081107082373    0.6279145428497  0.01081107082373    293.3552576002
0.00150335333755    0.001503353337531   0.001503353337531   0.8785435425234  0.001503353337582   618.5754737903
4.126548293386e-06  4.126548293452e-06  4.126548293469e-06  0.99729939758    4.126548293473e-06  673.2429034259
2.063813940498e-10  2.063814559148e-10  2.063814538311e-10  0.9999499869317  2.06381853248e-10   673.369600975
Optimization terminated successfully.
         Current function value: 673.369601
         Iterations: 6
Primal Feasibility  Dual Feasibility    Duality Gap         Step             Path Parameter      Objective
1.0                 1.0                 1.0                 -                1.0                 225.7506433631
0.07675788061753    0.07675788061753    0.07675788061751    0.9271368109475  0.07675788061753    1.307751786207
0.01889464099889    0.01889464099889    0.01889464099888    0.7970949255449  0.01889464099889    19.4989221165
0.007543783963398   0.007543783963394   0.007543783963392   0.6153162486386  0.007543783963394   16.52048983639
0.002504946781023   0.002504946781021   0.00250494678102    0.7004074675406  0.002504946781021   72.319459457
0.0004445444355394  0.000444544435539   0.0004445444355388  0.8689995047748  0.000444544435539   423.758860583
1.214501095331e-05  1.214501095325e-05  1.214501095324e-05  0.9820606335229  1.214501095322e-05  675.1646942955
1.070439776445e-09  1.07043978707e-09   1.070439766673e-09  0.9999120772687  1.070439801419e-09  681.9179209593
5.353929135063e-14  5.353592310399e-14  5.35373239097e-14   0.9999499866071  5.352230256347e-14  681.9185224492
Optimization terminated successfully.
         Current function value: 681.918522
         Iterations: 8
\end{sphinxVerbatim}
}

{

\kern-\sphinxverbatimsmallskipamount\kern-\baselineskip
\kern+\FrameHeightAdjust\kern-\fboxrule
\vspace{\nbsphinxcodecellspacing}

\sphinxsetup{VerbatimColor={named}{nbsphinx-stderr}}
\sphinxsetup{VerbatimBorderColor={named}{nbsphinx-code-border}}
\begin{sphinxVerbatim}[commandchars=\\\{\}]
-- 2020-11-09 15:08:48 - muse.mca - WARNING
Check growth constraints for wind.

\end{sphinxVerbatim}
}

{

\kern-\sphinxverbatimsmallskipamount\kern-\baselineskip
\kern+\FrameHeightAdjust\kern-\fboxrule
\vspace{\nbsphinxcodecellspacing}

\sphinxsetup{VerbatimColor={named}{white}}
\sphinxsetup{VerbatimBorderColor={named}{nbsphinx-code-border}}
\begin{sphinxVerbatim}[commandchars=\\\{\}]
Primal Feasibility  Dual Feasibility    Duality Gap         Step             Path Parameter      Objective
1.0                 1.0                 1.0                 -                1.0                 359.2443189825
0.2131118149695     0.2131118149695     0.2131118149695     0.7960343319409  0.2131118149695     262.2885392799
0.05758094728718    0.05758094728718    0.05758094728718    0.7523513573813  0.05758094728718    13.16175379893
0.01048349585899    0.01048349585899    0.01048349585899    0.8203940076989  0.01048349585899    9.036399448741
0.009005598049604   0.009005598049604   0.009005598049604   0.1488155054953  0.009005598049604   19.40296370843
0.002317604003518   0.002317604003518   0.002317604003518   0.8098673571979  0.002317604003518   226.5492459765
0.0009784310820962  0.0009784310820963  0.0009784310820963  0.5991550275447  0.0009784310820971  289.4684048776
0.0001326875071986  0.0001326875071986  0.0001326875071986  0.8836343167337  0.0001326875071987  358.6919881748
6.688262823007e-08  6.688262823276e-08  6.688262823391e-08  0.9995454959391  6.688262822145e-08  365.8223849509
3.344208692489e-12  3.34420597683e-12   3.344204125008e-12  0.9999499989235  3.344177862259e-12  365.8250744356
Optimization terminated successfully.
         Current function value: 365.825074
         Iterations: 9
HIHIHIHIHI
Primal Feasibility  Dual Feasibility    Duality Gap         Step             Path Parameter      Objective
1.0                 1.0                 1.0                 -                1.0                 179.6221594912
0.06792119055695    0.06792119055695    0.06792119055695    0.9362017234058  0.06792119055695    76.26287556551
0.009746786382626   0.009746786382626   0.009746786382626   0.8719786940257  0.009746786382625   34.28929487798
0.00504956460969    0.005049564609689   0.005049564609689   0.5134005827998  0.005049564609689   115.6145513358
0.003132107070668   0.003132107070663   0.003132107070663   0.3922116719677  0.003132107070663   175.0444435627
0.0005331490663204  0.000533149066321   0.000533149066321   0.8631478238108  0.0005331490663195  430.7265923665
6.809758501168e-06  6.809758501084e-06  6.809758501069e-06  0.9896962642248  6.809758501216e-06  482.9615875673
3.566453823927e-10  3.56645399101e-10   3.566453982014e-10  0.9999476448412  3.566453742836e-10  483.66367672
Optimization terminated successfully.
         Current function value: 483.663677
         Iterations: 7
Primal Feasibility  Dual Feasibility    Duality Gap         Step             Path Parameter      Objective
1.0                 1.0                 1.0                 -                1.0                 547.5170704708
0.06099397574481    0.06099397574481    0.06099397574482    0.944387525785   0.06099397574482    192.4316112656
0.01386491315737    0.01386491315737    0.01386491315737    0.7896145976147  0.01386491315737    3.157754203839
0.00713143943229    0.00713143943229    0.007131439432292   0.5116778496497  0.007131439432291   6.913227303095
0.0007854492963609  0.0007854492963609  0.0007854492963611  0.9057153968575  0.000785449296361   6.591304425235
0.0003968642772996  0.0003968642772997  0.0003968642772998  0.5195625901748  0.0003968642772996  5.218772647071
5.276614006577e-06  5.276614006577e-06  5.276614006576e-06  0.9941140946865  5.276614006565e-06  5.272614338781
3.679893198137e-10  3.679893242575e-10  3.679893250277e-10  0.9999303117912  3.679893357516e-10  5.266601980833
1.790517523351e-14  1.842347640038e-14  1.841384711876e-14  0.9999499611495  1.844295319492e-14  5.266601636509
Optimization terminated successfully.
         Current function value: 5.266602
         Iterations: 8
HIHIHIHIHI
Primal Feasibility  Dual Feasibility    Duality Gap         Step             Path Parameter      Objective
1.0                 1.0                 1.0                 -                1.0                 273.0918767992
0.007784675344484   0.007784675344483   0.007784675344496   0.9922157975144  0.007784675344482   1.218389503684
0.0007796035919664  0.0007796035919663  0.0007796035919676  0.9305974535198  0.0007796035919662  8.689368495091
0.0002432663284089  0.0002432663284089  0.0002432663284093  0.7114098156148  0.0002432663284089  45.24959963709
7.450312562646e-05  7.450312562646e-05  7.450312562658e-05  0.7438433474555  7.450312562644e-05  232.9668001869
1.824586184364e-06  1.824586184364e-06  1.824586184368e-06  0.9786324171177  1.824586184364e-06  272.0879301696
1.362513233892e-10  1.362513236089e-10  1.36251323629e-10   0.9999263111913  1.362513232934e-10  273.0918022961
6.81275233453e-15   6.813117868374e-15  6.81302616198e-15   0.9999499990795  6.812566211363e-15  273.0918768406
Optimization terminated successfully.
         Current function value: 273.091877
         Iterations: 7
HIHIHIHIHI
Primal Feasibility  Dual Feasibility    Duality Gap         Step             Path Parameter      Objective
1.0                 1.0                 1.0                 -                1.0                 24.59260538017
0.01678939478565    0.01678939478565    0.01678939478565    0.9931439495955  0.01678939478565    0.6055223166906
0.009965674493075   0.009965674493075   0.009965674493078   0.4242810241262  0.009965674493075   3.244714326293
0.003397925619967   0.003397925619967   0.003397925619968   0.7237970170791  0.003397925619967   63.79567916947
2.830731029084e-05  2.830731029084e-05  2.830731029086e-05  0.9917790420884  2.830731029084e-05  77.95525849671
1.416553419389e-09  1.416553580528e-09  1.416553598767e-09  0.999949961019   1.416553321704e-09  78.09518029969
7.075972310081e-14  7.098971338123e-14  7.097069021718e-14  0.9999498833268  7.082766760582e-14  78.095187309
Optimization terminated successfully.
         Current function value: 78.095187
         Iterations: 6
\end{sphinxVerbatim}
}

{

\kern-\sphinxverbatimsmallskipamount\kern-\baselineskip
\kern+\FrameHeightAdjust\kern-\fboxrule
\vspace{\nbsphinxcodecellspacing}

\sphinxsetup{VerbatimColor={named}{nbsphinx-stderr}}
\sphinxsetup{VerbatimBorderColor={named}{nbsphinx-code-border}}
\begin{sphinxVerbatim}[commandchars=\\\{\}]
-- 2020-11-09 15:08:54 - muse.mca - WARNING
Check growth constraints for wind.

\end{sphinxVerbatim}
}

{

\kern-\sphinxverbatimsmallskipamount\kern-\baselineskip
\kern+\FrameHeightAdjust\kern-\fboxrule
\vspace{\nbsphinxcodecellspacing}

\sphinxsetup{VerbatimColor={named}{white}}
\sphinxsetup{VerbatimBorderColor={named}{nbsphinx-code-border}}
\begin{sphinxVerbatim}[commandchars=\\\{\}]
Primal Feasibility  Dual Feasibility    Duality Gap         Step             Path Parameter      Objective
1.0                 1.0                 1.0                 -                1.0                 359.2443189825
0.2131118149695     0.2131118149695     0.2131118149695     0.7960343319409  0.2131118149695     262.2885392799
0.05758094728718    0.05758094728718    0.05758094728718    0.7523513573813  0.05758094728718    13.16175379893
0.01048349585899    0.01048349585899    0.01048349585899    0.8203940076989  0.01048349585899    9.036399448741
0.009005598049604   0.009005598049604   0.009005598049604   0.1488155054953  0.009005598049604   19.40296370843
0.002317604003518   0.002317604003518   0.002317604003518   0.8098673571979  0.002317604003518   226.5492459765
0.0009784310820962  0.0009784310820963  0.0009784310820963  0.5991550275447  0.0009784310820971  289.4684048776
0.0001326875071986  0.0001326875071986  0.0001326875071986  0.8836343167337  0.0001326875071987  358.6919881748
6.688262823007e-08  6.688262823276e-08  6.688262823391e-08  0.9995454959391  6.688262822145e-08  365.8223849509
3.344208692489e-12  3.34420597683e-12   3.344204125008e-12  0.9999499989235  3.344177862259e-12  365.8250744356
Optimization terminated successfully.
         Current function value: 365.825074
         Iterations: 9
HIHIHIHIHI
Primal Feasibility  Dual Feasibility    Duality Gap         Step             Path Parameter      Objective
1.0                 1.0                 1.0                 -                1.0                 179.6221594912
0.06792119055695    0.06792119055695    0.06792119055695    0.9362017234058  0.06792119055695    76.26287556551
0.009746786382626   0.009746786382626   0.009746786382626   0.8719786940257  0.009746786382625   34.28929487798
0.00504956460969    0.005049564609689   0.005049564609689   0.5134005827998  0.005049564609689   115.6145513358
0.003132107070668   0.003132107070663   0.003132107070663   0.3922116719677  0.003132107070663   175.0444435627
0.0005331490663204  0.000533149066321   0.000533149066321   0.8631478238108  0.0005331490663195  430.7265923665
6.809758501168e-06  6.809758501084e-06  6.809758501069e-06  0.9896962642248  6.809758501216e-06  482.9615875673
3.566453823927e-10  3.56645399101e-10   3.566453982014e-10  0.9999476448412  3.566453742836e-10  483.66367672
Optimization terminated successfully.
         Current function value: 483.663677
         Iterations: 7
Primal Feasibility  Dual Feasibility    Duality Gap         Step             Path Parameter      Objective
1.0                 1.0                 1.0                 -                1.0                 547.5170704708
0.06099397574481    0.06099397574481    0.06099397574482    0.944387525785   0.06099397574482    192.4316112656
0.01386491315737    0.01386491315737    0.01386491315737    0.7896145976147  0.01386491315737    3.157754203839
0.00713143943229    0.00713143943229    0.007131439432292   0.5116778496497  0.007131439432291   6.913227303095
0.0007854492963609  0.0007854492963609  0.0007854492963611  0.9057153968575  0.000785449296361   6.591304425235
0.0003968642772996  0.0003968642772997  0.0003968642772998  0.5195625901748  0.0003968642772996  5.218772647071
5.276614006577e-06  5.276614006577e-06  5.276614006576e-06  0.9941140946865  5.276614006565e-06  5.272614338781
3.679893198137e-10  3.679893242575e-10  3.679893250277e-10  0.9999303117912  3.679893357516e-10  5.266601980833
1.790517523351e-14  1.842347640038e-14  1.841384711876e-14  0.9999499611495  1.844295319492e-14  5.266601636509
Optimization terminated successfully.
         Current function value: 5.266602
         Iterations: 8
HIHIHIHIHI
Primal Feasibility  Dual Feasibility    Duality Gap         Step             Path Parameter      Objective
1.0                 1.0                 1.0                 -                1.0                 273.0918767992
0.007784675344484   0.007784675344483   0.007784675344496   0.9922157975144  0.007784675344482   1.218389503684
0.0007796035919664  0.0007796035919663  0.0007796035919676  0.9305974535198  0.0007796035919662  8.689368495091
0.0002432663284089  0.0002432663284089  0.0002432663284093  0.7114098156148  0.0002432663284089  45.24959963709
7.450312562646e-05  7.450312562646e-05  7.450312562658e-05  0.7438433474555  7.450312562644e-05  232.9668001869
1.824586184364e-06  1.824586184364e-06  1.824586184368e-06  0.9786324171177  1.824586184364e-06  272.0879301696
1.362513233892e-10  1.362513236089e-10  1.36251323629e-10   0.9999263111913  1.362513232934e-10  273.0918022961
6.81275233453e-15   6.813117868374e-15  6.81302616198e-15   0.9999499990795  6.812566211363e-15  273.0918768406
Optimization terminated successfully.
         Current function value: 273.091877
         Iterations: 7
HIHIHIHIHI
Primal Feasibility  Dual Feasibility    Duality Gap         Step             Path Parameter      Objective
1.0                 1.0                 1.0                 -                1.0                 24.59260538017
0.01678939478565    0.01678939478565    0.01678939478565    0.9931439495955  0.01678939478565    0.6055223166906
0.009965674493075   0.009965674493075   0.009965674493078   0.4242810241262  0.009965674493075   3.244714326293
0.003397925619967   0.003397925619967   0.003397925619968   0.7237970170791  0.003397925619967   63.79567916947
2.830731029084e-05  2.830731029084e-05  2.830731029086e-05  0.9917790420884  2.830731029084e-05  77.95525849671
1.416553419389e-09  1.416553580528e-09  1.416553598767e-09  0.999949961019   1.416553321704e-09  78.09518029969
7.075972310081e-14  7.098971338123e-14  7.097069021718e-14  0.9999498833268  7.082766760582e-14  78.095187309
Optimization terminated successfully.
         Current function value: 78.095187
         Iterations: 6
\end{sphinxVerbatim}
}

{

\kern-\sphinxverbatimsmallskipamount\kern-\baselineskip
\kern+\FrameHeightAdjust\kern-\fboxrule
\vspace{\nbsphinxcodecellspacing}

\sphinxsetup{VerbatimColor={named}{nbsphinx-stderr}}
\sphinxsetup{VerbatimBorderColor={named}{nbsphinx-code-border}}
\begin{sphinxVerbatim}[commandchars=\\\{\}]
-- 2020-11-09 15:09:00 - muse.mca - WARNING
Check growth constraints for wind.

\end{sphinxVerbatim}
}

{

\kern-\sphinxverbatimsmallskipamount\kern-\baselineskip
\kern+\FrameHeightAdjust\kern-\fboxrule
\vspace{\nbsphinxcodecellspacing}

\sphinxsetup{VerbatimColor={named}{white}}
\sphinxsetup{VerbatimBorderColor={named}{nbsphinx-code-border}}
\begin{sphinxVerbatim}[commandchars=\\\{\}]
Primal Feasibility  Dual Feasibility    Duality Gap         Step             Path Parameter      Objective
1.0                 1.0                 1.0                 -                1.0                 414.316476678
0.1943112264485     0.1943112264485     0.1943112264485     0.8115414580695  0.1943112264485     293.1180738751
0.05737071259203    0.05737071259203    0.05737071259203    0.7291321197238  0.05737071259203    17.07106219013
0.01062277327644    0.01062277327644    0.01062277327644    0.8173814867435  0.01062277327644    12.44885046687
0.009064514495371   0.009064514495372   0.009064514495372   0.1550649500025  0.009064514495372   28.00868491133
0.002334119807294   0.002334119807293   0.002334119807293   0.8051326751218  0.002334119807295   275.8552661516
0.0006946146456239  0.0006946146456238  0.0006946146456238  0.7097377670861  0.0006946146456242  332.7560201847
0.000253341221249   0.0002533412212489  0.0002533412212489  0.6718481634042  0.0002533412212491  377.2933943083
1.33722226265e-06   1.337222262655e-06  1.337222262657e-06  0.9952598213341  1.337222262651e-06  392.3599718847
6.775876566156e-11  6.775875940049e-11  6.775875524037e-11  0.9999493313269  6.775868905564e-11  392.4568216146
Optimization terminated successfully.
         Current function value: 392.456822
         Iterations: 9
HIHIHIHIHI
Primal Feasibility  Dual Feasibility    Duality Gap         Step             Path Parameter      Objective
1.0                 1.0                 1.0                 -                1.0                 414.316476678
0.1030590385675     0.1030590385675     0.1030590385675     0.9011154818883  0.1030590385675     219.2748221074
0.02654054666621    0.02654054666621    0.02654054666621    0.7752562655553  0.02654054666621    276.7551091027
0.01427394633699    0.01427394633699    0.01427394633699    0.4912526709581  0.01427394633699    607.4391713228
0.001218580492651   0.001218580492601   0.001218580492601   0.9248974186534  0.001218580492676   1166.289755796
1.888403761715e-05  1.888403761622e-05  1.888403761624e-05  0.985551099478   1.888403761735e-05  1241.740132447
1.545745837373e-09  1.545745372552e-09  1.545745362755e-09  0.999922927438   1.545745648845e-09  1242.779847889
1.692223225044e-10  1.873153075245e-10  1.873152955153e-10  0.8801597673128  1.817506825368e-10  1242.779937345
Optimization terminated successfully.
         Current function value: 1242.779937
         Iterations: 7
Primal Feasibility  Dual Feasibility    Duality Gap         Step             Path Parameter      Objective
1.0                 1.0                 1.0                 -                1.0                 652.1069903706
0.06096672005102    0.06096672005103    0.060966720051      0.9444993128467  0.06096672005103    285.1503667166
0.01533649030442    0.01533649030442    0.01533649030441    0.760955159853   0.01533649030442    2.726105681791
0.006320164709415   0.006320164709416   0.006320164709413   0.6160648365512  0.006320164709416   8.631043022899
0.001197892469647   0.001197892469648   0.001197892469647   0.8360719893043  0.001197892469648   8.304987261732
0.0005713687249976  0.0005713687249978  0.0005713687249976  0.5371632525845  0.0005713687249978  6.593737888213
0.0001900730365094  0.0001900730365095  0.0001900730365094  0.7071525519853  0.0001900730365095  4.503362444169
1.636573313793e-05  1.636573313793e-05  1.636573313793e-05  0.9478311770592  1.636573313794e-05  4.082883819104
7.532302844599e-09  7.532302083082e-09  7.532302077642e-09  0.9995526057271  7.532302082324e-09  3.99997015618
3.762057344252e-13  3.766204589771e-13  3.766274429159e-13  0.9999499983548  3.766188718361e-13  3.99995064979
Optimization terminated successfully.
         Current function value: 3.999951
         Iterations: 9
HIHIHIHIHI
Primal Feasibility  Dual Feasibility    Duality Gap         Step             Path Parameter      Objective
1.0                 1.0                 1.0                 -                1.0                 24.59260538017
0.03041802557839    0.03041802557839    0.03041802557839    0.9772064682649  0.03041802557839    24.19388314957
0.009080107307655   0.009080107307655   0.009080107307656   0.7409207447829  0.009080107307655   102.5179052635
0.0005454700267143  0.000545470026725   0.000545470026725   0.95051584666    0.0005454700267272  180.762473129
4.528574406413e-08  4.528574409137e-08  4.528574404608e-08  0.9999182057566  4.528574406479e-08  184.4441432226
2.264218525495e-12  2.264162473465e-12  2.264155892645e-12  0.9999500028083  2.264287232588e-12  184.4445388081
Optimization terminated successfully.
         Current function value: 184.444539
         Iterations: 5
\end{sphinxVerbatim}
}

{

\kern-\sphinxverbatimsmallskipamount\kern-\baselineskip
\kern+\FrameHeightAdjust\kern-\fboxrule
\vspace{\nbsphinxcodecellspacing}

\sphinxsetup{VerbatimColor={named}{nbsphinx-stderr}}
\sphinxsetup{VerbatimBorderColor={named}{nbsphinx-code-border}}
\begin{sphinxVerbatim}[commandchars=\\\{\}]
-- 2020-11-09 15:09:07 - muse.mca - WARNING
Check growth constraints for wind.

\end{sphinxVerbatim}
}

{

\kern-\sphinxverbatimsmallskipamount\kern-\baselineskip
\kern+\FrameHeightAdjust\kern-\fboxrule
\vspace{\nbsphinxcodecellspacing}

\sphinxsetup{VerbatimColor={named}{white}}
\sphinxsetup{VerbatimBorderColor={named}{nbsphinx-code-border}}
\begin{sphinxVerbatim}[commandchars=\\\{\}]
Primal Feasibility  Dual Feasibility    Duality Gap         Step             Path Parameter      Objective
1.0                 1.0                 1.0                 -                1.0                 414.316476678
0.1943112264485     0.1943112264485     0.1943112264485     0.8115414580695  0.1943112264485     293.1180738751
0.05737071259203    0.05737071259203    0.05737071259203    0.7291321197238  0.05737071259203    17.07106219013
0.01062277327644    0.01062277327644    0.01062277327644    0.8173814867435  0.01062277327644    12.44885046687
0.009064514495371   0.009064514495372   0.009064514495372   0.1550649500025  0.009064514495372   28.00868491133
0.002334119807294   0.002334119807293   0.002334119807293   0.8051326751218  0.002334119807295   275.8552661516
0.0006946146456239  0.0006946146456238  0.0006946146456238  0.7097377670861  0.0006946146456242  332.7560201847
0.000253341221249   0.0002533412212489  0.0002533412212489  0.6718481634042  0.0002533412212491  377.2933943083
1.33722226265e-06   1.337222262655e-06  1.337222262657e-06  0.9952598213341  1.337222262651e-06  392.3599718847
6.775876566156e-11  6.775875940049e-11  6.775875524037e-11  0.9999493313269  6.775868905564e-11  392.4568216146
Optimization terminated successfully.
         Current function value: 392.456822
         Iterations: 9
HIHIHIHIHI
Primal Feasibility  Dual Feasibility    Duality Gap         Step             Path Parameter      Objective
1.0                 1.0                 1.0                 -                1.0                 414.316476678
0.1030590385675     0.1030590385675     0.1030590385675     0.9011154818883  0.1030590385675     219.2748221074
0.02654054666621    0.02654054666621    0.02654054666621    0.7752562655553  0.02654054666621    276.7551091027
0.01427394633699    0.01427394633699    0.01427394633699    0.4912526709581  0.01427394633699    607.4391713228
0.001218580492651   0.001218580492601   0.001218580492601   0.9248974186534  0.001218580492676   1166.289755796
1.888403761715e-05  1.888403761622e-05  1.888403761624e-05  0.985551099478   1.888403761735e-05  1241.740132447
1.545745837373e-09  1.545745372552e-09  1.545745362755e-09  0.999922927438   1.545745648845e-09  1242.779847889
1.692223225044e-10  1.873153075245e-10  1.873152955153e-10  0.8801597673128  1.817506825368e-10  1242.779937345
Optimization terminated successfully.
         Current function value: 1242.779937
         Iterations: 7
Primal Feasibility  Dual Feasibility    Duality Gap         Step             Path Parameter      Objective
1.0                 1.0                 1.0                 -                1.0                 652.1069903706
0.06096672005102    0.06096672005103    0.060966720051      0.9444993128467  0.06096672005103    285.1503667166
0.01533649030442    0.01533649030442    0.01533649030441    0.760955159853   0.01533649030442    2.726105681791
0.006320164709415   0.006320164709416   0.006320164709413   0.6160648365512  0.006320164709416   8.631043022899
0.001197892469647   0.001197892469648   0.001197892469647   0.8360719893043  0.001197892469648   8.304987261732
0.0005713687249976  0.0005713687249978  0.0005713687249976  0.5371632525845  0.0005713687249978  6.593737888213
0.0001900730365094  0.0001900730365095  0.0001900730365094  0.7071525519853  0.0001900730365095  4.503362444169
1.636573313793e-05  1.636573313793e-05  1.636573313793e-05  0.9478311770592  1.636573313794e-05  4.082883819104
7.532302844599e-09  7.532302083082e-09  7.532302077642e-09  0.9995526057271  7.532302082324e-09  3.99997015618
3.762057344252e-13  3.766204589771e-13  3.766274429159e-13  0.9999499983548  3.766188718361e-13  3.99995064979
Optimization terminated successfully.
         Current function value: 3.999951
         Iterations: 9
HIHIHIHIHI
Primal Feasibility  Dual Feasibility    Duality Gap         Step             Path Parameter      Objective
1.0                 1.0                 1.0                 -                1.0                 24.59260538017
0.03041802557839    0.03041802557839    0.03041802557839    0.9772064682649  0.03041802557839    24.19388314957
0.009080107307655   0.009080107307655   0.009080107307656   0.7409207447829  0.009080107307655   102.5179052635
0.0005454700267143  0.000545470026725   0.000545470026725   0.95051584666    0.0005454700267272  180.762473129
4.528574406413e-08  4.528574409137e-08  4.528574404608e-08  0.9999182057566  4.528574406479e-08  184.4441432226
2.264218525495e-12  2.264162473465e-12  2.264155892645e-12  0.9999500028083  2.264287232588e-12  184.4445388081
Optimization terminated successfully.
         Current function value: 184.444539
         Iterations: 5
\end{sphinxVerbatim}
}

{

\kern-\sphinxverbatimsmallskipamount\kern-\baselineskip
\kern+\FrameHeightAdjust\kern-\fboxrule
\vspace{\nbsphinxcodecellspacing}

\sphinxsetup{VerbatimColor={named}{nbsphinx-stderr}}
\sphinxsetup{VerbatimBorderColor={named}{nbsphinx-code-border}}
\begin{sphinxVerbatim}[commandchars=\\\{\}]
-- 2020-11-09 15:09:12 - muse.mca - WARNING
Check growth constraints for wind.

\end{sphinxVerbatim}
}

{

\kern-\sphinxverbatimsmallskipamount\kern-\baselineskip
\kern+\FrameHeightAdjust\kern-\fboxrule
\vspace{\nbsphinxcodecellspacing}

\sphinxsetup{VerbatimColor={named}{white}}
\sphinxsetup{VerbatimBorderColor={named}{nbsphinx-code-border}}
\begin{sphinxVerbatim}[commandchars=\\\{\}]
Primal Feasibility  Dual Feasibility    Duality Gap         Step             Path Parameter      Objective
1.0                 1.0                 1.0                 -                1.0                 454.6784294315
0.1862232334687     0.1862232334687     0.1862232334687     0.8174567476576  0.1862232334687     282.4632604722
0.02870747975048    0.02870747975048    0.02870747975048    0.8875534078522  0.02870747975048    46.09791383177
0.009705128410004   0.009705128410004   0.009705128410004   0.6723641007507  0.009705128410004   35.13402203751
0.007693915131578   0.007693915131578   0.007693915131578   0.2197837130033  0.007693915131578   80.93881539206
0.001699044794839   0.001699044794837   0.001699044794837   0.8250841182677  0.00169904479484    298.2653882331
0.0005650008980062  0.0005650008980057  0.0005650008980057  0.678565948553   0.0005650008980065  327.975809321
0.0002196033953225  0.0002196033953223  0.0002196033953223  0.6469207178836  0.0002196033953226  358.4708363196
2.579861478512e-06  2.579861478501e-06  2.579861478502e-06  0.9887803221269  2.579861478497e-06  368.5514208652
1.310502062448e-10  1.310502049574e-10  1.31050204896e-10   0.9999492039366  1.310502128569e-10  368.6632645094
Optimization terminated successfully.
         Current function value: 368.663265
         Iterations: 9
HIHIHIHIHI
Primal Feasibility  Dual Feasibility    Duality Gap         Step             Path Parameter      Objective
1.0                 1.0                 1.0                 -                1.0                 454.6784294315
0.06717002148527    0.06717002148527    0.06717002148527    0.939336263039   0.06717002148527    183.2225626364
0.0335506482529     0.0335506482529     0.0335506482529     0.5230681149682  0.0335506482529     256.2098343218
0.01040420231972    0.01040420231972    0.01040420231972    0.6975119886622  0.01040420231972    178.3301859411
0.004349337378734   0.004349337378734   0.004349337378734   0.6235886580387  0.004349337378734   396.8309868051
0.002928936786464   0.002928936786463   0.002928936786463   0.3467952576409  0.002928936786463   488.976382689
0.0001458208281248  0.0001458208281499  0.0001458208281499  0.9601001354313  0.0001458208281409  665.3317152907
1.593588680453e-07  1.593588684743e-07  1.593588684755e-07  0.9989479829423  1.593588661973e-07  675.8691141106
7.991316370683e-12  7.990632294946e-12  7.990631661844e-12  0.999949860274   7.990813072693e-12  675.8826828943
Optimization terminated successfully.
         Current function value: 675.882683
         Iterations: 8
Primal Feasibility  Dual Feasibility    Duality Gap         Step             Path Parameter      Objective
1.0                 1.0                 1.0                 -                1.0                 759.5666413636
0.06103168736637    0.06103168736636    0.06103168736636    0.9444947751848  0.06103168736636    386.894766737
0.01481387164895    0.01481387164895    0.01481387164895    0.7670031635967  0.01481387164895    2.63017161912
0.005246182165771   0.00524618216577    0.005246182165771   0.676128398925   0.00524618216577    11.26512518349
0.001435544233241   0.00143554423324    0.00143554423324    0.7534457362106  0.00143554423324    12.27554761078
0.000168669976258   0.000168669976258   0.000168669976258   0.8892431693931  0.000168669976258   6.683524472165
1.71245789073e-05   1.712457890729e-05  1.712457890729e-05  0.9390213283973  1.712457890728e-05  6.18232423823
6.830533706094e-08  6.830533980833e-08  6.830533980775e-08  0.9982184386442  6.830533976625e-08  6.066658211057
3.416078908143e-12  3.415974879856e-12  3.415978735438e-12  0.9999499895809  3.41597612292e-12   6.066591807906
Optimization terminated successfully.
         Current function value: 6.066592
         Iterations: 8
\end{sphinxVerbatim}
}

{

\kern-\sphinxverbatimsmallskipamount\kern-\baselineskip
\kern+\FrameHeightAdjust\kern-\fboxrule
\vspace{\nbsphinxcodecellspacing}

\sphinxsetup{VerbatimColor={named}{nbsphinx-stderr}}
\sphinxsetup{VerbatimBorderColor={named}{nbsphinx-code-border}}
\begin{sphinxVerbatim}[commandchars=\\\{\}]
-- 2020-11-09 15:09:17 - muse.mca - WARNING
Check growth constraints for wind.

\end{sphinxVerbatim}
}

{

\kern-\sphinxverbatimsmallskipamount\kern-\baselineskip
\kern+\FrameHeightAdjust\kern-\fboxrule
\vspace{\nbsphinxcodecellspacing}

\sphinxsetup{VerbatimColor={named}{white}}
\sphinxsetup{VerbatimBorderColor={named}{nbsphinx-code-border}}
\begin{sphinxVerbatim}[commandchars=\\\{\}]
Primal Feasibility  Dual Feasibility    Duality Gap         Step             Path Parameter      Objective
1.0                 1.0                 1.0                 -                1.0                 454.6784294315
0.1862232334687     0.1862232334687     0.1862232334687     0.8174567476576  0.1862232334687     282.4632604722
0.02870747975048    0.02870747975048    0.02870747975048    0.8875534078522  0.02870747975048    46.09791383177
0.009705128410004   0.009705128410004   0.009705128410004   0.6723641007507  0.009705128410004   35.13402203751
0.007693915131578   0.007693915131578   0.007693915131578   0.2197837130033  0.007693915131578   80.93881539206
0.001699044794839   0.001699044794837   0.001699044794837   0.8250841182677  0.00169904479484    298.2653882331
0.0005650008980062  0.0005650008980057  0.0005650008980057  0.678565948553   0.0005650008980065  327.975809321
0.0002196033953225  0.0002196033953223  0.0002196033953223  0.6469207178836  0.0002196033953226  358.4708363196
2.579861478512e-06  2.579861478501e-06  2.579861478502e-06  0.9887803221269  2.579861478497e-06  368.5514208652
1.310502062448e-10  1.310502049574e-10  1.31050204896e-10   0.9999492039366  1.310502128569e-10  368.6632645094
Optimization terminated successfully.
         Current function value: 368.663265
         Iterations: 9
HIHIHIHIHI
Primal Feasibility  Dual Feasibility    Duality Gap         Step             Path Parameter      Objective
1.0                 1.0                 1.0                 -                1.0                 454.6784294315
0.06717002148527    0.06717002148527    0.06717002148527    0.939336263039   0.06717002148527    183.2225626364
0.0335506482529     0.0335506482529     0.0335506482529     0.5230681149682  0.0335506482529     256.2098343218
0.01040420231972    0.01040420231972    0.01040420231972    0.6975119886622  0.01040420231972    178.3301859411
0.004349337378734   0.004349337378734   0.004349337378734   0.6235886580387  0.004349337378734   396.8309868051
0.002928936786464   0.002928936786463   0.002928936786463   0.3467952576409  0.002928936786463   488.976382689
0.0001458208281248  0.0001458208281499  0.0001458208281499  0.9601001354313  0.0001458208281409  665.3317152907
1.593588680453e-07  1.593588684743e-07  1.593588684755e-07  0.9989479829423  1.593588661973e-07  675.8691141106
7.991316370683e-12  7.990632294946e-12  7.990631661844e-12  0.999949860274   7.990813072693e-12  675.8826828943
Optimization terminated successfully.
         Current function value: 675.882683
         Iterations: 8
Primal Feasibility  Dual Feasibility    Duality Gap         Step             Path Parameter      Objective
1.0                 1.0                 1.0                 -                1.0                 759.5666413636
0.06103168736637    0.06103168736636    0.06103168736636    0.9444947751848  0.06103168736636    386.894766737
0.01481387164895    0.01481387164895    0.01481387164895    0.7670031635967  0.01481387164895    2.63017161912
0.005246182165771   0.00524618216577    0.005246182165771   0.676128398925   0.00524618216577    11.26512518349
0.001435544233241   0.00143554423324    0.00143554423324    0.7534457362106  0.00143554423324    12.27554761078
0.000168669976258   0.000168669976258   0.000168669976258   0.8892431693931  0.000168669976258   6.683524472165
1.71245789073e-05   1.712457890729e-05  1.712457890729e-05  0.9390213283973  1.712457890728e-05  6.18232423823
6.830533706094e-08  6.830533980833e-08  6.830533980775e-08  0.9982184386442  6.830533976625e-08  6.066658211057
3.416078908143e-12  3.415974879856e-12  3.415978735438e-12  0.9999499895809  3.41597612292e-12   6.066591807906
Optimization terminated successfully.
         Current function value: 6.066592
         Iterations: 8
\end{sphinxVerbatim}
}

{

\kern-\sphinxverbatimsmallskipamount\kern-\baselineskip
\kern+\FrameHeightAdjust\kern-\fboxrule
\vspace{\nbsphinxcodecellspacing}

\sphinxsetup{VerbatimColor={named}{nbsphinx-stderr}}
\sphinxsetup{VerbatimBorderColor={named}{nbsphinx-code-border}}
\begin{sphinxVerbatim}[commandchars=\\\{\}]
-- 2020-11-09 15:09:22 - muse.mca - WARNING
Check growth constraints for wind.

\end{sphinxVerbatim}
}

{

\kern-\sphinxverbatimsmallskipamount\kern-\baselineskip
\kern+\FrameHeightAdjust\kern-\fboxrule
\vspace{\nbsphinxcodecellspacing}

\sphinxsetup{VerbatimColor={named}{white}}
\sphinxsetup{VerbatimBorderColor={named}{nbsphinx-code-border}}
\begin{sphinxVerbatim}[commandchars=\\\{\}]
Primal Feasibility  Dual Feasibility    Duality Gap         Step             Path Parameter      Objective
1.0                 1.0                 1.0                 -                1.0                 506.4464461439
0.1717574295871     0.1717574295871     0.1717574295871     0.8319963473544  0.1717574295871     268.7406800963
0.06638577511032    0.06638577511032    0.06638577511033    0.6490768395246  0.06638577511033    376.5617800925
0.0108119991109     0.01081199911089    0.01081199911089    0.8457086649906  0.01081199911089    156.44831956
0.006166341998836   0.006166341998834   0.006166341998835   0.4575243211097  0.006166341998835   243.5697308399
0.0005157722017999  0.0005157722017955  0.0005157722017955  0.9303097249327  0.0005157722018036  353.933860755
7.752032087913e-08  7.75203208986e-08   7.752032089586e-08  0.9998584907938  7.75203208902e-08   354.2607816225
1.875610793654e-11  1.875609893145e-11  1.875610020548e-11  0.9997580492617  1.875613913502e-11  354.2617798007
Optimization terminated successfully.
         Current function value: 354.261780
         Iterations: 7
HIHIHIHIHI
Primal Feasibility  Dual Feasibility    Duality Gap         Step             Path Parameter      Objective
1.0                 1.0                 1.0                 -                1.0                 506.4464461439
0.1079826576971     0.1079826576971     0.1079826576971     0.8981297430914  0.1079826576971     201.9066839097
0.04065945390789    0.04065945390789    0.04065945390789    0.6593890232973  0.04065945390789    434.9407529189
0.01889246737726    0.01889246737727    0.01889246737727    0.5585864512458  0.01889246737727    712.8967272059
0.001989080291767   0.001989080291864   0.001989080291864   0.9101975479096  0.001989080291929   1386.629312789
2.319893722758e-06  2.31989372302e-06   2.319893722997e-06  0.9988740155508  2.319893723235e-06  1475.934762389
2.096073130738e-10  2.096071489352e-10  2.096071783066e-10  0.9999096479509  2.096074566229e-10  1476.090721225
Optimization terminated successfully.
         Current function value: 1476.090721
         Iterations: 6
Primal Feasibility  Dual Feasibility    Duality Gap         Step             Path Parameter      Objective
1.0                 1.0                 1.0                 -                1.0                 889.6273141593
0.06094445869208    0.06094445869208    0.06094445869209    0.9446346580036  0.06094445869208    507.5729159191
0.0140365256742     0.0140365256742     0.0140365256742     0.7774935142178  0.0140365256742     2.547633056773
0.004912856564238   0.004912856564238   0.004912856564239   0.6815812213596  0.004912856564237   14.44422608905
0.001078619945183   0.001078619945164   0.001078619945164   0.8144176821684  0.001078619945165   17.16956450895
0.0005055468167838  0.000505546816775   0.0005055468167751  0.5422993251434  0.0005055468167754  10.66487053069
0.0001219875076483  0.0001219875076462  0.0001219875076463  0.801041914532   0.0001219875076463  5.108155023129
2.275536452812e-05  2.275536452773e-05  2.275536452773e-05  0.8511017161659  2.275536452774e-05  4.268976835492
5.871314065763e-08  5.87131408013e-08   5.871314080683e-08  0.9974816389309  5.871314078161e-08  4.000401347274
2.935817263126e-12  2.935852912569e-12  2.935856570528e-12  0.9999499966411  2.935857873721e-12  3.999950673031
Optimization terminated successfully.
         Current function value: 3.999951
         Iterations: 9
HIHIHIHIHI
Primal Feasibility  Dual Feasibility    Duality Gap         Step             Path Parameter      Objective
1.0                 1.0                 1.0                 -                1.0                 888.9606557232
0.06089606977121    0.06089606977121    0.06089606977121    0.9428879411484  0.06089606977121    766.2404996017
0.005721125845739   0.005721125845738   0.005721125845739   0.9069719079827  0.005721125845738   5.33096424619
0.002161096200032   0.002161096200032   0.002161096200032   0.6566312030915  0.002161096200032   20.91332490724
0.0003725944226319  0.0003725944226319  0.0003725944226319  0.8524952692869  0.0003725944226319  8.881368524706
0.0001720989711016  0.0001720989711015  0.0001720989711015  0.5509232609834  0.0001720989711015  4.926154010814
1.220446310927e-05  1.22044631093e-05   1.220446310931e-05  1.0              1.220446310931e-05  0.246666757403
2.473680488043e-08  2.473680488002e-08  2.47368048808e-08   0.9984936714025  2.47368048808e-08   0.0001969335672683
1.907668751839e-12  1.907671450774e-12  1.90766909387e-12   0.9999228813462  1.90766909387e-12   1.551815937867e-08
5.529091076088e-13  5.529110976982e-13  5.529090987305e-13  0.7177319323253  5.529090987305e-13  5.40574204651e-09
2.202175165779e-13  2.202166746453e-13  2.202178545523e-13  0.6416601486717  2.202178545522e-13  2.996616136035e-08
6.244002250284e-14  6.244046003975e-14  6.244022524357e-14  0.7587549421636  6.244022524355e-14  8.360160383532e-08
Optimization terminated successfully.
         Current function value: 0.000000
         Iterations: 11
\end{sphinxVerbatim}
}

{

\kern-\sphinxverbatimsmallskipamount\kern-\baselineskip
\kern+\FrameHeightAdjust\kern-\fboxrule
\vspace{\nbsphinxcodecellspacing}

\sphinxsetup{VerbatimColor={named}{nbsphinx-stderr}}
\sphinxsetup{VerbatimBorderColor={named}{nbsphinx-code-border}}
\begin{sphinxVerbatim}[commandchars=\\\{\}]
-- 2020-11-09 15:09:28 - muse.mca - WARNING
Check growth constraints for wind.

\end{sphinxVerbatim}
}

{

\kern-\sphinxverbatimsmallskipamount\kern-\baselineskip
\kern+\FrameHeightAdjust\kern-\fboxrule
\vspace{\nbsphinxcodecellspacing}

\sphinxsetup{VerbatimColor={named}{white}}
\sphinxsetup{VerbatimBorderColor={named}{nbsphinx-code-border}}
\begin{sphinxVerbatim}[commandchars=\\\{\}]
Primal Feasibility  Dual Feasibility    Duality Gap         Step             Path Parameter      Objective
1.0                 1.0                 1.0                 -                1.0                 506.4464461439
0.1717574295871     0.1717574295871     0.1717574295871     0.8319963473544  0.1717574295871     268.7406800963
0.06638577511032    0.06638577511032    0.06638577511033    0.6490768395246  0.06638577511033    376.5617800925
0.0108119991109     0.01081199911089    0.01081199911089    0.8457086649906  0.01081199911089    156.44831956
0.006166341998836   0.006166341998834   0.006166341998835   0.4575243211097  0.006166341998835   243.5697308399
0.0005157722017999  0.0005157722017955  0.0005157722017955  0.9303097249327  0.0005157722018036  353.933860755
7.752032087913e-08  7.75203208986e-08   7.752032089586e-08  0.9998584907938  7.75203208902e-08   354.2607816225
1.875610793654e-11  1.875609893145e-11  1.875610020548e-11  0.9997580492617  1.875613913502e-11  354.2617798007
Optimization terminated successfully.
         Current function value: 354.261780
         Iterations: 7
HIHIHIHIHI
Primal Feasibility  Dual Feasibility    Duality Gap         Step             Path Parameter      Objective
1.0                 1.0                 1.0                 -                1.0                 506.4464461439
0.1079826576971     0.1079826576971     0.1079826576971     0.8981297430914  0.1079826576971     201.9066839097
0.04065945390789    0.04065945390789    0.04065945390789    0.6593890232973  0.04065945390789    434.9407529189
0.01889246737726    0.01889246737727    0.01889246737727    0.5585864512458  0.01889246737727    712.8967272059
0.001989080291767   0.001989080291864   0.001989080291864   0.9101975479096  0.001989080291929   1386.629312789
2.319893722758e-06  2.31989372302e-06   2.319893722997e-06  0.9988740155508  2.319893723235e-06  1475.934762389
2.096073130738e-10  2.096071489352e-10  2.096071783066e-10  0.9999096479509  2.096074566229e-10  1476.090721225
Optimization terminated successfully.
         Current function value: 1476.090721
         Iterations: 6
Primal Feasibility  Dual Feasibility    Duality Gap         Step             Path Parameter      Objective
1.0                 1.0                 1.0                 -                1.0                 889.6273141593
0.06094445869208    0.06094445869208    0.06094445869209    0.9446346580036  0.06094445869208    507.5729159191
0.0140365256742     0.0140365256742     0.0140365256742     0.7774935142178  0.0140365256742     2.547633056773
0.004912856564238   0.004912856564238   0.004912856564239   0.6815812213596  0.004912856564237   14.44422608905
0.001078619945183   0.001078619945164   0.001078619945164   0.8144176821684  0.001078619945165   17.16956450895
0.0005055468167838  0.000505546816775   0.0005055468167751  0.5422993251434  0.0005055468167754  10.66487053069
0.0001219875076483  0.0001219875076462  0.0001219875076463  0.801041914532   0.0001219875076463  5.108155023129
2.275536452812e-05  2.275536452773e-05  2.275536452773e-05  0.8511017161659  2.275536452774e-05  4.268976835492
5.871314065763e-08  5.87131408013e-08   5.871314080683e-08  0.9974816389309  5.871314078161e-08  4.000401347274
2.935817263126e-12  2.935852912569e-12  2.935856570528e-12  0.9999499966411  2.935857873721e-12  3.999950673031
Optimization terminated successfully.
         Current function value: 3.999951
         Iterations: 9
HIHIHIHIHI
Primal Feasibility  Dual Feasibility    Duality Gap         Step             Path Parameter      Objective
1.0                 1.0                 1.0                 -                1.0                 888.9606557232
0.06089606977121    0.06089606977121    0.06089606977121    0.9428879411484  0.06089606977121    766.2404996017
0.005721125845739   0.005721125845738   0.005721125845739   0.9069719079827  0.005721125845738   5.33096424619
0.002161096200032   0.002161096200032   0.002161096200032   0.6566312030915  0.002161096200032   20.91332490724
0.0003725944226319  0.0003725944226319  0.0003725944226319  0.8524952692869  0.0003725944226319  8.881368524706
0.0001720989711016  0.0001720989711015  0.0001720989711015  0.5509232609834  0.0001720989711015  4.926154010814
1.220446310927e-05  1.22044631093e-05   1.220446310931e-05  1.0              1.220446310931e-05  0.246666757403
2.473680488043e-08  2.473680488002e-08  2.47368048808e-08   0.9984936714025  2.47368048808e-08   0.0001969335672683
1.907668751839e-12  1.907671450774e-12  1.90766909387e-12   0.9999228813462  1.90766909387e-12   1.551815937867e-08
5.529091076088e-13  5.529110976982e-13  5.529090987305e-13  0.7177319323253  5.529090987305e-13  5.40574204651e-09
2.202175165779e-13  2.202166746453e-13  2.202178545523e-13  0.6416601486717  2.202178545522e-13  2.996616136035e-08
6.244002250284e-14  6.244046003975e-14  6.244022524357e-14  0.7587549421636  6.244022524355e-14  8.360160383532e-08
Optimization terminated successfully.
         Current function value: 0.000000
         Iterations: 11
\end{sphinxVerbatim}
}

{

\kern-\sphinxverbatimsmallskipamount\kern-\baselineskip
\kern+\FrameHeightAdjust\kern-\fboxrule
\vspace{\nbsphinxcodecellspacing}

\sphinxsetup{VerbatimColor={named}{nbsphinx-stderr}}
\sphinxsetup{VerbatimBorderColor={named}{nbsphinx-code-border}}
\begin{sphinxVerbatim}[commandchars=\\\{\}]
-- 2020-11-09 15:09:33 - muse.mca - WARNING
Check growth constraints for wind.

\end{sphinxVerbatim}
}

{

\kern-\sphinxverbatimsmallskipamount\kern-\baselineskip
\kern+\FrameHeightAdjust\kern-\fboxrule
\vspace{\nbsphinxcodecellspacing}

\sphinxsetup{VerbatimColor={named}{white}}
\sphinxsetup{VerbatimBorderColor={named}{nbsphinx-code-border}}
\begin{sphinxVerbatim}[commandchars=\\\{\}]
Primal Feasibility  Dual Feasibility    Duality Gap         Step             Path Parameter      Objective
1.0                 1.0                 1.0                 -                1.0                 568.8949309456
0.1574999940443     0.1574999940443     0.1574999940443     0.8462995490748  0.1574999940443     259.3692323549
0.07523425397025    0.07523425397025    0.07523425397025    0.5547720211262  0.07523425397025    453.5904211877
0.01015218722862    0.01015218722862    0.01015218722862    0.8831324835902  0.01015218722862    172.5109988087
0.005597873343747   0.005597873343747   0.005597873343747   0.4782155730769  0.005597873343747   257.1812475882
0.0004682435100659  0.0004682435100634  0.0004682435100634  0.9301215087302  0.000468243510069   357.0084554784
6.611502592752e-08  6.611502591615e-08  6.611502591873e-08  0.999863385888   6.611502594636e-08  355.7877874388
7.555502313764e-12  7.555508723405e-12  7.555505999397e-12  0.9998857217433  7.555493074812e-12  355.7884668282
Optimization terminated successfully.
         Current function value: 355.788467
         Iterations: 7
HIHIHIHIHI
Primal Feasibility  Dual Feasibility    Duality Gap         Step             Path Parameter      Objective
1.0                 1.0                 1.0                 -                1.0                 568.8949309456
0.07655552550188    0.07655552550188    0.07655552550188    0.9326533055761  0.07655552550188    158.9571104913
0.02915610316818    0.02915610316819    0.02915610316819    0.6583861263339  0.02915610316818    366.3114833866
0.01377238842158    0.01377238842158    0.01377238842158    0.5494198953479  0.01377238842158    509.1861831748
0.002113311162461   0.002113311162461   0.002113311162461   0.8672288939288  0.002113311162461   954.4881944043
1.802801647426e-06  1.802801647524e-06  1.802801647519e-06  0.9992160883437  1.802801646136e-06  1007.960176974
1.128093610417e-10  1.128094830078e-10  1.128094936008e-10  0.9999374254592  1.128097428918e-10  1008.067349764
Optimization terminated successfully.
         Current function value: 1008.067350
         Iterations: 6
Primal Feasibility  Dual Feasibility    Duality Gap         Step             Path Parameter      Objective
1.0                 1.0                 1.0                 -                1.0                 1027.252465794
0.06092556982108    0.06092556982105    0.06092556982104    0.9446936429103  0.06092556982104    639.5720130299
0.01301817113373    0.01301817113372    0.01301817113372    0.7926577624161  0.01301817113372    2.572827353367
0.004562995467295   0.004562995467293   0.004562995467292   0.6817945715562  0.004562995467292   18.38802438984
0.001187610205527   0.001187610205525   0.001187610205525   0.7731355396853  0.001187610205525   21.29230074562
0.0004286830611871  0.0004286830611869  0.0004286830611868  0.6428182053059  0.0004286830611868  11.09687380181
9.112758693237e-05  9.112758693242e-05  9.112758693242e-05  0.8412920104849  9.11275869324e-05   4.992047466728
2.247353056843e-05  2.247353056846e-05  2.247353056845e-05  0.7901238115153  2.247353056845e-05  4.333289150699
1.030728394581e-07  1.030728396726e-07  1.030728396812e-07  0.9955619145694  1.030728396813e-07  4.000897045216
5.155084903017e-12  5.15512049257e-12   5.155123215439e-12  0.9999499861783  5.155122339295e-12  3.99995066737
2.970831115849e-16  2.567650272094e-16  2.577287629196e-16  0.999949996482   2.580007603385e-16  3.99995062003
Optimization terminated successfully.
         Current function value: 3.999951
         Iterations: 10
HIHIHIHIHI
Primal Feasibility  Dual Feasibility    Duality Gap         Step             Path Parameter      Objective
1.0                 1.0                 1.0                 -                1.0                 513.6262328969
0.06112168433353    0.06112168433353    0.06112168433352    0.9426661283905  0.06112168433353    286.3443459917
0.01863946704738    0.01863946704738    0.01863946704738    0.7054291768553  0.01863946704738    1.912202845507
0.0064383053884     0.0064383053884     0.006438305388398   0.6923307144584  0.0064383053884     10.96992216863
0.0005492326542255  0.0005492326542115  0.0005492326542114  0.9375350483781  0.0005492326542115  9.941037067989
4.307299686793e-06  4.307299686644e-06  4.307299686641e-06  0.9947606446392  4.307299686645e-06  6.689178746935
2.155106691573e-10  2.155106735225e-10  2.155106567961e-10  0.9999499661842  2.15510656451e-10   6.666585491792
1.767668975714e-13  1.759148670162e-14  1.760582438766e-14  0.9999183071531  1.503270896521e-14  6.66658436143
Optimization terminated successfully.
         Current function value: 6.666584
         Iterations: 7
\end{sphinxVerbatim}
}

{

\kern-\sphinxverbatimsmallskipamount\kern-\baselineskip
\kern+\FrameHeightAdjust\kern-\fboxrule
\vspace{\nbsphinxcodecellspacing}

\sphinxsetup{VerbatimColor={named}{nbsphinx-stderr}}
\sphinxsetup{VerbatimBorderColor={named}{nbsphinx-code-border}}
\begin{sphinxVerbatim}[commandchars=\\\{\}]
-- 2020-11-09 15:09:39 - muse.mca - WARNING
Check growth constraints for wind.

\end{sphinxVerbatim}
}

{

\kern-\sphinxverbatimsmallskipamount\kern-\baselineskip
\kern+\FrameHeightAdjust\kern-\fboxrule
\vspace{\nbsphinxcodecellspacing}

\sphinxsetup{VerbatimColor={named}{white}}
\sphinxsetup{VerbatimBorderColor={named}{nbsphinx-code-border}}
\begin{sphinxVerbatim}[commandchars=\\\{\}]
Primal Feasibility  Dual Feasibility    Duality Gap         Step             Path Parameter      Objective
1.0                 1.0                 1.0                 -                1.0                 568.8949309456
0.1574999940443     0.1574999940443     0.1574999940443     0.8462995490748  0.1574999940443     259.3692323549
0.07523425397025    0.07523425397025    0.07523425397025    0.5547720211262  0.07523425397025    453.5904211877
0.01015218722862    0.01015218722862    0.01015218722862    0.8831324835902  0.01015218722862    172.5109988087
0.005597873343747   0.005597873343747   0.005597873343747   0.4782155730769  0.005597873343747   257.1812475882
0.0004682435100659  0.0004682435100634  0.0004682435100634  0.9301215087302  0.000468243510069   357.0084554784
6.611502592752e-08  6.611502591615e-08  6.611502591873e-08  0.999863385888   6.611502594636e-08  355.7877874388
7.555502313764e-12  7.555508723405e-12  7.555505999397e-12  0.9998857217433  7.555493074812e-12  355.7884668282
Optimization terminated successfully.
         Current function value: 355.788467
         Iterations: 7
HIHIHIHIHI
Primal Feasibility  Dual Feasibility    Duality Gap         Step             Path Parameter      Objective
1.0                 1.0                 1.0                 -                1.0                 568.8949309456
0.07655552550188    0.07655552550188    0.07655552550188    0.9326533055761  0.07655552550188    158.9571104913
0.02915610316818    0.02915610316819    0.02915610316819    0.6583861263339  0.02915610316818    366.3114833866
0.01377238842158    0.01377238842158    0.01377238842158    0.5494198953479  0.01377238842158    509.1861831748
0.002113311162461   0.002113311162461   0.002113311162461   0.8672288939288  0.002113311162461   954.4881944043
1.802801647426e-06  1.802801647524e-06  1.802801647519e-06  0.9992160883437  1.802801646136e-06  1007.960176974
1.128093610417e-10  1.128094830078e-10  1.128094936008e-10  0.9999374254592  1.128097428918e-10  1008.067349764
Optimization terminated successfully.
         Current function value: 1008.067350
         Iterations: 6
Primal Feasibility  Dual Feasibility    Duality Gap         Step             Path Parameter      Objective
1.0                 1.0                 1.0                 -                1.0                 1027.252465794
0.06092556982108    0.06092556982105    0.06092556982104    0.9446936429103  0.06092556982104    639.5720130299
0.01301817113373    0.01301817113372    0.01301817113372    0.7926577624161  0.01301817113372    2.572827353367
0.004562995467295   0.004562995467293   0.004562995467292   0.6817945715562  0.004562995467292   18.38802438984
0.001187610205527   0.001187610205525   0.001187610205525   0.7731355396853  0.001187610205525   21.29230074562
0.0004286830611871  0.0004286830611869  0.0004286830611868  0.6428182053059  0.0004286830611868  11.09687380181
9.112758693237e-05  9.112758693242e-05  9.112758693242e-05  0.8412920104849  9.11275869324e-05   4.992047466728
2.247353056843e-05  2.247353056846e-05  2.247353056845e-05  0.7901238115153  2.247353056845e-05  4.333289150699
1.030728394581e-07  1.030728396726e-07  1.030728396812e-07  0.9955619145694  1.030728396813e-07  4.000897045216
5.155084903017e-12  5.15512049257e-12   5.155123215439e-12  0.9999499861783  5.155122339295e-12  3.99995066737
2.970831115849e-16  2.567650272094e-16  2.577287629196e-16  0.999949996482   2.580007603385e-16  3.99995062003
Optimization terminated successfully.
         Current function value: 3.999951
         Iterations: 10
HIHIHIHIHI
Primal Feasibility  Dual Feasibility    Duality Gap         Step             Path Parameter      Objective
1.0                 1.0                 1.0                 -                1.0                 513.6262328969
0.06112168433353    0.06112168433353    0.06112168433352    0.9426661283905  0.06112168433353    286.3443459917
0.01863946704738    0.01863946704738    0.01863946704738    0.7054291768553  0.01863946704738    1.912202845507
0.0064383053884     0.0064383053884     0.006438305388398   0.6923307144584  0.0064383053884     10.96992216863
0.0005492326542255  0.0005492326542115  0.0005492326542114  0.9375350483781  0.0005492326542115  9.941037067989
4.307299686793e-06  4.307299686644e-06  4.307299686641e-06  0.9947606446392  4.307299686645e-06  6.689178746935
2.155106691573e-10  2.155106735225e-10  2.155106567961e-10  0.9999499661842  2.15510656451e-10   6.666585491792
1.767668975714e-13  1.759148670162e-14  1.760582438766e-14  0.9999183071531  1.503270896521e-14  6.66658436143
Optimization terminated successfully.
         Current function value: 6.666584
         Iterations: 7
\end{sphinxVerbatim}
}

{

\kern-\sphinxverbatimsmallskipamount\kern-\baselineskip
\kern+\FrameHeightAdjust\kern-\fboxrule
\vspace{\nbsphinxcodecellspacing}

\sphinxsetup{VerbatimColor={named}{nbsphinx-stderr}}
\sphinxsetup{VerbatimBorderColor={named}{nbsphinx-code-border}}
\begin{sphinxVerbatim}[commandchars=\\\{\}]
-- 2020-11-09 15:09:45 - muse.mca - WARNING
Check growth constraints for wind.

\end{sphinxVerbatim}
}

{

\kern-\sphinxverbatimsmallskipamount\kern-\baselineskip
\kern+\FrameHeightAdjust\kern-\fboxrule
\vspace{\nbsphinxcodecellspacing}

\sphinxsetup{VerbatimColor={named}{white}}
\sphinxsetup{VerbatimBorderColor={named}{nbsphinx-code-border}}
\begin{sphinxVerbatim}[commandchars=\\\{\}]
HIHIHIHIHI
Primal Feasibility  Dual Feasibility    Duality Gap         Step             Path Parameter      Objective
1.0                 1.0                 1.0                 -                1.0                 647.4402248427
0.1042702416317     0.1042702416317     0.1042702416317     0.9044651903652  0.1042702416317     197.7436068274
0.04330783922975    0.04330783922975    0.04330783922975    0.6213338928986  0.04330783922975    554.8458695929
0.02029339907042    0.02029339907042    0.02029339907042    0.5546164251713  0.02029339907042    887.503418141
0.002059630009005   0.002059630008646   0.002059630008646   0.9123945707354  0.002059630008855   1773.497308596
1.96164830749e-06   1.961648306624e-06  1.961648306623e-06  0.9995475315428  1.96164830693e-06   1899.795955905
1.183904657193e-10  1.183907029537e-10  1.183907180056e-10  0.9999396473343  1.183903667831e-10  1900.055708261
Optimization terminated successfully.
         Current function value: 1900.055708
         Iterations: 6
Primal Feasibility  Dual Feasibility    Duality Gap         Step             Path Parameter      Objective
1.0                 1.0                 1.0                 -                1.0                 1199.542590081
0.06080970203684    0.06080970203684    0.06080970203685    0.9448463893022  0.06080970203684    802.3521760538
0.01206790508619    0.01206790508619    0.01206790508619    0.8067148881887  0.01206790508618    2.242531350155
0.004235304009502   0.004235304009502   0.004235304009502   0.6829892140301  0.004235304009502   22.93309435161
0.001076107298245   0.001076107298268   0.001076107298268   0.7811880414727  0.001076107298267   27.34472817154
0.0001381467026663  0.0001381467026694  0.0001381467026694  0.9534954297556  0.0001381467026692  2.164474733354
0.0001283068198708  0.0001283068198737  0.0001283068198737  0.07280278532081 0.0001283068198735  2.0117557724
1.709535384529e-06  1.709535384566e-06  1.709535384568e-06  0.9870423815898  1.709535384566e-06  0.04804313579566
9.45480215837e-11   9.454801752592e-11  9.454801899378e-11  0.9999447725614  9.454801899363e-11  2.655850431687e-06
8.641189470312e-12  8.641191779805e-12  8.641199906593e-12  0.9099589255942  8.64119990658e-12   2.42972823089e-07
8.227204364827e-12  8.227212664943e-12  8.22721600211e-12   0.05277665620137 8.227216002098e-12  2.322662502381e-07
1.085487238602e-12  1.085487876362e-12  1.085489602398e-12  0.8709818852506  1.085489602396e-12  3.141737288768e-08
7.818953396427e-13  7.81893890829e-13   7.818962895149e-13  0.3024720499645  7.818962895138e-13  2.33935913105e-08
2.33759416086e-13   2.337579557771e-13  2.337613758542e-13  0.7142468606499  2.337613758539e-13  9.290715381741e-09
Optimization terminated successfully.
         Current function value: 0.000000
         Iterations: 13
HIHIHIHIHI
Primal Feasibility  Dual Feasibility    Duality Gap         Step             Path Parameter      Objective
1.0                 1.0                 1.0                 -                1.0                 599.7712950406
0.06106893109436    0.06106893109436    0.06106893109435    0.9427080518507  0.06106893109436    369.2205608767
0.0172839506186     0.0172839506186     0.0172839506186     0.7253158185948  0.0172839506186     1.815008327441
0.006506405689011   0.006506405689011   0.00650640568901    0.660706520954   0.006506405689012   13.26117009766
0.0007534849804658  0.0007534849804373  0.0007534849804372  0.9127389540888  0.0007534849804374  13.3711836365
0.0002310664967685  0.0002310664967596  0.0002310664967596  0.7145302516257  0.0002310664967597  7.652720903336
1.357431783524e-06  1.35743178356e-06   1.357431783558e-06  1.0              1.357431783551e-06  5.273685381749
7.205799155642e-11  7.205807383533e-11  7.205806647959e-11  0.9999469527435  7.205806692488e-11  5.266601996919
3.494750239057e-15  3.599609449551e-15  3.605486637761e-15  0.9999499638259  3.603660446069e-15  5.26660163652
Optimization terminated successfully.
         Current function value: 5.266602
         Iterations: 8
\end{sphinxVerbatim}
}

{

\kern-\sphinxverbatimsmallskipamount\kern-\baselineskip
\kern+\FrameHeightAdjust\kern-\fboxrule
\vspace{\nbsphinxcodecellspacing}

\sphinxsetup{VerbatimColor={named}{nbsphinx-stderr}}
\sphinxsetup{VerbatimBorderColor={named}{nbsphinx-code-border}}
\begin{sphinxVerbatim}[commandchars=\\\{\}]
-- 2020-11-09 15:09:50 - muse.mca - WARNING
Check growth constraints for wind.

\end{sphinxVerbatim}
}

{

\kern-\sphinxverbatimsmallskipamount\kern-\baselineskip
\kern+\FrameHeightAdjust\kern-\fboxrule
\vspace{\nbsphinxcodecellspacing}

\sphinxsetup{VerbatimColor={named}{white}}
\sphinxsetup{VerbatimBorderColor={named}{nbsphinx-code-border}}
\begin{sphinxVerbatim}[commandchars=\\\{\}]
HIHIHIHIHI
Primal Feasibility  Dual Feasibility    Duality Gap         Step             Path Parameter      Objective
1.0                 1.0                 1.0                 -                1.0                 647.4402248427
0.1042702416317     0.1042702416317     0.1042702416317     0.9044651903652  0.1042702416317     197.7436068274
0.04330783922975    0.04330783922975    0.04330783922975    0.6213338928986  0.04330783922975    554.8458695929
0.02029339907042    0.02029339907042    0.02029339907042    0.5546164251713  0.02029339907042    887.503418141
0.002059630009005   0.002059630008646   0.002059630008646   0.9123945707354  0.002059630008855   1773.497308596
1.96164830749e-06   1.961648306624e-06  1.961648306623e-06  0.9995475315428  1.96164830693e-06   1899.795955905
1.183904657193e-10  1.183907029537e-10  1.183907180056e-10  0.9999396473343  1.183903667831e-10  1900.055708261
Optimization terminated successfully.
         Current function value: 1900.055708
         Iterations: 6
Primal Feasibility  Dual Feasibility    Duality Gap         Step             Path Parameter      Objective
1.0                 1.0                 1.0                 -                1.0                 1199.542590081
0.06080970203684    0.06080970203684    0.06080970203685    0.9448463893022  0.06080970203684    802.3521760538
0.01206790508619    0.01206790508619    0.01206790508619    0.8067148881887  0.01206790508618    2.242531350155
0.004235304009502   0.004235304009502   0.004235304009502   0.6829892140301  0.004235304009502   22.93309435161
0.001076107298245   0.001076107298268   0.001076107298268   0.7811880414727  0.001076107298267   27.34472817154
0.0001381467026663  0.0001381467026694  0.0001381467026694  0.9534954297556  0.0001381467026692  2.164474733354
0.0001283068198708  0.0001283068198737  0.0001283068198737  0.07280278532081 0.0001283068198735  2.0117557724
1.709535384529e-06  1.709535384566e-06  1.709535384568e-06  0.9870423815898  1.709535384566e-06  0.04804313579566
9.45480215837e-11   9.454801752592e-11  9.454801899378e-11  0.9999447725614  9.454801899363e-11  2.655850431687e-06
8.641189470312e-12  8.641191779805e-12  8.641199906593e-12  0.9099589255942  8.64119990658e-12   2.42972823089e-07
8.227204364827e-12  8.227212664943e-12  8.22721600211e-12   0.05277665620137 8.227216002098e-12  2.322662502381e-07
1.085487238602e-12  1.085487876362e-12  1.085489602398e-12  0.8709818852506  1.085489602396e-12  3.141737288768e-08
7.818953396427e-13  7.81893890829e-13   7.818962895149e-13  0.3024720499645  7.818962895138e-13  2.33935913105e-08
2.33759416086e-13   2.337579557771e-13  2.337613758542e-13  0.7142468606499  2.337613758539e-13  9.290715381741e-09
Optimization terminated successfully.
         Current function value: 0.000000
         Iterations: 13
HIHIHIHIHI
Primal Feasibility  Dual Feasibility    Duality Gap         Step             Path Parameter      Objective
1.0                 1.0                 1.0                 -                1.0                 599.7712950406
0.06106893109436    0.06106893109436    0.06106893109435    0.9427080518507  0.06106893109436    369.2205608767
0.0172839506186     0.0172839506186     0.0172839506186     0.7253158185948  0.0172839506186     1.815008327441
0.006506405689011   0.006506405689011   0.00650640568901    0.660706520954   0.006506405689012   13.26117009766
0.0007534849804658  0.0007534849804373  0.0007534849804372  0.9127389540888  0.0007534849804374  13.3711836365
0.0002310664967685  0.0002310664967596  0.0002310664967596  0.7145302516257  0.0002310664967597  7.652720903336
1.357431783524e-06  1.35743178356e-06   1.357431783558e-06  1.0              1.357431783551e-06  5.273685381749
7.205799155642e-11  7.205807383533e-11  7.205806647959e-11  0.9999469527435  7.205806692488e-11  5.266601996919
3.494750239057e-15  3.599609449551e-15  3.605486637761e-15  0.9999499638259  3.603660446069e-15  5.26660163652
Optimization terminated successfully.
         Current function value: 5.266602
         Iterations: 8
\end{sphinxVerbatim}
}

{

\kern-\sphinxverbatimsmallskipamount\kern-\baselineskip
\kern+\FrameHeightAdjust\kern-\fboxrule
\vspace{\nbsphinxcodecellspacing}

\sphinxsetup{VerbatimColor={named}{nbsphinx-stderr}}
\sphinxsetup{VerbatimBorderColor={named}{nbsphinx-code-border}}
\begin{sphinxVerbatim}[commandchars=\\\{\}]
-- 2020-11-09 15:09:55 - muse.mca - WARNING
Check growth constraints for wind.

\end{sphinxVerbatim}
}

{
\sphinxsetup{VerbatimColor={named}{nbsphinx-code-bg}}
\sphinxsetup{VerbatimBorderColor={named}{nbsphinx-code-border}}
\begin{sphinxVerbatim}[commandchars=\\\{\}]
\llap{\color{nbsphinxin}[8]:\,\hspace{\fboxrule}\hspace{\fboxsep}}\PYG{k+kn}{import} \PYG{n+nn}{pandas} \PYG{k}{as} \PYG{n+nn}{pd}
\PYG{k+kn}{import} \PYG{n+nn}{seaborn} \PYG{k}{as} \PYG{n+nn}{sns}
\PYG{k+kn}{import} \PYG{n+nn}{matplotlib}\PYG{n+nn}{.}\PYG{n+nn}{pyplot}

\PYG{n}{results} \PYG{o}{=} \PYG{n}{pd}\PYG{o}{.}\PYG{n}{read\PYGZus{}csv}\PYG{p}{(}\PYG{l+s+s2}{\PYGZdq{}}\PYG{l+s+s2}{Results/MCACapacity.csv}\PYG{l+s+s2}{\PYGZdq{}}\PYG{p}{)}
\PYG{n}{results}
\PYG{n}{sns}\PYG{o}{.}\PYG{n}{lineplot}\PYG{p}{(}\PYG{n}{data}\PYG{o}{=}\PYG{n}{results}\PYG{p}{[}\PYG{n}{results}\PYG{o}{.}\PYG{n}{sector}\PYG{o}{==}\PYG{l+s+s2}{\PYGZdq{}}\PYG{l+s+s2}{power}\PYG{l+s+s2}{\PYGZdq{}}\PYG{p}{]}\PYG{p}{,} \PYG{n}{x}\PYG{o}{=}\PYG{l+s+s1}{\PYGZsq{}}\PYG{l+s+s1}{year}\PYG{l+s+s1}{\PYGZsq{}}\PYG{p}{,} \PYG{n}{y}\PYG{o}{=}\PYG{l+s+s1}{\PYGZsq{}}\PYG{l+s+s1}{capacity}\PYG{l+s+s1}{\PYGZsq{}}\PYG{p}{,} \PYG{n}{hue}\PYG{o}{=}\PYG{l+s+s1}{\PYGZsq{}}\PYG{l+s+s1}{technology}\PYG{l+s+s1}{\PYGZsq{}}\PYG{p}{)}
\end{sphinxVerbatim}
}

{

\kern-\sphinxverbatimsmallskipamount\kern-\baselineskip
\kern+\FrameHeightAdjust\kern-\fboxrule
\vspace{\nbsphinxcodecellspacing}

\sphinxsetup{VerbatimColor={named}{white}}
\sphinxsetup{VerbatimBorderColor={named}{nbsphinx-code-border}}
\begin{sphinxVerbatim}[commandchars=\\\{\}]
\llap{\color{nbsphinxout}[8]:\,\hspace{\fboxrule}\hspace{\fboxsep}}<matplotlib.axes.\_subplots.AxesSubplot at 0x7fe8f77df460>
\end{sphinxVerbatim}
}

\hrule height -\fboxrule\relax
\vspace{\nbsphinxcodecellspacing}

\makeatletter\setbox\nbsphinxpromptbox\box\voidb@x\makeatother

\begin{nbsphinxfancyoutput}

\noindent\sphinxincludegraphics[width=382\sphinxpxdimen,height=262\sphinxpxdimen]{{advanced-guide_further-extending-muse_3_1}.png}

\end{nbsphinxfancyoutput}


\section{Indices and tables}
\label{\detokenize{advanced-guide/index:indices-and-tables}}\begin{itemize}
\item {} 
\DUrole{xref,std,std-ref}{genindex}

\item {} 
\DUrole{xref,std,std-ref}{modindex}

\item {} 
\DUrole{xref,std,std-ref}{search}

\end{itemize}


\chapter{API}
\label{\detokenize{api:module-muse}}\label{\detokenize{api:api}}\label{\detokenize{api::doc}}\index{module@\spxentry{module}!muse@\spxentry{muse}}\index{muse@\spxentry{muse}!module@\spxentry{module}}
MUSE model.


\section{Market Clearing Algorithm}
\label{\detokenize{api:market-clearing-algorithm}}

\subsection{Main MCA}
\label{\detokenize{api:module-muse.mca}}\label{\detokenize{api:main-mca}}\index{module@\spxentry{module}!muse.mca@\spxentry{muse.mca}}\index{muse.mca@\spxentry{muse.mca}!module@\spxentry{module}}\index{FindEquilibriumResults (class in muse.mca)@\spxentry{FindEquilibriumResults}\spxextra{class in muse.mca}}

\begin{fulllineitems}
\phantomsection\label{\detokenize{api:muse.mca.FindEquilibriumResults}}\pysiglinewithargsret{\sphinxbfcode{\sphinxupquote{class }}\sphinxcode{\sphinxupquote{muse.mca.}}\sphinxbfcode{\sphinxupquote{FindEquilibriumResults}}}{\emph{\DUrole{n}{converged}\DUrole{p}{:} \DUrole{n}{bool}}, \emph{\DUrole{n}{market}\DUrole{p}{:} \DUrole{n}{Dataset}}, \emph{\DUrole{n}{sectors}\DUrole{p}{:} \DUrole{n}{List\DUrole{p}{{[}}AbstractSector\DUrole{p}{{]}}}}}{}
Result of find equilibrium.
\index{converged (muse.mca.FindEquilibriumResults attribute)@\spxentry{converged}\spxextra{muse.mca.FindEquilibriumResults attribute}}

\begin{fulllineitems}
\phantomsection\label{\detokenize{api:muse.mca.FindEquilibriumResults.converged}}\pysigline{\sphinxbfcode{\sphinxupquote{converged}}\sphinxbfcode{\sphinxupquote{: bool}}}
Alias for field number 0

\end{fulllineitems}

\index{market (muse.mca.FindEquilibriumResults attribute)@\spxentry{market}\spxextra{muse.mca.FindEquilibriumResults attribute}}

\begin{fulllineitems}
\phantomsection\label{\detokenize{api:muse.mca.FindEquilibriumResults.market}}\pysigline{\sphinxbfcode{\sphinxupquote{market}}\sphinxbfcode{\sphinxupquote{: Dataset}}}
Alias for field number 1

\end{fulllineitems}

\index{sectors (muse.mca.FindEquilibriumResults attribute)@\spxentry{sectors}\spxextra{muse.mca.FindEquilibriumResults attribute}}

\begin{fulllineitems}
\phantomsection\label{\detokenize{api:muse.mca.FindEquilibriumResults.sectors}}\pysigline{\sphinxbfcode{\sphinxupquote{sectors}}\sphinxbfcode{\sphinxupquote{: List\DUrole{p}{{[}}AbstractSector\DUrole{p}{{]}}}}}
Alias for field number 2

\end{fulllineitems}


\end{fulllineitems}

\index{MCA (class in muse.mca)@\spxentry{MCA}\spxextra{class in muse.mca}}

\begin{fulllineitems}
\phantomsection\label{\detokenize{api:muse.mca.MCA}}\pysiglinewithargsret{\sphinxbfcode{\sphinxupquote{class }}\sphinxcode{\sphinxupquote{muse.mca.}}\sphinxbfcode{\sphinxupquote{MCA}}}{\emph{\DUrole{n}{sectors}\DUrole{p}{:} \DUrole{n}{List\DUrole{p}{{[}}muse.sectors.abstract.AbstractSector\DUrole{p}{{]}}}}, \emph{\DUrole{n}{market}\DUrole{p}{:} \DUrole{n}{xarray.core.dataset.Dataset}}, \emph{\DUrole{n}{outputs}\DUrole{p}{:} \DUrole{n}{Optional\DUrole{p}{{[}}Callable\DUrole{p}{{[}}\DUrole{p}{{[}}List\DUrole{p}{{[}}muse.sectors.abstract.AbstractSector\DUrole{p}{{]}}\DUrole{p}{, }xarray.core.dataset.Dataset\DUrole{p}{{]}}\DUrole{p}{, }Any\DUrole{p}{{]}}\DUrole{p}{{]}}} \DUrole{o}{=} \DUrole{default_value}{None}}, \emph{\DUrole{n}{time\_framework}\DUrole{p}{:} \DUrole{n}{Sequence\DUrole{p}{{[}}int\DUrole{p}{{]}}} \DUrole{o}{=} \DUrole{default_value}{{[}2010, 2020, 2030, 2040, 2050, 2060, 2070, 2080, 2090{]}}}, \emph{\DUrole{n}{equilibrium}\DUrole{p}{:} \DUrole{n}{bool} \DUrole{o}{=} \DUrole{default_value}{True}}, \emph{\DUrole{n}{expect\_equilibrium}\DUrole{p}{:} \DUrole{n}{bool} \DUrole{o}{=} \DUrole{default_value}{True}}, \emph{\DUrole{n}{equilibrium\_variable}\DUrole{p}{:} \DUrole{n}{str} \DUrole{o}{=} \DUrole{default_value}{\textquotesingle{}demand\textquotesingle{}}}, \emph{\DUrole{n}{maximum\_iterations}\DUrole{p}{:} \DUrole{n}{int} \DUrole{o}{=} \DUrole{default_value}{3}}, \emph{\DUrole{n}{tolerance}\DUrole{p}{:} \DUrole{n}{float} \DUrole{o}{=} \DUrole{default_value}{0.1}}, \emph{\DUrole{n}{tolerance\_unmet\_demand}\DUrole{p}{:} \DUrole{n}{float} \DUrole{o}{=} \DUrole{default_value}{\sphinxhyphen{} 0.1}}, \emph{\DUrole{n}{excluded\_commodities}\DUrole{p}{:} \DUrole{n}{Optional\DUrole{p}{{[}}Sequence\DUrole{p}{{[}}str\DUrole{p}{{]}}\DUrole{p}{{]}}} \DUrole{o}{=} \DUrole{default_value}{None}}, \emph{\DUrole{n}{carbon\_budget}\DUrole{p}{:} \DUrole{n}{Optional\DUrole{p}{{[}}Sequence\DUrole{p}{{]}}} \DUrole{o}{=} \DUrole{default_value}{None}}, \emph{\DUrole{n}{carbon\_price}\DUrole{p}{:} \DUrole{n}{Optional\DUrole{p}{{[}}Sequence\DUrole{p}{{]}}} \DUrole{o}{=} \DUrole{default_value}{None}}, \emph{\DUrole{n}{carbon\_commodities}\DUrole{p}{:} \DUrole{n}{Optional\DUrole{p}{{[}}Sequence\DUrole{p}{{[}}str\DUrole{p}{{]}}\DUrole{p}{{]}}} \DUrole{o}{=} \DUrole{default_value}{None}}, \emph{\DUrole{n}{debug}\DUrole{p}{:} \DUrole{n}{bool} \DUrole{o}{=} \DUrole{default_value}{False}}, \emph{\DUrole{n}{control\_undershoot}\DUrole{p}{:} \DUrole{n}{bool} \DUrole{o}{=} \DUrole{default_value}{True}}, \emph{\DUrole{n}{control\_overshoot}\DUrole{p}{:} \DUrole{n}{bool} \DUrole{o}{=} \DUrole{default_value}{True}}, \emph{\DUrole{n}{carbon\_method}\DUrole{p}{:} \DUrole{n}{str} \DUrole{o}{=} \DUrole{default_value}{\textquotesingle{}fitting\textquotesingle{}}}, \emph{\DUrole{n}{method\_options}\DUrole{p}{:} \DUrole{n}{Optional\DUrole{p}{{[}}Mapping\DUrole{p}{{]}}} \DUrole{o}{=} \DUrole{default_value}{None}}}{}
Market Clearing Algorithm.

The market clearing algorithm is the main object implementing the MUSE model. It is
responsible for orchestrating  how the sectors are run, how they interface with one
another, with the general market and the carbon market.
\index{calibrate\_legacy\_sectors() (muse.mca.MCA method)@\spxentry{calibrate\_legacy\_sectors()}\spxextra{muse.mca.MCA method}}

\begin{fulllineitems}
\phantomsection\label{\detokenize{api:muse.mca.MCA.calibrate_legacy_sectors}}\pysiglinewithargsret{\sphinxbfcode{\sphinxupquote{calibrate\_legacy\_sectors}}}{}{}
Run a calibration step in the lagacy sectors.

TODO: Remove when LegacySectors are no longer needed.

\end{fulllineitems}

\index{factory() (muse.mca.MCA class method)@\spxentry{factory()}\spxextra{muse.mca.MCA class method}}

\begin{fulllineitems}
\phantomsection\label{\detokenize{api:muse.mca.MCA.factory}}\pysiglinewithargsret{\sphinxbfcode{\sphinxupquote{classmethod }}\sphinxbfcode{\sphinxupquote{factory}}}{\emph{\DUrole{n}{settings}\DUrole{p}{:} \DUrole{n}{Union\DUrole{p}{{[}}str\DUrole{p}{, }pathlib.Path\DUrole{p}{, }Mapping\DUrole{p}{, }Any\DUrole{p}{{]}}}}}{{ $\rightarrow$ muse.mca.MCA}}
Loads MCA from input settings and input files.
\begin{quote}\begin{description}
\item[{Parameters}] \leavevmode
\sphinxstyleliteralstrong{\sphinxupquote{settings}} \textendash{} namedtuple with the global MUSE input settings.

\item[{Returns}] \leavevmode
The loaded MCA

\end{description}\end{quote}

\end{fulllineitems}

\index{find\_equilibrium() (muse.mca.MCA method)@\spxentry{find\_equilibrium()}\spxextra{muse.mca.MCA method}}

\begin{fulllineitems}
\phantomsection\label{\detokenize{api:muse.mca.MCA.find_equilibrium}}\pysiglinewithargsret{\sphinxbfcode{\sphinxupquote{find\_equilibrium}}}{\emph{\DUrole{n}{market}\DUrole{p}{:} \DUrole{n}{xarray.core.dataset.Dataset}}, \emph{\DUrole{n}{sectors}\DUrole{p}{:} \DUrole{n}{Optional\DUrole{p}{{[}}List\DUrole{p}{{[}}muse.sectors.abstract.AbstractSector\DUrole{p}{{]}}\DUrole{p}{{]}}} \DUrole{o}{=} \DUrole{default_value}{None}}}{{ $\rightarrow$ muse.mca.FindEquilibriumResults}}
Specialised version of the find\_equilibrium function.
\begin{quote}\begin{description}
\item[{Parameters}] \leavevmode
\sphinxstyleliteralstrong{\sphinxupquote{market}} \textendash{} Commodities market, with the prices, supply, consumption and demand.

\item[{Returns}] \leavevmode
A tuple with the updated market (prices, supply, consumption and demand) and
sector.

\end{description}\end{quote}

\end{fulllineitems}

\index{run() (muse.mca.MCA method)@\spxentry{run()}\spxextra{muse.mca.MCA method}}

\begin{fulllineitems}
\phantomsection\label{\detokenize{api:muse.mca.MCA.run}}\pysiglinewithargsret{\sphinxbfcode{\sphinxupquote{run}}}{}{{ $\rightarrow$ None}}
Initiates the calculation, starting with the loop over years.

This method starts the main MUSE loop, going over the years of the simulation.
Internally, it runs the carbon budget loop, which updates the carbon prices, if
needed, and the equilibrium loop, which tries to reach an equilibrium between
prices, demand and supply.
\begin{quote}\begin{description}
\item[{Returns}] \leavevmode
None

\end{description}\end{quote}

\end{fulllineitems}

\index{update\_carbon\_budget() (muse.mca.MCA method)@\spxentry{update\_carbon\_budget()}\spxextra{muse.mca.MCA method}}

\begin{fulllineitems}
\phantomsection\label{\detokenize{api:muse.mca.MCA.update_carbon_budget}}\pysiglinewithargsret{\sphinxbfcode{\sphinxupquote{update\_carbon\_budget}}}{\emph{\DUrole{n}{market}\DUrole{p}{:} \DUrole{n}{xarray.core.dataset.Dataset}}, \emph{\DUrole{n}{year\_idx}\DUrole{p}{:} \DUrole{n}{int}}}{{ $\rightarrow$ float}}
Specialised version of the update\_carbon\_budget function.
\begin{quote}\begin{description}
\item[{Parameters}] \leavevmode\begin{itemize}
\item {} 
\sphinxstyleliteralstrong{\sphinxupquote{market}} \textendash{} Commodities market, with the prices, supply, consumption and demand.

\item {} 
\sphinxstyleliteralstrong{\sphinxupquote{year\_idx}} \textendash{} Index of the year of interest.

\end{itemize}

\item[{Returns}] \leavevmode
An updated market with prices, supply, consumption and demand.

\end{description}\end{quote}

\end{fulllineitems}

\index{update\_carbon\_price() (muse.mca.MCA method)@\spxentry{update\_carbon\_price()}\spxextra{muse.mca.MCA method}}

\begin{fulllineitems}
\phantomsection\label{\detokenize{api:muse.mca.MCA.update_carbon_price}}\pysiglinewithargsret{\sphinxbfcode{\sphinxupquote{update\_carbon\_price}}}{\emph{\DUrole{n}{market}}}{{ $\rightarrow$ Optional\DUrole{p}{{[}}float\DUrole{p}{{]}}}}
Calculates the updated carbon price, if required.

If the emission calculated for the next time period is larger than the
limit, then the carbon price needs to be updated to ensure that whatever the
sectors do, the carbon budget limit is not exceeded.
\begin{quote}\begin{description}
\item[{Parameters}] \leavevmode
\sphinxstyleliteralstrong{\sphinxupquote{market}} \textendash{} Market, with the prices, supply, consumption and demand.

\item[{Returns}] \leavevmode
The new carbon price or None

\end{description}\end{quote}

\end{fulllineitems}


\end{fulllineitems}

\index{SingleYearIterationResult (class in muse.mca)@\spxentry{SingleYearIterationResult}\spxextra{class in muse.mca}}

\begin{fulllineitems}
\phantomsection\label{\detokenize{api:muse.mca.SingleYearIterationResult}}\pysiglinewithargsret{\sphinxbfcode{\sphinxupquote{class }}\sphinxcode{\sphinxupquote{muse.mca.}}\sphinxbfcode{\sphinxupquote{SingleYearIterationResult}}}{\emph{\DUrole{n}{market}\DUrole{p}{:} \DUrole{n}{Dataset}}, \emph{\DUrole{n}{sectors}\DUrole{p}{:} \DUrole{n}{List\DUrole{p}{{[}}AbstractSector\DUrole{p}{{]}}}}}{}
Result of iterating over sectors for a year.

Convenience tuple naming naming the return values from  of
\sphinxcode{\sphinxupquote{single\_year\_iteration()}}.
\index{market (muse.mca.SingleYearIterationResult attribute)@\spxentry{market}\spxextra{muse.mca.SingleYearIterationResult attribute}}

\begin{fulllineitems}
\phantomsection\label{\detokenize{api:muse.mca.SingleYearIterationResult.market}}\pysigline{\sphinxbfcode{\sphinxupquote{market}}\sphinxbfcode{\sphinxupquote{: Dataset}}}
Alias for field number 0

\end{fulllineitems}

\index{sectors (muse.mca.SingleYearIterationResult attribute)@\spxentry{sectors}\spxextra{muse.mca.SingleYearIterationResult attribute}}

\begin{fulllineitems}
\phantomsection\label{\detokenize{api:muse.mca.SingleYearIterationResult.sectors}}\pysigline{\sphinxbfcode{\sphinxupquote{sectors}}\sphinxbfcode{\sphinxupquote{: List\DUrole{p}{{[}}AbstractSector\DUrole{p}{{]}}}}}
Alias for field number 1

\end{fulllineitems}


\end{fulllineitems}

\index{check\_demand\_fulfillment() (in module muse.mca)@\spxentry{check\_demand\_fulfillment()}\spxextra{in module muse.mca}}

\begin{fulllineitems}
\phantomsection\label{\detokenize{api:muse.mca.check_demand_fulfillment}}\pysiglinewithargsret{\sphinxcode{\sphinxupquote{muse.mca.}}\sphinxbfcode{\sphinxupquote{check\_demand\_fulfillment}}}{\emph{\DUrole{n}{market}\DUrole{p}{:} \DUrole{n}{xarray.core.dataset.Dataset}}, \emph{\DUrole{n}{tol}\DUrole{p}{:} \DUrole{n}{float}}, \emph{\DUrole{n}{excluded\_commodities}\DUrole{p}{:} \DUrole{n}{Optional\DUrole{p}{{[}}Sequence\DUrole{p}{{]}}} \DUrole{o}{=} \DUrole{default_value}{None}}}{{ $\rightarrow$ bool}}
Checks if the supply will fulfill all the demand in the future.

If it does not, it logs a warning.
\begin{quote}\begin{description}
\item[{Parameters}] \leavevmode\begin{itemize}
\item {} 
\sphinxstyleliteralstrong{\sphinxupquote{market}} \textendash{} Commodities market, with the prices, supply, consumption and demand.

\item {} 
\sphinxstyleliteralstrong{\sphinxupquote{tol}} \textendash{} Tolerance for the unmet demand.

\end{itemize}

\item[{Returns}] \leavevmode
True if the supply fulfils the demand; False otherwise

\end{description}\end{quote}

\end{fulllineitems}

\index{check\_equilibrium() (in module muse.mca)@\spxentry{check\_equilibrium()}\spxextra{in module muse.mca}}

\begin{fulllineitems}
\phantomsection\label{\detokenize{api:muse.mca.check_equilibrium}}\pysiglinewithargsret{\sphinxcode{\sphinxupquote{muse.mca.}}\sphinxbfcode{\sphinxupquote{check\_equilibrium}}}{\emph{\DUrole{n}{market}\DUrole{p}{:} \DUrole{n}{xarray.core.dataset.Dataset}}, \emph{\DUrole{n}{int\_market}\DUrole{p}{:} \DUrole{n}{xarray.core.dataset.Dataset}}, \emph{\DUrole{n}{tolerance}\DUrole{p}{:} \DUrole{n}{float}}, \emph{\DUrole{n}{equilibrium\_variable}\DUrole{p}{:} \DUrole{n}{str}}, \emph{\DUrole{n}{year}\DUrole{p}{:} \DUrole{n}{Optional\DUrole{p}{{[}}int\DUrole{p}{{]}}} \DUrole{o}{=} \DUrole{default_value}{None}}}{{ $\rightarrow$ bool}}
Checks if equilibrium has been reached.

This function checks if the difference in either the demand or the prices between
iterations if smaller than certain tolerance. If is, then it is assumed that
the process has converged.
\begin{quote}\begin{description}
\item[{Parameters}] \leavevmode\begin{itemize}
\item {} 
\sphinxstyleliteralstrong{\sphinxupquote{market}} \textendash{} The market values in this iteration.

\item {} 
\sphinxstyleliteralstrong{\sphinxupquote{int\_market}} \textendash{} The market values in the previous iteration.

\item {} 
\sphinxstyleliteralstrong{\sphinxupquote{tolerance}} \textendash{} Tolerance for reaching equilibrium.

\item {} 
\sphinxstyleliteralstrong{\sphinxupquote{equilibrium\_variable}} \textendash{} Variable to use to calculate the equilibrium condition.

\item {} 
\sphinxstyleliteralstrong{\sphinxupquote{year}} \textendash{} year for which to check changes. Default to minimum year in market.

\end{itemize}

\item[{Returns}] \leavevmode
True if converged, False otherwise.

\end{description}\end{quote}

\end{fulllineitems}

\index{find\_equilibrium() (in module muse.mca)@\spxentry{find\_equilibrium()}\spxextra{in module muse.mca}}

\begin{fulllineitems}
\phantomsection\label{\detokenize{api:muse.mca.find_equilibrium}}\pysiglinewithargsret{\sphinxcode{\sphinxupquote{muse.mca.}}\sphinxbfcode{\sphinxupquote{find\_equilibrium}}}{\emph{\DUrole{n}{market}\DUrole{p}{:} \DUrole{n}{xarray.core.dataset.Dataset}}, \emph{\DUrole{n}{sectors}\DUrole{p}{:} \DUrole{n}{List\DUrole{p}{{[}}muse.sectors.abstract.AbstractSector\DUrole{p}{{]}}}}, \emph{\DUrole{n}{maxiter}\DUrole{p}{:} \DUrole{n}{int} \DUrole{o}{=} \DUrole{default_value}{3}}, \emph{\DUrole{n}{tol}\DUrole{p}{:} \DUrole{n}{float} \DUrole{o}{=} \DUrole{default_value}{0.1}}, \emph{\DUrole{n}{equilibrium\_variable}\DUrole{p}{:} \DUrole{n}{str} \DUrole{o}{=} \DUrole{default_value}{\textquotesingle{}demand\textquotesingle{}}}, \emph{\DUrole{n}{tol\_unmet\_demand}\DUrole{p}{:} \DUrole{n}{float} \DUrole{o}{=} \DUrole{default_value}{\sphinxhyphen{} 0.1}}, \emph{\DUrole{n}{excluded\_commodities}\DUrole{p}{:} \DUrole{n}{Optional\DUrole{p}{{[}}Sequence\DUrole{p}{{]}}} \DUrole{o}{=} \DUrole{default_value}{None}}, \emph{\DUrole{n}{expect\_equilibrium}\DUrole{p}{:} \DUrole{n}{bool} \DUrole{o}{=} \DUrole{default_value}{True}}}{{ $\rightarrow$ muse.mca.FindEquilibriumResults}}
Runs the equilibrium loop.

If convergence is reached, then the function returns the new market. If the maximum
number of iterations are reached, then a warning issued in the log and the function
returns with the current status.
\begin{quote}\begin{description}
\item[{Parameters}] \leavevmode\begin{itemize}
\item {} 
\sphinxstyleliteralstrong{\sphinxupquote{market}} \textendash{} Commodities market, with the prices, supply, consumption and demand.

\item {} 
\sphinxstyleliteralstrong{\sphinxupquote{sectors}} \textendash{} A list of the sectors participating in the simulation.

\item {} 
\sphinxstyleliteralstrong{\sphinxupquote{maxiter}} \textendash{} Maximum number of iterations.

\item {} 
\sphinxstyleliteralstrong{\sphinxupquote{tol}} \textendash{} Tolerance for reaching equilibrium.

\item {} 
\sphinxstyleliteralstrong{\sphinxupquote{equilibrium\_variable}} \textendash{} Variable to use to calculate the equilibrium condition.

\item {} 
\sphinxstyleliteralstrong{\sphinxupquote{tol\_unmet\_demand}} \textendash{} Tolerance for the unmet demand.

\item {} 
\sphinxstyleliteralstrong{\sphinxupquote{excluded\_commodities}} \textendash{} Commodities to be excluded in check\_demand\_fulfillment

\item {} 
\sphinxstyleliteralstrong{\sphinxupquote{expect\_equilibrium}} \textendash{} if equilibrium should be reached. Useful to testing.

\end{itemize}

\item[{Returns}] \leavevmode
A tuple with the updated market (prices, supply, consumption and demand),
sectors, and convergence status.

\end{description}\end{quote}

\end{fulllineitems}

\index{single\_year\_iteration() (in module muse.mca)@\spxentry{single\_year\_iteration()}\spxextra{in module muse.mca}}

\begin{fulllineitems}
\phantomsection\label{\detokenize{api:muse.mca.single_year_iteration}}\pysiglinewithargsret{\sphinxcode{\sphinxupquote{muse.mca.}}\sphinxbfcode{\sphinxupquote{single\_year\_iteration}}}{\emph{\DUrole{n}{market}\DUrole{p}{:} \DUrole{n}{xarray.core.dataset.Dataset}}, \emph{\DUrole{n}{sectors}\DUrole{p}{:} \DUrole{n}{List\DUrole{p}{{[}}muse.sectors.abstract.AbstractSector\DUrole{p}{{]}}}}}{{ $\rightarrow$ muse.mca.SingleYearIterationResult}}
Runs one iteration of the sectors (runs each sector once).
\begin{quote}\begin{description}
\item[{Parameters}] \leavevmode\begin{itemize}
\item {} 
\sphinxstyleliteralstrong{\sphinxupquote{market}} \textendash{} An initial market with prices, supply, consumption.

\item {} 
\sphinxstyleliteralstrong{\sphinxupquote{sectors}} \textendash{} A list of the sectors participating in the simulation.

\end{itemize}

\item[{Returns}] \leavevmode
A tuple with the new market and sectors.

\end{description}\end{quote}

\end{fulllineitems}



\subsection{Carbon Budget}
\label{\detokenize{api:module-muse.carbon_budget}}\label{\detokenize{api:carbon-budget}}\index{module@\spxentry{module}!muse.carbon\_budget@\spxentry{muse.carbon\_budget}}\index{muse.carbon\_budget@\spxentry{muse.carbon\_budget}!module@\spxentry{module}}\index{CARBON\_BUDGET\_FITTERS (in module muse.carbon\_budget)@\spxentry{CARBON\_BUDGET\_FITTERS}\spxextra{in module muse.carbon\_budget}}

\begin{fulllineitems}
\phantomsection\label{\detokenize{api:muse.carbon_budget.CARBON_BUDGET_FITTERS}}\pysigline{\sphinxcode{\sphinxupquote{muse.carbon\_budget.}}\sphinxbfcode{\sphinxupquote{CARBON\_BUDGET\_FITTERS}}\sphinxbfcode{\sphinxupquote{: MutableMapping\DUrole{p}{{[}}str\DUrole{p}{, }Callable\DUrole{p}{{[}}\DUrole{p}{{[}}numpy.ndarray\DUrole{p}{, }numpy.ndarray\DUrole{p}{, }int\DUrole{p}{{]}}\DUrole{p}{, }float\DUrole{p}{{]}}\DUrole{p}{{]}}}}\sphinxbfcode{\sphinxupquote{ = \{\textquotesingle{}Exponential\textquotesingle{}: \textless{}function exponential\textgreater{}, \textquotesingle{}Linear\textquotesingle{}: \textless{}function linear\textgreater{}, \textquotesingle{}exponential\textquotesingle{}: \textless{}function exponential\textgreater{}, \textquotesingle{}linear\textquotesingle{}: \textless{}function linear\textgreater{}\}}}}
Dictionary of carbon budget fitters.

\end{fulllineitems}

\index{CARBON\_BUDGET\_FITTERS\_SIGNATURE (in module muse.carbon\_budget)@\spxentry{CARBON\_BUDGET\_FITTERS\_SIGNATURE}\spxextra{in module muse.carbon\_budget}}

\begin{fulllineitems}
\phantomsection\label{\detokenize{api:muse.carbon_budget.CARBON_BUDGET_FITTERS_SIGNATURE}}\pysigline{\sphinxcode{\sphinxupquote{muse.carbon\_budget.}}\sphinxbfcode{\sphinxupquote{CARBON\_BUDGET\_FITTERS\_SIGNATURE}}}
carbon budget fitters signature.

alias of Callable{[}{[}numpy.ndarray, numpy.ndarray, int{]}, float{]}

\end{fulllineitems}

\index{CARBON\_BUDGET\_METHODS (in module muse.carbon\_budget)@\spxentry{CARBON\_BUDGET\_METHODS}\spxextra{in module muse.carbon\_budget}}

\begin{fulllineitems}
\phantomsection\label{\detokenize{api:muse.carbon_budget.CARBON_BUDGET_METHODS}}\pysigline{\sphinxcode{\sphinxupquote{muse.carbon\_budget.}}\sphinxbfcode{\sphinxupquote{CARBON\_BUDGET\_METHODS}}}
Dictionary of carbon budget methods checks.

\end{fulllineitems}

\index{CARBON\_BUDGET\_METHODS\_SIGNATURE (in module muse.carbon\_budget)@\spxentry{CARBON\_BUDGET\_METHODS\_SIGNATURE}\spxextra{in module muse.carbon\_budget}}

\begin{fulllineitems}
\phantomsection\label{\detokenize{api:muse.carbon_budget.CARBON_BUDGET_METHODS_SIGNATURE}}\pysigline{\sphinxcode{\sphinxupquote{muse.carbon\_budget.}}\sphinxbfcode{\sphinxupquote{CARBON\_BUDGET\_METHODS\_SIGNATURE}}}
carbon budget fitters signature.

alias of Callable{[}{[}xarray.core.dataset.Dataset, list, Callable, xarray.core.dataarray.DataArray, xarray.core.dataarray.DataArray{]}, float{]}

\end{fulllineitems}

\index{create\_sample() (in module muse.carbon\_budget)@\spxentry{create\_sample()}\spxextra{in module muse.carbon\_budget}}

\begin{fulllineitems}
\phantomsection\label{\detokenize{api:muse.carbon_budget.create_sample}}\pysiglinewithargsret{\sphinxcode{\sphinxupquote{muse.carbon\_budget.}}\sphinxbfcode{\sphinxupquote{create\_sample}}}{\emph{\DUrole{n}{carbon\_price}}, \emph{\DUrole{n}{current\_emissions}}, \emph{\DUrole{n}{budget}}, \emph{\DUrole{n}{size}\DUrole{o}{=}\DUrole{default_value}{4}}}{}
Calculates a sample of carbon prices to estimate the adjusted carbon
price.

For each of these prices, the equilibrium loop will be run, obtaining a new value
for the emissions. Out of those price\sphinxhyphen{}emissions pairs, the final carbon price will
be estimated.
\begin{quote}\begin{description}
\item[{Parameters}] \leavevmode\begin{itemize}
\item {} 
\sphinxstyleliteralstrong{\sphinxupquote{carbon\_price}} \textendash{} Current carbon price

\item {} 
\sphinxstyleliteralstrong{\sphinxupquote{current\_emissions}} \textendash{} Current emissions

\item {} 
\sphinxstyleliteralstrong{\sphinxupquote{budget}} \textendash{} Carbon budget

\item {} 
\sphinxstyleliteralstrong{\sphinxupquote{size}} \textendash{} Number of points in the sample

\end{itemize}

\item[{Returns}] \leavevmode
An array with the sample prices.

\end{description}\end{quote}

\end{fulllineitems}

\index{exp\_guess\_and\_weights() (in module muse.carbon\_budget)@\spxentry{exp\_guess\_and\_weights()}\spxextra{in module muse.carbon\_budget}}

\begin{fulllineitems}
\phantomsection\label{\detokenize{api:muse.carbon_budget.exp_guess_and_weights}}\pysiglinewithargsret{\sphinxcode{\sphinxupquote{muse.carbon\_budget.}}\sphinxbfcode{\sphinxupquote{exp\_guess\_and\_weights}}}{\emph{\DUrole{n}{prices}\DUrole{p}{:} \DUrole{n}{numpy.ndarray}}, \emph{\DUrole{n}{emissions}\DUrole{p}{:} \DUrole{n}{numpy.ndarray}}, \emph{\DUrole{n}{budget}\DUrole{p}{:} \DUrole{n}{int}}}{{ $\rightarrow$ tuple}}
Estimates initial values for the exponential fitting algorithm and the
weights.

The points closest to the budget are used to estimate the initial guess. They also
have the highest weight.
\begin{quote}\begin{description}
\item[{Parameters}] \leavevmode\begin{itemize}
\item {} 
\sphinxstyleliteralstrong{\sphinxupquote{prices}} \textendash{} An array with the sample carbon prices

\item {} 
\sphinxstyleliteralstrong{\sphinxupquote{emissions}} \textendash{} An array with the corresponding emissions

\item {} 
\sphinxstyleliteralstrong{\sphinxupquote{budget}} \textendash{} The carbon budget for the time period

\end{itemize}

\item[{Returns}] \leavevmode
The initial guess and weights

\end{description}\end{quote}

\end{fulllineitems}

\index{exponential() (in module muse.carbon\_budget)@\spxentry{exponential()}\spxextra{in module muse.carbon\_budget}}

\begin{fulllineitems}
\phantomsection\label{\detokenize{api:muse.carbon_budget.exponential}}\pysiglinewithargsret{\sphinxcode{\sphinxupquote{muse.carbon\_budget.}}\sphinxbfcode{\sphinxupquote{exponential}}}{\emph{\DUrole{n}{prices}\DUrole{p}{:} \DUrole{n}{numpy.ndarray}}, \emph{\DUrole{n}{emissions}\DUrole{p}{:} \DUrole{n}{numpy.ndarray}}, \emph{\DUrole{n}{budget}\DUrole{p}{:} \DUrole{n}{int}}}{{ $\rightarrow$ float}}
Fits the prices\sphinxhyphen{}emissions pairs to an exponential function.

Once that is done, an optimal carbon price is estimated
\begin{quote}\begin{description}
\item[{Parameters}] \leavevmode\begin{itemize}
\item {} 
\sphinxstyleliteralstrong{\sphinxupquote{prices}} \textendash{} An array with the sample carbon prices

\item {} 
\sphinxstyleliteralstrong{\sphinxupquote{emissions}} \textendash{} An array with the corresponding emissions

\item {} 
\sphinxstyleliteralstrong{\sphinxupquote{budget}} \textendash{} The carbon budget for the time period

\end{itemize}

\item[{Returns}] \leavevmode
The optimal carbon price.

\end{description}\end{quote}

\end{fulllineitems}

\index{linear() (in module muse.carbon\_budget)@\spxentry{linear()}\spxextra{in module muse.carbon\_budget}}

\begin{fulllineitems}
\phantomsection\label{\detokenize{api:muse.carbon_budget.linear}}\pysiglinewithargsret{\sphinxcode{\sphinxupquote{muse.carbon\_budget.}}\sphinxbfcode{\sphinxupquote{linear}}}{\emph{\DUrole{n}{prices}\DUrole{p}{:} \DUrole{n}{numpy.ndarray}}, \emph{\DUrole{n}{emissions}\DUrole{p}{:} \DUrole{n}{numpy.ndarray}}, \emph{\DUrole{n}{budget}\DUrole{p}{:} \DUrole{n}{int}}}{{ $\rightarrow$ float}}
Fits the prices\sphinxhyphen{}emissions pairs to a linear function.

Once that is done, an optimal carbon price is estimated
\begin{quote}\begin{description}
\item[{Parameters}] \leavevmode\begin{itemize}
\item {} 
\sphinxstyleliteralstrong{\sphinxupquote{prices}} \textendash{} An array with the sample carbon prices

\item {} 
\sphinxstyleliteralstrong{\sphinxupquote{emissions}} \textendash{} An array with the corresponding emissions

\item {} 
\sphinxstyleliteralstrong{\sphinxupquote{budget}} \textendash{} The carbon budget for the time period

\end{itemize}

\item[{Returns}] \leavevmode
The optimal carbon price.

\end{description}\end{quote}

\end{fulllineitems}

\index{linear\_guess\_and\_weights() (in module muse.carbon\_budget)@\spxentry{linear\_guess\_and\_weights()}\spxextra{in module muse.carbon\_budget}}

\begin{fulllineitems}
\phantomsection\label{\detokenize{api:muse.carbon_budget.linear_guess_and_weights}}\pysiglinewithargsret{\sphinxcode{\sphinxupquote{muse.carbon\_budget.}}\sphinxbfcode{\sphinxupquote{linear\_guess\_and\_weights}}}{\emph{\DUrole{n}{prices}\DUrole{p}{:} \DUrole{n}{numpy.ndarray}}, \emph{\DUrole{n}{emissions}\DUrole{p}{:} \DUrole{n}{numpy.ndarray}}, \emph{\DUrole{n}{budget}\DUrole{p}{:} \DUrole{n}{int}}}{{ $\rightarrow$ tuple}}
Estimates initial values for the linear fitting algorithm and the
weights.

The points closest to the budget are used to estimate the initial guess. They also
have the highest weight.
\begin{quote}\begin{description}
\item[{Returns}] \leavevmode
The initial guess and weights

\end{description}\end{quote}

\end{fulllineitems}

\index{refine\_new\_price() (in module muse.carbon\_budget)@\spxentry{refine\_new\_price()}\spxextra{in module muse.carbon\_budget}}

\begin{fulllineitems}
\phantomsection\label{\detokenize{api:muse.carbon_budget.refine_new_price}}\pysiglinewithargsret{\sphinxcode{\sphinxupquote{muse.carbon\_budget.}}\sphinxbfcode{\sphinxupquote{refine\_new\_price}}}{\emph{\DUrole{n}{market}\DUrole{p}{:} \DUrole{n}{xarray.core.dataset.Dataset}}, \emph{\DUrole{n}{historic\_price}\DUrole{p}{:} \DUrole{n}{xarray.core.dataarray.DataArray}}, \emph{\DUrole{n}{carbon\_budget}\DUrole{p}{:} \DUrole{n}{xarray.core.dataarray.DataArray}}, \emph{\DUrole{n}{sample}\DUrole{p}{:} \DUrole{n}{numpy.ndarray}}, \emph{\DUrole{n}{price}\DUrole{p}{:} \DUrole{n}{float}}, \emph{\DUrole{n}{commodities}\DUrole{p}{:} \DUrole{n}{list}}, \emph{\DUrole{n}{price\_too\_high\_threshold}\DUrole{p}{:} \DUrole{n}{float}}}{{ $\rightarrow$ float}}
Refine the value of the carbon price to ensure it is not too high or low.
:param market: Market, with the prices, supply, consumption and demand.
:param historic\_price: DataArray with the historic carbon prices.
:param carbon\_budget: DataArray with the carbon budget.
:param sample: Sample carbon price points.
:param price: Current carbon price, to be refined.
:param commodities: List of carbon\sphinxhyphen{}related commodities.
:param price\_too\_high\_threshold: Threshold to decide what is a price too high.
\begin{quote}\begin{description}
\item[{Returns}] \leavevmode
A refined carbon price.

\end{description}\end{quote}

\end{fulllineitems}

\index{register\_carbon\_budget\_fitter() (in module muse.carbon\_budget)@\spxentry{register\_carbon\_budget\_fitter()}\spxextra{in module muse.carbon\_budget}}

\begin{fulllineitems}
\phantomsection\label{\detokenize{api:muse.carbon_budget.register_carbon_budget_fitter}}\pysiglinewithargsret{\sphinxcode{\sphinxupquote{muse.carbon\_budget.}}\sphinxbfcode{\sphinxupquote{register\_carbon\_budget\_fitter}}}{\emph{\DUrole{n}{function}\DUrole{p}{:} \DUrole{n}{Callable\DUrole{p}{{[}}\DUrole{p}{{[}}numpy.ndarray\DUrole{p}{, }numpy.ndarray\DUrole{p}{, }int\DUrole{p}{{]}}\DUrole{p}{, }float\DUrole{p}{{]}}} \DUrole{o}{=} \DUrole{default_value}{None}}}{}
Decorator to register a carbon budget function.

\end{fulllineitems}

\index{register\_carbon\_budget\_method() (in module muse.carbon\_budget)@\spxentry{register\_carbon\_budget\_method()}\spxextra{in module muse.carbon\_budget}}

\begin{fulllineitems}
\phantomsection\label{\detokenize{api:muse.carbon_budget.register_carbon_budget_method}}\pysiglinewithargsret{\sphinxcode{\sphinxupquote{muse.carbon\_budget.}}\sphinxbfcode{\sphinxupquote{register\_carbon\_budget\_method}}}{\emph{\DUrole{n}{function}\DUrole{p}{:} \DUrole{n}{Callable\DUrole{p}{{[}}\DUrole{p}{{[}}xarray.core.dataset.Dataset\DUrole{p}{, }list\DUrole{p}{, }Callable\DUrole{p}{, }xarray.core.dataarray.DataArray\DUrole{p}{, }xarray.core.dataarray.DataArray\DUrole{p}{{]}}\DUrole{p}{, }float\DUrole{p}{{]}}} \DUrole{o}{=} \DUrole{default_value}{None}}}{}
Decorator to register a carbon budget function.

\end{fulllineitems}

\index{update\_carbon\_budget() (in module muse.carbon\_budget)@\spxentry{update\_carbon\_budget()}\spxextra{in module muse.carbon\_budget}}

\begin{fulllineitems}
\phantomsection\label{\detokenize{api:muse.carbon_budget.update_carbon_budget}}\pysiglinewithargsret{\sphinxcode{\sphinxupquote{muse.carbon\_budget.}}\sphinxbfcode{\sphinxupquote{update\_carbon\_budget}}}{\emph{\DUrole{n}{carbon\_budget}\DUrole{p}{:} \DUrole{n}{Sequence\DUrole{p}{{[}}float\DUrole{p}{{]}}}}, \emph{\DUrole{n}{emissions}\DUrole{p}{:} \DUrole{n}{float}}, \emph{\DUrole{n}{year\_idx}\DUrole{p}{:} \DUrole{n}{int}}, \emph{\DUrole{n}{over}\DUrole{p}{:} \DUrole{n}{bool} \DUrole{o}{=} \DUrole{default_value}{True}}, \emph{\DUrole{n}{under}\DUrole{p}{:} \DUrole{n}{bool} \DUrole{o}{=} \DUrole{default_value}{True}}}{{ $\rightarrow$ float}}
Adjust the carbon budget in the far future if emissions too high or low.
\begin{quote}\begin{description}
\item[{Returns}] \leavevmode
An adjusted threshold for the far future year

\end{description}\end{quote}

\end{fulllineitems}



\section{Sectors and associated functionality}
\label{\detokenize{api:module-muse.sectors}}\label{\detokenize{api:sectors-and-associated-functionality}}\index{module@\spxentry{module}!muse.sectors@\spxentry{muse.sectors}}\index{muse.sectors@\spxentry{muse.sectors}!module@\spxentry{module}}
Define a sector, e.g. aggregation of agents.

There are three main kinds of sectors classes, encompassing three use cases:
\begin{itemize}
\item {} 
\sphinxcode{\sphinxupquote{Sector}}: The main workhorse sector of the model. It
contains only on kind of data, namely the agents responsible for holding assets and
investing in new assets.

\item {} 
\sphinxcode{\sphinxupquote{PresetSector}}: A sector that is meant to generate
demand for the sectors above using a fixed formula or schedule.

\item {} 
\sphinxcode{\sphinxupquote{LegacySector}}: A wrapper around the original MUSE
sectors.

\end{itemize}

All the sectors derive from \sphinxcode{\sphinxupquote{AbstractSector}}. The \sphinxcode{\sphinxupquote{AbstractSector}} defines
two \sphinxhref{:https://docs.python.org/3/library/abc.html}{abstract} functions which should be declared by derived sectors. \sphinxhref{:https://www.python-course.eu/python3\_abstract\_classes.php}{Abstract}
here means a common programming practice where some concept in the code (e.g. a sector)
is given an explicit interface, with the goal of making it easier for other programmers
to use and implement the concept.
\begin{itemize}
\item {} 
\sphinxcode{\sphinxupquote{AbstractSector.factory()}}: Creates a sector from input data

\item {} 
\sphinxcode{\sphinxupquote{AbstractSector.next()}}: A function which takes a market (demand, supply,
prices) and returns a market.  What happens within could be anything, though it will
likely constists of dispatch and investment.

\end{itemize}

New sectors can be registered with the MUSE input files using
\sphinxcode{\sphinxupquote{muse.sectors.register.register\_sector()}}.
\index{register\_sector() (in module muse.sectors.register)@\spxentry{register\_sector()}\spxextra{in module muse.sectors.register}}

\begin{fulllineitems}
\phantomsection\label{\detokenize{api:muse.sectors.register.register_sector}}\pysiglinewithargsret{\sphinxcode{\sphinxupquote{@}}\sphinxcode{\sphinxupquote{muse.sectors.register.}}\sphinxbfcode{\sphinxupquote{register\_sector}}}{\emph{\DUrole{n}{sector\_class}\DUrole{p}{:} \DUrole{n}{Optional\DUrole{p}{{[}}Type\DUrole{p}{{[}}muse.sectors.abstract.AbstractSector\DUrole{p}{{]}}\DUrole{p}{{]}}} \DUrole{o}{=} \DUrole{default_value}{None}}, \emph{\DUrole{n}{name}\DUrole{p}{:} \DUrole{n}{Optional\DUrole{p}{{[}}Union\DUrole{p}{{[}}str\DUrole{p}{, }Sequence\DUrole{p}{{[}}str\DUrole{p}{{]}}\DUrole{p}{{]}}\DUrole{p}{{]}}} \DUrole{o}{=} \DUrole{default_value}{None}}}{{ $\rightarrow$ Type\DUrole{p}{{[}}muse.sectors.abstract.AbstractSector\DUrole{p}{{]}}}}
Registers a sector so it is available MUSE\sphinxhyphen{}wide.
\subsubsection*{Example}

\begin{sphinxVerbatim}[commandchars=\\\{\}]
\PYG{g+gp}{\PYGZgt{}\PYGZgt{}\PYGZgt{} }\PYG{k+kn}{from} \PYG{n+nn}{muse}\PYG{n+nn}{.}\PYG{n+nn}{sectors} \PYG{k+kn}{import} \PYG{n}{AbstractSector}\PYG{p}{,} \PYG{n}{register\PYGZus{}sector}
\PYG{g+gp}{\PYGZgt{}\PYGZgt{}\PYGZgt{} }\PYG{n+nd}{@register\PYGZus{}sector}\PYG{p}{(}\PYG{n}{name}\PYG{o}{=}\PYG{l+s+s2}{\PYGZdq{}}\PYG{l+s+s2}{MyResidence}\PYG{l+s+s2}{\PYGZdq{}}\PYG{p}{)}
\PYG{g+gp}{... }\PYG{k}{class} \PYG{n+nc}{ResidentialSector}\PYG{p}{(}\PYG{n}{AbstractSector}\PYG{p}{)}\PYG{p}{:}
\PYG{g+gp}{... }    \PYG{k}{pass}
\end{sphinxVerbatim}

\end{fulllineitems}



\subsection{AbstractSector}
\label{\detokenize{api:abstractsector}}\index{AbstractSector (class in muse.sectors)@\spxentry{AbstractSector}\spxextra{class in muse.sectors}}

\begin{fulllineitems}
\phantomsection\label{\detokenize{api:muse.sectors.AbstractSector}}\pysigline{\sphinxbfcode{\sphinxupquote{class }}\sphinxcode{\sphinxupquote{muse.sectors.}}\sphinxbfcode{\sphinxupquote{AbstractSector}}}
Abstract base class for sectors.

Sectors are part of type hierarchy with \sphinxcode{\sphinxupquote{AbstractSector}} at the apex: all
sectors should derive from \sphinxcode{\sphinxupquote{AbstractSector}} directly or indirectly.

MUSE only requires two things of a sector. Sector should be instanstiable via a
\sphinxcode{\sphinxupquote{factory()}} function. And they should be callable via
\sphinxcode{\sphinxupquote{next()}}.

\sphinxcode{\sphinxupquote{AbstractSector}} declares an interface with these two functions. Sectors
which derive from it will be warned if either method is not implemented.
\index{factory() (muse.sectors.AbstractSector class method)@\spxentry{factory()}\spxextra{muse.sectors.AbstractSector class method}}

\begin{fulllineitems}
\phantomsection\label{\detokenize{api:muse.sectors.AbstractSector.factory}}\pysiglinewithargsret{\sphinxbfcode{\sphinxupquote{abstract classmethod }}\sphinxbfcode{\sphinxupquote{factory}}}{\emph{\DUrole{n}{name}\DUrole{p}{:} \DUrole{n}{str}}, \emph{\DUrole{n}{settings}\DUrole{p}{:} \DUrole{n}{Any}}}{{ $\rightarrow$ muse.sectors.abstract.AbstractSector}}
Creates class from settings named\sphinxhyphen{}tuple.

\end{fulllineitems}

\index{next() (muse.sectors.AbstractSector method)@\spxentry{next()}\spxextra{muse.sectors.AbstractSector method}}

\begin{fulllineitems}
\phantomsection\label{\detokenize{api:muse.sectors.AbstractSector.next}}\pysiglinewithargsret{\sphinxbfcode{\sphinxupquote{abstract }}\sphinxbfcode{\sphinxupquote{next}}}{\emph{\DUrole{n}{mca\_market}\DUrole{p}{:} \DUrole{n}{xarray.core.dataset.Dataset}}}{{ $\rightarrow$ xarray.core.dataset.Dataset}}
Advance sector by one time period.

\end{fulllineitems}


\end{fulllineitems}



\subsection{Sector}
\label{\detokenize{api:sector}}\index{Sector (class in muse.sectors.sector)@\spxentry{Sector}\spxextra{class in muse.sectors.sector}}

\begin{fulllineitems}
\phantomsection\label{\detokenize{api:muse.sectors.sector.Sector}}\pysiglinewithargsret{\sphinxbfcode{\sphinxupquote{class }}\sphinxcode{\sphinxupquote{muse.sectors.sector.}}\sphinxbfcode{\sphinxupquote{Sector}}}{\emph{\DUrole{n}{name}\DUrole{p}{:} \DUrole{n}{str}}, \emph{\DUrole{n}{technologies}\DUrole{p}{:} \DUrole{n}{xarray.core.dataset.Dataset}}, \emph{\DUrole{n}{subsectors}\DUrole{p}{:} \DUrole{n}{Sequence\DUrole{p}{{[}}muse.sectors.subsector.Subsector\DUrole{p}{{]}}} \DUrole{o}{=} \DUrole{default_value}{{[}{]}}}, \emph{\DUrole{n}{timeslices}\DUrole{p}{:} \DUrole{n}{Optional\DUrole{p}{{[}}pandas.core.indexes.multi.MultiIndex\DUrole{p}{{]}}} \DUrole{o}{=} \DUrole{default_value}{None}}, \emph{\DUrole{n}{interactions}\DUrole{p}{:} \DUrole{n}{Optional{[}Callable{[}{[}Sequence{[}muse.agents.agent.AbstractAgent{]}{]}, None{]}{]}} \DUrole{o}{=} \DUrole{default_value}{None}}, \emph{\DUrole{n}{interpolation}\DUrole{p}{:} \DUrole{n}{str} \DUrole{o}{=} \DUrole{default_value}{\textquotesingle{}linear\textquotesingle{}}}, \emph{\DUrole{n}{outputs}\DUrole{p}{:} \DUrole{n}{Optional\DUrole{p}{{[}}Callable\DUrole{p}{{]}}} \DUrole{o}{=} \DUrole{default_value}{None}}, \emph{\DUrole{n}{supply\_prod}\DUrole{p}{:} \DUrole{n}{Optional\DUrole{p}{{[}}Callable\DUrole{p}{{[}}\DUrole{p}{{[}}xarray.core.dataarray.DataArray\DUrole{p}{, }xarray.core.dataarray.DataArray\DUrole{p}{, }xarray.core.dataset.Dataset\DUrole{p}{{]}}\DUrole{p}{, }xarray.core.dataarray.DataArray\DUrole{p}{{]}}\DUrole{p}{{]}}} \DUrole{o}{=} \DUrole{default_value}{None}}}{}
Base class for all sectors.
\index{agents() (muse.sectors.sector.Sector property)@\spxentry{agents()}\spxextra{muse.sectors.sector.Sector property}}

\begin{fulllineitems}
\phantomsection\label{\detokenize{api:muse.sectors.sector.Sector.agents}}\pysigline{\sphinxbfcode{\sphinxupquote{property }}\sphinxbfcode{\sphinxupquote{agents}}}
Iterator over all agents in the sector.

\end{fulllineitems}

\index{capacity() (muse.sectors.sector.Sector property)@\spxentry{capacity()}\spxextra{muse.sectors.sector.Sector property}}

\begin{fulllineitems}
\phantomsection\label{\detokenize{api:muse.sectors.sector.Sector.capacity}}\pysigline{\sphinxbfcode{\sphinxupquote{property }}\sphinxbfcode{\sphinxupquote{capacity}}}
Aggregates capacity across agents.

The capacities are aggregated leaving only two
dimensions: asset (technology, installation date,
region), year.

\end{fulllineitems}

\index{convert\_market\_timeslice() (muse.sectors.sector.Sector static method)@\spxentry{convert\_market\_timeslice()}\spxextra{muse.sectors.sector.Sector static method}}

\begin{fulllineitems}
\phantomsection\label{\detokenize{api:muse.sectors.sector.Sector.convert_market_timeslice}}\pysiglinewithargsret{\sphinxbfcode{\sphinxupquote{static }}\sphinxbfcode{\sphinxupquote{convert\_market\_timeslice}}}{\emph{\DUrole{n}{market}\DUrole{p}{:} \DUrole{n}{xarray.core.dataset.Dataset}}, \emph{\DUrole{n}{timeslice}\DUrole{p}{:} \DUrole{n}{pandas.core.indexes.multi.MultiIndex}}, \emph{\DUrole{n}{intensive}\DUrole{p}{:} \DUrole{n}{Union\DUrole{p}{{[}}str\DUrole{p}{, }Tuple\DUrole{p}{{[}}str\DUrole{p}{{]}}\DUrole{p}{{]}}} \DUrole{o}{=} \DUrole{default_value}{\textquotesingle{}prices\textquotesingle{}}}}{{ $\rightarrow$ xarray.core.dataset.Dataset}}
Converts market from one to another timeslice.

\end{fulllineitems}

\index{factory() (muse.sectors.sector.Sector class method)@\spxentry{factory()}\spxextra{muse.sectors.sector.Sector class method}}

\begin{fulllineitems}
\phantomsection\label{\detokenize{api:muse.sectors.sector.Sector.factory}}\pysiglinewithargsret{\sphinxbfcode{\sphinxupquote{classmethod }}\sphinxbfcode{\sphinxupquote{factory}}}{\emph{\DUrole{n}{name}\DUrole{p}{:} \DUrole{n}{str}}, \emph{\DUrole{n}{settings}\DUrole{p}{:} \DUrole{n}{Any}}}{{ $\rightarrow$ muse.sectors.sector.Sector}}
Creates class from settings named\sphinxhyphen{}tuple.

\end{fulllineitems}

\index{forecast() (muse.sectors.sector.Sector property)@\spxentry{forecast()}\spxextra{muse.sectors.sector.Sector property}}

\begin{fulllineitems}
\phantomsection\label{\detokenize{api:muse.sectors.sector.Sector.forecast}}\pysigline{\sphinxbfcode{\sphinxupquote{property }}\sphinxbfcode{\sphinxupquote{forecast}}}
Maximum forecast horizon across agents.

If no agents with a “forecast” attribute are found, defaults to 5. It cannot be
lower than 1 year.

\end{fulllineitems}

\index{interactions (muse.sectors.sector.Sector attribute)@\spxentry{interactions}\spxextra{muse.sectors.sector.Sector attribute}}

\begin{fulllineitems}
\phantomsection\label{\detokenize{api:muse.sectors.sector.Sector.interactions}}\pysigline{\sphinxbfcode{\sphinxupquote{interactions}}}
Interactions between agents.

Called right before computing new investments, this function should manage any
interactions between agents, e.g. passing assets from \sphinxstyleemphasis{new} agents  to \sphinxstyleemphasis{retro}
agents, and maket make\sphinxhyphen{}up from \sphinxstyleemphasis{retro} to \sphinxstyleemphasis{new}.

Defaults to doing nothing.

The function takes the sequence of agents as input, and returns nothing. It is
expected to modify the agents in\sphinxhyphen{}place.


\sphinxstrong{See also:}


\sphinxcode{\sphinxupquote{muse.interactions}}



\end{fulllineitems}

\index{interpolation (muse.sectors.sector.Sector attribute)@\spxentry{interpolation}\spxextra{muse.sectors.sector.Sector attribute}}

\begin{fulllineitems}
\phantomsection\label{\detokenize{api:muse.sectors.sector.Sector.interpolation}}\pysigline{\sphinxbfcode{\sphinxupquote{interpolation}}\sphinxbfcode{\sphinxupquote{: Mapping\DUrole{p}{{[}}Text\DUrole{p}{, }Any\DUrole{p}{{]}}}}}
Interpolation method and arguments when computing years.

\end{fulllineitems}

\index{market\_variables() (muse.sectors.sector.Sector method)@\spxentry{market\_variables()}\spxextra{muse.sectors.sector.Sector method}}

\begin{fulllineitems}
\phantomsection\label{\detokenize{api:muse.sectors.sector.Sector.market_variables}}\pysiglinewithargsret{\sphinxbfcode{\sphinxupquote{market\_variables}}}{\emph{\DUrole{n}{market}\DUrole{p}{:} \DUrole{n}{xarray.core.dataset.Dataset}}, \emph{\DUrole{n}{technologies}\DUrole{p}{:} \DUrole{n}{xarray.core.dataset.Dataset}}}{{ $\rightarrow$ xarray.core.dataset.Dataset}}
Computes resulting market: production, consumption, and costs.

\end{fulllineitems}

\index{name (muse.sectors.sector.Sector attribute)@\spxentry{name}\spxextra{muse.sectors.sector.Sector attribute}}

\begin{fulllineitems}
\phantomsection\label{\detokenize{api:muse.sectors.sector.Sector.name}}\pysigline{\sphinxbfcode{\sphinxupquote{name}}\sphinxbfcode{\sphinxupquote{: Text}}}
Name of the sector.

\end{fulllineitems}

\index{next() (muse.sectors.sector.Sector method)@\spxentry{next()}\spxextra{muse.sectors.sector.Sector method}}

\begin{fulllineitems}
\phantomsection\label{\detokenize{api:muse.sectors.sector.Sector.next}}\pysiglinewithargsret{\sphinxbfcode{\sphinxupquote{next}}}{\emph{\DUrole{n}{mca\_market}\DUrole{p}{:} \DUrole{n}{xarray.core.dataset.Dataset}}, \emph{\DUrole{n}{time\_period}\DUrole{p}{:} \DUrole{n}{Optional\DUrole{p}{{[}}int\DUrole{p}{{]}}} \DUrole{o}{=} \DUrole{default_value}{None}}, \emph{\DUrole{n}{current\_year}\DUrole{p}{:} \DUrole{n}{Optional\DUrole{p}{{[}}int\DUrole{p}{{]}}} \DUrole{o}{=} \DUrole{default_value}{None}}}{{ $\rightarrow$ xarray.core.dataset.Dataset}}
Advance sector by one time period.
\begin{quote}\begin{description}
\item[{Parameters}] \leavevmode\begin{itemize}
\item {} 
\sphinxstyleliteralstrong{\sphinxupquote{mca\_market}} \textendash{} Market with \sphinxcode{\sphinxupquote{demand}}, \sphinxcode{\sphinxupquote{supply}}, and \sphinxcode{\sphinxupquote{prices}}.

\item {} 
\sphinxstyleliteralstrong{\sphinxupquote{time\_period}} \textendash{} Length of the time period in the framework. Defaults to the range of
\sphinxcode{\sphinxupquote{mca\_market.year}}.

\end{itemize}

\item[{Returns}] \leavevmode
A market containing the \sphinxcode{\sphinxupquote{supply}} offered by the sector, it’s attendant
\sphinxcode{\sphinxupquote{consumption}} of fuels and materials and the associated \sphinxcode{\sphinxupquote{costs}}.

\end{description}\end{quote}

\end{fulllineitems}

\index{outputs (muse.sectors.sector.Sector attribute)@\spxentry{outputs}\spxextra{muse.sectors.sector.Sector attribute}}

\begin{fulllineitems}
\phantomsection\label{\detokenize{api:muse.sectors.sector.Sector.outputs}}\pysigline{\sphinxbfcode{\sphinxupquote{outputs}}\sphinxbfcode{\sphinxupquote{: Callable}}}
A function for outputing data for post\sphinxhyphen{}mortem analysis.

\end{fulllineitems}

\index{subsectors (muse.sectors.sector.Sector attribute)@\spxentry{subsectors}\spxextra{muse.sectors.sector.Sector attribute}}

\begin{fulllineitems}
\phantomsection\label{\detokenize{api:muse.sectors.sector.Sector.subsectors}}\pysigline{\sphinxbfcode{\sphinxupquote{subsectors}}\sphinxbfcode{\sphinxupquote{: Sequence\DUrole{p}{{[}}Subsector\DUrole{p}{{]}}}}}
Subsectors controlled by this object.

\end{fulllineitems}

\index{supply\_prod (muse.sectors.sector.Sector attribute)@\spxentry{supply\_prod}\spxextra{muse.sectors.sector.Sector attribute}}

\begin{fulllineitems}
\phantomsection\label{\detokenize{api:muse.sectors.sector.Sector.supply_prod}}\pysigline{\sphinxbfcode{\sphinxupquote{supply\_prod}}}
Computes production as used to return the supply to the MCA.

It can be anything registered with
\sphinxcode{\sphinxupquote{@register\_production}}.

\end{fulllineitems}

\index{technologies (muse.sectors.sector.Sector attribute)@\spxentry{technologies}\spxextra{muse.sectors.sector.Sector attribute}}

\begin{fulllineitems}
\phantomsection\label{\detokenize{api:muse.sectors.sector.Sector.technologies}}\pysigline{\sphinxbfcode{\sphinxupquote{technologies}}\sphinxbfcode{\sphinxupquote{: xr.Dataset}}}
Parameters describing the sector’s technologies.

\end{fulllineitems}

\index{timeslices (muse.sectors.sector.Sector attribute)@\spxentry{timeslices}\spxextra{muse.sectors.sector.Sector attribute}}

\begin{fulllineitems}
\phantomsection\label{\detokenize{api:muse.sectors.sector.Sector.timeslices}}\pysigline{\sphinxbfcode{\sphinxupquote{timeslices}}\sphinxbfcode{\sphinxupquote{: Optional\DUrole{p}{{[}}pd.MultiIndex\DUrole{p}{{]}}}}}
Timeslice at which this sector operates.

If None, it will operate using the timeslice of the input market.

\end{fulllineitems}


\end{fulllineitems}



\subsection{Subsector}
\label{\detokenize{api:subsector}}\index{Subsector (class in muse.sectors.subsector)@\spxentry{Subsector}\spxextra{class in muse.sectors.subsector}}

\begin{fulllineitems}
\phantomsection\label{\detokenize{api:muse.sectors.subsector.Subsector}}\pysiglinewithargsret{\sphinxbfcode{\sphinxupquote{class }}\sphinxcode{\sphinxupquote{muse.sectors.subsector.}}\sphinxbfcode{\sphinxupquote{Subsector}}}{\emph{\DUrole{n}{agents}\DUrole{p}{:} \DUrole{n}{Sequence\DUrole{p}{{[}}muse.agents.agent.Agent\DUrole{p}{{]}}}}, \emph{\DUrole{n}{commodities}\DUrole{p}{:} \DUrole{n}{Sequence\DUrole{p}{{[}}str\DUrole{p}{{]}}}}, \emph{\DUrole{n}{demand\_share}\DUrole{p}{:} \DUrole{n}{Optional\DUrole{p}{{[}}Callable\DUrole{p}{{]}}} \DUrole{o}{=} \DUrole{default_value}{None}}, \emph{\DUrole{n}{constraints}\DUrole{p}{:} \DUrole{n}{Optional\DUrole{p}{{[}}Callable\DUrole{p}{{]}}} \DUrole{o}{=} \DUrole{default_value}{None}}, \emph{\DUrole{n}{name}\DUrole{p}{:} \DUrole{n}{str} \DUrole{o}{=} \DUrole{default_value}{\textquotesingle{}subsector\textquotesingle{}}}, \emph{\DUrole{n}{forecast}\DUrole{p}{:} \DUrole{n}{int} \DUrole{o}{=} \DUrole{default_value}{5}}}{}
Agent group servicing a subset of the sectorial commodities.

\end{fulllineitems}



\subsection{PresetSector}
\label{\detokenize{api:presetsector}}\index{PresetSector (class in muse.sectors.preset\_sector)@\spxentry{PresetSector}\spxextra{class in muse.sectors.preset\_sector}}

\begin{fulllineitems}
\phantomsection\label{\detokenize{api:muse.sectors.preset_sector.PresetSector}}\pysiglinewithargsret{\sphinxbfcode{\sphinxupquote{class }}\sphinxcode{\sphinxupquote{muse.sectors.preset\_sector.}}\sphinxbfcode{\sphinxupquote{PresetSector}}}{\emph{\DUrole{n}{presets}\DUrole{p}{:} \DUrole{n}{xarray.core.dataset.Dataset}}, \emph{\DUrole{n}{interpolation\_mode}\DUrole{p}{:} \DUrole{n}{str} \DUrole{o}{=} \DUrole{default_value}{\textquotesingle{}linear\textquotesingle{}}}, \emph{\DUrole{n}{name}\DUrole{p}{:} \DUrole{n}{str} \DUrole{o}{=} \DUrole{default_value}{\textquotesingle{}preset\textquotesingle{}}}}{}
Sector with outcomes fixed from the start.
\index{factory() (muse.sectors.preset\_sector.PresetSector class method)@\spxentry{factory()}\spxextra{muse.sectors.preset\_sector.PresetSector class method}}

\begin{fulllineitems}
\phantomsection\label{\detokenize{api:muse.sectors.preset_sector.PresetSector.factory}}\pysiglinewithargsret{\sphinxbfcode{\sphinxupquote{classmethod }}\sphinxbfcode{\sphinxupquote{factory}}}{\emph{\DUrole{n}{name}\DUrole{p}{:} \DUrole{n}{str}}, \emph{\DUrole{n}{settings}\DUrole{p}{:} \DUrole{n}{Any}}}{{ $\rightarrow$ muse.sectors.preset\_sector.PresetSector}}
Constructs a PresetSectors from input data.

\end{fulllineitems}

\index{interpolation\_mode (muse.sectors.preset\_sector.PresetSector attribute)@\spxentry{interpolation\_mode}\spxextra{muse.sectors.preset\_sector.PresetSector attribute}}

\begin{fulllineitems}
\phantomsection\label{\detokenize{api:muse.sectors.preset_sector.PresetSector.interpolation_mode}}\pysigline{\sphinxbfcode{\sphinxupquote{interpolation\_mode}}\sphinxbfcode{\sphinxupquote{: Text}}}
Interpolation method

\end{fulllineitems}

\index{name (muse.sectors.preset\_sector.PresetSector attribute)@\spxentry{name}\spxextra{muse.sectors.preset\_sector.PresetSector attribute}}

\begin{fulllineitems}
\phantomsection\label{\detokenize{api:muse.sectors.preset_sector.PresetSector.name}}\pysigline{\sphinxbfcode{\sphinxupquote{name}}}
Name by which to identify a sector

\end{fulllineitems}

\index{next() (muse.sectors.preset\_sector.PresetSector method)@\spxentry{next()}\spxextra{muse.sectors.preset\_sector.PresetSector method}}

\begin{fulllineitems}
\phantomsection\label{\detokenize{api:muse.sectors.preset_sector.PresetSector.next}}\pysiglinewithargsret{\sphinxbfcode{\sphinxupquote{next}}}{\emph{\DUrole{n}{mca\_market}\DUrole{p}{:} \DUrole{n}{xarray.core.dataset.Dataset}}}{{ $\rightarrow$ xarray.core.dataset.Dataset}}
Advance sector by one time period.

\end{fulllineitems}

\index{presets (muse.sectors.preset\_sector.PresetSector attribute)@\spxentry{presets}\spxextra{muse.sectors.preset\_sector.PresetSector attribute}}

\begin{fulllineitems}
\phantomsection\label{\detokenize{api:muse.sectors.preset_sector.PresetSector.presets}}\pysigline{\sphinxbfcode{\sphinxupquote{presets}}\sphinxbfcode{\sphinxupquote{: Dataset}}}
Market across time and space.

\end{fulllineitems}


\end{fulllineitems}



\subsection{LegacySector}
\label{\detokenize{api:legacysector}}\index{LegacySector (class in muse.sectors.legacy\_sector)@\spxentry{LegacySector}\spxextra{class in muse.sectors.legacy\_sector}}

\begin{fulllineitems}
\phantomsection\label{\detokenize{api:muse.sectors.legacy_sector.LegacySector}}\pysiglinewithargsret{\sphinxbfcode{\sphinxupquote{class }}\sphinxcode{\sphinxupquote{muse.sectors.legacy\_sector.}}\sphinxbfcode{\sphinxupquote{LegacySector}}}{\emph{\DUrole{n}{name}\DUrole{p}{:} \DUrole{n}{str}}, \emph{\DUrole{n}{old\_sector}}, \emph{\DUrole{n}{timeslices}\DUrole{p}{:} \DUrole{n}{Dict}}, \emph{\DUrole{n}{commodities}\DUrole{p}{:} \DUrole{n}{Dict}}, \emph{\DUrole{n}{commodity\_price}\DUrole{p}{:} \DUrole{n}{xarray.core.dataarray.DataArray}}, \emph{\DUrole{n}{static\_trade}\DUrole{p}{:} \DUrole{n}{xarray.core.dataarray.DataArray}}, \emph{\DUrole{n}{regions}\DUrole{p}{:} \DUrole{n}{Sequence}}, \emph{\DUrole{n}{time\_framework}\DUrole{p}{:} \DUrole{n}{numpy.ndarray}}, \emph{\DUrole{n}{mode}\DUrole{p}{:} \DUrole{n}{str}}, \emph{\DUrole{n}{excess}\DUrole{p}{:} \DUrole{n}{Union\DUrole{p}{{[}}int\DUrole{p}{, }float\DUrole{p}{{]}}}}, \emph{\DUrole{n}{market\_iterative}\DUrole{p}{:} \DUrole{n}{str}}, \emph{\DUrole{n}{sectors\_dir}\DUrole{p}{:} \DUrole{n}{str}}, \emph{\DUrole{n}{output\_dir}\DUrole{p}{:} \DUrole{n}{str}}}{}~\index{calibrated (muse.sectors.legacy\_sector.LegacySector attribute)@\spxentry{calibrated}\spxextra{muse.sectors.legacy\_sector.LegacySector attribute}}

\begin{fulllineitems}
\phantomsection\label{\detokenize{api:muse.sectors.legacy_sector.LegacySector.calibrated}}\pysigline{\sphinxbfcode{\sphinxupquote{calibrated}}}
Flag if the sector has gone through the calibration process.

\end{fulllineitems}

\index{commodities (muse.sectors.legacy\_sector.LegacySector attribute)@\spxentry{commodities}\spxextra{muse.sectors.legacy\_sector.LegacySector attribute}}

\begin{fulllineitems}
\phantomsection\label{\detokenize{api:muse.sectors.legacy_sector.LegacySector.commodities}}\pysigline{\sphinxbfcode{\sphinxupquote{commodities}}}
Commodities for each sector, as well as global commodities.

\end{fulllineitems}

\index{commodity\_price (muse.sectors.legacy\_sector.LegacySector attribute)@\spxentry{commodity\_price}\spxextra{muse.sectors.legacy\_sector.LegacySector attribute}}

\begin{fulllineitems}
\phantomsection\label{\detokenize{api:muse.sectors.legacy_sector.LegacySector.commodity_price}}\pysigline{\sphinxbfcode{\sphinxupquote{commodity\_price}}}
Initial price of all the commodities.

\end{fulllineitems}

\index{dims (muse.sectors.legacy\_sector.LegacySector attribute)@\spxentry{dims}\spxextra{muse.sectors.legacy\_sector.LegacySector attribute}}

\begin{fulllineitems}
\phantomsection\label{\detokenize{api:muse.sectors.legacy_sector.LegacySector.dims}}\pysigline{\sphinxbfcode{\sphinxupquote{dims}}}
Order of the input and output dimensions.

\end{fulllineitems}

\index{excess (muse.sectors.legacy\_sector.LegacySector attribute)@\spxentry{excess}\spxextra{muse.sectors.legacy\_sector.LegacySector attribute}}

\begin{fulllineitems}
\phantomsection\label{\detokenize{api:muse.sectors.legacy_sector.LegacySector.excess}}\pysigline{\sphinxbfcode{\sphinxupquote{excess}}}
Allowed excess of capacity.

\end{fulllineitems}

\index{factory() (muse.sectors.legacy\_sector.LegacySector class method)@\spxentry{factory()}\spxextra{muse.sectors.legacy\_sector.LegacySector class method}}

\begin{fulllineitems}
\phantomsection\label{\detokenize{api:muse.sectors.legacy_sector.LegacySector.factory}}\pysiglinewithargsret{\sphinxbfcode{\sphinxupquote{classmethod }}\sphinxbfcode{\sphinxupquote{factory}}}{\emph{\DUrole{n}{name}\DUrole{p}{:} \DUrole{n}{str}}, \emph{\DUrole{n}{settings}\DUrole{p}{:} \DUrole{n}{Any}}, \emph{\DUrole{o}{**}\DUrole{n}{kwargs}}}{{ $\rightarrow$ muse.sectors.legacy\_sector.LegacySector}}
Creates class from settings named\sphinxhyphen{}tuple.

\end{fulllineitems}

\index{global\_commodities() (muse.sectors.legacy\_sector.LegacySector property)@\spxentry{global\_commodities()}\spxextra{muse.sectors.legacy\_sector.LegacySector property}}

\begin{fulllineitems}
\phantomsection\label{\detokenize{api:muse.sectors.legacy_sector.LegacySector.global_commodities}}\pysigline{\sphinxbfcode{\sphinxupquote{property }}\sphinxbfcode{\sphinxupquote{global\_commodities}}}
List of all commodities used by the MCA.

\end{fulllineitems}

\index{inputs() (muse.sectors.legacy\_sector.LegacySector method)@\spxentry{inputs()}\spxextra{muse.sectors.legacy\_sector.LegacySector method}}

\begin{fulllineitems}
\phantomsection\label{\detokenize{api:muse.sectors.legacy_sector.LegacySector.inputs}}\pysiglinewithargsret{\sphinxbfcode{\sphinxupquote{inputs}}}{\emph{\DUrole{n}{consumption}\DUrole{p}{:} \DUrole{n}{xarray.core.dataarray.DataArray}}, \emph{\DUrole{n}{prices}\DUrole{p}{:} \DUrole{n}{xarray.core.dataarray.DataArray}}, \emph{\DUrole{n}{supply}\DUrole{p}{:} \DUrole{n}{xarray.core.dataarray.DataArray}}}{}
Converts xarray to MUSE numpy input arrays.

\end{fulllineitems}

\index{load\_timeslices\_and\_aggregation() (muse.sectors.legacy\_sector.LegacySector static method)@\spxentry{load\_timeslices\_and\_aggregation()}\spxextra{muse.sectors.legacy\_sector.LegacySector static method}}

\begin{fulllineitems}
\phantomsection\label{\detokenize{api:muse.sectors.legacy_sector.LegacySector.load_timeslices_and_aggregation}}\pysiglinewithargsret{\sphinxbfcode{\sphinxupquote{static }}\sphinxbfcode{\sphinxupquote{load\_timeslices\_and\_aggregation}}}{\emph{\DUrole{n}{timeslices}}, \emph{\DUrole{n}{sectors}}}{{ $\rightarrow$ Tuple\DUrole{p}{{[}}dict\DUrole{p}{, }str\DUrole{p}{{]}}}}
Loads all sector timeslices and finds the finest one.

\end{fulllineitems}

\index{market\_iterative (muse.sectors.legacy\_sector.LegacySector attribute)@\spxentry{market\_iterative}\spxextra{muse.sectors.legacy\_sector.LegacySector attribute}}

\begin{fulllineitems}
\phantomsection\label{\detokenize{api:muse.sectors.legacy_sector.LegacySector.market_iterative}}\pysigline{\sphinxbfcode{\sphinxupquote{market\_iterative}}}
—\textendash{}\textgreater{} TODO what’s this parameter?

\end{fulllineitems}

\index{mode (muse.sectors.legacy\_sector.LegacySector attribute)@\spxentry{mode}\spxextra{muse.sectors.legacy\_sector.LegacySector attribute}}

\begin{fulllineitems}
\phantomsection\label{\detokenize{api:muse.sectors.legacy_sector.LegacySector.mode}}\pysigline{\sphinxbfcode{\sphinxupquote{mode}}}
If ‘Calibration’, the sector runs in calibration mode

\end{fulllineitems}

\index{name (muse.sectors.legacy\_sector.LegacySector attribute)@\spxentry{name}\spxextra{muse.sectors.legacy\_sector.LegacySector attribute}}

\begin{fulllineitems}
\phantomsection\label{\detokenize{api:muse.sectors.legacy_sector.LegacySector.name}}\pysigline{\sphinxbfcode{\sphinxupquote{name}}}
Name of the sector

\end{fulllineitems}

\index{next() (muse.sectors.legacy\_sector.LegacySector method)@\spxentry{next()}\spxextra{muse.sectors.legacy\_sector.LegacySector method}}

\begin{fulllineitems}
\phantomsection\label{\detokenize{api:muse.sectors.legacy_sector.LegacySector.next}}\pysiglinewithargsret{\sphinxbfcode{\sphinxupquote{next}}}{\emph{\DUrole{n}{market}\DUrole{p}{:} \DUrole{n}{xarray.core.dataset.Dataset}}}{{ $\rightarrow$ xarray.core.dataset.Dataset}}
Adapter between the old and the new.

\end{fulllineitems}

\index{old\_sector (muse.sectors.legacy\_sector.LegacySector attribute)@\spxentry{old\_sector}\spxextra{muse.sectors.legacy\_sector.LegacySector attribute}}

\begin{fulllineitems}
\phantomsection\label{\detokenize{api:muse.sectors.legacy_sector.LegacySector.old_sector}}\pysigline{\sphinxbfcode{\sphinxupquote{old\_sector}}}
Legacy sector method to run the calculation

\end{fulllineitems}

\index{output\_dir (muse.sectors.legacy\_sector.LegacySector attribute)@\spxentry{output\_dir}\spxextra{muse.sectors.legacy\_sector.LegacySector attribute}}

\begin{fulllineitems}
\phantomsection\label{\detokenize{api:muse.sectors.legacy_sector.LegacySector.output_dir}}\pysigline{\sphinxbfcode{\sphinxupquote{output\_dir}}}
Outputs directory.

\end{fulllineitems}

\index{outputs() (muse.sectors.legacy\_sector.LegacySector method)@\spxentry{outputs()}\spxextra{muse.sectors.legacy\_sector.LegacySector method}}

\begin{fulllineitems}
\phantomsection\label{\detokenize{api:muse.sectors.legacy_sector.LegacySector.outputs}}\pysiglinewithargsret{\sphinxbfcode{\sphinxupquote{outputs}}}{\emph{\DUrole{n}{consumption}\DUrole{p}{:} \DUrole{n}{numpy.ndarray}}, \emph{\DUrole{n}{prices}\DUrole{p}{:} \DUrole{n}{numpy.ndarray}}, \emph{\DUrole{n}{supply}\DUrole{p}{:} \DUrole{n}{numpy.ndarray}}}{{ $\rightarrow$ xarray.core.dataset.Dataset}}
Converts MUSE numpy outputs to xarray.

\end{fulllineitems}

\index{regions (muse.sectors.legacy\_sector.LegacySector attribute)@\spxentry{regions}\spxextra{muse.sectors.legacy\_sector.LegacySector attribute}}

\begin{fulllineitems}
\phantomsection\label{\detokenize{api:muse.sectors.legacy_sector.LegacySector.regions}}\pysigline{\sphinxbfcode{\sphinxupquote{regions}}}
Regions taking part in the simulation.

\end{fulllineitems}

\index{sector\_commodities() (muse.sectors.legacy\_sector.LegacySector property)@\spxentry{sector\_commodities()}\spxextra{muse.sectors.legacy\_sector.LegacySector property}}

\begin{fulllineitems}
\phantomsection\label{\detokenize{api:muse.sectors.legacy_sector.LegacySector.sector_commodities}}\pysigline{\sphinxbfcode{\sphinxupquote{property }}\sphinxbfcode{\sphinxupquote{sector\_commodities}}}
List of all commodities used by the Sector.

\end{fulllineitems}

\index{sector\_timeslices() (muse.sectors.legacy\_sector.LegacySector property)@\spxentry{sector\_timeslices()}\spxextra{muse.sectors.legacy\_sector.LegacySector property}}

\begin{fulllineitems}
\phantomsection\label{\detokenize{api:muse.sectors.legacy_sector.LegacySector.sector_timeslices}}\pysigline{\sphinxbfcode{\sphinxupquote{property }}\sphinxbfcode{\sphinxupquote{sector\_timeslices}}}
List of all commodities used by the MCA.

\end{fulllineitems}

\index{sectors\_dir (muse.sectors.legacy\_sector.LegacySector attribute)@\spxentry{sectors\_dir}\spxextra{muse.sectors.legacy\_sector.LegacySector attribute}}

\begin{fulllineitems}
\phantomsection\label{\detokenize{api:muse.sectors.legacy_sector.LegacySector.sectors_dir}}\pysigline{\sphinxbfcode{\sphinxupquote{sectors\_dir}}}
Sectors directory.

\end{fulllineitems}

\index{static\_trade (muse.sectors.legacy\_sector.LegacySector attribute)@\spxentry{static\_trade}\spxextra{muse.sectors.legacy\_sector.LegacySector attribute}}

\begin{fulllineitems}
\phantomsection\label{\detokenize{api:muse.sectors.legacy_sector.LegacySector.static_trade}}\pysigline{\sphinxbfcode{\sphinxupquote{static\_trade}}}
Static trade needed for the conversion and supply sectors.

\end{fulllineitems}

\index{time\_framework (muse.sectors.legacy\_sector.LegacySector attribute)@\spxentry{time\_framework}\spxextra{muse.sectors.legacy\_sector.LegacySector attribute}}

\begin{fulllineitems}
\phantomsection\label{\detokenize{api:muse.sectors.legacy_sector.LegacySector.time_framework}}\pysigline{\sphinxbfcode{\sphinxupquote{time\_framework}}}
Time framework of the complete simulation.

\end{fulllineitems}

\index{timeslices (muse.sectors.legacy\_sector.LegacySector attribute)@\spxentry{timeslices}\spxextra{muse.sectors.legacy\_sector.LegacySector attribute}}

\begin{fulllineitems}
\phantomsection\label{\detokenize{api:muse.sectors.legacy_sector.LegacySector.timeslices}}\pysigline{\sphinxbfcode{\sphinxupquote{timeslices}}}
Timeslices for sectors and mca.

\end{fulllineitems}


\end{fulllineitems}



\subsection{Production}
\label{\detokenize{api:module-muse.production}}\label{\detokenize{api:production}}\index{module@\spxentry{module}!muse.production@\spxentry{muse.production}}\index{muse.production@\spxentry{muse.production}!module@\spxentry{module}}
Various ways and means to compute production.

Production is the amount of commodities produced by an asset. However, depending on the
context, it could be computed several ways. For  instace, it can be obtained straight
from the capacity of the asset. Or it can be obtained by matching for the same
commodities with a set of assets.

Production methods can be registered via the \sphinxcode{\sphinxupquote{@register\_production}} production decorator.  Registering a function makes the function
accessible from MUSE’s input file. Production methods are not expected to modify their
arguments. Furthermore they should conform the
following signatures:

\begin{sphinxVerbatim}[commandchars=\\\{\}]
\PYG{n+nd}{@register\PYGZus{}production}
\PYG{k}{def} \PYG{n+nf}{production}\PYG{p}{(}
    \PYG{n}{market}\PYG{p}{:} \PYG{n}{xr}\PYG{o}{.}\PYG{n}{Dataset}\PYG{p}{,} \PYG{n}{capacity}\PYG{p}{:} \PYG{n}{xr}\PYG{o}{.}\PYG{n}{DataArray}\PYG{p}{,} \PYG{n}{technologies}\PYG{p}{:} \PYG{n}{xr}\PYG{o}{.}\PYG{n}{Dataset}\PYG{p}{,} \PYG{o}{*}\PYG{o}{*}\PYG{n}{kwargs}
\PYG{p}{)} \PYG{o}{\PYGZhy{}}\PYG{o}{\PYGZgt{}} \PYG{n}{xr}\PYG{o}{.}\PYG{n}{DataArray}\PYG{p}{:}
    \PYG{k}{pass}
\end{sphinxVerbatim}
\begin{quote}\begin{description}
\item[{param market}] \leavevmode
Market, including demand and prices.

\item[{param capacity}] \leavevmode
The capacity of each asset within a market.

\item[{param technologies}] \leavevmode
A dataset characterising the technologies of the same assets.

\item[{param **kwargs}] \leavevmode
Any number of keyword arguments

\item[{returns}] \leavevmode
A \sphinxtitleref{xr.DataArray} with the amount produced for each good from each asset.

\end{description}\end{quote}
\index{PRODUCTION\_SIGNATURE (in module muse.production)@\spxentry{PRODUCTION\_SIGNATURE}\spxextra{in module muse.production}}

\begin{fulllineitems}
\phantomsection\label{\detokenize{api:muse.production.PRODUCTION_SIGNATURE}}\pysigline{\sphinxcode{\sphinxupquote{muse.production.}}\sphinxbfcode{\sphinxupquote{PRODUCTION\_SIGNATURE}}}
Production signature.

alias of Callable{[}{[}xarray.core.dataarray.DataArray, xarray.core.dataarray.DataArray, xarray.core.dataset.Dataset{]}, xarray.core.dataarray.DataArray{]}

\end{fulllineitems}

\index{demand\_matched\_production() (in module muse.production)@\spxentry{demand\_matched\_production()}\spxextra{in module muse.production}}

\begin{fulllineitems}
\phantomsection\label{\detokenize{api:muse.production.demand_matched_production}}\pysiglinewithargsret{\sphinxcode{\sphinxupquote{muse.production.}}\sphinxbfcode{\sphinxupquote{demand\_matched\_production}}}{\emph{\DUrole{n}{market}\DUrole{p}{:} \DUrole{n}{xarray.core.dataset.Dataset}}, \emph{\DUrole{n}{capacity}\DUrole{p}{:} \DUrole{n}{xarray.core.dataarray.DataArray}}, \emph{\DUrole{n}{technologies}\DUrole{p}{:} \DUrole{n}{xarray.core.dataset.Dataset}}, \emph{\DUrole{n}{costs}\DUrole{p}{:} \DUrole{n}{str} \DUrole{o}{=} \DUrole{default_value}{\textquotesingle{}prices\textquotesingle{}}}}{{ $\rightarrow$ xarray.core.dataarray.DataArray}}
Production from matching demand via annual lcoe.

\end{fulllineitems}

\index{factory() (in module muse.production)@\spxentry{factory()}\spxextra{in module muse.production}}

\begin{fulllineitems}
\phantomsection\label{\detokenize{api:muse.production.factory}}\pysiglinewithargsret{\sphinxcode{\sphinxupquote{muse.production.}}\sphinxbfcode{\sphinxupquote{factory}}}{\emph{\DUrole{n}{settings}\DUrole{p}{:} \DUrole{n}{Union\DUrole{p}{{[}}str\DUrole{p}{, }Mapping\DUrole{p}{{]}}} \DUrole{o}{=} \DUrole{default_value}{\textquotesingle{}maximum\_production\textquotesingle{}}}, \emph{\DUrole{o}{**}\DUrole{n}{kwargs}}}{{ $\rightarrow$ Callable\DUrole{p}{{[}}\DUrole{p}{{[}}xarray.core.dataarray.DataArray\DUrole{p}{, }xarray.core.dataarray.DataArray\DUrole{p}{, }xarray.core.dataset.Dataset\DUrole{p}{{]}}\DUrole{p}{, }xarray.core.dataarray.DataArray\DUrole{p}{{]}}}}
Creates a production functor.

This function’s raison d’être is to convert the input from a TOML file into an
actual functor usable within the model, i.e. it converts data into logic.
\begin{quote}\begin{description}
\item[{Parameters}] \leavevmode\begin{itemize}
\item {} 
\sphinxstyleliteralstrong{\sphinxupquote{name}} \textendash{} Registered production method to create. The name is resolved when the
function returned by the factory is called. Hence, it could refer to a
function yet to be registered when this factory method is called.

\item {} 
\sphinxstyleliteralstrong{\sphinxupquote{**kwargs}} \textendash{} any keyword argument the production method accepts.

\end{itemize}

\end{description}\end{quote}

\end{fulllineitems}

\index{maximum\_production() (in module muse.production)@\spxentry{maximum\_production()}\spxextra{in module muse.production}}

\begin{fulllineitems}
\phantomsection\label{\detokenize{api:muse.production.maximum_production}}\pysiglinewithargsret{\sphinxcode{\sphinxupquote{muse.production.}}\sphinxbfcode{\sphinxupquote{maximum\_production}}}{\emph{\DUrole{n}{market}\DUrole{p}{:} \DUrole{n}{xarray.core.dataset.Dataset}}, \emph{\DUrole{n}{capacity}\DUrole{p}{:} \DUrole{n}{xarray.core.dataarray.DataArray}}, \emph{\DUrole{n}{technologies}\DUrole{p}{:} \DUrole{n}{xarray.core.dataset.Dataset}}}{{ $\rightarrow$ xarray.core.dataarray.DataArray}}
Production when running at full capacity.

\sphinxstyleemphasis{Full capacity} is limited by the utilitization factor. For more details, see
\sphinxcode{\sphinxupquote{muse.quantities.maximum\_production()}}.

\end{fulllineitems}

\index{register\_production() (in module muse.production)@\spxentry{register\_production()}\spxextra{in module muse.production}}

\begin{fulllineitems}
\phantomsection\label{\detokenize{api:muse.production.register_production}}\pysiglinewithargsret{\sphinxcode{\sphinxupquote{muse.production.}}\sphinxbfcode{\sphinxupquote{register\_production}}}{\emph{\DUrole{n}{function}\DUrole{p}{:} \DUrole{n}{Callable\DUrole{p}{{[}}\DUrole{p}{{[}}xarray.core.dataarray.DataArray\DUrole{p}{, }xarray.core.dataarray.DataArray\DUrole{p}{, }xarray.core.dataset.Dataset\DUrole{p}{{]}}\DUrole{p}{, }xarray.core.dataarray.DataArray\DUrole{p}{{]}}} \DUrole{o}{=} \DUrole{default_value}{None}}}{}
Decorator to register a function as a production method.


\sphinxstrong{See also:}


\sphinxcode{\sphinxupquote{muse.production}}



\end{fulllineitems}

\index{supply() (in module muse.production)@\spxentry{supply()}\spxextra{in module muse.production}}

\begin{fulllineitems}
\phantomsection\label{\detokenize{api:muse.production.supply}}\pysiglinewithargsret{\sphinxcode{\sphinxupquote{muse.production.}}\sphinxbfcode{\sphinxupquote{supply}}}{\emph{\DUrole{n}{market}\DUrole{p}{:} \DUrole{n}{xarray.core.dataset.Dataset}}, \emph{\DUrole{n}{capacity}\DUrole{p}{:} \DUrole{n}{xarray.core.dataarray.DataArray}}, \emph{\DUrole{n}{technologies}\DUrole{p}{:} \DUrole{n}{xarray.core.dataset.Dataset}}}{{ $\rightarrow$ xarray.core.dataarray.DataArray}}
Service current demand equally from all assets.

“Equally” means that equivalent technologies are used to the same percentage of
their respective capacity.

\end{fulllineitems}



\subsection{Agent Interactions}
\label{\detokenize{api:module-muse.interactions}}\label{\detokenize{api:agent-interactions}}\index{module@\spxentry{module}!muse.interactions@\spxentry{muse.interactions}}\index{muse.interactions@\spxentry{muse.interactions}!module@\spxentry{module}}
Modes of interactions between agents.

Interactions between agents are modelled via two orthogonal concepts:
\begin{itemize}
\item {} 
a \sphinxstyleemphasis{net} is a set of agents which interact in some way

\item {} 
an \sphinxstyleemphasis{interaction} proper is a function that takes a net and actually performs the
interaction.

\end{itemize}

Hence, there are two registrators in this this module,
\sphinxcode{\sphinxupquote{register\_interaction\_net()}}, and \sphinxcode{\sphinxupquote{register\_agent\_interaction()}}. The
first registers functions that take the full set of agents as input and returns a
sequence of nets. It is expected each net of the sequence will be applied the same
interaction. The second registrator registers the interaction proper: it takes agents as
arguments and returns nothing. It is expected to modify the agents in\sphinxhyphen{}place.
\index{factory() (in module muse.interactions)@\spxentry{factory()}\spxextra{in module muse.interactions}}

\begin{fulllineitems}
\phantomsection\label{\detokenize{api:muse.interactions.factory}}\pysiglinewithargsret{\sphinxcode{\sphinxupquote{muse.interactions.}}\sphinxbfcode{\sphinxupquote{factory}}}{\emph{\DUrole{n}{inputs}\DUrole{p}{:} \DUrole{n}{Optional\DUrole{p}{{[}}Sequence\DUrole{p}{{[}}Union\DUrole{p}{{[}}Mapping\DUrole{p}{, }Tuple\DUrole{p}{{[}}str\DUrole{p}{, }str\DUrole{p}{{]}}\DUrole{p}{{]}}\DUrole{p}{{]}}\DUrole{p}{{]}}} \DUrole{o}{=} \DUrole{default_value}{None}}}{{ $\rightarrow$ Callable{[}{[}Sequence{[}muse.agents.agent.AbstractAgent{]}{]}, None{]}}}
Creates an interaction functor.

\end{fulllineitems}

\index{new\_to\_retro\_net() (in module muse.interactions)@\spxentry{new\_to\_retro\_net()}\spxextra{in module muse.interactions}}

\begin{fulllineitems}
\phantomsection\label{\detokenize{api:muse.interactions.new_to_retro_net}}\pysiglinewithargsret{\sphinxcode{\sphinxupquote{muse.interactions.}}\sphinxbfcode{\sphinxupquote{new\_to\_retro\_net}}}{\emph{\DUrole{n}{agents}\DUrole{p}{:} \DUrole{n}{Sequence\DUrole{p}{{[}}muse.agents.agent.Agent\DUrole{p}{{]}}}}, \emph{\DUrole{n}{first\_category}\DUrole{p}{:} \DUrole{n}{str} \DUrole{o}{=} \DUrole{default_value}{\textquotesingle{}newcapa\textquotesingle{}}}}{{ $\rightarrow$ Sequence\DUrole{p}{{[}}Sequence\DUrole{p}{{[}}muse.agents.agent.Agent\DUrole{p}{{]}}\DUrole{p}{{]}}}}
Interactions between new and retrofit agents.

\end{fulllineitems}

\index{register\_agent\_interaction() (in module muse.interactions)@\spxentry{register\_agent\_interaction()}\spxextra{in module muse.interactions}}

\begin{fulllineitems}
\phantomsection\label{\detokenize{api:muse.interactions.register_agent_interaction}}\pysiglinewithargsret{\sphinxcode{\sphinxupquote{muse.interactions.}}\sphinxbfcode{\sphinxupquote{register\_agent\_interaction}}}{\emph{\DUrole{n}{function}\DUrole{p}{:} \DUrole{n}{Callable{[}{[}muse.agents.agent.Agent, muse.agents.agent.Agent{]}, None{]}}}}{}
Decorator to register an agent to agent(s) interaction function.

An agent interaction function takes at least two agents and makes them
interact in some way.

An agent interaction function also takes as argument a sector object.
This object should not be modified in any way. But it can be queried for
parameters, if the specific agent interaction function requires it. This is
most likely the same configuration object passed on to the interaction net
function.

\end{fulllineitems}

\index{register\_interaction\_net() (in module muse.interactions)@\spxentry{register\_interaction\_net()}\spxextra{in module muse.interactions}}

\begin{fulllineitems}
\phantomsection\label{\detokenize{api:muse.interactions.register_interaction_net}}\pysiglinewithargsret{\sphinxcode{\sphinxupquote{muse.interactions.}}\sphinxbfcode{\sphinxupquote{register\_interaction\_net}}}{\emph{\DUrole{n}{function}\DUrole{p}{:} \DUrole{n}{Callable\DUrole{p}{{[}}\DUrole{p}{{[}}Sequence\DUrole{p}{{[}}muse.agents.agent.Agent\DUrole{p}{{]}}\DUrole{p}{{]}}\DUrole{p}{, }Sequence\DUrole{p}{{[}}Sequence\DUrole{p}{{[}}muse.agents.agent.Agent\DUrole{p}{{]}}\DUrole{p}{{]}}\DUrole{p}{{]}}}}}{}
Decorator to register a function computing interaction nets.

An interaction net function takes as input the list of all agents and
returns the list of all interactions, where an interaction is a list of at
least two interacting agents.

An interactiont\sphinxhyphen{}net function also takes as argument a sector object.
This object should not be modified in any way. But it can be queried for
parameters, if the specific interaction\sphinxhyphen{}net function requires it.

\end{fulllineitems}

\index{transfer\_assets() (in module muse.interactions)@\spxentry{transfer\_assets()}\spxextra{in module muse.interactions}}

\begin{fulllineitems}
\phantomsection\label{\detokenize{api:muse.interactions.transfer_assets}}\pysiglinewithargsret{\sphinxcode{\sphinxupquote{muse.interactions.}}\sphinxbfcode{\sphinxupquote{transfer\_assets}}}{\emph{\DUrole{n}{from\_}\DUrole{p}{:} \DUrole{n}{muse.agents.agent.Agent}}, \emph{\DUrole{n}{to\_}\DUrole{p}{:} \DUrole{n}{muse.agents.agent.Agent}}}{{ $\rightarrow$ None}}
Transfer assets from first agent to second agent.

\end{fulllineitems}



\section{Agents and associated functionalities}
\label{\detokenize{api:module-muse.agents.factories}}\label{\detokenize{api:agents-and-associated-functionalities}}\index{module@\spxentry{module}!muse.agents.factories@\spxentry{muse.agents.factories}}\index{muse.agents.factories@\spxentry{muse.agents.factories}!module@\spxentry{module}}
Holds all building agents.
\index{agents\_factory() (in module muse.agents.factories)@\spxentry{agents\_factory()}\spxextra{in module muse.agents.factories}}

\begin{fulllineitems}
\phantomsection\label{\detokenize{api:muse.agents.factories.agents_factory}}\pysiglinewithargsret{\sphinxcode{\sphinxupquote{muse.agents.factories.}}\sphinxbfcode{\sphinxupquote{agents\_factory}}}{\emph{\DUrole{n}{params\_or\_path}\DUrole{p}{:} \DUrole{n}{Union\DUrole{p}{{[}}str\DUrole{p}{, }pathlib.Path\DUrole{p}{, }List\DUrole{p}{{]}}}}, \emph{\DUrole{n}{capacity}\DUrole{p}{:} \DUrole{n}{Union\DUrole{p}{{[}}xarray.core.dataarray.DataArray\DUrole{p}{, }str\DUrole{p}{, }pathlib.Path\DUrole{p}{{]}}}}, \emph{\DUrole{n}{technologies}\DUrole{p}{:} \DUrole{n}{xarray.core.dataset.Dataset}}, \emph{\DUrole{n}{regions}\DUrole{p}{:} \DUrole{n}{Optional\DUrole{p}{{[}}Sequence\DUrole{p}{{[}}str\DUrole{p}{{]}}\DUrole{p}{{]}}} \DUrole{o}{=} \DUrole{default_value}{None}}, \emph{\DUrole{n}{year}\DUrole{p}{:} \DUrole{n}{Optional\DUrole{p}{{[}}int\DUrole{p}{{]}}} \DUrole{o}{=} \DUrole{default_value}{None}}, \emph{\DUrole{o}{**}\DUrole{n}{kwargs}}}{{ $\rightarrow$ List\DUrole{p}{{[}}muse.agents.agent.Agent\DUrole{p}{{]}}}}
Creates a list of agents for the chosen sector.

\end{fulllineitems}

\index{create\_newcapa\_agent() (in module muse.agents.factories)@\spxentry{create\_newcapa\_agent()}\spxextra{in module muse.agents.factories}}

\begin{fulllineitems}
\phantomsection\label{\detokenize{api:muse.agents.factories.create_newcapa_agent}}\pysiglinewithargsret{\sphinxcode{\sphinxupquote{muse.agents.factories.}}\sphinxbfcode{\sphinxupquote{create\_newcapa\_agent}}}{\emph{\DUrole{n}{capacity}\DUrole{p}{:} \DUrole{n}{xarray.core.dataarray.DataArray}}, \emph{\DUrole{n}{year}\DUrole{p}{:} \DUrole{n}{int}}, \emph{\DUrole{n}{region}\DUrole{p}{:} \DUrole{n}{str}}, \emph{\DUrole{n}{search\_rules}\DUrole{p}{:} \DUrole{n}{Union\DUrole{p}{{[}}str\DUrole{p}{, }Sequence\DUrole{p}{{[}}str\DUrole{p}{{]}}\DUrole{p}{{]}}} \DUrole{o}{=} \DUrole{default_value}{\textquotesingle{}all\textquotesingle{}}}, \emph{\DUrole{n}{interpolation}\DUrole{p}{:} \DUrole{n}{str} \DUrole{o}{=} \DUrole{default_value}{\textquotesingle{}linear\textquotesingle{}}}, \emph{\DUrole{n}{merge\_transform}\DUrole{p}{:} \DUrole{n}{Union\DUrole{p}{{[}}str\DUrole{p}{, }Mapping\DUrole{p}{, }Callable\DUrole{p}{{]}}} \DUrole{o}{=} \DUrole{default_value}{\textquotesingle{}new\textquotesingle{}}}, \emph{\DUrole{n}{quantity}\DUrole{p}{:} \DUrole{n}{float} \DUrole{o}{=} \DUrole{default_value}{0.3}}, \emph{\DUrole{n}{housekeeping}\DUrole{p}{:} \DUrole{n}{Union\DUrole{p}{{[}}str\DUrole{p}{, }Mapping\DUrole{p}{, }Callable\DUrole{p}{{]}}} \DUrole{o}{=} \DUrole{default_value}{\textquotesingle{}noop\textquotesingle{}}}, \emph{\DUrole{o}{**}\DUrole{n}{kwargs}}}{}
Creates newcapa agent from muse primitives.

\end{fulllineitems}

\index{create\_retrofit\_agent() (in module muse.agents.factories)@\spxentry{create\_retrofit\_agent()}\spxextra{in module muse.agents.factories}}

\begin{fulllineitems}
\phantomsection\label{\detokenize{api:muse.agents.factories.create_retrofit_agent}}\pysiglinewithargsret{\sphinxcode{\sphinxupquote{muse.agents.factories.}}\sphinxbfcode{\sphinxupquote{create\_retrofit\_agent}}}{\emph{\DUrole{n}{technologies}\DUrole{p}{:} \DUrole{n}{xarray.core.dataset.Dataset}}, \emph{\DUrole{n}{capacity}\DUrole{p}{:} \DUrole{n}{xarray.core.dataarray.DataArray}}, \emph{\DUrole{n}{share}\DUrole{p}{:} \DUrole{n}{str}}, \emph{\DUrole{n}{year}\DUrole{p}{:} \DUrole{n}{int}}, \emph{\DUrole{n}{region}\DUrole{p}{:} \DUrole{n}{str}}, \emph{\DUrole{n}{interpolation}\DUrole{p}{:} \DUrole{n}{str} \DUrole{o}{=} \DUrole{default_value}{\textquotesingle{}linear\textquotesingle{}}}, \emph{\DUrole{n}{decision}\DUrole{p}{:} \DUrole{n}{Union\DUrole{p}{{[}}Callable\DUrole{p}{, }str\DUrole{p}{, }Mapping\DUrole{p}{{]}}} \DUrole{o}{=} \DUrole{default_value}{\textquotesingle{}mean\textquotesingle{}}}, \emph{\DUrole{o}{**}\DUrole{n}{kwargs}}}{}
Creates retrofit agent from muse primitives.

\end{fulllineitems}

\index{factory() (in module muse.agents.factories)@\spxentry{factory()}\spxextra{in module muse.agents.factories}}

\begin{fulllineitems}
\phantomsection\label{\detokenize{api:muse.agents.factories.factory}}\pysiglinewithargsret{\sphinxcode{\sphinxupquote{muse.agents.factories.}}\sphinxbfcode{\sphinxupquote{factory}}}{\emph{\DUrole{n}{existing\_capacity\_path}\DUrole{p}{:} \DUrole{n}{Optional\DUrole{p}{{[}}Union\DUrole{p}{{[}}str\DUrole{p}{, }pathlib.Path\DUrole{p}{{]}}\DUrole{p}{{]}}} \DUrole{o}{=} \DUrole{default_value}{None}}, \emph{\DUrole{n}{agent\_parameters\_path}\DUrole{p}{:} \DUrole{n}{Optional\DUrole{p}{{[}}Union\DUrole{p}{{[}}str\DUrole{p}{, }pathlib.Path\DUrole{p}{{]}}\DUrole{p}{{]}}} \DUrole{o}{=} \DUrole{default_value}{None}}, \emph{\DUrole{n}{technodata\_path}\DUrole{p}{:} \DUrole{n}{Optional\DUrole{p}{{[}}Union\DUrole{p}{{[}}str\DUrole{p}{, }pathlib.Path\DUrole{p}{{]}}\DUrole{p}{{]}}} \DUrole{o}{=} \DUrole{default_value}{None}}, \emph{\DUrole{n}{sector}\DUrole{p}{:} \DUrole{n}{Optional\DUrole{p}{{[}}str\DUrole{p}{{]}}} \DUrole{o}{=} \DUrole{default_value}{None}}, \emph{\DUrole{n}{sectors\_directory}\DUrole{p}{:} \DUrole{n}{Union\DUrole{p}{{[}}str\DUrole{p}{, }pathlib.Path\DUrole{p}{{]}}} \DUrole{o}{=} \DUrole{default_value}{PosixPath(\textquotesingle{}/Users/alexkell/Documents/SGI/2\sphinxhyphen{}documentation/StarMuse/docs/data\textquotesingle{})}}, \emph{\DUrole{n}{baseyear}\DUrole{p}{:} \DUrole{n}{int} \DUrole{o}{=} \DUrole{default_value}{2010}}}{{ $\rightarrow$ List\DUrole{p}{{[}}muse.agents.agent.Agent\DUrole{p}{{]}}}}
Reads list of agents from standard MUSE input files.

\end{fulllineitems}

\index{AbstractAgent (class in muse.agents.agent)@\spxentry{AbstractAgent}\spxextra{class in muse.agents.agent}}

\begin{fulllineitems}
\phantomsection\label{\detokenize{api:muse.agents.agent.AbstractAgent}}\pysiglinewithargsret{\sphinxbfcode{\sphinxupquote{class }}\sphinxcode{\sphinxupquote{muse.agents.agent.}}\sphinxbfcode{\sphinxupquote{AbstractAgent}}}{\emph{\DUrole{n}{name}\DUrole{p}{:} \DUrole{n}{str} \DUrole{o}{=} \DUrole{default_value}{\textquotesingle{}Agent\textquotesingle{}}}, \emph{\DUrole{n}{region}\DUrole{p}{:} \DUrole{n}{str} \DUrole{o}{=} \DUrole{default_value}{\textquotesingle{}\textquotesingle{}}}, \emph{\DUrole{n}{assets}\DUrole{p}{:} \DUrole{n}{Optional\DUrole{p}{{[}}xarray.core.dataset.Dataset\DUrole{p}{{]}}} \DUrole{o}{=} \DUrole{default_value}{None}}, \emph{\DUrole{n}{interpolation}\DUrole{p}{:} \DUrole{n}{str} \DUrole{o}{=} \DUrole{default_value}{\textquotesingle{}linear\textquotesingle{}}}, \emph{\DUrole{n}{category}\DUrole{p}{:} \DUrole{n}{Optional\DUrole{p}{{[}}str\DUrole{p}{{]}}} \DUrole{o}{=} \DUrole{default_value}{None}}}{}
Base class for all agents.
\index{assets (muse.agents.agent.AbstractAgent attribute)@\spxentry{assets}\spxextra{muse.agents.agent.AbstractAgent attribute}}

\begin{fulllineitems}
\phantomsection\label{\detokenize{api:muse.agents.agent.AbstractAgent.assets}}\pysigline{\sphinxbfcode{\sphinxupquote{assets}}}
Current stock of technologies.

\end{fulllineitems}

\index{category (muse.agents.agent.AbstractAgent attribute)@\spxentry{category}\spxextra{muse.agents.agent.AbstractAgent attribute}}

\begin{fulllineitems}
\phantomsection\label{\detokenize{api:muse.agents.agent.AbstractAgent.category}}\pysigline{\sphinxbfcode{\sphinxupquote{category}}}
Attribute to classify different sets of agents.

\end{fulllineitems}

\index{filter\_input() (muse.agents.agent.AbstractAgent method)@\spxentry{filter\_input()}\spxextra{muse.agents.agent.AbstractAgent method}}

\begin{fulllineitems}
\phantomsection\label{\detokenize{api:muse.agents.agent.AbstractAgent.filter_input}}\pysiglinewithargsret{\sphinxbfcode{\sphinxupquote{filter\_input}}}{\emph{\DUrole{n}{dataset}\DUrole{p}{:} \DUrole{n}{Union\DUrole{p}{{[}}xarray.core.dataset.Dataset\DUrole{p}{, }xarray.core.dataarray.DataArray\DUrole{p}{{]}}}}, \emph{\DUrole{n}{year}\DUrole{p}{:} \DUrole{n}{Optional\DUrole{p}{{[}}Union\DUrole{p}{{[}}Sequence\DUrole{p}{{[}}int\DUrole{p}{{]}}\DUrole{p}{, }int\DUrole{p}{{]}}\DUrole{p}{{]}}} \DUrole{o}{=} \DUrole{default_value}{None}}, \emph{\DUrole{o}{**}\DUrole{n}{kwargs}}}{{ $\rightarrow$ Union\DUrole{p}{{[}}xarray.core.dataset.Dataset\DUrole{p}{, }xarray.core.dataarray.DataArray\DUrole{p}{{]}}}}
Filter inputs for usage in agent.

For instance, filters down to agent’s region, etc.

\end{fulllineitems}

\index{interpolation (muse.agents.agent.AbstractAgent attribute)@\spxentry{interpolation}\spxextra{muse.agents.agent.AbstractAgent attribute}}

\begin{fulllineitems}
\phantomsection\label{\detokenize{api:muse.agents.agent.AbstractAgent.interpolation}}\pysigline{\sphinxbfcode{\sphinxupquote{interpolation}}}
Interpolation method.

\end{fulllineitems}

\index{name (muse.agents.agent.AbstractAgent attribute)@\spxentry{name}\spxextra{muse.agents.agent.AbstractAgent attribute}}

\begin{fulllineitems}
\phantomsection\label{\detokenize{api:muse.agents.agent.AbstractAgent.name}}\pysigline{\sphinxbfcode{\sphinxupquote{name}}}
Name associated with the agent

\end{fulllineitems}

\index{next() (muse.agents.agent.AbstractAgent method)@\spxentry{next()}\spxextra{muse.agents.agent.AbstractAgent method}}

\begin{fulllineitems}
\phantomsection\label{\detokenize{api:muse.agents.agent.AbstractAgent.next}}\pysiglinewithargsret{\sphinxbfcode{\sphinxupquote{abstract }}\sphinxbfcode{\sphinxupquote{next}}}{\emph{\DUrole{n}{technologies}\DUrole{p}{:} \DUrole{n}{xarray.core.dataset.Dataset}}, \emph{\DUrole{n}{market}\DUrole{p}{:} \DUrole{n}{xarray.core.dataset.Dataset}}, \emph{\DUrole{n}{demand}\DUrole{p}{:} \DUrole{n}{xarray.core.dataarray.DataArray}}, \emph{\DUrole{n}{time\_period}\DUrole{p}{:} \DUrole{n}{int} \DUrole{o}{=} \DUrole{default_value}{1}}}{}
Iterates agent one turn.

The goal is to figure out from market variables which technologies to invest in
and by how much.

\end{fulllineitems}

\index{region (muse.agents.agent.AbstractAgent attribute)@\spxentry{region}\spxextra{muse.agents.agent.AbstractAgent attribute}}

\begin{fulllineitems}
\phantomsection\label{\detokenize{api:muse.agents.agent.AbstractAgent.region}}\pysigline{\sphinxbfcode{\sphinxupquote{region}}}
Region the agent operates in

\end{fulllineitems}

\index{tolerance (muse.agents.agent.AbstractAgent attribute)@\spxentry{tolerance}\spxextra{muse.agents.agent.AbstractAgent attribute}}

\begin{fulllineitems}
\phantomsection\label{\detokenize{api:muse.agents.agent.AbstractAgent.tolerance}}\pysigline{\sphinxbfcode{\sphinxupquote{tolerance}}\sphinxbfcode{\sphinxupquote{ = 1e\sphinxhyphen{}12}}}
tolerance criteria for floating point comparisons.

\end{fulllineitems}

\index{uuid (muse.agents.agent.AbstractAgent attribute)@\spxentry{uuid}\spxextra{muse.agents.agent.AbstractAgent attribute}}

\begin{fulllineitems}
\phantomsection\label{\detokenize{api:muse.agents.agent.AbstractAgent.uuid}}\pysigline{\sphinxbfcode{\sphinxupquote{uuid}}}
A unique identifier for the agent.

\end{fulllineitems}


\end{fulllineitems}

\index{Agent (class in muse.agents.agent)@\spxentry{Agent}\spxextra{class in muse.agents.agent}}

\begin{fulllineitems}
\phantomsection\label{\detokenize{api:muse.agents.agent.Agent}}\pysiglinewithargsret{\sphinxbfcode{\sphinxupquote{class }}\sphinxcode{\sphinxupquote{muse.agents.agent.}}\sphinxbfcode{\sphinxupquote{Agent}}}{\emph{\DUrole{n}{name}\DUrole{p}{:} \DUrole{n}{str} \DUrole{o}{=} \DUrole{default_value}{\textquotesingle{}Agent\textquotesingle{}}}, \emph{\DUrole{n}{region}\DUrole{p}{:} \DUrole{n}{str} \DUrole{o}{=} \DUrole{default_value}{\textquotesingle{}USA\textquotesingle{}}}, \emph{\DUrole{n}{assets}\DUrole{p}{:} \DUrole{n}{Optional\DUrole{p}{{[}}xarray.core.dataset.Dataset\DUrole{p}{{]}}} \DUrole{o}{=} \DUrole{default_value}{None}}, \emph{\DUrole{n}{interpolation}\DUrole{p}{:} \DUrole{n}{str} \DUrole{o}{=} \DUrole{default_value}{\textquotesingle{}linear\textquotesingle{}}}, \emph{\DUrole{n}{search\_rules}\DUrole{p}{:} \DUrole{n}{Optional\DUrole{p}{{[}}Callable\DUrole{p}{{]}}} \DUrole{o}{=} \DUrole{default_value}{None}}, \emph{\DUrole{n}{objectives}\DUrole{p}{:} \DUrole{n}{Optional\DUrole{p}{{[}}Callable\DUrole{p}{{]}}} \DUrole{o}{=} \DUrole{default_value}{None}}, \emph{\DUrole{n}{decision}\DUrole{p}{:} \DUrole{n}{Optional\DUrole{p}{{[}}Callable\DUrole{p}{{]}}} \DUrole{o}{=} \DUrole{default_value}{None}}, \emph{\DUrole{n}{year}\DUrole{p}{:} \DUrole{n}{int} \DUrole{o}{=} \DUrole{default_value}{2010}}, \emph{\DUrole{n}{maturity\_threshhold}\DUrole{p}{:} \DUrole{n}{float} \DUrole{o}{=} \DUrole{default_value}{0}}, \emph{\DUrole{n}{forecast}\DUrole{p}{:} \DUrole{n}{int} \DUrole{o}{=} \DUrole{default_value}{5}}, \emph{\DUrole{n}{housekeeping}\DUrole{p}{:} \DUrole{n}{Optional\DUrole{p}{{[}}Callable\DUrole{p}{{]}}} \DUrole{o}{=} \DUrole{default_value}{None}}, \emph{\DUrole{n}{merge\_transform}\DUrole{p}{:} \DUrole{n}{Optional\DUrole{p}{{[}}Callable\DUrole{p}{{]}}} \DUrole{o}{=} \DUrole{default_value}{None}}, \emph{\DUrole{n}{demand\_threshhold}\DUrole{p}{:} \DUrole{n}{Optional\DUrole{p}{{[}}float\DUrole{p}{{]}}} \DUrole{o}{=} \DUrole{default_value}{None}}, \emph{\DUrole{n}{category}\DUrole{p}{:} \DUrole{n}{Optional\DUrole{p}{{[}}str\DUrole{p}{{]}}} \DUrole{o}{=} \DUrole{default_value}{None}}, \emph{\DUrole{o}{**}\DUrole{n}{kwargs}}}{}
Agent that is capable of computing a search\sphinxhyphen{}space and a cost metric.

This agent will not perform any investment itself.
\index{decision (muse.agents.agent.Agent attribute)@\spxentry{decision}\spxextra{muse.agents.agent.Agent attribute}}

\begin{fulllineitems}
\phantomsection\label{\detokenize{api:muse.agents.agent.Agent.decision}}\pysigline{\sphinxbfcode{\sphinxupquote{decision}}}
Creates single decision objective from one or more objectives.

\end{fulllineitems}

\index{demand\_threshhold (muse.agents.agent.Agent attribute)@\spxentry{demand\_threshhold}\spxextra{muse.agents.agent.Agent attribute}}

\begin{fulllineitems}
\phantomsection\label{\detokenize{api:muse.agents.agent.Agent.demand_threshhold}}\pysigline{\sphinxbfcode{\sphinxupquote{demand\_threshhold}}}
Threshhold below which the demand share is zero.

This criteria avoids fulfilling demand for very small values. If None,
then the criteria is not applied.

\end{fulllineitems}

\index{forecast (muse.agents.agent.Agent attribute)@\spxentry{forecast}\spxextra{muse.agents.agent.Agent attribute}}

\begin{fulllineitems}
\phantomsection\label{\detokenize{api:muse.agents.agent.Agent.forecast}}\pysigline{\sphinxbfcode{\sphinxupquote{forecast}}}
Number of years to look into the future for forecating purposed.

\end{fulllineitems}

\index{forecast\_year() (muse.agents.agent.Agent property)@\spxentry{forecast\_year()}\spxextra{muse.agents.agent.Agent property}}

\begin{fulllineitems}
\phantomsection\label{\detokenize{api:muse.agents.agent.Agent.forecast_year}}\pysigline{\sphinxbfcode{\sphinxupquote{property }}\sphinxbfcode{\sphinxupquote{forecast\_year}}}
Year to consider when forecasting.

\end{fulllineitems}

\index{housekeeping (muse.agents.agent.Agent attribute)@\spxentry{housekeeping}\spxextra{muse.agents.agent.Agent attribute}}

\begin{fulllineitems}
\phantomsection\label{\detokenize{api:muse.agents.agent.Agent.housekeeping}}\pysigline{\sphinxbfcode{\sphinxupquote{housekeeping}}}
Tranforms applied on the assets at the start of each iteration.

It could mean keeping the assets as are, or removing assets with no
capacity in the current year and beyond, etc…
It can be any function registered with
\sphinxcode{\sphinxupquote{register\_initial\_asset\_transform()}}.

\end{fulllineitems}

\index{maturity\_threshhold (muse.agents.agent.Agent attribute)@\spxentry{maturity\_threshhold}\spxextra{muse.agents.agent.Agent attribute}}

\begin{fulllineitems}
\phantomsection\label{\detokenize{api:muse.agents.agent.Agent.maturity_threshhold}}\pysigline{\sphinxbfcode{\sphinxupquote{maturity\_threshhold}}}
Market share threshhold.

Threshhold when and if filtering replacement technologies with respect
to market share.

\end{fulllineitems}

\index{merge\_transform (muse.agents.agent.Agent attribute)@\spxentry{merge\_transform}\spxextra{muse.agents.agent.Agent attribute}}

\begin{fulllineitems}
\phantomsection\label{\detokenize{api:muse.agents.agent.Agent.merge_transform}}\pysigline{\sphinxbfcode{\sphinxupquote{merge\_transform}}}
Tranforms applied on the old and new assets.

It could mean using only the new assets, or merging old and new, etc…
It can be any function registered with
\sphinxcode{\sphinxupquote{register\_final\_asset\_transform()}}.

\end{fulllineitems}

\index{next() (muse.agents.agent.Agent method)@\spxentry{next()}\spxextra{muse.agents.agent.Agent method}}

\begin{fulllineitems}
\phantomsection\label{\detokenize{api:muse.agents.agent.Agent.next}}\pysiglinewithargsret{\sphinxbfcode{\sphinxupquote{next}}}{\emph{\DUrole{n}{technologies}\DUrole{p}{:} \DUrole{n}{xarray.core.dataset.Dataset}}, \emph{\DUrole{n}{market}\DUrole{p}{:} \DUrole{n}{xarray.core.dataset.Dataset}}, \emph{\DUrole{n}{demand}\DUrole{p}{:} \DUrole{n}{xarray.core.dataarray.DataArray}}, \emph{\DUrole{n}{time\_period}\DUrole{p}{:} \DUrole{n}{int} \DUrole{o}{=} \DUrole{default_value}{1}}}{{ $\rightarrow$ Optional\DUrole{p}{{[}}xarray.core.dataset.Dataset\DUrole{p}{{]}}}}
Iterates agent one turn.

The goal is to figure out from market variables which technologies to
invest in and by how much.

This function will modify \sphinxtitleref{self.assets} and increment \sphinxtitleref{self.year}.
Other attributes are left unchanged. Arguments to the function are
never modified.

\end{fulllineitems}

\index{objectives (muse.agents.agent.Agent attribute)@\spxentry{objectives}\spxextra{muse.agents.agent.Agent attribute}}

\begin{fulllineitems}
\phantomsection\label{\detokenize{api:muse.agents.agent.Agent.objectives}}\pysigline{\sphinxbfcode{\sphinxupquote{objectives}}}
One or more objectives by which to decide next investments.

\end{fulllineitems}

\index{search\_rules (muse.agents.agent.Agent attribute)@\spxentry{search\_rules}\spxextra{muse.agents.agent.Agent attribute}}

\begin{fulllineitems}
\phantomsection\label{\detokenize{api:muse.agents.agent.Agent.search_rules}}\pysigline{\sphinxbfcode{\sphinxupquote{search\_rules}}\sphinxbfcode{\sphinxupquote{: Callable}}}
Search rule(s) determining potential replacement technologies.

This is a string referring to a filter, or a sequence of strings
referring to multiple filters, applied one after the other.  Any
function registered via \sphinxtitleref{muse.filters.register\_filter} can be
used to filter the search space.

\end{fulllineitems}

\index{year (muse.agents.agent.Agent attribute)@\spxentry{year}\spxextra{muse.agents.agent.Agent attribute}}

\begin{fulllineitems}
\phantomsection\label{\detokenize{api:muse.agents.agent.Agent.year}}\pysigline{\sphinxbfcode{\sphinxupquote{year}}}
Current year.

The year is incremented by one everytime next is called.

\end{fulllineitems}


\end{fulllineitems}

\index{InvestingAgent (class in muse.agents.agent)@\spxentry{InvestingAgent}\spxextra{class in muse.agents.agent}}

\begin{fulllineitems}
\phantomsection\label{\detokenize{api:muse.agents.agent.InvestingAgent}}\pysiglinewithargsret{\sphinxbfcode{\sphinxupquote{class }}\sphinxcode{\sphinxupquote{muse.agents.agent.}}\sphinxbfcode{\sphinxupquote{InvestingAgent}}}{\emph{\DUrole{o}{*}\DUrole{n}{args}}, \emph{\DUrole{n}{constraints}\DUrole{p}{:} \DUrole{n}{Optional\DUrole{p}{{[}}Callable\DUrole{p}{{]}}} \DUrole{o}{=} \DUrole{default_value}{None}}, \emph{\DUrole{n}{investment}\DUrole{p}{:} \DUrole{n}{Optional\DUrole{p}{{[}}Callable\DUrole{p}{{]}}} \DUrole{o}{=} \DUrole{default_value}{None}}, \emph{\DUrole{o}{**}\DUrole{n}{kwargs}}}{}
Agent that performs investment for itself.
\index{\_compute\_new\_assets() (muse.agents.agent.InvestingAgent method)@\spxentry{\_compute\_new\_assets()}\spxextra{muse.agents.agent.InvestingAgent method}}

\begin{fulllineitems}
\phantomsection\label{\detokenize{api:muse.agents.agent.InvestingAgent._compute_new_assets}}\pysiglinewithargsret{\sphinxbfcode{\sphinxupquote{\_compute\_new\_assets}}}{\emph{\DUrole{n}{demand}\DUrole{p}{:} \DUrole{n}{xarray.core.dataarray.DataArray}}, \emph{\DUrole{n}{search}\DUrole{p}{:} \DUrole{n}{xarray.core.dataset.Dataset}}, \emph{\DUrole{n}{technologies}\DUrole{p}{:} \DUrole{n}{xarray.core.dataset.Dataset}}, \emph{\DUrole{n}{market}\DUrole{p}{:} \DUrole{n}{xarray.core.dataset.Dataset}}, \emph{\DUrole{n}{time\_period}\DUrole{p}{:} \DUrole{n}{int}}, \emph{\DUrole{n}{current\_year}\DUrole{p}{:} \DUrole{n}{int}}}{{ $\rightarrow$ xarray.core.dataarray.DataArray}}
Computes investment and retirement profile.

\end{fulllineitems}

\index{add\_assets() (muse.agents.agent.InvestingAgent method)@\spxentry{add\_assets()}\spxextra{muse.agents.agent.InvestingAgent method}}

\begin{fulllineitems}
\phantomsection\label{\detokenize{api:muse.agents.agent.InvestingAgent.add_assets}}\pysiglinewithargsret{\sphinxbfcode{\sphinxupquote{add\_assets}}}{\emph{\DUrole{n}{newassets}\DUrole{p}{:} \DUrole{n}{xarray.core.dataset.Dataset}}}{}
Add new assets to the agent.

\end{fulllineitems}

\index{constraints (muse.agents.agent.InvestingAgent attribute)@\spxentry{constraints}\spxextra{muse.agents.agent.InvestingAgent attribute}}

\begin{fulllineitems}
\phantomsection\label{\detokenize{api:muse.agents.agent.InvestingAgent.constraints}}\pysigline{\sphinxbfcode{\sphinxupquote{constraints}}}
Creates a set of constraints limiting investment.

\end{fulllineitems}

\index{invest (muse.agents.agent.InvestingAgent attribute)@\spxentry{invest}\spxextra{muse.agents.agent.InvestingAgent attribute}}

\begin{fulllineitems}
\phantomsection\label{\detokenize{api:muse.agents.agent.InvestingAgent.invest}}\pysigline{\sphinxbfcode{\sphinxupquote{invest}}}
Method to use when fulfilling demand from rated set of techs.

\end{fulllineitems}

\index{next() (muse.agents.agent.InvestingAgent method)@\spxentry{next()}\spxextra{muse.agents.agent.InvestingAgent method}}

\begin{fulllineitems}
\phantomsection\label{\detokenize{api:muse.agents.agent.InvestingAgent.next}}\pysiglinewithargsret{\sphinxbfcode{\sphinxupquote{next}}}{\emph{\DUrole{n}{technologies}\DUrole{p}{:} \DUrole{n}{xarray.core.dataset.Dataset}}, \emph{\DUrole{n}{market}\DUrole{p}{:} \DUrole{n}{xarray.core.dataset.Dataset}}, \emph{\DUrole{n}{demand}\DUrole{p}{:} \DUrole{n}{xarray.core.dataarray.DataArray}}, \emph{\DUrole{n}{time\_period}\DUrole{p}{:} \DUrole{n}{int} \DUrole{o}{=} \DUrole{default_value}{1}}}{}
Iterates agent one turn.

The goal is to figure out from market variables which technologies to
invest in and by how much.

This function will modify \sphinxtitleref{self.assets} and increment \sphinxtitleref{self.year}.
Other attributes are left unchanged. Arguments to the function are
never modified.

\end{fulllineitems}


\end{fulllineitems}



\subsection{Objectives}
\label{\detokenize{api:module-muse.objectives}}\label{\detokenize{api:objectives}}\index{module@\spxentry{module}!muse.objectives@\spxentry{muse.objectives}}\index{muse.objectives@\spxentry{muse.objectives}!module@\spxentry{module}}
Valuation functions for replacement technologies.

Objectives are used to compare replacement technologies. They should correspond to
a single well defined economic concept. Multiple objectives can later be combined
via decision functions.

Objectives should be registered via the
\sphinxcode{\sphinxupquote{@register\_objective}} decorator. This makes it possible to
refer to them by name in agent input files, and nominally to set extra input parameters.

The \sphinxcode{\sphinxupquote{factory()}} function creates a function that calls all objectives defined in
its input argument and returns a dataset with each objective as a separate data array.

Objectives are not expected to modify their arguments. Furthermore they should
conform the following signatures:

\begin{sphinxVerbatim}[commandchars=\\\{\}]
\PYG{n+nd}{@register\PYGZus{}objective}
\PYG{k}{def} \PYG{n+nf}{comfort}\PYG{p}{(}
    \PYG{n}{agent}\PYG{p}{:} \PYG{n}{Agent}\PYG{p}{,}
    \PYG{n}{demand}\PYG{p}{:} \PYG{n}{xr}\PYG{o}{.}\PYG{n}{DataArray}\PYG{p}{,}
    \PYG{n}{search\PYGZus{}space}\PYG{p}{:} \PYG{n}{xr}\PYG{o}{.}\PYG{n}{DataArray}\PYG{p}{,}
    \PYG{n}{technologies}\PYG{p}{:} \PYG{n}{xr}\PYG{o}{.}\PYG{n}{Dataset}\PYG{p}{,}
    \PYG{n}{market}\PYG{p}{:} \PYG{n}{xr}\PYG{o}{.}\PYG{n}{Dataset}\PYG{p}{,}
    \PYG{o}{*}\PYG{o}{*}\PYG{n}{kwargs}
\PYG{p}{)} \PYG{o}{\PYGZhy{}}\PYG{o}{\PYGZgt{}} \PYG{n}{xr}\PYG{o}{.}\PYG{n}{DataArray}\PYG{p}{:}
    \PYG{k}{pass}
\end{sphinxVerbatim}
\begin{quote}\begin{description}
\item[{param agent}] \leavevmode
the agent relevant to the search space. The filters may need to query
the agent for parameters, e.g. the current year, the interpolation
method, the tolerance, etc.

\item[{param demand}] \leavevmode
Demand to fulfill.

\item[{param search\_space}] \leavevmode
A boolean matrix represented as a \sphinxcode{\sphinxupquote{xr.DataArray}}, listing
replacement technologies for each asset.

\item[{param technologies}] \leavevmode
A data set characterising the technologies from which the
agent can draw assets.

\item[{param market}] \leavevmode
Market variables, such as prices or current capacity and retirement
profile.

\item[{param kwargs}] \leavevmode
Extra input parameters. These parameters are expected to be set from the
input file.

\begin{sphinxadmonition}{warning}{Warning:}
The standard {\hyperref[\detokenize{inputs/agents:inputs-agents}]{\sphinxcrossref{\DUrole{std,std-ref}{agent csv file}}}} does not allow to set
these parameters.
\end{sphinxadmonition}

\item[{returns}] \leavevmode
A dataArray with at least one dimension corresponding to \sphinxcode{\sphinxupquote{replacement}}.  Only the
technologies in \sphinxcode{\sphinxupquote{search\_space.replacement}} should be present.  Furthermore, if an
\sphinxcode{\sphinxupquote{asset}} dimension is present, then it should correspond to \sphinxcode{\sphinxupquote{search\_space.asset}}.
Other dimensions can be present, as long as the subsequent decision function nows
how to reduce them.

\end{description}\end{quote}
\index{capacity\_to\_service\_demand() (in module muse.objectives)@\spxentry{capacity\_to\_service\_demand()}\spxextra{in module muse.objectives}}

\begin{fulllineitems}
\phantomsection\label{\detokenize{api:muse.objectives.capacity_to_service_demand}}\pysiglinewithargsret{\sphinxcode{\sphinxupquote{muse.objectives.}}\sphinxbfcode{\sphinxupquote{capacity\_to\_service\_demand}}}{\emph{\DUrole{n}{agent}\DUrole{p}{:} \DUrole{n}{muse.agents.agent.Agent}}, \emph{\DUrole{n}{demand}\DUrole{p}{:} \DUrole{n}{xarray.core.dataarray.DataArray}}, \emph{\DUrole{n}{search\_space}\DUrole{p}{:} \DUrole{n}{xarray.core.dataarray.DataArray}}, \emph{\DUrole{n}{technologies}\DUrole{p}{:} \DUrole{n}{xarray.core.dataset.Dataset}}, \emph{\DUrole{n}{market}\DUrole{p}{:} \DUrole{n}{xarray.core.dataset.Dataset}}, \emph{\DUrole{o}{*}\DUrole{n}{args}}, \emph{\DUrole{o}{**}\DUrole{n}{kwargs}}}{{ $\rightarrow$ xarray.core.dataarray.DataArray}}
Minimum capacity required to fulfill the demand.

\end{fulllineitems}

\index{capital\_costs() (in module muse.objectives)@\spxentry{capital\_costs()}\spxextra{in module muse.objectives}}

\begin{fulllineitems}
\phantomsection\label{\detokenize{api:muse.objectives.capital_costs}}\pysiglinewithargsret{\sphinxcode{\sphinxupquote{muse.objectives.}}\sphinxbfcode{\sphinxupquote{capital\_costs}}}{\emph{\DUrole{n}{agent}\DUrole{p}{:} \DUrole{n}{muse.agents.agent.Agent}}, \emph{\DUrole{n}{demand}\DUrole{p}{:} \DUrole{n}{xarray.core.dataarray.DataArray}}, \emph{\DUrole{n}{search\_space}\DUrole{p}{:} \DUrole{n}{xarray.core.dataarray.DataArray}}, \emph{\DUrole{n}{technologies}\DUrole{p}{:} \DUrole{n}{xarray.core.dataset.Dataset}}, \emph{\DUrole{o}{*}\DUrole{n}{args}}, \emph{\DUrole{o}{**}\DUrole{n}{kwargs}}}{{ $\rightarrow$ xarray.core.dataarray.DataArray}}
Capital costs for input technologies.

The capital costs are computed as \(a * b^\alpha\), where \(a\) is
“cap\_par” from the {\hyperref[\detokenize{inputs/technodata:inputs-technodata}]{\sphinxcrossref{\DUrole{std,std-ref}{Techno\sphinxhyphen{}data}}}}, \(b\) is the “scaling\_size”, and
\(\alpha\) is “cap\_exp”. In other words, capital costs are constant across the
simulation for each technology.

\end{fulllineitems}

\index{comfort() (in module muse.objectives)@\spxentry{comfort()}\spxextra{in module muse.objectives}}

\begin{fulllineitems}
\phantomsection\label{\detokenize{api:muse.objectives.comfort}}\pysiglinewithargsret{\sphinxcode{\sphinxupquote{muse.objectives.}}\sphinxbfcode{\sphinxupquote{comfort}}}{\emph{\DUrole{n}{agent}\DUrole{p}{:} \DUrole{n}{muse.agents.agent.Agent}}, \emph{\DUrole{n}{demand}\DUrole{p}{:} \DUrole{n}{xarray.core.dataarray.DataArray}}, \emph{\DUrole{n}{search\_space}\DUrole{p}{:} \DUrole{n}{xarray.core.dataarray.DataArray}}, \emph{\DUrole{n}{technologies}\DUrole{p}{:} \DUrole{n}{xarray.core.dataset.Dataset}}, \emph{\DUrole{o}{*}\DUrole{n}{args}}, \emph{\DUrole{o}{**}\DUrole{n}{kwargs}}}{{ $\rightarrow$ xarray.core.dataarray.DataArray}}
Comfort value provided by technologies.

\end{fulllineitems}

\index{efficiency() (in module muse.objectives)@\spxentry{efficiency()}\spxextra{in module muse.objectives}}

\begin{fulllineitems}
\phantomsection\label{\detokenize{api:muse.objectives.efficiency}}\pysiglinewithargsret{\sphinxcode{\sphinxupquote{muse.objectives.}}\sphinxbfcode{\sphinxupquote{efficiency}}}{\emph{\DUrole{n}{agent}\DUrole{p}{:} \DUrole{n}{muse.agents.agent.Agent}}, \emph{\DUrole{n}{demand}\DUrole{p}{:} \DUrole{n}{xarray.core.dataarray.DataArray}}, \emph{\DUrole{n}{search\_space}\DUrole{p}{:} \DUrole{n}{xarray.core.dataarray.DataArray}}, \emph{\DUrole{n}{technologies}\DUrole{p}{:} \DUrole{n}{xarray.core.dataset.Dataset}}, \emph{\DUrole{o}{*}\DUrole{n}{args}}, \emph{\DUrole{o}{**}\DUrole{n}{kwargs}}}{{ $\rightarrow$ xarray.core.dataarray.DataArray}}
Efficiency of the technologies.

\end{fulllineitems}

\index{emission\_cost() (in module muse.objectives)@\spxentry{emission\_cost()}\spxextra{in module muse.objectives}}

\begin{fulllineitems}
\phantomsection\label{\detokenize{api:muse.objectives.emission_cost}}\pysiglinewithargsret{\sphinxcode{\sphinxupquote{muse.objectives.}}\sphinxbfcode{\sphinxupquote{emission\_cost}}}{\emph{\DUrole{n}{agent}\DUrole{p}{:} \DUrole{n}{muse.agents.agent.Agent}}, \emph{\DUrole{n}{demand}\DUrole{p}{:} \DUrole{n}{xarray.core.dataarray.DataArray}}, \emph{\DUrole{n}{search\_space}\DUrole{p}{:} \DUrole{n}{xarray.core.dataarray.DataArray}}, \emph{\DUrole{n}{technologies}\DUrole{p}{:} \DUrole{n}{xarray.core.dataset.Dataset}}, \emph{\DUrole{n}{market}\DUrole{p}{:} \DUrole{n}{xarray.core.dataset.Dataset}}, \emph{\DUrole{o}{*}\DUrole{n}{args}}, \emph{\DUrole{o}{**}\DUrole{n}{kwargs}}}{{ $\rightarrow$ xarray.core.dataarray.DataArray}}
Emission cost for each technology when fultfilling whole demand.

Given the demand share \(D\), the emissions per amount produced \(E\), and
the prices per emittant \(P\), then emissions costs \(C\) are computed
as:
\begin{equation*}
\begin{split}C = \sum_s \left(\sum_cD\right)\left(\sum_cEP\right),\end{split}
\end{equation*}
with \(s\) the timeslices and \(c\) the commodity.

\end{fulllineitems}

\index{equivalent\_annual\_cost() (in module muse.objectives)@\spxentry{equivalent\_annual\_cost()}\spxextra{in module muse.objectives}}

\begin{fulllineitems}
\phantomsection\label{\detokenize{api:muse.objectives.equivalent_annual_cost}}\pysiglinewithargsret{\sphinxcode{\sphinxupquote{muse.objectives.}}\sphinxbfcode{\sphinxupquote{equivalent\_annual\_cost}}}{\emph{\DUrole{n}{agent}\DUrole{p}{:} \DUrole{n}{muse.agents.agent.Agent}}, \emph{\DUrole{n}{demand}\DUrole{p}{:} \DUrole{n}{xarray.core.dataarray.DataArray}}, \emph{\DUrole{n}{search\_space}\DUrole{p}{:} \DUrole{n}{xarray.core.dataarray.DataArray}}, \emph{\DUrole{n}{technologies}\DUrole{p}{:} \DUrole{n}{xarray.core.dataset.Dataset}}, \emph{\DUrole{n}{market}\DUrole{p}{:} \DUrole{n}{xarray.core.dataset.Dataset}}, \emph{\DUrole{o}{*}\DUrole{n}{args}}, \emph{\DUrole{o}{**}\DUrole{n}{kwargs}}}{}
Equivalent annual costs (or annualized cost) of a technology.

This is the cost that, if it were to occur equally in every year of the
project lifetime, would give the same net present cost as the actual cash
flow sequence associated with that component. The cost is computed using the
\sphinxhref{https://www.homerenergy.com/products/pro/docs/3.11/annualized\_cost.html}{annualized cost} expression given by HOMER Energy.
\begin{quote}\begin{description}
\item[{Parameters}] \leavevmode\begin{itemize}
\item {} 
\sphinxstyleliteralstrong{\sphinxupquote{agent}} \textendash{} The agent of interest

\item {} 
\sphinxstyleliteralstrong{\sphinxupquote{search\_space}} \textendash{} The search space space for replacement technologies

\item {} 
\sphinxstyleliteralstrong{\sphinxupquote{technologies}} \textendash{} All the technologies

\item {} 
\sphinxstyleliteralstrong{\sphinxupquote{market}} \textendash{} The market parameters

\end{itemize}

\item[{Returns}] \leavevmode
xr.DataArray with the EAC calculated for the relevant technologies

\end{description}\end{quote}

\end{fulllineitems}

\index{factory() (in module muse.objectives)@\spxentry{factory()}\spxextra{in module muse.objectives}}

\begin{fulllineitems}
\phantomsection\label{\detokenize{api:muse.objectives.factory}}\pysiglinewithargsret{\sphinxcode{\sphinxupquote{muse.objectives.}}\sphinxbfcode{\sphinxupquote{factory}}}{\emph{\DUrole{n}{settings}\DUrole{p}{:} \DUrole{n}{Union\DUrole{p}{{[}}str\DUrole{p}{, }Mapping\DUrole{p}{, }Sequence\DUrole{p}{{[}}Union\DUrole{p}{{[}}str\DUrole{p}{, }Mapping\DUrole{p}{{]}}\DUrole{p}{{]}}\DUrole{p}{{]}}} \DUrole{o}{=} \DUrole{default_value}{\textquotesingle{}LCOE\textquotesingle{}}}}{{ $\rightarrow$ Callable}}
Creates a function computing multiple objectives.

The input can be a single objective defined by its name alone. Or it can be a single
objective defined by a dictionary which must include at least a “name” item, as well
as any extra parameters to pass to the objective. Or it can be a sequence of
objectives defined by name or by dictionary.

\end{fulllineitems}

\index{fixed\_costs() (in module muse.objectives)@\spxentry{fixed\_costs()}\spxextra{in module muse.objectives}}

\begin{fulllineitems}
\phantomsection\label{\detokenize{api:muse.objectives.fixed_costs}}\pysiglinewithargsret{\sphinxcode{\sphinxupquote{muse.objectives.}}\sphinxbfcode{\sphinxupquote{fixed\_costs}}}{\emph{\DUrole{n}{agent}\DUrole{p}{:} \DUrole{n}{muse.agents.agent.Agent}}, \emph{\DUrole{n}{demand}\DUrole{p}{:} \DUrole{n}{xarray.core.dataarray.DataArray}}, \emph{\DUrole{n}{search\_space}\DUrole{p}{:} \DUrole{n}{xarray.core.dataarray.DataArray}}, \emph{\DUrole{n}{technologies}\DUrole{p}{:} \DUrole{n}{xarray.core.dataset.Dataset}}, \emph{\DUrole{n}{market}\DUrole{p}{:} \DUrole{n}{xarray.core.dataset.Dataset}}, \emph{\DUrole{o}{*}\DUrole{n}{args}}, \emph{\DUrole{o}{**}\DUrole{n}{kwargs}}}{{ $\rightarrow$ xarray.core.dataarray.DataArray}}
Fixed costs associated with a technology.

Given a factor \(\alpha\) and an  exponent \(\beta\), the fixed costs
\(F\) are computed from the \sphinxcode{\sphinxupquote{capacity fulfilling the current demand}} \(C\) as:
\begin{equation*}
\begin{split}F = \alpha * C^\beta\end{split}
\end{equation*}
\(\alpha\) and \(\beta\) are “fix\_par” and “fix\_exp” in
{\hyperref[\detokenize{inputs/technodata:inputs-technodata}]{\sphinxcrossref{\DUrole{std,std-ref}{Techno\sphinxhyphen{}data}}}}, respectively.

\end{fulllineitems}

\index{fuel\_consumption\_cost() (in module muse.objectives)@\spxentry{fuel\_consumption\_cost()}\spxextra{in module muse.objectives}}

\begin{fulllineitems}
\phantomsection\label{\detokenize{api:muse.objectives.fuel_consumption_cost}}\pysiglinewithargsret{\sphinxcode{\sphinxupquote{muse.objectives.}}\sphinxbfcode{\sphinxupquote{fuel\_consumption\_cost}}}{\emph{\DUrole{n}{agent}\DUrole{p}{:} \DUrole{n}{muse.agents.agent.Agent}}, \emph{\DUrole{n}{demand}\DUrole{p}{:} \DUrole{n}{xarray.core.dataarray.DataArray}}, \emph{\DUrole{n}{search\_space}\DUrole{p}{:} \DUrole{n}{xarray.core.dataarray.DataArray}}, \emph{\DUrole{n}{technologies}\DUrole{p}{:} \DUrole{n}{xarray.core.dataset.Dataset}}, \emph{\DUrole{n}{market}\DUrole{p}{:} \DUrole{n}{xarray.core.dataset.Dataset}}, \emph{\DUrole{o}{*}\DUrole{n}{args}}, \emph{\DUrole{o}{**}\DUrole{n}{kwargs}}}{}
Cost of fuels when fulfilling whole demand.

\end{fulllineitems}

\index{lifetime\_levelized\_cost\_of\_energy() (in module muse.objectives)@\spxentry{lifetime\_levelized\_cost\_of\_energy()}\spxextra{in module muse.objectives}}

\begin{fulllineitems}
\phantomsection\label{\detokenize{api:muse.objectives.lifetime_levelized_cost_of_energy}}\pysiglinewithargsret{\sphinxcode{\sphinxupquote{muse.objectives.}}\sphinxbfcode{\sphinxupquote{lifetime\_levelized\_cost\_of\_energy}}}{\emph{\DUrole{n}{agent}\DUrole{p}{:} \DUrole{n}{muse.agents.agent.Agent}}, \emph{\DUrole{n}{demand}\DUrole{p}{:} \DUrole{n}{xarray.core.dataarray.DataArray}}, \emph{\DUrole{n}{search\_space}\DUrole{p}{:} \DUrole{n}{xarray.core.dataarray.DataArray}}, \emph{\DUrole{n}{technologies}\DUrole{p}{:} \DUrole{n}{xarray.core.dataset.Dataset}}, \emph{\DUrole{n}{market}\DUrole{p}{:} \DUrole{n}{xarray.core.dataset.Dataset}}, \emph{\DUrole{o}{*}\DUrole{n}{args}}, \emph{\DUrole{o}{**}\DUrole{n}{kwargs}}}{}
Levelized cost of energy (LCOE) of technologies over their lifetime.

It follows the \sphinxtitleref{simpified LCOE} given by NREL.
\begin{quote}\begin{description}
\item[{Parameters}] \leavevmode\begin{itemize}
\item {} 
\sphinxstyleliteralstrong{\sphinxupquote{agent}} \textendash{} The agent of interest

\item {} 
\sphinxstyleliteralstrong{\sphinxupquote{search\_space}} \textendash{} The search space space for replacement technologies

\item {} 
\sphinxstyleliteralstrong{\sphinxupquote{technologies}} \textendash{} All the technologies

\item {} 
\sphinxstyleliteralstrong{\sphinxupquote{market}} \textendash{} The market parameters

\end{itemize}

\item[{Returns}] \leavevmode
xr.DataArray with the LCOE calculated for the relevant technologies

\end{description}\end{quote}

\end{fulllineitems}

\index{net\_present\_value() (in module muse.objectives)@\spxentry{net\_present\_value()}\spxextra{in module muse.objectives}}

\begin{fulllineitems}
\phantomsection\label{\detokenize{api:muse.objectives.net_present_value}}\pysiglinewithargsret{\sphinxcode{\sphinxupquote{muse.objectives.}}\sphinxbfcode{\sphinxupquote{net\_present\_value}}}{\emph{\DUrole{n}{agent}\DUrole{p}{:} \DUrole{n}{muse.agents.agent.Agent}}, \emph{\DUrole{n}{demand}\DUrole{p}{:} \DUrole{n}{xarray.core.dataarray.DataArray}}, \emph{\DUrole{n}{search\_space}\DUrole{p}{:} \DUrole{n}{xarray.core.dataarray.DataArray}}, \emph{\DUrole{n}{technologies}\DUrole{p}{:} \DUrole{n}{xarray.core.dataset.Dataset}}, \emph{\DUrole{n}{market}\DUrole{p}{:} \DUrole{n}{xarray.core.dataset.Dataset}}, \emph{\DUrole{o}{*}\DUrole{n}{args}}, \emph{\DUrole{o}{**}\DUrole{n}{kwargs}}}{}
Net present value (NPV) of the relevant technologies.

The net present value of a Component is the present value  of all the revenues that
a Component earns over its lifetime minus all the costs of installing and operating
it. Follows the definition of the \sphinxhref{https://www.homerenergy.com/products/pro/docs/3.11/net\_present\_cost.html}{net present cost} given by HOMER Energy.
\begin{itemize}
\item {} 
energy commodities INPUTS are related to fuel costs

\item {} 
environmental commodities OUTPUTS are related to environmental costs

\item {} 
material and service commodities INPUTS are related to consumable costs

\item {} 
fixed and variable costs are given as technodata inputs and depend on the
installed capacity and production (non\sphinxhyphen{}environmental), respectively

\item {} 
capacity costs are given as technodata inputs and depend on the installed capacity

\end{itemize}

\begin{sphinxadmonition}{note}{Note:}
Here, the installation year is always agent.year, since objectives compute the
NPV for technologies to be installed in the current year. A more general NPV
computation (which would then live in quantities.py) would have to refer to
installation year of the technology.
\end{sphinxadmonition}
\begin{quote}\begin{description}
\item[{Parameters}] \leavevmode\begin{itemize}
\item {} 
\sphinxstyleliteralstrong{\sphinxupquote{agent}} \textendash{} The agent of interest

\item {} 
\sphinxstyleliteralstrong{\sphinxupquote{search\_space}} \textendash{} The search space space for replacement technologies

\item {} 
\sphinxstyleliteralstrong{\sphinxupquote{technologies}} \textendash{} All the technologies

\item {} 
\sphinxstyleliteralstrong{\sphinxupquote{market}} \textendash{} The market parameters

\end{itemize}

\item[{Returns}] \leavevmode
xr.DataArray with the NPV calculated for the relevant technologies

\end{description}\end{quote}

\end{fulllineitems}

\index{register\_objective() (in module muse.objectives)@\spxentry{register\_objective()}\spxextra{in module muse.objectives}}

\begin{fulllineitems}
\phantomsection\label{\detokenize{api:muse.objectives.register_objective}}\pysiglinewithargsret{\sphinxcode{\sphinxupquote{muse.objectives.}}\sphinxbfcode{\sphinxupquote{register\_objective}}}{\emph{\DUrole{n}{function}\DUrole{p}{:} \DUrole{n}{Callable\DUrole{p}{{[}}\DUrole{p}{{[}}muse.agents.agent.Agent\DUrole{p}{, }xarray.core.dataarray.DataArray\DUrole{p}{, }xarray.core.dataarray.DataArray\DUrole{p}{, }xarray.core.dataset.Dataset\DUrole{p}{, }xarray.core.dataset.Dataset\DUrole{p}{, }Any\DUrole{p}{{]}}\DUrole{p}{, }xarray.core.dataarray.DataArray\DUrole{p}{{]}}}}}{}
Decorator to register a function as a objective.

Registers a function as a objective so that it can be applied easily
when sorting technologies one against the other.

The input name is expected to be in lower\_snake\_case, since it ought to be a
python function. CamelCase, lowerCamelCase, and kebab\sphinxhyphen{}case names are also
registered.

\end{fulllineitems}



\subsection{Search Rules}
\label{\detokenize{api:module-muse.filters}}\label{\detokenize{api:search-rules}}\index{module@\spxentry{module}!muse.filters@\spxentry{muse.filters}}\index{muse.filters@\spxentry{muse.filters}!module@\spxentry{module}}
Various search\sphinxhyphen{}space filters.

Search\sphinxhyphen{}space filters return a modified matrix of booleans, with dimension
\sphinxtitleref{(asset, replacement)}, where \sphinxtitleref{asset} refer to technologies currently managed by
the agent, and \sphinxtitleref{replacement} to all technologies the agent could consider, prior
to filtering.

Filters should be registered using the decorator \sphinxcode{\sphinxupquote{register\_filter()}}. The
registration makes it possible to call then from the agent by specifying the
\sphinxtitleref{search\_rule} attribute. The \sphinxtitleref{search\_rule} attribute is string or list of
strings specifying the filters to apply one after the other when considering the
search space.

Filters are not expected to modify any of their arguments. They should all
follow the same signature:

\begin{sphinxVerbatim}[commandchars=\\\{\}]
\PYG{n+nd}{@register\PYGZus{}filter}
\PYG{k}{def} \PYG{n+nf}{search\PYGZus{}space\PYGZus{}filter}\PYG{p}{(}
    \PYG{n}{agent}\PYG{p}{:} \PYG{n}{Agent}\PYG{p}{,}
    \PYG{n}{search\PYGZus{}space}\PYG{p}{:} \PYG{n}{xr}\PYG{o}{.}\PYG{n}{DataArray}\PYG{p}{,}
    \PYG{n}{technologies}\PYG{p}{:} \PYG{n}{xr}\PYG{o}{.}\PYG{n}{Dataset}\PYG{p}{,}
    \PYG{n}{market}\PYG{p}{:} \PYG{n}{xr}\PYG{o}{.}\PYG{n}{Dataset}
\PYG{p}{)} \PYG{o}{\PYGZhy{}}\PYG{o}{\PYGZgt{}} \PYG{n}{xr}\PYG{o}{.}\PYG{n}{DataArray}\PYG{p}{:}
    \PYG{k}{pass}
\end{sphinxVerbatim}
\begin{quote}\begin{description}
\item[{param agent}] \leavevmode
the agent relevant to the search space. The filters may need to query
the agent for parameters, e.g. the current year, the interpolation
method, the tolerance, etc.

\item[{param search\_space}] \leavevmode
the current search space.

\item[{param technologies}] \leavevmode
A data set characterising the technologies from which the
agent can draw assets.

\item[{param market}] \leavevmode
Market variables, such as prices or current capacity and retirement
profile.

\item[{returns}] \leavevmode
A new search space with the same data\sphinxhyphen{}type as the input search\sphinxhyphen{}space, but
with potentially different values.

\end{description}\end{quote}

In practice, an initial search space is created by calling a function with the signature
given below, and registered with \sphinxcode{\sphinxupquote{register\_initializer()}}. The initializer
function returns a search space which is passed on to a chain of filters, as done in the
\sphinxcode{\sphinxupquote{factory()}} function.

Functions creating initial search spaces should have the following signature:

\begin{sphinxVerbatim}[commandchars=\\\{\}]
\PYG{n+nd}{@register\PYGZus{}initializer}
\PYG{k}{def} \PYG{n+nf}{search\PYGZus{}space\PYGZus{}initializer}\PYG{p}{(}
    \PYG{n}{agent}\PYG{p}{:} \PYG{n}{Agent}\PYG{p}{,}
    \PYG{n}{demand}\PYG{p}{:} \PYG{n}{xr}\PYG{o}{.}\PYG{n}{DataArray}\PYG{p}{,}
    \PYG{n}{technologies}\PYG{p}{:} \PYG{n}{xr}\PYG{o}{.}\PYG{n}{Dataset}\PYG{p}{,}
    \PYG{n}{market}\PYG{p}{:} \PYG{n}{xr}\PYG{o}{.}\PYG{n}{Dataset}
\PYG{p}{)} \PYG{o}{\PYGZhy{}}\PYG{o}{\PYGZgt{}} \PYG{n}{xr}\PYG{o}{.}\PYG{n}{DataArray}\PYG{p}{:}
    \PYG{k}{pass}
\end{sphinxVerbatim}
\begin{quote}\begin{description}
\item[{param agent}] \leavevmode
the agent relevant to the search space. The filters may need to query
the agent for parameters, e.g. the current year, the interpolation
method, the tolerance, etc.

\item[{param demand}] \leavevmode
share of the demand per existing reference technology (e.g.
assets).

\item[{param technologies}] \leavevmode
A data set characterising the technologies from which the
agent can draw assets.

\item[{param market}] \leavevmode
Market variables, such as prices or current capacity and retirement
profile.

\item[{returns}] \leavevmode
An initial search space

\end{description}\end{quote}
\index{compress() (in module muse.filters)@\spxentry{compress()}\spxextra{in module muse.filters}}

\begin{fulllineitems}
\phantomsection\label{\detokenize{api:muse.filters.compress}}\pysiglinewithargsret{\sphinxcode{\sphinxupquote{muse.filters.}}\sphinxbfcode{\sphinxupquote{compress}}}{\emph{\DUrole{n}{agent}\DUrole{p}{:} \DUrole{n}{muse.agents.agent.Agent}}, \emph{\DUrole{n}{search\_space}\DUrole{p}{:} \DUrole{n}{xarray.core.dataarray.DataArray}}, \emph{\DUrole{n}{technologies}\DUrole{p}{:} \DUrole{n}{xarray.core.dataset.Dataset}}, \emph{\DUrole{n}{market}\DUrole{p}{:} \DUrole{n}{xarray.core.dataset.Dataset}}, \emph{\DUrole{o}{**}\DUrole{n}{kwargs}}}{{ $\rightarrow$ xarray.core.dataarray.DataArray}}
Compress search space to include only potential technologies.

This operation reduces the \sphinxstyleemphasis{size} of the search space along the
\sphinxtitleref{replacement} dimension, such that are left only technologies that
will be considered as replacement for at least by one asset. Unlike
most filters, it does not change the data, but rather changes how
the data is represented. In other words, this is mostly an
\sphinxstyleemphasis{optimization} for later steps, to avoid unnecessary computations.

\end{fulllineitems}

\index{currently\_existing\_tech() (in module muse.filters)@\spxentry{currently\_existing\_tech()}\spxextra{in module muse.filters}}

\begin{fulllineitems}
\phantomsection\label{\detokenize{api:muse.filters.currently_existing_tech}}\pysiglinewithargsret{\sphinxcode{\sphinxupquote{muse.filters.}}\sphinxbfcode{\sphinxupquote{currently\_existing\_tech}}}{\emph{\DUrole{n}{agent}\DUrole{p}{:} \DUrole{n}{muse.agents.agent.Agent}}, \emph{\DUrole{n}{search\_space}\DUrole{p}{:} \DUrole{n}{xarray.core.dataarray.DataArray}}, \emph{\DUrole{n}{technologies}\DUrole{p}{:} \DUrole{n}{xarray.core.dataset.Dataset}}, \emph{\DUrole{n}{market}\DUrole{p}{:} \DUrole{n}{xarray.core.dataset.Dataset}}}{{ $\rightarrow$ xarray.core.dataarray.DataArray}}
Only consider technologies that currently exist in the market.

This filter only allows technologies that exists in the market and have non\sphinxhyphen{} zero
capacity in the current year. See \sphinxtitleref{currently\_referenced\_tech} for a similar filter
that does not check the capacity.

\end{fulllineitems}

\index{currently\_referenced\_tech() (in module muse.filters)@\spxentry{currently\_referenced\_tech()}\spxextra{in module muse.filters}}

\begin{fulllineitems}
\phantomsection\label{\detokenize{api:muse.filters.currently_referenced_tech}}\pysiglinewithargsret{\sphinxcode{\sphinxupquote{muse.filters.}}\sphinxbfcode{\sphinxupquote{currently\_referenced\_tech}}}{\emph{\DUrole{n}{agent}\DUrole{p}{:} \DUrole{n}{muse.agents.agent.Agent}}, \emph{\DUrole{n}{search\_space}\DUrole{p}{:} \DUrole{n}{xarray.core.dataarray.DataArray}}, \emph{\DUrole{n}{technologies}\DUrole{p}{:} \DUrole{n}{xarray.core.dataset.Dataset}}, \emph{\DUrole{n}{market}\DUrole{p}{:} \DUrole{n}{xarray.core.dataset.Dataset}}}{{ $\rightarrow$ xarray.core.dataarray.DataArray}}
Only consider technologies that are currently referenced in the market.

This filter will allow any technology that exists in the market, even if it
currently sits at zero capacity (unlike \sphinxtitleref{currently\_existing\_tech} which requires
non\sphinxhyphen{}zero capacity in the current year).

\end{fulllineitems}

\index{factory() (in module muse.filters)@\spxentry{factory()}\spxextra{in module muse.filters}}

\begin{fulllineitems}
\phantomsection\label{\detokenize{api:muse.filters.factory}}\pysiglinewithargsret{\sphinxcode{\sphinxupquote{muse.filters.}}\sphinxbfcode{\sphinxupquote{factory}}}{\emph{\DUrole{n}{settings}\DUrole{p}{:} \DUrole{n}{Optional\DUrole{p}{{[}}Union\DUrole{p}{{[}}str\DUrole{p}{, }Mapping\DUrole{p}{, }Sequence\DUrole{p}{{[}}Union\DUrole{p}{{[}}str\DUrole{p}{, }Mapping\DUrole{p}{{]}}\DUrole{p}{{]}}\DUrole{p}{{]}}\DUrole{p}{{]}}} \DUrole{o}{=} \DUrole{default_value}{None}}, \emph{\DUrole{n}{separator}\DUrole{p}{:} \DUrole{n}{str} \DUrole{o}{=} \DUrole{default_value}{\textquotesingle{}\sphinxhyphen{}\textgreater{}\textquotesingle{}}}}{}
Creates filters from input TOML data.

The input data is standardized to a list of dictionaries where each dictionary
contains at least one member, “name”.

The first dictionary specifies the initial function which creates the search space
from the demand share, the market, and the dataset describing technologies in the
sectors.

The next entries are applied in turn and transform the search space in some way.
In other words the process is more or less:

\begin{sphinxVerbatim}[commandchars=\\\{\}]
\PYG{n}{search\PYGZus{}space} \PYG{o}{=} \PYG{n}{initial\PYGZus{}filter}\PYG{p}{(}
    \PYG{n}{agent}\PYG{p}{,} \PYG{n}{demand}\PYG{p}{,} \PYG{n}{technologies}\PYG{o}{=}\PYG{n}{technologies}\PYG{p}{,} \PYG{n}{market}\PYG{o}{=}\PYG{n}{market}
\PYG{p}{)}
\PYG{k}{for} \PYG{n}{afilter} \PYG{o+ow}{in} \PYG{n}{filters}\PYG{p}{:}
    \PYG{n}{search\PYGZus{}space} \PYG{o}{=} \PYG{n}{afilter}\PYG{p}{(}
        \PYG{n}{agent}\PYG{p}{,} \PYG{n}{search\PYGZus{}space}\PYG{p}{,} \PYG{n}{technologies}\PYG{o}{=}\PYG{n}{technologies}\PYG{p}{,} \PYG{n}{market}\PYG{o}{=}\PYG{n}{market}
    \PYG{p}{)}
\PYG{k}{return} \PYG{n}{search\PYGZus{}space}
\end{sphinxVerbatim}

\sphinxcode{\sphinxupquote{initial\_filter}} is simply first filter given on input, if that filter is
registered with \sphinxcode{\sphinxupquote{register\_initializer()}}. Otherwise,
\sphinxcode{\sphinxupquote{initialize\_from\_technologies()}} is automatically inserted.

\end{fulllineitems}

\index{identity() (in module muse.filters)@\spxentry{identity()}\spxextra{in module muse.filters}}

\begin{fulllineitems}
\phantomsection\label{\detokenize{api:muse.filters.identity}}\pysiglinewithargsret{\sphinxcode{\sphinxupquote{muse.filters.}}\sphinxbfcode{\sphinxupquote{identity}}}{\emph{\DUrole{n}{agent}\DUrole{p}{:} \DUrole{n}{muse.agents.agent.Agent}}, \emph{\DUrole{n}{search\_space}\DUrole{p}{:} \DUrole{n}{xarray.core.dataarray.DataArray}}, \emph{\DUrole{o}{*}\DUrole{n}{args}}, \emph{\DUrole{o}{**}\DUrole{n}{kwargs}}}{{ $\rightarrow$ xarray.core.dataarray.DataArray}}
Returns search space as given.

\end{fulllineitems}

\index{initialize\_from\_technologies() (in module muse.filters)@\spxentry{initialize\_from\_technologies()}\spxextra{in module muse.filters}}

\begin{fulllineitems}
\phantomsection\label{\detokenize{api:muse.filters.initialize_from_technologies}}\pysiglinewithargsret{\sphinxcode{\sphinxupquote{muse.filters.}}\sphinxbfcode{\sphinxupquote{initialize\_from\_technologies}}}{\emph{\DUrole{n}{agent}\DUrole{p}{:} \DUrole{n}{muse.agents.agent.Agent}}, \emph{\DUrole{n}{demand}\DUrole{p}{:} \DUrole{n}{xarray.core.dataarray.DataArray}}, \emph{\DUrole{n}{technologies}\DUrole{p}{:} \DUrole{n}{xarray.core.dataset.Dataset}}, \emph{\DUrole{o}{*}\DUrole{n}{args}}, \emph{\DUrole{o}{**}\DUrole{n}{kwargs}}}{}
Initialize a search space from existing technologies.

\end{fulllineitems}

\index{maturity() (in module muse.filters)@\spxentry{maturity()}\spxextra{in module muse.filters}}

\begin{fulllineitems}
\phantomsection\label{\detokenize{api:muse.filters.maturity}}\pysiglinewithargsret{\sphinxcode{\sphinxupquote{muse.filters.}}\sphinxbfcode{\sphinxupquote{maturity}}}{\emph{\DUrole{n}{agent}\DUrole{p}{:} \DUrole{n}{muse.agents.agent.Agent}}, \emph{\DUrole{n}{search\_space}\DUrole{p}{:} \DUrole{n}{xarray.core.dataarray.DataArray}}, \emph{\DUrole{n}{technologies}\DUrole{p}{:} \DUrole{n}{xarray.core.dataset.Dataset}}, \emph{\DUrole{n}{market}\DUrole{p}{:} \DUrole{n}{xarray.core.dataset.Dataset}}, \emph{\DUrole{n}{enduse\_label}\DUrole{p}{:} \DUrole{n}{str} \DUrole{o}{=} \DUrole{default_value}{\textquotesingle{}service\textquotesingle{}}}, \emph{\DUrole{o}{**}\DUrole{n}{kwargs}}}{{ $\rightarrow$ xarray.core.dataarray.DataArray}}
Only allows technologies that have achieve a given market share.

Specifically, the market share refers to the capacity for each end\sphinxhyphen{} use.

\end{fulllineitems}

\index{reduce\_asset() (in module muse.filters)@\spxentry{reduce\_asset()}\spxextra{in module muse.filters}}

\begin{fulllineitems}
\phantomsection\label{\detokenize{api:muse.filters.reduce_asset}}\pysiglinewithargsret{\sphinxcode{\sphinxupquote{muse.filters.}}\sphinxbfcode{\sphinxupquote{reduce\_asset}}}{\emph{\DUrole{n}{agent}\DUrole{p}{:} \DUrole{n}{muse.agents.agent.Agent}}, \emph{\DUrole{n}{search\_space}\DUrole{p}{:} \DUrole{n}{xarray.core.dataarray.DataArray}}, \emph{\DUrole{n}{technologies}\DUrole{p}{:} \DUrole{n}{xarray.core.dataset.Dataset}}, \emph{\DUrole{n}{market}\DUrole{p}{:} \DUrole{n}{xarray.core.dataset.Dataset}}, \emph{\DUrole{o}{**}\DUrole{n}{kwargs}}}{{ $\rightarrow$ xarray.core.dataarray.DataArray}}
Reduce over assets.

\end{fulllineitems}

\index{register\_filter() (in module muse.filters)@\spxentry{register\_filter()}\spxextra{in module muse.filters}}

\begin{fulllineitems}
\phantomsection\label{\detokenize{api:muse.filters.register_filter}}\pysiglinewithargsret{\sphinxcode{\sphinxupquote{muse.filters.}}\sphinxbfcode{\sphinxupquote{register\_filter}}}{\emph{\DUrole{n}{function}\DUrole{p}{:} \DUrole{n}{Callable\DUrole{p}{{[}}\DUrole{p}{{[}}muse.agents.agent.Agent\DUrole{p}{, }xarray.core.dataarray.DataArray\DUrole{p}{, }xarray.core.dataset.Dataset\DUrole{p}{, }xarray.core.dataset.Dataset\DUrole{p}{{]}}\DUrole{p}{, }xarray.core.dataarray.DataArray\DUrole{p}{{]}}}}}{{ $\rightarrow$ Callable}}
Decorator to register a function as a filter.

Registers a function as a filter so that it can be applied easily
when constraining the technology search\sphinxhyphen{}space.

The name that the function is registered with defaults to the function name.
However, it can also be specified explicitly as a \sphinxstyleemphasis{keyword} argument. In any
case, it must be unique amongst all search\sphinxhyphen{}space filters.

\end{fulllineitems}

\index{register\_initializer() (in module muse.filters)@\spxentry{register\_initializer()}\spxextra{in module muse.filters}}

\begin{fulllineitems}
\phantomsection\label{\detokenize{api:muse.filters.register_initializer}}\pysiglinewithargsret{\sphinxcode{\sphinxupquote{muse.filters.}}\sphinxbfcode{\sphinxupquote{register\_initializer}}}{\emph{\DUrole{n}{function}\DUrole{p}{:} \DUrole{n}{Callable\DUrole{p}{{[}}\DUrole{p}{{[}}muse.agents.agent.Agent\DUrole{p}{, }xarray.core.dataarray.DataArray\DUrole{p}{, }xarray.core.dataset.Dataset\DUrole{p}{, }xarray.core.dataset.Dataset\DUrole{p}{{]}}\DUrole{p}{, }xarray.core.dataarray.DataArray\DUrole{p}{{]}}}}}{{ $\rightarrow$ Callable}}
Decorator to register a function as a search\sphinxhyphen{}space initializer.

\end{fulllineitems}

\index{same\_enduse() (in module muse.filters)@\spxentry{same\_enduse()}\spxextra{in module muse.filters}}

\begin{fulllineitems}
\phantomsection\label{\detokenize{api:muse.filters.same_enduse}}\pysiglinewithargsret{\sphinxcode{\sphinxupquote{muse.filters.}}\sphinxbfcode{\sphinxupquote{same\_enduse}}}{\emph{\DUrole{n}{agent}\DUrole{p}{:} \DUrole{n}{muse.agents.agent.Agent}}, \emph{\DUrole{n}{search\_space}\DUrole{p}{:} \DUrole{n}{xarray.core.dataarray.DataArray}}, \emph{\DUrole{n}{technologies}\DUrole{p}{:} \DUrole{n}{xarray.core.dataset.Dataset}}, \emph{\DUrole{o}{*}\DUrole{n}{args}}, \emph{\DUrole{n}{enduse\_label}\DUrole{p}{:} \DUrole{n}{str} \DUrole{o}{=} \DUrole{default_value}{\textquotesingle{}service\textquotesingle{}}}, \emph{\DUrole{o}{**}\DUrole{n}{kwargs}}}{{ $\rightarrow$ xarray.core.dataarray.DataArray}}
Only allow for technologies with at least the same end\sphinxhyphen{}use.

\end{fulllineitems}

\index{same\_fuels() (in module muse.filters)@\spxentry{same\_fuels()}\spxextra{in module muse.filters}}

\begin{fulllineitems}
\phantomsection\label{\detokenize{api:muse.filters.same_fuels}}\pysiglinewithargsret{\sphinxcode{\sphinxupquote{muse.filters.}}\sphinxbfcode{\sphinxupquote{same\_fuels}}}{\emph{\DUrole{n}{agent}\DUrole{p}{:} \DUrole{n}{muse.agents.agent.Agent}}, \emph{\DUrole{n}{search\_space}\DUrole{p}{:} \DUrole{n}{xarray.core.dataarray.DataArray}}, \emph{\DUrole{n}{technologies}\DUrole{p}{:} \DUrole{n}{xarray.core.dataset.Dataset}}, \emph{\DUrole{o}{*}\DUrole{n}{args}}, \emph{\DUrole{o}{**}\DUrole{n}{kwargs}}}{}
Filters technologies with the same fuel type.

\end{fulllineitems}

\index{similar\_technology() (in module muse.filters)@\spxentry{similar\_technology()}\spxextra{in module muse.filters}}

\begin{fulllineitems}
\phantomsection\label{\detokenize{api:muse.filters.similar_technology}}\pysiglinewithargsret{\sphinxcode{\sphinxupquote{muse.filters.}}\sphinxbfcode{\sphinxupquote{similar\_technology}}}{\emph{\DUrole{n}{agent}\DUrole{p}{:} \DUrole{n}{muse.agents.agent.Agent}}, \emph{\DUrole{n}{search\_space}\DUrole{p}{:} \DUrole{n}{xarray.core.dataarray.DataArray}}, \emph{\DUrole{n}{technologies}\DUrole{p}{:} \DUrole{n}{xarray.core.dataset.Dataset}}, \emph{\DUrole{o}{*}\DUrole{n}{args}}, \emph{\DUrole{o}{**}\DUrole{n}{kwargs}}}{}
Filters technologies with the same type.

\end{fulllineitems}

\index{with\_asset\_technology() (in module muse.filters)@\spxentry{with\_asset\_technology()}\spxextra{in module muse.filters}}

\begin{fulllineitems}
\phantomsection\label{\detokenize{api:muse.filters.with_asset_technology}}\pysiglinewithargsret{\sphinxcode{\sphinxupquote{muse.filters.}}\sphinxbfcode{\sphinxupquote{with\_asset\_technology}}}{\emph{\DUrole{n}{agent}\DUrole{p}{:} \DUrole{n}{muse.agents.agent.Agent}}, \emph{\DUrole{n}{search\_space}\DUrole{p}{:} \DUrole{n}{xarray.core.dataarray.DataArray}}, \emph{\DUrole{n}{technologies}\DUrole{p}{:} \DUrole{n}{xarray.core.dataset.Dataset}}, \emph{\DUrole{n}{market}\DUrole{p}{:} \DUrole{n}{xarray.core.dataset.Dataset}}, \emph{\DUrole{o}{**}\DUrole{n}{kwargs}}}{{ $\rightarrow$ xarray.core.dataarray.DataArray}}
Search space \sphinxstyleemphasis{also} contains its asset technology for each asset.

\end{fulllineitems}



\subsection{Decision Methods}
\label{\detokenize{api:module-muse.decisions}}\label{\detokenize{api:decision-methods}}\index{module@\spxentry{module}!muse.decisions@\spxentry{muse.decisions}}\index{muse.decisions@\spxentry{muse.decisions}!module@\spxentry{module}}
Decision methods combining several objectives into ones.

Decisions methods create a single scalar from multiple objectives. To be available from
the input, functions implementing decision methods should follow a specific signature:

\begin{sphinxVerbatim}[commandchars=\\\{\}]
\PYG{n+nd}{@register\PYGZus{}decision}
\PYG{k}{def} \PYG{n+nf}{weighted\PYGZus{}sum}\PYG{p}{(}\PYG{n}{objectives}\PYG{p}{:} \PYG{n}{Dataset}\PYG{p}{,} \PYG{n}{parameters}\PYG{p}{:} \PYG{n}{Any}\PYG{p}{,} \PYG{o}{*}\PYG{o}{*}\PYG{n}{kwargs}\PYG{p}{)} \PYG{o}{\PYGZhy{}}\PYG{o}{\PYGZgt{}} \PYG{n}{DataArray}\PYG{p}{:}
    \PYG{k}{pass}
\end{sphinxVerbatim}
\begin{quote}\begin{description}
\item[{param objectives}] \leavevmode
An dataset where each array is a separate objective

\item[{param parameters}] \leavevmode
parameters, such as weigths, whether to minimize or maximize, the names
of objectives to consider, etc.

\item[{param kwargs}] \leavevmode
Extra input parameters. These parameters are expected to be set from the
input file.

\begin{sphinxadmonition}{warning}{Warning:}
The standard {\hyperref[\detokenize{inputs/agents:inputs-agents}]{\sphinxcrossref{\DUrole{std,std-ref}{agent csv file}}}} does not allow to set
these parameters.
\end{sphinxadmonition}

\item[{returns}] \leavevmode
A data array with ranked replacement technologies.

\end{description}\end{quote}
\index{epsilon\_constraints() (in module muse.decisions)@\spxentry{epsilon\_constraints()}\spxextra{in module muse.decisions}}

\begin{fulllineitems}
\phantomsection\label{\detokenize{api:muse.decisions.epsilon_constraints}}\pysiglinewithargsret{\sphinxcode{\sphinxupquote{muse.decisions.}}\sphinxbfcode{\sphinxupquote{epsilon\_constraints}}}{\emph{\DUrole{n}{objectives}\DUrole{p}{:} \DUrole{n}{xarray.core.dataset.Dataset}}, \emph{\DUrole{n}{parameters}\DUrole{p}{:} \DUrole{n}{Sequence\DUrole{p}{{[}}Tuple\DUrole{p}{{[}}str\DUrole{p}{, }bool\DUrole{p}{, }float\DUrole{p}{{]}}\DUrole{p}{{]}}}}, \emph{\DUrole{n}{mask}\DUrole{p}{:} \DUrole{n}{Optional\DUrole{p}{{[}}Any\DUrole{p}{{]}}} \DUrole{o}{=} \DUrole{default_value}{None}}}{{ $\rightarrow$ xarray.core.dataarray.DataArray}}
Minimizes first objective subject to constraints on other objectives.

The parameters are a sequence of tuples \sphinxtitleref{(name, minimize, epsilon)}, where
\sphinxtitleref{name} is the name of the objective, \sphinxtitleref{minimze} is \sphinxtitleref{True} if minimizing and
false if maximizing that objective, and \sphinxtitleref{epsilon} is the constraint. The
first objective is the one that will be minimized according to:

Given objectives \(O^{(i)}_t\), with \(i \in [|1, N|]\) and \(t\) the
replacement technologies, this function computes the ranking with respect to
\(t\):
\begin{equation*}
\begin{split}\mathrm{ranking}_{O^{(i)}_t < \epsilon_i} O^{(0)}_t\end{split}
\end{equation*}
The first tuple can be restricted to \sphinxtitleref{(name, minimize)}, since \sphinxtitleref{epsilon} is ignored.

The result is the matrix \(O^{(0)}\) modified such minimizing over the
replacement dimension value would take into account the constraints and the
optimization direction (minimize or maximize). In other words, calling
\sphinxtitleref{result.rank(‘replacement’)} will yield the expected result.

\end{fulllineitems}

\index{factory() (in module muse.decisions)@\spxentry{factory()}\spxextra{in module muse.decisions}}

\begin{fulllineitems}
\phantomsection\label{\detokenize{api:muse.decisions.factory}}\pysiglinewithargsret{\sphinxcode{\sphinxupquote{muse.decisions.}}\sphinxbfcode{\sphinxupquote{factory}}}{\emph{\DUrole{n}{settings}\DUrole{p}{:} \DUrole{n}{Union\DUrole{p}{{[}}str\DUrole{p}{, }Mapping\DUrole{p}{{]}}} \DUrole{o}{=} \DUrole{default_value}{\textquotesingle{}mean\textquotesingle{}}}}{{ $\rightarrow$ Callable}}
Creates a decision method based on the input settings.

\end{fulllineitems}

\index{lexical\_comparison() (in module muse.decisions)@\spxentry{lexical\_comparison()}\spxextra{in module muse.decisions}}

\begin{fulllineitems}
\phantomsection\label{\detokenize{api:muse.decisions.lexical_comparison}}\pysiglinewithargsret{\sphinxcode{\sphinxupquote{muse.decisions.}}\sphinxbfcode{\sphinxupquote{lexical\_comparison}}}{\emph{\DUrole{n}{objectives}\DUrole{p}{:} \DUrole{n}{xarray.core.dataset.Dataset}}, \emph{\DUrole{n}{parameters}\DUrole{p}{:} \DUrole{n}{Union\DUrole{p}{{[}}Sequence\DUrole{p}{{[}}Tuple\DUrole{p}{{[}}str\DUrole{p}{, }bool\DUrole{p}{, }float\DUrole{p}{{]}}\DUrole{p}{{]}}\DUrole{p}{, }Sequence\DUrole{p}{{[}}Tuple\DUrole{p}{{[}}str\DUrole{p}{, }float\DUrole{p}{{]}}\DUrole{p}{{]}}\DUrole{p}{{]}}}}}{{ $\rightarrow$ xarray.core.dataarray.DataArray}}
Lexical comparison over the objectives.

Lexical comparison operates by binning the objectives into bins of width
w\_i = min\_j(p\_i o\_i\textasciicircum{}j). Once binned, dimensions other than \sphinxtitleref{asset} and
\sphinxtitleref{technology} are reduced by taking the max, e.g. the largest constraint.
Finally, the objectives are ranked lexographically, in the order given by the
parameters.

The result is an array of tuples which can subsquently be compared
lexicographically.

\end{fulllineitems}

\index{mean() (in module muse.decisions)@\spxentry{mean()}\spxextra{in module muse.decisions}}

\begin{fulllineitems}
\phantomsection\label{\detokenize{api:muse.decisions.mean}}\pysiglinewithargsret{\sphinxcode{\sphinxupquote{muse.decisions.}}\sphinxbfcode{\sphinxupquote{mean}}}{\emph{\DUrole{n}{objectives}\DUrole{p}{:} \DUrole{n}{xarray.core.dataset.Dataset}}, \emph{\DUrole{o}{*}\DUrole{n}{args}}, \emph{\DUrole{o}{**}\DUrole{n}{kwargs}}}{{ $\rightarrow$ xarray.core.dataarray.DataArray}}
Mean over objectives.

\end{fulllineitems}

\index{register\_decision() (in module muse.decisions)@\spxentry{register\_decision()}\spxextra{in module muse.decisions}}

\begin{fulllineitems}
\phantomsection\label{\detokenize{api:muse.decisions.register_decision}}\pysiglinewithargsret{\sphinxcode{\sphinxupquote{muse.decisions.}}\sphinxbfcode{\sphinxupquote{register\_decision}}}{\emph{\DUrole{n}{function}\DUrole{p}{:} \DUrole{n}{Callable\DUrole{p}{{[}}\DUrole{p}{{[}}xarray.core.dataset.Dataset\DUrole{p}{, }Sequence\DUrole{p}{{[}}Tuple\DUrole{p}{{[}}str\DUrole{p}{, }bool\DUrole{p}{, }float\DUrole{p}{{]}}\DUrole{p}{{]}}\DUrole{p}{{]}}\DUrole{p}{, }xarray.core.dataarray.DataArray\DUrole{p}{{]}}}}, \emph{\DUrole{n}{name}\DUrole{p}{:} \DUrole{n}{str}}}{}
Decorator to register a function as a decision.

Registers a function as a decision so that it can be applied easily when aggregating
different objectives together.

\end{fulllineitems}

\index{retro\_epsilon\_constraints() (in module muse.decisions)@\spxentry{retro\_epsilon\_constraints()}\spxextra{in module muse.decisions}}

\begin{fulllineitems}
\phantomsection\label{\detokenize{api:muse.decisions.retro_epsilon_constraints}}\pysiglinewithargsret{\sphinxcode{\sphinxupquote{muse.decisions.}}\sphinxbfcode{\sphinxupquote{retro\_epsilon\_constraints}}}{\emph{\DUrole{n}{objectives}\DUrole{p}{:} \DUrole{n}{xarray.core.dataset.Dataset}}, \emph{\DUrole{n}{parameters}\DUrole{p}{:} \DUrole{n}{Sequence\DUrole{p}{{[}}Tuple\DUrole{p}{{[}}str\DUrole{p}{, }bool\DUrole{p}{, }float\DUrole{p}{{]}}\DUrole{p}{{]}}}}}{{ $\rightarrow$ xarray.core.dataarray.DataArray}}
Epsilon constraints where the current tech is included.

Modifies the parameters to the function such that the existing technologies are
always competitive.

\end{fulllineitems}

\index{retro\_lexical\_comparison() (in module muse.decisions)@\spxentry{retro\_lexical\_comparison()}\spxextra{in module muse.decisions}}

\begin{fulllineitems}
\phantomsection\label{\detokenize{api:muse.decisions.retro_lexical_comparison}}\pysiglinewithargsret{\sphinxcode{\sphinxupquote{muse.decisions.}}\sphinxbfcode{\sphinxupquote{retro\_lexical\_comparison}}}{\emph{\DUrole{n}{objectives}\DUrole{p}{:} \DUrole{n}{xarray.core.dataset.Dataset}}, \emph{\DUrole{n}{parameters}\DUrole{p}{:} \DUrole{n}{Union\DUrole{p}{{[}}Sequence\DUrole{p}{{[}}Tuple\DUrole{p}{{[}}str\DUrole{p}{, }bool\DUrole{p}{, }float\DUrole{p}{{]}}\DUrole{p}{{]}}\DUrole{p}{, }Sequence\DUrole{p}{{[}}Tuple\DUrole{p}{{[}}str\DUrole{p}{, }float\DUrole{p}{{]}}\DUrole{p}{{]}}\DUrole{p}{{]}}}}}{{ $\rightarrow$ xarray.core.dataarray.DataArray}}
Lexical comparison over the objectives.

Lexical comparison operates by binning the objectives into bins of width
w\_i = p\_i o\_i, where i are the current assets. Once binned, dimensions other
than \sphinxtitleref{asset} and \sphinxtitleref{replacement} are reduced by taking the max, e.g. the
largest constraint.  Finally, the objectives are ranked lexographically, in
the order given by the parameters.

The result is an array of tuples which can subsquently be compared
lexicographically.

\end{fulllineitems}

\index{single\_objective() (in module muse.decisions)@\spxentry{single\_objective()}\spxextra{in module muse.decisions}}

\begin{fulllineitems}
\phantomsection\label{\detokenize{api:muse.decisions.single_objective}}\pysiglinewithargsret{\sphinxcode{\sphinxupquote{muse.decisions.}}\sphinxbfcode{\sphinxupquote{single\_objective}}}{\emph{\DUrole{n}{objectives}\DUrole{p}{:} \DUrole{n}{xarray.core.dataset.Dataset}}, \emph{\DUrole{n}{parameters}\DUrole{p}{:} \DUrole{n}{Union\DUrole{p}{{[}}str\DUrole{p}{, }Tuple\DUrole{p}{{[}}str\DUrole{p}{, }bool\DUrole{p}{{]}}\DUrole{p}{, }Tuple\DUrole{p}{{[}}str\DUrole{p}{, }bool\DUrole{p}{, }float\DUrole{p}{{]}}\DUrole{p}{, }Sequence\DUrole{p}{{[}}Tuple\DUrole{p}{{[}}str\DUrole{p}{, }bool\DUrole{p}{, }float\DUrole{p}{{]}}\DUrole{p}{{]}}\DUrole{p}{{]}}}}}{{ $\rightarrow$ xarray.core.dataarray.DataArray}}
Single objective decision method.

It only decides on minimization vs maximization and multiplies by a given factor.
The input parameters can take the following forms:
\begin{itemize}
\item {} 
Standard sequence \sphinxtitleref{{[}(objective, direction, factor){]}}, in which case it must have
only one element.

\item {} 
A single string: defaults to standard sequence \sphinxtitleref{{[}(string, 1, 1){]}}

\item {} 
A tuple (string, bool): defaults to standard sequence
\sphinxtitleref{{[}(string, direction, 1){]}}

\item {} 
A tuple (string, bool, factor): defaults to standard sequence
\sphinxtitleref{{[}(string, direction, factor){]}}

\end{itemize}

\end{fulllineitems}

\index{weighted\_sum() (in module muse.decisions)@\spxentry{weighted\_sum()}\spxextra{in module muse.decisions}}

\begin{fulllineitems}
\phantomsection\label{\detokenize{api:muse.decisions.weighted_sum}}\pysiglinewithargsret{\sphinxcode{\sphinxupquote{muse.decisions.}}\sphinxbfcode{\sphinxupquote{weighted\_sum}}}{\emph{\DUrole{n}{objectives}\DUrole{p}{:} \DUrole{n}{xarray.core.dataset.Dataset}}, \emph{\DUrole{n}{parameters}\DUrole{p}{:} \DUrole{n}{Mapping\DUrole{p}{{[}}str\DUrole{p}{, }float\DUrole{p}{{]}}}}}{{ $\rightarrow$ xarray.core.dataarray.DataArray}}
Weighted sum over normalized objectives.

The objectives are each normalized to {[}0, 1{]} over the \sphinxtitleref{replacement}
dimension. Furthermore, the dimensions other than \sphinxtitleref{asset} and \sphinxtitleref{replacement}
are reduced by taking the mean.

More specifically, the objective function is:
\begin{equation*}
\begin{split}\sum_m c_m \frac{A_m - \min(A_m)}{\max(A_m) - \min(A_m)}\end{split}
\end{equation*}
where sum runs over the different objectives, c\_m is a scalar coefficient,
A\_m is a matrix with dimensions (existing tech, replacemnt tech). \sphinxtitleref{max(A)}
and \sphinxtitleref{min(A)} return the largest and smallest component of the input matrix.
If c\_m is positive, then that particular objective is minimized, whereas if
it is negative, that particular objective is maximized.

\end{fulllineitems}



\subsection{Investment Methods}
\label{\detokenize{api:module-muse.investments}}\label{\detokenize{api:investment-methods}}\index{module@\spxentry{module}!muse.investments@\spxentry{muse.investments}}\index{muse.investments@\spxentry{muse.investments}!module@\spxentry{module}}
Investment decision.

An investment determines which technologies to invest given a metric to
determine preferred technologies, a corresponding search space of technologies,
and the demand to fulfill.

Investments should be registered via the decorator \sphinxtitleref{register\_investment}. The
registration makes it possible to call investments dynamically through
\sphinxtitleref{compute\_investment}, by specifying the name of the investment. It is part of
MUSE’s plugin platform.

Investments are not expected to modify any of their arguments. They should all
have the following signature:

\begin{sphinxVerbatim}[commandchars=\\\{\}]
\PYG{n+nd}{@register\PYGZus{}investment}
\PYG{k}{def} \PYG{n+nf}{investment}\PYG{p}{(}
    \PYG{n}{costs}\PYG{p}{:} \PYG{n}{DataArray}\PYG{p}{,}
    \PYG{n}{search\PYGZus{}space}\PYG{p}{:} \PYG{n}{DataArray}\PYG{p}{,}
    \PYG{n}{technologies}\PYG{p}{:} \PYG{n}{Dataset}\PYG{p}{,}
    \PYG{n}{constraints}\PYG{p}{:} \PYG{n}{List}\PYG{p}{[}\PYG{n}{Constraint}\PYG{p}{]}\PYG{p}{,}
    \PYG{n}{year}\PYG{p}{:} \PYG{n+nb}{int}\PYG{p}{,}
    \PYG{o}{*}\PYG{o}{*}\PYG{n}{kwargs}
\PYG{p}{)} \PYG{o}{\PYGZhy{}}\PYG{o}{\PYGZgt{}} \PYG{n}{DataArray}\PYG{p}{:}
    \PYG{k}{pass}
\end{sphinxVerbatim}
\begin{quote}\begin{description}
\item[{param costs}] \leavevmode
specifies for each \sphinxtitleref{asset} which \sphinxtitleref{replacement} technology should be invested
in preferentially. This should be an integer or floating point array with
dimensions \sphinxtitleref{asset} and \sphinxtitleref{replacement}.

\item[{param search\_space}] \leavevmode
an \sphinxtitleref{asset} by \sphinxtitleref{replacement} matrix defining allowed and disallowed
replacement technologies for each asset

\item[{param technologies}] \leavevmode
a dataset containing all constant data characterizing the
technologies.

\item[{param constraints}] \leavevmode
a list of constraints as defined in \sphinxcode{\sphinxupquote{constraints}}.

\item[{param year}] \leavevmode
the current year.

\item[{returns}] \leavevmode
A data array with dimensions \sphinxtitleref{asset} and \sphinxtitleref{technology} specifying the amount
of newly invested capacity.

\end{description}\end{quote}
\index{INVESTMENT\_SIGNATURE (in module muse.investments)@\spxentry{INVESTMENT\_SIGNATURE}\spxextra{in module muse.investments}}

\begin{fulllineitems}
\phantomsection\label{\detokenize{api:muse.investments.INVESTMENT_SIGNATURE}}\pysigline{\sphinxcode{\sphinxupquote{muse.investments.}}\sphinxbfcode{\sphinxupquote{INVESTMENT\_SIGNATURE}}}
Investment signature.

alias of Callable{[}{[}xarray.core.dataarray.DataArray, xarray.core.dataarray.DataArray, xarray.core.dataset.Dataset, List{[}xarray.core.dataset.Dataset{]}, Any{]}, xarray.core.dataarray.DataArray{]}

\end{fulllineitems}

\index{cliff\_retirement\_profile() (in module muse.investments)@\spxentry{cliff\_retirement\_profile()}\spxextra{in module muse.investments}}

\begin{fulllineitems}
\phantomsection\label{\detokenize{api:muse.investments.cliff_retirement_profile}}\pysiglinewithargsret{\sphinxcode{\sphinxupquote{muse.investments.}}\sphinxbfcode{\sphinxupquote{cliff\_retirement\_profile}}}{\emph{\DUrole{n}{technical\_life}\DUrole{p}{:} \DUrole{n}{xarray.core.dataarray.DataArray}}, \emph{\DUrole{n}{current\_year}\DUrole{p}{:} \DUrole{n}{int} \DUrole{o}{=} \DUrole{default_value}{0}}, \emph{\DUrole{n}{protected}\DUrole{p}{:} \DUrole{n}{int} \DUrole{o}{=} \DUrole{default_value}{0}}, \emph{\DUrole{n}{interpolation}\DUrole{p}{:} \DUrole{n}{str} \DUrole{o}{=} \DUrole{default_value}{\textquotesingle{}linear\textquotesingle{}}}, \emph{\DUrole{o}{**}\DUrole{n}{kwargs}}}{{ $\rightarrow$ xarray.core.dataarray.DataArray}}
Cliff\sphinxhyphen{}like retirement profile from current year.

Computes the retirement profile of all technologies in \sphinxcode{\sphinxupquote{technical\_life}}.
Assets with a technical life smaller than the input time\sphinxhyphen{}period should automatically
be renewed.

Hence, if \sphinxcode{\sphinxupquote{technical\_life \textless{}= protected}}, then effectively, the technical life is
rewritten as \sphinxcode{\sphinxupquote{technical\_life * n}} with \sphinxcode{\sphinxupquote{n = int(protected // technical\_life) +
1}}.

We could just return an array where each year is repesented. Instead, to save
memory, we return a compact view of the same where years where no change happens are
removed.
\begin{quote}\begin{description}
\item[{Parameters}] \leavevmode\begin{itemize}
\item {} 
\sphinxstyleliteralstrong{\sphinxupquote{technical\_life}} \textendash{} lifetimes for each technology

\item {} 
\sphinxstyleliteralstrong{\sphinxupquote{current\_year}} \textendash{} current year

\item {} 
\sphinxstyleliteralstrong{\sphinxupquote{protected}} \textendash{} The technologies are assumed to be renewed between years
\sphinxtitleref{current\_year} and \sphinxtitleref{current\_year + protected}

\item {} 
\sphinxstyleliteralstrong{\sphinxupquote{**kwargs}} \textendash{} arguments by which to filter technical\_life, if any.

\end{itemize}

\item[{Returns}] \leavevmode
A boolean DataArray where each each element along the year dimension is
true if the technology is still not retired for the given year.

\end{description}\end{quote}

\end{fulllineitems}

\index{register\_investment() (in module muse.investments)@\spxentry{register\_investment()}\spxextra{in module muse.investments}}

\begin{fulllineitems}
\phantomsection\label{\detokenize{api:muse.investments.register_investment}}\pysiglinewithargsret{\sphinxcode{\sphinxupquote{muse.investments.}}\sphinxbfcode{\sphinxupquote{register\_investment}}}{\emph{\DUrole{n}{function}\DUrole{p}{:} \DUrole{n}{Callable\DUrole{p}{{[}}\DUrole{p}{{[}}xarray.core.dataarray.DataArray\DUrole{p}{, }xarray.core.dataarray.DataArray\DUrole{p}{, }xarray.core.dataset.Dataset\DUrole{p}{, }List\DUrole{p}{{[}}xarray.core.dataset.Dataset\DUrole{p}{{]}}\DUrole{p}{, }Any\DUrole{p}{{]}}\DUrole{p}{, }xarray.core.dataarray.DataArray\DUrole{p}{{]}}}}}{{ $\rightarrow$ Callable\DUrole{p}{{[}}\DUrole{p}{{[}}xarray.core.dataarray.DataArray\DUrole{p}{, }xarray.core.dataarray.DataArray\DUrole{p}{, }xarray.core.dataset.Dataset\DUrole{p}{, }List\DUrole{p}{{[}}xarray.core.dataset.Dataset\DUrole{p}{{]}}\DUrole{p}{, }Any\DUrole{p}{{]}}\DUrole{p}{, }xarray.core.dataarray.DataArray\DUrole{p}{{]}}}}
Decorator to register a function as an investment.

\end{fulllineitems}



\subsection{Demand Share}
\label{\detokenize{api:module-muse.demand_share}}\label{\detokenize{api:demand-share}}\index{module@\spxentry{module}!muse.demand\_share@\spxentry{muse.demand\_share}}\index{muse.demand\_share@\spxentry{muse.demand\_share}!module@\spxentry{module}}
Demand share computations.

The demand share splits a demand amongst agents. It is used within a sector to assign
part of the input MCA demand to each agent.

Demand shares functions should be registered via the decorator \sphinxtitleref{register\_demand\_share}.

Demand share functions are not expected to modify any of their arguments. They
should all have the following signature:

\begin{sphinxVerbatim}[commandchars=\\\{\}]
\PYG{n+nd}{@register\PYGZus{}demand\PYGZus{}share}
\PYG{k}{def} \PYG{n+nf}{demand\PYGZus{}share}\PYG{p}{(}
    \PYG{n}{agents}\PYG{p}{:} \PYG{n}{Sequence}\PYG{p}{[}\PYG{n}{AbstractAgent}\PYG{p}{]}\PYG{p}{,}
    \PYG{n}{market}\PYG{p}{:} \PYG{n}{xr}\PYG{o}{.}\PYG{n}{Dataset}\PYG{p}{,}
    \PYG{n}{technologies}\PYG{p}{:} \PYG{n}{xr}\PYG{o}{.}\PYG{n}{Dataset}\PYG{p}{,}
    \PYG{o}{*}\PYG{o}{*}\PYG{n}{kwargs}
\PYG{p}{)} \PYG{o}{\PYGZhy{}}\PYG{o}{\PYGZgt{}} \PYG{n}{xr}\PYG{o}{.}\PYG{n}{DataArray}\PYG{p}{:}
    \PYG{k}{pass}
\end{sphinxVerbatim}
\begin{quote}\begin{description}
\item[{param agents}] \leavevmode
a sequence of  agent relevant to the demand share procedure. The agent can
be queried for parameters specific to the demand share procedure. For instance,
:py:func\textasciigrave{}new\_and\_retro\textasciigrave{} will query the agents for the assets they own, the
region they are contained with, their category (new or retrofit), etc…

\item[{param market}] \leavevmode
Market variables, including prices, consumption and supply.

\item[{param technologies}] \leavevmode
a dataset containing all constant data characterizing the
technologies.

\item[{param kwargs}] \leavevmode
Any number of keyword arguments that can parametrize how the demand is
shared. These keyword arguments can be modified from the TOML file.

\item[{returns}] \leavevmode
The unmet consumption. Unless indicated, all agents will compete for a the full
demand. However, if there exists a coordinate “agent” of dimension “asset” giving
the \sphinxcode{\sphinxupquote{uuid}} of the agent, then agents will
only service that par of the demand.

\end{description}\end{quote}
\index{DEMAND\_SHARE\_SIGNATURE (in module muse.demand\_share)@\spxentry{DEMAND\_SHARE\_SIGNATURE}\spxextra{in module muse.demand\_share}}

\begin{fulllineitems}
\phantomsection\label{\detokenize{api:muse.demand_share.DEMAND_SHARE_SIGNATURE}}\pysigline{\sphinxcode{\sphinxupquote{muse.demand\_share.}}\sphinxbfcode{\sphinxupquote{DEMAND\_SHARE\_SIGNATURE}}}
Demand share signature.

alias of Callable{[}{[}Sequence{[}muse.agents.agent.AbstractAgent{]}, xarray.core.dataset.Dataset, xarray.core.dataset.Dataset, Any{]}, xarray.core.dataarray.DataArray{]}

\end{fulllineitems}

\index{new\_and\_retro() (in module muse.demand\_share)@\spxentry{new\_and\_retro()}\spxextra{in module muse.demand\_share}}

\begin{fulllineitems}
\phantomsection\label{\detokenize{api:muse.demand_share.new_and_retro}}\pysiglinewithargsret{\sphinxcode{\sphinxupquote{muse.demand\_share.}}\sphinxbfcode{\sphinxupquote{new\_and\_retro}}}{\emph{\DUrole{n}{agents}\DUrole{p}{:} \DUrole{n}{Sequence\DUrole{p}{{[}}muse.agents.agent.AbstractAgent\DUrole{p}{{]}}}}, \emph{\DUrole{n}{market}\DUrole{p}{:} \DUrole{n}{xarray.core.dataset.Dataset}}, \emph{\DUrole{n}{technologies}\DUrole{p}{:} \DUrole{n}{xarray.core.dataset.Dataset}}, \emph{\DUrole{n}{production}\DUrole{p}{:} \DUrole{n}{Union\DUrole{p}{{[}}str\DUrole{p}{, }Mapping\DUrole{p}{, }Callable\DUrole{p}{{]}}} \DUrole{o}{=} \DUrole{default_value}{\textquotesingle{}maximum\_production\textquotesingle{}}}, \emph{\DUrole{n}{current\_year}\DUrole{p}{:} \DUrole{n}{Optional\DUrole{p}{{[}}int\DUrole{p}{{]}}} \DUrole{o}{=} \DUrole{default_value}{None}}, \emph{\DUrole{n}{forecast}\DUrole{p}{:} \DUrole{n}{int} \DUrole{o}{=} \DUrole{default_value}{5}}}{{ $\rightarrow$ xarray.core.dataarray.DataArray}}
Splits demand across new and retro agents.

The input demand is split amongst both \sphinxstyleemphasis{new} and \sphinxstyleemphasis{retro} agents. \sphinxstyleemphasis{New} agents get a
share of the increase in demand for the forecast year, whereas \sphinxstyleemphasis{retrofi} agents are
assigned a share of the demand that occurs from decommissioned assets.
\begin{quote}\begin{description}
\item[{Parameters}] \leavevmode\begin{itemize}
\item {} 
\sphinxstyleliteralstrong{\sphinxupquote{agents}} \textendash{} a list of all agents. This list should mainly be used to determine the
type of an agent and the assets it owns. The agents will not be modified in
any way.

\item {} 
\sphinxstyleliteralstrong{\sphinxupquote{market}} \textendash{} the market for which to satisfy the demand. It should contain at\sphinxhyphen{}least
\sphinxcode{\sphinxupquote{consumption}} and \sphinxcode{\sphinxupquote{supply}}. It may contain \sphinxcode{\sphinxupquote{prices}} if that is of use
to the production method. The \sphinxcode{\sphinxupquote{consumption}} reflects the demand for the
commodities produced by the current sector.

\item {} 
\sphinxstyleliteralstrong{\sphinxupquote{technologies}} \textendash{} quantities describing the technologies.

\end{itemize}

\end{description}\end{quote}

Pseudo\sphinxhyphen{}code:
\begin{enumerate}
\sphinxsetlistlabels{\arabic}{enumi}{enumii}{}{.}%
\item {} 
the capacity is reduced over agents and  expanded over timeslices (extensive
quantity) and aggregated over agents. Generally:
\begin{equation*}
\begin{split}A_{a, s}^r = w_s\sum_i A_a^{r, i}\end{split}
\end{equation*}
with \(w_s\) a weight associated with each timeslice and determined via
\sphinxcode{\sphinxupquote{muse.timeslices.convert\_timeslice()}}.

\item {} 
An intermediate quantity, the \sphinxcode{\sphinxupquote{unmet demand}} \(U\) is defined from
\(P[\mathcal{M}, \mathcal{A}]\), a function giving the production for a given
market \(\mathcal{M}\), the associated consumption \(\mathcal{C}\), and
aggregate assets \(\mathcal{A}\):
\begin{equation*}
\begin{split}U[\mathcal{M}, \mathcal{A}] =
  \max(\mathcal{C} - P[\mathcal{M}, \mathcal{A}], 0)\end{split}
\end{equation*}
where \(\max\) operates element\sphinxhyphen{}wise, and indices have been dropped for
simplicity. The resulting expression has the same indices as the consumption
\(\mathcal{C}_{c, s}^r\).

\(P\) is any function registered with
\sphinxcode{\sphinxupquote{@register\_production}}.

\item {} 
the \sphinxstyleemphasis{new} demand \(N\) is defined as:
\begin{quote}
\begin{equation*}
\begin{split}N = \min\left(
    \mathcal{C}_{c, s}^r(y + \Delta y) - \mathcal{C}_{c, s}^r(y),
    U[\mathcal{M}^r(y + \Delta y), \mathcal{A}_{a, s}^r(y)]
\right)\end{split}
\end{equation*}\end{quote}

\item {} 
the \sphinxstyleemphasis{retrofit} demand \(R\) is defined from the identity
\begin{equation*}
\begin{split}C_{c, s}^r(y + \Delta y) =
 P[\mathcal{M}^r(y+\Delta y), \mathcal{A}_{a, s}^r(y + \Delta y)]
 + N_{c, s}^r
 + R_{c, s}^r\end{split}
\end{equation*}
In other words, it is the share of the forecasted consumption that is serviced
neither by the current assets still present in the forecast year, nor by the
\sphinxstyleemphasis{new} agent.

\item {} \begin{description}
\item[{then each \sphinxstyleemphasis{new} agent gets a share of \(N\) proportional to it’s}] \leavevmode
share of the \sphinxcode{\sphinxupquote{production}},
\(P[\mathcal{A}_{a, s}^{r, i}(y)]\).  Then the share of the demand for new
agent \(i\) is:
\begin{equation*}
\begin{split}N_{c, s, t}^{i, r}(y) = N_{c, s}^r
    \frac{\sum_\iota P[\mathcal{A}_{s, t, \iota}^{r, i}(y)]}
         {\sum_{i, t, \iota}P[\mathcal{A}_{s, t, \iota}^{r, i}(y)]}\end{split}
\end{equation*}
\end{description}

\item {} \begin{description}
\item[{similarly, each \sphinxstyleemphasis{retrofit} agent gets a share of \(N\) proportional to it’s}] \leavevmode
share of the \sphinxcode{\sphinxupquote{decommissioning demand}}, \(D^{r, i}_{t, c}\).
Then the share of the demand for retrofit agent \(i\) is:
\begin{equation*}
\begin{split}R_{c, s, t}^{i, r}(y) = R_{c, s}^r
    \frac{\sum_\iota\mathcal{D}_{t, c, \iota}^{i, r}(y)}
        {\sum_{i, t, \iota}\mathcal{D}_{t, c, \iota}^{i, r}(y)}\end{split}
\end{equation*}
\end{description}

\end{enumerate}

Note that tin the last two steps, the assets owned by the agent are aggregated over
the installation year. The effect is that the demand serviced by agents is
disaggregated over each technology, rather than not over each \sphinxstyleemphasis{model} of each
technology.


\sphinxstrong{See also:}


\DUrole{xref,std,std-ref}{indices}, {\hyperref[\detokenize{api:module-muse.quantities}]{\sphinxcrossref{\DUrole{std,std-ref}{Quantities}}}},
\DUrole{xref,std,std-ref}{Agent investments},
\sphinxcode{\sphinxupquote{decommissioning\_demand()}},
\sphinxcode{\sphinxupquote{maximum\_production()}}



\end{fulllineitems}

\index{register\_demand\_share() (in module muse.demand\_share)@\spxentry{register\_demand\_share()}\spxextra{in module muse.demand\_share}}

\begin{fulllineitems}
\phantomsection\label{\detokenize{api:muse.demand_share.register_demand_share}}\pysiglinewithargsret{\sphinxcode{\sphinxupquote{muse.demand\_share.}}\sphinxbfcode{\sphinxupquote{register\_demand\_share}}}{\emph{\DUrole{n}{function}\DUrole{p}{:} \DUrole{n}{Callable\DUrole{p}{{[}}\DUrole{p}{{[}}Sequence\DUrole{p}{{[}}muse.agents.agent.AbstractAgent\DUrole{p}{{]}}\DUrole{p}{, }xarray.core.dataset.Dataset\DUrole{p}{, }xarray.core.dataset.Dataset\DUrole{p}{, }Any\DUrole{p}{{]}}\DUrole{p}{, }xarray.core.dataarray.DataArray\DUrole{p}{{]}}}}}{}
Decorator to register a function as a demand share calculation.

\end{fulllineitems}

\index{unmet\_demand() (in module muse.demand\_share)@\spxentry{unmet\_demand()}\spxextra{in module muse.demand\_share}}

\begin{fulllineitems}
\phantomsection\label{\detokenize{api:muse.demand_share.unmet_demand}}\pysiglinewithargsret{\sphinxcode{\sphinxupquote{muse.demand\_share.}}\sphinxbfcode{\sphinxupquote{unmet\_demand}}}{\emph{\DUrole{n}{market}\DUrole{p}{:} \DUrole{n}{xarray.core.dataset.Dataset}}, \emph{\DUrole{n}{capacity}\DUrole{p}{:} \DUrole{n}{xarray.core.dataarray.DataArray}}, \emph{\DUrole{n}{technologies}\DUrole{p}{:} \DUrole{n}{xarray.core.dataset.Dataset}}, \emph{\DUrole{n}{production}\DUrole{p}{:} \DUrole{n}{Union\DUrole{p}{{[}}str\DUrole{p}{, }Mapping\DUrole{p}{, }Callable\DUrole{p}{{]}}} \DUrole{o}{=} \DUrole{default_value}{\textquotesingle{}maximum\_production\textquotesingle{}}}}{}
Share of the demand that cannot be serviced by the existing assets.
\begin{equation*}
\begin{split}U[\mathcal{M}, \mathcal{A}] =
  \max(\mathcal{C} - P[\mathcal{M}, \mathcal{A}], 0)\end{split}
\end{equation*}
\(\max\) operates element\sphinxhyphen{}wise, and indices have been dropped for simplicity.
The resulting expression has the same indices as the consumption
\(\mathcal{C}_{c, s}^r\).

\(P\) is any function registered with
\sphinxcode{\sphinxupquote{@register\_production}}.

\end{fulllineitems}

\index{unmet\_forecasted\_demand() (in module muse.demand\_share)@\spxentry{unmet\_forecasted\_demand()}\spxextra{in module muse.demand\_share}}

\begin{fulllineitems}
\phantomsection\label{\detokenize{api:muse.demand_share.unmet_forecasted_demand}}\pysiglinewithargsret{\sphinxcode{\sphinxupquote{muse.demand\_share.}}\sphinxbfcode{\sphinxupquote{unmet\_forecasted\_demand}}}{\emph{\DUrole{n}{agents}\DUrole{p}{:} \DUrole{n}{Sequence\DUrole{p}{{[}}muse.agents.agent.AbstractAgent\DUrole{p}{{]}}}}, \emph{\DUrole{n}{market}\DUrole{p}{:} \DUrole{n}{xarray.core.dataset.Dataset}}, \emph{\DUrole{n}{technologies}\DUrole{p}{:} \DUrole{n}{xarray.core.dataset.Dataset}}, \emph{\DUrole{n}{current\_year}\DUrole{p}{:} \DUrole{n}{Optional\DUrole{p}{{[}}int\DUrole{p}{{]}}} \DUrole{o}{=} \DUrole{default_value}{None}}, \emph{\DUrole{n}{production}\DUrole{p}{:} \DUrole{n}{Union\DUrole{p}{{[}}str\DUrole{p}{, }Mapping\DUrole{p}{, }Callable\DUrole{p}{{]}}} \DUrole{o}{=} \DUrole{default_value}{\textquotesingle{}maximum\_production\textquotesingle{}}}, \emph{\DUrole{n}{forecast}\DUrole{p}{:} \DUrole{n}{int} \DUrole{o}{=} \DUrole{default_value}{5}}}{{ $\rightarrow$ xarray.core.dataarray.DataArray}}
Forecast demand that cannot be serviced by non\sphinxhyphen{}decommissioned current assets.

\end{fulllineitems}



\subsection{Constraints:}
\label{\detokenize{api:module-muse.constraints}}\label{\detokenize{api:constraints}}\index{module@\spxentry{module}!muse.constraints@\spxentry{muse.constraints}}\index{muse.constraints@\spxentry{muse.constraints}!module@\spxentry{module}}
Investment constraints.

Constraints on investements ensure that investements match some given criteria. For
instance, the constraints could ensure that only so much of a new asset can be built
every year.

Functions to compute constraints should be registered via the decorator
\sphinxcode{\sphinxupquote{register\_constraints()}}. This registration step makes it
possible for constraints to be declared in the TOML file.

Generally, LP solvers accept linear constraint defined as:
\begin{equation*}
\begin{split}A x \leq b\end{split}
\end{equation*}
with \(A\) a matrix, \(x\) the decision variables, and \(b\) a vector.
However, these quantities are dimensionless. They do no have timeslices, assets, or
replacement technologies, or any other dimensions that users have set\sphinxhyphen{}up in their model.
The crux is to translates from MUSE’s data\sphinxhyphen{}structures to a consistent dimensionless
format.

In MUSE, users can register constraints functions that return fully dimensional
quantities. The matrix operator is split over the capacity decision variables and the
production decision variables:
\begin{equation*}
\begin{split}A_c .* x_c + A_p .* x_p \leq b\end{split}
\end{equation*}
The operator \(.*\) means the standard elementwise multiplication of xarray,
including automatic broadcasting (adding missing dimensions by repeating the smaller
matrix along the missing dimension).  Constraint functions return the three quantities
\(A_c\), \(A_p\), and \(b\). These three quantities will often not have the
same dimension. E.g. one might include timeslices where another might not. The
transformation from \(A_c\), \(A_p\), \(b\) to \(A\) and \(b\)
happens as described below.
\begin{itemize}
\item {} 
\(b\) remains the same. It defines the rows of \(A\).

\item {} 
\(x_c\) and \(x_p\) are concatenated one on top of the other and define the
columns of \(A\).

\item {} 
\(A\) is split into a left submatrix for capacities and a right submatrix for
production, following the concatenation of \(x_c\) and \(x_p\)

\item {} 
Any dimension in \(A_c .* x_c\) (\(A_p .* x_p\)) that is also in \(b\)
defines diagonal entries into the left (right) submatrix of \(A\).

\item {} 
Any dimension in \(A_c .* x_c\) (\(A_p .* x_b\)) and missing from
\(b\) is reduce by summation over a row in the left (right) submatrix of
\(A\). In other words, those dimension do become part of a standard tensor
reduction or matrix multiplication.

\end{itemize}

There are two additional rules. However, they are likely to be the result of an
inefficient defininition of \(A_c\), \(A_p\) and \(b\).
\begin{itemize}
\item {} 
Any dimension in \(A_c\) (\(A_b\)) that is neither in \(b\) nor in
\(x_c\) (\(x_p\)) is reduced by summation before consideration for the
elementwise multiplication. For instance, if \(d\) is such a dimension, present
only in \(A_c\), then the problem becomes \((\sum_d A_c) .* x_c + A_p .* x_p
\leq b\).

\item {} 
Any dimension missing from \(A_c .* x_c\) (\(A_p .* x_p\)) and present in
\(b\) is added by repeating the resulting row in \(A\).

\end{itemize}

Constraints are registered using the decorator
\sphinxcode{\sphinxupquote{register\_constraints()}}. The decorated functions must follow
the following signature:

\begin{sphinxVerbatim}[commandchars=\\\{\}]
\PYG{n+nd}{@register\PYGZus{}constraints}
\PYG{k}{def} \PYG{n+nf}{constraints}\PYG{p}{(}
    \PYG{n}{demand}\PYG{p}{:} \PYG{n}{xr}\PYG{o}{.}\PYG{n}{DataArray}\PYG{p}{,}
    \PYG{n}{assets}\PYG{p}{:} \PYG{n}{xr}\PYG{o}{.}\PYG{n}{Dataset}\PYG{p}{,}
    \PYG{n}{search\PYGZus{}space}\PYG{p}{:} \PYG{n}{xr}\PYG{o}{.}\PYG{n}{DataArray}\PYG{p}{,}
    \PYG{n}{market}\PYG{p}{:} \PYG{n}{xr}\PYG{o}{.}\PYG{n}{Dataset}\PYG{p}{,}
    \PYG{n}{technologies}\PYG{p}{:} \PYG{n}{xr}\PYG{o}{.}\PYG{n}{Dataset}\PYG{p}{,}
    \PYG{n}{year}\PYG{p}{:} \PYG{n}{Optional}\PYG{p}{[}\PYG{n+nb}{int}\PYG{p}{]} \PYG{o}{=} \PYG{k+kc}{None}\PYG{p}{,}
    \PYG{o}{*}\PYG{o}{*}\PYG{n}{kwargs}\PYG{p}{,}
\PYG{p}{)} \PYG{o}{\PYGZhy{}}\PYG{o}{\PYGZgt{}} \PYG{n}{Constraint}\PYG{p}{:}
    \PYG{k}{pass}
\end{sphinxVerbatim}
\begin{description}
\item[{demand:}] \leavevmode
The demand for the sectors products. In practice it is a demand share obtained in
\sphinxcode{\sphinxupquote{demand\_share}}. It is a data\sphinxhyphen{}array with dimensions including \sphinxtitleref{asset},
\sphinxtitleref{commodity}, \sphinxtitleref{timeslice}.

\item[{assets:}] \leavevmode
The capacity of the assets owned by the agent.

\item[{search\_space:}] \leavevmode
A matrix \sphinxtitleref{asset} vs \sphinxtitleref{replacement} technology defining which replacement technologies
will be considered for each existing asset.

\item[{market:}] \leavevmode
The market as obtained from the MCA.

\item[{technologies:}] \leavevmode
Technodata characterizing the competing technologies.

\item[{year:}] \leavevmode
current year.

\item[{\sphinxcode{\sphinxupquote{**kwargs}}:}] \leavevmode
Any other parameter.

\end{description}
\index{ScipyAdapter (class in muse.constraints)@\spxentry{ScipyAdapter}\spxextra{class in muse.constraints}}

\begin{fulllineitems}
\phantomsection\label{\detokenize{api:muse.constraints.ScipyAdapter}}\pysiglinewithargsret{\sphinxbfcode{\sphinxupquote{class }}\sphinxcode{\sphinxupquote{muse.constraints.}}\sphinxbfcode{\sphinxupquote{ScipyAdapter}}}{\emph{\DUrole{n}{c}\DUrole{p}{:} \DUrole{n}{numpy.ndarray}}, \emph{\DUrole{n}{to\_muse}\DUrole{p}{:} \DUrole{n}{Callable\DUrole{p}{{[}}\DUrole{p}{{[}}numpy.ndarray\DUrole{p}{{]}}\DUrole{p}{, }xarray.core.dataset.Dataset\DUrole{p}{{]}}}}, \emph{\DUrole{n}{bounds}\DUrole{p}{:} \DUrole{n}{Tuple\DUrole{p}{{[}}Optional\DUrole{p}{{[}}float\DUrole{p}{{]}}\DUrole{p}{, }Optional\DUrole{p}{{[}}float\DUrole{p}{{]}}\DUrole{p}{{]}}} \DUrole{o}{=} \DUrole{default_value}{0, inf}}, \emph{\DUrole{n}{A\_ub}\DUrole{p}{:} \DUrole{n}{Optional\DUrole{p}{{[}}numpy.ndarray\DUrole{p}{{]}}} \DUrole{o}{=} \DUrole{default_value}{None}}, \emph{\DUrole{n}{b\_ub}\DUrole{p}{:} \DUrole{n}{Optional\DUrole{p}{{[}}numpy.ndarray\DUrole{p}{{]}}} \DUrole{o}{=} \DUrole{default_value}{None}}, \emph{\DUrole{n}{A\_eq}\DUrole{p}{:} \DUrole{n}{Optional\DUrole{p}{{[}}numpy.ndarray\DUrole{p}{{]}}} \DUrole{o}{=} \DUrole{default_value}{None}}, \emph{\DUrole{n}{b\_eq}\DUrole{p}{:} \DUrole{n}{Optional\DUrole{p}{{[}}numpy.ndarray\DUrole{p}{{]}}} \DUrole{o}{=} \DUrole{default_value}{None}}}{}
Creates the input for the scipy solvers.
\subsubsection*{Example}

Lets give a fist simple example. The constraint
\sphinxcode{\sphinxupquote{max\_capacity\_expansion()}} limits how much each
capacity can be expanded in a given year.

\begin{sphinxVerbatim}[commandchars=\\\{\}]
\PYG{g+gp}{\PYGZgt{}\PYGZgt{}\PYGZgt{} }\PYG{k+kn}{from} \PYG{n+nn}{muse} \PYG{k+kn}{import} \PYG{n}{examples}
\PYG{g+gp}{\PYGZgt{}\PYGZgt{}\PYGZgt{} }\PYG{k+kn}{from} \PYG{n+nn}{muse}\PYG{n+nn}{.}\PYG{n+nn}{quantities} \PYG{k+kn}{import} \PYG{n}{maximum\PYGZus{}production}
\PYG{g+gp}{\PYGZgt{}\PYGZgt{}\PYGZgt{} }\PYG{k+kn}{from} \PYG{n+nn}{muse}\PYG{n+nn}{.}\PYG{n+nn}{timeslices} \PYG{k+kn}{import} \PYG{n}{convert\PYGZus{}timeslice}
\PYG{g+gp}{\PYGZgt{}\PYGZgt{}\PYGZgt{} }\PYG{k+kn}{from} \PYG{n+nn}{muse} \PYG{k+kn}{import} \PYG{n}{constraints} \PYG{k}{as} \PYG{n}{cs}
\PYG{g+gp}{\PYGZgt{}\PYGZgt{}\PYGZgt{} }\PYG{n}{res} \PYG{o}{=} \PYG{n}{examples}\PYG{o}{.}\PYG{n}{sector}\PYG{p}{(}\PYG{l+s+s2}{\PYGZdq{}}\PYG{l+s+s2}{residential}\PYG{l+s+s2}{\PYGZdq{}}\PYG{p}{,} \PYG{n}{model}\PYG{o}{=}\PYG{l+s+s2}{\PYGZdq{}}\PYG{l+s+s2}{medium}\PYG{l+s+s2}{\PYGZdq{}}\PYG{p}{)}
\PYG{g+gp}{\PYGZgt{}\PYGZgt{}\PYGZgt{} }\PYG{n}{market} \PYG{o}{=} \PYG{n}{examples}\PYG{o}{.}\PYG{n}{residential\PYGZus{}market}\PYG{p}{(}\PYG{l+s+s2}{\PYGZdq{}}\PYG{l+s+s2}{medium}\PYG{l+s+s2}{\PYGZdq{}}\PYG{p}{)}
\PYG{g+gp}{\PYGZgt{}\PYGZgt{}\PYGZgt{} }\PYG{n}{search} \PYG{o}{=} \PYG{n}{examples}\PYG{o}{.}\PYG{n}{search\PYGZus{}space}\PYG{p}{(}\PYG{l+s+s2}{\PYGZdq{}}\PYG{l+s+s2}{residential}\PYG{l+s+s2}{\PYGZdq{}}\PYG{p}{,} \PYG{n}{model}\PYG{o}{=}\PYG{l+s+s2}{\PYGZdq{}}\PYG{l+s+s2}{medium}\PYG{l+s+s2}{\PYGZdq{}}\PYG{p}{)}
\PYG{g+gp}{\PYGZgt{}\PYGZgt{}\PYGZgt{} }\PYG{n}{assets} \PYG{o}{=} \PYG{n+nb}{next}\PYG{p}{(}\PYG{n}{a}\PYG{o}{.}\PYG{n}{assets} \PYG{k}{for} \PYG{n}{a} \PYG{o+ow}{in} \PYG{n}{res}\PYG{o}{.}\PYG{n}{agents} \PYG{k}{if} \PYG{n}{a}\PYG{o}{.}\PYG{n}{category} \PYG{o}{==} \PYG{l+s+s2}{\PYGZdq{}}\PYG{l+s+s2}{retrofit}\PYG{l+s+s2}{\PYGZdq{}}\PYG{p}{)}
\PYG{g+gp}{\PYGZgt{}\PYGZgt{}\PYGZgt{} }\PYG{n}{market\PYGZus{}demand} \PYG{o}{=}  \PYG{l+m+mf}{0.8} \PYG{o}{*} \PYG{n}{maximum\PYGZus{}production}\PYG{p}{(}
\PYG{g+gp}{... }    \PYG{n}{res}\PYG{o}{.}\PYG{n}{technologies}\PYG{o}{.}\PYG{n}{interp}\PYG{p}{(}\PYG{n}{year}\PYG{o}{=}\PYG{l+m+mi}{2025}\PYG{p}{)}\PYG{p}{,}
\PYG{g+gp}{... }    \PYG{n}{convert\PYGZus{}timeslice}\PYG{p}{(}
\PYG{g+gp}{... }        \PYG{n}{assets}\PYG{o}{.}\PYG{n}{capacity}\PYG{o}{.}\PYG{n}{sel}\PYG{p}{(}\PYG{n}{year}\PYG{o}{=}\PYG{l+m+mi}{2025}\PYG{p}{)}\PYG{o}{.}\PYG{n}{groupby}\PYG{p}{(}\PYG{l+s+s2}{\PYGZdq{}}\PYG{l+s+s2}{technology}\PYG{l+s+s2}{\PYGZdq{}}\PYG{p}{)}\PYG{o}{.}\PYG{n}{sum}\PYG{p}{(}\PYG{l+s+s2}{\PYGZdq{}}\PYG{l+s+s2}{asset}\PYG{l+s+s2}{\PYGZdq{}}\PYG{p}{)}\PYG{p}{,}
\PYG{g+gp}{... }        \PYG{n}{market}\PYG{o}{.}\PYG{n}{timeslice}\PYG{p}{,}
\PYG{g+gp}{... }    \PYG{p}{)}\PYG{p}{,}
\PYG{g+gp}{... }\PYG{p}{)}\PYG{o}{.}\PYG{n}{rename}\PYG{p}{(}\PYG{n}{technology}\PYG{o}{=}\PYG{l+s+s2}{\PYGZdq{}}\PYG{l+s+s2}{asset}\PYG{l+s+s2}{\PYGZdq{}}\PYG{p}{)}
\PYG{g+gp}{\PYGZgt{}\PYGZgt{}\PYGZgt{} }\PYG{n}{costs} \PYG{o}{=} \PYG{n}{search} \PYG{o}{*} \PYG{n}{np}\PYG{o}{.}\PYG{n}{arange}\PYG{p}{(}\PYG{n}{np}\PYG{o}{.}\PYG{n}{prod}\PYG{p}{(}\PYG{n}{search}\PYG{o}{.}\PYG{n}{shape}\PYG{p}{)}\PYG{p}{)}\PYG{o}{.}\PYG{n}{reshape}\PYG{p}{(}\PYG{n}{search}\PYG{o}{.}\PYG{n}{shape}\PYG{p}{)}
\PYG{g+gp}{\PYGZgt{}\PYGZgt{}\PYGZgt{} }\PYG{n}{constraint} \PYG{o}{=} \PYG{n}{cs}\PYG{o}{.}\PYG{n}{max\PYGZus{}capacity\PYGZus{}expansion}\PYG{p}{(}
\PYG{g+gp}{... }    \PYG{n}{market\PYGZus{}demand}\PYG{p}{,} \PYG{n}{assets}\PYG{p}{,} \PYG{n}{search}\PYG{p}{,} \PYG{n}{market}\PYG{p}{,} \PYG{n}{res}\PYG{o}{.}\PYG{n}{technologies}\PYG{p}{,}
\PYG{g+gp}{... }\PYG{p}{)}
\end{sphinxVerbatim}

The constraint acts over capacity decision variables only:

\begin{sphinxVerbatim}[commandchars=\\\{\}]
\PYG{g+gp}{\PYGZgt{}\PYGZgt{}\PYGZgt{} }\PYG{k}{assert} \PYG{n}{constraint}\PYG{o}{.}\PYG{n}{production}\PYG{o}{.}\PYG{n}{data} \PYG{o}{==} \PYG{n}{np}\PYG{o}{.}\PYG{n}{array}\PYG{p}{(}\PYG{l+m+mi}{0}\PYG{p}{)}
\PYG{g+gp}{\PYGZgt{}\PYGZgt{}\PYGZgt{} }\PYG{k}{assert} \PYG{n+nb}{len}\PYG{p}{(}\PYG{n}{constraint}\PYG{o}{.}\PYG{n}{production}\PYG{o}{.}\PYG{n}{dims}\PYG{p}{)} \PYG{o}{==} \PYG{l+m+mi}{0}
\end{sphinxVerbatim}

It is an upper bound for a straightforward sum over the capacities for a given
technology. The matrix operator is simply the identity:

\begin{sphinxVerbatim}[commandchars=\\\{\}]
\PYG{g+gp}{\PYGZgt{}\PYGZgt{}\PYGZgt{} }\PYG{k}{assert} \PYG{n}{constraint}\PYG{o}{.}\PYG{n}{capacity}\PYG{o}{.}\PYG{n}{data} \PYG{o}{==} \PYG{n}{np}\PYG{o}{.}\PYG{n}{array}\PYG{p}{(}\PYG{l+m+mi}{1}\PYG{p}{)}
\PYG{g+gp}{\PYGZgt{}\PYGZgt{}\PYGZgt{} }\PYG{k}{assert} \PYG{n+nb}{len}\PYG{p}{(}\PYG{n}{constraint}\PYG{o}{.}\PYG{n}{capacity}\PYG{o}{.}\PYG{n}{dims}\PYG{p}{)} \PYG{o}{==} \PYG{l+m+mi}{0}
\end{sphinxVerbatim}

And the upperbound is exanded over the replacement technologies,
but not over the assets. Hence the assets will be summed over in the final
constraint:

\begin{sphinxVerbatim}[commandchars=\\\{\}]
\PYG{g+gp}{\PYGZgt{}\PYGZgt{}\PYGZgt{} }\PYG{k}{assert} \PYG{p}{(}\PYG{n}{constraint}\PYG{o}{.}\PYG{n}{b}\PYG{o}{.}\PYG{n}{data} \PYG{o}{==} \PYG{n}{np}\PYG{o}{.}\PYG{n}{array}\PYG{p}{(}\PYG{p}{[}\PYG{l+m+mf}{500.0}\PYG{p}{,} \PYG{l+m+mf}{55.0}\PYG{p}{,} \PYG{l+m+mf}{55.0}\PYG{p}{,} \PYG{l+m+mf}{500.0} \PYG{p}{]}\PYG{p}{)}\PYG{p}{)}\PYG{o}{.}\PYG{n}{all}\PYG{p}{(}\PYG{p}{)}
\PYG{g+gp}{\PYGZgt{}\PYGZgt{}\PYGZgt{} }\PYG{k}{assert} \PYG{n+nb}{set}\PYG{p}{(}\PYG{n}{constraint}\PYG{o}{.}\PYG{n}{b}\PYG{o}{.}\PYG{n}{dims}\PYG{p}{)} \PYG{o}{==} \PYG{p}{\PYGZob{}}\PYG{l+s+s2}{\PYGZdq{}}\PYG{l+s+s2}{replacement}\PYG{l+s+s2}{\PYGZdq{}}\PYG{p}{\PYGZcb{}}
\PYG{g+gp}{\PYGZgt{}\PYGZgt{}\PYGZgt{} }\PYG{k}{assert} \PYG{n}{constraint}\PYG{o}{.}\PYG{n}{kind} \PYG{o}{==} \PYG{n}{cs}\PYG{o}{.}\PYG{n}{ConstraintKind}\PYG{o}{.}\PYG{n}{UPPER\PYGZus{}BOUND}
\end{sphinxVerbatim}

As shown above, it does not bind the production decision variables. Hence,
production is zero. The matrix operator for the capacity is simply the identity.
Hence it can be inputed as the dimensionless scalar 1. The upper bound is simply
the maximum for replacement technology (and region, if that particular dimension
exists in the problem).

The lp problem then becomes:

\begin{sphinxVerbatim}[commandchars=\\\{\}]
\PYG{g+gp}{\PYGZgt{}\PYGZgt{}\PYGZgt{} }\PYG{n}{technologies} \PYG{o}{=} \PYG{n}{res}\PYG{o}{.}\PYG{n}{technologies}\PYG{o}{.}\PYG{n}{interp}\PYG{p}{(}\PYG{n}{year}\PYG{o}{=}\PYG{n}{market}\PYG{o}{.}\PYG{n}{year}\PYG{o}{.}\PYG{n}{min}\PYG{p}{(}\PYG{p}{)} \PYG{o}{+} \PYG{l+m+mi}{5}\PYG{p}{)}
\PYG{g+gp}{\PYGZgt{}\PYGZgt{}\PYGZgt{} }\PYG{n}{inputs} \PYG{o}{=} \PYG{n}{cs}\PYG{o}{.}\PYG{n}{ScipyAdapter}\PYG{o}{.}\PYG{n}{factory}\PYG{p}{(}
\PYG{g+gp}{... }    \PYG{n}{technologies}\PYG{p}{,} \PYG{n}{costs}\PYG{p}{,} \PYG{n}{market}\PYG{o}{.}\PYG{n}{timeslice}\PYG{p}{,} \PYG{n}{constraint}
\PYG{g+gp}{... }\PYG{p}{)}
\end{sphinxVerbatim}

The decision variables are always constrained between zero and infinity:

\begin{sphinxVerbatim}[commandchars=\\\{\}]
\PYG{g+gp}{\PYGZgt{}\PYGZgt{}\PYGZgt{} }\PYG{k}{assert} \PYG{n}{inputs}\PYG{o}{.}\PYG{n}{bounds} \PYG{o}{==} \PYG{p}{(}\PYG{l+m+mi}{0}\PYG{p}{,} \PYG{n}{np}\PYG{o}{.}\PYG{n}{inf}\PYG{p}{)}
\end{sphinxVerbatim}

The problem is an upper\sphinxhyphen{}bound one. There are no equality constraints:

\begin{sphinxVerbatim}[commandchars=\\\{\}]
\PYG{g+gp}{\PYGZgt{}\PYGZgt{}\PYGZgt{} }\PYG{k}{assert} \PYG{n}{inputs}\PYG{o}{.}\PYG{n}{A\PYGZus{}eq} \PYG{o+ow}{is} \PYG{k+kc}{None}
\PYG{g+gp}{\PYGZgt{}\PYGZgt{}\PYGZgt{} }\PYG{k}{assert} \PYG{n}{inputs}\PYG{o}{.}\PYG{n}{b\PYGZus{}eq} \PYG{o+ow}{is} \PYG{k+kc}{None}
\end{sphinxVerbatim}

The upper bound matrix and vector, and the costs are consistent in their
dimensions:

\begin{sphinxVerbatim}[commandchars=\\\{\}]
\PYG{g+gp}{\PYGZgt{}\PYGZgt{}\PYGZgt{} }\PYG{k}{assert} \PYG{n}{inputs}\PYG{o}{.}\PYG{n}{c}\PYG{o}{.}\PYG{n}{ndim} \PYG{o}{==} \PYG{l+m+mi}{1}
\PYG{g+gp}{\PYGZgt{}\PYGZgt{}\PYGZgt{} }\PYG{k}{assert} \PYG{n}{inputs}\PYG{o}{.}\PYG{n}{b\PYGZus{}ub}\PYG{o}{.}\PYG{n}{ndim} \PYG{o}{==} \PYG{l+m+mi}{1}
\PYG{g+gp}{\PYGZgt{}\PYGZgt{}\PYGZgt{} }\PYG{k}{assert} \PYG{n}{inputs}\PYG{o}{.}\PYG{n}{A\PYGZus{}ub}\PYG{o}{.}\PYG{n}{ndim} \PYG{o}{==} \PYG{l+m+mi}{2}
\PYG{g+gp}{\PYGZgt{}\PYGZgt{}\PYGZgt{} }\PYG{k}{assert} \PYG{n}{inputs}\PYG{o}{.}\PYG{n}{b\PYGZus{}ub}\PYG{o}{.}\PYG{n}{size} \PYG{o}{==} \PYG{n}{inputs}\PYG{o}{.}\PYG{n}{A\PYGZus{}ub}\PYG{o}{.}\PYG{n}{shape}\PYG{p}{[}\PYG{l+m+mi}{0}\PYG{p}{]}
\PYG{g+gp}{\PYGZgt{}\PYGZgt{}\PYGZgt{} }\PYG{k}{assert} \PYG{n}{inputs}\PYG{o}{.}\PYG{n}{c}\PYG{o}{.}\PYG{n}{size} \PYG{o}{==} \PYG{n}{inputs}\PYG{o}{.}\PYG{n}{A\PYGZus{}ub}\PYG{o}{.}\PYG{n}{shape}\PYG{p}{[}\PYG{l+m+mi}{1}\PYG{p}{]}
\PYG{g+gp}{\PYGZgt{}\PYGZgt{}\PYGZgt{} }\PYG{k}{assert} \PYG{n}{inputs}\PYG{o}{.}\PYG{n}{c}\PYG{o}{.}\PYG{n}{ndim} \PYG{o}{==} \PYG{l+m+mi}{1}
\end{sphinxVerbatim}

In practice, \sphinxcode{\sphinxupquote{lp\_costs()}} helps us define the decision
variables (and \sphinxcode{\sphinxupquote{c}}). We can verify that the sizes are consistent:

\begin{sphinxVerbatim}[commandchars=\\\{\}]
\PYG{g+gp}{\PYGZgt{}\PYGZgt{}\PYGZgt{} }\PYG{n}{lpcosts} \PYG{o}{=} \PYG{n}{cs}\PYG{o}{.}\PYG{n}{lp\PYGZus{}costs}\PYG{p}{(}\PYG{n}{technologies}\PYG{p}{,} \PYG{n}{costs}\PYG{p}{,} \PYG{n}{market}\PYG{o}{.}\PYG{n}{timeslice}\PYG{p}{)}
\PYG{g+gp}{\PYGZgt{}\PYGZgt{}\PYGZgt{} }\PYG{n}{capsize} \PYG{o}{=} \PYG{n}{lpcosts}\PYG{o}{.}\PYG{n}{capacity}\PYG{o}{.}\PYG{n}{size}
\PYG{g+gp}{\PYGZgt{}\PYGZgt{}\PYGZgt{} }\PYG{n}{prodsize} \PYG{o}{=} \PYG{n}{lpcosts}\PYG{o}{.}\PYG{n}{production}\PYG{o}{.}\PYG{n}{size}
\PYG{g+gp}{\PYGZgt{}\PYGZgt{}\PYGZgt{} }\PYG{k}{assert} \PYG{n}{inputs}\PYG{o}{.}\PYG{n}{c}\PYG{o}{.}\PYG{n}{size} \PYG{o}{==} \PYG{n}{capsize} \PYG{o}{+} \PYG{n}{prodsize}
\end{sphinxVerbatim}

The upper bound itself is over each replacement technology:

\begin{sphinxVerbatim}[commandchars=\\\{\}]
\PYG{g+gp}{\PYGZgt{}\PYGZgt{}\PYGZgt{} }\PYG{k}{assert} \PYG{n}{inputs}\PYG{o}{.}\PYG{n}{b\PYGZus{}ub}\PYG{o}{.}\PYG{n}{size} \PYG{o}{==} \PYG{n}{lpcosts}\PYG{o}{.}\PYG{n}{replacement}\PYG{o}{.}\PYG{n}{size}
\end{sphinxVerbatim}

The production decision variables are not involved:

\begin{sphinxVerbatim}[commandchars=\\\{\}]
\PYG{g+gp}{\PYGZgt{}\PYGZgt{}\PYGZgt{} }\PYG{k+kn}{from} \PYG{n+nn}{pytest} \PYG{k+kn}{import} \PYG{n}{approx}
\PYG{g+gp}{\PYGZgt{}\PYGZgt{}\PYGZgt{} }\PYG{k}{assert} \PYG{n}{inputs}\PYG{o}{.}\PYG{n}{A\PYGZus{}ub}\PYG{p}{[}\PYG{p}{:}\PYG{p}{,} \PYG{n}{capsize}\PYG{p}{:}\PYG{p}{]} \PYG{o}{==} \PYG{n}{approx}\PYG{p}{(}\PYG{l+m+mi}{0}\PYG{p}{)}
\end{sphinxVerbatim}

The matrix for the capacity decision variables is a sum over assets for a given
replacement technology. Hence, each row is constituted of zeros and ones and
sums to the number of assets:

\begin{sphinxVerbatim}[commandchars=\\\{\}]
\PYG{g+gp}{\PYGZgt{}\PYGZgt{}\PYGZgt{} }\PYG{k}{assert} \PYG{n}{inputs}\PYG{o}{.}\PYG{n}{A\PYGZus{}ub}\PYG{p}{[}\PYG{p}{:}\PYG{p}{,} \PYG{p}{:}\PYG{n}{capsize}\PYG{p}{]}\PYG{o}{.}\PYG{n}{sum}\PYG{p}{(}\PYG{n}{axis}\PYG{o}{=}\PYG{l+m+mi}{1}\PYG{p}{)} \PYG{o}{==} \PYG{n}{approx}\PYG{p}{(}\PYG{n}{lpcosts}\PYG{o}{.}\PYG{n}{asset}\PYG{o}{.}\PYG{n}{size}\PYG{p}{)}
\PYG{g+gp}{\PYGZgt{}\PYGZgt{}\PYGZgt{} }\PYG{k}{assert} \PYG{n+nb}{set}\PYG{p}{(}\PYG{n}{inputs}\PYG{o}{.}\PYG{n}{A\PYGZus{}ub}\PYG{p}{[}\PYG{p}{:}\PYG{p}{,} \PYG{p}{:}\PYG{n}{capsize}\PYG{p}{]}\PYG{o}{.}\PYG{n}{flatten}\PYG{p}{(}\PYG{p}{)}\PYG{p}{)} \PYG{o}{==} \PYG{p}{\PYGZob{}}\PYG{l+m+mf}{0.0}\PYG{p}{,} \PYG{l+m+mf}{1.0}\PYG{p}{\PYGZcb{}}
\end{sphinxVerbatim}

\end{fulllineitems}

\index{demand() (in module muse.constraints)@\spxentry{demand()}\spxextra{in module muse.constraints}}

\begin{fulllineitems}
\phantomsection\label{\detokenize{api:muse.constraints.demand}}\pysiglinewithargsret{\sphinxcode{\sphinxupquote{muse.constraints.}}\sphinxbfcode{\sphinxupquote{demand}}}{\emph{\DUrole{n}{demand}\DUrole{p}{:} \DUrole{n}{xarray.core.dataarray.DataArray}}, \emph{\DUrole{n}{assets}\DUrole{p}{:} \DUrole{n}{xarray.core.dataset.Dataset}}, \emph{\DUrole{n}{search\_space}\DUrole{p}{:} \DUrole{n}{xarray.core.dataarray.DataArray}}, \emph{\DUrole{n}{market}\DUrole{p}{:} \DUrole{n}{xarray.core.dataset.Dataset}}, \emph{\DUrole{n}{technologies}\DUrole{p}{:} \DUrole{n}{xarray.core.dataset.Dataset}}, \emph{\DUrole{n}{year}\DUrole{p}{:} \DUrole{n}{Optional\DUrole{p}{{[}}int\DUrole{p}{{]}}} \DUrole{o}{=} \DUrole{default_value}{None}}, \emph{\DUrole{n}{forecast}\DUrole{p}{:} \DUrole{n}{int} \DUrole{o}{=} \DUrole{default_value}{5}}, \emph{\DUrole{n}{interpolation}\DUrole{p}{:} \DUrole{n}{str} \DUrole{o}{=} \DUrole{default_value}{\textquotesingle{}linear\textquotesingle{}}}}{{ $\rightarrow$ xarray.core.dataset.Dataset}}
Constraints production to meet demand.

\end{fulllineitems}

\index{factory() (in module muse.constraints)@\spxentry{factory()}\spxextra{in module muse.constraints}}

\begin{fulllineitems}
\phantomsection\label{\detokenize{api:muse.constraints.factory}}\pysiglinewithargsret{\sphinxcode{\sphinxupquote{muse.constraints.}}\sphinxbfcode{\sphinxupquote{factory}}}{\emph{\DUrole{n}{settings}\DUrole{p}{:} \DUrole{n}{Optional\DUrole{p}{{[}}Union\DUrole{p}{{[}}str\DUrole{p}{, }Mapping\DUrole{p}{, }Sequence\DUrole{p}{{[}}str\DUrole{p}{{]}}\DUrole{p}{, }Sequence\DUrole{p}{{[}}Union\DUrole{p}{{[}}str\DUrole{p}{, }Mapping\DUrole{p}{{]}}\DUrole{p}{{]}}\DUrole{p}{{]}}\DUrole{p}{{]}}} \DUrole{o}{=} \DUrole{default_value}{None}}}{{ $\rightarrow$ Callable}}
Creates a list of constraints from standard settings.

The standard settings can be a string naming the constraint, a dictionary including
at least “name”, or a list of strings and dictionaries.

\end{fulllineitems}

\index{lp\_constraint() (in module muse.constraints)@\spxentry{lp\_constraint()}\spxextra{in module muse.constraints}}

\begin{fulllineitems}
\phantomsection\label{\detokenize{api:muse.constraints.lp_constraint}}\pysiglinewithargsret{\sphinxcode{\sphinxupquote{muse.constraints.}}\sphinxbfcode{\sphinxupquote{lp\_constraint}}}{\emph{\DUrole{n}{constraint}\DUrole{p}{:} \DUrole{n}{xarray.core.dataset.Dataset}}, \emph{\DUrole{n}{lpcosts}\DUrole{p}{:} \DUrole{n}{xarray.core.dataset.Dataset}}}{{ $\rightarrow$ xarray.core.dataset.Dataset}}
Transforms the constraint to LP data.

The goal is to create from \sphinxcode{\sphinxupquote{lpcosts.capacity}}, \sphinxcode{\sphinxupquote{constraint.capacity}}, and
\sphinxcode{\sphinxupquote{constraint.b}} a 2d\sphinxhyphen{}matrix \sphinxcode{\sphinxupquote{constraint}} vs \sphinxcode{\sphinxupquote{decision variables}}.
\begin{enumerate}
\sphinxsetlistlabels{\arabic}{enumi}{enumii}{}{.}%
\item {} \begin{description}
\item[{The dimensions of \sphinxcode{\sphinxupquote{constraint.b}} are the constraint dimensions. They are}] \leavevmode
renamed \sphinxcode{\sphinxupquote{"c(xxx)"}}.

\end{description}

\item {} \begin{description}
\item[{The dimensions of \sphinxcode{\sphinxupquote{lpcosts}} are the decision\sphinxhyphen{}variable dimensions. They are}] \leavevmode
renamed \sphinxcode{\sphinxupquote{"d(xxx)"}}.

\end{description}

\item {} \begin{description}
\item[{\sphinxcode{\sphinxupquote{set(b.dims).intersection(lpcosts.xxx.dims)}} are diagonal}] \leavevmode
in constraint dimensions and decision variables dimension, with \sphinxcode{\sphinxupquote{xxx}} the
capacity or the production

\end{description}

\item {} \begin{description}
\item[{\sphinxcode{\sphinxupquote{set(constraint.xxx.dims) \sphinxhyphen{} set(lpcosts.xxx.dims) \sphinxhyphen{} set(b.dims)}} are reduced by}] \leavevmode
summation, with \sphinxcode{\sphinxupquote{xxx}} the capacity or the production

\end{description}

\item {} \begin{description}
\item[{\sphinxcode{\sphinxupquote{set(lpcosts.xxx.dims) \sphinxhyphen{} set(constraint.xxx.dims) \sphinxhyphen{} set(b.dims)}} are added for}] \leavevmode
expansion, with \sphinxcode{\sphinxupquote{xxx}} the capacity or the production

\end{description}

\end{enumerate}

See \sphinxcode{\sphinxupquote{muse.constraints.lp\_constraint\_matrix()}} for a more detailed explanation
of the transformations applied here.

\end{fulllineitems}

\index{lp\_constraint\_matrix() (in module muse.constraints)@\spxentry{lp\_constraint\_matrix()}\spxextra{in module muse.constraints}}

\begin{fulllineitems}
\phantomsection\label{\detokenize{api:muse.constraints.lp_constraint_matrix}}\pysiglinewithargsret{\sphinxcode{\sphinxupquote{muse.constraints.}}\sphinxbfcode{\sphinxupquote{lp\_constraint\_matrix}}}{\emph{\DUrole{n}{b}\DUrole{p}{:} \DUrole{n}{xarray.core.dataarray.DataArray}}, \emph{\DUrole{n}{constraint}\DUrole{p}{:} \DUrole{n}{xarray.core.dataarray.DataArray}}, \emph{\DUrole{n}{lpcosts}\DUrole{p}{:} \DUrole{n}{xarray.core.dataarray.DataArray}}}{}
Transforms one constraint block into an lp matrix.

The goal is to create from \sphinxcode{\sphinxupquote{lpcosts}}, \sphinxcode{\sphinxupquote{constraint}}, and \sphinxcode{\sphinxupquote{b}} a 2d\sphinxhyphen{}matrix of
constraints vs decision variables.
\begin{quote}
\begin{enumerate}
\sphinxsetlistlabels{\arabic}{enumi}{enumii}{}{.}%
\item {} \begin{description}
\item[{The dimensions of \sphinxcode{\sphinxupquote{b}} are the constraint dimensions. They are renamed}] \leavevmode
\sphinxcode{\sphinxupquote{"c(xxx)"}}.

\end{description}

\item {} \begin{description}
\item[{The dimensions of \sphinxcode{\sphinxupquote{lpcosts}} are the decision\sphinxhyphen{}variable dimensions. They are}] \leavevmode
renamed \sphinxcode{\sphinxupquote{"d(xxx)"}}.

\end{description}

\item {} \begin{description}
\item[{\sphinxcode{\sphinxupquote{set(b.dims).intersection(lpcosts.dims)}} are diagonal}] \leavevmode
in constraint dimensions and decision variables dimension

\end{description}

\item {} \begin{description}
\item[{\sphinxcode{\sphinxupquote{set(constraint.dims) \sphinxhyphen{} set(lpcosts.dims) \sphinxhyphen{} set(b.dims)}} are reduced by}] \leavevmode
summation

\end{description}

\item {} \begin{description}
\item[{\sphinxcode{\sphinxupquote{set(lpcosts.dims) \sphinxhyphen{} set(constraint.dims) \sphinxhyphen{} set(b.dims)}} are added for}] \leavevmode
expansion

\end{description}

\item {} \begin{description}
\item[{\sphinxcode{\sphinxupquote{set(b.dims) \sphinxhyphen{} set(constraint.dims) \sphinxhyphen{} set(lpcosts.dims)}} are added for}] \leavevmode
expansion. Such dimensions only make sense if they consist of one point.

\end{description}

\end{enumerate}

The result is the constraint matrix, expanded, reduced and diagonalized for the
conditions above.

Example:
\begin{quote}

Lets first setup a constraint and a cost matrix:

\begin{sphinxVerbatim}[commandchars=\\\{\}]
\PYG{g+gp}{\PYGZgt{}\PYGZgt{}\PYGZgt{} }\PYG{k+kn}{from} \PYG{n+nn}{muse} \PYG{k+kn}{import} \PYG{n}{examples}
\PYG{g+gp}{\PYGZgt{}\PYGZgt{}\PYGZgt{} }\PYG{k+kn}{from} \PYG{n+nn}{muse} \PYG{k+kn}{import} \PYG{n}{constraints} \PYG{k}{as} \PYG{n}{cs}
\PYG{g+gp}{\PYGZgt{}\PYGZgt{}\PYGZgt{} }\PYG{n}{res} \PYG{o}{=} \PYG{n}{examples}\PYG{o}{.}\PYG{n}{sector}\PYG{p}{(}\PYG{l+s+s2}{\PYGZdq{}}\PYG{l+s+s2}{residential}\PYG{l+s+s2}{\PYGZdq{}}\PYG{p}{,} \PYG{n}{model}\PYG{o}{=}\PYG{l+s+s2}{\PYGZdq{}}\PYG{l+s+s2}{medium}\PYG{l+s+s2}{\PYGZdq{}}\PYG{p}{)}
\PYG{g+gp}{\PYGZgt{}\PYGZgt{}\PYGZgt{} }\PYG{n}{technologies} \PYG{o}{=} \PYG{n}{res}\PYG{o}{.}\PYG{n}{technologies}
\PYG{g+gp}{\PYGZgt{}\PYGZgt{}\PYGZgt{} }\PYG{n}{market} \PYG{o}{=} \PYG{n}{examples}\PYG{o}{.}\PYG{n}{residential\PYGZus{}market}\PYG{p}{(}\PYG{l+s+s2}{\PYGZdq{}}\PYG{l+s+s2}{medium}\PYG{l+s+s2}{\PYGZdq{}}\PYG{p}{)}
\PYG{g+gp}{\PYGZgt{}\PYGZgt{}\PYGZgt{} }\PYG{n}{search} \PYG{o}{=} \PYG{n}{examples}\PYG{o}{.}\PYG{n}{search\PYGZus{}space}\PYG{p}{(}\PYG{l+s+s2}{\PYGZdq{}}\PYG{l+s+s2}{residential}\PYG{l+s+s2}{\PYGZdq{}}\PYG{p}{,} \PYG{n}{model}\PYG{o}{=}\PYG{l+s+s2}{\PYGZdq{}}\PYG{l+s+s2}{medium}\PYG{l+s+s2}{\PYGZdq{}}\PYG{p}{)}
\PYG{g+gp}{\PYGZgt{}\PYGZgt{}\PYGZgt{} }\PYG{n}{assets} \PYG{o}{=} \PYG{n+nb}{next}\PYG{p}{(}\PYG{n}{a}\PYG{o}{.}\PYG{n}{assets} \PYG{k}{for} \PYG{n}{a} \PYG{o+ow}{in} \PYG{n}{res}\PYG{o}{.}\PYG{n}{agents} \PYG{k}{if} \PYG{n}{a}\PYG{o}{.}\PYG{n}{category} \PYG{o}{==} \PYG{l+s+s2}{\PYGZdq{}}\PYG{l+s+s2}{retrofit}\PYG{l+s+s2}{\PYGZdq{}}\PYG{p}{)}
\PYG{g+gp}{\PYGZgt{}\PYGZgt{}\PYGZgt{} }\PYG{n}{demand} \PYG{o}{=} \PYG{k+kc}{None} \PYG{c+c1}{\PYGZsh{} not used in max production}
\PYG{g+gp}{\PYGZgt{}\PYGZgt{}\PYGZgt{} }\PYG{n}{constraint} \PYG{o}{=} \PYG{n}{cs}\PYG{o}{.}\PYG{n}{max\PYGZus{}production}\PYG{p}{(}\PYG{n}{demand}\PYG{p}{,} \PYG{n}{assets}\PYG{p}{,} \PYG{n}{search}\PYG{p}{,} \PYG{n}{market}\PYG{p}{,} \PYG{n}{technologies}\PYG{p}{)}
\PYG{g+gp}{\PYGZgt{}\PYGZgt{}\PYGZgt{} }\PYG{n}{lpcosts} \PYG{o}{=} \PYG{n}{cs}\PYG{o}{.}\PYG{n}{lp\PYGZus{}costs}\PYG{p}{(}
\PYG{g+gp}{... }    \PYG{p}{(}
\PYG{g+gp}{... }        \PYG{n}{technologies}
\PYG{g+gp}{... }        \PYG{o}{.}\PYG{n}{interp}\PYG{p}{(}\PYG{n}{year}\PYG{o}{=}\PYG{n}{market}\PYG{o}{.}\PYG{n}{year}\PYG{o}{.}\PYG{n}{min}\PYG{p}{(}\PYG{p}{)} \PYG{o}{+} \PYG{l+m+mi}{5}\PYG{p}{)}
\PYG{g+gp}{... }        \PYG{o}{.}\PYG{n}{drop\PYGZus{}vars}\PYG{p}{(}\PYG{l+s+s2}{\PYGZdq{}}\PYG{l+s+s2}{year}\PYG{l+s+s2}{\PYGZdq{}}\PYG{p}{)}
\PYG{g+gp}{... }        \PYG{o}{.}\PYG{n}{sel}\PYG{p}{(}\PYG{n}{region}\PYG{o}{=}\PYG{n}{assets}\PYG{o}{.}\PYG{n}{region}\PYG{p}{)}
\PYG{g+gp}{... }    \PYG{p}{)}\PYG{p}{,}
\PYG{g+gp}{... }    \PYG{n}{costs}\PYG{o}{=}\PYG{n}{search} \PYG{o}{*} \PYG{n}{np}\PYG{o}{.}\PYG{n}{arange}\PYG{p}{(}\PYG{n}{np}\PYG{o}{.}\PYG{n}{prod}\PYG{p}{(}\PYG{n}{search}\PYG{o}{.}\PYG{n}{shape}\PYG{p}{)}\PYG{p}{)}\PYG{o}{.}\PYG{n}{reshape}\PYG{p}{(}\PYG{n}{search}\PYG{o}{.}\PYG{n}{shape}\PYG{p}{)}\PYG{p}{,}
\PYG{g+gp}{... }    \PYG{n}{timeslices}\PYG{o}{=}\PYG{n}{market}\PYG{o}{.}\PYG{n}{timeslice}\PYG{p}{,}
\PYG{g+gp}{... }\PYG{p}{)}
\end{sphinxVerbatim}

For a simple example, we can first check the case where b is scalar. The result
ought to be a single row of a matrix, or a vector with only decision variables:

\begin{sphinxVerbatim}[commandchars=\\\{\}]
\PYG{g+gp}{\PYGZgt{}\PYGZgt{}\PYGZgt{} }\PYG{k+kn}{from} \PYG{n+nn}{pytest} \PYG{k+kn}{import} \PYG{n}{approx}
\PYG{g+gp}{\PYGZgt{}\PYGZgt{}\PYGZgt{} }\PYG{n}{result} \PYG{o}{=} \PYG{n}{cs}\PYG{o}{.}\PYG{n}{lp\PYGZus{}constraint\PYGZus{}matrix}\PYG{p}{(}
\PYG{g+gp}{... }    \PYG{n}{xr}\PYG{o}{.}\PYG{n}{DataArray}\PYG{p}{(}\PYG{l+m+mi}{1}\PYG{p}{)}\PYG{p}{,} \PYG{n}{constraint}\PYG{o}{.}\PYG{n}{capacity}\PYG{p}{,} \PYG{n}{lpcosts}\PYG{o}{.}\PYG{n}{capacity}
\PYG{g+gp}{... }\PYG{p}{)}
\PYG{g+gp}{\PYGZgt{}\PYGZgt{}\PYGZgt{} }\PYG{k}{assert} \PYG{n}{result}\PYG{o}{.}\PYG{n}{values} \PYG{o}{==} \PYG{n}{approx}\PYG{p}{(}\PYG{o}{\PYGZhy{}}\PYG{l+m+mi}{1}\PYG{p}{)}
\PYG{g+gp}{\PYGZgt{}\PYGZgt{}\PYGZgt{} }\PYG{k}{assert} \PYG{n+nb}{set}\PYG{p}{(}\PYG{n}{result}\PYG{o}{.}\PYG{n}{dims}\PYG{p}{)} \PYG{o}{==} \PYG{p}{\PYGZob{}}\PYG{l+s+sa}{f}\PYG{l+s+s2}{\PYGZdq{}}\PYG{l+s+s2}{d(}\PYG{l+s+si}{\PYGZob{}}\PYG{n}{x}\PYG{l+s+si}{\PYGZcb{}}\PYG{l+s+s2}{)}\PYG{l+s+s2}{\PYGZdq{}} \PYG{k}{for} \PYG{n}{x} \PYG{o+ow}{in} \PYG{n}{lpcosts}\PYG{o}{.}\PYG{n}{capacity}\PYG{o}{.}\PYG{n}{dims}\PYG{p}{\PYGZcb{}}
\PYG{g+gp}{\PYGZgt{}\PYGZgt{}\PYGZgt{} }\PYG{n}{result} \PYG{o}{=} \PYG{n}{cs}\PYG{o}{.}\PYG{n}{lp\PYGZus{}constraint\PYGZus{}matrix}\PYG{p}{(}
\PYG{g+gp}{... }    \PYG{n}{xr}\PYG{o}{.}\PYG{n}{DataArray}\PYG{p}{(}\PYG{l+m+mi}{1}\PYG{p}{)}\PYG{p}{,} \PYG{n}{constraint}\PYG{o}{.}\PYG{n}{production}\PYG{p}{,} \PYG{n}{lpcosts}\PYG{o}{.}\PYG{n}{production}
\PYG{g+gp}{... }\PYG{p}{)}
\PYG{g+gp}{\PYGZgt{}\PYGZgt{}\PYGZgt{} }\PYG{k}{assert} \PYG{n+nb}{set}\PYG{p}{(}\PYG{n}{result}\PYG{o}{.}\PYG{n}{dims}\PYG{p}{)} \PYG{o}{==} \PYG{p}{\PYGZob{}}\PYG{l+s+sa}{f}\PYG{l+s+s2}{\PYGZdq{}}\PYG{l+s+s2}{d(}\PYG{l+s+si}{\PYGZob{}}\PYG{n}{x}\PYG{l+s+si}{\PYGZcb{}}\PYG{l+s+s2}{)}\PYG{l+s+s2}{\PYGZdq{}} \PYG{k}{for} \PYG{n}{x} \PYG{o+ow}{in} \PYG{n}{lpcosts}\PYG{o}{.}\PYG{n}{production}\PYG{o}{.}\PYG{n}{dims}\PYG{p}{\PYGZcb{}}
\PYG{g+gp}{\PYGZgt{}\PYGZgt{}\PYGZgt{} }\PYG{k}{assert} \PYG{n}{result}\PYG{o}{.}\PYG{n}{values} \PYG{o}{==} \PYG{n}{approx}\PYG{p}{(}\PYG{l+m+mi}{1}\PYG{p}{)}
\end{sphinxVerbatim}

As expected, the capacity vector is 1, whereas the production vector is \sphinxhyphen{}1.
These are the values the \sphinxcode{\sphinxupquote{max\_production()}} is set up
to create.

Now, let’s check the case where \sphinxcode{\sphinxupquote{b}} is the one from the
\sphinxcode{\sphinxupquote{max\_production()}} constraint. In that case, all the
dimensions should end up as constraint dimensions: the production for each
timeslice, region, asset, and replacement technology should not outstrip the
capacity assigned for the asset and replacement technology.

\begin{sphinxVerbatim}[commandchars=\\\{\}]
\PYG{g+gp}{\PYGZgt{}\PYGZgt{}\PYGZgt{} }\PYG{n}{result} \PYG{o}{=} \PYG{n}{cs}\PYG{o}{.}\PYG{n}{lp\PYGZus{}constraint\PYGZus{}matrix}\PYG{p}{(}
\PYG{g+gp}{... }    \PYG{n}{constraint}\PYG{o}{.}\PYG{n}{b}\PYG{p}{,} \PYG{n}{constraint}\PYG{o}{.}\PYG{n}{capacity}\PYG{p}{,} \PYG{n}{lpcosts}\PYG{o}{.}\PYG{n}{capacity}
\PYG{g+gp}{... }\PYG{p}{)}
\PYG{g+gp}{\PYGZgt{}\PYGZgt{}\PYGZgt{} }\PYG{n}{decision\PYGZus{}dims} \PYG{o}{=} \PYG{p}{\PYGZob{}}\PYG{l+s+sa}{f}\PYG{l+s+s2}{\PYGZdq{}}\PYG{l+s+s2}{d(}\PYG{l+s+si}{\PYGZob{}}\PYG{n}{x}\PYG{l+s+si}{\PYGZcb{}}\PYG{l+s+s2}{)}\PYG{l+s+s2}{\PYGZdq{}} \PYG{k}{for} \PYG{n}{x} \PYG{o+ow}{in} \PYG{n}{lpcosts}\PYG{o}{.}\PYG{n}{capacity}\PYG{o}{.}\PYG{n}{dims}\PYG{p}{\PYGZcb{}}
\PYG{g+gp}{\PYGZgt{}\PYGZgt{}\PYGZgt{} }\PYG{n}{constraint\PYGZus{}dims} \PYG{o}{=} \PYG{p}{\PYGZob{}}
\PYG{g+gp}{... }    \PYG{l+s+sa}{f}\PYG{l+s+s2}{\PYGZdq{}}\PYG{l+s+s2}{c(}\PYG{l+s+si}{\PYGZob{}}\PYG{n}{x}\PYG{l+s+si}{\PYGZcb{}}\PYG{l+s+s2}{)}\PYG{l+s+s2}{\PYGZdq{}} \PYG{k}{for} \PYG{n}{x} \PYG{o+ow}{in} \PYG{n+nb}{set}\PYG{p}{(}\PYG{n}{lpcosts}\PYG{o}{.}\PYG{n}{production}\PYG{o}{.}\PYG{n}{dims}\PYG{p}{)}\PYG{o}{.}\PYG{n}{union}\PYG{p}{(}\PYG{n}{constraint}\PYG{o}{.}\PYG{n}{b}\PYG{o}{.}\PYG{n}{dims}\PYG{p}{)}
\PYG{g+gp}{... }\PYG{p}{\PYGZcb{}}
\PYG{g+gp}{\PYGZgt{}\PYGZgt{}\PYGZgt{} }\PYG{k}{assert} \PYG{n+nb}{set}\PYG{p}{(}\PYG{n}{result}\PYG{o}{.}\PYG{n}{dims}\PYG{p}{)} \PYG{o}{==} \PYG{n}{decision\PYGZus{}dims}\PYG{o}{.}\PYG{n}{union}\PYG{p}{(}\PYG{n}{constraint\PYGZus{}dims}\PYG{p}{)}
\end{sphinxVerbatim}

The \sphinxcode{\sphinxupquote{max\_production()}} constraint on the production
side is the identy matrix with a factor \(-1\). We can easily check this
by stacking the decision and constraint dimensions in the result:

\begin{sphinxVerbatim}[commandchars=\\\{\}]
\PYG{g+gp}{\PYGZgt{}\PYGZgt{}\PYGZgt{} }\PYG{n}{result} \PYG{o}{=} \PYG{n}{cs}\PYG{o}{.}\PYG{n}{lp\PYGZus{}constraint\PYGZus{}matrix}\PYG{p}{(}
\PYG{g+gp}{... }    \PYG{n}{constraint}\PYG{o}{.}\PYG{n}{b}\PYG{p}{,} \PYG{n}{constraint}\PYG{o}{.}\PYG{n}{production}\PYG{p}{,} \PYG{n}{lpcosts}\PYG{o}{.}\PYG{n}{production}
\PYG{g+gp}{... }\PYG{p}{)}
\PYG{g+gp}{\PYGZgt{}\PYGZgt{}\PYGZgt{} }\PYG{n}{decision\PYGZus{}dims} \PYG{o}{=} \PYG{p}{\PYGZob{}}\PYG{l+s+sa}{f}\PYG{l+s+s2}{\PYGZdq{}}\PYG{l+s+s2}{d(}\PYG{l+s+si}{\PYGZob{}}\PYG{n}{x}\PYG{l+s+si}{\PYGZcb{}}\PYG{l+s+s2}{)}\PYG{l+s+s2}{\PYGZdq{}} \PYG{k}{for} \PYG{n}{x} \PYG{o+ow}{in} \PYG{n}{lpcosts}\PYG{o}{.}\PYG{n}{production}\PYG{o}{.}\PYG{n}{dims}\PYG{p}{\PYGZcb{}}
\PYG{g+gp}{\PYGZgt{}\PYGZgt{}\PYGZgt{} }\PYG{k}{assert} \PYG{n+nb}{set}\PYG{p}{(}\PYG{n}{result}\PYG{o}{.}\PYG{n}{dims}\PYG{p}{)} \PYG{o}{==} \PYG{n}{decision\PYGZus{}dims}\PYG{o}{.}\PYG{n}{union}\PYG{p}{(}\PYG{n}{constraint\PYGZus{}dims}\PYG{p}{)}
\PYG{g+gp}{\PYGZgt{}\PYGZgt{}\PYGZgt{} }\PYG{n}{stacked} \PYG{o}{=} \PYG{n}{result}\PYG{o}{.}\PYG{n}{stack}\PYG{p}{(}\PYG{n}{d}\PYG{o}{=}\PYG{n+nb}{sorted}\PYG{p}{(}\PYG{n}{decision\PYGZus{}dims}\PYG{p}{)}\PYG{p}{,} \PYG{n}{c}\PYG{o}{=}\PYG{n+nb}{sorted}\PYG{p}{(}\PYG{n}{constraint\PYGZus{}dims}\PYG{p}{)}\PYG{p}{)}
\PYG{g+gp}{\PYGZgt{}\PYGZgt{}\PYGZgt{} }\PYG{k}{assert} \PYG{n}{stacked}\PYG{o}{.}\PYG{n}{shape}\PYG{p}{[}\PYG{l+m+mi}{0}\PYG{p}{]} \PYG{o}{==} \PYG{n}{stacked}\PYG{o}{.}\PYG{n}{shape}\PYG{p}{[}\PYG{l+m+mi}{1}\PYG{p}{]}
\PYG{g+gp}{\PYGZgt{}\PYGZgt{}\PYGZgt{} }\PYG{k}{assert} \PYG{n}{stacked}\PYG{o}{.}\PYG{n}{values} \PYG{o}{==} \PYG{n}{approx}\PYG{p}{(}\PYG{n}{np}\PYG{o}{.}\PYG{n}{eye}\PYG{p}{(}\PYG{n}{stacked}\PYG{o}{.}\PYG{n}{shape}\PYG{p}{[}\PYG{l+m+mi}{0}\PYG{p}{]}\PYG{p}{)}\PYG{p}{)}
\end{sphinxVerbatim}
\end{quote}
\end{quote}

\end{fulllineitems}

\index{lp\_costs() (in module muse.constraints)@\spxentry{lp\_costs()}\spxextra{in module muse.constraints}}

\begin{fulllineitems}
\phantomsection\label{\detokenize{api:muse.constraints.lp_costs}}\pysiglinewithargsret{\sphinxcode{\sphinxupquote{muse.constraints.}}\sphinxbfcode{\sphinxupquote{lp\_costs}}}{\emph{\DUrole{n}{technologies}\DUrole{p}{:} \DUrole{n}{xarray.core.dataset.Dataset}}, \emph{\DUrole{n}{costs}\DUrole{p}{:} \DUrole{n}{xarray.core.dataarray.DataArray}}, \emph{\DUrole{n}{timeslices}\DUrole{p}{:} \DUrole{n}{xarray.core.dataarray.DataArray}}}{{ $\rightarrow$ xarray.core.dataset.Dataset}}
Creates costs for solving with scipy’s LP solver.
\subsubsection*{Example}

We can now construct example inputs to the funtion from the sample model. The
costs will be a matrix where each assets has a candidate replacement technology.

\begin{sphinxVerbatim}[commandchars=\\\{\}]
\PYG{g+gp}{\PYGZgt{}\PYGZgt{}\PYGZgt{} }\PYG{k+kn}{from} \PYG{n+nn}{muse} \PYG{k+kn}{import} \PYG{n}{examples}
\PYG{g+gp}{\PYGZgt{}\PYGZgt{}\PYGZgt{} }\PYG{n}{technologies} \PYG{o}{=} \PYG{n}{examples}\PYG{o}{.}\PYG{n}{technodata}\PYG{p}{(}\PYG{l+s+s2}{\PYGZdq{}}\PYG{l+s+s2}{residential}\PYG{l+s+s2}{\PYGZdq{}}\PYG{p}{,} \PYG{n}{model}\PYG{o}{=}\PYG{l+s+s2}{\PYGZdq{}}\PYG{l+s+s2}{medium}\PYG{l+s+s2}{\PYGZdq{}}\PYG{p}{)}
\PYG{g+gp}{\PYGZgt{}\PYGZgt{}\PYGZgt{} }\PYG{n}{search\PYGZus{}space} \PYG{o}{=} \PYG{n}{examples}\PYG{o}{.}\PYG{n}{search\PYGZus{}space}\PYG{p}{(}\PYG{l+s+s2}{\PYGZdq{}}\PYG{l+s+s2}{residential}\PYG{l+s+s2}{\PYGZdq{}}\PYG{p}{,} \PYG{n}{model}\PYG{o}{=}\PYG{l+s+s2}{\PYGZdq{}}\PYG{l+s+s2}{medium}\PYG{l+s+s2}{\PYGZdq{}}\PYG{p}{)}
\PYG{g+gp}{\PYGZgt{}\PYGZgt{}\PYGZgt{} }\PYG{n}{timeslices} \PYG{o}{=} \PYG{n}{examples}\PYG{o}{.}\PYG{n}{sector}\PYG{p}{(}\PYG{l+s+s2}{\PYGZdq{}}\PYG{l+s+s2}{residential}\PYG{l+s+s2}{\PYGZdq{}}\PYG{p}{,} \PYG{n}{model}\PYG{o}{=}\PYG{l+s+s2}{\PYGZdq{}}\PYG{l+s+s2}{medium}\PYG{l+s+s2}{\PYGZdq{}}\PYG{p}{)}\PYG{o}{.}\PYG{n}{timeslices}
\PYG{g+gp}{\PYGZgt{}\PYGZgt{}\PYGZgt{} }\PYG{n}{costs} \PYG{o}{=} \PYG{p}{(}
\PYG{g+gp}{... }    \PYG{n}{search\PYGZus{}space}
\PYG{g+gp}{... }    \PYG{o}{*} \PYG{n}{np}\PYG{o}{.}\PYG{n}{arange}\PYG{p}{(}\PYG{n}{np}\PYG{o}{.}\PYG{n}{prod}\PYG{p}{(}\PYG{n}{search\PYGZus{}space}\PYG{o}{.}\PYG{n}{shape}\PYG{p}{)}\PYG{p}{)}\PYG{o}{.}\PYG{n}{reshape}\PYG{p}{(}\PYG{n}{search\PYGZus{}space}\PYG{o}{.}\PYG{n}{shape}\PYG{p}{)}
\PYG{g+gp}{... }\PYG{p}{)}
\end{sphinxVerbatim}

The function returns the LP vector split along capacity and production
variables.

\begin{sphinxVerbatim}[commandchars=\\\{\}]
\PYG{g+gp}{\PYGZgt{}\PYGZgt{}\PYGZgt{} }\PYG{k+kn}{from} \PYG{n+nn}{muse}\PYG{n+nn}{.}\PYG{n+nn}{constraints} \PYG{k+kn}{import} \PYG{n}{lp\PYGZus{}costs}
\PYG{g+gp}{\PYGZgt{}\PYGZgt{}\PYGZgt{} }\PYG{n}{lpcosts} \PYG{o}{=} \PYG{n}{lp\PYGZus{}costs}\PYG{p}{(}
\PYG{g+gp}{... }    \PYG{n}{technologies}\PYG{o}{.}\PYG{n}{sel}\PYG{p}{(}\PYG{n}{year}\PYG{o}{=}\PYG{l+m+mi}{2020}\PYG{p}{,} \PYG{n}{region}\PYG{o}{=}\PYG{l+s+s2}{\PYGZdq{}}\PYG{l+s+s2}{R1}\PYG{l+s+s2}{\PYGZdq{}}\PYG{p}{)}\PYG{p}{,} \PYG{n}{costs}\PYG{p}{,} \PYG{n}{timeslices}
\PYG{g+gp}{... }\PYG{p}{)}
\PYG{g+gp}{\PYGZgt{}\PYGZgt{}\PYGZgt{} }\PYG{k}{assert} \PYG{l+s+s2}{\PYGZdq{}}\PYG{l+s+s2}{capacity}\PYG{l+s+s2}{\PYGZdq{}} \PYG{o+ow}{in} \PYG{n}{lpcosts}\PYG{o}{.}\PYG{n}{data\PYGZus{}vars}
\PYG{g+gp}{\PYGZgt{}\PYGZgt{}\PYGZgt{} }\PYG{k}{assert} \PYG{l+s+s2}{\PYGZdq{}}\PYG{l+s+s2}{production}\PYG{l+s+s2}{\PYGZdq{}} \PYG{o+ow}{in} \PYG{n}{lpcosts}\PYG{o}{.}\PYG{n}{data\PYGZus{}vars}
\end{sphinxVerbatim}

The capacity costs correspond exactly to the input costs:

\begin{sphinxVerbatim}[commandchars=\\\{\}]
\PYG{g+gp}{\PYGZgt{}\PYGZgt{}\PYGZgt{} }\PYG{k}{assert} \PYG{p}{(}\PYG{n}{costs} \PYG{o}{==} \PYG{n}{lpcosts}\PYG{o}{.}\PYG{n}{capacity}\PYG{p}{)}\PYG{o}{.}\PYG{n}{all}\PYG{p}{(}\PYG{p}{)}
\end{sphinxVerbatim}

The production is zero in this context. It does not enter the cost function of
the LP problem:

\begin{sphinxVerbatim}[commandchars=\\\{\}]
\PYG{g+gp}{\PYGZgt{}\PYGZgt{}\PYGZgt{} }\PYG{k}{assert} \PYG{p}{(}\PYG{n}{lpcosts}\PYG{o}{.}\PYG{n}{production} \PYG{o}{==} \PYG{l+m+mi}{0}\PYG{p}{)}\PYG{o}{.}\PYG{n}{all}\PYG{p}{(}\PYG{p}{)}
\end{sphinxVerbatim}

They should correspond to a data\sphinxhyphen{}array with dimensions \sphinxcode{\sphinxupquote{(asset, replacement)}}
(and possibly \sphinxcode{\sphinxupquote{region}} as well).

\begin{sphinxVerbatim}[commandchars=\\\{\}]
\PYG{g+gp}{\PYGZgt{}\PYGZgt{}\PYGZgt{} }\PYG{n}{lpcosts}\PYG{o}{.}\PYG{n}{capacity}\PYG{o}{.}\PYG{n}{dims}
\PYG{g+go}{(\PYGZsq{}asset\PYGZsq{}, \PYGZsq{}replacement\PYGZsq{})}
\end{sphinxVerbatim}

The production costs are zero by default. However, the production expands over
not only the dimensions of the capacity, but also the \sphinxcode{\sphinxupquote{timeslice}} during
which production occurs and the \sphinxcode{\sphinxupquote{commodity}} produced.

\begin{sphinxVerbatim}[commandchars=\\\{\}]
\PYG{g+gp}{\PYGZgt{}\PYGZgt{}\PYGZgt{} }\PYG{n}{lpcosts}\PYG{o}{.}\PYG{n}{production}\PYG{o}{.}\PYG{n}{dims}
\PYG{g+go}{(\PYGZsq{}timeslice\PYGZsq{}, \PYGZsq{}asset\PYGZsq{}, \PYGZsq{}replacement\PYGZsq{}, \PYGZsq{}commodity\PYGZsq{})}
\end{sphinxVerbatim}

\end{fulllineitems}

\index{max\_capacity\_expansion() (in module muse.constraints)@\spxentry{max\_capacity\_expansion()}\spxextra{in module muse.constraints}}

\begin{fulllineitems}
\phantomsection\label{\detokenize{api:muse.constraints.max_capacity_expansion}}\pysiglinewithargsret{\sphinxcode{\sphinxupquote{muse.constraints.}}\sphinxbfcode{\sphinxupquote{max\_capacity\_expansion}}}{\emph{\DUrole{n}{demand}\DUrole{p}{:} \DUrole{n}{xarray.core.dataarray.DataArray}}, \emph{\DUrole{n}{assets}\DUrole{p}{:} \DUrole{n}{xarray.core.dataset.Dataset}}, \emph{\DUrole{n}{search\_space}\DUrole{p}{:} \DUrole{n}{xarray.core.dataarray.DataArray}}, \emph{\DUrole{n}{market}\DUrole{p}{:} \DUrole{n}{xarray.core.dataset.Dataset}}, \emph{\DUrole{n}{technologies}\DUrole{p}{:} \DUrole{n}{xarray.core.dataset.Dataset}}, \emph{\DUrole{n}{year}\DUrole{p}{:} \DUrole{n}{Optional\DUrole{p}{{[}}int\DUrole{p}{{]}}} \DUrole{o}{=} \DUrole{default_value}{None}}, \emph{\DUrole{n}{forecast}\DUrole{p}{:} \DUrole{n}{Optional\DUrole{p}{{[}}int\DUrole{p}{{]}}} \DUrole{o}{=} \DUrole{default_value}{None}}, \emph{\DUrole{n}{interpolation}\DUrole{p}{:} \DUrole{n}{str} \DUrole{o}{=} \DUrole{default_value}{\textquotesingle{}linear\textquotesingle{}}}}{{ $\rightarrow$ xarray.core.dataset.Dataset}}
Max\sphinxhyphen{}capacity addition, max\sphinxhyphen{}capacity growth, and capacity limits constraints.

Limits by how much the capacity of each technology owned by an agent can grow in
a given year. This is a constraint on the agent’s ability to invest in a
technology.

Let \(L_t^r(y)\) be the total capacity limit for a given year, technology,
and region. \(G_t^r(y)\) is the maximum growth. And \(W_t^r(y)\) is
the maximum additional capacity. \(y=y_0\) is the current year and
\(y=y_1\) is the year marking the end of the investment period.

Let \(\mathcal{A}^{i, r}_{t, \iota}(y)\) be the current assets, before
invesment, and let \(\Delta\mathcal{A}^{i,r}_t\) be the future investements.
The the constraint on agent \(i\) are given as:
\begin{align*}\!\begin{aligned}
L_t^r(y_0) - \sum_\iota \mathcal{A}^{i, r}_{t, \iota}(y_1)
    \geq \Delta\mathcal{A}^{i,r}_t\\
(y_1 - y_0 + 1) G_t^r(y_0) \sum_\iota \mathcal{A}^{i, r}_{t, \iota}(y_0)
    - \sum_\iota \mathcal{A}^{i, r}_{t, \iota}(y_1)
    \geq \Delta\mathcal{A}^{i,r}_t\\
(y_1 - y_0)W_t^r(y_0) \geq  \Delta\mathcal{A}^{i,r}_t\\
\end{aligned}\end{align*}
The three constraints are combined into a single one which is returned as the
maximum capacity expansion, \(\Gamma_t^{r, i}\). The maximum capacity
expansion cannot impose negative investments:
Maximum capacity addition:
\begin{quote}
\begin{equation*}
\begin{split}\Gamma_t^{r, i} \geq 0\end{split}
\end{equation*}\end{quote}

\end{fulllineitems}

\index{max\_production() (in module muse.constraints)@\spxentry{max\_production()}\spxextra{in module muse.constraints}}

\begin{fulllineitems}
\phantomsection\label{\detokenize{api:muse.constraints.max_production}}\pysiglinewithargsret{\sphinxcode{\sphinxupquote{muse.constraints.}}\sphinxbfcode{\sphinxupquote{max\_production}}}{\emph{\DUrole{n}{demand}\DUrole{p}{:} \DUrole{n}{xarray.core.dataarray.DataArray}}, \emph{\DUrole{n}{assets}\DUrole{p}{:} \DUrole{n}{xarray.core.dataset.Dataset}}, \emph{\DUrole{n}{search\_space}\DUrole{p}{:} \DUrole{n}{xarray.core.dataarray.DataArray}}, \emph{\DUrole{n}{market}\DUrole{p}{:} \DUrole{n}{xarray.core.dataset.Dataset}}, \emph{\DUrole{n}{technologies}\DUrole{p}{:} \DUrole{n}{xarray.core.dataset.Dataset}}, \emph{\DUrole{n}{year}\DUrole{p}{:} \DUrole{n}{Optional\DUrole{p}{{[}}int\DUrole{p}{{]}}} \DUrole{o}{=} \DUrole{default_value}{None}}}{{ $\rightarrow$ xarray.core.dataset.Dataset}}
Constructs constraint between capacity and maximum production.

Constrains the production decision variable by the maximum production for a given
capacity.

\end{fulllineitems}

\index{register\_constraints() (in module muse.constraints)@\spxentry{register\_constraints()}\spxextra{in module muse.constraints}}

\begin{fulllineitems}
\phantomsection\label{\detokenize{api:muse.constraints.register_constraints}}\pysiglinewithargsret{\sphinxcode{\sphinxupquote{muse.constraints.}}\sphinxbfcode{\sphinxupquote{register\_constraints}}}{\emph{\DUrole{n}{function}\DUrole{p}{:} \DUrole{n}{Optional\DUrole{p}{{[}}Callable\DUrole{p}{{[}}\DUrole{p}{{[}}xarray.core.dataarray.DataArray\DUrole{p}{, }xarray.core.dataset.Dataset\DUrole{p}{, }xarray.core.dataarray.DataArray\DUrole{p}{, }xarray.core.dataset.Dataset\DUrole{p}{, }xarray.core.dataset.Dataset\DUrole{p}{, }Any\DUrole{p}{{]}}\DUrole{p}{, }Optional\DUrole{p}{{[}}xarray.core.dataset.Dataset\DUrole{p}{{]}}\DUrole{p}{{]}}\DUrole{p}{{]}}}}}{{ $\rightarrow$ Callable\DUrole{p}{{[}}\DUrole{p}{{[}}xarray.core.dataarray.DataArray\DUrole{p}{, }xarray.core.dataset.Dataset\DUrole{p}{, }xarray.core.dataarray.DataArray\DUrole{p}{, }xarray.core.dataset.Dataset\DUrole{p}{, }xarray.core.dataset.Dataset\DUrole{p}{, }Any\DUrole{p}{{]}}\DUrole{p}{, }Optional\DUrole{p}{{[}}xarray.core.dataset.Dataset\DUrole{p}{{]}}\DUrole{p}{{]}}}}
Registers a constraint with MUSE.

See \sphinxcode{\sphinxupquote{muse.constraints}}.

\end{fulllineitems}

\index{search\_space() (in module muse.constraints)@\spxentry{search\_space()}\spxextra{in module muse.constraints}}

\begin{fulllineitems}
\phantomsection\label{\detokenize{api:muse.constraints.search_space}}\pysiglinewithargsret{\sphinxcode{\sphinxupquote{muse.constraints.}}\sphinxbfcode{\sphinxupquote{search\_space}}}{\emph{\DUrole{n}{demand}\DUrole{p}{:} \DUrole{n}{xarray.core.dataarray.DataArray}}, \emph{\DUrole{n}{assets}\DUrole{p}{:} \DUrole{n}{xarray.core.dataset.Dataset}}, \emph{\DUrole{n}{search\_space}\DUrole{p}{:} \DUrole{n}{xarray.core.dataarray.DataArray}}, \emph{\DUrole{n}{market}\DUrole{p}{:} \DUrole{n}{xarray.core.dataset.Dataset}}, \emph{\DUrole{n}{technologies}\DUrole{p}{:} \DUrole{n}{xarray.core.dataset.Dataset}}, \emph{\DUrole{n}{year}\DUrole{p}{:} \DUrole{n}{Optional\DUrole{p}{{[}}int\DUrole{p}{{]}}} \DUrole{o}{=} \DUrole{default_value}{None}}, \emph{\DUrole{n}{forecast}\DUrole{p}{:} \DUrole{n}{int} \DUrole{o}{=} \DUrole{default_value}{5}}}{{ $\rightarrow$ Optional\DUrole{p}{{[}}xarray.core.dataset.Dataset\DUrole{p}{{]}}}}
Removes disabled technologies.

\end{fulllineitems}



\subsection{Initial and Final Asset Transforms}
\label{\detokenize{api:module-muse.hooks}}\label{\detokenize{api:initial-and-final-asset-transforms}}\index{module@\spxentry{module}!muse.hooks@\spxentry{muse.hooks}}\index{muse.hooks@\spxentry{muse.hooks}!module@\spxentry{module}}
Pre and post hooks on agents.
\index{asset\_merge\_factory() (in module muse.hooks)@\spxentry{asset\_merge\_factory()}\spxextra{in module muse.hooks}}

\begin{fulllineitems}
\phantomsection\label{\detokenize{api:muse.hooks.asset_merge_factory}}\pysiglinewithargsret{\sphinxcode{\sphinxupquote{muse.hooks.}}\sphinxbfcode{\sphinxupquote{asset\_merge\_factory}}}{\emph{\DUrole{n}{settings}\DUrole{p}{:} \DUrole{n}{Union\DUrole{p}{{[}}str\DUrole{p}{, }Mapping\DUrole{p}{{]}}} \DUrole{o}{=} \DUrole{default_value}{\textquotesingle{}new\textquotesingle{}}}}{{ $\rightarrow$ Callable}}
Returns a function for merging new investments into assets.

Available merging functions should be registered with
\sphinxcode{\sphinxupquote{@register\_final\_asset\_transform}}.

\end{fulllineitems}

\index{clean() (in module muse.hooks)@\spxentry{clean()}\spxextra{in module muse.hooks}}

\begin{fulllineitems}
\phantomsection\label{\detokenize{api:muse.hooks.clean}}\pysiglinewithargsret{\sphinxcode{\sphinxupquote{muse.hooks.}}\sphinxbfcode{\sphinxupquote{clean}}}{\emph{\DUrole{n}{agent}\DUrole{p}{:} \DUrole{n}{muse.agents.agent.Agent}}, \emph{\DUrole{n}{assets}\DUrole{p}{:} \DUrole{n}{xarray.core.dataset.Dataset}}}{{ $\rightarrow$ xarray.core.dataset.Dataset}}
Removes empty assets.

\end{fulllineitems}

\index{housekeeping\_factory() (in module muse.hooks)@\spxentry{housekeeping\_factory()}\spxextra{in module muse.hooks}}

\begin{fulllineitems}
\phantomsection\label{\detokenize{api:muse.hooks.housekeeping_factory}}\pysiglinewithargsret{\sphinxcode{\sphinxupquote{muse.hooks.}}\sphinxbfcode{\sphinxupquote{housekeeping\_factory}}}{\emph{\DUrole{n}{settings}\DUrole{p}{:} \DUrole{n}{Union\DUrole{p}{{[}}str\DUrole{p}{, }Mapping\DUrole{p}{{]}}} \DUrole{o}{=} \DUrole{default_value}{\textquotesingle{}noop\textquotesingle{}}}}{{ $\rightarrow$ Callable}}
Returns a function for performing initial housekeeping.

For instance, remove technologies with no capacity now or in the future.
Available housekeeping functions should be registered with
\sphinxcode{\sphinxupquote{@register\_initial\_asset\_transform}}.

\end{fulllineitems}

\index{merge\_assets() (in module muse.hooks)@\spxentry{merge\_assets()}\spxextra{in module muse.hooks}}

\begin{fulllineitems}
\phantomsection\label{\detokenize{api:muse.hooks.merge_assets}}\pysiglinewithargsret{\sphinxcode{\sphinxupquote{muse.hooks.}}\sphinxbfcode{\sphinxupquote{merge\_assets}}}{\emph{\DUrole{n}{old\_assets}\DUrole{p}{:} \DUrole{n}{xarray.core.dataset.Dataset}}, \emph{\DUrole{n}{new\_assets}\DUrole{p}{:} \DUrole{n}{xarray.core.dataset.Dataset}}}{{ $\rightarrow$ xarray.core.dataset.Dataset}}
Adds new assets to old along asset dimension.

New assets are assumed to be unequivalent to any old\_assets. Indeed,
it is expected that the asset dimension does not have coordinates (i.e.
it is a combination of coordinates, such as technology and installation
year).

After merging the new assets, quantities are back\sphinxhyphen{}filled along the year
dimension. Further missing values (i.e. future years the old\_assets
did not take into account) are set to zero.

\end{fulllineitems}

\index{new\_assets\_only() (in module muse.hooks)@\spxentry{new\_assets\_only()}\spxextra{in module muse.hooks}}

\begin{fulllineitems}
\phantomsection\label{\detokenize{api:muse.hooks.new_assets_only}}\pysiglinewithargsret{\sphinxcode{\sphinxupquote{muse.hooks.}}\sphinxbfcode{\sphinxupquote{new\_assets\_only}}}{\emph{\DUrole{n}{old\_assets}\DUrole{p}{:} \DUrole{n}{xarray.core.dataset.Dataset}}, \emph{\DUrole{n}{new\_assets}\DUrole{p}{:} \DUrole{n}{xarray.core.dataset.Dataset}}}{{ $\rightarrow$ xarray.core.dataset.Dataset}}
Returns newly invested assets and ignores old assets.

\end{fulllineitems}

\index{noop() (in module muse.hooks)@\spxentry{noop()}\spxextra{in module muse.hooks}}

\begin{fulllineitems}
\phantomsection\label{\detokenize{api:muse.hooks.noop}}\pysiglinewithargsret{\sphinxcode{\sphinxupquote{muse.hooks.}}\sphinxbfcode{\sphinxupquote{noop}}}{\emph{\DUrole{n}{agent}\DUrole{p}{:} \DUrole{n}{muse.agents.agent.Agent}}, \emph{\DUrole{n}{assets}\DUrole{p}{:} \DUrole{n}{xarray.core.dataset.Dataset}}}{{ $\rightarrow$ xarray.core.dataset.Dataset}}
Return assets as they are.

\end{fulllineitems}

\index{old\_assets\_only() (in module muse.hooks)@\spxentry{old\_assets\_only()}\spxextra{in module muse.hooks}}

\begin{fulllineitems}
\phantomsection\label{\detokenize{api:muse.hooks.old_assets_only}}\pysiglinewithargsret{\sphinxcode{\sphinxupquote{muse.hooks.}}\sphinxbfcode{\sphinxupquote{old\_assets\_only}}}{\emph{\DUrole{n}{old\_assets}\DUrole{p}{:} \DUrole{n}{xarray.core.dataset.Dataset}}, \emph{\DUrole{n}{new\_assets}\DUrole{p}{:} \DUrole{n}{xarray.core.dataset.Dataset}}}{{ $\rightarrow$ xarray.core.dataset.Dataset}}
Returns old assets and ignores newly invested assets.

\end{fulllineitems}

\index{register\_final\_asset\_transform() (in module muse.hooks)@\spxentry{register\_final\_asset\_transform()}\spxextra{in module muse.hooks}}

\begin{fulllineitems}
\phantomsection\label{\detokenize{api:muse.hooks.register_final_asset_transform}}\pysiglinewithargsret{\sphinxcode{\sphinxupquote{muse.hooks.}}\sphinxbfcode{\sphinxupquote{register\_final\_asset\_transform}}}{\emph{\DUrole{n}{function}\DUrole{p}{:} \DUrole{n}{Callable\DUrole{p}{{[}}\DUrole{p}{{[}}xarray.core.dataset.Dataset\DUrole{p}{, }xarray.core.dataset.Dataset\DUrole{p}{{]}}\DUrole{p}{, }xarray.core.dataset.Dataset\DUrole{p}{{]}}}}}{{ $\rightarrow$ Callable}}
Decorator to register a function to merge new investments into current assets.

The transform is applied a the very end of the agent iteration. It can be any
function which takes as input the current set of assets, the new assets, and any
number of keyword arguments. The function must return a “merge” of the two assets.

For instance, the new assets could completely replace the old assets
(\sphinxcode{\sphinxupquote{new\_assets\_only()}}), or they could be summed to the old assets
(\sphinxcode{\sphinxupquote{merge\_assets()}}).

\end{fulllineitems}

\index{register\_initial\_asset\_transform() (in module muse.hooks)@\spxentry{register\_initial\_asset\_transform()}\spxextra{in module muse.hooks}}

\begin{fulllineitems}
\phantomsection\label{\detokenize{api:muse.hooks.register_initial_asset_transform}}\pysiglinewithargsret{\sphinxcode{\sphinxupquote{muse.hooks.}}\sphinxbfcode{\sphinxupquote{register\_initial\_asset\_transform}}}{\emph{\DUrole{n}{function}\DUrole{p}{:} \DUrole{n}{Callable\DUrole{p}{{[}}\DUrole{p}{{[}}muse.agents.agent.Agent\DUrole{p}{, }xarray.core.dataset.Dataset\DUrole{p}{{]}}\DUrole{p}{, }xarray.core.dataset.Dataset\DUrole{p}{{]}}}}}{{ $\rightarrow$ Callable}}
Decorator to register a function for cleaning or transforming assets.

The transformation is applied at the start of each iteration. It any function which
take an agent and assets as input and any number of keyword arguments, and returns
the transformed assets. The agent should not be modified. It is only there to query
the current year, the region, etc.

\end{fulllineitems}



\section{Reading the inputs}
\label{\detokenize{api:module-muse.readers.toml}}\label{\detokenize{api:reading-the-inputs}}\index{module@\spxentry{module}!muse.readers.toml@\spxentry{muse.readers.toml}}\index{muse.readers.toml@\spxentry{muse.readers.toml}!module@\spxentry{module}}
Ensemble of functions to read MUSE data.
\index{read\_settings() (in module muse.readers.toml)@\spxentry{read\_settings()}\spxextra{in module muse.readers.toml}}

\begin{fulllineitems}
\phantomsection\label{\detokenize{api:muse.readers.toml.read_settings}}\pysiglinewithargsret{\sphinxcode{\sphinxupquote{muse.readers.toml.}}\sphinxbfcode{\sphinxupquote{read\_settings}}}{\emph{\DUrole{n}{settings\_file}\DUrole{p}{:} \DUrole{n}{Union\DUrole{p}{{[}}str\DUrole{p}{, }pathlib.Path\DUrole{p}{, }IO\DUrole{p}{{[}}str\DUrole{p}{{]}}\DUrole{p}{, }Mapping\DUrole{p}{{]}}}}, \emph{\DUrole{n}{path}\DUrole{p}{:} \DUrole{n}{Optional\DUrole{p}{{[}}Union\DUrole{p}{{[}}str\DUrole{p}{, }pathlib.Path\DUrole{p}{{]}}\DUrole{p}{{]}}} \DUrole{o}{=} \DUrole{default_value}{None}}}{{ $\rightarrow$ Any}}
Loads the input settings for any MUSE simulation.

Loads a MUSE settings file. This must be a TOML formatted file. Missing settings are
loaded from the DEFAULT\_SETTINGS. Custom pythom modules, if present, are loaded
and checks are run to validate the settings and ensure that they are compatible with
a MUSE simulation.

Arguments:
settings\_file: A string or a Path to the settings file
\begin{quote}\begin{description}
\item[{Returns}] \leavevmode
A dictionary with the settings

\end{description}\end{quote}

\end{fulllineitems}

\phantomsection\label{\detokenize{api:module-muse.readers.csv}}\index{module@\spxentry{module}!muse.readers.csv@\spxentry{muse.readers.csv}}\index{muse.readers.csv@\spxentry{muse.readers.csv}!module@\spxentry{module}}
Ensemble of functions to read MUSE data.
\index{read\_attribute\_table() (in module muse.readers.csv)@\spxentry{read\_attribute\_table()}\spxextra{in module muse.readers.csv}}

\begin{fulllineitems}
\phantomsection\label{\detokenize{api:muse.readers.csv.read_attribute_table}}\pysiglinewithargsret{\sphinxcode{\sphinxupquote{muse.readers.csv.}}\sphinxbfcode{\sphinxupquote{read\_attribute\_table}}}{\emph{\DUrole{n}{path}\DUrole{p}{:} \DUrole{n}{Union\DUrole{p}{{[}}str\DUrole{p}{, }pathlib.Path\DUrole{p}{{]}}}}}{{ $\rightarrow$ xarray.core.dataarray.DataArray}}
Read a standard MUSE csv file for price projections.

\end{fulllineitems}

\index{read\_csv\_agent\_parameters() (in module muse.readers.csv)@\spxentry{read\_csv\_agent\_parameters()}\spxextra{in module muse.readers.csv}}

\begin{fulllineitems}
\phantomsection\label{\detokenize{api:muse.readers.csv.read_csv_agent_parameters}}\pysiglinewithargsret{\sphinxcode{\sphinxupquote{muse.readers.csv.}}\sphinxbfcode{\sphinxupquote{read\_csv\_agent\_parameters}}}{\emph{\DUrole{n}{filename}}}{{ $\rightarrow$ List}}
Reads standard MUSE agent\sphinxhyphen{}declaration csv\sphinxhyphen{}files.

Returns a list of dictionaries, where each dictionary can be used to instantiate an
agent in \sphinxcode{\sphinxupquote{muse.agents.factories.factory()}}.

\end{fulllineitems}

\index{read\_csv\_outputs() (in module muse.readers.csv)@\spxentry{read\_csv\_outputs()}\spxextra{in module muse.readers.csv}}

\begin{fulllineitems}
\phantomsection\label{\detokenize{api:muse.readers.csv.read_csv_outputs}}\pysiglinewithargsret{\sphinxcode{\sphinxupquote{muse.readers.csv.}}\sphinxbfcode{\sphinxupquote{read\_csv\_outputs}}}{\emph{\DUrole{n}{paths}\DUrole{p}{:} \DUrole{n}{Union\DUrole{p}{{[}}str\DUrole{p}{, }pathlib.Path\DUrole{p}{, }Sequence\DUrole{p}{{[}}Union\DUrole{p}{{[}}str\DUrole{p}{, }pathlib.Path\DUrole{p}{{]}}\DUrole{p}{{]}}\DUrole{p}{{]}}}}, \emph{\DUrole{n}{columns}\DUrole{p}{:} \DUrole{n}{str} \DUrole{o}{=} \DUrole{default_value}{\textquotesingle{}commodity\textquotesingle{}}}, \emph{\DUrole{n}{indices}\DUrole{p}{:} \DUrole{n}{Sequence\DUrole{p}{{[}}str\DUrole{p}{{]}}} \DUrole{o}{=} \DUrole{default_value}{\textquotesingle{}RegionName\textquotesingle{}, \textquotesingle{}ProcessName\textquotesingle{}, \textquotesingle{}Timeslice\textquotesingle{}}}, \emph{\DUrole{n}{drop}\DUrole{p}{:} \DUrole{n}{Sequence\DUrole{p}{{[}}str\DUrole{p}{{]}}} \DUrole{o}{=} \DUrole{default_value}{\textquotesingle{}Unnamed: 0\textquotesingle{}}}}{{ $\rightarrow$ xarray.core.dataset.Dataset}}
Read standard MUSE output files for consumption or supply.

\end{fulllineitems}

\index{read\_csv\_timeslices() (in module muse.readers.csv)@\spxentry{read\_csv\_timeslices()}\spxextra{in module muse.readers.csv}}

\begin{fulllineitems}
\phantomsection\label{\detokenize{api:muse.readers.csv.read_csv_timeslices}}\pysiglinewithargsret{\sphinxcode{\sphinxupquote{muse.readers.csv.}}\sphinxbfcode{\sphinxupquote{read\_csv\_timeslices}}}{\emph{\DUrole{n}{path}\DUrole{p}{:} \DUrole{n}{Union\DUrole{p}{{[}}str\DUrole{p}{, }pathlib.Path\DUrole{p}{{]}}}}, \emph{\DUrole{o}{**}\DUrole{n}{kwargs}}}{{ $\rightarrow$ xarray.core.dataarray.DataArray}}
Reads timeslice information from input.

\end{fulllineitems}

\index{read\_global\_commodities() (in module muse.readers.csv)@\spxentry{read\_global\_commodities()}\spxextra{in module muse.readers.csv}}

\begin{fulllineitems}
\phantomsection\label{\detokenize{api:muse.readers.csv.read_global_commodities}}\pysiglinewithargsret{\sphinxcode{\sphinxupquote{muse.readers.csv.}}\sphinxbfcode{\sphinxupquote{read\_global\_commodities}}}{\emph{\DUrole{n}{path}\DUrole{p}{:} \DUrole{n}{Union\DUrole{p}{{[}}str\DUrole{p}{, }pathlib.Path\DUrole{p}{{]}}}}}{{ $\rightarrow$ xarray.core.dataset.Dataset}}
Reads commodities information from input.

\end{fulllineitems}

\index{read\_initial\_assets() (in module muse.readers.csv)@\spxentry{read\_initial\_assets()}\spxextra{in module muse.readers.csv}}

\begin{fulllineitems}
\phantomsection\label{\detokenize{api:muse.readers.csv.read_initial_assets}}\pysiglinewithargsret{\sphinxcode{\sphinxupquote{muse.readers.csv.}}\sphinxbfcode{\sphinxupquote{read\_initial\_assets}}}{\emph{\DUrole{n}{filename}\DUrole{p}{:} \DUrole{n}{Union\DUrole{p}{{[}}str\DUrole{p}{, }pathlib.Path\DUrole{p}{{]}}}}}{{ $\rightarrow$ xarray.core.dataarray.DataArray}}
Reads and formats data about initial capacity into a dataframe.

\end{fulllineitems}

\index{read\_initial\_market() (in module muse.readers.csv)@\spxentry{read\_initial\_market()}\spxextra{in module muse.readers.csv}}

\begin{fulllineitems}
\phantomsection\label{\detokenize{api:muse.readers.csv.read_initial_market}}\pysiglinewithargsret{\sphinxcode{\sphinxupquote{muse.readers.csv.}}\sphinxbfcode{\sphinxupquote{read\_initial\_market}}}{\emph{\DUrole{n}{projections}\DUrole{p}{:} \DUrole{n}{Union\DUrole{p}{{[}}xarray.core.dataarray.DataArray\DUrole{p}{, }pathlib.Path\DUrole{p}{, }str\DUrole{p}{{]}}}}, \emph{\DUrole{n}{base\_year\_import}\DUrole{p}{:} \DUrole{n}{Optional\DUrole{p}{{[}}Union\DUrole{p}{{[}}str\DUrole{p}{, }pathlib.Path\DUrole{p}{, }xarray.core.dataarray.DataArray\DUrole{p}{{]}}\DUrole{p}{{]}}} \DUrole{o}{=} \DUrole{default_value}{None}}, \emph{\DUrole{n}{base\_year\_export}\DUrole{p}{:} \DUrole{n}{Optional\DUrole{p}{{[}}Union\DUrole{p}{{[}}str\DUrole{p}{, }pathlib.Path\DUrole{p}{, }xarray.core.dataarray.DataArray\DUrole{p}{{]}}\DUrole{p}{{]}}} \DUrole{o}{=} \DUrole{default_value}{None}}, \emph{\DUrole{n}{timeslices}\DUrole{p}{:} \DUrole{n}{Optional\DUrole{p}{{[}}xarray.core.dataarray.DataArray\DUrole{p}{{]}}} \DUrole{o}{=} \DUrole{default_value}{None}}}{{ $\rightarrow$ xarray.core.dataset.Dataset}}
Read projections, import and export csv files.

\end{fulllineitems}

\index{read\_io\_technodata() (in module muse.readers.csv)@\spxentry{read\_io\_technodata()}\spxextra{in module muse.readers.csv}}

\begin{fulllineitems}
\phantomsection\label{\detokenize{api:muse.readers.csv.read_io_technodata}}\pysiglinewithargsret{\sphinxcode{\sphinxupquote{muse.readers.csv.}}\sphinxbfcode{\sphinxupquote{read\_io\_technodata}}}{\emph{\DUrole{n}{filename}\DUrole{p}{:} \DUrole{n}{Union\DUrole{p}{{[}}str\DUrole{p}{, }pathlib.Path\DUrole{p}{{]}}}}}{{ $\rightarrow$ xarray.core.dataset.Dataset}}
Reads process inputs or ouputs.

There are four axes: (technology, region, year, commodity)

\end{fulllineitems}

\index{read\_macro\_drivers() (in module muse.readers.csv)@\spxentry{read\_macro\_drivers()}\spxextra{in module muse.readers.csv}}

\begin{fulllineitems}
\phantomsection\label{\detokenize{api:muse.readers.csv.read_macro_drivers}}\pysiglinewithargsret{\sphinxcode{\sphinxupquote{muse.readers.csv.}}\sphinxbfcode{\sphinxupquote{read\_macro\_drivers}}}{\emph{\DUrole{n}{path}\DUrole{p}{:} \DUrole{n}{Union\DUrole{p}{{[}}str\DUrole{p}{, }pathlib.Path\DUrole{p}{{]}}}}}{{ $\rightarrow$ xarray.core.dataset.Dataset}}
Reads a standard MUSE csv file for macro drivers.

\end{fulllineitems}

\index{read\_regression\_parameters() (in module muse.readers.csv)@\spxentry{read\_regression\_parameters()}\spxextra{in module muse.readers.csv}}

\begin{fulllineitems}
\phantomsection\label{\detokenize{api:muse.readers.csv.read_regression_parameters}}\pysiglinewithargsret{\sphinxcode{\sphinxupquote{muse.readers.csv.}}\sphinxbfcode{\sphinxupquote{read\_regression\_parameters}}}{\emph{\DUrole{n}{path}\DUrole{p}{:} \DUrole{n}{Union\DUrole{p}{{[}}str\DUrole{p}{, }pathlib.Path\DUrole{p}{{]}}}}}{{ $\rightarrow$ xarray.core.dataset.Dataset}}
Reads the regression parameters from a standard MUSE csv file.

\end{fulllineitems}

\index{read\_technodictionary() (in module muse.readers.csv)@\spxentry{read\_technodictionary()}\spxextra{in module muse.readers.csv}}

\begin{fulllineitems}
\phantomsection\label{\detokenize{api:muse.readers.csv.read_technodictionary}}\pysiglinewithargsret{\sphinxcode{\sphinxupquote{muse.readers.csv.}}\sphinxbfcode{\sphinxupquote{read\_technodictionary}}}{\emph{\DUrole{n}{filename}\DUrole{p}{:} \DUrole{n}{Union\DUrole{p}{{[}}str\DUrole{p}{, }pathlib.Path\DUrole{p}{{]}}}}}{{ $\rightarrow$ xarray.core.dataset.Dataset}}
Reads and formats technodata into a dataset.

There are three axes: technologies, regions, and year.

\end{fulllineitems}

\index{read\_technologies() (in module muse.readers.csv)@\spxentry{read\_technologies()}\spxextra{in module muse.readers.csv}}

\begin{fulllineitems}
\phantomsection\label{\detokenize{api:muse.readers.csv.read_technologies}}\pysiglinewithargsret{\sphinxcode{\sphinxupquote{muse.readers.csv.}}\sphinxbfcode{\sphinxupquote{read\_technologies}}}{\emph{\DUrole{n}{technodata\_path\_or\_sector}\DUrole{p}{:} \DUrole{n}{Optional\DUrole{p}{{[}}Union\DUrole{p}{{[}}str\DUrole{p}{, }pathlib.Path\DUrole{p}{{]}}\DUrole{p}{{]}}} \DUrole{o}{=} \DUrole{default_value}{None}}, \emph{\DUrole{n}{comm\_out\_path}\DUrole{p}{:} \DUrole{n}{Optional\DUrole{p}{{[}}Union\DUrole{p}{{[}}str\DUrole{p}{, }pathlib.Path\DUrole{p}{{]}}\DUrole{p}{{]}}} \DUrole{o}{=} \DUrole{default_value}{None}}, \emph{\DUrole{n}{comm\_in\_path}\DUrole{p}{:} \DUrole{n}{Optional\DUrole{p}{{[}}Union\DUrole{p}{{[}}str\DUrole{p}{, }pathlib.Path\DUrole{p}{{]}}\DUrole{p}{{]}}} \DUrole{o}{=} \DUrole{default_value}{None}}, \emph{\DUrole{n}{commodities}\DUrole{p}{:} \DUrole{n}{Optional\DUrole{p}{{[}}Union\DUrole{p}{{[}}str\DUrole{p}{, }pathlib.Path\DUrole{p}{, }xarray.core.dataset.Dataset\DUrole{p}{{]}}\DUrole{p}{{]}}} \DUrole{o}{=} \DUrole{default_value}{None}}, \emph{\DUrole{n}{sectors\_directory}\DUrole{p}{:} \DUrole{n}{Union\DUrole{p}{{[}}str\DUrole{p}{, }pathlib.Path\DUrole{p}{{]}}} \DUrole{o}{=} \DUrole{default_value}{PosixPath(\textquotesingle{}/Users/alexkell/Documents/SGI/2\sphinxhyphen{}documentation/StarMuse/docs/data\textquotesingle{})}}}{{ $\rightarrow$ xarray.core.dataset.Dataset}}
Reads data characterising technologies from files.
\begin{quote}\begin{description}
\item[{Parameters}] \leavevmode\begin{itemize}
\item {} 
\sphinxstyleliteralstrong{\sphinxupquote{technodata\_path\_or\_sector}} \textendash{} If \sphinxtitleref{comm\_out\_path} and \sphinxtitleref{comm\_in\_path} are not given,
then this argument refers to the name of the sector. The three paths are
then determined using standard locations and name. Specifically, thechnodata
looks for a “technodataSECTORNAME.csv” file in the standard location for
that sector. However, if  \sphinxtitleref{comm\_out\_path} and \sphinxtitleref{comm\_in\_path} are given, then
this should be the path to the the technodata file.

\item {} 
\sphinxstyleliteralstrong{\sphinxupquote{comm\_out\_path}} \textendash{} If given, then refers to the path of the file specifying output
commmodities. If not given, then defaults to
“commOUTtechnodataSECTORNAME.csv” in the relevant sector directory.

\item {} 
\sphinxstyleliteralstrong{\sphinxupquote{comm\_in\_path}} \textendash{} If given, then refers to the path of the file specifying input
commmodities. If not given, then defaults to
“commINtechnodataSECTORNAME.csv” in the relevant sector directory.

\item {} 
\sphinxstyleliteralstrong{\sphinxupquote{commodities}} \textendash{} Optional. If commodities is given, it should point to a global
commodities file, or a dataset akin to reading such a file with
\sphinxtitleref{read\_global\_commodities}. In either case, the information pertaining to
commodities will be added to the technologies dataset.

\item {} 
\sphinxstyleliteralstrong{\sphinxupquote{sectors\_directory}} \textendash{} Optional. If \sphinxtitleref{paths\_or\_sector} is a string indicating the
name of the sector, then this is a path to a directory where standard input
files are contained.

\end{itemize}

\item[{Returns}] \leavevmode
A dataset with all the characteristics of the technologies.

\end{description}\end{quote}

\end{fulllineitems}

\index{read\_timeslice\_shares() (in module muse.readers.csv)@\spxentry{read\_timeslice\_shares()}\spxextra{in module muse.readers.csv}}

\begin{fulllineitems}
\phantomsection\label{\detokenize{api:muse.readers.csv.read_timeslice_shares}}\pysiglinewithargsret{\sphinxcode{\sphinxupquote{muse.readers.csv.}}\sphinxbfcode{\sphinxupquote{read\_timeslice\_shares}}}{\emph{\DUrole{n}{path}\DUrole{p}{:} \DUrole{n}{Union\DUrole{p}{{[}}str\DUrole{p}{, }pathlib.Path\DUrole{p}{{]}}} \DUrole{o}{=} \DUrole{default_value}{PosixPath(\textquotesingle{}/Users/alexkell/Documents/SGI/2\sphinxhyphen{}documentation/StarMuse/docs/data\textquotesingle{})}}, \emph{\DUrole{n}{sector}\DUrole{p}{:} \DUrole{n}{Optional\DUrole{p}{{[}}str\DUrole{p}{{]}}} \DUrole{o}{=} \DUrole{default_value}{None}}, \emph{\DUrole{n}{timeslice}\DUrole{p}{:} \DUrole{n}{Union\DUrole{p}{{[}}str\DUrole{p}{, }pathlib.Path\DUrole{p}{, }xarray.core.dataarray.DataArray\DUrole{p}{{]}}} \DUrole{o}{=} \DUrole{default_value}{\textquotesingle{}Timeslices\{sector\}.csv\textquotesingle{}}}}{{ $\rightarrow$ xarray.core.dataset.Dataset}}
Reads sliceshare information into a xr.Dataset.

Additionaly, this function will try and recover the timeslice multi\sphinxhyphen{} index from a
import file “Timeslices\{sector\}.csv” in the same directory as the timeslice shares.
Pass \sphinxtitleref{None} if this behaviour is not required.

\end{fulllineitems}

\phantomsection\label{\detokenize{api:module-muse.decorators}}\index{module@\spxentry{module}!muse.decorators@\spxentry{muse.decorators}}\index{muse.decorators@\spxentry{muse.decorators}!module@\spxentry{module}}\index{SETTINGS\_CHECKS (in module muse.decorators)@\spxentry{SETTINGS\_CHECKS}\spxextra{in module muse.decorators}}

\begin{fulllineitems}
\phantomsection\label{\detokenize{api:muse.decorators.SETTINGS_CHECKS}}\pysigline{\sphinxcode{\sphinxupquote{muse.decorators.}}\sphinxbfcode{\sphinxupquote{SETTINGS\_CHECKS}}\sphinxbfcode{\sphinxupquote{: Mapping{[}str, Callable{[}{[}dict{]}, None{]}{]}}}\sphinxbfcode{\sphinxupquote{ = \{\textquotesingle{}check\_budget\_parameters\textquotesingle{}: \textless{}function check\_budget\_parameters\textgreater{}, \textquotesingle{}check\_foresight\textquotesingle{}: \textless{}function check\_foresight\textgreater{}, \textquotesingle{}check\_global\_data\_files\textquotesingle{}: \textless{}function check\_global\_data\_files\textgreater{}, \textquotesingle{}check\_interpolation\_mode\textquotesingle{}: \textless{}function check\_interpolation\_mode\textgreater{}, \textquotesingle{}check\_iteration\_control\textquotesingle{}: \textless{}function check\_iteration\_control\textgreater{}, \textquotesingle{}check\_log\_level\textquotesingle{}: \textless{}function check\_log\_level\textgreater{}, \textquotesingle{}check\_sectors\_files\textquotesingle{}: \textless{}function check\_sectors\_files\textgreater{}, \textquotesingle{}check\_time\_slices\textquotesingle{}: \textless{}function check\_time\_slices\textgreater{}\}}}}
Dictionary of settings checks.

\end{fulllineitems}

\index{SETTINGS\_CHECKS\_SIGNATURE (in module muse.decorators)@\spxentry{SETTINGS\_CHECKS\_SIGNATURE}\spxextra{in module muse.decorators}}

\begin{fulllineitems}
\phantomsection\label{\detokenize{api:muse.decorators.SETTINGS_CHECKS_SIGNATURE}}\pysigline{\sphinxcode{\sphinxupquote{muse.decorators.}}\sphinxbfcode{\sphinxupquote{SETTINGS\_CHECKS\_SIGNATURE}}}
settings checks signature.

alias of Callable{[}{[}dict{]}, None{]}

\end{fulllineitems}

\index{register\_settings\_check() (in module muse.decorators)@\spxentry{register\_settings\_check()}\spxextra{in module muse.decorators}}

\begin{fulllineitems}
\phantomsection\label{\detokenize{api:muse.decorators.register_settings_check}}\pysiglinewithargsret{\sphinxcode{\sphinxupquote{muse.decorators.}}\sphinxbfcode{\sphinxupquote{register\_settings\_check}}}{\emph{\DUrole{n}{function}\DUrole{p}{:} \DUrole{n}{Callable{[}{[}dict{]}, None{]}}}}{}
Decorator to register a function as a settings check.

Registers a function as a settings check so that it can be applied easily
when validating the MUSE input settings.

There is no restriction on the function name, although is should be
in lower\_snake\_case, as it is a python function.

\end{fulllineitems}



\section{Writing Outputs}
\label{\detokenize{api:module-muse.outputs}}\label{\detokenize{api:writing-outputs}}\index{module@\spxentry{module}!muse.outputs@\spxentry{muse.outputs}}\index{muse.outputs@\spxentry{muse.outputs}!module@\spxentry{module}}\index{register\_output\_quantity() (in module muse.outputs)@\spxentry{register\_output\_quantity()}\spxextra{in module muse.outputs}}

\begin{fulllineitems}
\phantomsection\label{\detokenize{api:muse.outputs.register_output_quantity}}\pysiglinewithargsret{\sphinxcode{\sphinxupquote{muse.outputs.}}\sphinxbfcode{\sphinxupquote{register\_output\_quantity}}}{\emph{\DUrole{n}{function}\DUrole{p}{:} \DUrole{n}{Callable\DUrole{p}{{[}}\DUrole{p}{{[}}xarray.core.dataset.Dataset\DUrole{p}{, }xarray.core.dataarray.DataArray\DUrole{p}{, }xarray.core.dataset.Dataset\DUrole{p}{, }Any\DUrole{p}{{]}}\DUrole{p}{, }Union\DUrole{p}{{[}}pandas.core.frame.DataFrame\DUrole{p}{, }xarray.core.dataarray.DataArray\DUrole{p}{{]}}\DUrole{p}{{]}}} \DUrole{o}{=} \DUrole{default_value}{None}}}{{ $\rightarrow$ Callable}}
Registers a function to compute an output quantity.

\end{fulllineitems}

\index{register\_output\_sink() (in module muse.outputs)@\spxentry{register\_output\_sink()}\spxextra{in module muse.outputs}}

\begin{fulllineitems}
\phantomsection\label{\detokenize{api:muse.outputs.register_output_sink}}\pysiglinewithargsret{\sphinxcode{\sphinxupquote{muse.outputs.}}\sphinxbfcode{\sphinxupquote{register\_output\_sink}}}{\emph{\DUrole{n}{function}\DUrole{p}{:} \DUrole{n}{Callable\DUrole{p}{{[}}\DUrole{p}{{[}}Union\DUrole{p}{{[}}xarray.core.dataarray.DataArray\DUrole{p}{, }pandas.core.frame.DataFrame\DUrole{p}{{]}}\DUrole{p}{, }int\DUrole{p}{, }Any\DUrole{p}{{]}}\DUrole{p}{, }Optional\DUrole{p}{{[}}str\DUrole{p}{{]}}\DUrole{p}{{]}}} \DUrole{o}{=} \DUrole{default_value}{None}}}{{ $\rightarrow$ Callable}}
Registers a function to save quantities.

\end{fulllineitems}



\subsection{Sinks}
\label{\detokenize{api:module-muse.outputs.sinks}}\label{\detokenize{api:sinks}}\index{module@\spxentry{module}!muse.outputs.sinks@\spxentry{muse.outputs.sinks}}\index{muse.outputs.sinks@\spxentry{muse.outputs.sinks}!module@\spxentry{module}}
Sinks where output quantities can be stored.

Sinks take as argument a DataArray and store it somewhere. Additionally they
take a dictionary as argument. This dictionary will always contains the items
(‘quantity’, ‘sector’, ‘year’) referring to the name of the quantity, the name
of the calling sector, the current year. They may contain additional parameters
which depend on the actual sink, such as ‘filename’.

Optionally, a description of the storage (filename, etc) can be returned.

The signature of a sink is:

\begin{sphinxVerbatim}[commandchars=\\\{\}]
\PYG{n+nd}{@register\PYGZus{}output\PYGZus{}sink}\PYG{p}{(}\PYG{n}{name}\PYG{o}{=}\PYG{l+s+s2}{\PYGZdq{}}\PYG{l+s+s2}{netcfd}\PYG{l+s+s2}{\PYGZdq{}}\PYG{p}{)}
\PYG{k}{def} \PYG{n+nf}{to\PYGZus{}netcfd}\PYG{p}{(}\PYG{n}{quantity}\PYG{p}{:} \PYG{n}{DataArray}\PYG{p}{,} \PYG{n}{config}\PYG{p}{:} \PYG{n}{Mapping}\PYG{p}{)} \PYG{o}{\PYGZhy{}}\PYG{o}{\PYGZgt{}} \PYG{n}{Optional}\PYG{p}{[}\PYG{n}{Text}\PYG{p}{]}\PYG{p}{:}
    \PYG{k}{pass}
\end{sphinxVerbatim}
\index{FiniteResourceException@\spxentry{FiniteResourceException}}

\begin{fulllineitems}
\phantomsection\label{\detokenize{api:muse.outputs.sinks.FiniteResourceException}}\pysigline{\sphinxbfcode{\sphinxupquote{exception }}\sphinxcode{\sphinxupquote{muse.outputs.sinks.}}\sphinxbfcode{\sphinxupquote{FiniteResourceException}}}
Raised when a finite resource is exceeded.

\end{fulllineitems}

\index{OUTPUT\_SINKS (in module muse.outputs.sinks)@\spxentry{OUTPUT\_SINKS}\spxextra{in module muse.outputs.sinks}}

\begin{fulllineitems}
\phantomsection\label{\detokenize{api:muse.outputs.sinks.OUTPUT_SINKS}}\pysigline{\sphinxcode{\sphinxupquote{muse.outputs.sinks.}}\sphinxbfcode{\sphinxupquote{OUTPUT\_SINKS}}\sphinxbfcode{\sphinxupquote{: MutableMapping\DUrole{p}{{[}}str\DUrole{p}{, }Union\DUrole{p}{{[}}Callable\DUrole{p}{{[}}\DUrole{p}{{[}}Union\DUrole{p}{{[}}xarray.core.dataarray.DataArray\DUrole{p}{, }pandas.core.frame.DataFrame\DUrole{p}{{]}}\DUrole{p}{, }int\DUrole{p}{, }Any\DUrole{p}{{]}}\DUrole{p}{, }Optional\DUrole{p}{{[}}str\DUrole{p}{{]}}\DUrole{p}{{]}}\DUrole{p}{, }Callable\DUrole{p}{{]}}\DUrole{p}{{]}}}}\sphinxbfcode{\sphinxupquote{ = \{\textquotesingle{}Aggregate\textquotesingle{}: \textless{}class \textquotesingle{}muse.outputs.sinks.YearlyAggregate\textquotesingle{}\textgreater{}, \textquotesingle{}Csv\textquotesingle{}: \textless{}function to\_csv\textgreater{}, \textquotesingle{}Excel\textquotesingle{}: \textless{}function to\_excel\textgreater{}, \textquotesingle{}FiniteResourceLogger\textquotesingle{}: \textless{}function finite\_resource\_logger\textgreater{}, \textquotesingle{}Nc\textquotesingle{}: \textless{}function to\_netcdf\textgreater{}, \textquotesingle{}Netcdf\textquotesingle{}: \textless{}function to\_netcdf\textgreater{}, \textquotesingle{}ToCsv\textquotesingle{}: \textless{}function to\_csv\textgreater{}, \textquotesingle{}ToExcel\textquotesingle{}: \textless{}function to\_excel\textgreater{}, \textquotesingle{}ToNetcdf\textquotesingle{}: \textless{}function to\_netcdf\textgreater{}, \textquotesingle{}Xlsx\textquotesingle{}: \textless{}function to\_excel\textgreater{}, \textquotesingle{}YearlyAggregate\textquotesingle{}: \textless{}class \textquotesingle{}muse.outputs.sinks.YearlyAggregate\textquotesingle{}\textgreater{}, \textquotesingle{}Yearlyaggregate\textquotesingle{}: \textless{}class \textquotesingle{}muse.outputs.sinks.YearlyAggregate\textquotesingle{}\textgreater{}, \textquotesingle{}aggregate\textquotesingle{}: \textless{}class \textquotesingle{}muse.outputs.sinks.YearlyAggregate\textquotesingle{}\textgreater{}, \textquotesingle{}csv\textquotesingle{}: \textless{}function to\_csv\textgreater{}, \textquotesingle{}excel\textquotesingle{}: \textless{}function to\_excel\textgreater{}, \textquotesingle{}finite\sphinxhyphen{}resource\sphinxhyphen{}logger\textquotesingle{}: \textless{}function finite\_resource\_logger\textgreater{}, \textquotesingle{}finiteResourceLogger\textquotesingle{}: \textless{}function finite\_resource\_logger\textgreater{}, \textquotesingle{}finite\_resource\_logger\textquotesingle{}: \textless{}function finite\_resource\_logger\textgreater{}, \textquotesingle{}finiteresourcelogger\textquotesingle{}: \textless{}function finite\_resource\_logger\textgreater{}, \textquotesingle{}nc\textquotesingle{}: \textless{}function to\_netcdf\textgreater{}, \textquotesingle{}netcdf\textquotesingle{}: \textless{}function to\_netcdf\textgreater{}, \textquotesingle{}to\sphinxhyphen{}csv\textquotesingle{}: \textless{}function to\_csv\textgreater{}, \textquotesingle{}to\sphinxhyphen{}excel\textquotesingle{}: \textless{}function to\_excel\textgreater{}, \textquotesingle{}to\sphinxhyphen{}netcdf\textquotesingle{}: \textless{}function to\_netcdf\textgreater{}, \textquotesingle{}toCsv\textquotesingle{}: \textless{}function to\_csv\textgreater{}, \textquotesingle{}toExcel\textquotesingle{}: \textless{}function to\_excel\textgreater{}, \textquotesingle{}toNetcdf\textquotesingle{}: \textless{}function to\_netcdf\textgreater{}, \textquotesingle{}to\_csv\textquotesingle{}: \textless{}function to\_csv\textgreater{}, \textquotesingle{}to\_excel\textquotesingle{}: \textless{}function to\_excel\textgreater{}, \textquotesingle{}to\_netcdf\textquotesingle{}: \textless{}function to\_netcdf\textgreater{}, \textquotesingle{}tocsv\textquotesingle{}: \textless{}function to\_csv\textgreater{}, \textquotesingle{}toexcel\textquotesingle{}: \textless{}function to\_excel\textgreater{}, \textquotesingle{}tonetcdf\textquotesingle{}: \textless{}function to\_netcdf\textgreater{}, \textquotesingle{}xlsx\textquotesingle{}: \textless{}function to\_excel\textgreater{}\}}}}
Stores a quantity somewhere.

\end{fulllineitems}

\index{OUTPUT\_SINK\_SIGNATURE (in module muse.outputs.sinks)@\spxentry{OUTPUT\_SINK\_SIGNATURE}\spxextra{in module muse.outputs.sinks}}

\begin{fulllineitems}
\phantomsection\label{\detokenize{api:muse.outputs.sinks.OUTPUT_SINK_SIGNATURE}}\pysigline{\sphinxcode{\sphinxupquote{muse.outputs.sinks.}}\sphinxbfcode{\sphinxupquote{OUTPUT\_SINK\_SIGNATURE}}}
Signature of functions used to save quantities.

alias of Callable{[}{[}Union{[}xarray.core.dataarray.DataArray, pandas.core.frame.DataFrame{]}, int, Any{]}, Optional{[}str{]}{]}

\end{fulllineitems}

\index{YearlyAggregate (class in muse.outputs.sinks)@\spxentry{YearlyAggregate}\spxextra{class in muse.outputs.sinks}}

\begin{fulllineitems}
\phantomsection\label{\detokenize{api:muse.outputs.sinks.YearlyAggregate}}\pysiglinewithargsret{\sphinxbfcode{\sphinxupquote{class }}\sphinxcode{\sphinxupquote{muse.outputs.sinks.}}\sphinxbfcode{\sphinxupquote{YearlyAggregate}}}{\emph{\DUrole{n}{final\_sink}\DUrole{p}{:} \DUrole{n}{Optional\DUrole{p}{{[}}MutableMapping\DUrole{p}{{[}}str\DUrole{p}{, }Any\DUrole{p}{{]}}\DUrole{p}{{]}}} \DUrole{o}{=} \DUrole{default_value}{None}}, \emph{\DUrole{n}{sector}\DUrole{p}{:} \DUrole{n}{str} \DUrole{o}{=} \DUrole{default_value}{\textquotesingle{}\textquotesingle{}}}, \emph{\DUrole{n}{axis}\DUrole{o}{=}\DUrole{default_value}{\textquotesingle{}year\textquotesingle{}}}, \emph{\DUrole{o}{**}\DUrole{n}{kwargs}}}{}
Incrementally aggregates data from year to year.

\end{fulllineitems}

\index{register\_output\_sink() (in module muse.outputs.sinks)@\spxentry{register\_output\_sink()}\spxextra{in module muse.outputs.sinks}}

\begin{fulllineitems}
\phantomsection\label{\detokenize{api:muse.outputs.sinks.register_output_sink}}\pysiglinewithargsret{\sphinxcode{\sphinxupquote{muse.outputs.sinks.}}\sphinxbfcode{\sphinxupquote{register\_output\_sink}}}{\emph{\DUrole{n}{function}\DUrole{p}{:} \DUrole{n}{Callable\DUrole{p}{{[}}\DUrole{p}{{[}}Union\DUrole{p}{{[}}xarray.core.dataarray.DataArray\DUrole{p}{, }pandas.core.frame.DataFrame\DUrole{p}{{]}}\DUrole{p}{, }int\DUrole{p}{, }Any\DUrole{p}{{]}}\DUrole{p}{, }Optional\DUrole{p}{{[}}str\DUrole{p}{{]}}\DUrole{p}{{]}}} \DUrole{o}{=} \DUrole{default_value}{None}}}{{ $\rightarrow$ Callable}}
Registers a function to save quantities.

\end{fulllineitems}

\index{sink\_to\_file() (in module muse.outputs.sinks)@\spxentry{sink\_to\_file()}\spxextra{in module muse.outputs.sinks}}

\begin{fulllineitems}
\phantomsection\label{\detokenize{api:muse.outputs.sinks.sink_to_file}}\pysiglinewithargsret{\sphinxcode{\sphinxupquote{muse.outputs.sinks.}}\sphinxbfcode{\sphinxupquote{sink\_to\_file}}}{\emph{\DUrole{n}{suffix}\DUrole{p}{:} \DUrole{n}{str}}}{}
Simplifies sinks to files.

The decorator takes care of figuring out the path to the file, as well as trims the
configuration dictionary to include only parameters for the sink itself. The
decorated function returns the path to the output file.

\end{fulllineitems}

\index{to\_csv() (in module muse.outputs.sinks)@\spxentry{to\_csv()}\spxextra{in module muse.outputs.sinks}}

\begin{fulllineitems}
\phantomsection\label{\detokenize{api:muse.outputs.sinks.to_csv}}\pysiglinewithargsret{\sphinxcode{\sphinxupquote{muse.outputs.sinks.}}\sphinxbfcode{\sphinxupquote{to\_csv}}}{\emph{\DUrole{n}{quantity}\DUrole{p}{:} \DUrole{n}{Union\DUrole{p}{{[}}pandas.core.frame.DataFrame\DUrole{p}{, }xarray.core.dataarray.DataArray\DUrole{p}{{]}}}}, \emph{\DUrole{n}{filename}\DUrole{p}{:} \DUrole{n}{str}}, \emph{\DUrole{o}{**}\DUrole{n}{params}}}{{ $\rightarrow$ None}}
Saves data array to csv format, using pandas.to\_csv.
\begin{quote}\begin{description}
\item[{Parameters}] \leavevmode\begin{itemize}
\item {} 
\sphinxstyleliteralstrong{\sphinxupquote{quantity}} \textendash{} The data to be saved

\item {} 
\sphinxstyleliteralstrong{\sphinxupquote{filename}} \textendash{} File to which the data should be saved

\item {} 
\sphinxstyleliteralstrong{\sphinxupquote{params}} \textendash{} A configuration dictionary accepting any argument to \sphinxtitleref{pandas.to\_csv}

\end{itemize}

\end{description}\end{quote}

\end{fulllineitems}

\index{to\_excel() (in module muse.outputs.sinks)@\spxentry{to\_excel()}\spxextra{in module muse.outputs.sinks}}

\begin{fulllineitems}
\phantomsection\label{\detokenize{api:muse.outputs.sinks.to_excel}}\pysiglinewithargsret{\sphinxcode{\sphinxupquote{muse.outputs.sinks.}}\sphinxbfcode{\sphinxupquote{to\_excel}}}{\emph{\DUrole{n}{quantity}\DUrole{p}{:} \DUrole{n}{Union\DUrole{p}{{[}}pandas.core.frame.DataFrame\DUrole{p}{, }xarray.core.dataarray.DataArray\DUrole{p}{{]}}}}, \emph{\DUrole{n}{filename}\DUrole{p}{:} \DUrole{n}{str}}, \emph{\DUrole{o}{**}\DUrole{n}{params}}}{{ $\rightarrow$ None}}
Saves data array to csv format, using pandas.to\_excel.
\begin{quote}\begin{description}
\item[{Parameters}] \leavevmode\begin{itemize}
\item {} 
\sphinxstyleliteralstrong{\sphinxupquote{quantity}} \textendash{} The data to be saved

\item {} 
\sphinxstyleliteralstrong{\sphinxupquote{filename}} \textendash{} File to which the data should be saved

\item {} 
\sphinxstyleliteralstrong{\sphinxupquote{params}} \textendash{} A configuration dictionary accepting any argument to \sphinxtitleref{pandas.to\_excel}

\end{itemize}

\end{description}\end{quote}

\end{fulllineitems}

\index{to\_netcdf() (in module muse.outputs.sinks)@\spxentry{to\_netcdf()}\spxextra{in module muse.outputs.sinks}}

\begin{fulllineitems}
\phantomsection\label{\detokenize{api:muse.outputs.sinks.to_netcdf}}\pysiglinewithargsret{\sphinxcode{\sphinxupquote{muse.outputs.sinks.}}\sphinxbfcode{\sphinxupquote{to\_netcdf}}}{\emph{\DUrole{n}{quantity}\DUrole{p}{:} \DUrole{n}{Union\DUrole{p}{{[}}xarray.core.dataarray.DataArray\DUrole{p}{, }pandas.core.frame.DataFrame\DUrole{p}{{]}}}}, \emph{\DUrole{n}{filename}\DUrole{p}{:} \DUrole{n}{str}}, \emph{\DUrole{o}{**}\DUrole{n}{params}}}{{ $\rightarrow$ None}}
Saves data array to csv format, using xarray.to\_netcdf.
\begin{quote}\begin{description}
\item[{Parameters}] \leavevmode\begin{itemize}
\item {} 
\sphinxstyleliteralstrong{\sphinxupquote{quantity}} \textendash{} The data to be saved

\item {} 
\sphinxstyleliteralstrong{\sphinxupquote{filename}} \textendash{} File to which the data should be saved

\item {} 
\sphinxstyleliteralstrong{\sphinxupquote{params}} \textendash{} A configuration dictionary accepting any argument to \sphinxtitleref{xarray.to\_netcdf}

\end{itemize}

\end{description}\end{quote}

\end{fulllineitems}



\subsection{Sectorial Outputs}
\label{\detokenize{api:module-muse.outputs.sector}}\label{\detokenize{api:sectorial-outputs}}\index{module@\spxentry{module}!muse.outputs.sector@\spxentry{muse.outputs.sector}}\index{muse.outputs.sector@\spxentry{muse.outputs.sector}!module@\spxentry{module}}
Output quantities.

Functions that compute sectorial quantities for post\sphinxhyphen{}simulation analysis should all
follow the same signature:

\begin{sphinxVerbatim}[commandchars=\\\{\}]
\PYG{n+nd}{@register\PYGZus{}output\PYGZus{}quantity}
\PYG{k}{def} \PYG{n+nf}{quantity}\PYG{p}{(}
    \PYG{n}{capacity}\PYG{p}{:} \PYG{n}{xr}\PYG{o}{.}\PYG{n}{DataArray}\PYG{p}{,}
    \PYG{n}{market}\PYG{p}{:} \PYG{n}{xr}\PYG{o}{.}\PYG{n}{Dataset}\PYG{p}{,}
    \PYG{n}{technologies}\PYG{p}{:} \PYG{n}{xr}\PYG{o}{.}\PYG{n}{Dataset}
\PYG{p}{)} \PYG{o}{\PYGZhy{}}\PYG{o}{\PYGZgt{}} \PYG{n}{Union}\PYG{p}{[}\PYG{n}{xr}\PYG{o}{.}\PYG{n}{DataArray}\PYG{p}{,} \PYG{n}{DataFrame}\PYG{p}{]}\PYG{p}{:}
    \PYG{k}{pass}
\end{sphinxVerbatim}

They take as input the current capacity profile, aggregated across a sectoar,
a dataset containing market\sphinxhyphen{}related quantities, and a dataset characterizing the
technologies in the market. It returns a single xr.DataArray object.

The function should never modify it’s arguments.
\index{OUTPUTS\_PARAMETERS (in module muse.outputs.sector)@\spxentry{OUTPUTS\_PARAMETERS}\spxextra{in module muse.outputs.sector}}

\begin{fulllineitems}
\phantomsection\label{\detokenize{api:muse.outputs.sector.OUTPUTS_PARAMETERS}}\pysigline{\sphinxcode{\sphinxupquote{muse.outputs.sector.}}\sphinxbfcode{\sphinxupquote{OUTPUTS\_PARAMETERS}}}
Acceptable Datastructures for outputs parameters

alias of Union{[}str, Mapping{]}

\end{fulllineitems}

\index{OUTPUT\_QUANTITIES (in module muse.outputs.sector)@\spxentry{OUTPUT\_QUANTITIES}\spxextra{in module muse.outputs.sector}}

\begin{fulllineitems}
\phantomsection\label{\detokenize{api:muse.outputs.sector.OUTPUT_QUANTITIES}}\pysigline{\sphinxcode{\sphinxupquote{muse.outputs.sector.}}\sphinxbfcode{\sphinxupquote{OUTPUT\_QUANTITIES}}}
Quantity for post\sphinxhyphen{}simulation analysis.

\end{fulllineitems}

\index{OUTPUT\_QUANTITY\_SIGNATURE (in module muse.outputs.sector)@\spxentry{OUTPUT\_QUANTITY\_SIGNATURE}\spxextra{in module muse.outputs.sector}}

\begin{fulllineitems}
\phantomsection\label{\detokenize{api:muse.outputs.sector.OUTPUT_QUANTITY_SIGNATURE}}\pysigline{\sphinxcode{\sphinxupquote{muse.outputs.sector.}}\sphinxbfcode{\sphinxupquote{OUTPUT\_QUANTITY\_SIGNATURE}}}
Signature of functions computing quantities for later analysis.

alias of Callable{[}{[}xarray.core.dataset.Dataset, xarray.core.dataarray.DataArray, xarray.core.dataset.Dataset, Any{]}, Union{[}pandas.core.frame.DataFrame, xarray.core.dataarray.DataArray{]}{]}

\end{fulllineitems}

\index{capacity() (in module muse.outputs.sector)@\spxentry{capacity()}\spxextra{in module muse.outputs.sector}}

\begin{fulllineitems}
\phantomsection\label{\detokenize{api:muse.outputs.sector.capacity}}\pysiglinewithargsret{\sphinxcode{\sphinxupquote{muse.outputs.sector.}}\sphinxbfcode{\sphinxupquote{capacity}}}{\emph{\DUrole{n}{market}\DUrole{p}{:} \DUrole{n}{xarray.core.dataset.Dataset}}, \emph{\DUrole{n}{capacity}\DUrole{p}{:} \DUrole{n}{xarray.core.dataarray.DataArray}}, \emph{\DUrole{n}{technologies}\DUrole{p}{:} \DUrole{n}{xarray.core.dataset.Dataset}}, \emph{\DUrole{n}{rounding}\DUrole{p}{:} \DUrole{n}{int} \DUrole{o}{=} \DUrole{default_value}{4}}}{{ $\rightarrow$ pandas.core.frame.DataFrame}}
Current capacity.

\end{fulllineitems}

\index{consumption() (in module muse.outputs.sector)@\spxentry{consumption()}\spxextra{in module muse.outputs.sector}}

\begin{fulllineitems}
\phantomsection\label{\detokenize{api:muse.outputs.sector.consumption}}\pysiglinewithargsret{\sphinxcode{\sphinxupquote{muse.outputs.sector.}}\sphinxbfcode{\sphinxupquote{consumption}}}{\emph{\DUrole{n}{market}\DUrole{p}{:} \DUrole{n}{xarray.core.dataset.Dataset}}, \emph{\DUrole{n}{capacity}\DUrole{p}{:} \DUrole{n}{xarray.core.dataarray.DataArray}}, \emph{\DUrole{n}{technologies}\DUrole{p}{:} \DUrole{n}{xarray.core.dataset.Dataset}}, \emph{\DUrole{n}{sum\_over}\DUrole{p}{:} \DUrole{n}{Optional\DUrole{p}{{[}}List\DUrole{p}{{[}}str\DUrole{p}{{]}}\DUrole{p}{{]}}} \DUrole{o}{=} \DUrole{default_value}{None}}, \emph{\DUrole{n}{drop}\DUrole{p}{:} \DUrole{n}{Optional\DUrole{p}{{[}}List\DUrole{p}{{[}}str\DUrole{p}{{]}}\DUrole{p}{{]}}} \DUrole{o}{=} \DUrole{default_value}{None}}, \emph{\DUrole{n}{rounding}\DUrole{p}{:} \DUrole{n}{int} \DUrole{o}{=} \DUrole{default_value}{4}}}{{ $\rightarrow$ xarray.core.dataarray.DataArray}}
Current consumption.

\end{fulllineitems}

\index{costs() (in module muse.outputs.sector)@\spxentry{costs()}\spxextra{in module muse.outputs.sector}}

\begin{fulllineitems}
\phantomsection\label{\detokenize{api:muse.outputs.sector.costs}}\pysiglinewithargsret{\sphinxcode{\sphinxupquote{muse.outputs.sector.}}\sphinxbfcode{\sphinxupquote{costs}}}{\emph{\DUrole{n}{market}\DUrole{p}{:} \DUrole{n}{xarray.core.dataset.Dataset}}, \emph{\DUrole{n}{capacity}\DUrole{p}{:} \DUrole{n}{xarray.core.dataarray.DataArray}}, \emph{\DUrole{n}{technologies}\DUrole{p}{:} \DUrole{n}{xarray.core.dataset.Dataset}}, \emph{\DUrole{n}{sum\_over}\DUrole{p}{:} \DUrole{n}{Optional\DUrole{p}{{[}}List\DUrole{p}{{[}}str\DUrole{p}{{]}}\DUrole{p}{{]}}} \DUrole{o}{=} \DUrole{default_value}{None}}, \emph{\DUrole{n}{drop}\DUrole{p}{:} \DUrole{n}{Optional\DUrole{p}{{[}}List\DUrole{p}{{[}}str\DUrole{p}{{]}}\DUrole{p}{{]}}} \DUrole{o}{=} \DUrole{default_value}{None}}, \emph{\DUrole{n}{rounding}\DUrole{p}{:} \DUrole{n}{int} \DUrole{o}{=} \DUrole{default_value}{4}}}{{ $\rightarrow$ xarray.core.dataarray.DataArray}}
Current costs.

\end{fulllineitems}

\index{factory() (in module muse.outputs.sector)@\spxentry{factory()}\spxextra{in module muse.outputs.sector}}

\begin{fulllineitems}
\phantomsection\label{\detokenize{api:muse.outputs.sector.factory}}\pysiglinewithargsret{\sphinxcode{\sphinxupquote{muse.outputs.sector.}}\sphinxbfcode{\sphinxupquote{factory}}}{\emph{\DUrole{o}{*}\DUrole{n}{parameters}\DUrole{p}{:} \DUrole{n}{Union\DUrole{p}{{[}}str\DUrole{p}{, }Mapping\DUrole{p}{{]}}}}, \emph{\DUrole{n}{sector\_name}\DUrole{p}{:} \DUrole{n}{str} \DUrole{o}{=} \DUrole{default_value}{\textquotesingle{}default\textquotesingle{}}}}{{ $\rightarrow$ Callable\DUrole{p}{{[}}\DUrole{p}{{[}}xarray.core.dataset.Dataset\DUrole{p}{, }xarray.core.dataarray.DataArray\DUrole{p}{, }xarray.core.dataset.Dataset\DUrole{p}{{]}}\DUrole{p}{, }List\DUrole{p}{{[}}Any\DUrole{p}{{]}}\DUrole{p}{{]}}}}
Creates outputs functions for post\sphinxhyphen{}mortem analysis.

Each parameter is a dictionary containing the following:
\begin{itemize}
\item {} 
quantity (mandatory): name of the quantity to output. Mandatory.

\item {} 
sink (optional): name of the storage procedure, e.g. the file format
or database format. When it cannot be guessed from \sphinxtitleref{filename}, it defaults to
“csv”.

\item {} 
filename (optional): path to a directory or a file where to store the quantity. In
the latter case, if sink is not given, it will be determined from the file
extension. The filename can incorporate markers. By default, it is
“\{default\_output\_dir\}/\{sector\}\{year\}\{quantity\}\{suffix\}”.

\item {} 
any other parameter relevant to the sink, e.g. \sphinxtitleref{pandas.to\_csv} keyword
arguments.

\end{itemize}

For simplicity, it is also possible to given lone strings as input.
They default to \sphinxtitleref{\{‘quantity’: string\}} (and the sink will default to
“csv”).

\end{fulllineitems}

\index{register\_output\_quantity() (in module muse.outputs.sector)@\spxentry{register\_output\_quantity()}\spxextra{in module muse.outputs.sector}}

\begin{fulllineitems}
\phantomsection\label{\detokenize{api:muse.outputs.sector.register_output_quantity}}\pysiglinewithargsret{\sphinxcode{\sphinxupquote{muse.outputs.sector.}}\sphinxbfcode{\sphinxupquote{register\_output\_quantity}}}{\emph{\DUrole{n}{function}\DUrole{p}{:} \DUrole{n}{Callable\DUrole{p}{{[}}\DUrole{p}{{[}}xarray.core.dataset.Dataset\DUrole{p}{, }xarray.core.dataarray.DataArray\DUrole{p}{, }xarray.core.dataset.Dataset\DUrole{p}{, }Any\DUrole{p}{{]}}\DUrole{p}{, }Union\DUrole{p}{{[}}pandas.core.frame.DataFrame\DUrole{p}{, }xarray.core.dataarray.DataArray\DUrole{p}{{]}}\DUrole{p}{{]}}} \DUrole{o}{=} \DUrole{default_value}{None}}}{{ $\rightarrow$ Callable}}
Registers a function to compute an output quantity.

\end{fulllineitems}

\index{supply() (in module muse.outputs.sector)@\spxentry{supply()}\spxextra{in module muse.outputs.sector}}

\begin{fulllineitems}
\phantomsection\label{\detokenize{api:muse.outputs.sector.supply}}\pysiglinewithargsret{\sphinxcode{\sphinxupquote{muse.outputs.sector.}}\sphinxbfcode{\sphinxupquote{supply}}}{\emph{\DUrole{n}{market}\DUrole{p}{:} \DUrole{n}{xarray.core.dataset.Dataset}}, \emph{\DUrole{n}{capacity}\DUrole{p}{:} \DUrole{n}{xarray.core.dataarray.DataArray}}, \emph{\DUrole{n}{technologies}\DUrole{p}{:} \DUrole{n}{xarray.core.dataset.Dataset}}, \emph{\DUrole{n}{sum\_over}\DUrole{p}{:} \DUrole{n}{Optional\DUrole{p}{{[}}List\DUrole{p}{{[}}str\DUrole{p}{{]}}\DUrole{p}{{]}}} \DUrole{o}{=} \DUrole{default_value}{None}}, \emph{\DUrole{n}{drop}\DUrole{p}{:} \DUrole{n}{Optional\DUrole{p}{{[}}List\DUrole{p}{{[}}str\DUrole{p}{{]}}\DUrole{p}{{]}}} \DUrole{o}{=} \DUrole{default_value}{None}}, \emph{\DUrole{n}{rounding}\DUrole{p}{:} \DUrole{n}{int} \DUrole{o}{=} \DUrole{default_value}{4}}}{{ $\rightarrow$ xarray.core.dataarray.DataArray}}
Current supply.

\end{fulllineitems}



\section{Quantities}
\label{\detokenize{api:module-muse.quantities}}\label{\detokenize{api:quantities}}\index{module@\spxentry{module}!muse.quantities@\spxentry{muse.quantities}}\index{muse.quantities@\spxentry{muse.quantities}!module@\spxentry{module}}
Collection of functions to compute model quantities.

This module is meant to collect functions computing quantities of interest to the model,
e.g. lcoe, maximum production for a given capacity, etc, especially where these
functions are used in different areas of the model.
\index{annual\_levelized\_cost\_of\_energy() (in module muse.quantities)@\spxentry{annual\_levelized\_cost\_of\_energy()}\spxextra{in module muse.quantities}}

\begin{fulllineitems}
\phantomsection\label{\detokenize{api:muse.quantities.annual_levelized_cost_of_energy}}\pysiglinewithargsret{\sphinxcode{\sphinxupquote{muse.quantities.}}\sphinxbfcode{\sphinxupquote{annual\_levelized\_cost\_of\_energy}}}{\emph{\DUrole{n}{prices}\DUrole{p}{:} \DUrole{n}{xarray.core.dataarray.DataArray}}, \emph{\DUrole{n}{technologies}\DUrole{p}{:} \DUrole{n}{xarray.core.dataset.Dataset}}, \emph{\DUrole{n}{interpolation}\DUrole{p}{:} \DUrole{n}{str} \DUrole{o}{=} \DUrole{default_value}{\textquotesingle{}linear\textquotesingle{}}}, \emph{\DUrole{n}{fill\_value}\DUrole{p}{:} \DUrole{n}{Union\DUrole{p}{{[}}int\DUrole{p}{, }str\DUrole{p}{{]}}} \DUrole{o}{=} \DUrole{default_value}{\textquotesingle{}extrapolate\textquotesingle{}}}, \emph{\DUrole{o}{**}\DUrole{n}{filters}}}{{ $\rightarrow$ xarray.core.dataarray.DataArray}}
Levelized cost of energy (LCOE) of technologies on each given year.

It mostly follows the \sphinxhref{https://www.nrel.gov/analysis/tech-lcoe-documentation.html}{simplified LCOE} given by NREL. However, the
units are sometimes different. In the argument description, we use the following:
\begin{itemize}
\item {} 
{[}h{]}: hour

\item {} 
{[}y{]}: year

\item {} 
{[}\${]}: unit of currency

\item {} 
{[}E{]}: unit of energy

\item {} 
{[}1{]}: dimensionless

\end{itemize}
\begin{quote}\begin{description}
\item[{Parameters}] \leavevmode\begin{itemize}
\item {} 
\sphinxstyleliteralstrong{\sphinxupquote{prices}} \textendash{} {[}\$/(Eh){]} the price of all commodities, including consumables and fuels.
This dataarray contains at least timeslice and commodity dimensions.

\item {} 
\sphinxstyleliteralstrong{\sphinxupquote{technologies}} \textendash{} 
Describe the technologies, with at least the following parameters:
\begin{itemize}
\item {} 
cap\_par: {[}\$/E{]} overnight capital cost

\item {} 
interest\_rate: {[}1{]}

\item {} 
fix\_par: {[}\$/(Eh){]} fixed costs of operation and maintenance costs

\item {} 
var\_par: {[}\$/(Eh){]} variable costs of operation and maintenance costs

\item {} \begin{description}
\item[{fixed\_inputs: {[}1{]} == {[}(Eh)/(Eh){]} ratio indicating the amount of commodity}] \leavevmode
consumed per units of energy created.

\end{description}

\item {} \begin{description}
\item[{fixed\_outputs: {[}1{]} == {[}(Eh)/(Eh){]} ration indicating the amount of}] \leavevmode
environmental pollutants produced per units of energy created.

\end{description}

\end{itemize}


\item {} 
\sphinxstyleliteralstrong{\sphinxupquote{interpolation}} \textendash{} interpolation method.

\item {} 
\sphinxstyleliteralstrong{\sphinxupquote{fill\_value}} \textendash{} Fill value for values outside the extrapolation range.

\item {} 
\sphinxstyleliteralstrong{\sphinxupquote{**filters}} \textendash{} Anything by which prices can be filtered.

\end{itemize}

\item[{Returns}] \leavevmode
The lifetime LCOE in {[}\$/(Eh){]} for each technology at each timeslice.

\end{description}\end{quote}

\end{fulllineitems}

\index{capacity\_in\_use() (in module muse.quantities)@\spxentry{capacity\_in\_use()}\spxextra{in module muse.quantities}}

\begin{fulllineitems}
\phantomsection\label{\detokenize{api:muse.quantities.capacity_in_use}}\pysiglinewithargsret{\sphinxcode{\sphinxupquote{muse.quantities.}}\sphinxbfcode{\sphinxupquote{capacity\_in\_use}}}{\emph{\DUrole{n}{production}\DUrole{p}{:} \DUrole{n}{xarray.core.dataarray.DataArray}}, \emph{\DUrole{n}{technologies}\DUrole{p}{:} \DUrole{n}{xarray.core.dataset.Dataset}}, \emph{\DUrole{n}{max\_dim}\DUrole{p}{:} \DUrole{n}{Optional\DUrole{p}{{[}}Union\DUrole{p}{{[}}str\DUrole{p}{, }Tuple\DUrole{p}{{[}}str\DUrole{p}{{]}}\DUrole{p}{{]}}\DUrole{p}{{]}}} \DUrole{o}{=} \DUrole{default_value}{\textquotesingle{}commodity\textquotesingle{}}}, \emph{\DUrole{o}{**}\DUrole{n}{filters}}}{}
Capacity\sphinxhyphen{}in\sphinxhyphen{}use for each asset, given production.

Conceptually, this operation is the inverse of \sphinxtitleref{production}.
\begin{quote}\begin{description}
\item[{Parameters}] \leavevmode\begin{itemize}
\item {} 
\sphinxstyleliteralstrong{\sphinxupquote{production}} \textendash{} Production from each technology of interest.

\item {} 
\sphinxstyleliteralstrong{\sphinxupquote{technologies}} \textendash{} xr.Dataset describing the features of the technologies of
interests.  It should contain \sphinxtitleref{fixed\_outputs} and \sphinxtitleref{utilization\_factor}. It’s
shape is matched to \sphinxtitleref{capacity} using \sphinxtitleref{muse.utilities.broadcast\_techs}.

\item {} 
\sphinxstyleliteralstrong{\sphinxupquote{max\_dim}} \textendash{} reduces the given dimensions using \sphinxtitleref{max}. Defaults to “commodity”. If
None, then no reduction is performed.

\item {} 
\sphinxstyleliteralstrong{\sphinxupquote{filters}} \textendash{} keyword arguments are used to filter down the capacity and
technologies. Filters not relevant to the quantities of interest, i.e.
filters that are not a dimension of \sphinxtitleref{capacity} or \sphinxtitleref{techologies}, are
silently ignored.

\end{itemize}

\item[{Returns}] \leavevmode
Capacity\sphinxhyphen{}in\sphinxhyphen{}use for each technology, whittled down by the filters.

\end{description}\end{quote}

\end{fulllineitems}

\index{consumption() (in module muse.quantities)@\spxentry{consumption()}\spxextra{in module muse.quantities}}

\begin{fulllineitems}
\phantomsection\label{\detokenize{api:muse.quantities.consumption}}\pysiglinewithargsret{\sphinxcode{\sphinxupquote{muse.quantities.}}\sphinxbfcode{\sphinxupquote{consumption}}}{\emph{\DUrole{n}{technologies}\DUrole{p}{:} \DUrole{n}{xarray.core.dataset.Dataset}}, \emph{\DUrole{n}{production}\DUrole{p}{:} \DUrole{n}{xarray.core.dataarray.DataArray}}, \emph{\DUrole{n}{prices}\DUrole{p}{:} \DUrole{n}{Optional\DUrole{p}{{[}}xarray.core.dataarray.DataArray\DUrole{p}{{]}}} \DUrole{o}{=} \DUrole{default_value}{None}}, \emph{\DUrole{o}{**}\DUrole{n}{kwargs}}}{{ $\rightarrow$ xarray.core.dataarray.DataArray}}
Commodity consumption when fulfilling the whole production.

Currently, the consumption is implemented for commodity\_max == +infinity. If prices
are not given, then flexible consumption is \sphinxstyleemphasis{not} considered.

\end{fulllineitems}

\index{costed\_production() (in module muse.quantities)@\spxentry{costed\_production()}\spxextra{in module muse.quantities}}

\begin{fulllineitems}
\phantomsection\label{\detokenize{api:muse.quantities.costed_production}}\pysiglinewithargsret{\sphinxcode{\sphinxupquote{muse.quantities.}}\sphinxbfcode{\sphinxupquote{costed\_production}}}{\emph{\DUrole{n}{demand}\DUrole{p}{:} \DUrole{n}{xarray.core.dataset.Dataset}}, \emph{\DUrole{n}{costs}\DUrole{p}{:} \DUrole{n}{xarray.core.dataarray.DataArray}}, \emph{\DUrole{n}{capacity}\DUrole{p}{:} \DUrole{n}{xarray.core.dataarray.DataArray}}, \emph{\DUrole{n}{technologies}\DUrole{p}{:} \DUrole{n}{xarray.core.dataset.Dataset}}, \emph{\DUrole{n}{with\_minimum\_service}\DUrole{p}{:} \DUrole{n}{bool} \DUrole{o}{=} \DUrole{default_value}{True}}}{{ $\rightarrow$ xarray.core.dataarray.DataArray}}
Computes production from ranked assets.

The assets are ranked according to their cost. The asset with least cost are allowed
to service the demand first, up to the maximum production. By default, the mininum
service is applied first.

\end{fulllineitems}

\index{decommissioning\_demand() (in module muse.quantities)@\spxentry{decommissioning\_demand()}\spxextra{in module muse.quantities}}

\begin{fulllineitems}
\phantomsection\label{\detokenize{api:muse.quantities.decommissioning_demand}}\pysiglinewithargsret{\sphinxcode{\sphinxupquote{muse.quantities.}}\sphinxbfcode{\sphinxupquote{decommissioning\_demand}}}{\emph{\DUrole{n}{technologies}\DUrole{p}{:} \DUrole{n}{xarray.core.dataset.Dataset}}, \emph{\DUrole{n}{capacity}\DUrole{p}{:} \DUrole{n}{xarray.core.dataarray.DataArray}}, \emph{\DUrole{n}{year}\DUrole{p}{:} \DUrole{n}{Optional\DUrole{p}{{[}}Sequence\DUrole{p}{{[}}int\DUrole{p}{{]}}\DUrole{p}{{]}}} \DUrole{o}{=} \DUrole{default_value}{None}}}{{ $\rightarrow$ xarray.core.dataarray.DataArray}}
Computes demand from process decommissioning.

If \sphinxtitleref{year} is not given, it defaults to all years in capacity. If there are more than
two years, then decommissioning is with respect to first (or minimum) year.

Let \(M_t^r(y)\) be the retrofit demand, \(^{(s)}\mathcal{D}_t^r(y)\) be the
decommissioning demand at the level of the sector, and \(A^r_{t, \iota}(y)\) be
the assets owned by the agent. Then, the decommissioning demand for agent \(i\)
is :
\begin{equation*}
\begin{split}\mathcal{D}^{r, i}_{t, c}(y) =
    \sum_\iota \alpha_{t, \iota}^r \beta_{t, \iota, c}^r
        \left(A^{i, r}_{t, \iota}(y) - A^{i, r}_{t, \iota, c}(y + 1) \right)\end{split}
\end{equation*}
given the utilization factor \(\alpha_{t, \iota}\) and the fixed output factor
\(\beta_{t, \iota, c}\).

Furthermore, decommissioning demand is non\sphinxhyphen{}zero only for end\sphinxhyphen{}use commodities.
\begin{description}
\item[{ncsearch\sphinxhyphen{}nohlsearch).. SeeAlso:}] \leavevmode
\DUrole{xref,std,std-ref}{indices}, {\hyperref[\detokenize{api:module-muse.quantities}]{\sphinxcrossref{\DUrole{std,std-ref}{Quantities}}}},
\sphinxcode{\sphinxupquote{maximum\_production()}}
\sphinxcode{\sphinxupquote{is\_enduse()}}

\end{description}

\end{fulllineitems}

\index{demand\_matched\_production() (in module muse.quantities)@\spxentry{demand\_matched\_production()}\spxextra{in module muse.quantities}}

\begin{fulllineitems}
\phantomsection\label{\detokenize{api:muse.quantities.demand_matched_production}}\pysiglinewithargsret{\sphinxcode{\sphinxupquote{muse.quantities.}}\sphinxbfcode{\sphinxupquote{demand\_matched\_production}}}{\emph{\DUrole{n}{demand}\DUrole{p}{:} \DUrole{n}{xarray.core.dataarray.DataArray}}, \emph{\DUrole{n}{prices}\DUrole{p}{:} \DUrole{n}{xarray.core.dataarray.DataArray}}, \emph{\DUrole{n}{capacity}\DUrole{p}{:} \DUrole{n}{xarray.core.dataarray.DataArray}}, \emph{\DUrole{n}{technologies}\DUrole{p}{:} \DUrole{n}{xarray.core.dataset.Dataset}}, \emph{\DUrole{o}{**}\DUrole{n}{filters}}}{{ $\rightarrow$ xarray.core.dataarray.DataArray}}
Production matching the input demand.
\begin{quote}\begin{description}
\item[{Parameters}] \leavevmode\begin{itemize}
\item {} 
\sphinxstyleliteralstrong{\sphinxupquote{demand}} \textendash{} demand to match.

\item {} 
\sphinxstyleliteralstrong{\sphinxupquote{prices}} \textendash{} price from which to compute the annual levelized cost of energy.

\item {} 
\sphinxstyleliteralstrong{\sphinxupquote{capacity}} \textendash{} capacity from which to obtain the maximum production constraints.

\item {} 
\sphinxstyleliteralstrong{\sphinxupquote{**filters}} \textendash{} keyword arguments with which to filter the input datasets and
data arrays., e.g. region, or year.

\end{itemize}

\end{description}\end{quote}

\end{fulllineitems}

\index{emission() (in module muse.quantities)@\spxentry{emission()}\spxextra{in module muse.quantities}}

\begin{fulllineitems}
\phantomsection\label{\detokenize{api:muse.quantities.emission}}\pysiglinewithargsret{\sphinxcode{\sphinxupquote{muse.quantities.}}\sphinxbfcode{\sphinxupquote{emission}}}{\emph{\DUrole{n}{production}\DUrole{p}{:} \DUrole{n}{xarray.core.dataarray.DataArray}}, \emph{\DUrole{n}{fixed\_outputs}\DUrole{p}{:} \DUrole{n}{xarray.core.dataarray.DataArray}}}{}
Computes emission from current products.

Emissions are computed as \sphinxtitleref{sum(product) * fixed\_outputs}.
\begin{quote}\begin{description}
\item[{Parameters}] \leavevmode\begin{itemize}
\item {} 
\sphinxstyleliteralstrong{\sphinxupquote{production}} \textendash{} Produced goods. Only those with non\sphinxhyphen{}environmental products are used
when computing emissions.

\item {} 
\sphinxstyleliteralstrong{\sphinxupquote{fixed\_outputs}} \textendash{} factor relating total production to emissions. For convenience,
this can also be a \sphinxtitleref{technologies} dataset containing \sphinxtitleref{fixed\_output}.

\end{itemize}

\item[{Returns}] \leavevmode
A data array containing emissions (and only emissions).

\end{description}\end{quote}

\end{fulllineitems}

\index{gross\_margin() (in module muse.quantities)@\spxentry{gross\_margin()}\spxextra{in module muse.quantities}}

\begin{fulllineitems}
\phantomsection\label{\detokenize{api:muse.quantities.gross_margin}}\pysiglinewithargsret{\sphinxcode{\sphinxupquote{muse.quantities.}}\sphinxbfcode{\sphinxupquote{gross\_margin}}}{\emph{\DUrole{n}{technologies}\DUrole{p}{:} \DUrole{n}{xarray.core.dataset.Dataset}}, \emph{\DUrole{n}{capacity}\DUrole{p}{:} \DUrole{n}{xarray.core.dataarray.DataArray}}, \emph{\DUrole{n}{prices}\DUrole{p}{:} \DUrole{n}{xarray.core.dataset.Dataset}}}{{ $\rightarrow$ xarray.core.dataarray.DataArray}}
profit of increasing the production by one unit.
\begin{itemize}
\item {} 
energy commodities INPUTS are related to fuel costs

\item {} 
environmental commodities OUTPUTS are related to environmental costs

\item {} 
variable costs is given as technodata inputs

\item {} 
non\sphinxhyphen{}environmental commodities OUTPUTS are related to revenues

\end{itemize}

\end{fulllineitems}

\index{lifetime\_levelized\_cost\_of\_energy() (in module muse.quantities)@\spxentry{lifetime\_levelized\_cost\_of\_energy()}\spxextra{in module muse.quantities}}

\begin{fulllineitems}
\phantomsection\label{\detokenize{api:muse.quantities.lifetime_levelized_cost_of_energy}}\pysiglinewithargsret{\sphinxcode{\sphinxupquote{muse.quantities.}}\sphinxbfcode{\sphinxupquote{lifetime\_levelized\_cost\_of\_energy}}}{\emph{\DUrole{n}{prices}\DUrole{p}{:} \DUrole{n}{xarray.core.dataarray.DataArray}}, \emph{\DUrole{n}{technologies}\DUrole{p}{:} \DUrole{n}{xarray.core.dataset.Dataset}}, \emph{\DUrole{n}{installation\_year}\DUrole{p}{:} \DUrole{n}{Optional\DUrole{p}{{[}}int\DUrole{p}{{]}}} \DUrole{o}{=} \DUrole{default_value}{None}}, \emph{\DUrole{o}{**}\DUrole{n}{filters}}}{}
Levelized cost of energy (LCOE) of technologies over their lifetime.

It mostly follows the \sphinxtitleref{simplified LCOE} given by NREL. However, the units are
sometimes different. In the argument description, we use the following:
\begin{itemize}
\item {} 
{[}h{]}: hour

\item {} 
{[}y{]}: year

\item {} 
{[}\${]}: unit of currency

\item {} 
{[}E{]}: unit of energy

\item {} 
{[}1{]}: dimensionless

\end{itemize}
\begin{quote}\begin{description}
\item[{Parameters}] \leavevmode\begin{itemize}
\item {} 
\sphinxstyleliteralstrong{\sphinxupquote{prices}} \textendash{} {[}\$/(Eh){]} the price of all commodities, including consumables and fuels.
This dataarray contains at least timeslice and commodity dimensions.

\item {} 
\sphinxstyleliteralstrong{\sphinxupquote{technologies}} \textendash{} 
Describe the technologies, with at least the following parameters:
\begin{itemize}
\item {} 
technical life: {[}a{]} lifetime of each technology

\item {} 
cap\_par: {[}\$/E{]} overnight capital cost

\item {} 
interest\_rate: {[}1{]}

\item {} 
fix\_par: {[}\$/(Eh){]} fixed costs of operation and maintenance costs

\item {} 
var\_par: {[}\$/(Eh){]} variable costs of operation and maintenance costs

\item {} \begin{description}
\item[{fixed\_inputs: {[}1{]} == {[}(Eh)/(Eh){]} ratio indicating the amount of commodity}] \leavevmode
consumed per units of energy created.

\end{description}

\item {} \begin{description}
\item[{fixed\_outputs: {[}1{]} == {[}(Eh)/(Eh){]} ration indicating the amount of}] \leavevmode
environmental pollutants produced per units of energy created.

\end{description}

\end{itemize}


\item {} 
\sphinxstyleliteralstrong{\sphinxupquote{installation\_year}} \textendash{} year when the technologies are installed. If not given, it
defaults to the first year in \sphinxtitleref{prices}. This should be a single value, there
is currently no provision for computing LCOE over different installation
years.

\end{itemize}

\item[{Returns}] \leavevmode
The lifetime LCOE in {[}\$/(Eh){]} for each technology at each timeslice.

\end{description}\end{quote}

\end{fulllineitems}

\index{maximum\_production() (in module muse.quantities)@\spxentry{maximum\_production()}\spxextra{in module muse.quantities}}

\begin{fulllineitems}
\phantomsection\label{\detokenize{api:muse.quantities.maximum_production}}\pysiglinewithargsret{\sphinxcode{\sphinxupquote{muse.quantities.}}\sphinxbfcode{\sphinxupquote{maximum\_production}}}{\emph{\DUrole{n}{technologies}\DUrole{p}{:} \DUrole{n}{xarray.core.dataset.Dataset}}, \emph{\DUrole{n}{capacity}\DUrole{p}{:} \DUrole{n}{xarray.core.dataarray.DataArray}}, \emph{\DUrole{o}{**}\DUrole{n}{filters}}}{}
Production for a given capacity.

Given a capacity \(\mathcal{A}_{t, \iota}^r\), the utilization factor
\(\alpha^r_{t, \iota}\) and the the fixed outputs of each technology
\(\beta^r_{t, \iota, c}\), then the result production is:
\begin{equation*}
\begin{split}P_{t, \iota}^r =
    \alpha^r_{t, \iota}\beta^r_{t, \iota, c}\mathcal{A}_{t, \iota}^r\end{split}
\end{equation*}
The dimensions above are only indicative. The function should work with many
different input values, e.g. with capacities expanded over time\sphinxhyphen{}slices \(t\) or
agents \(i\).
\begin{quote}\begin{description}
\item[{Parameters}] \leavevmode\begin{itemize}
\item {} 
\sphinxstyleliteralstrong{\sphinxupquote{capacity}} \textendash{} Capacity of each technology of interest. In practice, the capacity can
refer to asset capacity, the max capacity, or the capacity\sphinxhyphen{}in\sphinxhyphen{}use.

\item {} 
\sphinxstyleliteralstrong{\sphinxupquote{technologies}} \textendash{} xr.Dataset describing the features of the technologies of
interests.  It should contain \sphinxtitleref{fixed\_outputs} and \sphinxtitleref{utilization\_factor}. It’s
shape is matched to \sphinxtitleref{capacity} using \sphinxtitleref{muse.utilities.broadcast\_techs}.

\item {} 
\sphinxstyleliteralstrong{\sphinxupquote{filters}} \textendash{} keyword arguments are used to filter down the capacity and
technologies. Filters not relevant to the quantities of interest, i.e.
filters that are not a dimension of \sphinxtitleref{capacity} or \sphinxtitleref{techologies}, are
silently ignored.

\end{itemize}

\item[{Returns}] \leavevmode
\sphinxtitleref{capacity * fixed\_outputs * utilization\_factor}, whittled down according to the
filters and the set of technologies in \sphinxtitleref{capacity}.

\end{description}\end{quote}

\end{fulllineitems}

\index{supply() (in module muse.quantities)@\spxentry{supply()}\spxextra{in module muse.quantities}}

\begin{fulllineitems}
\phantomsection\label{\detokenize{api:muse.quantities.supply}}\pysiglinewithargsret{\sphinxcode{\sphinxupquote{muse.quantities.}}\sphinxbfcode{\sphinxupquote{supply}}}{\emph{\DUrole{n}{capacity}\DUrole{p}{:} \DUrole{n}{xarray.core.dataarray.DataArray}}, \emph{\DUrole{n}{demand}\DUrole{p}{:} \DUrole{n}{xarray.core.dataarray.DataArray}}, \emph{\DUrole{n}{technologies}\DUrole{p}{:} \DUrole{n}{Union\DUrole{p}{{[}}xarray.core.dataset.Dataset\DUrole{p}{, }xarray.core.dataarray.DataArray\DUrole{p}{{]}}}}, \emph{\DUrole{n}{interpolation}\DUrole{p}{:} \DUrole{n}{str} \DUrole{o}{=} \DUrole{default_value}{\textquotesingle{}linear\textquotesingle{}}}, \emph{\DUrole{n}{production\_method}\DUrole{p}{:} \DUrole{n}{Optional\DUrole{p}{{[}}Callable\DUrole{p}{{]}}} \DUrole{o}{=} \DUrole{default_value}{None}}}{{ $\rightarrow$ xarray.core.dataarray.DataArray}}
Production and emission for a given capacity servicing a given demand.

Supply includes two components, end\sphinxhyphen{}uses outputs and environmental pollutants. The
former consists of the demand that the current capacity is capable of servicing.
Where there is excess capacity, then service is assigned to each asset a share of
the maximum production (e.g. utilization across similar assets is the same in
percentage). Then, environmental pollutants are computing as a function of
commodity outputs.
\begin{quote}\begin{description}
\item[{Parameters}] \leavevmode\begin{itemize}
\item {} 
\sphinxstyleliteralstrong{\sphinxupquote{capacity}} \textendash{} number/quantity of assets that can service the demand

\item {} 
\sphinxstyleliteralstrong{\sphinxupquote{demand}} \textendash{} amount of each end\sphinxhyphen{}use required. The supply of each process will not
exceed it’s share of the demand.

\item {} 
\sphinxstyleliteralstrong{\sphinxupquote{technologies}} \textendash{} factors bindings the capacity of an asset with its production of
commodities and environmental pollutants.

\end{itemize}

\item[{Returns}] \leavevmode
A data array where the commodity dimension only contains actual outputs (i.e. no
input commodities).

\end{description}\end{quote}

\end{fulllineitems}

\index{supply\_cost() (in module muse.quantities)@\spxentry{supply\_cost()}\spxextra{in module muse.quantities}}

\begin{fulllineitems}
\phantomsection\label{\detokenize{api:muse.quantities.supply_cost}}\pysiglinewithargsret{\sphinxcode{\sphinxupquote{muse.quantities.}}\sphinxbfcode{\sphinxupquote{supply\_cost}}}{\emph{\DUrole{n}{production}\DUrole{p}{:} \DUrole{n}{xarray.core.dataarray.DataArray}}, \emph{\DUrole{n}{lcoe}\DUrole{p}{:} \DUrole{n}{xarray.core.dataarray.DataArray}}, \emph{\DUrole{n}{asset\_dim}\DUrole{p}{:} \DUrole{n}{Optional\DUrole{p}{{[}}str\DUrole{p}{{]}}} \DUrole{o}{=} \DUrole{default_value}{\textquotesingle{}asset\textquotesingle{}}}}{{ $\rightarrow$ xarray.core.dataarray.DataArray}}
Supply cost given production and the levelized cost of energy.

In practice, the supply cost is the weighted average LCOE over assets (\sphinxtitleref{asset\_dim}),
where the weights are the production.
\begin{quote}\begin{description}
\item[{Parameters}] \leavevmode\begin{itemize}
\item {} 
\sphinxstyleliteralstrong{\sphinxupquote{production}} \textendash{} Amount of goods produced. In practice, production can be obtained
from the capacity for each asset via the method
\sphinxtitleref{muse.quantities.production}.

\item {} 
\sphinxstyleliteralstrong{\sphinxupquote{lcoe}} \textendash{} Levelized cost of energy for each good produced. In practice, it can be
obtained from market prices via
\sphinxtitleref{muse.quantities.annual\_levelized\_cost\_of\_energy} or
\sphinxtitleref{muse.quantities.lifetime\_levelized\_cost\_of\_energy}.

\item {} 
\sphinxstyleliteralstrong{\sphinxupquote{asset\_dim}} \textendash{} Name of the dimension(s) holding assets, processes or technologies.

\end{itemize}

\end{description}\end{quote}

\end{fulllineitems}



\section{Demand Matching Algorithm}
\label{\detokenize{api:module-muse.demand_matching}}\label{\detokenize{api:demand-matching-algorithm}}\index{module@\spxentry{module}!muse.demand\_matching@\spxentry{muse.demand\_matching}}\index{muse.demand\_matching@\spxentry{muse.demand\_matching}!module@\spxentry{module}}
Collection of demand\sphinxhyphen{}matching algorithms.

At it’s simplest, the demand matching algorithm solves the following problem,
\begin{itemize}
\item {} 
given a demand for a commodity \(D_d\), with \(d\in\mathcal{D}\)

\item {} 
given processes to supply these commodities, with an associated cost per process,
\(C_{d, i}\), with \(i\in\mathcal{I}\)

\end{itemize}

Match demand and supply while minimizing the associated cost.
\begin{align*}\!\begin{aligned}
\min_{X} \sum_{d, i} C_{d,i} X_{d, i}\\
X_{d, i} \geq 0\\
\sum_o X_o \geq D_d\\
\end{aligned}\end{align*}
The basic algorithm proceeds as follows:
\begin{enumerate}
\sphinxsetlistlabels{\arabic}{enumi}{enumii}{}{.}%
\item {} 
sort all costs \(C_{d, i}\) accross both \(d\) and \(i\)

\item {} 
for each cost \(c_0\) in order:
\begin{enumerate}
\sphinxsetlistlabels{\arabic}{enumii}{enumiii}{}{.}%
\item {} 
find the set of indices \(\mathcal{C}\subseteq\mathcal{D}\cup\mathcal{I}\)
for which
\begin{quote}
\begin{equation*}
\begin{split}\forall (d, i) \in \mathcal{C}\quad C_{d, i} == c_0\end{split}
\end{equation*}\end{quote}

\item {} 
determine the partial result for the current cost
\begin{quote}
\begin{equation*}
\begin{split}\forall (d, i) \in \mathcal{C}\quad X_{d, i} = \frac{D_d}{|i\in\mathcal{C}|}\end{split}
\end{equation*}
Where \(|i\in\mathcal{C}|\) indicates the number of indices \(i\) in
\(\mathcal{C}\).
\end{quote}

\end{enumerate}

\end{enumerate}

However, in practice, the problem to solve often contains constraints, e.g. a constraint
on production \(\sum_d X_{d, i} \leq M_i\). The algorithms in this module try and
solve these constrained problems one way or another.
\index{demand\_matching() (in module muse.demand\_matching)@\spxentry{demand\_matching()}\spxextra{in module muse.demand\_matching}}

\begin{fulllineitems}
\phantomsection\label{\detokenize{api:muse.demand_matching.demand_matching}}\pysiglinewithargsret{\sphinxcode{\sphinxupquote{muse.demand\_matching.}}\sphinxbfcode{\sphinxupquote{demand\_matching}}}{\emph{\DUrole{n}{demand}\DUrole{p}{:} \DUrole{n}{xarray.core.dataarray.DataArray}}, \emph{\DUrole{n}{cost}\DUrole{p}{:} \DUrole{n}{xarray.core.dataarray.DataArray}}, \emph{\DUrole{o}{*}\DUrole{n}{constraints}\DUrole{p}{:} \DUrole{n}{xarray.core.dataarray.DataArray}}, \emph{\DUrole{n}{protected\_dims}\DUrole{p}{:} \DUrole{n}{Optional\DUrole{p}{{[}}Set\DUrole{p}{{]}}} \DUrole{o}{=} \DUrole{default_value}{None}}}{{ $\rightarrow$ xarray.core.dataarray.DataArray}}
Demand matching over heterogenous dimensions.

This algorithm enables demand matching while enforcing constraints on how much an
asset can produce. Any set of dimensions can be matched. The algorithm is general
with respect to the dimensions in demand and cost. It also enforces constraints over
sets of indices.
\begin{align*}\!\begin{aligned}
\min_{X} \sum_{d, i} C_{d, i} X_{d, i}\\
X_{d, i} \geq 0\\
\sum_i X_{d, i} \geq D_d\\
M_{(d, i) \in \mathcal{R}^{(\alpha)}}^{(\alpha)}
    \geq \sum_{(d, i)\notin\mathcal{R}^{(\alpha)}} X_{d, i}\\
\end{aligned}\end{align*}
Where \(\alpha\) is an index running over constraints,
\(\mathcal{R}^{(\alpha)}\subseteq\mathcal{D}\cup\mathcal{I}\) is a subset of
indices.

The algorithm proceeds as described in \sphinxcode{\sphinxupquote{muse.demand\_matching}}.
However, an extra step is added to ensure that the solutions falls within the
convex\sphinxhyphen{}hull formed by the constraints. This projects the current solution onto the
constraint. Hence, the solution will depend on the order in which the constraints
are given.
\begin{enumerate}
\sphinxsetlistlabels{\arabic}{enumi}{enumii}{}{.}%
\item {} 
sort all costs \(C_{d, m}\) accross both \(d\) and \(m\)

\item {} 
for each cost \(c_0\) in order:
\begin{enumerate}
\sphinxsetlistlabels{\arabic}{enumii}{enumiii}{}{.}%
\item {} 
find the set of indices \(\mathcal{C}\)
\begin{quote}
\begin{align*}\!\begin{aligned}
\mathcal{C}\subseteq\mathcal{D}\cup\mathcal{I}\\
\forall (d, i) \in \mathcal{C}\quad C_{d, i} == c_0\\
\end{aligned}\end{align*}\end{quote}

\item {} 
determine an interim partial result for the current cost
\begin{quote}
\begin{equation*}
\begin{split}\forall (d, i) \in \mathcal{C}\quad
\delta X_{d, i} = \frac{1}{|i\in\mathcal{C}|}\left(
    D_d - \sum_{j \in \mathcal{I}} X_{d, j}\right)\end{split}
\end{equation*}
Where \(|i\in\mathcal{C}|\) indicates the number of \(i\) indices in
\(\mathcal{C}\). The expression in the parenthesis is the currently
unserviced demand.
\end{quote}

\item {} 
Loop over each constraint \(\alpha\). Below we drop the index
\(\alpha\) over constraints for simplicity.
\begin{enumerate}
\sphinxsetlistlabels{\arabic}{enumiii}{enumiv}{}{.}%
\item {} 
Determine the excess over the constraint:
\begin{quote}
\begin{equation*}
\begin{split}E_{(d, i) \in \mathcal{R}} = \max\left\{
    0,
    \sum_{(d, i)\notin\mathcal{R}}\left(
        X_{d, i} + \delta X_{d, i}
    \right) - M_{(d, i) \in \mathcal{R}}
\right\}\end{split}
\end{equation*}\end{quote}

\item {} 
Correct \(\delta X\) as follows:
\begin{quote}
\begin{align*}\!\begin{aligned}
\forall (d, i) \in \mathcal{C}\cap\mathcal{R}\quad
\delta X\prime_{d, i} =
    E_{(d, i)}
    \frac{\delta X_{(d, i)}}{
        \sum_{(e, j)\in \mathcal{C}\cap\mathcal{R}} \delta X_{(e,j)}
    }\\
\forall (d, i) \notin \mathcal{R}, (d, i)\in\mathcal{C} \quad
\delta X\prime_{d, i} = 0\\
\end{aligned}\end{align*}\end{quote}

\item {} 
Set \(\delta X = \max(0, \delta X - \delta X\prime)\)

\end{enumerate}

\end{enumerate}

\end{enumerate}

A more complex problem would see independant dimensions for each quantity. In that,
case we can reduce to the original problem as shown here
\begin{align*}\!\begin{aligned}
C_{d, i, c} = \min_cC\prime_{d, i, c}\\
D_d = \sum_{d\prime} D\prime_{d, d\prime}\\
M_r = \sum_m M\prime_{r, m}\\
X_{d, d\prime, i, m, c} =
    \left(C\prime_{d, i, c} == C_{d, i}\right)
    \frac{M\prime_{r, m}}{M_r} \frac{D\prime_{d, d\prime}}{D_d} X_{d, i}\\
\end{aligned}\end{align*}
A dimension could be shared by all quantities, in which case each point along that
dimension is treated as independant.

Similarly, if a dimension is shared only by the demand and a constraint but not by
the cost, then the problem can be reduced a set of problems independant along that
direction.
\begin{quote}\begin{description}
\item[{Parameters}] \leavevmode\begin{itemize}
\item {} 
\sphinxstyleliteralstrong{\sphinxupquote{demand}} \textendash{} Demand to match with production. It should have the same physical units
as \sphinxtitleref{max\_production}.

\item {} 
\sphinxstyleliteralstrong{\sphinxupquote{cost}} \textendash{} Cost to minimize while fulfiling the demand.

\item {} 
\sphinxstyleliteralstrong{\sphinxupquote{*constraints}} \textendash{} each item is a seperate constraint \(M_r\).

\end{itemize}

\item[{Returns}] \leavevmode
An array with the joint dimensionality of \sphinxtitleref{max\_production}, \sphinxtitleref{cost}, and
\sphinxtitleref{demand}, containing the supply that fulfills the demand. The units of this
supply are the same as \sphinxtitleref{demand} and \sphinxtitleref{max\_production}.

\end{description}\end{quote}

\end{fulllineitems}



\section{Miscellaneous}
\label{\detokenize{api:miscellaneous}}

\subsection{Timeslices}
\label{\detokenize{api:module-muse.timeslices}}\label{\detokenize{api:timeslices}}\index{module@\spxentry{module}!muse.timeslices@\spxentry{muse.timeslices}}\index{muse.timeslices@\spxentry{muse.timeslices}!module@\spxentry{module}}
Timeslice utility functions.
\index{aggregate\_transforms() (in module muse.timeslices)@\spxentry{aggregate\_transforms()}\spxextra{in module muse.timeslices}}

\begin{fulllineitems}
\phantomsection\label{\detokenize{api:muse.timeslices.aggregate_transforms}}\pysiglinewithargsret{\sphinxcode{\sphinxupquote{muse.timeslices.}}\sphinxbfcode{\sphinxupquote{aggregate\_transforms}}}{\emph{\DUrole{n}{settings}\DUrole{p}{:} \DUrole{n}{Optional\DUrole{p}{{[}}Union\DUrole{p}{{[}}Mapping\DUrole{p}{, }str\DUrole{p}{{]}}\DUrole{p}{{]}}} \DUrole{o}{=} \DUrole{default_value}{None}}, \emph{\DUrole{n}{timeslice}\DUrole{p}{:} \DUrole{n}{Optional\DUrole{p}{{[}}xarray.core.dataarray.DataArray\DUrole{p}{{]}}} \DUrole{o}{=} \DUrole{default_value}{None}}}{{ $\rightarrow$ Dict\DUrole{p}{{[}}Tuple\DUrole{p}{, }numpy.ndarray\DUrole{p}{{]}}}}
Creates dictionay of transforms for aggregate levels.

The transforms are used to create the projectors towards the finest timeslice.
\begin{quote}\begin{description}
\item[{Parameters}] \leavevmode\begin{itemize}
\item {} 
\sphinxstyleliteralstrong{\sphinxupquote{timeslice}} \textendash{} a \sphinxcode{\sphinxupquote{DataArray}} with the timeslice dimension.

\item {} 
\sphinxstyleliteralstrong{\sphinxupquote{settings}} \textendash{} A dictionary mapping the name of an aggregate with the values it
aggregates, or a string that toml will parse as such. If not given, only the
unit transforms are returned.

\end{itemize}

\item[{Returns}] \leavevmode
A dictionary of transforms for each possible slice to it’s corresponding finest
timeslices.

\end{description}\end{quote}
\subsubsection*{Example}

\begin{sphinxVerbatim}[commandchars=\\\{\}]
\PYG{g+gp}{\PYGZgt{}\PYGZgt{}\PYGZgt{} }\PYG{n}{toml} \PYG{o}{=} \PYG{l+s+s2}{\PYGZdq{}\PYGZdq{}\PYGZdq{}}
\PYG{g+gp}{... }\PYG{l+s+s2}{    [timeslices]}
\PYG{g+gp}{... }\PYG{l+s+s2}{    spring.weekday = 5}
\PYG{g+gp}{... }\PYG{l+s+s2}{    spring.weekend = 2}
\PYG{g+gp}{... }\PYG{l+s+s2}{    autumn.weekday = 5}
\PYG{g+gp}{... }\PYG{l+s+s2}{    autumn.weekend = 2}
\PYG{g+gp}{... }\PYG{l+s+s2}{    winter.weekday = 5}
\PYG{g+gp}{... }\PYG{l+s+s2}{    winter.weekend = 2}
\PYG{g+gp}{... }\PYG{l+s+s2}{    summer.weekday = 5}
\PYG{g+gp}{... }\PYG{l+s+s2}{    summer.weekend = 2}
\PYG{g+gp}{...}
\PYG{g+gp}{... }\PYG{l+s+s2}{    [timeslices.aggregates]}
\PYG{g+gp}{... }\PYG{l+s+s2}{    spautumn = [}\PYG{l+s+s2}{\PYGZdq{}}\PYG{l+s+s2}{spring}\PYG{l+s+s2}{\PYGZdq{}}\PYG{l+s+s2}{, }\PYG{l+s+s2}{\PYGZdq{}}\PYG{l+s+s2}{autumn}\PYG{l+s+s2}{\PYGZdq{}}\PYG{l+s+s2}{]}
\PYG{g+gp}{... }\PYG{l+s+s2}{    week = [}\PYG{l+s+s2}{\PYGZdq{}}\PYG{l+s+s2}{weekday}\PYG{l+s+s2}{\PYGZdq{}}\PYG{l+s+s2}{, }\PYG{l+s+s2}{\PYGZdq{}}\PYG{l+s+s2}{weekend}\PYG{l+s+s2}{\PYGZdq{}}\PYG{l+s+s2}{]}
\PYG{g+gp}{... }\PYG{l+s+s2}{\PYGZdq{}\PYGZdq{}\PYGZdq{}}
\PYG{g+gp}{\PYGZgt{}\PYGZgt{}\PYGZgt{} }\PYG{k+kn}{from} \PYG{n+nn}{muse}\PYG{n+nn}{.}\PYG{n+nn}{timeslices} \PYG{k+kn}{import} \PYG{n}{reference\PYGZus{}timeslice}\PYG{p}{,} \PYG{n}{aggregate\PYGZus{}transforms}
\PYG{g+gp}{\PYGZgt{}\PYGZgt{}\PYGZgt{} }\PYG{n}{ref} \PYG{o}{=} \PYG{n}{reference\PYGZus{}timeslice}\PYG{p}{(}\PYG{n}{toml}\PYG{p}{)}
\PYG{g+gp}{\PYGZgt{}\PYGZgt{}\PYGZgt{} }\PYG{n}{transforms} \PYG{o}{=} \PYG{n}{aggregate\PYGZus{}transforms}\PYG{p}{(}\PYG{n}{toml}\PYG{p}{,} \PYG{n}{ref}\PYG{p}{)}
\PYG{g+gp}{\PYGZgt{}\PYGZgt{}\PYGZgt{} }\PYG{n}{transforms}\PYG{p}{[}\PYG{p}{(}\PYG{l+s+s2}{\PYGZdq{}}\PYG{l+s+s2}{spring}\PYG{l+s+s2}{\PYGZdq{}}\PYG{p}{,} \PYG{l+s+s2}{\PYGZdq{}}\PYG{l+s+s2}{weekend}\PYG{l+s+s2}{\PYGZdq{}}\PYG{p}{)}\PYG{p}{]}
\PYG{g+go}{array([0, 1, 0, 0, 0, 0, 0, 0])}
\PYG{g+gp}{\PYGZgt{}\PYGZgt{}\PYGZgt{} }\PYG{n}{transforms}\PYG{p}{[}\PYG{p}{(}\PYG{l+s+s2}{\PYGZdq{}}\PYG{l+s+s2}{spautumn}\PYG{l+s+s2}{\PYGZdq{}}\PYG{p}{,} \PYG{l+s+s2}{\PYGZdq{}}\PYG{l+s+s2}{weekday}\PYG{l+s+s2}{\PYGZdq{}}\PYG{p}{)}\PYG{p}{]}
\PYG{g+go}{array([1, 0, 1, 0, 0, 0, 0, 0])}
\PYG{g+gp}{\PYGZgt{}\PYGZgt{}\PYGZgt{} }\PYG{n}{transforms}\PYG{p}{[}\PYG{p}{(}\PYG{l+s+s2}{\PYGZdq{}}\PYG{l+s+s2}{autumn}\PYG{l+s+s2}{\PYGZdq{}}\PYG{p}{,} \PYG{l+s+s2}{\PYGZdq{}}\PYG{l+s+s2}{week}\PYG{l+s+s2}{\PYGZdq{}}\PYG{p}{)}\PYG{p}{]}\PYG{o}{.}\PYG{n}{T}
\PYG{g+go}{array([0, 0, 1, 1, 0, 0, 0, 0])}
\PYG{g+gp}{\PYGZgt{}\PYGZgt{}\PYGZgt{} }\PYG{n}{transforms}\PYG{p}{[}\PYG{p}{(}\PYG{l+s+s2}{\PYGZdq{}}\PYG{l+s+s2}{spautumn}\PYG{l+s+s2}{\PYGZdq{}}\PYG{p}{,} \PYG{l+s+s2}{\PYGZdq{}}\PYG{l+s+s2}{week}\PYG{l+s+s2}{\PYGZdq{}}\PYG{p}{)}\PYG{p}{]}\PYG{o}{.}\PYG{n}{T}
\PYG{g+go}{array([1, 1, 1, 1, 0, 0, 0, 0])}
\end{sphinxVerbatim}

\end{fulllineitems}

\index{convert\_timeslice() (in module muse.timeslices)@\spxentry{convert\_timeslice()}\spxextra{in module muse.timeslices}}

\begin{fulllineitems}
\phantomsection\label{\detokenize{api:muse.timeslices.convert_timeslice}}\pysiglinewithargsret{\sphinxcode{\sphinxupquote{muse.timeslices.}}\sphinxbfcode{\sphinxupquote{convert\_timeslice}}}{\emph{x: Union{[}xarray.core.dataarray.DataArray, xarray.core.dataset.Dataset{]}, ts: Union{[}xarray.core.dataarray.DataArray, xarray.core.dataset.Dataset, pandas.core.indexes.multi.MultiIndex{]}, quantity: Union{[}muse.timeslices.QuantityType, str{]} = \textless{}QuantityType.EXTENSIVE: \textquotesingle{}extensive\textquotesingle{}\textgreater{}, finest: Optional{[}xarray.core.dataarray.DataArray{]} = None, transforms: Optional{[}Dict{[}Tuple, numpy.ndarray{]}{]} = None}}{{ $\rightarrow$ Union\DUrole{p}{{[}}xarray.core.dataarray.DataArray\DUrole{p}{, }xarray.core.dataset.Dataset\DUrole{p}{{]}}}}
Adjusts the timeslice of x to match that of ts.

The conversion can be done in on of two ways, depending on whether the
quantity is extensive or intensive. See \sphinxtitleref{QuantityType}.
\subsubsection*{Example}

Lets define three timeslices from finest, to fine, to rough:

\begin{sphinxVerbatim}[commandchars=\\\{\}]
\PYG{g+gp}{\PYGZgt{}\PYGZgt{}\PYGZgt{} }\PYG{n}{toml} \PYG{o}{=} \PYG{l+s+s2}{\PYGZdq{}\PYGZdq{}\PYGZdq{}}
\PYG{g+gp}{... }\PYG{l+s+s2}{    [}\PYG{l+s+s2}{\PYGZdq{}}\PYG{l+s+s2}{timeslices}\PYG{l+s+s2}{\PYGZdq{}}\PYG{l+s+s2}{]}
\PYG{g+gp}{... }\PYG{l+s+s2}{    winter.weekday.day = 5}
\PYG{g+gp}{... }\PYG{l+s+s2}{    winter.weekday.night = 5}
\PYG{g+gp}{... }\PYG{l+s+s2}{    winter.weekend.day = 2}
\PYG{g+gp}{... }\PYG{l+s+s2}{    winter.weekend.night = 2}
\PYG{g+gp}{... }\PYG{l+s+s2}{    summer.weekday.day = 5}
\PYG{g+gp}{... }\PYG{l+s+s2}{    summer.weekday.night = 5}
\PYG{g+gp}{... }\PYG{l+s+s2}{    summer.weekend.day = 2}
\PYG{g+gp}{... }\PYG{l+s+s2}{    summer.weekend.night = 2}
\PYG{g+gp}{... }\PYG{l+s+s2}{    level\PYGZus{}names = [}\PYG{l+s+s2}{\PYGZdq{}}\PYG{l+s+s2}{semester}\PYG{l+s+s2}{\PYGZdq{}}\PYG{l+s+s2}{, }\PYG{l+s+s2}{\PYGZdq{}}\PYG{l+s+s2}{week}\PYG{l+s+s2}{\PYGZdq{}}\PYG{l+s+s2}{, }\PYG{l+s+s2}{\PYGZdq{}}\PYG{l+s+s2}{day}\PYG{l+s+s2}{\PYGZdq{}}\PYG{l+s+s2}{]}
\PYG{g+gp}{... }\PYG{l+s+s2}{    aggregates.allday = [}\PYG{l+s+s2}{\PYGZdq{}}\PYG{l+s+s2}{day}\PYG{l+s+s2}{\PYGZdq{}}\PYG{l+s+s2}{, }\PYG{l+s+s2}{\PYGZdq{}}\PYG{l+s+s2}{night}\PYG{l+s+s2}{\PYGZdq{}}\PYG{l+s+s2}{]}
\PYG{g+gp}{... }\PYG{l+s+s2}{    aggregates.allweek = [}\PYG{l+s+s2}{\PYGZdq{}}\PYG{l+s+s2}{weekend}\PYG{l+s+s2}{\PYGZdq{}}\PYG{l+s+s2}{, }\PYG{l+s+s2}{\PYGZdq{}}\PYG{l+s+s2}{weekday}\PYG{l+s+s2}{\PYGZdq{}}\PYG{l+s+s2}{]}
\PYG{g+gp}{... }\PYG{l+s+s2}{    aggregates.allyear = [}\PYG{l+s+s2}{\PYGZdq{}}\PYG{l+s+s2}{winter}\PYG{l+s+s2}{\PYGZdq{}}\PYG{l+s+s2}{, }\PYG{l+s+s2}{\PYGZdq{}}\PYG{l+s+s2}{summer}\PYG{l+s+s2}{\PYGZdq{}}\PYG{l+s+s2}{]}
\PYG{g+gp}{... }\PYG{l+s+s2}{\PYGZdq{}\PYGZdq{}\PYGZdq{}}
\PYG{g+gp}{\PYGZgt{}\PYGZgt{}\PYGZgt{} }\PYG{k+kn}{from} \PYG{n+nn}{muse}\PYG{n+nn}{.}\PYG{n+nn}{timeslices} \PYG{k+kn}{import} \PYG{n}{setup\PYGZus{}module}
\PYG{g+gp}{\PYGZgt{}\PYGZgt{}\PYGZgt{} }\PYG{k+kn}{from} \PYG{n+nn}{muse}\PYG{n+nn}{.}\PYG{n+nn}{readers} \PYG{k+kn}{import} \PYG{n}{read\PYGZus{}timeslices}
\PYG{g+gp}{\PYGZgt{}\PYGZgt{}\PYGZgt{} }\PYG{n}{setup\PYGZus{}module}\PYG{p}{(}\PYG{n}{toml}\PYG{p}{)}
\PYG{g+gp}{\PYGZgt{}\PYGZgt{}\PYGZgt{} }\PYG{n}{finest\PYGZus{}ts} \PYG{o}{=} \PYG{n}{read\PYGZus{}timeslices}\PYG{p}{(}\PYG{p}{)}
\PYG{g+gp}{\PYGZgt{}\PYGZgt{}\PYGZgt{} }\PYG{n}{fine\PYGZus{}ts} \PYG{o}{=} \PYG{n}{read\PYGZus{}timeslices}\PYG{p}{(}\PYG{n+nb}{dict}\PYG{p}{(}\PYG{n}{week}\PYG{o}{=}\PYG{p}{[}\PYG{l+s+s2}{\PYGZdq{}}\PYG{l+s+s2}{allweek}\PYG{l+s+s2}{\PYGZdq{}}\PYG{p}{]}\PYG{p}{)}\PYG{p}{)}
\PYG{g+gp}{\PYGZgt{}\PYGZgt{}\PYGZgt{} }\PYG{n}{rough\PYGZus{}ts} \PYG{o}{=} \PYG{n}{read\PYGZus{}timeslices}\PYG{p}{(}\PYG{n+nb}{dict}\PYG{p}{(}\PYG{n}{semester}\PYG{o}{=}\PYG{p}{[}\PYG{l+s+s2}{\PYGZdq{}}\PYG{l+s+s2}{allyear}\PYG{l+s+s2}{\PYGZdq{}}\PYG{p}{]}\PYG{p}{,} \PYG{n}{day}\PYG{o}{=}\PYG{p}{[}\PYG{l+s+s2}{\PYGZdq{}}\PYG{l+s+s2}{allday}\PYG{l+s+s2}{\PYGZdq{}}\PYG{p}{]}\PYG{p}{)}\PYG{p}{)}
\end{sphinxVerbatim}

Lets also define to other data\sphinxhyphen{}arrays to demonstrate how we can play with
dimensions:

\begin{sphinxVerbatim}[commandchars=\\\{\}]
\PYG{g+gp}{\PYGZgt{}\PYGZgt{}\PYGZgt{} }\PYG{k+kn}{from} \PYG{n+nn}{numpy} \PYG{k+kn}{import} \PYG{n}{array}
\PYG{g+gp}{\PYGZgt{}\PYGZgt{}\PYGZgt{} }\PYG{n}{x} \PYG{o}{=} \PYG{n}{DataArray}\PYG{p}{(}
\PYG{g+gp}{... }    \PYG{p}{[}\PYG{l+m+mi}{5}\PYG{p}{,} \PYG{l+m+mi}{2}\PYG{p}{,} \PYG{l+m+mi}{3}\PYG{p}{]}\PYG{p}{,}
\PYG{g+gp}{... }    \PYG{n}{coords}\PYG{o}{=}\PYG{p}{\PYGZob{}}\PYG{l+s+s1}{\PYGZsq{}}\PYG{l+s+s1}{a}\PYG{l+s+s1}{\PYGZsq{}}\PYG{p}{:} \PYG{n}{array}\PYG{p}{(}\PYG{p}{[}\PYG{l+m+mi}{1}\PYG{p}{,} \PYG{l+m+mi}{2}\PYG{p}{,} \PYG{l+m+mi}{3}\PYG{p}{]}\PYG{p}{,} \PYG{n}{dtype}\PYG{o}{=}\PYG{l+s+s2}{\PYGZdq{}}\PYG{l+s+s2}{int64}\PYG{l+s+s2}{\PYGZdq{}}\PYG{p}{)}\PYG{p}{\PYGZcb{}}\PYG{p}{,}
\PYG{g+gp}{... }    \PYG{n}{dims}\PYG{o}{=}\PYG{l+s+s1}{\PYGZsq{}}\PYG{l+s+s1}{a}\PYG{l+s+s1}{\PYGZsq{}}
\PYG{g+gp}{... }\PYG{p}{)}
\PYG{g+gp}{\PYGZgt{}\PYGZgt{}\PYGZgt{} }\PYG{n}{y} \PYG{o}{=} \PYG{n}{DataArray}\PYG{p}{(}\PYG{p}{[}\PYG{l+m+mi}{1}\PYG{p}{,} \PYG{l+m+mi}{1}\PYG{p}{,} \PYG{l+m+mi}{2}\PYG{p}{]}\PYG{p}{,} \PYG{n}{coords}\PYG{o}{=}\PYG{p}{\PYGZob{}}\PYG{l+s+s1}{\PYGZsq{}}\PYG{l+s+s1}{b}\PYG{l+s+s1}{\PYGZsq{}}\PYG{p}{:} \PYG{p}{[}\PYG{l+s+s2}{\PYGZdq{}}\PYG{l+s+s2}{d}\PYG{l+s+s2}{\PYGZdq{}}\PYG{p}{,} \PYG{l+s+s2}{\PYGZdq{}}\PYG{l+s+s2}{e}\PYG{l+s+s2}{\PYGZdq{}}\PYG{p}{,} \PYG{l+s+s2}{\PYGZdq{}}\PYG{l+s+s2}{f}\PYG{l+s+s2}{\PYGZdq{}}\PYG{p}{]}\PYG{p}{\PYGZcb{}}\PYG{p}{,} \PYG{n}{dims}\PYG{o}{=}\PYG{l+s+s1}{\PYGZsq{}}\PYG{l+s+s1}{b}\PYG{l+s+s1}{\PYGZsq{}}\PYG{p}{)}
\end{sphinxVerbatim}

We can now easily convert arrays with different dimensions. First, lets check
conversion from an array with no timeslices:

\begin{sphinxVerbatim}[commandchars=\\\{\}]
\PYG{g+gp}{\PYGZgt{}\PYGZgt{}\PYGZgt{} }\PYG{k+kn}{from} \PYG{n+nn}{xarray} \PYG{k+kn}{import} \PYG{n}{ones\PYGZus{}like}
\PYG{g+gp}{\PYGZgt{}\PYGZgt{}\PYGZgt{} }\PYG{k+kn}{from} \PYG{n+nn}{muse}\PYG{n+nn}{.}\PYG{n+nn}{timeslices} \PYG{k+kn}{import} \PYG{n}{convert\PYGZus{}timeslice}\PYG{p}{,} \PYG{n}{QuantityType}
\PYG{g+gp}{\PYGZgt{}\PYGZgt{}\PYGZgt{} }\PYG{n}{z} \PYG{o}{=} \PYG{n}{convert\PYGZus{}timeslice}\PYG{p}{(}\PYG{n}{x}\PYG{p}{,} \PYG{n}{finest\PYGZus{}ts}\PYG{p}{,} \PYG{n}{QuantityType}\PYG{o}{.}\PYG{n}{EXTENSIVE}\PYG{p}{)}
\PYG{g+gp}{\PYGZgt{}\PYGZgt{}\PYGZgt{} }\PYG{n}{z}\PYG{o}{.}\PYG{n}{round}\PYG{p}{(}\PYG{l+m+mi}{6}\PYG{p}{)}
\PYG{g+go}{\PYGZlt{}xarray.DataArray (timeslice: 8, a: 3)\PYGZgt{}}
\PYG{g+go}{array([[0.892857, 0.357143, 0.535714],}
\PYG{g+go}{       [0.892857, 0.357143, 0.535714],}
\PYG{g+go}{       [0.357143, 0.142857, 0.214286],}
\PYG{g+go}{       [0.357143, 0.142857, 0.214286],}
\PYG{g+go}{       [0.892857, 0.357143, 0.535714],}
\PYG{g+go}{       [0.892857, 0.357143, 0.535714],}
\PYG{g+go}{       [0.357143, 0.142857, 0.214286],}
\PYG{g+go}{       [0.357143, 0.142857, 0.214286]])}
\PYG{g+go}{Coordinates:}
\PYG{g+go}{  * timeslice  (timeslice) MultiIndex}
\PYG{g+go}{  \PYGZhy{} semester   (timeslice) object \PYGZsq{}winter\PYGZsq{} \PYGZsq{}winter\PYGZsq{} ... \PYGZsq{}summer\PYGZsq{} \PYGZsq{}summer\PYGZsq{}}
\PYG{g+go}{  \PYGZhy{} week       (timeslice) object \PYGZsq{}weekday\PYGZsq{} \PYGZsq{}weekday\PYGZsq{} ... \PYGZsq{}weekend\PYGZsq{} \PYGZsq{}weekend\PYGZsq{}}
\PYG{g+go}{  \PYGZhy{} day        (timeslice) object \PYGZsq{}day\PYGZsq{} \PYGZsq{}night\PYGZsq{} \PYGZsq{}day\PYGZsq{} ... \PYGZsq{}night\PYGZsq{} \PYGZsq{}day\PYGZsq{} \PYGZsq{}night\PYGZsq{}}
\PYG{g+go}{  * a          (a) int64 1 2 3}
\PYG{g+gp}{\PYGZgt{}\PYGZgt{}\PYGZgt{} }\PYG{n}{z}\PYG{o}{.}\PYG{n}{sum}\PYG{p}{(}\PYG{l+s+s2}{\PYGZdq{}}\PYG{l+s+s2}{timeslice}\PYG{l+s+s2}{\PYGZdq{}}\PYG{p}{)}
\PYG{g+go}{\PYGZlt{}xarray.DataArray (a: 3)\PYGZgt{}}
\PYG{g+go}{array([5., 2., 3.])}
\PYG{g+go}{Coordinates:}
\PYG{g+go}{  * a        (a) int64 1 2 3}
\end{sphinxVerbatim}

As expected, the sum over timeslices recovers the original array.

In the case of an intensive quantity without a timeslice dimension, the
operation does not do anything:

\begin{sphinxVerbatim}[commandchars=\\\{\}]
\PYG{g+gp}{\PYGZgt{}\PYGZgt{}\PYGZgt{} }\PYG{n}{convert\PYGZus{}timeslice}\PYG{p}{(}\PYG{p}{[}\PYG{l+m+mi}{1}\PYG{p}{,} \PYG{l+m+mi}{2}\PYG{p}{]}\PYG{p}{,} \PYG{n}{rough\PYGZus{}ts}\PYG{p}{,} \PYG{n}{QuantityType}\PYG{o}{.}\PYG{n}{INTENSIVE}\PYG{p}{)}
\PYG{g+go}{[1, 2]}
\end{sphinxVerbatim}

More interesting is the conversion between different timeslices:

\begin{sphinxVerbatim}[commandchars=\\\{\}]
\PYG{g+gp}{\PYGZgt{}\PYGZgt{}\PYGZgt{} }\PYG{k+kn}{from} \PYG{n+nn}{xarray} \PYG{k+kn}{import} \PYG{n}{zeros\PYGZus{}like}
\PYG{g+gp}{\PYGZgt{}\PYGZgt{}\PYGZgt{} }\PYG{n}{zfine} \PYG{o}{=} \PYG{n}{x} \PYG{o}{+} \PYG{n}{y} \PYG{o}{+} \PYG{n}{zeros\PYGZus{}like}\PYG{p}{(}\PYG{n}{fine\PYGZus{}ts}\PYG{o}{.}\PYG{n}{timeslice}\PYG{p}{,} \PYG{n}{dtype}\PYG{o}{=}\PYG{n+nb}{int}\PYG{p}{)}
\PYG{g+gp}{\PYGZgt{}\PYGZgt{}\PYGZgt{} }\PYG{n}{zrough} \PYG{o}{=} \PYG{n}{convert\PYGZus{}timeslice}\PYG{p}{(}\PYG{n}{zfine}\PYG{p}{,} \PYG{n}{rough\PYGZus{}ts}\PYG{p}{)}
\PYG{g+gp}{\PYGZgt{}\PYGZgt{}\PYGZgt{} }\PYG{n}{zrough}\PYG{o}{.}\PYG{n}{round}\PYG{p}{(}\PYG{l+m+mi}{6}\PYG{p}{)}
\PYG{g+go}{\PYGZlt{}xarray.DataArray (timeslice: 2, a: 3, b: 3)\PYGZgt{}}
\PYG{g+go}{array([[[17.142857, 17.142857, 20.      ],}
\PYG{g+go}{        [ 8.571429,  8.571429, 11.428571],}
\PYG{g+go}{        [11.428571, 11.428571, 14.285714]],}

\PYG{g+go}{       [[ 6.857143,  6.857143,  8.      ],}
\PYG{g+go}{        [ 3.428571,  3.428571,  4.571429],}
\PYG{g+go}{        [ 4.571429,  4.571429,  5.714286]]])}
\PYG{g+go}{Coordinates:}
\PYG{g+go}{  * timeslice  (timeslice) MultiIndex}
\PYG{g+go}{  \PYGZhy{} semester   (timeslice) object \PYGZsq{}allyear\PYGZsq{} \PYGZsq{}allyear\PYGZsq{}}
\PYG{g+go}{  \PYGZhy{} week       (timeslice) object \PYGZsq{}weekday\PYGZsq{} \PYGZsq{}weekend\PYGZsq{}}
\PYG{g+go}{  \PYGZhy{} day        (timeslice) object \PYGZsq{}allday\PYGZsq{} \PYGZsq{}allday\PYGZsq{}}
\PYG{g+go}{  * a          (a) int64 1 2 3}
\PYG{g+go}{  * b          (b) \PYGZlt{}U1 \PYGZsq{}d\PYGZsq{} \PYGZsq{}e\PYGZsq{} \PYGZsq{}f\PYGZsq{}}
\end{sphinxVerbatim}

We can check that nothing has been added to z (the quantity is \sphinxcode{\sphinxupquote{EXTENSIVE}} by
default):

\begin{sphinxVerbatim}[commandchars=\\\{\}]
\PYG{g+gp}{\PYGZgt{}\PYGZgt{}\PYGZgt{} }\PYG{k+kn}{from} \PYG{n+nn}{numpy} \PYG{k+kn}{import} \PYG{n+nb}{all}
\PYG{g+gp}{\PYGZgt{}\PYGZgt{}\PYGZgt{} }\PYG{n+nb}{all}\PYG{p}{(}\PYG{n}{zfine}\PYG{o}{.}\PYG{n}{sum}\PYG{p}{(}\PYG{l+s+s2}{\PYGZdq{}}\PYG{l+s+s2}{timeslice}\PYG{l+s+s2}{\PYGZdq{}}\PYG{p}{)}\PYG{o}{.}\PYG{n}{round}\PYG{p}{(}\PYG{l+m+mi}{6}\PYG{p}{)} \PYG{o}{==} \PYG{n}{zrough}\PYG{o}{.}\PYG{n}{sum}\PYG{p}{(}\PYG{l+s+s2}{\PYGZdq{}}\PYG{l+s+s2}{timeslice}\PYG{l+s+s2}{\PYGZdq{}}\PYG{p}{)}\PYG{o}{.}\PYG{n}{round}\PYG{p}{(}\PYG{l+m+mi}{6}\PYG{p}{)}\PYG{p}{)}
\PYG{g+go}{\PYGZlt{}xarray.DataArray ()\PYGZgt{}}
\PYG{g+go}{array(True)}
\end{sphinxVerbatim}

Or that the ratio of weekdays to weekends makes sense:
\textgreater{}\textgreater{}\textgreater{} weekdays = (
…    zrough
…    .unstack(“timeslice”)
…    .sel(week=”weekday”)
…    .stack(timeslice={[}“semester”, “day”{]})
…    .squeeze()
… )
\textgreater{}\textgreater{}\textgreater{} weekend = (
…    zrough
…    .unstack(“timeslice”)
…    .sel(week=”weekend”)
…    .stack(timeslice={[}“semester”, “day”{]})
…    .squeeze()
… )
\textgreater{}\textgreater{}\textgreater{} bool(all((weekend * 5).round(6) == (weekdays * 2).round(6)))
True

\end{fulllineitems}

\index{reference\_timeslice() (in module muse.timeslices)@\spxentry{reference\_timeslice()}\spxextra{in module muse.timeslices}}

\begin{fulllineitems}
\phantomsection\label{\detokenize{api:muse.timeslices.reference_timeslice}}\pysiglinewithargsret{\sphinxcode{\sphinxupquote{muse.timeslices.}}\sphinxbfcode{\sphinxupquote{reference\_timeslice}}}{\emph{\DUrole{n}{settings}\DUrole{p}{:} \DUrole{n}{Union\DUrole{p}{{[}}Mapping\DUrole{p}{, }str\DUrole{p}{{]}}}}, \emph{\DUrole{n}{level\_names}\DUrole{p}{:} \DUrole{n}{Sequence\DUrole{p}{{[}}str\DUrole{p}{{]}}} \DUrole{o}{=} \DUrole{default_value}{\textquotesingle{}month\textquotesingle{}, \textquotesingle{}day\textquotesingle{}, \textquotesingle{}hour\textquotesingle{}}}, \emph{\DUrole{n}{name}\DUrole{p}{:} \DUrole{n}{str} \DUrole{o}{=} \DUrole{default_value}{\textquotesingle{}timeslice\textquotesingle{}}}}{{ $\rightarrow$ xarray.core.dataarray.DataArray}}
Reads reference timeslice from toml like input.
\begin{quote}\begin{description}
\item[{Parameters}] \leavevmode\begin{itemize}
\item {} 
\sphinxstyleliteralstrong{\sphinxupquote{settings}} \textendash{} A dictionary of nested dictionaries or a string that toml will
interpret as such. The nesting specifies different levels of the timeslice.
If a dictionary and it contains “timeslices” key, then the associated value
is used as the root dictionary. Ultimately, the most nested values should be
relative weights for each slice in the timeslice, e.g. the corresponding
number of hours.

\item {} 
\sphinxstyleliteralstrong{\sphinxupquote{level\_names}} \textendash{} Hints indicating the names of each level. Can also be given a
“level\_names” key in \sphinxcode{\sphinxupquote{settings}}.

\item {} 
\sphinxstyleliteralstrong{\sphinxupquote{name}} \textendash{} name of the reference array

\end{itemize}

\item[{Returns}] \leavevmode
A \sphinxcode{\sphinxupquote{DataArray}} with dimension \sphinxstyleemphasis{timeslice} and values representing the relative
weight of each timeslice.

\end{description}\end{quote}
\subsubsection*{Example}

\begin{sphinxVerbatim}[commandchars=\\\{\}]
\PYG{g+gp}{\PYGZgt{}\PYGZgt{}\PYGZgt{} }\PYG{k+kn}{from} \PYG{n+nn}{muse}\PYG{n+nn}{.}\PYG{n+nn}{timeslices} \PYG{k+kn}{import} \PYG{n}{reference\PYGZus{}timeslice}
\PYG{g+gp}{\PYGZgt{}\PYGZgt{}\PYGZgt{} }\PYG{n}{reference\PYGZus{}timeslice}\PYG{p}{(}
\PYG{g+gp}{... }    \PYG{l+s+sd}{\PYGZdq{}\PYGZdq{}\PYGZdq{}}
\PYG{g+gp}{... }\PYG{l+s+sd}{    [timeslices]}
\PYG{g+gp}{... }\PYG{l+s+sd}{    spring.weekday = 5}
\PYG{g+gp}{... }\PYG{l+s+sd}{    spring.weekend = 2}
\PYG{g+gp}{... }\PYG{l+s+sd}{    autumn.weekday = 5}
\PYG{g+gp}{... }\PYG{l+s+sd}{    autumn.weekend = 2}
\PYG{g+gp}{... }\PYG{l+s+sd}{    winter.weekday = 5}
\PYG{g+gp}{... }\PYG{l+s+sd}{    winter.weekend = 2}
\PYG{g+gp}{... }\PYG{l+s+sd}{    summer.weekday = 5}
\PYG{g+gp}{... }\PYG{l+s+sd}{    summer.weekend = 2}
\PYG{g+gp}{... }\PYG{l+s+sd}{    level\PYGZus{}names = [\PYGZdq{}season\PYGZdq{}, \PYGZdq{}week\PYGZdq{}]}
\PYG{g+gp}{... }\PYG{l+s+sd}{    \PYGZdq{}\PYGZdq{}\PYGZdq{}}
\PYG{g+gp}{... }\PYG{p}{)}
\PYG{g+go}{\PYGZlt{}xarray.DataArray (timeslice: 8)\PYGZgt{}}
\PYG{g+go}{array([5, 2, 5, 2, 5, 2, 5, 2])}
\PYG{g+go}{Coordinates:}
\PYG{g+go}{  * timeslice  (timeslice) MultiIndex}
\PYG{g+go}{  \PYGZhy{} season     (timeslice) object \PYGZsq{}spring\PYGZsq{} \PYGZsq{}spring\PYGZsq{} ... \PYGZsq{}summer\PYGZsq{} \PYGZsq{}summer\PYGZsq{}}
\PYG{g+go}{  \PYGZhy{} week       (timeslice) object \PYGZsq{}weekday\PYGZsq{} \PYGZsq{}weekend\PYGZsq{} ... \PYGZsq{}weekday\PYGZsq{} \PYGZsq{}weekend\PYGZsq{}}
\end{sphinxVerbatim}

\end{fulllineitems}

\index{represent\_hours() (in module muse.timeslices)@\spxentry{represent\_hours()}\spxextra{in module muse.timeslices}}

\begin{fulllineitems}
\phantomsection\label{\detokenize{api:muse.timeslices.represent_hours}}\pysiglinewithargsret{\sphinxcode{\sphinxupquote{muse.timeslices.}}\sphinxbfcode{\sphinxupquote{represent\_hours}}}{\emph{\DUrole{n}{timeslices}\DUrole{p}{:} \DUrole{n}{xarray.core.dataarray.DataArray}}, \emph{\DUrole{n}{nhours}\DUrole{p}{:} \DUrole{n}{Union\DUrole{p}{{[}}int\DUrole{p}{, }float\DUrole{p}{{]}}} \DUrole{o}{=} \DUrole{default_value}{8765.82}}}{{ $\rightarrow$ xarray.core.dataarray.DataArray}}
Number of hours per timeslice.
\begin{quote}\begin{description}
\item[{Parameters}] \leavevmode\begin{itemize}
\item {} 
\sphinxstyleliteralstrong{\sphinxupquote{timeslices}} \textendash{} The timeslice for which to compute the number of hours

\item {} 
\sphinxstyleliteralstrong{\sphinxupquote{nhours}} \textendash{} The total number of hours represented in the timeslice. Defaults to the
average number of hours in year.

\end{itemize}

\end{description}\end{quote}

\end{fulllineitems}

\index{setup\_module() (in module muse.timeslices)@\spxentry{setup\_module()}\spxextra{in module muse.timeslices}}

\begin{fulllineitems}
\phantomsection\label{\detokenize{api:muse.timeslices.setup_module}}\pysiglinewithargsret{\sphinxcode{\sphinxupquote{muse.timeslices.}}\sphinxbfcode{\sphinxupquote{setup\_module}}}{\emph{\DUrole{n}{settings}\DUrole{p}{:} \DUrole{n}{Union\DUrole{p}{{[}}str\DUrole{p}{, }Mapping\DUrole{p}{{]}}}}}{}
Sets up module singletons.

\end{fulllineitems}

\index{timeslice\_projector() (in module muse.timeslices)@\spxentry{timeslice\_projector()}\spxextra{in module muse.timeslices}}

\begin{fulllineitems}
\phantomsection\label{\detokenize{api:muse.timeslices.timeslice_projector}}\pysiglinewithargsret{\sphinxcode{\sphinxupquote{muse.timeslices.}}\sphinxbfcode{\sphinxupquote{timeslice\_projector}}}{\emph{\DUrole{n}{x}\DUrole{p}{:} \DUrole{n}{Union\DUrole{p}{{[}}xarray.core.dataarray.DataArray\DUrole{p}{, }pandas.core.indexes.multi.MultiIndex\DUrole{p}{{]}}}}, \emph{\DUrole{n}{finest}\DUrole{p}{:} \DUrole{n}{Optional\DUrole{p}{{[}}xarray.core.dataarray.DataArray\DUrole{p}{{]}}} \DUrole{o}{=} \DUrole{default_value}{None}}, \emph{\DUrole{n}{transforms}\DUrole{p}{:} \DUrole{n}{Optional\DUrole{p}{{[}}Dict\DUrole{p}{{[}}Tuple\DUrole{p}{, }numpy.ndarray\DUrole{p}{{]}}\DUrole{p}{{]}}} \DUrole{o}{=} \DUrole{default_value}{None}}}{{ $\rightarrow$ xarray.core.dataarray.DataArray}}
Project time\sphinxhyphen{}slice to standardized finest time\sphinxhyphen{}slices.

Returns a matrix from the input timeslice \sphinxcode{\sphinxupquote{x}} to the \sphinxcode{\sphinxupquote{finest}} timeslice, using
the input \sphinxcode{\sphinxupquote{transforms}}. The latter are a set of transforms that map indices from
one timeslice to indices in another.
\subsubsection*{Example}

Lets define the following timeslices and aggregates:

\begin{sphinxVerbatim}[commandchars=\\\{\}]
\PYG{g+gp}{\PYGZgt{}\PYGZgt{}\PYGZgt{} }\PYG{n}{toml} \PYG{o}{=} \PYG{l+s+s2}{\PYGZdq{}\PYGZdq{}\PYGZdq{}}
\PYG{g+gp}{... }\PYG{l+s+s2}{    [}\PYG{l+s+s2}{\PYGZdq{}}\PYG{l+s+s2}{timeslices}\PYG{l+s+s2}{\PYGZdq{}}\PYG{l+s+s2}{]}
\PYG{g+gp}{... }\PYG{l+s+s2}{    winter.weekday.day = 5}
\PYG{g+gp}{... }\PYG{l+s+s2}{    winter.weekday.night = 5}
\PYG{g+gp}{... }\PYG{l+s+s2}{    winter.weekend.day = 2}
\PYG{g+gp}{... }\PYG{l+s+s2}{    winter.weekend.night = 2}
\PYG{g+gp}{... }\PYG{l+s+s2}{    winter.weekend.dusk = 1}
\PYG{g+gp}{... }\PYG{l+s+s2}{    summer.weekday.day = 5}
\PYG{g+gp}{... }\PYG{l+s+s2}{    summer.weekday.night = 5}
\PYG{g+gp}{... }\PYG{l+s+s2}{    summer.weekend.day = 2}
\PYG{g+gp}{... }\PYG{l+s+s2}{    summer.weekend.night = 2}
\PYG{g+gp}{... }\PYG{l+s+s2}{    summer.weekend.dusk = 1}
\PYG{g+gp}{... }\PYG{l+s+s2}{    level\PYGZus{}names = [}\PYG{l+s+s2}{\PYGZdq{}}\PYG{l+s+s2}{semester}\PYG{l+s+s2}{\PYGZdq{}}\PYG{l+s+s2}{, }\PYG{l+s+s2}{\PYGZdq{}}\PYG{l+s+s2}{week}\PYG{l+s+s2}{\PYGZdq{}}\PYG{l+s+s2}{, }\PYG{l+s+s2}{\PYGZdq{}}\PYG{l+s+s2}{day}\PYG{l+s+s2}{\PYGZdq{}}\PYG{l+s+s2}{]}
\PYG{g+gp}{... }\PYG{l+s+s2}{    aggregates.allday = [}\PYG{l+s+s2}{\PYGZdq{}}\PYG{l+s+s2}{day}\PYG{l+s+s2}{\PYGZdq{}}\PYG{l+s+s2}{, }\PYG{l+s+s2}{\PYGZdq{}}\PYG{l+s+s2}{night}\PYG{l+s+s2}{\PYGZdq{}}\PYG{l+s+s2}{]}
\PYG{g+gp}{... }\PYG{l+s+s2}{\PYGZdq{}\PYGZdq{}\PYGZdq{}}
\PYG{g+gp}{\PYGZgt{}\PYGZgt{}\PYGZgt{} }\PYG{k+kn}{from} \PYG{n+nn}{muse}\PYG{n+nn}{.}\PYG{n+nn}{timeslices} \PYG{k+kn}{import} \PYG{p}{(}
\PYG{g+gp}{... }    \PYG{n}{reference\PYGZus{}timeslice}\PYG{p}{,}  \PYG{n}{aggregate\PYGZus{}transforms}
\PYG{g+gp}{... }\PYG{p}{)}
\PYG{g+gp}{\PYGZgt{}\PYGZgt{}\PYGZgt{} }\PYG{n}{ref} \PYG{o}{=} \PYG{n}{reference\PYGZus{}timeslice}\PYG{p}{(}\PYG{n}{toml}\PYG{p}{)}
\PYG{g+gp}{\PYGZgt{}\PYGZgt{}\PYGZgt{} }\PYG{n}{transforms} \PYG{o}{=} \PYG{n}{aggregate\PYGZus{}transforms}\PYG{p}{(}\PYG{n}{toml}\PYG{p}{,} \PYG{n}{ref}\PYG{p}{)}
\PYG{g+gp}{\PYGZgt{}\PYGZgt{}\PYGZgt{} }\PYG{k+kn}{from} \PYG{n+nn}{pandas} \PYG{k+kn}{import} \PYG{n}{MultiIndex}
\PYG{g+gp}{\PYGZgt{}\PYGZgt{}\PYGZgt{} }\PYG{n}{input\PYGZus{}ts} \PYG{o}{=} \PYG{n}{DataArray}\PYG{p}{(}
\PYG{g+gp}{... }    \PYG{p}{[}\PYG{l+m+mi}{1}\PYG{p}{,} \PYG{l+m+mi}{2}\PYG{p}{,} \PYG{l+m+mi}{3}\PYG{p}{]}\PYG{p}{,}
\PYG{g+gp}{... }    \PYG{n}{coords}\PYG{o}{=}\PYG{p}{\PYGZob{}}
\PYG{g+gp}{... }        \PYG{l+s+s2}{\PYGZdq{}}\PYG{l+s+s2}{timeslice}\PYG{l+s+s2}{\PYGZdq{}}\PYG{p}{:} \PYG{n}{MultiIndex}\PYG{o}{.}\PYG{n}{from\PYGZus{}tuples}\PYG{p}{(}
\PYG{g+gp}{... }            \PYG{p}{[}
\PYG{g+gp}{... }                \PYG{p}{(}\PYG{l+s+s2}{\PYGZdq{}}\PYG{l+s+s2}{winter}\PYG{l+s+s2}{\PYGZdq{}}\PYG{p}{,} \PYG{l+s+s2}{\PYGZdq{}}\PYG{l+s+s2}{weekday}\PYG{l+s+s2}{\PYGZdq{}}\PYG{p}{,} \PYG{l+s+s2}{\PYGZdq{}}\PYG{l+s+s2}{allday}\PYG{l+s+s2}{\PYGZdq{}}\PYG{p}{)}\PYG{p}{,}
\PYG{g+gp}{... }                \PYG{p}{(}\PYG{l+s+s2}{\PYGZdq{}}\PYG{l+s+s2}{winter}\PYG{l+s+s2}{\PYGZdq{}}\PYG{p}{,} \PYG{l+s+s2}{\PYGZdq{}}\PYG{l+s+s2}{weekend}\PYG{l+s+s2}{\PYGZdq{}}\PYG{p}{,} \PYG{l+s+s2}{\PYGZdq{}}\PYG{l+s+s2}{dusk}\PYG{l+s+s2}{\PYGZdq{}}\PYG{p}{)}\PYG{p}{,}
\PYG{g+gp}{... }                \PYG{p}{(}\PYG{l+s+s2}{\PYGZdq{}}\PYG{l+s+s2}{summer}\PYG{l+s+s2}{\PYGZdq{}}\PYG{p}{,} \PYG{l+s+s2}{\PYGZdq{}}\PYG{l+s+s2}{weekend}\PYG{l+s+s2}{\PYGZdq{}}\PYG{p}{,} \PYG{l+s+s2}{\PYGZdq{}}\PYG{l+s+s2}{night}\PYG{l+s+s2}{\PYGZdq{}}\PYG{p}{)}\PYG{p}{,}
\PYG{g+gp}{... }            \PYG{p}{]}\PYG{p}{,}
\PYG{g+gp}{... }            \PYG{n}{names}\PYG{o}{=}\PYG{n}{ref}\PYG{o}{.}\PYG{n}{get\PYGZus{}index}\PYG{p}{(}\PYG{l+s+s2}{\PYGZdq{}}\PYG{l+s+s2}{timeslice}\PYG{l+s+s2}{\PYGZdq{}}\PYG{p}{)}\PYG{o}{.}\PYG{n}{names}\PYG{p}{,}
\PYG{g+gp}{... }        \PYG{p}{)}\PYG{p}{,}
\PYG{g+gp}{... }    \PYG{p}{\PYGZcb{}}\PYG{p}{,}
\PYG{g+gp}{... }    \PYG{n}{dims}\PYG{o}{=}\PYG{l+s+s2}{\PYGZdq{}}\PYG{l+s+s2}{timeslice}\PYG{l+s+s2}{\PYGZdq{}}
\PYG{g+gp}{... }\PYG{p}{)}
\PYG{g+gp}{\PYGZgt{}\PYGZgt{}\PYGZgt{} }\PYG{n}{input\PYGZus{}ts}
\PYG{g+go}{\PYGZlt{}xarray.DataArray (timeslice: 3)\PYGZgt{}}
\PYG{g+go}{array([1, 2, 3])}
\PYG{g+go}{Coordinates:}
\PYG{g+go}{  * timeslice  (timeslice) MultiIndex}
\PYG{g+go}{  \PYGZhy{} semester   (timeslice) object \PYGZsq{}winter\PYGZsq{} \PYGZsq{}winter\PYGZsq{} \PYGZsq{}summer\PYGZsq{}}
\PYG{g+go}{  \PYGZhy{} week       (timeslice) object \PYGZsq{}weekday\PYGZsq{} \PYGZsq{}weekend\PYGZsq{} \PYGZsq{}weekend\PYGZsq{}}
\PYG{g+go}{  \PYGZhy{} day        (timeslice) object \PYGZsq{}allday\PYGZsq{} \PYGZsq{}dusk\PYGZsq{} \PYGZsq{}night\PYGZsq{}}
\end{sphinxVerbatim}

The input timeslice does not have to be complete. In any case, we can now
compute a transform, i.e. a matrix that will take this timeslice and transform
it to the equivalent times in the finest timeslice:

\begin{sphinxVerbatim}[commandchars=\\\{\}]
\PYG{g+gp}{\PYGZgt{}\PYGZgt{}\PYGZgt{} }\PYG{k+kn}{from} \PYG{n+nn}{muse}\PYG{n+nn}{.}\PYG{n+nn}{timeslices} \PYG{k+kn}{import} \PYG{n}{timeslice\PYGZus{}projector}
\PYG{g+gp}{\PYGZgt{}\PYGZgt{}\PYGZgt{} }\PYG{n}{timeslice\PYGZus{}projector}\PYG{p}{(}\PYG{n}{input\PYGZus{}ts}\PYG{p}{,} \PYG{n}{ref}\PYG{p}{,} \PYG{n}{transforms}\PYG{p}{)}
\PYG{g+go}{\PYGZlt{}xarray.DataArray \PYGZsq{}projector\PYGZsq{} (finest\PYGZus{}timeslice: 10, timeslice: 3)\PYGZgt{}}
\PYG{g+go}{array([[1, 0, 0],}
\PYG{g+go}{       [1, 0, 0],}
\PYG{g+go}{       [0, 0, 0],}
\PYG{g+go}{       [0, 0, 0],}
\PYG{g+go}{       [0, 1, 0],}
\PYG{g+go}{       [0, 0, 0],}
\PYG{g+go}{       [0, 0, 0],}
\PYG{g+go}{       [0, 0, 0],}
\PYG{g+go}{       [0, 0, 1],}
\PYG{g+go}{       [0, 0, 0]])}
\PYG{g+go}{Coordinates:}
\PYG{g+go}{  * finest\PYGZus{}timeslice  (finest\PYGZus{}timeslice) MultiIndex}
\PYG{g+go}{  \PYGZhy{} finest\PYGZus{}semester   (finest\PYGZus{}timeslice) object \PYGZsq{}winter\PYGZsq{} \PYGZsq{}winter\PYGZsq{} ... \PYGZsq{}summer\PYGZsq{}}
\PYG{g+go}{  \PYGZhy{} finest\PYGZus{}week       (finest\PYGZus{}timeslice) object \PYGZsq{}weekday\PYGZsq{} ... \PYGZsq{}weekend\PYGZsq{}}
\PYG{g+go}{  \PYGZhy{} finest\PYGZus{}day        (finest\PYGZus{}timeslice) object \PYGZsq{}day\PYGZsq{} \PYGZsq{}night\PYGZsq{} ... \PYGZsq{}night\PYGZsq{} \PYGZsq{}dusk\PYGZsq{}}
\PYG{g+go}{  * timeslice         (timeslice) MultiIndex}
\PYG{g+go}{  \PYGZhy{} semester          (timeslice) object \PYGZsq{}winter\PYGZsq{} \PYGZsq{}winter\PYGZsq{} \PYGZsq{}summer\PYGZsq{}}
\PYG{g+go}{  \PYGZhy{} week              (timeslice) object \PYGZsq{}weekday\PYGZsq{} \PYGZsq{}weekend\PYGZsq{} \PYGZsq{}weekend\PYGZsq{}}
\PYG{g+go}{  \PYGZhy{} day               (timeslice) object \PYGZsq{}allday\PYGZsq{} \PYGZsq{}dusk\PYGZsq{} \PYGZsq{}night\PYGZsq{}}
\end{sphinxVerbatim}

It is possible to give as input an array which does not have a timeslice of its
own:

\begin{sphinxVerbatim}[commandchars=\\\{\}]
\PYG{g+gp}{\PYGZgt{}\PYGZgt{}\PYGZgt{} }\PYG{n}{nots} \PYG{o}{=} \PYG{n}{DataArray}\PYG{p}{(}\PYG{p}{[}\PYG{l+m+mf}{5.0}\PYG{p}{,} \PYG{l+m+mf}{1.0}\PYG{p}{,} \PYG{l+m+mf}{2.0}\PYG{p}{]}\PYG{p}{,} \PYG{n}{dims}\PYG{o}{=}\PYG{l+s+s2}{\PYGZdq{}}\PYG{l+s+s2}{a}\PYG{l+s+s2}{\PYGZdq{}}\PYG{p}{,} \PYG{n}{coords}\PYG{o}{=}\PYG{p}{\PYGZob{}}\PYG{l+s+s1}{\PYGZsq{}}\PYG{l+s+s1}{a}\PYG{l+s+s1}{\PYGZsq{}}\PYG{p}{:} \PYG{p}{[}\PYG{l+m+mi}{1}\PYG{p}{,} \PYG{l+m+mi}{2}\PYG{p}{,} \PYG{l+m+mi}{3}\PYG{p}{]}\PYG{p}{\PYGZcb{}}\PYG{p}{)}
\PYG{g+gp}{\PYGZgt{}\PYGZgt{}\PYGZgt{} }\PYG{n}{timeslice\PYGZus{}projector}\PYG{p}{(}\PYG{n}{nots}\PYG{p}{,} \PYG{n}{ref}\PYG{p}{,} \PYG{n}{transforms}\PYG{p}{)}\PYG{o}{.}\PYG{n}{T}
\PYG{g+go}{\PYGZlt{}xarray.DataArray (timeslice: 1, finest\PYGZus{}timeslice: 10)\PYGZgt{}}
\PYG{g+go}{array([[1, 1, 1, 1, 1, 1, 1, 1, 1, 1]])}
\PYG{g+go}{Coordinates:}
\PYG{g+go}{  * finest\PYGZus{}timeslice  (finest\PYGZus{}timeslice) MultiIndex}
\PYG{g+go}{  \PYGZhy{} finest\PYGZus{}semester   (finest\PYGZus{}timeslice) object \PYGZsq{}winter\PYGZsq{} \PYGZsq{}winter\PYGZsq{} ... \PYGZsq{}summer\PYGZsq{}}
\PYG{g+go}{  \PYGZhy{} finest\PYGZus{}week       (finest\PYGZus{}timeslice) object \PYGZsq{}weekday\PYGZsq{} ... \PYGZsq{}weekend\PYGZsq{}}
\PYG{g+go}{  \PYGZhy{} finest\PYGZus{}day        (finest\PYGZus{}timeslice) object \PYGZsq{}day\PYGZsq{} \PYGZsq{}night\PYGZsq{} ... \PYGZsq{}night\PYGZsq{} \PYGZsq{}dusk\PYGZsq{}}
\PYG{g+go}{Dimensions without coordinates: timeslice}
\end{sphinxVerbatim}

\end{fulllineitems}



\subsection{Commodities}
\label{\detokenize{api:module-muse.commodities}}\label{\detokenize{api:commodities}}\index{module@\spxentry{module}!muse.commodities@\spxentry{muse.commodities}}\index{muse.commodities@\spxentry{muse.commodities}!module@\spxentry{module}}
Methods and types around commodities.
\index{CommodityUsage (class in muse.commodities)@\spxentry{CommodityUsage}\spxextra{class in muse.commodities}}

\begin{fulllineitems}
\phantomsection\label{\detokenize{api:muse.commodities.CommodityUsage}}\pysiglinewithargsret{\sphinxbfcode{\sphinxupquote{class }}\sphinxcode{\sphinxupquote{muse.commodities.}}\sphinxbfcode{\sphinxupquote{CommodityUsage}}}{\emph{\DUrole{n}{value}}}{}
Flags to specify the different kinds of commodities.

For details on how \sphinxcode{\sphinxupquote{enum}}’s work, see \sphinxhref{https://docs.python.org/3/library/enum.html\#intflag}{python’s documentation}. In practice,
\sphinxcode{\sphinxupquote{CommodityUsage}} centralizes in one place the different kinds of
commodities that are meaningfull to the generalized sector, e.g. commodities that
are consumed by the sector, and commodities that produced by the sectors, as well
commodities that are, somehow, \sphinxstyleemphasis{environmental}.

With the exception of \sphinxcode{\sphinxupquote{CommodityUsage.OTHER}}, flags can be combined in any
fashion. \sphinxcode{\sphinxupquote{CommodityUsage.PRODUCT | CommodityUsage.CONSUMABLE}} is a commodity that
is both consumed and produced by a sector. \sphinxcode{\sphinxupquote{CommodityUsage.ENVIRONMENTAL |
CommodityUsage.ENERGY | CommodityUsage.CONSUMABLE}} is an environmental energy
commodity consumed by the sector.

\sphinxcode{\sphinxupquote{CommodityUsage.OTHER}} is an alias for \sphinxstyleemphasis{no} flag. It is meant for commodities
that should be ignored by the sector.
\index{CONSUMABLE (muse.commodities.CommodityUsage attribute)@\spxentry{CONSUMABLE}\spxextra{muse.commodities.CommodityUsage attribute}}

\begin{fulllineitems}
\phantomsection\label{\detokenize{api:muse.commodities.CommodityUsage.CONSUMABLE}}\pysigline{\sphinxbfcode{\sphinxupquote{CONSUMABLE}}\sphinxbfcode{\sphinxupquote{ = 1}}}
Commodity which can be consumed by the sector.

\end{fulllineitems}

\index{ENERGY (muse.commodities.CommodityUsage attribute)@\spxentry{ENERGY}\spxextra{muse.commodities.CommodityUsage attribute}}

\begin{fulllineitems}
\phantomsection\label{\detokenize{api:muse.commodities.CommodityUsage.ENERGY}}\pysigline{\sphinxbfcode{\sphinxupquote{ENERGY}}\sphinxbfcode{\sphinxupquote{ = 8}}}
Commodity which is a fuel for this or another sector.

\end{fulllineitems}

\index{ENVIRONMENTAL (muse.commodities.CommodityUsage attribute)@\spxentry{ENVIRONMENTAL}\spxextra{muse.commodities.CommodityUsage attribute}}

\begin{fulllineitems}
\phantomsection\label{\detokenize{api:muse.commodities.CommodityUsage.ENVIRONMENTAL}}\pysigline{\sphinxbfcode{\sphinxupquote{ENVIRONMENTAL}}\sphinxbfcode{\sphinxupquote{ = 4}}}
Commodity which is a pollutant.

\end{fulllineitems}

\index{OTHER (muse.commodities.CommodityUsage attribute)@\spxentry{OTHER}\spxextra{muse.commodities.CommodityUsage attribute}}

\begin{fulllineitems}
\phantomsection\label{\detokenize{api:muse.commodities.CommodityUsage.OTHER}}\pysigline{\sphinxbfcode{\sphinxupquote{OTHER}}\sphinxbfcode{\sphinxupquote{ = 0}}}
Not relevant for current sector.

\end{fulllineitems}

\index{PRODUCT (muse.commodities.CommodityUsage attribute)@\spxentry{PRODUCT}\spxextra{muse.commodities.CommodityUsage attribute}}

\begin{fulllineitems}
\phantomsection\label{\detokenize{api:muse.commodities.CommodityUsage.PRODUCT}}\pysigline{\sphinxbfcode{\sphinxupquote{PRODUCT}}\sphinxbfcode{\sphinxupquote{ = 2}}}
Commodity which can be produced by the sector.

\end{fulllineitems}


\end{fulllineitems}

\index{check\_usage() (in module muse.commodities)@\spxentry{check\_usage()}\spxextra{in module muse.commodities}}

\begin{fulllineitems}
\phantomsection\label{\detokenize{api:muse.commodities.check_usage}}\pysiglinewithargsret{\sphinxcode{\sphinxupquote{muse.commodities.}}\sphinxbfcode{\sphinxupquote{check\_usage}}}{\emph{\DUrole{n}{data}\DUrole{p}{:} \DUrole{n}{Sequence\DUrole{p}{{[}}muse.commodities.CommodityUsage\DUrole{p}{{]}}}}, \emph{\DUrole{n}{flag}\DUrole{p}{:} \DUrole{n}{Optional\DUrole{p}{{[}}Union\DUrole{p}{{[}}str\DUrole{p}{, }muse.commodities.CommodityUsage\DUrole{p}{{]}}\DUrole{p}{{]}}}}, \emph{\DUrole{n}{match}\DUrole{p}{:} \DUrole{n}{str} \DUrole{o}{=} \DUrole{default_value}{\textquotesingle{}all\textquotesingle{}}}}{{ $\rightarrow$ numpy.ndarray}}
Match usage flags with input data array.
\begin{quote}\begin{description}
\item[{Parameters}] \leavevmode\begin{itemize}
\item {} 
\sphinxstyleliteralstrong{\sphinxupquote{data}} \textendash{} sequence for which to match flags elementwise.

\item {} 
\sphinxstyleliteralstrong{\sphinxupquote{flag}} \textendash{} flag or combination of flags to match. The input can be a string, such as
“product | environmental”, or a CommodityUsage instance.
Defaults to “other”.

\item {} 
\sphinxstyleliteralstrong{\sphinxupquote{match}} \textendash{} one of:
\sphinxhyphen{} “all”: should all flag match. Default.
\sphinxhyphen{} “any”, should match at least one flags.
\sphinxhyphen{} “exact”, should match each flag and nothing else.

\end{itemize}

\end{description}\end{quote}
\subsubsection*{Examples}

\begin{sphinxVerbatim}[commandchars=\\\{\}]
\PYG{g+gp}{\PYGZgt{}\PYGZgt{}\PYGZgt{} }\PYG{k+kn}{from} \PYG{n+nn}{muse}\PYG{n+nn}{.}\PYG{n+nn}{commodities} \PYG{k+kn}{import} \PYG{n}{CommodityUsage}\PYG{p}{,} \PYG{n}{check\PYGZus{}usage}
\PYG{g+gp}{\PYGZgt{}\PYGZgt{}\PYGZgt{} }\PYG{n}{data} \PYG{o}{=} \PYG{p}{[}
\PYG{g+gp}{... }    \PYG{n}{CommodityUsage}\PYG{o}{.}\PYG{n}{OTHER}\PYG{p}{,}
\PYG{g+gp}{... }    \PYG{n}{CommodityUsage}\PYG{o}{.}\PYG{n}{PRODUCT}\PYG{p}{,}
\PYG{g+gp}{... }    \PYG{n}{CommodityUsage}\PYG{o}{.}\PYG{n}{ENVIRONMENTAL} \PYG{o}{|} \PYG{n}{CommodityUsage}\PYG{o}{.}\PYG{n}{PRODUCT}\PYG{p}{,}
\PYG{g+gp}{... }    \PYG{n}{CommodityUsage}\PYG{o}{.}\PYG{n}{ENVIRONMENTAL}\PYG{p}{,}
\PYG{g+gp}{... }\PYG{p}{]}
\end{sphinxVerbatim}

Matching “all”:

\begin{sphinxVerbatim}[commandchars=\\\{\}]
\PYG{g+gp}{\PYGZgt{}\PYGZgt{}\PYGZgt{} }\PYG{n}{check\PYGZus{}usage}\PYG{p}{(}\PYG{n}{data}\PYG{p}{,} \PYG{n}{CommodityUsage}\PYG{o}{.}\PYG{n}{PRODUCT}\PYG{p}{)}\PYG{o}{.}\PYG{n}{tolist}\PYG{p}{(}\PYG{p}{)}
\PYG{g+go}{[False, True, True, False]}
\end{sphinxVerbatim}

\begin{sphinxVerbatim}[commandchars=\\\{\}]
\PYG{g+gp}{\PYGZgt{}\PYGZgt{}\PYGZgt{} }\PYG{n}{check\PYGZus{}usage}\PYG{p}{(}\PYG{n}{data}\PYG{p}{,} \PYG{n}{CommodityUsage}\PYG{o}{.}\PYG{n}{ENVIRONMENTAL}\PYG{p}{)}\PYG{o}{.}\PYG{n}{tolist}\PYG{p}{(}\PYG{p}{)}
\PYG{g+go}{[False, False, True, True]}
\end{sphinxVerbatim}

\begin{sphinxVerbatim}[commandchars=\\\{\}]
\PYG{g+gp}{\PYGZgt{}\PYGZgt{}\PYGZgt{} }\PYG{n}{check\PYGZus{}usage}\PYG{p}{(}
\PYG{g+gp}{... }    \PYG{n}{data}\PYG{p}{,} \PYG{n}{CommodityUsage}\PYG{o}{.}\PYG{n}{ENVIRONMENTAL} \PYG{o}{|} \PYG{n}{CommodityUsage}\PYG{o}{.}\PYG{n}{PRODUCT}
\PYG{g+gp}{... }\PYG{p}{)}\PYG{o}{.}\PYG{n}{tolist}\PYG{p}{(}\PYG{p}{)}
\PYG{g+go}{[False, False, True, False]}
\end{sphinxVerbatim}

Matching “any”:

\begin{sphinxVerbatim}[commandchars=\\\{\}]
\PYG{g+gp}{\PYGZgt{}\PYGZgt{}\PYGZgt{} }\PYG{n}{check\PYGZus{}usage}\PYG{p}{(}\PYG{n}{data}\PYG{p}{,} \PYG{n}{CommodityUsage}\PYG{o}{.}\PYG{n}{PRODUCT}\PYG{p}{,} \PYG{n}{match}\PYG{o}{=}\PYG{l+s+s2}{\PYGZdq{}}\PYG{l+s+s2}{any}\PYG{l+s+s2}{\PYGZdq{}}\PYG{p}{)}\PYG{o}{.}\PYG{n}{tolist}\PYG{p}{(}\PYG{p}{)}
\PYG{g+go}{[False, True, True, False]}
\end{sphinxVerbatim}

\begin{sphinxVerbatim}[commandchars=\\\{\}]
\PYG{g+gp}{\PYGZgt{}\PYGZgt{}\PYGZgt{} }\PYG{n}{check\PYGZus{}usage}\PYG{p}{(}\PYG{n}{data}\PYG{p}{,} \PYG{n}{CommodityUsage}\PYG{o}{.}\PYG{n}{ENVIRONMENTAL}\PYG{p}{,} \PYG{n}{match}\PYG{o}{=}\PYG{l+s+s2}{\PYGZdq{}}\PYG{l+s+s2}{any}\PYG{l+s+s2}{\PYGZdq{}}\PYG{p}{)}\PYG{o}{.}\PYG{n}{tolist}\PYG{p}{(}\PYG{p}{)}
\PYG{g+go}{[False, False, True, True]}
\end{sphinxVerbatim}

\begin{sphinxVerbatim}[commandchars=\\\{\}]
\PYG{g+gp}{\PYGZgt{}\PYGZgt{}\PYGZgt{} }\PYG{n}{check\PYGZus{}usage}\PYG{p}{(}\PYG{n}{data}\PYG{p}{,} \PYG{l+s+s2}{\PYGZdq{}}\PYG{l+s+s2}{environmental | product}\PYG{l+s+s2}{\PYGZdq{}}\PYG{p}{,} \PYG{n}{match}\PYG{o}{=}\PYG{l+s+s2}{\PYGZdq{}}\PYG{l+s+s2}{any}\PYG{l+s+s2}{\PYGZdq{}}\PYG{p}{)}\PYG{o}{.}\PYG{n}{tolist}\PYG{p}{(}\PYG{p}{)}
\PYG{g+go}{[False, True, True, True]}
\end{sphinxVerbatim}

Matching “exact”:

\begin{sphinxVerbatim}[commandchars=\\\{\}]
\PYG{g+gp}{\PYGZgt{}\PYGZgt{}\PYGZgt{} }\PYG{n}{check\PYGZus{}usage}\PYG{p}{(}\PYG{n}{data}\PYG{p}{,} \PYG{l+s+s2}{\PYGZdq{}}\PYG{l+s+s2}{PRODUCT}\PYG{l+s+s2}{\PYGZdq{}}\PYG{p}{,} \PYG{n}{match}\PYG{o}{=}\PYG{l+s+s2}{\PYGZdq{}}\PYG{l+s+s2}{exact}\PYG{l+s+s2}{\PYGZdq{}}\PYG{p}{)}\PYG{o}{.}\PYG{n}{tolist}\PYG{p}{(}\PYG{p}{)}
\PYG{g+go}{[False, True, False, False]}
\end{sphinxVerbatim}

\begin{sphinxVerbatim}[commandchars=\\\{\}]
\PYG{g+gp}{\PYGZgt{}\PYGZgt{}\PYGZgt{} }\PYG{n}{check\PYGZus{}usage}\PYG{p}{(}\PYG{n}{data}\PYG{p}{,} \PYG{n}{CommodityUsage}\PYG{o}{.}\PYG{n}{ENVIRONMENTAL}\PYG{p}{,} \PYG{n}{match}\PYG{o}{=}\PYG{l+s+s2}{\PYGZdq{}}\PYG{l+s+s2}{exact}\PYG{l+s+s2}{\PYGZdq{}}\PYG{p}{)}\PYG{o}{.}\PYG{n}{tolist}\PYG{p}{(}\PYG{p}{)}
\PYG{g+go}{[False, False, False, True]}
\end{sphinxVerbatim}

\begin{sphinxVerbatim}[commandchars=\\\{\}]
\PYG{g+gp}{\PYGZgt{}\PYGZgt{}\PYGZgt{} }\PYG{n}{check\PYGZus{}usage}\PYG{p}{(}\PYG{n}{data}\PYG{p}{,} \PYG{l+s+s2}{\PYGZdq{}}\PYG{l+s+s2}{ENVIRONMENTAL | PRODUCT}\PYG{l+s+s2}{\PYGZdq{}}\PYG{p}{,} \PYG{n}{match}\PYG{o}{=}\PYG{l+s+s2}{\PYGZdq{}}\PYG{l+s+s2}{exact}\PYG{l+s+s2}{\PYGZdq{}}\PYG{p}{)}\PYG{o}{.}\PYG{n}{tolist}\PYG{p}{(}\PYG{p}{)}
\PYG{g+go}{[False, False, True, False]}
\end{sphinxVerbatim}

Finally, checking no flags has been set can be done with:

\begin{sphinxVerbatim}[commandchars=\\\{\}]
\PYG{g+gp}{\PYGZgt{}\PYGZgt{}\PYGZgt{} }\PYG{n}{check\PYGZus{}usage}\PYG{p}{(}\PYG{n}{data}\PYG{p}{,} \PYG{n}{CommodityUsage}\PYG{o}{.}\PYG{n}{OTHER}\PYG{p}{,} \PYG{n}{match}\PYG{o}{=}\PYG{l+s+s2}{\PYGZdq{}}\PYG{l+s+s2}{exact}\PYG{l+s+s2}{\PYGZdq{}}\PYG{p}{)}\PYG{o}{.}\PYG{n}{tolist}\PYG{p}{(}\PYG{p}{)}
\PYG{g+go}{[True, False, False, False]}
\PYG{g+gp}{\PYGZgt{}\PYGZgt{}\PYGZgt{} }\PYG{n}{check\PYGZus{}usage}\PYG{p}{(}\PYG{n}{data}\PYG{p}{,} \PYG{k+kc}{None}\PYG{p}{,} \PYG{n}{match}\PYG{o}{=}\PYG{l+s+s2}{\PYGZdq{}}\PYG{l+s+s2}{exact}\PYG{l+s+s2}{\PYGZdq{}}\PYG{p}{)}\PYG{o}{.}\PYG{n}{tolist}\PYG{p}{(}\PYG{p}{)}
\PYG{g+go}{[True, False, False, False]}
\end{sphinxVerbatim}

\end{fulllineitems}

\index{is\_consumable() (in module muse.commodities)@\spxentry{is\_consumable()}\spxextra{in module muse.commodities}}

\begin{fulllineitems}
\phantomsection\label{\detokenize{api:muse.commodities.is_consumable}}\pysiglinewithargsret{\sphinxcode{\sphinxupquote{muse.commodities.}}\sphinxbfcode{\sphinxupquote{is\_consumable}}}{\emph{\DUrole{n}{data}\DUrole{p}{:} \DUrole{n}{Sequence\DUrole{p}{{[}}muse.commodities.CommodityUsage\DUrole{p}{{]}}}}}{{ $\rightarrow$ numpy.ndarray}}
Any consumable.
\subsubsection*{Examples}

\begin{sphinxVerbatim}[commandchars=\\\{\}]
\PYG{g+gp}{\PYGZgt{}\PYGZgt{}\PYGZgt{} }\PYG{k+kn}{from} \PYG{n+nn}{muse}\PYG{n+nn}{.}\PYG{n+nn}{commodities} \PYG{k+kn}{import} \PYG{n}{CommodityUsage}\PYG{p}{,} \PYG{n}{is\PYGZus{}consumable}
\PYG{g+gp}{\PYGZgt{}\PYGZgt{}\PYGZgt{} }\PYG{n}{data} \PYG{o}{=} \PYG{p}{[}
\PYG{g+gp}{... }    \PYG{n}{CommodityUsage}\PYG{o}{.}\PYG{n}{CONSUMABLE}\PYG{p}{,}
\PYG{g+gp}{... }    \PYG{n}{CommodityUsage}\PYG{o}{.}\PYG{n}{PRODUCT}\PYG{p}{,}
\PYG{g+gp}{... }    \PYG{n}{CommodityUsage}\PYG{o}{.}\PYG{n}{ENVIRONMENTAL}\PYG{p}{,}
\PYG{g+gp}{... }    \PYG{n}{CommodityUsage}\PYG{o}{.}\PYG{n}{PRODUCT} \PYG{o}{|} \PYG{n}{CommodityUsage}\PYG{o}{.}\PYG{n}{CONSUMABLE}\PYG{p}{,}
\PYG{g+gp}{... }    \PYG{n}{CommodityUsage}\PYG{o}{.}\PYG{n}{ENVIRONMENTAL} \PYG{o}{|} \PYG{n}{CommodityUsage}\PYG{o}{.}\PYG{n}{PRODUCT}\PYG{p}{,}
\PYG{g+gp}{... }\PYG{p}{]}
\PYG{g+gp}{\PYGZgt{}\PYGZgt{}\PYGZgt{} }\PYG{n}{is\PYGZus{}consumable}\PYG{p}{(}\PYG{n}{data}\PYG{p}{)}\PYG{o}{.}\PYG{n}{tolist}\PYG{p}{(}\PYG{p}{)}
\PYG{g+go}{[True, False, False, True, False]}
\end{sphinxVerbatim}

\end{fulllineitems}

\index{is\_enduse() (in module muse.commodities)@\spxentry{is\_enduse()}\spxextra{in module muse.commodities}}

\begin{fulllineitems}
\phantomsection\label{\detokenize{api:muse.commodities.is_enduse}}\pysiglinewithargsret{\sphinxcode{\sphinxupquote{muse.commodities.}}\sphinxbfcode{\sphinxupquote{is\_enduse}}}{\emph{\DUrole{n}{data}\DUrole{p}{:} \DUrole{n}{Sequence\DUrole{p}{{[}}muse.commodities.CommodityUsage\DUrole{p}{{]}}}}}{{ $\rightarrow$ numpy.ndarray}}
Non\sphinxhyphen{}environmental product.
\subsubsection*{Examples}

\begin{sphinxVerbatim}[commandchars=\\\{\}]
\PYG{g+gp}{\PYGZgt{}\PYGZgt{}\PYGZgt{} }\PYG{k+kn}{from} \PYG{n+nn}{muse}\PYG{n+nn}{.}\PYG{n+nn}{commodities} \PYG{k+kn}{import} \PYG{n}{CommodityUsage}\PYG{p}{,} \PYG{n}{is\PYGZus{}enduse}
\PYG{g+gp}{\PYGZgt{}\PYGZgt{}\PYGZgt{} }\PYG{n}{data} \PYG{o}{=} \PYG{p}{[}
\PYG{g+gp}{... }    \PYG{n}{CommodityUsage}\PYG{o}{.}\PYG{n}{CONSUMABLE}\PYG{p}{,}
\PYG{g+gp}{... }    \PYG{n}{CommodityUsage}\PYG{o}{.}\PYG{n}{PRODUCT}\PYG{p}{,}
\PYG{g+gp}{... }    \PYG{n}{CommodityUsage}\PYG{o}{.}\PYG{n}{ENVIRONMENTAL}\PYG{p}{,}
\PYG{g+gp}{... }    \PYG{n}{CommodityUsage}\PYG{o}{.}\PYG{n}{PRODUCT} \PYG{o}{|} \PYG{n}{CommodityUsage}\PYG{o}{.}\PYG{n}{CONSUMABLE}\PYG{p}{,}
\PYG{g+gp}{... }    \PYG{n}{CommodityUsage}\PYG{o}{.}\PYG{n}{ENVIRONMENTAL} \PYG{o}{|} \PYG{n}{CommodityUsage}\PYG{o}{.}\PYG{n}{PRODUCT}\PYG{p}{,}
\PYG{g+gp}{... }\PYG{p}{]}
\PYG{g+gp}{\PYGZgt{}\PYGZgt{}\PYGZgt{} }\PYG{n}{is\PYGZus{}enduse}\PYG{p}{(}\PYG{n}{data}\PYG{p}{)}\PYG{o}{.}\PYG{n}{tolist}\PYG{p}{(}\PYG{p}{)}
\PYG{g+go}{[False, True, False, True, False]}
\end{sphinxVerbatim}

\end{fulllineitems}

\index{is\_fuel() (in module muse.commodities)@\spxentry{is\_fuel()}\spxextra{in module muse.commodities}}

\begin{fulllineitems}
\phantomsection\label{\detokenize{api:muse.commodities.is_fuel}}\pysiglinewithargsret{\sphinxcode{\sphinxupquote{muse.commodities.}}\sphinxbfcode{\sphinxupquote{is\_fuel}}}{\emph{\DUrole{n}{data}\DUrole{p}{:} \DUrole{n}{Sequence\DUrole{p}{{[}}muse.commodities.CommodityUsage\DUrole{p}{{]}}}}}{{ $\rightarrow$ numpy.ndarray}}
Any consumable energy.
\subsubsection*{Examples}

\begin{sphinxVerbatim}[commandchars=\\\{\}]
\PYG{g+gp}{\PYGZgt{}\PYGZgt{}\PYGZgt{} }\PYG{k+kn}{from} \PYG{n+nn}{muse}\PYG{n+nn}{.}\PYG{n+nn}{commodities} \PYG{k+kn}{import} \PYG{n}{CommodityUsage}\PYG{p}{,} \PYG{n}{is\PYGZus{}fuel}
\PYG{g+gp}{\PYGZgt{}\PYGZgt{}\PYGZgt{} }\PYG{n}{data} \PYG{o}{=} \PYG{p}{[}
\PYG{g+gp}{... }    \PYG{n}{CommodityUsage}\PYG{o}{.}\PYG{n}{CONSUMABLE}\PYG{p}{,}
\PYG{g+gp}{... }    \PYG{n}{CommodityUsage}\PYG{o}{.}\PYG{n}{PRODUCT}\PYG{p}{,}
\PYG{g+gp}{... }    \PYG{n}{CommodityUsage}\PYG{o}{.}\PYG{n}{ENERGY}\PYG{p}{,}
\PYG{g+gp}{... }    \PYG{n}{CommodityUsage}\PYG{o}{.}\PYG{n}{ENERGY} \PYG{o}{|} \PYG{n}{CommodityUsage}\PYG{o}{.}\PYG{n}{CONSUMABLE}\PYG{p}{,}
\PYG{g+gp}{... }    \PYG{n}{CommodityUsage}\PYG{o}{.}\PYG{n}{ENERGY} \PYG{o}{|} \PYG{n}{CommodityUsage}\PYG{o}{.}\PYG{n}{CONSUMABLE}
\PYG{g+gp}{... }        \PYG{o}{|} \PYG{n}{CommodityUsage}\PYG{o}{.}\PYG{n}{ENVIRONMENTAL}\PYG{p}{,}
\PYG{g+gp}{... }    \PYG{n}{CommodityUsage}\PYG{o}{.}\PYG{n}{ENERGY} \PYG{o}{|} \PYG{n}{CommodityUsage}\PYG{o}{.}\PYG{n}{CONSUMABLE}
\PYG{g+gp}{... }        \PYG{o}{|} \PYG{n}{CommodityUsage}\PYG{o}{.}\PYG{n}{PRODUCT}\PYG{p}{,}
\PYG{g+gp}{... }    \PYG{n}{CommodityUsage}\PYG{o}{.}\PYG{n}{ENERGY} \PYG{o}{|} \PYG{n}{CommodityUsage}\PYG{o}{.}\PYG{n}{PRODUCT}\PYG{p}{,}
\PYG{g+gp}{... }\PYG{p}{]}
\PYG{g+gp}{\PYGZgt{}\PYGZgt{}\PYGZgt{} }\PYG{n}{is\PYGZus{}fuel}\PYG{p}{(}\PYG{n}{data}\PYG{p}{)}\PYG{o}{.}\PYG{n}{tolist}\PYG{p}{(}\PYG{p}{)}
\PYG{g+go}{[False, False, False, True, True, True, False]}
\end{sphinxVerbatim}

\end{fulllineitems}

\index{is\_material() (in module muse.commodities)@\spxentry{is\_material()}\spxextra{in module muse.commodities}}

\begin{fulllineitems}
\phantomsection\label{\detokenize{api:muse.commodities.is_material}}\pysiglinewithargsret{\sphinxcode{\sphinxupquote{muse.commodities.}}\sphinxbfcode{\sphinxupquote{is\_material}}}{\emph{\DUrole{n}{data}\DUrole{p}{:} \DUrole{n}{Sequence\DUrole{p}{{[}}muse.commodities.CommodityUsage\DUrole{p}{{]}}}}}{{ $\rightarrow$ numpy.ndarray}}
Any non\sphinxhyphen{}energy non\sphinxhyphen{}environmental consumable.
\subsubsection*{Examples}

\begin{sphinxVerbatim}[commandchars=\\\{\}]
\PYG{g+gp}{\PYGZgt{}\PYGZgt{}\PYGZgt{} }\PYG{k+kn}{from} \PYG{n+nn}{muse}\PYG{n+nn}{.}\PYG{n+nn}{commodities} \PYG{k+kn}{import} \PYG{n}{CommodityUsage}\PYG{p}{,} \PYG{n}{is\PYGZus{}material}
\PYG{g+gp}{\PYGZgt{}\PYGZgt{}\PYGZgt{} }\PYG{n}{data} \PYG{o}{=} \PYG{p}{[}
\PYG{g+gp}{... }    \PYG{n}{CommodityUsage}\PYG{o}{.}\PYG{n}{CONSUMABLE}\PYG{p}{,}
\PYG{g+gp}{... }    \PYG{n}{CommodityUsage}\PYG{o}{.}\PYG{n}{PRODUCT}\PYG{p}{,}
\PYG{g+gp}{... }    \PYG{n}{CommodityUsage}\PYG{o}{.}\PYG{n}{ENERGY}\PYG{p}{,}
\PYG{g+gp}{... }    \PYG{n}{CommodityUsage}\PYG{o}{.}\PYG{n}{ENERGY} \PYG{o}{|} \PYG{n}{CommodityUsage}\PYG{o}{.}\PYG{n}{CONSUMABLE}\PYG{p}{,}
\PYG{g+gp}{... }    \PYG{n}{CommodityUsage}\PYG{o}{.}\PYG{n}{CONSUMABLE} \PYG{o}{|} \PYG{n}{CommodityUsage}\PYG{o}{.}\PYG{n}{ENVIRONMENTAL}\PYG{p}{,}
\PYG{g+gp}{... }    \PYG{n}{CommodityUsage}\PYG{o}{.}\PYG{n}{ENERGY} \PYG{o}{|} \PYG{n}{CommodityUsage}\PYG{o}{.}\PYG{n}{CONSUMABLE}
\PYG{g+gp}{... }        \PYG{o}{|} \PYG{n}{CommodityUsage}\PYG{o}{.}\PYG{n}{PRODUCT}\PYG{p}{,}
\PYG{g+gp}{... }    \PYG{n}{CommodityUsage}\PYG{o}{.}\PYG{n}{CONSUMABLE} \PYG{o}{|} \PYG{n}{CommodityUsage}\PYG{o}{.}\PYG{n}{PRODUCT}\PYG{p}{,}
\PYG{g+gp}{... }\PYG{p}{]}
\PYG{g+gp}{\PYGZgt{}\PYGZgt{}\PYGZgt{} }\PYG{n}{is\PYGZus{}material}\PYG{p}{(}\PYG{n}{data}\PYG{p}{)}\PYG{o}{.}\PYG{n}{tolist}\PYG{p}{(}\PYG{p}{)}
\PYG{g+go}{[True, False, False, False, False, False, True]}
\end{sphinxVerbatim}

\end{fulllineitems}

\index{is\_other() (in module muse.commodities)@\spxentry{is\_other()}\spxextra{in module muse.commodities}}

\begin{fulllineitems}
\phantomsection\label{\detokenize{api:muse.commodities.is_other}}\pysiglinewithargsret{\sphinxcode{\sphinxupquote{muse.commodities.}}\sphinxbfcode{\sphinxupquote{is\_other}}}{\emph{\DUrole{n}{data}\DUrole{p}{:} \DUrole{n}{Sequence\DUrole{p}{{[}}muse.commodities.CommodityUsage\DUrole{p}{{]}}}}}{{ $\rightarrow$ numpy.ndarray}}
No flags are set.
\subsubsection*{Examples}

\begin{sphinxVerbatim}[commandchars=\\\{\}]
\PYG{g+gp}{\PYGZgt{}\PYGZgt{}\PYGZgt{} }\PYG{k+kn}{from} \PYG{n+nn}{muse}\PYG{n+nn}{.}\PYG{n+nn}{commodities} \PYG{k+kn}{import} \PYG{n}{CommodityUsage}\PYG{p}{,} \PYG{n}{is\PYGZus{}other}
\PYG{g+gp}{\PYGZgt{}\PYGZgt{}\PYGZgt{} }\PYG{n}{data} \PYG{o}{=} \PYG{p}{[}
\PYG{g+gp}{... }    \PYG{n}{CommodityUsage}\PYG{o}{.}\PYG{n}{OTHER}\PYG{p}{,}
\PYG{g+gp}{... }    \PYG{n}{CommodityUsage}\PYG{o}{.}\PYG{n}{PRODUCT}\PYG{p}{,}
\PYG{g+gp}{... }    \PYG{n}{CommodityUsage}\PYG{o}{.}\PYG{n}{PRODUCT} \PYG{o}{|} \PYG{n}{CommodityUsage}\PYG{o}{.}\PYG{n}{OTHER}\PYG{p}{,}
\PYG{g+gp}{... }\PYG{p}{]}
\PYG{g+gp}{\PYGZgt{}\PYGZgt{}\PYGZgt{} }\PYG{n}{is\PYGZus{}other}\PYG{p}{(}\PYG{n}{data}\PYG{p}{)}\PYG{o}{.}\PYG{n}{tolist}\PYG{p}{(}\PYG{p}{)}
\PYG{g+go}{[True, False, False]}
\end{sphinxVerbatim}

\end{fulllineitems}

\index{is\_pollutant() (in module muse.commodities)@\spxentry{is\_pollutant()}\spxextra{in module muse.commodities}}

\begin{fulllineitems}
\phantomsection\label{\detokenize{api:muse.commodities.is_pollutant}}\pysiglinewithargsret{\sphinxcode{\sphinxupquote{muse.commodities.}}\sphinxbfcode{\sphinxupquote{is\_pollutant}}}{\emph{\DUrole{n}{data}\DUrole{p}{:} \DUrole{n}{Sequence\DUrole{p}{{[}}muse.commodities.CommodityUsage\DUrole{p}{{]}}}}}{{ $\rightarrow$ numpy.ndarray}}
Environmental product.
\subsubsection*{Examples}

\begin{sphinxVerbatim}[commandchars=\\\{\}]
\PYG{g+gp}{\PYGZgt{}\PYGZgt{}\PYGZgt{} }\PYG{k+kn}{from} \PYG{n+nn}{muse}\PYG{n+nn}{.}\PYG{n+nn}{commodities} \PYG{k+kn}{import} \PYG{n}{CommodityUsage}\PYG{p}{,} \PYG{n}{is\PYGZus{}pollutant}
\PYG{g+gp}{\PYGZgt{}\PYGZgt{}\PYGZgt{} }\PYG{n}{data} \PYG{o}{=} \PYG{p}{[}
\PYG{g+gp}{... }    \PYG{n}{CommodityUsage}\PYG{o}{.}\PYG{n}{CONSUMABLE}\PYG{p}{,}
\PYG{g+gp}{... }    \PYG{n}{CommodityUsage}\PYG{o}{.}\PYG{n}{PRODUCT}\PYG{p}{,}
\PYG{g+gp}{... }    \PYG{n}{CommodityUsage}\PYG{o}{.}\PYG{n}{ENVIRONMENTAL}\PYG{p}{,}
\PYG{g+gp}{... }    \PYG{n}{CommodityUsage}\PYG{o}{.}\PYG{n}{PRODUCT} \PYG{o}{|} \PYG{n}{CommodityUsage}\PYG{o}{.}\PYG{n}{CONSUMABLE}\PYG{p}{,}
\PYG{g+gp}{... }    \PYG{n}{CommodityUsage}\PYG{o}{.}\PYG{n}{ENVIRONMENTAL} \PYG{o}{|} \PYG{n}{CommodityUsage}\PYG{o}{.}\PYG{n}{PRODUCT}\PYG{p}{,}
\PYG{g+gp}{... }\PYG{p}{]}
\PYG{g+gp}{\PYGZgt{}\PYGZgt{}\PYGZgt{} }\PYG{n}{is\PYGZus{}pollutant}\PYG{p}{(}\PYG{n}{data}\PYG{p}{)}\PYG{o}{.}\PYG{n}{tolist}\PYG{p}{(}\PYG{p}{)}
\PYG{g+go}{[False, False, False, False, True]}
\end{sphinxVerbatim}

\end{fulllineitems}



\subsection{Regression functions}
\label{\detokenize{api:module-muse.regressions}}\label{\detokenize{api:regression-functions}}\index{module@\spxentry{module}!muse.regressions@\spxentry{muse.regressions}}\index{muse.regressions@\spxentry{muse.regressions}!module@\spxentry{module}}
Functions and functors to compute macro\sphinxhyphen{}drivers.
\index{Exponential (class in muse.regressions)@\spxentry{Exponential}\spxextra{class in muse.regressions}}

\begin{fulllineitems}
\phantomsection\label{\detokenize{api:muse.regressions.Exponential}}\pysiglinewithargsret{\sphinxbfcode{\sphinxupquote{class }}\sphinxcode{\sphinxupquote{muse.regressions.}}\sphinxbfcode{\sphinxupquote{Exponential}}}{\emph{\DUrole{o}{*}\DUrole{n}{args}}, \emph{\DUrole{o}{**}\DUrole{n}{kwds}}}{}
Regression function: exponential

This functor is a regression function registered with MUSE as ‘exponential’.

\end{fulllineitems}

\index{ExponentialAdj (class in muse.regressions)@\spxentry{ExponentialAdj}\spxextra{class in muse.regressions}}

\begin{fulllineitems}
\phantomsection\label{\detokenize{api:muse.regressions.ExponentialAdj}}\pysiglinewithargsret{\sphinxbfcode{\sphinxupquote{class }}\sphinxcode{\sphinxupquote{muse.regressions.}}\sphinxbfcode{\sphinxupquote{ExponentialAdj}}}{\emph{\DUrole{o}{*}\DUrole{n}{args}}, \emph{\DUrole{o}{**}\DUrole{n}{kwds}}}{}
Regression function: exponentialadj

This functor is a regression function registered with MUSE as ‘exponentialadj’.

\end{fulllineitems}

\index{Linear (class in muse.regressions)@\spxentry{Linear}\spxextra{class in muse.regressions}}

\begin{fulllineitems}
\phantomsection\label{\detokenize{api:muse.regressions.Linear}}\pysiglinewithargsret{\sphinxbfcode{\sphinxupquote{class }}\sphinxcode{\sphinxupquote{muse.regressions.}}\sphinxbfcode{\sphinxupquote{Linear}}}{\emph{\DUrole{o}{*}\DUrole{n}{args}}, \emph{\DUrole{o}{**}\DUrole{n}{kwds}}}{}
a * population + b * (gdp \sphinxhyphen{} gdp{[}2010{]}/population{[}2010{]} * population)

\end{fulllineitems}

\index{Logistic (class in muse.regressions)@\spxentry{Logistic}\spxextra{class in muse.regressions}}

\begin{fulllineitems}
\phantomsection\label{\detokenize{api:muse.regressions.Logistic}}\pysiglinewithargsret{\sphinxbfcode{\sphinxupquote{class }}\sphinxcode{\sphinxupquote{muse.regressions.}}\sphinxbfcode{\sphinxupquote{Logistic}}}{\emph{\DUrole{o}{*}\DUrole{n}{args}}, \emph{\DUrole{o}{**}\DUrole{n}{kwds}}}{}
Regression function: logistic

This functor is a regression function registered with MUSE as ‘logistic’.

\end{fulllineitems}

\index{LogisticSigmoid (class in muse.regressions)@\spxentry{LogisticSigmoid}\spxextra{class in muse.regressions}}

\begin{fulllineitems}
\phantomsection\label{\detokenize{api:muse.regressions.LogisticSigmoid}}\pysiglinewithargsret{\sphinxbfcode{\sphinxupquote{class }}\sphinxcode{\sphinxupquote{muse.regressions.}}\sphinxbfcode{\sphinxupquote{LogisticSigmoid}}}{\emph{\DUrole{o}{*}\DUrole{n}{args}}, \emph{\DUrole{o}{**}\DUrole{n}{kwds}}}{}
Regression function: logisticsigmoid

This functor is a regression function registered with MUSE as ‘logisticsigmoid’.

\end{fulllineitems}

\index{Loglog (class in muse.regressions)@\spxentry{Loglog}\spxextra{class in muse.regressions}}

\begin{fulllineitems}
\phantomsection\label{\detokenize{api:muse.regressions.Loglog}}\pysiglinewithargsret{\sphinxbfcode{\sphinxupquote{class }}\sphinxcode{\sphinxupquote{muse.regressions.}}\sphinxbfcode{\sphinxupquote{Loglog}}}{\emph{\DUrole{o}{*}\DUrole{n}{args}}, \emph{\DUrole{o}{**}\DUrole{n}{kwds}}}{}
Regression function: loglog

This functor is a regression function registered with MUSE as ‘loglog’.

\end{fulllineitems}

\index{endogenous\_demand() (in module muse.regressions)@\spxentry{endogenous\_demand()}\spxextra{in module muse.regressions}}

\begin{fulllineitems}
\phantomsection\label{\detokenize{api:muse.regressions.endogenous_demand}}\pysiglinewithargsret{\sphinxcode{\sphinxupquote{muse.regressions.}}\sphinxbfcode{\sphinxupquote{endogenous\_demand}}}{\emph{\DUrole{n}{regression\_parameters}\DUrole{p}{:} \DUrole{n}{Union\DUrole{p}{{[}}str\DUrole{p}{, }pathlib.Path\DUrole{p}{, }xarray.core.dataset.Dataset\DUrole{p}{{]}}}}, \emph{\DUrole{n}{drivers}\DUrole{p}{:} \DUrole{n}{Union\DUrole{p}{{[}}str\DUrole{p}{, }pathlib.Path\DUrole{p}{, }xarray.core.dataset.Dataset\DUrole{p}{{]}}}}, \emph{\DUrole{n}{sector}\DUrole{p}{:} \DUrole{n}{Optional\DUrole{p}{{[}}Union\DUrole{p}{{[}}str\DUrole{p}{, }Sequence\DUrole{p}{{]}}\DUrole{p}{{]}}} \DUrole{o}{=} \DUrole{default_value}{None}}, \emph{\DUrole{o}{**}\DUrole{n}{kwargs}}}{{ $\rightarrow$ xarray.core.dataset.Dataset}}
Endogenous demand based on macro drivers and regression parameters.

\end{fulllineitems}

\index{factory() (in module muse.regressions)@\spxentry{factory()}\spxextra{in module muse.regressions}}

\begin{fulllineitems}
\phantomsection\label{\detokenize{api:muse.regressions.factory}}\pysiglinewithargsret{\sphinxcode{\sphinxupquote{muse.regressions.}}\sphinxbfcode{\sphinxupquote{factory}}}{\emph{\DUrole{n}{regression\_parameters}\DUrole{p}{:} \DUrole{n}{Union\DUrole{p}{{[}}str\DUrole{p}{, }pathlib.Path\DUrole{p}{, }xarray.core.dataset.Dataset\DUrole{p}{{]}}}}, \emph{\DUrole{n}{sector}\DUrole{p}{:} \DUrole{n}{Optional\DUrole{p}{{[}}Union\DUrole{p}{{[}}str\DUrole{p}{, }Sequence\DUrole{p}{{[}}str\DUrole{p}{{]}}\DUrole{p}{{]}}\DUrole{p}{{]}}} \DUrole{o}{=} \DUrole{default_value}{None}}}{{ $\rightarrow$ muse.regressions.Regression}}
Creates regression functor from standard MUSE data for given sector.

\end{fulllineitems}

\index{register\_regression() (in module muse.regressions)@\spxentry{register\_regression()}\spxextra{in module muse.regressions}}

\begin{fulllineitems}
\phantomsection\label{\detokenize{api:muse.regressions.register_regression}}\pysiglinewithargsret{\sphinxcode{\sphinxupquote{muse.regressions.}}\sphinxbfcode{\sphinxupquote{register\_regression}}}{\emph{\DUrole{n}{Functor}\DUrole{p}{:} \DUrole{n}{Optional\DUrole{p}{{[}}muse.regressions.Regression\DUrole{p}{{]}}} \DUrole{o}{=} \DUrole{default_value}{None}}, \emph{\DUrole{n}{name}\DUrole{p}{:} \DUrole{n}{Optional\DUrole{p}{{[}}str\DUrole{p}{{]}}} \DUrole{o}{=} \DUrole{default_value}{None}}}{{ $\rightarrow$ muse.regressions.Regression}}
Registers a functor with MUSE regressions.

Regression functors are registered with MUSE so that the functors can be
called easily on created.

functor name that the functor is registered with defaults to the snake\_case
version of the functor name.  However, it can also be specified explicitly
as a \sphinxstyleemphasis{keyword} argument. In any case, it must be unique amongst all
registered regression functor.

\end{fulllineitems}



\subsection{Functionality Registration}
\label{\detokenize{api:module-muse.registration}}\label{\detokenize{api:functionality-registration}}\index{module@\spxentry{module}!muse.registration@\spxentry{muse.registration}}\index{muse.registration@\spxentry{muse.registration}!module@\spxentry{module}}
Registrators that allow pluggable data to logic transforms.
\index{registrator() (in module muse.registration)@\spxentry{registrator()}\spxextra{in module muse.registration}}

\begin{fulllineitems}
\phantomsection\label{\detokenize{api:muse.registration.registrator}}\pysiglinewithargsret{\sphinxcode{\sphinxupquote{muse.registration.}}\sphinxbfcode{\sphinxupquote{registrator}}}{\emph{\DUrole{n}{decorator}\DUrole{p}{:} \DUrole{n}{Callable} \DUrole{o}{=} \DUrole{default_value}{None}}, \emph{\DUrole{n}{registry}\DUrole{p}{:} \DUrole{n}{MutableMapping} \DUrole{o}{=} \DUrole{default_value}{None}}, \emph{\DUrole{n}{logname}\DUrole{p}{:} \DUrole{n}{Optional\DUrole{p}{{[}}str\DUrole{p}{{]}}} \DUrole{o}{=} \DUrole{default_value}{None}}, \emph{\DUrole{n}{loglevel}\DUrole{p}{:} \DUrole{n}{Optional\DUrole{p}{{[}}str\DUrole{p}{{]}}} \DUrole{o}{=} \DUrole{default_value}{\textquotesingle{}Debug\textquotesingle{}}}}{{ $\rightarrow$ Callable}}
A decorator to create a decorator that registers functions with MUSE.

This is a decorator that takes another decorator as an argument. Hence it
returns a decorator. It simplifies and standardizes creating decorators to
register functions with muse.

The registrator expects as non\sphinxhyphen{}optional keyword argument a registry where
the resulting decorator will register functions.

Furthermore, the final function (the one passed to the decorator passed to
this function) will emit a standardized log\sphinxhyphen{}call.
\subsubsection*{Example}

At it’s simplest, creating a registrator and registrating happens by
first declaring a registry.

\begin{sphinxVerbatim}[commandchars=\\\{\}]
\PYG{g+gp}{\PYGZgt{}\PYGZgt{}\PYGZgt{} }\PYG{n}{REGISTRY} \PYG{o}{=} \PYG{p}{\PYGZob{}}\PYG{p}{\PYGZcb{}}
\end{sphinxVerbatim}

In general, it will be a variable owned directly by a module, hence the
all\sphinxhyphen{}caps. Creating the registrator then follows:

\begin{sphinxVerbatim}[commandchars=\\\{\}]
\PYG{g+gp}{\PYGZgt{}\PYGZgt{}\PYGZgt{} }\PYG{k+kn}{from} \PYG{n+nn}{muse}\PYG{n+nn}{.}\PYG{n+nn}{registration} \PYG{k+kn}{import} \PYG{n}{registrator}
\PYG{g+gp}{\PYGZgt{}\PYGZgt{}\PYGZgt{} }\PYG{n+nd}{@registrator}\PYG{p}{(}\PYG{n}{registry}\PYG{o}{=}\PYG{n}{REGISTRY}\PYG{p}{,} \PYG{n}{logname}\PYG{o}{=}\PYG{l+s+s1}{\PYGZsq{}}\PYG{l+s+s1}{my stuff}\PYG{l+s+s1}{\PYGZsq{}}\PYG{p}{,}
\PYG{g+gp}{... }             \PYG{n}{loglevel}\PYG{o}{=}\PYG{l+s+s1}{\PYGZsq{}}\PYG{l+s+s1}{Info}\PYG{l+s+s1}{\PYGZsq{}}\PYG{p}{)}
\PYG{g+gp}{... }\PYG{k}{def} \PYG{n+nf}{register\PYGZus{}mystuff}\PYG{p}{(}\PYG{n}{function}\PYG{p}{)}\PYG{p}{:}
\PYG{g+gp}{... }    \PYG{k}{return} \PYG{n}{function}
\end{sphinxVerbatim}

This registrator does nothing more than register the function. A more
interesting example is given below. Then a function can be registered:

\begin{sphinxVerbatim}[commandchars=\\\{\}]
\PYG{g+gp}{\PYGZgt{}\PYGZgt{}\PYGZgt{} }\PYG{n+nd}{@register\PYGZus{}mystuff}\PYG{p}{(}\PYG{n}{name}\PYG{o}{=}\PYG{l+s+s1}{\PYGZsq{}}\PYG{l+s+s1}{yoyo}\PYG{l+s+s1}{\PYGZsq{}}\PYG{p}{)}
\PYG{g+gp}{... }\PYG{k}{def} \PYG{n+nf}{my\PYGZus{}registered\PYGZus{}function}\PYG{p}{(}\PYG{n}{a}\PYG{p}{,} \PYG{n}{b}\PYG{p}{)}\PYG{p}{:}
\PYG{g+gp}{... }    \PYG{k}{return} \PYG{n}{a} \PYG{o}{+} \PYG{n}{b}
\end{sphinxVerbatim}

The argument ‘yoyo’ is optional. It adds aliases for the function in the
registry. In any case, functions are registered with default aliases
corresponding to standard name variations, e.g. CamelCase, camelCase,
and kebab\sphinxhyphen{}case, as illustrated below:

\begin{sphinxVerbatim}[commandchars=\\\{\}]
\PYG{g+gp}{\PYGZgt{}\PYGZgt{}\PYGZgt{} }\PYG{n}{REGISTRY}\PYG{p}{[}\PYG{l+s+s1}{\PYGZsq{}}\PYG{l+s+s1}{my\PYGZus{}registered\PYGZus{}function}\PYG{l+s+s1}{\PYGZsq{}}\PYG{p}{]} \PYG{o+ow}{is} \PYG{n}{my\PYGZus{}registered\PYGZus{}function}
\PYG{g+go}{True}
\PYG{g+gp}{\PYGZgt{}\PYGZgt{}\PYGZgt{} }\PYG{n}{REGISTRY}\PYG{p}{[}\PYG{l+s+s1}{\PYGZsq{}}\PYG{l+s+s1}{my\PYGZhy{}registered\PYGZhy{}function}\PYG{l+s+s1}{\PYGZsq{}}\PYG{p}{]} \PYG{o+ow}{is} \PYG{n}{my\PYGZus{}registered\PYGZus{}function}
\PYG{g+go}{True}
\PYG{g+gp}{\PYGZgt{}\PYGZgt{}\PYGZgt{} }\PYG{n}{REGISTRY}\PYG{p}{[}\PYG{l+s+s1}{\PYGZsq{}}\PYG{l+s+s1}{yoyo}\PYG{l+s+s1}{\PYGZsq{}}\PYG{p}{]} \PYG{o+ow}{is} \PYG{n}{my\PYGZus{}registered\PYGZus{}function}
\PYG{g+go}{True}
\end{sphinxVerbatim}

A more interesting case would involve the registrator automatically
adding functionality to the input function. For instance, the inputs
could be manipulated and the result of the function could be
automatically transformed to a string:

\begin{sphinxVerbatim}[commandchars=\\\{\}]
\PYG{g+gp}{\PYGZgt{}\PYGZgt{}\PYGZgt{} }\PYG{k+kn}{from} \PYG{n+nn}{muse}\PYG{n+nn}{.}\PYG{n+nn}{registration} \PYG{k+kn}{import} \PYG{n}{registrator}
\PYG{g+gp}{\PYGZgt{}\PYGZgt{}\PYGZgt{} }\PYG{n+nd}{@registrator}\PYG{p}{(}\PYG{n}{registry}\PYG{o}{=}\PYG{n}{REGISTRY}\PYG{p}{)}
\PYG{g+gp}{... }\PYG{k}{def} \PYG{n+nf}{register\PYGZus{}mystuff}\PYG{p}{(}\PYG{n}{function}\PYG{p}{)}\PYG{p}{:}
\PYG{g+gp}{... }    \PYG{k+kn}{from} \PYG{n+nn}{functools} \PYG{k+kn}{import} \PYG{n}{wraps}
\PYG{g+gp}{...}
\PYG{g+gp}{... }    \PYG{n+nd}{@wraps}\PYG{p}{(}\PYG{n}{function}\PYG{p}{)}
\PYG{g+gp}{... }    \PYG{k}{def} \PYG{n+nf}{decorated}\PYG{p}{(}\PYG{n}{a}\PYG{p}{,} \PYG{n}{b}\PYG{p}{)} \PYG{o}{\PYGZhy{}}\PYG{o}{\PYGZgt{}} \PYG{n+nb}{str}\PYG{p}{:}
\PYG{g+gp}{... }        \PYG{n}{result} \PYG{o}{=} \PYG{n}{function}\PYG{p}{(}\PYG{l+m+mi}{2} \PYG{o}{*} \PYG{n}{a}\PYG{p}{,} \PYG{l+m+mi}{3} \PYG{o}{*} \PYG{n}{b}\PYG{p}{)}
\PYG{g+gp}{... }        \PYG{k}{return} \PYG{n+nb}{str}\PYG{p}{(}\PYG{n}{result}\PYG{p}{)}
\PYG{g+gp}{...}
\PYG{g+gp}{... }    \PYG{k}{return} \PYG{n}{decorated}
\end{sphinxVerbatim}

\begin{sphinxVerbatim}[commandchars=\\\{\}]
\PYG{g+gp}{\PYGZgt{}\PYGZgt{}\PYGZgt{} }\PYG{n+nd}{@register\PYGZus{}mystuff}
\PYG{g+gp}{... }\PYG{k}{def} \PYG{n+nf}{other}\PYG{p}{(}\PYG{n}{a}\PYG{p}{,} \PYG{n}{b}\PYG{p}{)}\PYG{p}{:}
\PYG{g+gp}{... }    \PYG{k}{return} \PYG{n}{a} \PYG{o}{+} \PYG{n}{b}
\end{sphinxVerbatim}

\begin{sphinxVerbatim}[commandchars=\\\{\}]
\PYG{g+gp}{\PYGZgt{}\PYGZgt{}\PYGZgt{} }\PYG{n+nb}{isinstance}\PYG{p}{(}\PYG{n}{REGISTRY}\PYG{p}{[}\PYG{l+s+s1}{\PYGZsq{}}\PYG{l+s+s1}{other}\PYG{l+s+s1}{\PYGZsq{}}\PYG{p}{]}\PYG{p}{(}\PYG{o}{\PYGZhy{}}\PYG{l+m+mi}{3}\PYG{p}{,} \PYG{l+m+mi}{2}\PYG{p}{)}\PYG{p}{,} \PYG{n+nb}{str}\PYG{p}{)}
\PYG{g+go}{True}
\PYG{g+gp}{\PYGZgt{}\PYGZgt{}\PYGZgt{} }\PYG{n}{REGISTRY}\PYG{p}{[}\PYG{l+s+s1}{\PYGZsq{}}\PYG{l+s+s1}{other}\PYG{l+s+s1}{\PYGZsq{}}\PYG{p}{]}\PYG{p}{(}\PYG{o}{\PYGZhy{}}\PYG{l+m+mi}{3}\PYG{p}{,} \PYG{l+m+mi}{2}\PYG{p}{)} \PYG{o}{==} \PYG{l+s+s2}{\PYGZdq{}}\PYG{l+s+s2}{0}\PYG{l+s+s2}{\PYGZdq{}}
\PYG{g+go}{True}
\end{sphinxVerbatim}

\end{fulllineitems}



\subsection{Utilities}
\label{\detokenize{api:module-muse.utilities}}\label{\detokenize{api:utilities}}\index{module@\spxentry{module}!muse.utilities@\spxentry{muse.utilities}}\index{muse.utilities@\spxentry{muse.utilities}!module@\spxentry{module}}
Collection of functions and stand\sphinxhyphen{}alone algorithms.
\index{agent\_concatenation() (in module muse.utilities)@\spxentry{agent\_concatenation()}\spxextra{in module muse.utilities}}

\begin{fulllineitems}
\phantomsection\label{\detokenize{api:muse.utilities.agent_concatenation}}\pysiglinewithargsret{\sphinxcode{\sphinxupquote{muse.utilities.}}\sphinxbfcode{\sphinxupquote{agent\_concatenation}}}{\emph{\DUrole{n}{data}\DUrole{p}{:} \DUrole{n}{Mapping\DUrole{p}{{[}}Hashable\DUrole{p}{, }Union\DUrole{p}{{[}}xarray.core.dataarray.DataArray\DUrole{p}{, }xarray.core.dataset.Dataset\DUrole{p}{{]}}\DUrole{p}{{]}}}}, \emph{\DUrole{n}{dim}\DUrole{p}{:} \DUrole{n}{str} \DUrole{o}{=} \DUrole{default_value}{\textquotesingle{}asset\textquotesingle{}}}, \emph{\DUrole{n}{name}\DUrole{p}{:} \DUrole{n}{str} \DUrole{o}{=} \DUrole{default_value}{\textquotesingle{}agent\textquotesingle{}}}, \emph{\DUrole{n}{fill\_value}\DUrole{p}{:} \DUrole{n}{Any} \DUrole{o}{=} \DUrole{default_value}{0}}}{{ $\rightarrow$ Union\DUrole{p}{{[}}xarray.core.dataarray.DataArray\DUrole{p}{, }xarray.core.dataset.Dataset\DUrole{p}{{]}}}}
Concatenates input map along given dimension.
\subsubsection*{Example}

Lets create sets of random assets to work with. We set the seed so that this
test can be reproduced exactly.

\begin{sphinxVerbatim}[commandchars=\\\{\}]
\PYG{g+gp}{\PYGZgt{}\PYGZgt{}\PYGZgt{} }\PYG{k+kn}{from} \PYG{n+nn}{muse}\PYG{n+nn}{.}\PYG{n+nn}{examples} \PYG{k+kn}{import} \PYG{n}{random\PYGZus{}agent\PYGZus{}assets}
\PYG{g+gp}{\PYGZgt{}\PYGZgt{}\PYGZgt{} }\PYG{n}{rng} \PYG{o}{=} \PYG{n}{np}\PYG{o}{.}\PYG{n}{random}\PYG{o}{.}\PYG{n}{default\PYGZus{}rng}\PYG{p}{(}\PYG{l+m+mi}{1234}\PYG{p}{)}
\PYG{g+gp}{\PYGZgt{}\PYGZgt{}\PYGZgt{} }\PYG{n}{assets} \PYG{o}{=} \PYG{p}{\PYGZob{}}\PYG{n}{i}\PYG{p}{:} \PYG{n}{random\PYGZus{}agent\PYGZus{}assets}\PYG{p}{(}\PYG{n}{rng}\PYG{p}{)} \PYG{k}{for} \PYG{n}{i} \PYG{o+ow}{in} \PYG{n+nb}{range}\PYG{p}{(}\PYG{l+m+mi}{5}\PYG{p}{)}\PYG{p}{\PYGZcb{}}
\end{sphinxVerbatim}

The concatenation will create a new dataset (or datarray) combining all the
inputs along the dimension “asset”. The origin of each datum is retained in a
new coordinate “agent” with dimension “asset”.

\begin{sphinxVerbatim}[commandchars=\\\{\}]
\PYG{g+gp}{\PYGZgt{}\PYGZgt{}\PYGZgt{} }\PYG{k+kn}{from} \PYG{n+nn}{muse}\PYG{n+nn}{.}\PYG{n+nn}{utilities} \PYG{k+kn}{import} \PYG{n}{agent\PYGZus{}concatenation}
\PYG{g+gp}{\PYGZgt{}\PYGZgt{}\PYGZgt{} }\PYG{n}{aggregate} \PYG{o}{=} \PYG{n}{agent\PYGZus{}concatenation}\PYG{p}{(}\PYG{n}{assets}\PYG{p}{)}
\PYG{g+gp}{\PYGZgt{}\PYGZgt{}\PYGZgt{} }\PYG{n}{aggregate}
\PYG{g+go}{\PYGZlt{}xarray.Dataset\PYGZgt{}}
\PYG{g+go}{Dimensions:     (asset: 19, year: 12)}
\PYG{g+go}{Coordinates:}
\PYG{g+go}{  * year        (year) int64 2033 2035 2036 2037 2039 ... 2046 2047 2048 2049}
\PYG{g+go}{    technology  (asset) \PYGZlt{}U9 \PYGZsq{}oven\PYGZsq{} \PYGZsq{}stove\PYGZsq{} \PYGZsq{}oven\PYGZsq{} ... \PYGZsq{}stove\PYGZsq{} \PYGZsq{}oven\PYGZsq{} \PYGZsq{}thermomix\PYGZsq{}}
\PYG{g+go}{    region      (asset) \PYGZlt{}U9 \PYGZsq{}Brexitham\PYGZsq{} \PYGZsq{}Brexitham\PYGZsq{} ... \PYGZsq{}Brexitham\PYGZsq{} \PYGZsq{}Brexitham\PYGZsq{}}
\PYG{g+go}{    agent       (asset) ... 0 0 0 0 0 1 1 1 2 2 2 2 3 3 3 4 4 4 4}
\PYG{g+go}{    installed   (asset) int64 2030 2025 2030 2010 2030 ... 2025 2030 2010 2025}
\PYG{g+go}{Dimensions without coordinates: asset}
\PYG{g+go}{Data variables:}
\PYG{g+go}{    capacity    (asset, year) float64 26.0 26.0 26.0 56.0 ... 62.0 62.0 62.0}
\end{sphinxVerbatim}

Note that the \sphinxtitleref{dtype} of the capacity has changed from integers to floating
points. This is due to how \sphinxcode{\sphinxupquote{xarray}} performs the operation.

We can check that all the data from each agent is indeed present in the
aggregate.

\begin{sphinxVerbatim}[commandchars=\\\{\}]
\PYG{g+gp}{\PYGZgt{}\PYGZgt{}\PYGZgt{} }\PYG{k}{for} \PYG{n}{agent}\PYG{p}{,} \PYG{n}{inventory} \PYG{o+ow}{in} \PYG{n}{assets}\PYG{o}{.}\PYG{n}{items}\PYG{p}{(}\PYG{p}{)}\PYG{p}{:}
\PYG{g+gp}{... }   \PYG{k}{assert} \PYG{p}{(}\PYG{n}{aggregate}\PYG{o}{.}\PYG{n}{sel}\PYG{p}{(}\PYG{n}{asset}\PYG{o}{=}\PYG{n}{aggregate}\PYG{o}{.}\PYG{n}{agent} \PYG{o}{==} \PYG{n}{agent}\PYG{p}{)} \PYG{o}{==} \PYG{n}{inventory}\PYG{p}{)}\PYG{o}{.}\PYG{n}{all}\PYG{p}{(}\PYG{p}{)}
\end{sphinxVerbatim}

However, it should be noted that the data is not always strictly equivalent:
dimensions outside of “assets” (most notably “year”) will include all points
from all agents. Missing values for the “year” dimension are forward filled (and
backfilled with zeros). Others are left with “NaN”.

\end{fulllineitems}

\index{aggregate\_technology\_model() (in module muse.utilities)@\spxentry{aggregate\_technology\_model()}\spxextra{in module muse.utilities}}

\begin{fulllineitems}
\phantomsection\label{\detokenize{api:muse.utilities.aggregate_technology_model}}\pysiglinewithargsret{\sphinxcode{\sphinxupquote{muse.utilities.}}\sphinxbfcode{\sphinxupquote{aggregate\_technology\_model}}}{\emph{\DUrole{n}{data}\DUrole{p}{:} \DUrole{n}{Union\DUrole{p}{{[}}xarray.core.dataarray.DataArray\DUrole{p}{, }xarray.core.dataset.Dataset\DUrole{p}{{]}}}}, \emph{\DUrole{n}{dim}\DUrole{p}{:} \DUrole{n}{str} \DUrole{o}{=} \DUrole{default_value}{\textquotesingle{}asset\textquotesingle{}}}, \emph{\DUrole{n}{drop}\DUrole{p}{:} \DUrole{n}{Union\DUrole{p}{{[}}str\DUrole{p}{, }Sequence\DUrole{p}{{[}}str\DUrole{p}{{]}}\DUrole{p}{{]}}} \DUrole{o}{=} \DUrole{default_value}{\textquotesingle{}installed\textquotesingle{}}}}{{ $\rightarrow$ Union\DUrole{p}{{[}}xarray.core.dataarray.DataArray\DUrole{p}{, }xarray.core.dataset.Dataset\DUrole{p}{{]}}}}
Aggregate together assets with the same installation year.

The assets of a given agent, region, and technology but different installation year
are grouped together and summed over.
\subsubsection*{Example}

We first create a random set of agent assets and aggregate them.
Some of these agents own assets from the same technology but potentially with
different installation year. This function will aggregate together all assets
of a given agent with same technology.

\begin{sphinxVerbatim}[commandchars=\\\{\}]
\PYG{g+gp}{\PYGZgt{}\PYGZgt{}\PYGZgt{} }\PYG{k+kn}{from} \PYG{n+nn}{muse}\PYG{n+nn}{.}\PYG{n+nn}{examples} \PYG{k+kn}{import} \PYG{n}{random\PYGZus{}agent\PYGZus{}assets}
\PYG{g+gp}{\PYGZgt{}\PYGZgt{}\PYGZgt{} }\PYG{k+kn}{from} \PYG{n+nn}{muse}\PYG{n+nn}{.}\PYG{n+nn}{utilities} \PYG{k+kn}{import} \PYG{n}{agent\PYGZus{}concatenation}\PYG{p}{,} \PYG{n}{aggregate\PYGZus{}technology\PYGZus{}model}
\PYG{g+gp}{\PYGZgt{}\PYGZgt{}\PYGZgt{} }\PYG{n}{rng} \PYG{o}{=} \PYG{n}{np}\PYG{o}{.}\PYG{n}{random}\PYG{o}{.}\PYG{n}{default\PYGZus{}rng}\PYG{p}{(}\PYG{l+m+mi}{1234}\PYG{p}{)}
\PYG{g+gp}{\PYGZgt{}\PYGZgt{}\PYGZgt{} }\PYG{n}{agent\PYGZus{}assets} \PYG{o}{=} \PYG{p}{\PYGZob{}}\PYG{n}{i}\PYG{p}{:} \PYG{n}{random\PYGZus{}agent\PYGZus{}assets}\PYG{p}{(}\PYG{n}{rng}\PYG{p}{)} \PYG{k}{for} \PYG{n}{i} \PYG{o+ow}{in} \PYG{n+nb}{range}\PYG{p}{(}\PYG{l+m+mi}{5}\PYG{p}{)}\PYG{p}{\PYGZcb{}}
\PYG{g+gp}{\PYGZgt{}\PYGZgt{}\PYGZgt{} }\PYG{n}{assets} \PYG{o}{=} \PYG{n}{agent\PYGZus{}concatenation}\PYG{p}{(}\PYG{n}{agent\PYGZus{}assets}\PYG{p}{)}
\PYG{g+gp}{\PYGZgt{}\PYGZgt{}\PYGZgt{} }\PYG{n}{reduced} \PYG{o}{=} \PYG{n}{aggregate\PYGZus{}technology\PYGZus{}model}\PYG{p}{(}\PYG{n}{assets}\PYG{p}{)}
\end{sphinxVerbatim}

We can check that the tuples (agent, technology) are unique (each agent works in
a single region):

\begin{sphinxVerbatim}[commandchars=\\\{\}]
\PYG{g+gp}{\PYGZgt{}\PYGZgt{}\PYGZgt{} }\PYG{n}{ids} \PYG{o}{=} \PYG{n+nb}{list}\PYG{p}{(}\PYG{n+nb}{zip}\PYG{p}{(}\PYG{n}{reduced}\PYG{o}{.}\PYG{n}{agent}\PYG{o}{.}\PYG{n}{values}\PYG{p}{,} \PYG{n}{reduced}\PYG{o}{.}\PYG{n}{technology}\PYG{o}{.}\PYG{n}{values}\PYG{p}{)}\PYG{p}{)}
\PYG{g+gp}{\PYGZgt{}\PYGZgt{}\PYGZgt{} }\PYG{k}{assert} \PYG{n+nb}{len}\PYG{p}{(}\PYG{n+nb}{set}\PYG{p}{(}\PYG{n}{ids}\PYG{p}{)}\PYG{p}{)} \PYG{o}{==} \PYG{n+nb}{len}\PYG{p}{(}\PYG{n}{ids}\PYG{p}{)}
\end{sphinxVerbatim}

And we can check they correspond to the right summation:

\begin{sphinxVerbatim}[commandchars=\\\{\}]
\PYG{g+gp}{\PYGZgt{}\PYGZgt{}\PYGZgt{} }\PYG{k}{for} \PYG{n}{agent}\PYG{p}{,} \PYG{n}{technology} \PYG{o+ow}{in} \PYG{n+nb}{set}\PYG{p}{(}\PYG{n}{ids}\PYG{p}{)}\PYG{p}{:}
\PYG{g+gp}{... }    \PYG{n}{techsel} \PYG{o}{=} \PYG{n}{assets}\PYG{o}{.}\PYG{n}{technology} \PYG{o}{==} \PYG{n}{technology}
\PYG{g+gp}{... }    \PYG{n}{agsel} \PYG{o}{=} \PYG{n}{assets}\PYG{o}{.}\PYG{n}{agent} \PYG{o}{==} \PYG{n}{agent}
\PYG{g+gp}{... }    \PYG{n}{expected} \PYG{o}{=} \PYG{n}{assets}\PYG{o}{.}\PYG{n}{sel}\PYG{p}{(}\PYG{n}{asset}\PYG{o}{=}\PYG{n}{techsel} \PYG{o}{\PYGZam{}} \PYG{n}{agsel}\PYG{p}{)}\PYG{o}{.}\PYG{n}{sum}\PYG{p}{(}\PYG{l+s+s2}{\PYGZdq{}}\PYG{l+s+s2}{asset}\PYG{l+s+s2}{\PYGZdq{}}\PYG{p}{)}
\PYG{g+gp}{... }    \PYG{n}{techsel} \PYG{o}{=} \PYG{n}{reduced}\PYG{o}{.}\PYG{n}{technology} \PYG{o}{==} \PYG{n}{technology}
\PYG{g+gp}{... }    \PYG{n}{agsel} \PYG{o}{=} \PYG{n}{reduced}\PYG{o}{.}\PYG{n}{agent} \PYG{o}{==} \PYG{n}{agent}
\PYG{g+gp}{... }    \PYG{n}{actual} \PYG{o}{=} \PYG{n}{reduced}\PYG{o}{.}\PYG{n}{sel}\PYG{p}{(}\PYG{n}{asset}\PYG{o}{=}\PYG{n}{techsel} \PYG{o}{\PYGZam{}} \PYG{n}{agsel}\PYG{p}{)}
\PYG{g+gp}{... }    \PYG{k}{assert} \PYG{n+nb}{len}\PYG{p}{(}\PYG{n}{actual}\PYG{o}{.}\PYG{n}{asset}\PYG{p}{)} \PYG{o}{==} \PYG{l+m+mi}{1}
\PYG{g+gp}{... }    \PYG{k}{assert} \PYG{p}{(}\PYG{n}{actual} \PYG{o}{==} \PYG{n}{expected}\PYG{p}{)}\PYG{o}{.}\PYG{n}{all}\PYG{p}{(}\PYG{p}{)}
\end{sphinxVerbatim}

\end{fulllineitems}

\index{avoid\_repetitions() (in module muse.utilities)@\spxentry{avoid\_repetitions()}\spxextra{in module muse.utilities}}

\begin{fulllineitems}
\phantomsection\label{\detokenize{api:muse.utilities.avoid_repetitions}}\pysiglinewithargsret{\sphinxcode{\sphinxupquote{muse.utilities.}}\sphinxbfcode{\sphinxupquote{avoid\_repetitions}}}{\emph{\DUrole{n}{data}\DUrole{p}{:} \DUrole{n}{xarray.core.dataarray.DataArray}}, \emph{\DUrole{n}{dim}\DUrole{p}{:} \DUrole{n}{str} \DUrole{o}{=} \DUrole{default_value}{\textquotesingle{}year\textquotesingle{}}}}{{ $\rightarrow$ xarray.core.dataarray.DataArray}}
list of years such that there is no repetition in the data.

It removes the central year of any three consecutive years where all data is
the same. This means the original data can be reobtained via a linear
interpolation or a forward fill.

The first and last year are always preserved.

\end{fulllineitems}

\index{broadcast\_techs() (in module muse.utilities)@\spxentry{broadcast\_techs()}\spxextra{in module muse.utilities}}

\begin{fulllineitems}
\phantomsection\label{\detokenize{api:muse.utilities.broadcast_techs}}\pysiglinewithargsret{\sphinxcode{\sphinxupquote{muse.utilities.}}\sphinxbfcode{\sphinxupquote{broadcast\_techs}}}{\emph{\DUrole{n}{technologies}\DUrole{p}{:} \DUrole{n}{Union\DUrole{p}{{[}}xarray.core.dataset.Dataset\DUrole{p}{, }xarray.core.dataarray.DataArray\DUrole{p}{{]}}}}, \emph{\DUrole{n}{template}\DUrole{p}{:} \DUrole{n}{Union\DUrole{p}{{[}}xarray.core.dataarray.DataArray\DUrole{p}{, }xarray.core.dataset.Dataset\DUrole{p}{{]}}}}, \emph{\DUrole{n}{dimension}\DUrole{p}{:} \DUrole{n}{str} \DUrole{o}{=} \DUrole{default_value}{\textquotesingle{}asset\textquotesingle{}}}, \emph{\DUrole{n}{interpolation}\DUrole{p}{:} \DUrole{n}{str} \DUrole{o}{=} \DUrole{default_value}{\textquotesingle{}linear\textquotesingle{}}}, \emph{\DUrole{n}{installed\_as\_year}\DUrole{p}{:} \DUrole{n}{bool} \DUrole{o}{=} \DUrole{default_value}{True}}, \emph{\DUrole{o}{**}\DUrole{n}{kwargs}}}{{ $\rightarrow$ Union\DUrole{p}{{[}}xarray.core.dataset.Dataset\DUrole{p}{, }xarray.core.dataarray.DataArray\DUrole{p}{{]}}}}
Broadcasts technologies to the shape of template in given dimension.

The dimensions of the technologies are fully explicit, in that each concept
‘technology’, ‘region’, ‘year’ (for year of issue) is a separate dimension.
However, the dataset or data arrays representing other quantities, such as
capacity, are often flattened out with coordinates ‘region’, ‘installed’,
and ‘technology’ represented in a single ‘asset’ dimension. This latter
representation is sparse if not all combinations of ‘region’, ‘installed’,
and ‘technology’ are present, whereas the former represention makes it
easier to select a subset of the same.

This function broadcast the first represention to the shape and coordinates
of the second.
\begin{quote}\begin{description}
\item[{Parameters}] \leavevmode\begin{itemize}
\item {} 
\sphinxstyleliteralstrong{\sphinxupquote{technologies}} \textendash{} The dataset to broadcast

\item {} 
\sphinxstyleliteralstrong{\sphinxupquote{template}} \textendash{} the dataset or data\sphinxhyphen{}array to use as a template

\item {} 
\sphinxstyleliteralstrong{\sphinxupquote{dimension}} \textendash{} the name of the dimensiom from \sphinxtitleref{template} over which to
broadcast

\item {} 
\sphinxstyleliteralstrong{\sphinxupquote{interpolation}} \textendash{} interpolation method used across \sphinxtitleref{year}

\item {} 
\sphinxstyleliteralstrong{\sphinxupquote{installed\_as\_year}} \textendash{} if the coordinate \sphinxtitleref{installed} exists, then it is
applied to the \sphinxtitleref{year} dimension of the technologies dataset

\item {} 
\sphinxstyleliteralstrong{\sphinxupquote{kwargs}} \textendash{} further arguments are used initial filters over the
\sphinxtitleref{technologies} dataset.

\end{itemize}

\end{description}\end{quote}

\end{fulllineitems}

\index{clean\_assets() (in module muse.utilities)@\spxentry{clean\_assets()}\spxextra{in module muse.utilities}}

\begin{fulllineitems}
\phantomsection\label{\detokenize{api:muse.utilities.clean_assets}}\pysiglinewithargsret{\sphinxcode{\sphinxupquote{muse.utilities.}}\sphinxbfcode{\sphinxupquote{clean\_assets}}}{\emph{\DUrole{n}{assets}\DUrole{p}{:} \DUrole{n}{xarray.core.dataset.Dataset}}, \emph{\DUrole{n}{years}\DUrole{p}{:} \DUrole{n}{Union\DUrole{p}{{[}}int\DUrole{p}{, }Sequence\DUrole{p}{{[}}int\DUrole{p}{{]}}\DUrole{p}{{]}}}}}{}
Cleans up and prepares asset for current iteration.
\begin{itemize}
\item {} 
adds current and forecast year by backfilling missing entries

\item {} 
removes assets for which there is no capacity now or in the future

\end{itemize}

\end{fulllineitems}

\index{coords\_to\_multiindex() (in module muse.utilities)@\spxentry{coords\_to\_multiindex()}\spxextra{in module muse.utilities}}

\begin{fulllineitems}
\phantomsection\label{\detokenize{api:muse.utilities.coords_to_multiindex}}\pysiglinewithargsret{\sphinxcode{\sphinxupquote{muse.utilities.}}\sphinxbfcode{\sphinxupquote{coords\_to\_multiindex}}}{\emph{\DUrole{n}{data}\DUrole{p}{:} \DUrole{n}{Union\DUrole{p}{{[}}xarray.core.dataset.Dataset\DUrole{p}{, }xarray.core.dataarray.DataArray\DUrole{p}{{]}}}}, \emph{\DUrole{n}{dimension}\DUrole{p}{:} \DUrole{n}{str} \DUrole{o}{=} \DUrole{default_value}{\textquotesingle{}asset\textquotesingle{}}}}{{ $\rightarrow$ Union\DUrole{p}{{[}}xarray.core.dataset.Dataset\DUrole{p}{, }xarray.core.dataarray.DataArray\DUrole{p}{{]}}}}
Creates a multi\sphinxhyphen{}index from flattened multiple coords.

\end{fulllineitems}

\index{filter\_input() (in module muse.utilities)@\spxentry{filter\_input()}\spxextra{in module muse.utilities}}

\begin{fulllineitems}
\phantomsection\label{\detokenize{api:muse.utilities.filter_input}}\pysiglinewithargsret{\sphinxcode{\sphinxupquote{muse.utilities.}}\sphinxbfcode{\sphinxupquote{filter\_input}}}{\emph{\DUrole{n}{dataset}\DUrole{p}{:} \DUrole{n}{Union\DUrole{p}{{[}}xarray.core.dataset.Dataset\DUrole{p}{, }xarray.core.dataarray.DataArray\DUrole{p}{{]}}}}, \emph{\DUrole{n}{year}\DUrole{p}{:} \DUrole{n}{Optional\DUrole{p}{{[}}Union\DUrole{p}{{[}}int\DUrole{p}{, }Iterable\DUrole{p}{{[}}int\DUrole{p}{{]}}\DUrole{p}{{]}}\DUrole{p}{{]}}} \DUrole{o}{=} \DUrole{default_value}{None}}, \emph{\DUrole{n}{interpolation}\DUrole{p}{:} \DUrole{n}{str} \DUrole{o}{=} \DUrole{default_value}{\textquotesingle{}linear\textquotesingle{}}}, \emph{\DUrole{o}{**}\DUrole{n}{kwargs}}}{{ $\rightarrow$ Union\DUrole{p}{{[}}xarray.core.dataset.Dataset\DUrole{p}{, }xarray.core.dataarray.DataArray\DUrole{p}{{]}}}}
Filter inputs, taking care to interpolate years.

\end{fulllineitems}

\index{filter\_with\_template() (in module muse.utilities)@\spxentry{filter\_with\_template()}\spxextra{in module muse.utilities}}

\begin{fulllineitems}
\phantomsection\label{\detokenize{api:muse.utilities.filter_with_template}}\pysiglinewithargsret{\sphinxcode{\sphinxupquote{muse.utilities.}}\sphinxbfcode{\sphinxupquote{filter\_with\_template}}}{\emph{\DUrole{n}{data}\DUrole{p}{:} \DUrole{n}{Union\DUrole{p}{{[}}xarray.core.dataset.Dataset\DUrole{p}{, }xarray.core.dataarray.DataArray\DUrole{p}{{]}}}}, \emph{\DUrole{n}{template}\DUrole{p}{:} \DUrole{n}{Union\DUrole{p}{{[}}xarray.core.dataarray.DataArray\DUrole{p}{, }xarray.core.dataset.Dataset\DUrole{p}{{]}}}}, \emph{\DUrole{n}{asset\_dimension}\DUrole{p}{:} \DUrole{n}{str} \DUrole{o}{=} \DUrole{default_value}{\textquotesingle{}asset\textquotesingle{}}}, \emph{\DUrole{o}{**}\DUrole{n}{kwargs}}}{}
Filters data to match template.

If the \sphinxtitleref{asset\_dimension} is present in \sphinxtitleref{template.dims}, then the call is
forwarded to \sphinxtitleref{broadcast\_techs}. Otherwise, the set of dimensions and indices
in common between \sphinxtitleref{template} and \sphinxtitleref{data} are determined, and the resulting
call is forwarded to \sphinxtitleref{filter\_input}.
\begin{quote}\begin{description}
\item[{Parameters}] \leavevmode\begin{itemize}
\item {} 
\sphinxstyleliteralstrong{\sphinxupquote{data}} \textendash{} Data to transform

\item {} 
\sphinxstyleliteralstrong{\sphinxupquote{template}} \textendash{} Data from which to figure coordinates and dimensions

\item {} 
\sphinxstyleliteralstrong{\sphinxupquote{asset\_dimension}} \textendash{} Name of the dimension which if present indicates the
format is that of an \sphinxstyleemphasis{asset} (see \sphinxtitleref{broadcast\_techs})

\item {} 
\sphinxstyleliteralstrong{\sphinxupquote{kwargs}} \textendash{} passed on to \sphinxtitleref{broadcast\_techs} or \sphinxtitleref{filter\_input}

\end{itemize}

\end{description}\end{quote}
\begin{description}
\item[{Returns}] \leavevmode
\sphinxtitleref{data} transformed to match the form of \sphinxtitleref{template}

\end{description}

\end{fulllineitems}

\index{future\_propagation() (in module muse.utilities)@\spxentry{future\_propagation()}\spxextra{in module muse.utilities}}

\begin{fulllineitems}
\phantomsection\label{\detokenize{api:muse.utilities.future_propagation}}\pysiglinewithargsret{\sphinxcode{\sphinxupquote{muse.utilities.}}\sphinxbfcode{\sphinxupquote{future\_propagation}}}{\emph{\DUrole{n}{data}\DUrole{p}{:} \DUrole{n}{xarray.core.dataarray.DataArray}}, \emph{\DUrole{n}{future}\DUrole{p}{:} \DUrole{n}{xarray.core.dataarray.DataArray}}, \emph{\DUrole{n}{threshhold}\DUrole{p}{:} \DUrole{n}{float} \DUrole{o}{=} \DUrole{default_value}{1e\sphinxhyphen{}12}}, \emph{\DUrole{n}{dim}\DUrole{p}{:} \DUrole{n}{str} \DUrole{o}{=} \DUrole{default_value}{\textquotesingle{}year\textquotesingle{}}}}{{ $\rightarrow$ xarray.core.dataarray.DataArray}}
Propagates values into the future.
\subsubsection*{Example}

\sphinxcode{\sphinxupquote{Data}} should be an array with at least one dimension, “year”:

\begin{sphinxVerbatim}[commandchars=\\\{\}]
\PYG{g+gp}{\PYGZgt{}\PYGZgt{}\PYGZgt{} }\PYG{n}{coords} \PYG{o}{=} \PYG{n+nb}{dict}\PYG{p}{(}\PYG{n}{year}\PYG{o}{=}\PYG{n+nb}{list}\PYG{p}{(}\PYG{n+nb}{range}\PYG{p}{(}\PYG{l+m+mi}{2020}\PYG{p}{,} \PYG{l+m+mi}{2040}\PYG{p}{,} \PYG{l+m+mi}{5}\PYG{p}{)}\PYG{p}{)}\PYG{p}{,} \PYG{n}{fuel}\PYG{o}{=}\PYG{p}{[}\PYG{l+s+s2}{\PYGZdq{}}\PYG{l+s+s2}{gas}\PYG{l+s+s2}{\PYGZdq{}}\PYG{p}{,} \PYG{l+s+s2}{\PYGZdq{}}\PYG{l+s+s2}{coal}\PYG{l+s+s2}{\PYGZdq{}}\PYG{p}{]}\PYG{p}{)}
\PYG{g+gp}{\PYGZgt{}\PYGZgt{}\PYGZgt{} }\PYG{n}{data} \PYG{o}{=} \PYG{n}{xr}\PYG{o}{.}\PYG{n}{DataArray}\PYG{p}{(}
\PYG{g+gp}{... }    \PYG{p}{[}\PYG{n+nb}{list}\PYG{p}{(}\PYG{n+nb}{range}\PYG{p}{(}\PYG{l+m+mi}{4}\PYG{p}{)}\PYG{p}{)}\PYG{p}{,} \PYG{n+nb}{list}\PYG{p}{(}\PYG{n+nb}{range}\PYG{p}{(}\PYG{o}{\PYGZhy{}}\PYG{l+m+mi}{5}\PYG{p}{,} \PYG{o}{\PYGZhy{}}\PYG{l+m+mi}{1}\PYG{p}{)}\PYG{p}{)}\PYG{p}{]}\PYG{p}{,}
\PYG{g+gp}{... }    \PYG{n}{coords}\PYG{o}{=}\PYG{n}{coords}\PYG{p}{,}
\PYG{g+gp}{... }    \PYG{n}{dims}\PYG{o}{=}\PYG{p}{(}\PYG{l+s+s2}{\PYGZdq{}}\PYG{l+s+s2}{fuel}\PYG{l+s+s2}{\PYGZdq{}}\PYG{p}{,} \PYG{l+s+s2}{\PYGZdq{}}\PYG{l+s+s2}{year}\PYG{l+s+s2}{\PYGZdq{}}\PYG{p}{)}
\PYG{g+gp}{... }\PYG{p}{)}
\end{sphinxVerbatim}

\sphinxcode{\sphinxupquote{future}} is an array with  exactly one year in its \sphinxcode{\sphinxupquote{year}} coordinate, or
that coordinate must correspond to a scalar. That one year should also be
present in \sphinxcode{\sphinxupquote{data}}.

\begin{sphinxVerbatim}[commandchars=\\\{\}]
\PYG{g+gp}{\PYGZgt{}\PYGZgt{}\PYGZgt{} }\PYG{n}{future} \PYG{o}{=} \PYG{n}{xr}\PYG{o}{.}\PYG{n}{DataArray}\PYG{p}{(}
\PYG{g+gp}{... }    \PYG{p}{[}\PYG{l+m+mf}{1.2}\PYG{p}{,} \PYG{o}{\PYGZhy{}}\PYG{l+m+mf}{3.95}\PYG{p}{]}\PYG{p}{,} \PYG{n}{coords}\PYG{o}{=}\PYG{n+nb}{dict}\PYG{p}{(}\PYG{n}{fuel}\PYG{o}{=}\PYG{n}{coords}\PYG{p}{[}\PYG{l+s+s1}{\PYGZsq{}}\PYG{l+s+s1}{fuel}\PYG{l+s+s1}{\PYGZsq{}}\PYG{p}{]}\PYG{p}{,} \PYG{n}{year}\PYG{o}{=}\PYG{l+m+mi}{2025}\PYG{p}{)}\PYG{p}{,} \PYG{n}{dims}\PYG{o}{=}\PYG{l+s+s2}{\PYGZdq{}}\PYG{l+s+s2}{fuel}\PYG{l+s+s2}{\PYGZdq{}}\PYG{p}{,}
\PYG{g+gp}{... }\PYG{p}{)}
\end{sphinxVerbatim}

This function propagates into \sphinxcode{\sphinxupquote{data}} values from \sphinxcode{\sphinxupquote{future}}, but only if those
values differed for the current year beyond a given threshhold:

\begin{sphinxVerbatim}[commandchars=\\\{\}]
\PYG{g+gp}{\PYGZgt{}\PYGZgt{}\PYGZgt{} }\PYG{k+kn}{from} \PYG{n+nn}{muse}\PYG{n+nn}{.}\PYG{n+nn}{utilities} \PYG{k+kn}{import} \PYG{n}{future\PYGZus{}propagation}
\PYG{g+gp}{\PYGZgt{}\PYGZgt{}\PYGZgt{} }\PYG{n}{future\PYGZus{}propagation}\PYG{p}{(}\PYG{n}{data}\PYG{p}{,} \PYG{n}{future}\PYG{p}{,} \PYG{n}{threshhold}\PYG{o}{=}\PYG{l+m+mf}{0.1}\PYG{p}{)}
\PYG{g+go}{\PYGZlt{}xarray.DataArray (fuel: 2, year: 4)\PYGZgt{}}
\PYG{g+go}{array([[ 0. ,  1.2,  1.2,  1.2],}
\PYG{g+go}{       [\PYGZhy{}5. , \PYGZhy{}4. , \PYGZhy{}3. , \PYGZhy{}2. ]])}
\PYG{g+go}{Coordinates:}
\PYG{g+go}{  * year     (year) ... 2020 2025 2030 2035}
\PYG{g+go}{  * fuel     (fuel) \PYGZlt{}U4 \PYGZsq{}gas\PYGZsq{} \PYGZsq{}coal\PYGZsq{}}
\end{sphinxVerbatim}

Above, the data for coal is not sufficiently different given the threshhold.
hence, the future values for coal remain as they where.

The dimensions of \sphinxcode{\sphinxupquote{future}} do not have to match exactly those of \sphinxcode{\sphinxupquote{data}}.
Standard broadcasting is used if they do not match:

\begin{sphinxVerbatim}[commandchars=\\\{\}]
\PYG{g+gp}{\PYGZgt{}\PYGZgt{}\PYGZgt{} }\PYG{n}{future\PYGZus{}propagation}\PYG{p}{(}\PYG{n}{data}\PYG{p}{,} \PYG{n}{future}\PYG{o}{.}\PYG{n}{sel}\PYG{p}{(}\PYG{n}{fuel}\PYG{o}{=}\PYG{l+s+s2}{\PYGZdq{}}\PYG{l+s+s2}{gas}\PYG{l+s+s2}{\PYGZdq{}}\PYG{p}{,} \PYG{n}{drop}\PYG{o}{=}\PYG{k+kc}{True}\PYG{p}{)}\PYG{p}{,} \PYG{n}{threshhold}\PYG{o}{=}\PYG{l+m+mf}{0.1}\PYG{p}{)}
\PYG{g+go}{\PYGZlt{}xarray.DataArray (fuel: 2, year: 4)\PYGZgt{}}
\PYG{g+go}{array([[ 0. ,  1.2,  1.2,  1.2],}
\PYG{g+go}{       [\PYGZhy{}5. ,  1.2,  1.2,  1.2]])}
\PYG{g+go}{Coordinates:}
\PYG{g+go}{  * year     (year) ... 2020 2025 2030 2035}
\PYG{g+go}{  * fuel     (fuel) \PYGZlt{}U4 \PYGZsq{}gas\PYGZsq{} \PYGZsq{}coal\PYGZsq{}}
\PYG{g+gp}{\PYGZgt{}\PYGZgt{}\PYGZgt{} }\PYG{n}{future\PYGZus{}propagation}\PYG{p}{(}\PYG{n}{data}\PYG{p}{,} \PYG{n}{future}\PYG{o}{.}\PYG{n}{sel}\PYG{p}{(}\PYG{n}{fuel}\PYG{o}{=}\PYG{l+s+s2}{\PYGZdq{}}\PYG{l+s+s2}{coal}\PYG{l+s+s2}{\PYGZdq{}}\PYG{p}{,} \PYG{n}{drop}\PYG{o}{=}\PYG{k+kc}{True}\PYG{p}{)}\PYG{p}{,} \PYG{n}{threshhold}\PYG{o}{=}\PYG{l+m+mf}{0.1}\PYG{p}{)}
\PYG{g+go}{\PYGZlt{}xarray.DataArray (fuel: 2, year: 4)\PYGZgt{}}
\PYG{g+go}{array([[ 0.  , \PYGZhy{}3.95, \PYGZhy{}3.95, \PYGZhy{}3.95],}
\PYG{g+go}{       [\PYGZhy{}5.  , \PYGZhy{}4.  , \PYGZhy{}3.  , \PYGZhy{}2.  ]])}
\PYG{g+go}{Coordinates:}
\PYG{g+go}{  * year     (year) ... 2020 2025 2030 2035}
\PYG{g+go}{  * fuel     (fuel) \PYGZlt{}U4 \PYGZsq{}gas\PYGZsq{} \PYGZsq{}coal\PYGZsq{}}
\end{sphinxVerbatim}

\end{fulllineitems}

\index{lexical\_comparison() (in module muse.utilities)@\spxentry{lexical\_comparison()}\spxextra{in module muse.utilities}}

\begin{fulllineitems}
\phantomsection\label{\detokenize{api:muse.utilities.lexical_comparison}}\pysiglinewithargsret{\sphinxcode{\sphinxupquote{muse.utilities.}}\sphinxbfcode{\sphinxupquote{lexical\_comparison}}}{\emph{\DUrole{n}{objectives}\DUrole{p}{:} \DUrole{n}{xarray.core.dataset.Dataset}}, \emph{\DUrole{n}{binsize}\DUrole{p}{:} \DUrole{n}{xarray.core.dataset.Dataset}}, \emph{\DUrole{n}{order}\DUrole{p}{:} \DUrole{n}{Optional\DUrole{p}{{[}}Sequence\DUrole{p}{{[}}Hashable\DUrole{p}{{]}}\DUrole{p}{{]}}} \DUrole{o}{=} \DUrole{default_value}{None}}, \emph{\DUrole{n}{bin\_last}\DUrole{p}{:} \DUrole{n}{bool} \DUrole{o}{=} \DUrole{default_value}{True}}}{{ $\rightarrow$ xarray.core.dataarray.DataArray}}
Lexical comparison over the objectives.

Lexical comparison operates by binning the objectives into bins of width
\sphinxtitleref{binsize}. Once binned, dimensions other than \sphinxtitleref{asset} and \sphinxtitleref{technology} are
reduced by taking the max, e.g. the largest constraint. Finally, the
objectives are ranked lexographically, in the order given by the parameters.
\begin{quote}\begin{description}
\item[{Parameters}] \leavevmode\begin{itemize}
\item {} 
\sphinxstyleliteralstrong{\sphinxupquote{objectives}} \textendash{} xr.Dataset containing the objectives to rank

\item {} 
\sphinxstyleliteralstrong{\sphinxupquote{binsize}} \textendash{} bin size, minimization direction
(+ \sphinxhyphen{}\textgreater{} minimize, \sphinxhyphen{} \sphinxhyphen{}\textgreater{} maximize), and (optionally) order of
lexicographical comparison. The order is the one given
\sphinxtitleref{binsize.data\_vars} if the argument \sphinxtitleref{order} is None.

\item {} 
\sphinxstyleliteralstrong{\sphinxupquote{order}} \textendash{} Optional array indicating the order in which to rank the tuples.

\item {} 
\sphinxstyleliteralstrong{\sphinxupquote{bin\_last}} \textendash{} Whether the last metric should be binned, or whether it
should be left as a the type it already is (e.g. no flooring and
no turning to integer.)

\end{itemize}

\end{description}\end{quote}
\begin{description}
\item[{Result:}] \leavevmode
An array of tuples which can subsquently be compared lexicographically.

\end{description}

\end{fulllineitems}

\index{merge\_assets() (in module muse.utilities)@\spxentry{merge\_assets()}\spxextra{in module muse.utilities}}

\begin{fulllineitems}
\phantomsection\label{\detokenize{api:muse.utilities.merge_assets}}\pysiglinewithargsret{\sphinxcode{\sphinxupquote{muse.utilities.}}\sphinxbfcode{\sphinxupquote{merge\_assets}}}{\emph{\DUrole{n}{capa\_a}\DUrole{p}{:} \DUrole{n}{xarray.core.dataarray.DataArray}}, \emph{\DUrole{n}{capa\_b}\DUrole{p}{:} \DUrole{n}{xarray.core.dataarray.DataArray}}, \emph{\DUrole{n}{interpolation}\DUrole{p}{:} \DUrole{n}{str} \DUrole{o}{=} \DUrole{default_value}{\textquotesingle{}linear\textquotesingle{}}}, \emph{\DUrole{n}{dimension}\DUrole{p}{:} \DUrole{n}{str} \DUrole{o}{=} \DUrole{default_value}{\textquotesingle{}asset\textquotesingle{}}}}{{ $\rightarrow$ xarray.core.dataarray.DataArray}}
Merge two capacity arrays.

\end{fulllineitems}

\index{multiindex\_to\_coords() (in module muse.utilities)@\spxentry{multiindex\_to\_coords()}\spxextra{in module muse.utilities}}

\begin{fulllineitems}
\phantomsection\label{\detokenize{api:muse.utilities.multiindex_to_coords}}\pysiglinewithargsret{\sphinxcode{\sphinxupquote{muse.utilities.}}\sphinxbfcode{\sphinxupquote{multiindex\_to\_coords}}}{\emph{\DUrole{n}{data}\DUrole{p}{:} \DUrole{n}{Union\DUrole{p}{{[}}xarray.core.dataset.Dataset\DUrole{p}{, }xarray.core.dataarray.DataArray\DUrole{p}{{]}}}}, \emph{\DUrole{n}{dimension}\DUrole{p}{:} \DUrole{n}{str} \DUrole{o}{=} \DUrole{default_value}{\textquotesingle{}asset\textquotesingle{}}}}{}
Flattens multi\sphinxhyphen{}index dimension into multi\sphinxhyphen{}coord dimension.

\end{fulllineitems}

\index{nametuple\_to\_dict() (in module muse.utilities)@\spxentry{nametuple\_to\_dict()}\spxextra{in module muse.utilities}}

\begin{fulllineitems}
\phantomsection\label{\detokenize{api:muse.utilities.nametuple_to_dict}}\pysiglinewithargsret{\sphinxcode{\sphinxupquote{muse.utilities.}}\sphinxbfcode{\sphinxupquote{nametuple\_to\_dict}}}{\emph{\DUrole{n}{nametup}\DUrole{p}{:} \DUrole{n}{Union\DUrole{p}{{[}}Mapping\DUrole{p}{, }NamedTuple\DUrole{p}{{]}}}}}{{ $\rightarrow$ Mapping}}
Transforms a nametuple of type GenericDict into an OrderDict.

\end{fulllineitems}

\index{reduce\_assets() (in module muse.utilities)@\spxentry{reduce\_assets()}\spxextra{in module muse.utilities}}

\begin{fulllineitems}
\phantomsection\label{\detokenize{api:muse.utilities.reduce_assets}}\pysiglinewithargsret{\sphinxcode{\sphinxupquote{muse.utilities.}}\sphinxbfcode{\sphinxupquote{reduce\_assets}}}{\emph{\DUrole{n}{assets}\DUrole{p}{:} \DUrole{n}{Union\DUrole{p}{{[}}xarray.core.dataarray.DataArray\DUrole{p}{, }xarray.core.dataset.Dataset\DUrole{p}{, }Sequence\DUrole{p}{{[}}Union\DUrole{p}{{[}}xarray.core.dataset.Dataset\DUrole{p}{, }xarray.core.dataarray.DataArray\DUrole{p}{{]}}\DUrole{p}{{]}}\DUrole{p}{{]}}}}, \emph{\DUrole{n}{coords}\DUrole{p}{:} \DUrole{n}{Optional\DUrole{p}{{[}}Union\DUrole{p}{{[}}str\DUrole{p}{, }Sequence\DUrole{p}{{[}}str\DUrole{p}{{]}}\DUrole{p}{, }Iterable\DUrole{p}{{[}}str\DUrole{p}{{]}}\DUrole{p}{{]}}\DUrole{p}{{]}}} \DUrole{o}{=} \DUrole{default_value}{None}}, \emph{\DUrole{n}{dim}\DUrole{p}{:} \DUrole{n}{str} \DUrole{o}{=} \DUrole{default_value}{\textquotesingle{}asset\textquotesingle{}}}, \emph{\DUrole{n}{operation}\DUrole{p}{:} \DUrole{n}{Optional\DUrole{p}{{[}}Callable\DUrole{p}{{]}}} \DUrole{o}{=} \DUrole{default_value}{None}}}{{ $\rightarrow$ xarray.core.dataarray.DataArray}}
Combine assets along given asset dimension.

This method simplifies combining assets accross multiple agents, or combining assets
across a given dimension. By default, it will sum together assets from the same
region which have the same technology and the same installation date. In other
words, assets are identified by the technology, installation year and region. The
reduction happens over other possible coordinates, e.g. the owning agent.

More specifically, assets are often indexed using what xarray calls a \sphinxstylestrong{dimension
without coordinates}. In practice, there are still coordinates associated with
assets, e.g. \sphinxstyleemphasis{technology} and \sphinxstyleemphasis{installed} (installation year or version), but the
value associated with these coordinates are not unique.  There may be more than one
asset with the same technology or installation year.

For instance, with assets per agent defined as \(A^{i, r}_o\), with \(i\) an
agent index, \(r\) a region, \(o\) is the coordinates identified in
\sphinxcode{\sphinxupquote{coords}}, and \(T\) the transformation identified by \sphinxcode{\sphinxupquote{operation}}, then this
function computes:
\begin{equation*}
\begin{split}R_{r, o} = T[\{A^{i, r}_o; \forall i\}]\end{split}
\end{equation*}
If \(T\) is the sum operation, then:
\begin{equation*}
\begin{split}R_{r, o} = \sum_i  A^{i, r}_o\end{split}
\end{equation*}\subsubsection*{Example}

Lets construct assets that do have duplicates assets. First we construct the
dimensions, using fake data:

\begin{sphinxVerbatim}[commandchars=\\\{\}]
\PYG{g+gp}{\PYGZgt{}\PYGZgt{}\PYGZgt{} }\PYG{n}{data} \PYG{o}{=} \PYG{n}{xr}\PYG{o}{.}\PYG{n}{Dataset}\PYG{p}{(}\PYG{p}{)}
\PYG{g+gp}{\PYGZgt{}\PYGZgt{}\PYGZgt{} }\PYG{n}{data}\PYG{p}{[}\PYG{l+s+s1}{\PYGZsq{}}\PYG{l+s+s1}{year}\PYG{l+s+s1}{\PYGZsq{}}\PYG{p}{]} \PYG{o}{=} \PYG{l+s+s1}{\PYGZsq{}}\PYG{l+s+s1}{year}\PYG{l+s+s1}{\PYGZsq{}}\PYG{p}{,} \PYG{p}{[}\PYG{l+m+mi}{2010}\PYG{p}{,} \PYG{l+m+mi}{2015}\PYG{p}{,} \PYG{l+m+mi}{2017}\PYG{p}{]}
\PYG{g+gp}{\PYGZgt{}\PYGZgt{}\PYGZgt{} }\PYG{n}{data}\PYG{p}{[}\PYG{l+s+s1}{\PYGZsq{}}\PYG{l+s+s1}{installed}\PYG{l+s+s1}{\PYGZsq{}}\PYG{p}{]} \PYG{o}{=} \PYG{l+s+s1}{\PYGZsq{}}\PYG{l+s+s1}{asset}\PYG{l+s+s1}{\PYGZsq{}}\PYG{p}{,} \PYG{p}{[}\PYG{l+m+mi}{1990}\PYG{p}{,} \PYG{l+m+mi}{1991}\PYG{p}{,} \PYG{l+m+mi}{1991}\PYG{p}{,} \PYG{l+m+mi}{1990}\PYG{p}{]}
\PYG{g+gp}{\PYGZgt{}\PYGZgt{}\PYGZgt{} }\PYG{n}{data}\PYG{p}{[}\PYG{l+s+s1}{\PYGZsq{}}\PYG{l+s+s1}{technology}\PYG{l+s+s1}{\PYGZsq{}}\PYG{p}{]} \PYG{o}{=} \PYG{l+s+s1}{\PYGZsq{}}\PYG{l+s+s1}{asset}\PYG{l+s+s1}{\PYGZsq{}}\PYG{p}{,} \PYG{p}{[}\PYG{l+s+s1}{\PYGZsq{}}\PYG{l+s+s1}{a}\PYG{l+s+s1}{\PYGZsq{}}\PYG{p}{,} \PYG{l+s+s1}{\PYGZsq{}}\PYG{l+s+s1}{b}\PYG{l+s+s1}{\PYGZsq{}}\PYG{p}{,} \PYG{l+s+s1}{\PYGZsq{}}\PYG{l+s+s1}{b}\PYG{l+s+s1}{\PYGZsq{}}\PYG{p}{,} \PYG{l+s+s1}{\PYGZsq{}}\PYG{l+s+s1}{c}\PYG{l+s+s1}{\PYGZsq{}}\PYG{p}{]}
\PYG{g+gp}{\PYGZgt{}\PYGZgt{}\PYGZgt{} }\PYG{n}{data}\PYG{p}{[}\PYG{l+s+s1}{\PYGZsq{}}\PYG{l+s+s1}{region}\PYG{l+s+s1}{\PYGZsq{}}\PYG{p}{]} \PYG{o}{=} \PYG{l+s+s1}{\PYGZsq{}}\PYG{l+s+s1}{asset}\PYG{l+s+s1}{\PYGZsq{}}\PYG{p}{,} \PYG{p}{[}\PYG{l+s+s1}{\PYGZsq{}}\PYG{l+s+s1}{x}\PYG{l+s+s1}{\PYGZsq{}}\PYG{p}{,} \PYG{l+s+s1}{\PYGZsq{}}\PYG{l+s+s1}{x}\PYG{l+s+s1}{\PYGZsq{}}\PYG{p}{,} \PYG{l+s+s1}{\PYGZsq{}}\PYG{l+s+s1}{x}\PYG{l+s+s1}{\PYGZsq{}}\PYG{p}{,} \PYG{l+s+s1}{\PYGZsq{}}\PYG{l+s+s1}{y}\PYG{l+s+s1}{\PYGZsq{}}\PYG{p}{]}
\PYG{g+gp}{\PYGZgt{}\PYGZgt{}\PYGZgt{} }\PYG{n}{data} \PYG{o}{=} \PYG{n}{data}\PYG{o}{.}\PYG{n}{set\PYGZus{}coords}\PYG{p}{(}\PYG{p}{(}\PYG{l+s+s1}{\PYGZsq{}}\PYG{l+s+s1}{installed}\PYG{l+s+s1}{\PYGZsq{}}\PYG{p}{,} \PYG{l+s+s1}{\PYGZsq{}}\PYG{l+s+s1}{technology}\PYG{l+s+s1}{\PYGZsq{}}\PYG{p}{,} \PYG{l+s+s1}{\PYGZsq{}}\PYG{l+s+s1}{region}\PYG{l+s+s1}{\PYGZsq{}}\PYG{p}{)}\PYG{p}{)}
\end{sphinxVerbatim}

We can check there are duplicate assets in this coordinate system:

\begin{sphinxVerbatim}[commandchars=\\\{\}]
\PYG{g+gp}{\PYGZgt{}\PYGZgt{}\PYGZgt{} }\PYG{n}{processes} \PYG{o}{=} \PYG{n+nb}{set}\PYG{p}{(}
\PYG{g+gp}{... }    \PYG{n+nb}{zip}\PYG{p}{(}\PYG{n}{data}\PYG{o}{.}\PYG{n}{installed}\PYG{o}{.}\PYG{n}{values}\PYG{p}{,} \PYG{n}{data}\PYG{o}{.}\PYG{n}{technology}\PYG{o}{.}\PYG{n}{values}\PYG{p}{,} \PYG{n}{data}\PYG{o}{.}\PYG{n}{region}\PYG{o}{.}\PYG{n}{values}\PYG{p}{)}
\PYG{g+gp}{... }\PYG{p}{)}
\PYG{g+gp}{\PYGZgt{}\PYGZgt{}\PYGZgt{} }\PYG{n+nb}{len}\PYG{p}{(}\PYG{n}{processes}\PYG{p}{)} \PYG{o}{\PYGZlt{}} \PYG{n+nb}{len}\PYG{p}{(}\PYG{n}{data}\PYG{o}{.}\PYG{n}{asset}\PYG{p}{)}
\PYG{g+go}{True}
\end{sphinxVerbatim}

Now we can easily create a fake two dimensional quantity per process and
per year:

\begin{sphinxVerbatim}[commandchars=\\\{\}]
\PYG{g+gp}{\PYGZgt{}\PYGZgt{}\PYGZgt{} }\PYG{n}{data}\PYG{p}{[}\PYG{l+s+s1}{\PYGZsq{}}\PYG{l+s+s1}{capacity}\PYG{l+s+s1}{\PYGZsq{}}\PYG{p}{]} \PYG{o}{=} \PYG{p}{(}\PYG{l+s+s1}{\PYGZsq{}}\PYG{l+s+s1}{year}\PYG{l+s+s1}{\PYGZsq{}}\PYG{p}{,} \PYG{l+s+s1}{\PYGZsq{}}\PYG{l+s+s1}{asset}\PYG{l+s+s1}{\PYGZsq{}}\PYG{p}{)}\PYG{p}{,} \PYG{n}{np}\PYG{o}{.}\PYG{n}{arange}\PYG{p}{(}\PYG{l+m+mi}{3} \PYG{o}{*} \PYG{l+m+mi}{4}\PYG{p}{)}\PYG{o}{.}\PYG{n}{reshape}\PYG{p}{(}\PYG{l+m+mi}{3}\PYG{p}{,} \PYG{l+m+mi}{4}\PYG{p}{)}
\end{sphinxVerbatim}

The point of \sphinxtitleref{reduce\_assets} is to aggregate assets that refer to the
same process:

\begin{sphinxVerbatim}[commandchars=\\\{\}]
\PYG{g+gp}{\PYGZgt{}\PYGZgt{}\PYGZgt{} }\PYG{n}{reduce\PYGZus{}assets}\PYG{p}{(}\PYG{n}{data}\PYG{o}{.}\PYG{n}{capacity}\PYG{p}{)}
\PYG{g+go}{\PYGZlt{}xarray.DataArray \PYGZsq{}capacity\PYGZsq{} (year: 3, asset: 3)\PYGZgt{}}
\PYG{g+go}{array([[ 0,  3,  3],}
\PYG{g+go}{       [ 4,  7, 11],}
\PYG{g+go}{       [ 8, 11, 19]])}
\PYG{g+go}{Coordinates:}
\PYG{g+go}{  * year        (year) ... 2010 2015 2017}
\PYG{g+go}{    installed   (asset) ... 1990 1990 1991}
\PYG{g+go}{    technology  (asset) \PYGZlt{}U1 \PYGZsq{}a\PYGZsq{} \PYGZsq{}c\PYGZsq{} \PYGZsq{}b\PYGZsq{}}
\PYG{g+go}{    region      (asset) \PYGZlt{}U1 \PYGZsq{}x\PYGZsq{} \PYGZsq{}y\PYGZsq{} \PYGZsq{}x\PYGZsq{}}
\PYG{g+go}{Dimensions without coordinates: asset}
\end{sphinxVerbatim}

We can also specify explicitly which coordinates in the ‘asset’
dimension should be reduced, and how:

\begin{sphinxVerbatim}[commandchars=\\\{\}]
\PYG{g+gp}{\PYGZgt{}\PYGZgt{}\PYGZgt{} }\PYG{n}{reduce\PYGZus{}assets}\PYG{p}{(}
\PYG{g+gp}{... }    \PYG{n}{data}\PYG{o}{.}\PYG{n}{capacity}\PYG{p}{,}
\PYG{g+gp}{... }    \PYG{n}{coords}\PYG{o}{=}\PYG{p}{(}\PYG{l+s+s1}{\PYGZsq{}}\PYG{l+s+s1}{technology}\PYG{l+s+s1}{\PYGZsq{}}\PYG{p}{,} \PYG{l+s+s1}{\PYGZsq{}}\PYG{l+s+s1}{installed}\PYG{l+s+s1}{\PYGZsq{}}\PYG{p}{)}\PYG{p}{,}
\PYG{g+gp}{... }    \PYG{n}{operation} \PYG{o}{=} \PYG{k}{lambda} \PYG{n}{x}\PYG{p}{:} \PYG{n}{x}\PYG{o}{.}\PYG{n}{mean}\PYG{p}{(}\PYG{n}{dim}\PYG{o}{=}\PYG{l+s+s1}{\PYGZsq{}}\PYG{l+s+s1}{asset}\PYG{l+s+s1}{\PYGZsq{}}\PYG{p}{)}
\PYG{g+gp}{... }\PYG{p}{)}
\PYG{g+go}{\PYGZlt{}xarray.DataArray \PYGZsq{}capacity\PYGZsq{} (year: 3, asset: 3)\PYGZgt{}}
\PYG{g+go}{array([[ 0. ,  1.5,  3. ],}
\PYG{g+go}{       [ 4. ,  5.5,  7. ],}
\PYG{g+go}{       [ 8. ,  9.5, 11. ]])}
\PYG{g+go}{Coordinates:}
\PYG{g+go}{  * year        (year) ... 2010 2015 2017}
\PYG{g+go}{    technology  (asset) \PYGZlt{}U1 \PYGZsq{}a\PYGZsq{} \PYGZsq{}b\PYGZsq{} \PYGZsq{}c\PYGZsq{}}
\PYG{g+go}{    installed   (asset) ... 1990 1991 1990}
\PYG{g+go}{Dimensions without coordinates: asset}
\end{sphinxVerbatim}

\end{fulllineitems}

\index{tupled\_dimension() (in module muse.utilities)@\spxentry{tupled\_dimension()}\spxextra{in module muse.utilities}}

\begin{fulllineitems}
\phantomsection\label{\detokenize{api:muse.utilities.tupled_dimension}}\pysiglinewithargsret{\sphinxcode{\sphinxupquote{muse.utilities.}}\sphinxbfcode{\sphinxupquote{tupled\_dimension}}}{\emph{\DUrole{n}{array}\DUrole{p}{:} \DUrole{n}{numpy.ndarray}}, \emph{\DUrole{n}{axis}\DUrole{p}{:} \DUrole{n}{int}}}{}
Transforms one axis into a tuples.

\end{fulllineitems}



\subsection{Examples}
\label{\detokenize{api:module-muse.examples}}\label{\detokenize{api:examples}}\index{module@\spxentry{module}!muse.examples@\spxentry{muse.examples}}\index{muse.examples@\spxentry{muse.examples}!module@\spxentry{module}}
Example models and datasets.

Helps create and run small standard models from the command\sphinxhyphen{}line or directly from
python.

To run from the command\sphinxhyphen{}line:

\begin{sphinxVerbatim}[commandchars=\\\{\}]
python \PYGZhy{}m muse \PYGZhy{}\PYGZhy{}model default
\end{sphinxVerbatim}

Other models may be available. Check the command\sphinxhyphen{}line help:

\begin{sphinxVerbatim}[commandchars=\\\{\}]
python \PYGZhy{}m muse \PYGZhy{}\PYGZhy{}help
\end{sphinxVerbatim}

The same models can be instanciated in a python script as follows:

\begin{sphinxVerbatim}[commandchars=\\\{\}]
\PYG{k+kn}{from} \PYG{n+nn}{muse} \PYG{k+kn}{import} \PYG{n}{example}
\PYG{n}{model} \PYG{o}{=} \PYG{n}{example}\PYG{o}{.}\PYG{n}{model}\PYG{p}{(}\PYG{l+s+s2}{\PYGZdq{}}\PYG{l+s+s2}{default}\PYG{l+s+s2}{\PYGZdq{}}\PYG{p}{)}
\PYG{n}{model}\PYG{o}{.}\PYG{n}{run}\PYG{p}{(}\PYG{p}{)}
\end{sphinxVerbatim}
\index{model() (in module muse.examples)@\spxentry{model()}\spxextra{in module muse.examples}}

\begin{fulllineitems}
\phantomsection\label{\detokenize{api:muse.examples.model}}\pysiglinewithargsret{\sphinxcode{\sphinxupquote{muse.examples.}}\sphinxbfcode{\sphinxupquote{model}}}{\emph{\DUrole{n}{name}\DUrole{p}{:} \DUrole{n}{str} \DUrole{o}{=} \DUrole{default_value}{\textquotesingle{}default\textquotesingle{}}}}{{ $\rightarrow$ muse.mca.MCA}}
Fully constructs a given example model.

\end{fulllineitems}

\index{technodata() (in module muse.examples)@\spxentry{technodata()}\spxextra{in module muse.examples}}

\begin{fulllineitems}
\phantomsection\label{\detokenize{api:muse.examples.technodata}}\pysiglinewithargsret{\sphinxcode{\sphinxupquote{muse.examples.}}\sphinxbfcode{\sphinxupquote{technodata}}}{\emph{\DUrole{n}{sector}\DUrole{p}{:} \DUrole{n}{str}}, \emph{\DUrole{n}{model}\DUrole{p}{:} \DUrole{n}{str} \DUrole{o}{=} \DUrole{default_value}{\textquotesingle{}default\textquotesingle{}}}}{{ $\rightarrow$ xarray.core.dataset.Dataset}}
Technology for a sector of a given example model.

\end{fulllineitems}



\chapter{Indices and tables}
\label{\detokenize{index:indices-and-tables}}\begin{itemize}
\item {} 
\DUrole{xref,std,std-ref}{genindex}

\item {} 
\DUrole{xref,std,std-ref}{modindex}

\item {} 
\DUrole{xref,std,std-ref}{search}

\end{itemize}



\renewcommand{\indexname}{Index}
\printindex
\end{document}